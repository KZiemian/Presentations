% ---------------------------------------------------------------------
% Basic configuration of Beamera and Jagiellonian
% ---------------------------------------------------------------------
\RequirePackage[l2tabu, orthodox]{nag}



\ifx\PresentationStyle\notset
\def\PresentationStyle{dark}
\fi



\documentclass[10pt,t]{beamer}
\mode<presentation>
\usetheme[style=\PresentationStyle,logoColor=monochromaticJUwhite,JUlogotitle=yes]{jagiellonian}



% ---------------------------------------
% Configuration files of Jagiellonian loceted in catalog preambule
% ---------------------------------------
% Configuration for polish language
% Need description
\usepackage[polish]{babel}
% Need description
\usepackage[MeX]{polski}



% ------------------------------
% Better support of polish chars in technical parts of PDF
% ------------------------------
\hypersetup{pdfencoding=auto,psdextra}

% Package "textpos" give as enviroment "textblock" which is very usefull in
% arranging text on slides.

% This is standard configuration of "textpos"
\usepackage[overlay,absolute]{textpos}

% If you need to see bounds of "textblock's" comment line above and uncomment
% one below.

% Caution! When showboxes option is on significant ammunt of space is add
% to the top of textblock and as such, everyting put in them gone down.
% We need to check how to remove this bug.

% \usepackage[showboxes,overlay,absolute]{textpos}



% Setting scale length for package "textpos"
\setlength{\TPHorizModule}{10mm}
\setlength{\TPVertModule}{\TPHorizModule}


% ---------------------------------------
% TikZ
% ---------------------------------------
% Importing TikZ libraries
\usetikzlibrary{arrows.meta}
\usetikzlibrary{positioning}





% % Configuration package "bm" that need for making bold symbols
% \newcommand{\bmmax}{0}
% \newcommand{\hmmax}{0}
% \usepackage{bm}




% ---------------------------------------
% Packages for scientific texts
% ---------------------------------------
% \let\lll\undefined  % Sometimes you must use this line to allow
% "amsmath" package to works with packages with packages for polish
% languge imported
% /preambul/LanguageSettings/JagiellonianPolishLanguageSettings.tex.
% This comments (probably) removes polish letter Ł.
\usepackage{amsmath}  % Packages from American Mathematical Society (AMS)
\usepackage{amssymb}
\usepackage{amscd}
\usepackage{amsthm}
\usepackage{siunitx}  % Package for typsetting SI units.
\usepackage{upgreek}  % Better looking greek letters.
% Example of using upgreek: pi = \uppi


\usepackage{calrsfs}  % Zmienia czcionkę kaligraficzną w \mathcal
% na ładniejszą. Może w innych miejscach robi to samo, ale o tym nic
% nie wiem.










% ---------------------------------------
% Packages written for lectures "Geometria 3D dla twórców gier wideo"
% ---------------------------------------
% \usepackage{./ProgramowanieSymulacjiFizykiPaczki/ProgramowanieSymulacjiFizyki}
% \usepackage{./ProgramowanieSymulacjiFizykiPaczki/ProgramowanieSymulacjiFizykiIndeksy}
% \usepackage{./ProgramowanieSymulacjiFizykiPaczki/ProgramowanieSymulacjiFizykiTikZStyle}





% !!!!!!!!!!!!!!!!!!!!!!!!!!!!!!
% !!!!!!!!!!!!!!!!!!!!!!!!!!!!!!
% EVIL STUFF
\if\JUlogotitle1
\edef\LogoJUPath{LogoJU_\JUlogoLang/LogoJU_\JUlogoShape_\JUlogoColor.pdf}
\titlegraphic{\hfill\includegraphics[scale=0.22]
{./JagiellonianPictures/\LogoJUPath}}
\fi
% ---------------------------------------
% Commands for handling colors
% ---------------------------------------


% Command for setting normal text color for some text in math modestyle
% Text color depend on used style of Jagiellonian

% Beamer version of command
\newcommand{\TextWithNormalTextColor}[1]{%
  {\color{jNormalTextFGColor}
    \setbeamercolor{math text}{fg=jNormalTextFGColor} {#1}}
}

% Article and similar classes version of command
% \newcommand{\TextWithNormalTextColor}[1]{%
%   {\color{jNormalTextsFGColor} {#1}}
% }



% Beamer version of command
\newcommand{\NormalTextInMathMode}[1]{%
  {\color{jNormalTextFGColor}
    \setbeamercolor{math text}{fg=jNormalTextFGColor} \text{#1}}
}


% Article and similar classes version of command
% \newcommand{\NormalTextInMathMode}[1]{%
%   {\color{jNormalTextsFGColor} \text{#1}}
% }




% Command that sets color of some mathematical text to the same color
% that has normal text in header (?)

% Beamer version of the command
\newcommand{\MathTextFrametitleFGColor}[1]{%
  {\color{jFrametitleFGColor}
    \setbeamercolor{math text}{fg=jFrametitleFGColor} #1}
}

% Article and similar classes version of the command
% \newcommand{\MathTextWhiteColor}[1]{{\color{jFrametitleFGColor} #1}}





% Command for setting color of alert text for some text in math modestyle

% Beamer version of the command
\newcommand{\MathTextAlertColor}[1]{%
  {\color{jOrange} \setbeamercolor{math text}{fg=jOrange} #1}
}

% Article and similar classes version of the command
% \newcommand{\MathTextAlertColor}[1]{{\color{jOrange} #1}}





% Command that allow you to sets chosen color as the color of some text into
% math mode. Due to some nuances in the way that Beamer handle colors
% it not work in all cases. We hope that in the future we will improve it.

% Beamer version of the command
\newcommand{\SetMathTextsColor}[2]{%
  {\color{#1} \setbeamercolor{math text}{fg=#1} #2}
}


% Article and similar classes version of the command
% \newcommand{\SetMathTextColor}[2]{{\color{#1} #2}}










% ---------------------------------------
% Commands for setting background pictures for some slides
% ---------------------------------------
\newcommand{\TitleBackgroundPicture}
{./PresentationPictures/CommonPictures/Cute_dragon_BG_dark.png}
\newcommand{\SectionBackgroundPicture}
{./PresentationPictures/CommonPictures/Cute_dragon_small_BG_light.png}



\newcommand{\TitleSlideWithPicture}{
  \begingroup

  \usebackgroundtemplate{ % \hspace*{-11.5em}
    \includegraphics[height=\paperheight]{\TitleBackgroundPicture}}

  \maketitle

  \endgroup
}





\newcommand{\SectionSlideWithPicture}[1]{%
  \begingroup

  \usebackgroundtemplate{ % \hspace*{-11.5em}
    \includegraphics[height=\paperheight]{\SectionBackgroundPicture}}

  \setbeamercolor{titlelike}{fg=normal text.fg}

  \section{#1}

  \endgroup
}





\newcommand{\EndingSlide}[1]{%
  \begin{frame}[standout]

    \begingroup

    \color{jFrametitleFGColor}

    #1

    \endgroup

  \end{frame}
}










% ---------------------------------------
% Packages, libraries and their configuration
% ---------------------------------------
% \usepackage{latexgeneralcommands}
% \usepackage{mathcommands}





% ---------------------------------------
% Configuration for this particular presentation
% ---------------------------------------










% ---------------------------------------------------------------------
\title[Terrarium]{NKFowy Grupa Miłośników Python
  Terrarium}
\subtitle{Spotkanie pierwsze. Oby nie ostatnie.}

\author{Kamil Ziemian}


\institute{Uniwersytet Jagielloński w~Krakowie}

\date[27 III 2014]{27 marca 2014 r.}
% ---------------------------------------------------------------------










% ####################################################################
% Początek dokumentu
\begin{document}
% ####################################################################





% Wyrównanie do lewej z łamaniem wyrazów

\RaggedRight





% ######################################
\maketitle % Tytuł całego tekstu
% ######################################





% ######################################
\begin{frame}
  \frametitle{Plan prezentacji}


  \tableofcontents % Spis treści

\end{frame}
% ######################################





% ##################
\begin{frame}
  \frametitle{Cel i~plan}


  Cel spotkań \\
  Wspólna nauka języka programowania Python i jego zastosowań do
  rozwiązywania problemów naukowych symbolicznie i numerycznie, analizy
  i~wizualizacji danych etc.

  W~planie
  \begin{itemize}
    \RaggedRight

  \item Podstawy języka i~podstawowe wiadomości z informatyki.

  \item IPython.

  \item NumPy/SciPy.

  \item MathPlotLib.

  \item Zadania.

  \item Wasze pomysły.

  \end{itemize}

\end{frame}
% ##################





% ##################
\begin{frame}
  \frametitle{Przywitajmy się}


  % #############
  \begin{figure}

    \centering

    \includegraphics[height=1.8in]
    {./PresentationPictures/Guido-von-Rossum-01.jpeg}
    \includegraphics[height=1.8in]
    {./PresentationPictures/Guido-von-Rossum-02.jpg}


    \caption{Guido van Rossum (1956--), holender}

  \end{figure}
  % #############

\end{frame}
% ##################





% ##################
\begin{frame}
  \frametitle{Dlaczego Python?}


  \begin{itemize}
    \RaggedRight

  \item Jest bardzo prostym językiem.

  \item Ma dobrą i~przejrzystą składnie.

  \item Zawiera zaawansowane i dobrze dobrane typy danych.

  \item Jest interpretowany.

  \item Posiada OGROMNY zasób bibliotek.

  \item Powszechnie stosowany.

  \item Dynamicznie rozwijany.

  \item Wieloparadygmatowy.

  \end{itemize}


  Problemy
  \begin{itemize}
    \RaggedRight

  \item Jest często zbyt wolny.

  \end{itemize}

\end{frame}
% ##################





% ##################
\begin{frame}
  \frametitle{Kilka ważnych informacji}


  Uwagi o Python \\
  \begin{itemize}
    \RaggedRight

  \item Jest dwustandardowy: 2.7x, 3.x (3.3.x). Wersje te nie są wstecznie
    kompatybilne.

  \item „Rzeczy niebezpieczne mają być utrudnione, lecz nie zabronione.”

  \item Doświadczenie z nauki można łatwo przenieść na wiele innych języków.

  \end{itemize}

  Uwagi o~Terrarium \\
  \begin{itemize}
    \RaggedRight

  \item Będzie trochę matematyki.

  \item Prowadzący spotkania nie jest najlepszy. Za to ma oddane wsparcie.

  \end{itemize}

\end{frame}
% ##################





% ##################
\begin{frame}
  \frametitle{Ilustracja}

  Python
  def factorial(n): \\
  \hspace{5mm} if n > 1: \\
  \hspace{10mm} return n * factorial(n - 1) \\
  \hspace{5mm} else: \\
  \hspace{10mm} return 1 \\

  \vspace{8pt}

  print factorial(4)

\end{frame}
% ##################





% ##################
\begin{frame}[fragile]
  \frametitle{Ilustracja}


  C \\
\begin{verbatim}
int factorial(int n) {
  if (n > 1) {
    return n*factorial(n - 1);
  } else {
    return 1;
  }
}

int main() {
  printf("%i\n", factorial(4));

  return 0;
}
\end{verbatim}

\end{frame}
% ##################





% ##################
\begin{frame}[fragile]
  \frametitle{Ilustracja}


  Common Lisp
\begin{verbatim}
(defun factorial (x)
  (if (> x 1)
    (* x (factorial (- x 1)))
    1))

(factorial 4)
\end{verbatim}

\end{frame}
% ##################





% ##################
\begin{frame}
  \frametitle{Krótka lista innych bibliotek/rozszerzeń}


  \begin{itemize}

  \item PyFeyn. (Nie udało mi się uruchomić.)

  \item Biopython.

  \item Astropysics.

  \item PsychoPy.

  \end{itemize}

\end{frame}
% ##################

% ##################
\begin{frame}
  \frametitle{Materiały}


  \begin{itemize}

  \item Oficjalna strona Pythona: http://www.python.org/

  \item Stowarzyszenie Polska Grupa Użytkowników Pythona:
    http://pl.python.org

  \item Kurs z MIT „Introduction to Computer Science and Programming”.
    Specjalne podziękowania dla Johna Guttaga, Erica Grimsona i~MIT.

  \item Zanurkuj w Pythonie.

  \item A. B. Downey „Think Python”, „Think Complexity”
    http://greenteapress.com

  \end{itemize}

\end{frame}
% ##################





% ##################
\begin{frame}
  \frametitle{Materiały}


  \begin{itemize}

  \item M. Newman „Computational Physics with Python”
    http://www-personal.umich.edu/~mejn/computational-physics/

  \item E. Ayars „Computational Physics with Python”

  \item Oficjalna grupa dyskusyjna: comp.lang.python

  \item Cała masa innych.

  \end{itemize}

\end{frame}
% ##################





% ##################
\EndingSlide{Dziękuję. Pytania?}
% ##################










% ############################

% Koniec dokumentu
\end{document}
