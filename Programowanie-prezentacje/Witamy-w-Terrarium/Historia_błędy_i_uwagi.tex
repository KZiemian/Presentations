\documentclass[a4paper,11pt]{article}
\usepackage[utf8]{inputenc}% Pozwala pisać polskie znaki bezpośrednio.
\usepackage[polish]{babel}% Tłumaczy na polski teksty automatyczne LaTeXa i pomaga z typografią.
\usepackage[MeX]{polski}% Polska notacja we wzorach matematycznych. Ładne polskie
\usepackage{vmargin}
%----------------------------------------------------------------------------------------
%	MARGINS
%----------------------------------------------------------------------------------------
\setmarginsrb           { 0.7in}  % left margin
                        { 0.6in}  % top margin
                        { 0.7in}  % right margin
                        { 0.8in}  % bottom margin
                        {  20pt}  % head height
                        {0.25in}  % head sep
                        {   9pt}  % foot height
                        { 0.3in}  % foot sep
%\usepackage{amsfonts}% Czcionki matematyczne od American Mathematic Society.
%\usepackage{amsmath}% Dalsze wsparcie od AMS. Więc tego, co najlepsze w LaTeX, czyli trybu
%matematycznego.
%\usepackage{amscd}% Jeszcze wsparcie od AMS.
%\usepackage{latexsym}% Więcej symboli.
%\usepackage{textcomp}% Pakiet z dziwnymi symbolami.
%\usepackage{xy}% Pozwala rysować grafy.
%\usepackage{tensor}% Pozwala prosto używać notacji tensorowej. Albo nawet pięknej notacji
%tensorowej:).
\usepackage{hyperref}
\usepackage{graphicx}

\newcommand{\Center}[1]{\begin{center} #1 \end{center}}
\newcommand{\tb}{\textbf}
\newcommand{\Str}[1]{\textbf{Str. #1.}}
\newcommand{\StrWg}[2]{\textbf{Str. #1, wiersz #2.}}
\newcommand{\StrWd}[2]{\textbf{Str. #1, wiersz #2 (od dołu).}}
\newcommand{\noi}{\noindent}
\newcommand{\start}{\noi \tb{--} {}}
\newcommand{\Jest}{\textbf{Jest: }}
\newcommand{\Pow}{\textbf{Powinno być: }}
\newcommand{\Prze}{\textbf{Przemyśl.}}
\newcommand{\Dok}{\textbf{Dokończ.}}
\newcommand{\Field}[1]{ \begin{center} {\Large \tb{#1} } \end{center} }
\newcommand{\Work}[1]{ \begin{center} {\large #1} \end{center} }

\renewcommand{\arraystretch}{1.2}



\begin{document}


\Field{Historia, błędy i~uwagi.}



\Work{
  Nicolas Bourbaki \\
  ,,Elementy historii matematyki'', \cite{NBEHM}. }

Błędy:\\
\begin{center}
  \begin{tabular}{|c|c|c|c|c|}
    \hline
    & \multicolumn{2}{c|}{} & & \\
    Strona & \multicolumn{2}{c|}{Wiersz}& Jest & Powinno być \\ \cline{2-3}
    & Od góry & Od dołu &  &  \\ \hline
    & & & & \\
    8 & 11 & & metodzie '' & metodzie'' \\
    82 & & 9 & \emph{metafizycznego}''; & \emph{metafizycznego}''); \\
    & & & & \\
    & & & & \\
    & & & & \\ \hline
  \end{tabular}
\end{center}


\Work{
  C. B. Boyer \\
  ,,Historia rachunku różniczkowego i~całkowego i~rozwój jego pojęć'',
  \cite{Boy64}. }


\Center{Uwagi:}

\start W~całej książce angielskie zwarte i~treściwe słowo ,,calculus''
jest zastąpione długim polskim terminem ,,rachunek różniczkowy
i~całkowy'', co często prowadzi do bardzo niezgrabnych stylistycznie
zdań. Lepiej byłoby wprowadzi do książki, obok powyższego, termin
,,analiza matematyczna'', który można ładnie skrócić do ,,analizy''.

\vspace{2em}

\noi \textbf{Konkretne strony.}

\start \StrWd{18}{2} Umieszczenie w~tym samym zdaniu stwierdzenia
o~ścisłym sformułowaniu analizy już u~jej początków oraz faktu,
że~matematycy byli niewrażliwi na pewne subtelności, jest dość
karkołomne. Nie wspominając już o~tym, że~te ,,subtelności'' były
często bardzo poważne.

\start \StrWd{19}{18} Użyte tu określenie ,,mistycyzm imaginacyjnej
spekulacji'' jest wyraźnie niesprawiedliwe w~stosunku do metafizyki,
najważniejszego działu filozofii. Nie~zmienia tego fakt, że~możliwe
iż~odnosi~się tylko do transcendentalnej metafizyki ze~szkoły Kanta.

\start \Str{23} Stwierdzenie, że pewne podstawowe idea zostały
usunięte z~analizy matematycznej, szerzej zaś, z~matematyki, są~mocno
wątpliwe.

\start \StrWd{26}{6} Nazwanie podanych wyżej pojęć ,,sztucznymi'',
ciężko jest mi nazwać czymś innym, niż nieczułością na piękno
matematyki.

\start \Str{28} Ponieważ drugie wydanie tej książki ukazało~się w~1949
r., autor nie mógł wiedzieć, że~w~latach 60 XX w., głównie za sprawą
prac Abrahama Robinsona zostanie sformułowana analiza niestandardowa,
oparta na ścisły pojęciu nieskończenie małych liczb.

Błędy:\\
\begin{center}
  \begin{tabular}{|c|c|c|c|c|}
    \hline
    & \multicolumn{2}{c|}{} & & \\
    Strona & \multicolumn{2}{c|}{Wiersz}& Jest & Powinno być \\ \cline{2-3}
    & Od góry & Od dołu &  &  \\ \hline
    29 & & 4 & [(376] & ([376] \\
    42 & 13 & & [402 & [402] \\
    & & & & \\
    & & & & \\ \hline
  \end{tabular}
\end{center}


\Work{
  Christopher A. Ferrara\\
  ,,Liberty: The God That Failed'', \cite{Fer12}. }

Błędy:\\
\begin{center}
  \begin{tabular}{|c|c|c|c|c|}
    \hline
    & \multicolumn{2}{c|}{} & & \\
    Strona & \multicolumn{2}{c|}{Wiersz}& Jest & Powinno być \\ \cline{2-3}
    & Od góry & Od dołu &  &  \\ \hline
    & & & & \\
    23 & 3 & & in & in an other \\
    40 & & 1 & \emph{St. Saint} & \emph{Saint} \\
    207 & & 2 & Government & \emph{Government} \\
    227 & 11 & & doing-not & doing--not \\
    & & & & \\
    & & & & \\ \hline
  \end{tabular}
\end{center}

\begin{center}
  Martin Gilbert \\
  ,,Pierwsza wojna światowa'', \cite{Gil03}.
\end{center}

Uwagi:
\begin{itemize}
\item[] \StrWg{70}{2} Nie wiem kto popełnił błąd, żołnierz, autor czy
  tłumacz, ale to zdanie o martwym doboszu jest bez sensu.
\item[] \StrWd{70}{5} To zdanie jest na~pewno źle przetłumaczone,
  ale~nie~wiem jakie je poprawić.
\end{itemize}

Błędy:\\
\begin{center}
  \begin{tabular}{|c|c|c|c|c|}
    \hline
    & \multicolumn{2}{c|}{} & & \\
    Strona & \multicolumn{2}{c|}{Wiersz}& Jest & Powinno być \\ \cline{2-3}
    & Od góry & Od dołu &  &  \\ \hline
    & & & & \\
    21 & 1 & & 1996 & 1993 \\
    68 & & 2 & Wielka Brytania & Rosja \\
    69 & 2 & & kulturowo & kulturowo'' \\
    69 & 16 & & 1 sierpnia & 12 sierpnia \\
    70 & 8 & & Pułk Feuchtingera, kiedy & Kiedy pułk Feuchtingera \\
    70 & & 4 & Sir Edward Gray, kiedy & Kiedy sir Edward Gray \\
    122 & & 10 & dal & dał \\
    177 & 15 & & ,,Walcie, aż lufy pękną''. & <<Walcie, aż lufy pękną>>''. \\
    & & & & \\ \hline
  \end{tabular}
\end{center}

% \noindent\\
% \tb{Jest:} nieczystości pierworodnej córki jest ślepotą ducha. \\
% \tb{Powinno być:} pierworodną córką nieczystości jest ślepota ducha. \\


\begin{center}
  Paul Johnson\\
  ,,Historia świata XX wieku, od Rewolucji Październikowej do
  <<Solidarności>>. Tom I.'', \cite{Joh09a}.
\end{center}

Uwagi:
\begin{itemize}
\item[--] \StrWd{70}{4} Terminy id, ego, superego nie zostały
  wprowadzone przez Freuda, lecz przez jego tłumaczy na język
  angielski. Sam Freud używał zwykłych słów z~języka niemieckiego:
  das~Es, Ich, Uberich. (Powinno to być omówione w książce Burzyńskiej
  i~Markowskiego \cite{BM09}).
\item[--] \Str{56} Książka Keynesa \emph{Ekonomiczne konsekwencje
    pokoju}, nie mogła ukazać~się pod koniec 1917
  roku. Najprawdopodobniej chodzi tu o~koniec roku 1919. Powinno~się
  tu też znaleźć obszerniejsze omówienie treści tej książki.
\item[--] \Str{63} Głosowanie nad traktatem o~którym tu mowa
  nie~odbyło~się w~marcu 1919 roku, lecz w~marcu 1920 r.
\item[--] \Str{72} Wyrażoną tu opinię, że~Polska skorzystała z~obawy
  Wielkiej Brytanii przed zalewem bolszewizmu, warto skonfrontować
  z~tym co wielokrotnie mówił na dostępnych na YouTubie wystąpieniach
  Andrzej Nowak i~co opisał w~,,Pierwszej zdradzie zachodu''.
\end{itemize}

Błędy:\\
\begin{center}
  \begin{tabular}{|c|c|c|c|c|}
    \hline
    & \multicolumn{2}{c|}{} & & \\
    Strona & \multicolumn{2}{c|}{Wiersz}& Jest & Powinno być \\ \cline{2-3}
    & Od góry & Od dołu &  &  \\ \hline
    & & & & \\
    15 & 17 & & Mendla & prac Mendla \\
    28 & 8 & & cywilizacyjne & cywilizowane \\
    36 & 10 & & zastąpić & zaspokoić \\
    51 & & & Jedyny & Jeden \\
    54 & 11 & & [dotyczących planu) & [dotyczących planu] \\
    & & & ] & \\
    55 & & 1 & M. Keynes & J. M. Keynes \\
    64 & 2 & & kształt ów & ów kształt \\
    65 & & 14 & późniejszy doradca & doradca \\
    68 & 5 & & szśćdziesiątych & sześćdziesiątych \\
    82 & 6 & & Ghandi: & Ghandi. \\
    89 & 5 & & odbyć & przebiegać \\
    & & & & \\ \hline
  \end{tabular}
\end{center}

\noindent\\
\tb{Okładka.} \\
\Jest ''Solidarności'' \\
\Pow ,,Solidarności'' \\
\Str{1} \\
\Jest \tb{Historia świata} \\
\Pow \tb{Historia świata XX wieku} \\
\Str{3} \\
\Jest \tb{Historia świata} \\
\Pow \tb{Historia świata XX wieku} \\

\begin{center}
  Paul Johnson\\
  ,,Narodziny nowoczesności'', \cite{Joh95}.
\end{center}

Błędy:\\
\begin{center}
  \begin{tabular}{|c|c|c|c|c|}
    \hline
    & \multicolumn{2}{c|}{} & & \\
    Strona & \multicolumn{2}{c|}{Wiersz}& Jest & Powinno być \\ \cline{2-3}
    & Od góry & Od dołu &  &  \\ \hline
    & & & & \\
    29 & 2 & & cali, członie & cali, o członie \\
    142 & 1 & & Barbaji & Barbajowi \\
    142 & & 14 & w nową operą & z nową operą \\
    345 & 15 & & XIX & XVIII \\
    345 & 18 & & od & na od \\
    409 & 8 & & sposób & nie sposób \\ \hline
  \end{tabular}
\end{center}

\begin{center}
  Tony Judt \\
  ,,Powojnie. Historia Europy od roku 1945.'', \cite{Jud13}.
\end{center}

Uwagi:
\begin{itemize}
\item[--] \Str{28} Uwaga, że~jeszcze w~latach trzydziestych
  XIX~w. babcie hiszpańskie straszyły dzieci Napoleonem, jest zapewne
  pomyłkom. Wojny napoleońskie skończyły~się dopiero w~1815 roku, więc
  pamiętanie o~nich po kilkunastu latach nie~jest niczym
  niezwykłym. Zapewne chodziło o~lata trzydzieste XX~w.
\end{itemize}

Błędy:\\
\begin{center}
  \begin{tabular}{|c|c|c|c|c|}
    \hline
    & \multicolumn{2}{c|}{} & & \\
    Strona & \multicolumn{2}{c|}{Wiersz}& Jest & Powinno być \\ \cline{2-3}
    & Od góry & Od dołu &  &  \\ \hline
    & & & & \\
    35 & & 17 & miast, prądu & miast, lecz prądu \\
    58 & & 7 & postępującą & postępującej \\
    58 & & 6 & degeneracją & degeneracji \\
    & & & & \\ \hline
  \end{tabular}
\end{center}

\begin{center}
  J. Kofman, W. Roszkowski \\
  ,,Transformacja i postkomunizm'', \cite{KR99}.
\end{center}

Błędy:\\
\begin{center}
  \begin{tabular}{|c|c|c|c|c|}
    \hline
    & \multicolumn{2}{c|}{} & & \\
    Strona & \multicolumn{2}{c|}{Wiersz}& Jest & Powinno być \\ \cline{2-3}
    & Od góry & Od dołu &  &  \\ \hline
    & & & & \\
    10 & & 17 & jedynie & jedynej \\
    13 & & 4 & o ekspansji & do ekspansji \\
    16 & 9 & & marntrawstwem & marnotrawstwem \\
    & & & & \\
    & & & &  \\ \hline
  \end{tabular}
\end{center}


\begin{center}
  Andrzej Nowak \\
  ,,Strachy i lachy. Przemiany polskiej pamięci 1982-2012.'',
  \cite{Now12}.
\end{center}

\Center{Uwagi:}

\start \Str{47} T.~S.~Eliot jest na tej stronie nazwany ,,wielkim
poetą katolickim'', acz z~tego co wiem do Kościoła nigdy nie
przyszedł, zamiast tego dołączył do jakiegoś wyznania
anglokatolickiego. Zaś użycie przymiotnika ,,wielki'' w~odniesieniu to
tego poety, którego twórczości nie da~się czytać, jest już na~pewno
błędem.


% Błędy:\\
% \begin{center}
%   \begin{tabular}{|c|c|c|c|c|}
%     \hline
%     & \multicolumn{2}{c|}{} & & \\
%     Strona & \multicolumn{2}{c|}{Wiersz}& Jest & Powinno być \\ \cline{2-3}
%     & Od góry & Od dołu &  &  \\ \hline
%     & & & & \\
%     & & & & \\ \hline
%   \end{tabular}
% \end{center}


\begin{center}
  Andrzej Nowak \\
  ,,Intelektualna historia III RP. Rozmowy z lat 1990--2012.'',
  \cite{Now13}.
\end{center}

Błędy:\\
\begin{center}
  \begin{tabular}{|c|c|c|c|c|}
    \hline
    & \multicolumn{2}{c|}{} & & \\
    Strona & \multicolumn{2}{c|}{Wiersz}& Jest & Powinno być \\ \cline{2-3}
    & Od góry & Od dołu &  &  \\ \hline
    & & & & \\
    & & & & \\ \hline
  \end{tabular}
\end{center}
\noindent\\
\tb{Przednia okładka, wiersz 14.}\\
\Jest \emph{ImperologicalStudies.APolishPerspective}(2011);\emph{Czaswalki} \\
\Pow \emph{Imperological Studies. A Polish Perspective} (2011);\emph{Czas walki} \\
\tb{Przednia okładka, wiersz 10 (od dołu).}\\
\Jest ...w Brnie \\
\Pow w Brnie. \\


\Work{
  Andrzej Nowak \\
  ,,Dzieje Polski. Tom I do 1202: Skąd nasz ród.'', \cite{Now14a}. }

Uwagi: \\
\start \tb{Strona tytułowa.} W informacjach o~autorze jest podane,
że~był redaktorem naczelnym ARCANA w~latach 1994--2012, lecz
prawidłowy okres to 1995--2012. \\
\start \Str{96} W~swoim wykładzie z~cyklu \href{https://www.youtube.com/watch?v=QovVLT2fitc}{,,Filary Polskości: Mieszko i~Bolesław''} Nowak znacznie wyraźniej niż w~tej książce, pokazał cynizm polityczny Mieszka~I. O~możliwości takiego spojrzenia na tego władcę mówi on otwarcie na tym wykładzie, a~który to zarzut próbuje tak tam, jak i~tu, przykryć określeniem jego działań mianem ,,majstersztyku polityki polskiej''.

Błędy:\\
\begin{center}
  \begin{tabular}{|c|c|c|c|c|}
    \hline
    & \multicolumn{2}{c|}{} & & \\
    Strona & \multicolumn{2}{c|}{Wiersz}& Jest & Powinno być \\ \cline{2-3}
    & Od góry & Od dołu &  &  \\ \hline
    & & & & \\
    41 & 3 & & W X w. jeszcze & Jeszcze w X w. \\
    53 & & 1 & Księga Wyjścia & Księga Rodzaju \\
    61 & 10 & & do dziejów & dla dziejów \\
    % & & & & \\
    % & & & & \\
    % & & & & \\
    & & & & \\ \hline
  \end{tabular}
\end{center}
\StrWg{61}{8} \\
\Jest z~Rocznika kapituły krakowskiej dawnego \ldots \\
\Pow z~dawnego Rocznika kapituły krakowskiej \ldots \\


\begin{center}
  Red. A. Nowak \\
  ,,Historie Polski w~XIX wieku. Tom I: Kominy, ludzie i~obłoki:
  modernizacja i~kultura.'', \cite{HPXIX1}.
\end{center}

% Uwagi:
% \begin{itemize}
% \item[--] \Str{45} T.~S.~ Eliot jest na tej stronie nazwany
%   ,,wielkim poetą katolickim'', acz z~tego co wiem do Kościoła nigdy
%   nie przyszedł, zamiast tego dołączył do jakiegoś wyznania
%   anglokatolickiego. Zaś przymiotnik ,,wielki'' w~odniesieniu to
%   tego poety, którego twórczości nie da~się czytać, jest już
%   na~pewno błędny.
% \end{itemize}

Błędy:\\
\begin{center}
  \begin{tabular}{|c|c|c|c|c|}
    \hline
    & \multicolumn{2}{c|}{} & & \\
    Strona & \multicolumn{2}{c|}{Wiersz}& Jest & Powinno być \\ \cline{2-3}
    & Od góry & Od dołu &  &  \\ \hline
    & & & & \\
    16 & & 15 & równości równość & równości \\ \hline
  \end{tabular}
\end{center}



\begin{center}
  R. Rhodes \\
  ,,Jak powstała bomba atomowa'', \cite{Rho00}.
\end{center}

Błędy:\\
\begin{center}
  \begin{tabular}{|c|c|c|c|c|}
    \hline
    & \multicolumn{2}{c|}{} & & \\
    Strona & \multicolumn{2}{c|}{Wiersz}& Jest & Powinno być \\ \cline{2-3}
    & Od góry & Od dołu &  &  \\ \hline
    & & & & \\
    719 & & 6 & 1993 & 1933 \\
    & & & & \\ \hline
  \end{tabular}
\end{center}

\begin{center}
  W. Roszkowski \\
  ,,Najnowsza historia Polski: 1914--1939'', \cite{Ros11a}.
\end{center}

Uwagi:
\begin{itemize}
\item[--] Karygodną cechą całego tego wydania ,,Najnowszej historii
  Polski'', jest nieumieszczenie w~każdym tomie listy używanych w~nim
  skrótów, niezależnie od tego, czy zostały one pierwszy raz użyte
  w~tym, czy~w~poprzednim tomie.
\item[--] \Str{26} Podany tu opis przyczyn wybuchu I~Wojny Światowej,
  zwłaszcza bardzo silne stwierdzenie, że~Austro\dywiz Węgry
  wypowiedziały wojnę Serbii pod naciskiem Niemiec, warto
  skonfrontować z~tym co pisze M. Gilbert w~swojej książce na temat
  tego przedziwnego wydarzenia \cite{Gil03}.
\item[--] \StrWd{78}{4} W~tym wierszu jest niedomknięty cudzysłów,
  który sprawie, że~nie wiadomo czy dany fragment jest cytatem,
  a~jeśli tak to gdzie~się zaczyna.
\end{itemize}

Błędy:\\
\begin{center}
  \begin{tabular}{|c|c|c|c|c|}
    \hline
    & \multicolumn{2}{c|}{} & & \\
    Strona & \multicolumn{2}{c|}{Wiersz}& Jest & Powinno być \\ \cline{2-3}
    & Od góry & Od dołu &  &  \\ \hline
    & & & & \\
    20 & & 9 & sita & siła \\
    27 & & 16 & z agrozić & zagrozić \\
    31 & & 8 & Hans Beseler & Hans von Beseler \\
    36 & 3 & & POW & POW. \\
    37 & 12 & & LLOYDA & LOYDA \\
    47 & & 6 & przed nadchodzącą zimą & nadchodzącej zimy \\
    50 & 21 & & \emph{Pobki} & \emph{Polski} \\
    73 & & 13 & 1920 R & 1920 R. \\
    & & 17 & W braku & Z braku \\
    & & & & \\ \hline
  \end{tabular}
\end{center}

\begin{center}
  A. K. Wróblewski \\
  ,,Historia fizyki'', \cite{Wro06}.
\end{center}

Błędy:\\
\begin{center}
  \begin{tabular}{|c|c|c|c|c|}
    \hline
    % & \multicolumn{2}{c|}{} & & \\
      & \multicolumn{2}{c|}{Wiersz} & & \\ \cline{2-3}
    Strona & Od góry & Od dołu & Jest & Powinno być \\ 
      & (kolumna) & (kolumna) & & \\ \hline
      & & & & \\
    203 & 3 (2) & & Jacob 'sGravesande'a & Jacob's Gravesande'a \\
      & & & & \\ \hline
  \end{tabular}
\end{center}


\bibliographystyle{ieeetr} \bibliography{Bibliography}{}



\end{document}
