\documentclass[a4paper,11pt]{article}
\usepackage[utf8]{inputenc}% Pozwala pisać polskie znaki bezpośrednio.
\usepackage[polish]{babel}% Tłumaczy na polski teksty automatyczne LaTeXa i pomaga z typografią.
\usepackage[MeX]{polski}% Polska notacja we wzorach matematycznych. Ładne polskie
% \usepackage[T1]{fontenc}% Pozwala pisać znaki diakrytyczne z języków innych
\usepackage{amsfonts}% Czcionki matematyczne od American Mathematic Society.
\usepackage{amsmath}% Dalsze wsparcie od AMS. Więc tego, co najlepsze w LaTeX, czyli trybu
% matematycznego.
\usepackage{amscd}% Jeszcze wsparcie od AMS.
\usepackage{latexsym}% Więcej symboli.
\usepackage{upgreek}%Lepsze greckie czcionki. Przyklad skladni: pi = \uppi
\usepackage{textcomp}% Pakiet z dziwnymi symbolami.
% \usepackage{slashed}% Pozwala pisać slash Feynmana.
\usepackage{xy}% Pozwala rysować grafy.
\usepackage{tensor}% Pozwala prosto używać notacji tensorowej. Albo nawet pięknej notacji
% tensorowej:).
\usepackage{graphicx}% Pozwala wstawiać grafikę.
\usepackage{hyperref}
\usepackage{vmargin}
% ----------------------------------------------------------------------------------------
% MARGINS
% ----------------------------------------------------------------------------------------
\setmarginsrb { 0.7in} % left margin
{ 0.6in} % top margin
{ 0.7in} % right margin
{ 0.8in} % bottom margin
{ 20pt} % head height
{0.25in} % head sep
{ 9pt} % foot height
{ 0.3in} % foot sep

\newcommand{\Center}[1]{\begin{center} #1 \end{center}}
\newcommand{\e}{\mathrm{e}} 
\newcommand{\tb}{\textbf}
\newcommand{\noi}{\noindent}
\newcommand{\ii}{\mathrm{i}}
\newcommand{\de}{\mathrm{d}} 
\newcommand{\dA}[1]{\, d#1}
\newcommand{\dd}[3]{\frac{ \de^{ #1 } #2 }{ \de #3^{ #1 } }} 
\newcommand{\pd}[3]{\frac{ \partial^{ #1 } #2 }{ \partial #3^{ #1 } }} 
\newcommand{\s}{\sigma}
\newcommand{\Str}[1]{\textbf{Str. #1.}}
\newcommand{\StrWg}[2]{\textbf{Str. #1, wiersz #2.}}
\newcommand{\StrWd}[2]{\textbf{Str. #1, wiersz #2 (od dołu).}}
\newcommand{\Jest}{\textbf{Jest: }}
\newcommand{\Pow}{\textbf{Powinno być: }}
\newcommand{\Prze}{\textbf{Przemyśl.}}
\newcommand{\Dok}{\textbf{Dokończ.}}
\newcommand{\Field}[1]{ \begin{center} {\Large \tb{#1} } \end{center} }
\newcommand{\Work}[1]{ \begin{center} {\large #1} \end{center} }

\renewcommand{\arraystretch}{1.2}



\begin{document}



\Field{Analiza funkcjonalna, błędy i~uwagi.}

,,P\ldots'' oznacza, że w wydaniu ,,\ldots'' błąd został poprawiony.\\



\Work{ A. V. Balakrishnan \\
  ,,Analiza funkcjonalna stosowana'', \cite{Bal92}. }

% Uwagi:
% \begin{itemize}
% \item[--]
% \end{itemize}

Błędy:\\
\begin{center}
  \begin{tabular}{|c|c|c|c|c|}
    \hline
    & \multicolumn{2}{c|}{} & & \\
    Strona & \multicolumn{2}{c|}{Wiersz}& Jest & Powinno być \\ \cline{2-3}
    & Od góry & Od dołu &  &  \\ \hline
    & & & & \\
    13 & & 10 & $\int^{ 1 }_{ -1 }$ & $2 \int^{ 1 }_{ -1 }$ \\
    14 & 15 & & $<$ & $\leq$ \\
    15 & 18 & & otrzymywaliśmy & otrzymalibyśmy \\
    & & & & \\ \hline
  \end{tabular}
\end{center}

\Work{I. M. Gelfand, G. E. Shilov \\
  ,,Generalized Functions: Volume I, Properties and Operations'',
  \cite{GS64}. }

% Uwagi:
% \begin{itemize}
% \item[--]
% \end{itemize}

Błędy:\\
\begin{center}
  \begin{tabular}{|c|c|c|c|c|}
    \hline
    & \multicolumn{2}{c|}{} & & \\
    Strona & \multicolumn{2}{c|}{Wiersz}& Jest & Powinno być \\ \cline{2-3}
    & Od góry & Od dołu &  &  \\ \hline
    & & & & \\
    12 & 4 & & $f_{ 0 } = 0$ & $x_{ 0 } = 0$ \\
    28 & 16 & & $r \leq a$ & $r \geq a$ \\
    & & & & \\ \hline
  \end{tabular}
\end{center}


\Work{ M. Reed, B. Simon \\
  ,,Methods of Modern Mathematical Physics I: Functional Analysis'',
  \cite{RS80}. }



\Center{Uwagi:} \noi \tb{Rozdział I.}

\tb{--} Nie została podana definicja $\sup$ i~$\inf$ ani w~przypadku
ogólny relacji porządku, ani w~przypadku liczb rzeczywistych.

\tb{--} Nie~zdefiniowano zbiorów zwartych, ani~nie~wskazano, że~zbiór
domknięty i~ograniczony w $\mathbb{R}^{ n }$ jest zwarty.

\tb{--} Nie podana została definicja sumy miar borelowskich,
ani~nie~zostało udowodnione, że funkcja całkowalna względem miar
$\mu_{ 1 }$ i~$\mu_{ 2 }$ jest całkowalna względem ich
sumy. Uzupełnienie tych braków nie jest jednak trudne.  Dla dowolnych
dwóch miar dodatnich $\mu_{ 1 }$ i~$\mu_{ 2 }$, niekoniecznie
borelowskich, zdefiniowanych na wspólnej $\s$\dywiz pierścieniu
$\mathcal{R}$, określamy ich sumę jak:
\begin{displaymath}
  (\mu_{ 1 } + \mu_{ 2 })( A ) = \mu_{ 1 }( A ) + \mu_{ 2 }( A ), \quad \forall A \in \mathcal{R}.
\end{displaymath}
Należy zauważyć, że~dla przeliczalnej sumy rozłącznych zbiorów
$A_{ i }$, zachodzi:
\begin{displaymath}
  \begin{split}
    ( \mu_{ 1 } + \mu_{ 2 } )\left( \bigcup_{ i = 1 }^{ \infty } A_{ i } \right) =& \mu_{ 1 }\left( \bigcup_{ i = 1 }^{ \infty } A_{ i } \right) + \mu_{ 2 }\left( \bigcup_{ i = 1 }^{ \infty } A_{ i } \right) = \sum_{ i = 1 }^{ \infty } \mu_{ 1 }( A_{ i } ) + \sum_{ i = 1 }^{ \infty } \mu_{ 2 }( A_{ i } )\\
    =& \sum_{ i = 1 }^{ \infty } ( \mu_{ 1 }( A_{ i } ) + \mu_{ 2 }(
    A_{ i } ) ) = \sum_{ i = 1 }^{ \infty } ( \mu_{ 1 } + \mu_{ 2 } )(
    A_{ i } ).
  \end{split}
\end{displaymath}
Gdy miary są dodatnie nie ma żadnego problemu z~wciągnięciem dwóch sum
szeregów pod jedną sumę, gdy wszystkie zbiory mają miarę skończoną
względem $\mu_{ 1 }$ i $\mu_{ 2 }$, zaś gdy któraś z~tych miar jest
nieskończona, to obie strony są równe $+\infty$.  Jednak nie jestem
pewien, czy w~przypadku miar które nie są dodatnie też to
zachodzi. \Prze

Rozpatrzmy teraz problem całkowania funkcji $f$ całkowalnej względem
dwóch miar dodatnich $\mu_{ 1 }$ i~$\mu_{ 2 }$, względem sumy
miar. Ograniczymy się przy tym dla przypadku funkcji określonej na
$\mathbb{R}$. Jeżeli $f \geq 0$, wtedy z~równości:
\begin{displaymath}
  \begin{split}
    \sum_{ n }( f )_{ \mu_{ 1 } + \mu_{ 2 } } =& \sum_{ m = 0 }^{ \infty } \frac{ m }{ n } ( \mu_{ 1 } + \mu_{ 2 } )\left( f^{ -1 }\left[ \left[ \frac{ m }{ n }, \frac{ m + 1 }{ n } \right) \right] \right) = \sum_{ m = 0 }^{ \infty } \left\{ \frac{ m }{ n } \mu_{ 1 }\left( f^{ -1 }\left[ \left[ \frac{ m }{ n }, \frac{ m + 1 }{ n } \right) \right] \right) \right. \\
    &+ \left. \frac{ m }{ n } \mu_{ 2 }\left( f^{ -1 }\left[ \left[
            \frac{ m }{ n }, \frac{ m + 1 }{ n } \right) \right]
      \right) \right\} = \sum_{ n }( f )_{ \mu_{ 1 } } + \sum_{ n }( f
    )_{ \mu_{ 2 } }.
  \end{split}
\end{displaymath}
Dzięki dodatniości miar nie ma problemu z~przekształceniem szeregów,
jednak przypadek ogólniejszy wciąż pozostaje do zanalizowania. \Prze

Ponieważ w podejściu prezentowanym przez Reeda i Simona, całkowanie
funkcji zespolonych sprowadza~się do całkowania funkcji dodatnich,
widzimy więc, że~funkcja całkowalna względem każdej z~miar dodatnich
z~osobna jest też całkowalna względem ich sumy. Należy to jednak
odróżnić od~przypadku, gdy mamy dwie funkcje $f_{ 1 }$ i~$f_{ 2 }$,
całkowalne odpowiednio względem miar $\mu_{ 1 }$ i~$\mu_{ 2 }$,
i~chcemy powiedzieć coś o~całkowalności ich sumy. Wtedy bowiem może
okazać się, że~funkcja np.~$f_{ 1 }$ jest całkowalna bo jest duża na
zbiorze, którego miara $\mu_{ 1 }$ jest mała, ale którego miara
$\mu_{ 2 }$ jest duża.

Dla przykładu weźmy funkcję $f$ całkowalną względem miary
Lebesgue'a. Możemy zmienić jej wartość w~0 na $+\infty$, bez żadnych
konsekwencji. Jeżeli teraz dodamy do miary Lebesgue'a miarę Diraca
skupioną w~0 to ta funkcja jest niecałkowalna względem takiej miary.

\vspace{2em}

\noi \textbf{Konkretne strony.}

\tb{--} \Str{12} Ponieważ nie została podana definicja podciągu, nie
można było użyć następującej eleganckiej definicji. $b$ jest punktem
skupienia ciągu $\{ a_{ n } \}$, jeśli $\{ a_{ n } \}$ zawiera podciąg
zbieżny do $b$.

\tb{--} \Str{12} Zaproponowana na tej stronie alternatywna definicja
$\overline{\lim}$ ma sens tylko dla zbiorów ograniczonych. Tak jak
poprzednio należy ją rozszerzyć na przypadki zbiorów nieograniczonych.

\tb{--} \Str{17} W~twierdzeniu o~zbieżności monotonicznej jest
powiedziane,
że~$\int | f( p ) - f_{ n }( p ) | \dA{ p } \rightarrow 0$, co wymaga
pewnego komentarza. Ponieważ $f( p ) \geq f_{ n }( p )$ więc
$f( p ) - f_{ n }( p ) \geq 0$ więc zachodzi ciąg równoważności
$\int | f( p ) - f_{ n }( p ) | \dA{ p } = \int ( f( p ) - f_{ n }( p
) ) \dA{ p } = \int f( p ) \dA{ p } - \int f_{ n }( p ) \dA{ p }$.
Dowód tej ostatniej równości można znaleźć w~każdej książce do teorii
całki.

\tb{--} \Str{21} W~podanej charakterystyce liczbowej wzoru Cantora
jest błąd ,,miary zero''. Zbiór Cantora jest skonstruowany jako
dopełnienie do odcinka $[ 0, 1 ]$ zbioru otwartego
$S = \left( \frac{ 1 }{ 3 }, \frac{ 2 }{ 3 } \right) \cup \left(
  \frac{ 1 }{ 9 }, \frac{ 2 }{ 9 } \right) \cup \left( \frac{ 7 }{ 9
  }, \frac{ 8 }{ 9 } \right) \cup \ldots$, stąd zawiera np.~punkt
$\tfrac{ 1 }{ 3 }$, który ma rozwinięcie trójkowe $0,1000\ldots$
Jednak tego typu punktów, jest tylko przeliczalnie wiele, więc
stanowią one ,,mały zbiór''.
% \item

Błędy:\\
\begin{center}
  \begin{tabular}{|c|c|c|c|c|}
    \hline
    & \multicolumn{2}{c|}{} & & \\
    Strona & \multicolumn{2}{c|}{Wiersz}& Jest & Powinno być \\ \cline{2-3}
    & Od góry & Od dołu &  &  \\ \hline
    & & & & \\
    2 & 8 & & \tb{surjective} & surjective \\
    2 & 9 & & surjective & \tb{surjective} \\
    14 & 10 & & $\Sigma_{ 2^{ n } }( f )$ & $\Sigma_{ 2 n }( f )$ \\
    14 & 14 & & $\Sigma_{ 2^{ n } }( f )$ & $\Sigma_{ 2 n }( f )$ \\
    46 & & 9 & $(v_{ 1 }, u_{ 1 } ) v_{ k }$ & $(v_{ 1 }, u_{ 2 } ) v_{ 1 }$ \\
    50 & 2 & & $( \eta, \mu )$ & $( \mu, \eta )$ \\
    50 & 4 & & $( \eta, \mu ) = \bigg( \sum\limits_{ i = 1 }^{ N } c_{ i } ( \varphi_{ i } \otimes \psi_{ i } ), \mu \bigg)$ & $( \mu, \eta ) = \bigg( \mu, \sum\limits_{ i = 1 }^{ N } c_{ i } ( \varphi_{ i } \otimes \psi_{ i } )\bigg)$ \\
    53 & & 12 & $\varphi_{ k_{ \sigma( 1 ) } } \otimes \varphi_{ k_{ \sigma( 2 ) } } \cdots \otimes \varphi_{ k_{ n( p ) } }$ & $\varphi_{ k_{ \sigma( 1 ) } } \otimes \varphi_{ k_{ \sigma( 2 ) }  } \otimes \cdots \otimes \varphi_{ k_{ \sigma( p ) } }$ \\
    & & & & \\ \hline
  \end{tabular}
\end{center}









\bibliographystyle{ieeetr} \bibliography{Bibliography}{}



\end{document}
