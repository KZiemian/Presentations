% ------------------------------------------------------------------------------------------------------------------
% Basic configuration of Beamera and Jagiellonian
% ------------------------------------------------------------------------------------------------------------------
\RequirePackage[l2tabu, orthodox]{nag}



\ifx\PresentationStyle\notset
  \def\PresentationStyle{dark}
\fi



% Options: t -- align text to the top of the frame
\documentclass[10pt,t]{beamer}
\mode<presentation>
\usetheme[style=\PresentationStyle,JUlogotitle=no]{jagiellonian}



% ------------------------------------------------------------------------------------
% Procesing configuration files of Jagiellonian theme located in directory
% "preambule"
% ------------------------------------------------------------------------------------
% Configuration for polish language
% Need description
\usepackage[polish]{babel}
% Need description
\usepackage[MeX]{polski}



% ------------------------------
% Better support of polish chars in technical parts of PDF
% ------------------------------
\hypersetup{pdfencoding=auto,psdextra}

% Package "textpos" give as enviroment "textblock" which is very usefull in
% arranging text on slides.

% This is standard configuration of "textpos"
\usepackage[overlay,absolute]{textpos}

% If you need to see bounds of "textblock's" comment line above and uncomment
% one below.

% Caution! When showboxes option is on significant ammunt of space is add
% to the top of textblock and as such, everyting put in them gone down.
% We need to check how to remove this bug.

% \usepackage[showboxes,overlay,absolute]{textpos}



% Setting scale length for package "textpos"
\setlength{\TPHorizModule}{10mm}
\setlength{\TPVertModule}{\TPHorizModule}


% ---------------------------------------
% TikZ
% ---------------------------------------
% Importing TikZ libraries
\usetikzlibrary{arrows.meta}
\usetikzlibrary{positioning}





% % Configuration package "bm" that need for making bold symbols
% \newcommand{\bmmax}{0}
% \newcommand{\hmmax}{0}
% \usepackage{bm}




% ---------------------------------------
% Packages for scientific texts
% ---------------------------------------
% \let\lll\undefined  % Sometimes you must use this line to allow
% "amsmath" package to works with packages with packages for polish
% languge imported
% /preambul/LanguageSettings/JagiellonianPolishLanguageSettings.tex.
% This comments (probably) removes polish letter Ł.
\usepackage{amsmath}  % Packages from American Mathematical Society (AMS)
\usepackage{amssymb}
\usepackage{amscd}
\usepackage{amsthm}
\usepackage{siunitx}  % Package for typsetting SI units.
\usepackage{upgreek}  % Better looking greek letters.
% Example of using upgreek: pi = \uppi


\usepackage{calrsfs}  % Zmienia czcionkę kaligraficzną w \mathcal
% na ładniejszą. Może w innych miejscach robi to samo, ale o tym nic
% nie wiem.










% ---------------------------------------
% Packages written for lectures "Geometria 3D dla twórców gier wideo"
% ---------------------------------------
% \usepackage{./ProgramowanieSymulacjiFizykiPaczki/ProgramowanieSymulacjiFizyki}
% \usepackage{./ProgramowanieSymulacjiFizykiPaczki/ProgramowanieSymulacjiFizykiIndeksy}
% \usepackage{./ProgramowanieSymulacjiFizykiPaczki/ProgramowanieSymulacjiFizykiTikZStyle}





% !!!!!!!!!!!!!!!!!!!!!!!!!!!!!!
% !!!!!!!!!!!!!!!!!!!!!!!!!!!!!!
% EVIL STUFF
\if\JUlogotitle1
\edef\LogoJUPath{LogoJU_\JUlogoLang/LogoJU_\JUlogoShape_\JUlogoColor.pdf}
\titlegraphic{\hfill\includegraphics[scale=0.22]
{./JagiellonianPictures/\LogoJUPath}}
\fi
% ---------------------------------------
% Commands for handling colors
% ---------------------------------------


% Command for setting normal text color for some text in math modestyle
% Text color depend on used style of Jagiellonian

% Beamer version of command
\newcommand{\TextWithNormalTextColor}[1]{%
  {\color{jNormalTextFGColor}
    \setbeamercolor{math text}{fg=jNormalTextFGColor} {#1}}
}

% Article and similar classes version of command
% \newcommand{\TextWithNormalTextColor}[1]{%
%   {\color{jNormalTextsFGColor} {#1}}
% }



% Beamer version of command
\newcommand{\NormalTextInMathMode}[1]{%
  {\color{jNormalTextFGColor}
    \setbeamercolor{math text}{fg=jNormalTextFGColor} \text{#1}}
}


% Article and similar classes version of command
% \newcommand{\NormalTextInMathMode}[1]{%
%   {\color{jNormalTextsFGColor} \text{#1}}
% }




% Command that sets color of some mathematical text to the same color
% that has normal text in header (?)

% Beamer version of the command
\newcommand{\MathTextFrametitleFGColor}[1]{%
  {\color{jFrametitleFGColor}
    \setbeamercolor{math text}{fg=jFrametitleFGColor} #1}
}

% Article and similar classes version of the command
% \newcommand{\MathTextWhiteColor}[1]{{\color{jFrametitleFGColor} #1}}





% Command for setting color of alert text for some text in math modestyle

% Beamer version of the command
\newcommand{\MathTextAlertColor}[1]{%
  {\color{jOrange} \setbeamercolor{math text}{fg=jOrange} #1}
}

% Article and similar classes version of the command
% \newcommand{\MathTextAlertColor}[1]{{\color{jOrange} #1}}





% Command that allow you to sets chosen color as the color of some text into
% math mode. Due to some nuances in the way that Beamer handle colors
% it not work in all cases. We hope that in the future we will improve it.

% Beamer version of the command
\newcommand{\SetMathTextsColor}[2]{%
  {\color{#1} \setbeamercolor{math text}{fg=#1} #2}
}


% Article and similar classes version of the command
% \newcommand{\SetMathTextColor}[2]{{\color{#1} #2}}










% ---------------------------------------
% Commands for setting background pictures for some slides
% ---------------------------------------
\newcommand{\TitleBackgroundPicture}
{./PresentationPictures/CommonPictures/Cute_dragon_BG_dark.png}
\newcommand{\SectionBackgroundPicture}
{./PresentationPictures/CommonPictures/Cute_dragon_small_BG_light.png}



\newcommand{\TitleSlideWithPicture}{
  \begingroup

  \usebackgroundtemplate{ % \hspace*{-11.5em}
    \includegraphics[height=\paperheight]{\TitleBackgroundPicture}}

  \maketitle

  \endgroup
}





\newcommand{\SectionSlideWithPicture}[1]{%
  \begingroup

  \usebackgroundtemplate{ % \hspace*{-11.5em}
    \includegraphics[height=\paperheight]{\SectionBackgroundPicture}}

  \setbeamercolor{titlelike}{fg=normal text.fg}

  \section{#1}

  \endgroup
}





\newcommand{\EndingSlide}[1]{%
  \begin{frame}[standout]

    \begingroup

    \color{jFrametitleFGColor}

    #1

    \endgroup

  \end{frame}
}










% ------------------------------------------------------
% Packages, libraries and their configuration
% ------------------------------------------------------





% ------------------------------------------------------
% Configuration for this particular presentation
% ------------------------------------------------------










% ------------------------------------------------------------------------------------------------------------------
\title{Fizyka z~przymrużeniem oka}

\author{Autorzy tych żartów}


% \institute{Uniwersytet Jagielloński w~Krakowie}

% \date[11 December 2018]{Seminarium astrofizyczne PAU \\
%   11 grudnia 2019}
% ------------------------------------------------------------------------------------------------------------------










% ####################################################################
% Beginning of the document
\begin{document}
% ####################################################################





% ######################################
% Text is adjusted to the left and words are broken at the end of the line.
\RaggedRight
% Number of chars: 41k+,
% ######################################





% ######################################
\maketitle
% ######################################





% % ######################################
% \begin{frame}
%   \frametitle{Table of contents}


%   \tableofcontents % Spis treści

% \end{frame}
% % ######################################










% % ######################################
% \section{Różne żarty}
% % ######################################



% ##################
\begin{frame}
  \frametitle{???}


  \begin{figure}

    \centering


    \includegraphics[scale=0.23]
    {./Presentations-pictures/Schrodingers-cat-plan.jpg}

  \end{figure}

\end{frame}
% ##################





% ##################
\begin{frame}
  \frametitle{Julia creators}


  % \begin{figure}

  %   \centering

  %   \includegraphics[scale=0.30]{./PresentationPictures/Julia_creators.png}

  % \end{figure}

\end{frame}
% ##################





% % ##################
% \begin{frame}
%   \frametitle{LLVM and all of that}


%   \begin{figure}

%     \includegraphics[scale=0.16]{./PresentationPictures/LLVM_logo.png}


%     \caption{LLVM logo}

%   \end{figure}

% \end{frame}
% % ##################





% % ##################
% \begin{frame}
%   \frametitle{Aims of Julia (partialy by Alan Edelman)}



% \end{frame}
% % ##################





% % ##################
% \begin{frame}
%   \frametitle{Why not use Julia?}




% \end{frame}
% % ##################





% % ##################
% \begin{frame}
%   \frametitle{Why not use Julia?}




% \end{frame}
% % ##################





% % ##################
% \begin{frame}
%   \frametitle{Why use Julia?}




% \end{frame}
% % ##################





% % ##################
% \begin{frame}
%   \frametitle{Why use Julia?}



% \end{frame}
% % ##################





% % ##################
% \begin{frame}
%   \frametitle{There are lies, big lies and benchmarks}



% \end{frame}
% % ##################









% % ##################
% \begin{frame}
%   \frametitle{Why use Julia?}



% \end{frame}
% % ##################





% % ##################
% \begin{frame}
%   \frametitle{Bad code is~\alert{slow}, good code is~\alert{mostly}
%     fast}




% \end{frame}
% % ##################





% % ##################
% \begin{frame}
%   \frametitle{What make Julia code good?}




% \end{frame}
% % ##################





% % ##################
% \begin{frame}
%   \frametitle{Who use Julia?}


%   \begin{figure}

%     \centering


%     \includegraphics[scale=0.27]
%     {./PresentationPictures/Big_players_using_Julia.png}


%     \caption{From \colorhref{https://juliacomputing.com/}{Julia
%         Computing} page}

%   \end{figure}




% \end{frame}
% % ##################





% % ##################
% \begin{frame}
%   \frametitle{Celeste.jl}




% \end{frame}
% % ##################





% % ##################
% \begin{frame}
%   \frametitle{Celeste.jl}




% \end{frame}
% % ##################





% % ##################
% \begin{frame}
%   \frametitle{Gamedev in Julia is almost not existing}


%   \begin{figure}

%     \centering

%     \includegraphics[scale=0.17]{./PresentationPictures/Paddle_Battle.png}

%   \end{figure}

% \end{frame}
% % ##################










% % ######################################
% \section{Live coding in Julia}
% % ######################################










% % ######################################
% \section{Lies, big lies and~PDE benchmarks}
% % ######################################



% % ##################
% \begin{frame}
%   \frametitle{Solving Kuramoto-Sivashinsky PDE
%     in~$\MathTextFrametitleFGColor{1 + 1}$ dimensions}




% \end{frame}
% % ##################





% % ##################
% \begin{frame}
%   \frametitle{Solving Kuramoto-Sivashinsky PDE
%     in~$\MathTextFrametitleFGColor{1 + 1}$ dimensions}



% \end{frame}
% % ##################





% % ##################
% \begin{frame}
%   \frametitle{Solving Kuramoto-Sivashinsky PDE
%     in~$\MathTextFrametitleFGColor{1 + 1}$ dimensions}



% \end{frame}
% % ##################





% % ##################
% \begin{frame}
%   \frametitle{Solving Kuramoto-Sivashinsky PDE
%     in~$\MathTextFrametitleFGColor{1 + 1}$ dimensions}


%   \begin{figure}

%     \centering


%     \includegraphics[scale=0.22]{./PresentationPictures/KS_result.png}


%     \caption{Kuramoto-Sivashinsky heat evolution in~1~dimension}

%   \end{figure}

% \end{frame}
% % ##################





% % ##################
% \begin{frame}
%   \frametitle{Chart with linear scale}


%   \begin{figure}

%     \centering

%     \includegraphics[scale=0.22]
%     {./PresentationPictures/JFG_benchmarks_linear_scale.png}


%     \caption{Results for $N_{ x }$ points on $x$ axis, CPU time in
%       seconds}

%   \end{figure}


% \end{frame}
% % ##################





% % ##################
% \begin{frame}
%   \frametitle{Chart with logarithmic scale}


%   \begin{figure}

%     \centering

%     \includegraphics[scale=0.22]
%     {./PresentationPictures/JFG_benchmarks_logarithm_scale_01.png}


%     \caption{Results for $N_{ x }$ points on $x$ axis, CPU time in
%       seconds}

%   \end{figure}

% \end{frame}
% % ##################





% % ##################
% \begin{frame}
%   \frametitle{Time for maximal
%     $\MathTextFrametitleFGColor{N_{ x } = 2^{ 17 } = 131072}$ used
%     in~computation}



% \end{frame}
% % ##################





% % ##################
% \begin{frame}
%   \frametitle{Time over~line~of code}


%   \begin{figure}

%     \centering

%     \includegraphics[scale=0.22]
%     {./PresentationPictures/JFG_time_over_code.png}


%     \caption{Compression~of time and~lines~of code}

%   \end{figure}

% \end{frame}
% % ##################





% % ##################
% \begin{frame}
%   \frametitle{Speed/lines~of code}


%   \begin{figure}

%     \centering

%     \includegraphics[scale=0.22]
%     {./PresentationPictures/JFG_speed_over_code_01.png}


%     \caption{Compression~of ratio~of speed to the lines~of code}

%   \end{figure}

% \end{frame}
% % ##################










% % ######################################
% \section{Live coding with JuliaDiffEq}
% % ######################################










% % ######################################
% \section{Closing remarks}
% % ######################################



% % ##################
% \begin{frame}
%   \frametitle{Is someone doing\ldots}




% \end{frame}
% % ##################





% % ##################
% \begin{frame}
%   \frametitle{Reflections}



% \end{frame}
% % ##################





% % ##################
% \begin{frame}
%   \frametitle{Remember}




% \end{frame}
% % ##################










% % ######################################
% \appendix
% % ######################################





% % ##################
% \EndingSlide{Thank you. Questions?}
% % ##################











% % ######################################
% \section{Bibliography and~resources}
% % ######################################



% % ##################
% \begin{frame}
%   \frametitle{Relevant articles}




% \end{frame}
% % ##################





% % ##################
% \begin{frame}
%   \frametitle{Netography}



% \end{frame}
% % ##################





% % ##################
% \begin{frame}
%   \frametitle{Netography}




% \end{frame}
% % ##################





% % ##################
% \begin{frame}
%   \frametitle{Mentioned projects and articles}




% \end{frame}
% % ##################










% % ######################################
% \section{Additional information}
% % ######################################



% % ##################
% \begin{frame}
%   \frametitle{Some statistics}



% \end{frame}
% % ##################





% % ##################
% \begin{frame}
%   \frametitle{Julia source code}




% \end{frame}
% % ##################





% % ##################
% \begin{frame}
%   \frametitle{Gamedev in Julia is almost not existing}


%   \begin{figure}

%     \centering

%     \includegraphics[scale=0.17]{./PresentationPictures/Paddle_Battle.png}

%   \end{figure}




% \end{frame}
% % ##################





% % ##################
% \begin{frame}
%   \frametitle{Packages and projects}




% \end{frame}
% % ##################





% % ##################
% \begin{frame}
%   \frametitle{Things that are important or~promising}



% \end{frame}
% % ##################





% % ##################
% \begin{frame}
%   \frametitle{Things that are important or~promising}


%   Monte Carlo, very immature, \\
%   \colorhref{https://www.youtube.com/watch?v=BmVd7pw6Trc}
%   {https://www.youtube.com/watch?v=BmVd7pw6Trc}.

%   \vspace{0.3em}



%   Writing fast code, \\
%   \colorhref{https://www.youtube.com/watch?v=szE4txAD8mk}
%   {https://www.youtube.com/watch?v=szE4txAD8mk}.

%   \vspace{0.3em}



%   Parallel computing, \\
%   \colorhref{https://www.youtube.com/watch?v=euZkvgx0fG8}
%   {https://www.youtube.com/watch?v=euZkvgx0fG8}.

%   \vspace{0.3em}



%   Regulars expressions, \\
%   \colorhref{https://docs.julialang.org/en/v1/manual/strings/}
%   {https://docs.julialang.org/en/v1/manual/strings/}.

%   \vspace{0.3em}



%   Tensor compilers, \\
%   \colorhref{https://www.youtube.com/watch?v=Rp7sTl9oPNI}
%   {https://www.youtube.com/watch?v=Rp7sTl9Opin}.

%   \vspace{0.3em}



%   How Julia work inside, \\
%   \colorhref{https://www.youtube.com/watch?v=7KGZ\_9D\_DbI}
%   {https://www.youtube.com/watch?v=7KGZ\_9D\_DbI}.

% \end{frame}
% % ##################





% % ##################
% \begin{frame}
%   \frametitle{Lies, big lies and benchmarks from 2017}


%   {They are outdated now, but there aren't any newer}


%   \begin{figure}

%     \centering

%     \includegraphics[scale=0.7]{./PresentationPictures/benchmarks_QO.pdf}

%     \caption{QuTiP~-- Quantum Toolbox in Python}

%   \end{figure}

% \end{frame}
% % ##################





% % ##################
% \begin{frame}
%   \frametitle{Basics learning materials}


%   Most~of them are more or~less outdated, since they were created
%   before release~of Julia 1.x.
%   \begin{itemize}
%     \RaggedRight

%   \item JuliaBoxTutorials, version 1.x, GitHub
%     \colorhref{https://github.com/JuliaComputing/JuliaBoxTutorials}
%     {JuliaComputing/JuliaBoxTutorials}. Good starting point.

%   \item Julia 1.x Documentation. Always~up to~date, really good
%     written in~comparison to~others manuals,
%     \colorhref{https://docs.julialang.org/en/v1/}
%     {https://docs.julialang.org/en/v1/}.

%   \item David P.~Sanders, \textit{Introduction to~Julia for~scientific
%       Computing (Workshop)}, 2015. Outdated, but very good
%     introduction to~language,
%     \colorhref{https://www.youtube.com/watch?v=gQ1y5NUD\_RI}
%     {https://www.youtube.com/watch?v=gQ1y5NUD\_RI}.

%     % \item The Julia Language channel
%     %   on~\colorhref{https://www.youtube.com/user/JuliaLanguage}{YouTube}.
%     %   Contains dozens videos from JuliaCons and hold regulars
%     %   \textit{Intro to~Julia}, keeping it up to date. You~can find
%     %   next \textit{Intro to~Julia} and other introductions
%     %   on~\colorhref{https://www.facebook.com/Learn-Julia-529467964069525/?eid=ARDnSanRh6e3_6NQanjfziNeMYt3-hRUqje-5xOvRidGM5bmm6ZGCZ49fy5CZ1AYX9T503OEdMUMRlOe}{Facebook}.

%     % \item Ben Lauwens, Allen Downey, \textit{Think Julia: How
%     %   to~Think Like a~Computer Scientist},
%     %   \colorhref{https://benlauwens.github.io/ThinkJulia.jl/latest/book.html}
%     %   {https://benlauwens.github.io/ThinkJulia.jl/latest/book.html}.

%   \end{itemize}

% \end{frame}
% % ##################





% % ##################
% \begin{frame}
%   \frametitle{Basics learning materials}


%   Most~of them are more or~less outdated, since they were created
%   before release~of Julia 1.x.
%   \begin{itemize}
%     \RaggedRight

%     % \item JuliaBoxTutorials, version 1.x, GitHub
%     %   \colorhref{https://github.com/JuliaComputing/JuliaBoxTutorials}
%     %   {JuliaComputing/JuliaBoxTutorials}. Good starting point.

%     % \item Julia 1.x Documentation. Always~up to~date, really good
%     %   written in~comparison to~others manuals,
%     %   \colorhref{https://docs.julialang.org/en/v1/}
%     %   {https://docs.julialang.org/en/v1/}.

%     % \item David P.~Sanders, \textit{Introduction to~Julia
%     %   for~scientific Computing (Workshop)}, 2015. Outdated, but very
%     %   good introduction to~language,
%     %   \colorhref{https://www.youtube.com/watch?v=gQ1y5NUD\_RI}
%     %   {https://www.youtube.com/watch?v=gQ1y5NUD\_RI}.

%   \item The Julia Language channel
%     on~\colorhref{https://www.youtube.com/user/JuliaLanguage}{YouTube}.
%     Contains dozens videos from JuliaCons and hold regulars
%     \textit{Intro to~Julia}, keeping it up to date. You~can find next
%     \textit{Intro to~Julia} and other introductions
%     on~\colorhref{https://www.facebook.com/Learn-Julia-529467964069525/?eid=ARDnSanRh6e3_6NQanjfziNeMYt3-hRUqje-5xOvRidGM5bmm6ZGCZ49fy5CZ1AYX9T503OEdMUMRlOe}{Facebook}.

%   \item Ben Lauwens, Allen Downey, \textit{Think Julia: How to~Think
%       Like a~Computer Scientist},
%     \colorhref{https://benlauwens.github.io/ThinkJulia.jl/latest/book.html}
%     {https://benlauwens.github.io/ThinkJulia.jl/latest/book.html}.

%   \end{itemize}

% \end{frame}
% % ##################





% % ##################
% \begin{frame}
%   \frametitle{Practical introductions to many different topics}


%   Stefan Karpinski and~Kristoffer Carlsson,
%   \textit{Pkg3:~The~new Julia package manager}, JuliaCon 2018, \\
%   \colorhref{https://www.youtube.com/watch?v=HgFmiT5p0zU}
%   {https://www.youtube.com/watch?v=HgFmiT5p0zU}.

%   \vspace{0.3em}



%   Arch D. Robison, \textit{Introduction to~Writing High Performance
%     Julia (Workshop)}, JuliaCon 2016,
%   \colorhref{https://www.youtube.com/watch?v=szE4txAD8mk}
%   {https://www.youtube.com/watch?v=szE4txAD8mk}.

%   \vspace{0.3em}



%   Andy Ferris, \textit{A~practical introduction to~metaprogramming
%     in~Julia}, JuliaCon 2018,
%   \colorhref{https://www.youtube.com/watch?v=SeqAQHKLNj4}
%   {https://www.youtube.com/watch?v=SeqAQHKLNj4}.

%   \vspace{0.3em}



%   Chris Rackauckas, \textit{Intro to solving differential
%     equations in Julia}, \\
%   \colorhref{https://www.youtube.com/watch?v=KPEqYtEd-zY}
%   {https://www.youtube.com/watch?v=KPEqYtEd-zY}.

%   \vspace{0.3em}



%   DiffEqTutorials.jl. Tutorials~of JuliaDiffEq project, GitHub
%   \colorhref{https://github.com/JuliaDiffEq/DiffEqTutorials.jl}{JuliaDiffEq/DiffEqTutorials.jl}.

% \end{frame}
% % ##################





% % ##################
% \begin{frame}
%   \frametitle{More theoretical, less practical materials}



%   Jeff Bezanson, \textit{Why is~Julia fast?}, 2015, \\
%   \colorhref{https://www.youtube.com/watch?v=cjzcYM9YhwA}
%   {https://www.youtube.com/watch?v=cjzcYM9YhwA}.

%   \vspace{0.3em}



%   Jeff Bezanson, \textit{The~State~of the~Type System},
%   JuliaCon 2017, \\
%   \colorhref{https://www.youtube.com/watch?v=Z2LtJUe1q8c}
%   {https://www.youtube.com/watch?v=Z2LtJUe1q8c}.

%   \vspace{0.3em}



%   Jiahao Chen, \textit{Why language matters: Julia
%     and~multiple dispatch}, 2016, \\
%   \colorhref{https://www.youtube.com/watch?v=gZJFHrYopxw}
%   {https://www.youtube.com/watch?v=gZJFHrYopxw}.

%   \vspace{0.3em}



%   Jameson Nash, \textit{AoT or~JIT: How Does Julia Work?}, (AoT
%   --~Ahead~of Time compilation, JIT --~Just In~Time
%   compilation), JuliaCon 2017, \\
%   \colorhref{https://www.youtube.com/watch?v=7KGZ\_9D\_DbI}
%   {https://www.youtube.com/watch?v=7KGZ\_9D\_DbI}.

%   \vspace{0.3em}



%   John Lapyre, \textit{Symbolic Mathematics in Julia}, JuliaCon 2018, \\
%   \colorhref{https://www.youtube.com/watch?v=M742\_73edLA}
%   {https://www.youtube.com/watch?v=M742\_73edLA}.

%   \vspace{0.3em}



%   Julia Lab at~MIT, \textit{Parallel Computing (Workshop)},
%   JuliaCon 2016, \\
%   \colorhref{https://www.youtube.com/watch?v=euZkvgx0fG8}
%   {https://www.youtube.com/watch?v=euZkvgx0fG8}.

% \end{frame}
% % ##################





% % ##################
% \begin{frame}
%   \frametitle{More theoretical, less practical materials}


%   Tim Besard, Valentin Churavy and Simon Danisch,
%   \textit{GPU~Programming with~Julia}, JuliaCon 2017, \\
%   \colorhref{https://www.youtube.com/watch?v=6ntJ\_al4oXA}
%   {https://www.youtube.com/watch?v=6ntJ\_al4oXA}.

%   \vspace{0.3em}



%   Peter Ahrens, \emph{For~Loops~2.0: Index Notation
%     And~The~Future~Of Tensor Compilers}, JuliaCon 2018, \\
%   \colorhref{https://www.youtube.com/watch?v=Rp7sTl9oPNI}
%   {https://www.youtube.com/watch?v=Rp7sTl9oPNI}.

%   \vspace{0.3em}



%   Carsten Bauer, \emph{Julia for Physics: Quantum Monte Carlo},
%   JuliaCon 2018,
%   \colorhref{https://www.youtube.com/watch?v=BmVd7pw6Trc}
%   {https://www.youtube.com/watch?v=BmVd7pw6Trc}.

%   \vspace{0.3em}



%   George Datseris, \emph{Why Julia is the most suitable
%     language for science}, case study~of project JuliaDynamics, \\
%   \colorhref{https://www.youtube.com/watch?v=7y-ahkUsIrY}
%   {https://www.youtube.com/watch?v=7y-ahkUsIrY}.

%   \vspace{0.3em}



%   Nick Higham, \emph{Tricks and~Tips in~Numerical Computing}, JuliaCon
%   2018, \colorhref{https://www.youtube.com/watch?v=Q9OLOqEhc64}
%   {https://www.youtube.com/watch?v=Q9OLOqEhc64}.

% \end{frame}
% % ##################










% % ######################################
% \section{Additional information and topics}
% % ######################################



% % ##################
% \begin{frame}
%   \frametitle{Cxx.jl (probably still broken in~1.x)}


%   \begin{figure}

%     \centering

%     \includegraphics[scale=0.29]{./PresentationPictures/Cxx_jl.png}


%     \caption{RELP for C++}

%   \end{figure}

% \end{frame}
% % ##################





% % ##################
% \begin{frame}
%   \frametitle{Mentioned projects and articles}


%   {Written Julia if not mentioned otherwise}
%   Still many~of them don't work in Julia 1.x.
%   \begin{itemize}
%     \RaggedRight

%   \item \textit{Channelflow} (written in~C++),
%     \colorhref{http://channelflow.org/}{http://channelflow.org/}.

%   \item DynamicalSystems.jl, GitHub
%     \colorhref{https://github.com/JuliaDynamics/DynamicalSystems.jl}
%     {JuliaDynamics/DynamicalSystems.jl}.

%   \item QuantumOptics.jl. Numerical framework for numerical
%     solving open quantum systems, more on this page
%     \colorhref{https://qojulia.org/}{https://qojulia.org/}.

%   \item REPL (shell) for C++ (broken in Julia 1.x), GitHub
%     \colorhref{https://github.com/NHDaly/PaddleBattleJL}{Keno/Cxx.jl}.

%   \item Game and educational tool, \textit{Paddle Battle}, GitHub
%     \colorhref{https://github.com/NHDaly/PaddleBattleJL}
%     {NHDaly/PaddleBattleJL}.

%   \end{itemize}

% \end{frame}
% % ##################










% ######################################
\appendix
% ######################################





% ######################################
\EndingSlide{Thank you! Questions?}
% ######################################










% ####################################################################
% ####################################################################
% Bibliography

% \bibliographystyle{plalpha}

% \bibliography{}{}





% ############################

% Koniec dokumentu
\end{document}
