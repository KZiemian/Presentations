% ------------------------------------------------------------------------------------------------------------------
% Basic configuration of Beamera and Jagiellonian
% ------------------------------------------------------------------------------------------------------------------
\RequirePackage[l2tabu, orthodox]{nag}



\ifx\PresentationStyle\notset
  \def\PresentationStyle{dark}
\fi



% Options: t -- align text to the top of the frame
\documentclass[10pt,t]{beamer}
\mode<presentation>
\usetheme[style=\PresentationStyle,JUlogotitle=no]{jagiellonian}



% ------------------------------------------------------------------------------------
% Procesing configuration files of Jagiellonian theme located in directory
% "preambule"
% ------------------------------------------------------------------------------------
% Configuration for polish language
% Need description
\usepackage[polish]{babel}
% Need description
\usepackage[MeX]{polski}



% ------------------------------
% Better support of polish chars in technical parts of PDF
% ------------------------------
\hypersetup{pdfencoding=auto,psdextra}

% Package "textpos" give as enviroment "textblock" which is very usefull in
% arranging text on slides.

% This is standard configuration of "textpos"
\usepackage[overlay,absolute]{textpos}

% If you need to see bounds of "textblock's" comment line above and uncomment
% one below.

% Caution! When showboxes option is on significant ammunt of space is add
% to the top of textblock and as such, everyting put in them gone down.
% We need to check how to remove this bug.

% \usepackage[showboxes,overlay,absolute]{textpos}



% Setting scale length for package "textpos"
\setlength{\TPHorizModule}{10mm}
\setlength{\TPVertModule}{\TPHorizModule}


% ---------------------------------------
% TikZ
% ---------------------------------------
% Importing TikZ libraries
\usetikzlibrary{arrows.meta}
\usetikzlibrary{positioning}





% % Configuration package "bm" that need for making bold symbols
% \newcommand{\bmmax}{0}
% \newcommand{\hmmax}{0}
% \usepackage{bm}




% ---------------------------------------
% Packages for scientific texts
% ---------------------------------------
% \let\lll\undefined  % Sometimes you must use this line to allow
% "amsmath" package to works with packages with packages for polish
% languge imported
% /preambul/LanguageSettings/JagiellonianPolishLanguageSettings.tex.
% This comments (probably) removes polish letter Ł.
\usepackage{amsmath}  % Packages from American Mathematical Society (AMS)
\usepackage{amssymb}
\usepackage{amscd}
\usepackage{amsthm}
\usepackage{siunitx}  % Package for typsetting SI units.
\usepackage{upgreek}  % Better looking greek letters.
% Example of using upgreek: pi = \uppi


\usepackage{calrsfs}  % Zmienia czcionkę kaligraficzną w \mathcal
% na ładniejszą. Może w innych miejscach robi to samo, ale o tym nic
% nie wiem.










% ---------------------------------------
% Packages written for lectures "Geometria 3D dla twórców gier wideo"
% ---------------------------------------
% \usepackage{./ProgramowanieSymulacjiFizykiPaczki/ProgramowanieSymulacjiFizyki}
% \usepackage{./ProgramowanieSymulacjiFizykiPaczki/ProgramowanieSymulacjiFizykiIndeksy}
% \usepackage{./ProgramowanieSymulacjiFizykiPaczki/ProgramowanieSymulacjiFizykiTikZStyle}





% !!!!!!!!!!!!!!!!!!!!!!!!!!!!!!
% !!!!!!!!!!!!!!!!!!!!!!!!!!!!!!
% EVIL STUFF
\if\JUlogotitle1
\edef\LogoJUPath{LogoJU_\JUlogoLang/LogoJU_\JUlogoShape_\JUlogoColor.pdf}
\titlegraphic{\hfill\includegraphics[scale=0.22]
{./JagiellonianPictures/\LogoJUPath}}
\fi
% ---------------------------------------
% Commands for handling colors
% ---------------------------------------


% Command for setting normal text color for some text in math modestyle
% Text color depend on used style of Jagiellonian

% Beamer version of command
\newcommand{\TextWithNormalTextColor}[1]{%
  {\color{jNormalTextFGColor}
    \setbeamercolor{math text}{fg=jNormalTextFGColor} {#1}}
}

% Article and similar classes version of command
% \newcommand{\TextWithNormalTextColor}[1]{%
%   {\color{jNormalTextsFGColor} {#1}}
% }



% Beamer version of command
\newcommand{\NormalTextInMathMode}[1]{%
  {\color{jNormalTextFGColor}
    \setbeamercolor{math text}{fg=jNormalTextFGColor} \text{#1}}
}


% Article and similar classes version of command
% \newcommand{\NormalTextInMathMode}[1]{%
%   {\color{jNormalTextsFGColor} \text{#1}}
% }




% Command that sets color of some mathematical text to the same color
% that has normal text in header (?)

% Beamer version of the command
\newcommand{\MathTextFrametitleFGColor}[1]{%
  {\color{jFrametitleFGColor}
    \setbeamercolor{math text}{fg=jFrametitleFGColor} #1}
}

% Article and similar classes version of the command
% \newcommand{\MathTextWhiteColor}[1]{{\color{jFrametitleFGColor} #1}}





% Command for setting color of alert text for some text in math modestyle

% Beamer version of the command
\newcommand{\MathTextAlertColor}[1]{%
  {\color{jOrange} \setbeamercolor{math text}{fg=jOrange} #1}
}

% Article and similar classes version of the command
% \newcommand{\MathTextAlertColor}[1]{{\color{jOrange} #1}}





% Command that allow you to sets chosen color as the color of some text into
% math mode. Due to some nuances in the way that Beamer handle colors
% it not work in all cases. We hope that in the future we will improve it.

% Beamer version of the command
\newcommand{\SetMathTextsColor}[2]{%
  {\color{#1} \setbeamercolor{math text}{fg=#1} #2}
}


% Article and similar classes version of the command
% \newcommand{\SetMathTextColor}[2]{{\color{#1} #2}}










% ---------------------------------------
% Commands for setting background pictures for some slides
% ---------------------------------------
\newcommand{\TitleBackgroundPicture}
{./PresentationPictures/CommonPictures/Cute_dragon_BG_dark.png}
\newcommand{\SectionBackgroundPicture}
{./PresentationPictures/CommonPictures/Cute_dragon_small_BG_light.png}



\newcommand{\TitleSlideWithPicture}{
  \begingroup

  \usebackgroundtemplate{ % \hspace*{-11.5em}
    \includegraphics[height=\paperheight]{\TitleBackgroundPicture}}

  \maketitle

  \endgroup
}





\newcommand{\SectionSlideWithPicture}[1]{%
  \begingroup

  \usebackgroundtemplate{ % \hspace*{-11.5em}
    \includegraphics[height=\paperheight]{\SectionBackgroundPicture}}

  \setbeamercolor{titlelike}{fg=normal text.fg}

  \section{#1}

  \endgroup
}





\newcommand{\EndingSlide}[1]{%
  \begin{frame}[standout]

    \begingroup

    \color{jFrametitleFGColor}

    #1

    \endgroup

  \end{frame}
}










% ------------------------------------------------------
% Packages, libraries and their configuration
% ------------------------------------------------------





% ------------------------------------------------------
% Local packages
% ------------------------------------------------------





% ------------------------------------------------------
% Configuration for this particular presentation
% ------------------------------------------------------










% ------------------------------------------------------------------------------------------------------------------
\title{Matematyka z~przymrużeniem oka}

\author{Autorzy tych żartów}


% \institute{Uniwersytet Jagielloński w~Krakowie}

% \date[11 December 2018]{Seminarium astrofizyczne PAU \\
%   11 grudnia 2019}
% ------------------------------------------------------------------------------------------------------------------










% ####################################################################
% Beginning of the document
\begin{document}
% ####################################################################





% ######################################
% Text is adjusted to the left and words are broken at the end of the line.
\RaggedRight
% Number of chars: 41k+,
% ######################################





% ######################################
\maketitle
% ######################################





% % ######################################
% \begin{frame}
%   \frametitle{Table of contents}


%   \tableofcontents % Spis treści

% \end{frame}
% % ######################################










% ######################################
% \section{Różne żarty}
% ######################################



% ##################
\begin{frame}
  \frametitle{Musimy o~tym pamiętać}


  \begin{figure}

    \label{fig:aaa}

    \centering


    \includegraphics[scale=0.15]
    {./Presentations-pictures/Reading-mathematicals-mems-ETC.jpg}

  \end{figure}

\end{frame}
% ##################





% ##################
\begin{frame}
  \frametitle{LLVM and all of that}


  \begin{figure}

    \label{fig:aaa}

    \centering


    \includegraphics[scale=0.18]
    {./Presentations-pictures/It-should-be-like-that.jpg}

  \end{figure}

\end{frame}
% ##################





% ##################
\begin{frame}
  \frametitle{Problemy z~arytmetyką}


  \begin{figure}

    \label{fig:aaa}

    \centering


    \includegraphics[scale=0.15]
    {./Presentations-pictures/Problems-with-simple-arithmetics-01.jpg}

  \end{figure}

\end{frame}
% ##################











% % ##################
% \begin{frame}
%   \frametitle{Aims of Julia (partialy by Alan Edelman)}



% \end{frame}
% % ##################





% % ##################
% \begin{frame}
%   \frametitle{Why not use Julia?}




% \end{frame}
% % ##################





% % ##################
% \begin{frame}
%   \frametitle{Why not use Julia?}




% \end{frame}
% % ##################





% % ##################
% \begin{frame}
%   \frametitle{Why use Julia?}




% \end{frame}
% % ##################





% % ##################
% \begin{frame}
%   \frametitle{Why use Julia?}



% \end{frame}
% % ##################





% % ##################
% \begin{frame}
%   \frametitle{There are lies, big lies and benchmarks}

%   \begin{figure}

%     \centering

%     \includegraphics[scale=0.29]
%     {./PresentationPictures/Julia_micro_benchmarks.png}


%     \caption{\textbf{Warning!} Python code use numpy libraries that
%       is~written 52.8\% in C (state of GitHub repository at~4
%       January~2019)}

%   \end{figure}

% \end{frame}
% % ##################









% % ##################
% \begin{frame}
%   \frametitle{Why use Julia?}




% \end{frame}
% % ##################





% % ##################
% \begin{frame}
%   \frametitle{Bad code is~\alert{slow}, good code is~\alert{mostly}
%     fast}




% \end{frame}
% % ##################





% % ##################
% \begin{frame}
%   \frametitle{What make Julia code good?}




% \end{frame}
% % ##################





% % ##################
% \begin{frame}
%   \frametitle{Who use Julia?}


%   \begin{figure}

%     \centering

%     \includegraphics[scale=0.27]
%     {./PresentationPictures/Big_players_using_Julia.png}


%     \caption{From \colorhref{https://juliacomputing.com/}{Julia
%         Computing} page}

%   \end{figure}




% \end{frame}
% % ##################





% % ##################
% \begin{frame}
%   \frametitle{Celeste.jl}




% \end{frame}
% % ##################





% % ##################
% \begin{frame}
%   \frametitle{Celeste.jl}



% \end{frame}
% % ##################





% % ##################
% \begin{frame}
%   \frametitle{Gamedev in Julia is almost not existing}


%   \begin{figure}

%     \centering

%     \includegraphics[scale=0.17]{./PresentationPictures/Paddle_Battle.png}

%   \end{figure}




% \end{frame}
% % ##################










% % ######################################
% \section{Live coding in Julia}
% % ######################################










% % ######################################
% \section{Lies, big lies and~PDE benchmarks}
% % ######################################



% % ##################
% \begin{frame}
%   \frametitle{Solving Kuramoto-Sivashinsky PDE
%     in~$\MathTextFrametitleFGColor{1 + 1}$ dimensions}




% \end{frame}
% % ##################





% % ##################
% \begin{frame}
%   \frametitle{Solving Kuramoto-Sivashinsky PDE
%     in~$\MathTextFrametitleFGColor{1 + 1}$ dimensions}




% \end{frame}
% % ##################





% % ##################
% \begin{frame}
%   \frametitle{Solving Kuramoto-Sivashinsky PDE
%     in~$\MathTextFrametitleFGColor{1 + 1}$ dimensions}




% \end{frame}
% % ##################





% % ##################
% \begin{frame}
%   \frametitle{Solving Kuramoto-Sivashinsky PDE
%     in~$\MathTextFrametitleFGColor{1 + 1}$ dimensions}


%   \begin{figure}

%     \centering


%     \includegraphics[scale=0.22]{./PresentationPictures/KS_result.png}


%     \caption{Kuramoto-Sivashinsky heat evolution in~1~dimension}

%   \end{figure}

% \end{frame}
% % ##################





% % ##################
% \begin{frame}
%   \frametitle{Chart with linear scale}


%   \begin{figure}

%     \centering


%     \includegraphics[scale=0.22]
%     {./PresentationPictures/JFG_benchmarks_linear_scale.png}


%     \caption{Results for $N_{ x }$ points on $x$ axis, CPU time in
%       seconds}

%   \end{figure}


% \end{frame}
% % ##################





% % ##################
% \begin{frame}
%   \frametitle{Chart with logarithmic scale}


%   \begin{figure}

%     \centering


%     \includegraphics[scale=0.22]
%     {./PresentationPictures/JFG_benchmarks_logarithm_scale_01.png}


%     \caption{Results for $N_{ x }$ points on $x$ axis, CPU time in
%       seconds}

%   \end{figure}

% \end{frame}
% % ##################





% % ##################
% \begin{frame}
%   \frametitle{Time for maximal
%     $\MathTextFrametitleFGColor{N_{ x } = 2^{ 17 } = 131072}$ used
%     in~computation}



% \end{frame}
% % ##################





% % ##################
% \begin{frame}
%   \frametitle{Time over~line~of code}


%   \begin{figure}

%     \centering

%     \includegraphics[scale=0.22]
%     {./PresentationPictures/JFG_time_over_code.png}


%     \caption{Compression~of time and~lines~of code}

%   \end{figure}

% \end{frame}
% % ##################





% % ##################
% \begin{frame}
%   \frametitle{Speed/lines~of code}


%   \begin{figure}

%     \centering

%     \includegraphics[scale=0.22]
%     {./PresentationPictures/JFG_speed_over_code_01.png}


%     \caption{Compression~of ratio~of speed to the lines~of code}

%   \end{figure}

% \end{frame}
% % ##################










% % ######################################
% \section{Live coding with JuliaDiffEq}
% % ######################################










% % ######################################
% \section{Closing remarks}
% % ######################################



% % ##################
% \begin{frame}
%   \frametitle{Is someone doing\ldots}



% \end{frame}
% % ##################





% % ##################
% \begin{frame}
%   \frametitle{Reflections}




% \end{frame}
% % ##################





% % ##################
% \begin{frame}
%   \frametitle{Remember}




% \end{frame}
% % ##################










% % ######################################
% \appendix
% % ######################################





% % ##################
% \EndingSlide{Thank you. Questions?}
% % ##################











% % ######################################
% \section{Bibliography and~resources}
% % ######################################



% % ##################
% \begin{frame}
%   \frametitle{Relevant articles}




% \end{frame}
% % ##################





% % ##################
% \begin{frame}
%   \frametitle{Netography}




% \end{frame}
% % ##################





% % ##################
% \begin{frame}
%   \frametitle{Netography}




% \end{frame}
% % ##################





% % ##################
% \begin{frame}
%   \frametitle{Mentioned projects and articles}




% \end{frame}
% % ##################










% % ######################################
% \section{Additional information}
% % ######################################



% % ##################
% \begin{frame}
%   \frametitle{Some statistics}



% \end{frame}
% % ##################





% % ##################
% \begin{frame}
%   \frametitle{Julia source code}




% \end{frame}
% % ##################





% % ##################
% \begin{frame}
%   \frametitle{Gamedev in Julia is almost not existing}


%   \begin{figure}

%     \centering

%     \includegraphics[scale=0.17]{./PresentationPictures/Paddle_Battle.png}

%   \end{figure}




% \end{frame}
% % ##################





% % ##################
% \begin{frame}
%   \frametitle{Packages and projects}



% \end{frame}
% % ##################





% % ##################
% \begin{frame}
%   \frametitle{Things that are important or~promising}



% \end{frame}
% % ##################





% % ##################
% \begin{frame}
%   \frametitle{Things that are important or~promising}




% \end{frame}
% % ##################





% % ##################
% \begin{frame}
%   \frametitle{Lies, big lies and benchmarks from 2017}


%   \begin{figure}

%     \centering

%     \includegraphics[scale=0.7]{./PresentationPictures/benchmarks_QO.pdf}

%     \caption{QuTiP~-- Quantum Toolbox in Python}

%   \end{figure}

% \end{frame}
% % ##################





% % ##################
% \begin{frame}
%   \frametitle{Basics learning materials}




% \end{frame}
% % ##################





% % ##################
% \begin{frame}
%   \frametitle{Basics learning materials}




% \end{frame}
% % ##################





% % ##################
% \begin{frame}
%   \frametitle{Practical introductions to many different topics}




% \end{frame}
% % ##################





% % ##################
% \begin{frame}
%   \frametitle{More theoretical, less practical materials}




% \end{frame}
% % ##################





% % ##################
% \begin{frame}
%   \frametitle{More theoretical, less practical materials}



% \end{frame}
% % ##################










% % ######################################
% \section{Additional information and topics}
% % ######################################



% % ##################
% \begin{frame}
%   \frametitle{Cxx.jl (probably still broken in~1.x)}




% \end{frame}
% % ##################





% % ##################
% \begin{frame}
%   \frametitle{Mentioned projects and articles}




% \end{frame}
% % ##################










% ######################################
\appendix
% ######################################





% ######################################
\EndingSlide{Thank you! Questions?}
% ######################################










% ####################################################################
% ####################################################################
% Bibliography

% \bibliographystyle{plalpha}

% \bibliography{}{}





% ############################

% Koniec dokumentu
\end{document}
