% ---------------------------------------------------------------------
% Basic configuration of Beamera and Jagiellonian
% ---------------------------------------------------------------------
\RequirePackage[l2tabu, orthodox]{nag}



\ifx\PresentationStyle\notset
\def\PresentationStyle{dark}
\fi



\documentclass[10pt,t]{beamer}
\mode<presentation>
\usetheme[style=\PresentationStyle,logoColor=monochromaticJUwhite,JUlogotitle=yes]{jagiellonian}



% ---------------------------------------
% Configuration files of Jagiellonian loceted in catalog preambule
% ---------------------------------------
% Configuration for polish language
% Need description
\usepackage[polish]{babel}
% Need description
\usepackage[MeX]{polski}



% ------------------------------
% Better support of polish chars in technical parts of PDF
% ------------------------------
\hypersetup{pdfencoding=auto,psdextra}

% Package "textpos" give as enviroment "textblock" which is very usefull in
% arranging text on slides.

% This is standard configuration of "textpos"
\usepackage[overlay,absolute]{textpos}

% If you need to see bounds of "textblock's" comment line above and uncomment
% one below.

% Caution! When showboxes option is on significant ammunt of space is add
% to the top of textblock and as such, everyting put in them gone down.
% We need to check how to remove this bug.

% \usepackage[showboxes,overlay,absolute]{textpos}



% Setting scale length for package "textpos"
\setlength{\TPHorizModule}{10mm}
\setlength{\TPVertModule}{\TPHorizModule}


% ---------------------------------------
% TikZ
% ---------------------------------------
% Importing TikZ libraries
\usetikzlibrary{arrows.meta}
\usetikzlibrary{positioning}





% % Configuration package "bm" that need for making bold symbols
% \newcommand{\bmmax}{0}
% \newcommand{\hmmax}{0}
% \usepackage{bm}




% ---------------------------------------
% Packages for scientific texts
% ---------------------------------------
% \let\lll\undefined  % Sometimes you must use this line to allow
% "amsmath" package to works with packages with packages for polish
% languge imported
% /preambul/LanguageSettings/JagiellonianPolishLanguageSettings.tex.
% This comments (probably) removes polish letter Ł.
\usepackage{amsmath}  % Packages from American Mathematical Society (AMS)
\usepackage{amssymb}
\usepackage{amscd}
\usepackage{amsthm}
\usepackage{siunitx}  % Package for typsetting SI units.
\usepackage{upgreek}  % Better looking greek letters.
% Example of using upgreek: pi = \uppi


\usepackage{calrsfs}  % Zmienia czcionkę kaligraficzną w \mathcal
% na ładniejszą. Może w innych miejscach robi to samo, ale o tym nic
% nie wiem.










% ---------------------------------------
% Packages written for lectures "Geometria 3D dla twórców gier wideo"
% ---------------------------------------
% \usepackage{./ProgramowanieSymulacjiFizykiPaczki/ProgramowanieSymulacjiFizyki}
% \usepackage{./ProgramowanieSymulacjiFizykiPaczki/ProgramowanieSymulacjiFizykiIndeksy}
% \usepackage{./ProgramowanieSymulacjiFizykiPaczki/ProgramowanieSymulacjiFizykiTikZStyle}





% !!!!!!!!!!!!!!!!!!!!!!!!!!!!!!
% !!!!!!!!!!!!!!!!!!!!!!!!!!!!!!
% EVIL STUFF
\if\JUlogotitle1
\edef\LogoJUPath{LogoJU_\JUlogoLang/LogoJU_\JUlogoShape_\JUlogoColor.pdf}
\titlegraphic{\hfill\includegraphics[scale=0.22]
{./JagiellonianPictures/\LogoJUPath}}
\fi
% ---------------------------------------
% Commands for handling colors
% ---------------------------------------


% Command for setting normal text color for some text in math modestyle
% Text color depend on used style of Jagiellonian

% Beamer version of command
\newcommand{\TextWithNormalTextColor}[1]{%
  {\color{jNormalTextFGColor}
    \setbeamercolor{math text}{fg=jNormalTextFGColor} {#1}}
}

% Article and similar classes version of command
% \newcommand{\TextWithNormalTextColor}[1]{%
%   {\color{jNormalTextsFGColor} {#1}}
% }



% Beamer version of command
\newcommand{\NormalTextInMathMode}[1]{%
  {\color{jNormalTextFGColor}
    \setbeamercolor{math text}{fg=jNormalTextFGColor} \text{#1}}
}


% Article and similar classes version of command
% \newcommand{\NormalTextInMathMode}[1]{%
%   {\color{jNormalTextsFGColor} \text{#1}}
% }




% Command that sets color of some mathematical text to the same color
% that has normal text in header (?)

% Beamer version of the command
\newcommand{\MathTextFrametitleFGColor}[1]{%
  {\color{jFrametitleFGColor}
    \setbeamercolor{math text}{fg=jFrametitleFGColor} #1}
}

% Article and similar classes version of the command
% \newcommand{\MathTextWhiteColor}[1]{{\color{jFrametitleFGColor} #1}}





% Command for setting color of alert text for some text in math modestyle

% Beamer version of the command
\newcommand{\MathTextAlertColor}[1]{%
  {\color{jOrange} \setbeamercolor{math text}{fg=jOrange} #1}
}

% Article and similar classes version of the command
% \newcommand{\MathTextAlertColor}[1]{{\color{jOrange} #1}}





% Command that allow you to sets chosen color as the color of some text into
% math mode. Due to some nuances in the way that Beamer handle colors
% it not work in all cases. We hope that in the future we will improve it.

% Beamer version of the command
\newcommand{\SetMathTextColor}[2]{%
  {\color{#1} \setbeamercolor{math text}{fg=#1} #2}
}


% Article and similar classes version of the command
% \newcommand{\SetMathTextColor}[2]{{\color{#1} #2}}










% ---------------------------------------
% Commands for few special slides
% ---------------------------------------
\newcommand{\EndingSlide}[1]{%
  \begin{frame}[standout]

    \begingroup

    \color{jFrametitleFGColor}

    #1

    \endgroup

  \end{frame}
}










% ---------------------------------------
% Commands for setting background pictures for some slides
% ---------------------------------------
\newcommand{\TitleBackgroundPicture}
{./JagiellonianPictures/Backgrounds/LajkonikDark.png}
\newcommand{\SectionBackgroundPicture}
{./JagiellonianPictures/Backgrounds/LajkonikLight.png}



\newcommand{\TitleSlideWithPicture}{%
  \begingroup

  \usebackgroundtemplate{%
    \includegraphics[height=\paperheight]{\TitleBackgroundPicture}}

  \maketitle

  \endgroup
}





\newcommand{\SectionSlideWithPicture}[1]{%
  \begingroup

  \usebackgroundtemplate{%
    \includegraphics[height=\paperheight]{\SectionBackgroundPicture}}

  \setbeamercolor{titlelike}{fg=normal text.fg}

  \section{#1}

  \endgroup
}










% ---------------------------------------
% Commands for lectures "Geometria 3D dla twórców gier wideo"
% Polish version
% ---------------------------------------
% Komendy teraz wykomentowane były potrzebne, gdy loga były na niebieskim
% tle, nie na białym. A są na białym bo tego chcieli w biurze projektu.
% \newcommand{\FundingLogoWhitePicturePL}
% {./PresentationPictures/CommonPictures/logotypFundusze_biale_bez_tla2.pdf}
\newcommand{\FundingLogoColorPicturePL}
{./PresentationPictures/CommonPictures/European_Funds_color_PL.pdf}
% \newcommand{\EULogoWhitePicturePL}
% {./PresentationPictures/CommonPictures/logotypUE_biale_bez_tla2.pdf}
\newcommand{\EUSocialFundLogoColorPicturePL}
{./PresentationPictures/CommonPictures/EU_Social_Fund_color_PL.pdf}
% \newcommand{\ZintegrUJLogoWhitePicturePL}
% {./PresentationPictures/CommonPictures/zintegruj-logo-white.pdf}
\newcommand{\ZintegrUJLogoColorPicturePL}
{./PresentationPictures/CommonPictures/ZintegrUJ_color.pdf}
\newcommand{\JULogoColorPicturePL}
{./JagiellonianPictures/LogoJU_PL/LogoJU_A_color.pdf}





\newcommand{\GeometryThreeDSpecialBeginningSlidePL}{%
  \begin{frame}[standout]

    \begin{textblock}{11}(1,0.7)

      \begin{flushleft}

        \mdseries

        \footnotesize

        \color{jFrametitleFGColor}

        Materiał powstał w ramach projektu współfinansowanego ze środków
        Unii Europejskiej w ramach Europejskiego Funduszu Społecznego
        POWR.03.05.00-00-Z309/17-00.

      \end{flushleft}

    \end{textblock}





    \begin{textblock}{10}(0,2.2)

      \tikz \fill[color=jBackgroundStyleLight] (0,0) rectangle (12.8,-1.5);

    \end{textblock}


    \begin{textblock}{3.2}(1,2.45)

      \includegraphics[scale=0.3]{\FundingLogoColorPicturePL}

    \end{textblock}


    \begin{textblock}{2.5}(3.7,2.5)

      \includegraphics[scale=0.2]{\JULogoColorPicturePL}

    \end{textblock}


    \begin{textblock}{2.5}(6,2.4)

      \includegraphics[scale=0.1]{\ZintegrUJLogoColorPicturePL}

    \end{textblock}


    \begin{textblock}{4.2}(8.4,2.6)

      \includegraphics[scale=0.3]{\EUSocialFundLogoColorPicturePL}

    \end{textblock}

  \end{frame}
}



\newcommand{\GeometryThreeDTwoSpecialBeginningSlidesPL}{%
  \begin{frame}[standout]

    \begin{textblock}{11}(1,0.7)

      \begin{flushleft}

        \mdseries

        \footnotesize

        \color{jFrametitleFGColor}

        Materiał powstał w ramach projektu współfinansowanego ze środków
        Unii Europejskiej w ramach Europejskiego Funduszu Społecznego
        POWR.03.05.00-00-Z309/17-00.

      \end{flushleft}

    \end{textblock}





    \begin{textblock}{10}(0,2.2)

      \tikz \fill[color=jBackgroundStyleLight] (0,0) rectangle (12.8,-1.5);

    \end{textblock}


    \begin{textblock}{3.2}(1,2.45)

      \includegraphics[scale=0.3]{\FundingLogoColorPicturePL}

    \end{textblock}


    \begin{textblock}{2.5}(3.7,2.5)

      \includegraphics[scale=0.2]{\JULogoColorPicturePL}

    \end{textblock}


    \begin{textblock}{2.5}(6,2.4)

      \includegraphics[scale=0.1]{\ZintegrUJLogoColorPicturePL}

    \end{textblock}


    \begin{textblock}{4.2}(8.4,2.6)

      \includegraphics[scale=0.3]{\EUSocialFundLogoColorPicturePL}

    \end{textblock}

  \end{frame}





  \TitleSlideWithPicture
}



\newcommand{\GeometryThreeDSpecialEndingSlidePL}{%
  \begin{frame}[standout]

    \begin{textblock}{11}(1,0.7)

      \begin{flushleft}

        \mdseries

        \footnotesize

        \color{jFrametitleFGColor}

        Materiał powstał w ramach projektu współfinansowanego ze środków
        Unii Europejskiej w~ramach Europejskiego Funduszu Społecznego
        POWR.03.05.00-00-Z309/17-00.

      \end{flushleft}

    \end{textblock}





    \begin{textblock}{10}(0,2.2)

      \tikz \fill[color=jBackgroundStyleLight] (0,0) rectangle (12.8,-1.5);

    \end{textblock}


    \begin{textblock}{3.2}(1,2.45)

      \includegraphics[scale=0.3]{\FundingLogoColorPicturePL}

    \end{textblock}


    \begin{textblock}{2.5}(3.7,2.5)

      \includegraphics[scale=0.2]{\JULogoColorPicturePL}

    \end{textblock}


    \begin{textblock}{2.5}(6,2.4)

      \includegraphics[scale=0.1]{\ZintegrUJLogoColorPicturePL}

    \end{textblock}


    \begin{textblock}{4.2}(8.4,2.6)

      \includegraphics[scale=0.3]{\EUSocialFundLogoColorPicturePL}

    \end{textblock}





    \begin{textblock}{11}(1,4)

      \begin{flushleft}

        \mdseries

        \footnotesize

        \RaggedRight

        \color{jFrametitleFGColor}

        Treść niniejszego wykładu jest udostępniona na~licencji
        Creative Commons (\textsc{cc}), z~uzna\-niem autorstwa
        (\textsc{by}) oraz udostępnianiem na tych samych warunkach
        (\textsc{sa}). Rysunki i~wy\-kresy zawarte w~wykładzie są
        autorstwa dr.~hab.~Pawła Węgrzyna et~al. i~są dostępne
        na tej samej licencji, o~ile nie wskazano inaczej.
        W~prezentacji wykorzystano temat Beamera Jagiellonian,
        oparty na~temacie Metropolis Matthiasa Vogelgesanga,
        dostępnym na licencji \LaTeX{} Project Public License~1.3c
        pod adresem: \colorhref{https://github.com/matze/mtheme}
        {https://github.com/matze/mtheme}.

        Projekt typograficzny: Iwona Grabska-Gradzińska \\
        Skład: Kamil Ziemian;
        Korekta: Wojciech Palacz \\
        Modele: Dariusz Frymus, Kamil Nowakowski \\
        Rysunki i~wykresy: Kamil Ziemian, Paweł Węgrzyn, Wojciech Palacz

      \end{flushleft}

    \end{textblock}

  \end{frame}
}



\newcommand{\GeometryThreeDTwoSpecialEndingSlidesPL}[1]{%
  \begin{frame}[standout]


    \begin{textblock}{11}(1,0.7)

      \begin{flushleft}

        \mdseries

        \footnotesize

        \color{jFrametitleFGColor}

        Materiał powstał w ramach projektu współfinansowanego ze środków
        Unii Europejskiej w~ramach Europejskiego Funduszu Społecznego
        POWR.03.05.00-00-Z309/17-00.

      \end{flushleft}

    \end{textblock}





    \begin{textblock}{10}(0,2.2)

      \tikz \fill[color=jBackgroundStyleLight] (0,0) rectangle (12.8,-1.5);

    \end{textblock}


    \begin{textblock}{3.2}(1,2.45)

      \includegraphics[scale=0.3]{\FundingLogoColorPicturePL}

    \end{textblock}


    \begin{textblock}{2.5}(3.7,2.5)

      \includegraphics[scale=0.2]{\JULogoColorPicturePL}

    \end{textblock}


    \begin{textblock}{2.5}(6,2.4)

      \includegraphics[scale=0.1]{\ZintegrUJLogoColorPicturePL}

    \end{textblock}


    \begin{textblock}{4.2}(8.4,2.6)

      \includegraphics[scale=0.3]{\EUSocialFundLogoColorPicturePL}

    \end{textblock}





    \begin{textblock}{11}(1,4)

      \begin{flushleft}

        \mdseries

        \footnotesize

        \RaggedRight

        \color{jFrametitleFGColor}

        Treść niniejszego wykładu jest udostępniona na~licencji
        Creative Commons (\textsc{cc}), z~uzna\-niem autorstwa
        (\textsc{by}) oraz udostępnianiem na tych samych warunkach
        (\textsc{sa}). Rysunki i~wy\-kresy zawarte w~wykładzie są
        autorstwa dr.~hab.~Pawła Węgrzyna et~al. i~są dostępne
        na tej samej licencji, o~ile nie wskazano inaczej.
        W~prezentacji wykorzystano temat Beamera Jagiellonian,
        oparty na~temacie Metropolis Matthiasa Vogelgesanga,
        dostępnym na licencji \LaTeX{} Project Public License~1.3c
        pod adresem: \colorhref{https://github.com/matze/mtheme}
        {https://github.com/matze/mtheme}.

        Projekt typograficzny: Iwona Grabska-Gradzińska \\
        Skład: Kamil Ziemian;
        Korekta: Wojciech Palacz \\
        Modele: Dariusz Frymus, Kamil Nowakowski \\
        Rysunki i~wykresy: Kamil Ziemian, Paweł Węgrzyn, Wojciech Palacz

      \end{flushleft}

    \end{textblock}

  \end{frame}





  \begin{frame}[standout]

    \begingroup

    \color{jFrametitleFGColor}

    #1

    \endgroup

  \end{frame}
}



\newcommand{\GeometryThreeDSpecialEndingSlideVideoPL}{%
  \begin{frame}[standout]

    \begin{textblock}{11}(1,0.7)

      \begin{flushleft}

        \mdseries

        \footnotesize

        \color{jFrametitleFGColor}

        Materiał powstał w ramach projektu współfinansowanego ze środków
        Unii Europejskiej w~ramach Europejskiego Funduszu Społecznego
        POWR.03.05.00-00-Z309/17-00.

      \end{flushleft}

    \end{textblock}





    \begin{textblock}{10}(0,2.2)

      \tikz \fill[color=jBackgroundStyleLight] (0,0) rectangle (12.8,-1.5);

    \end{textblock}


    \begin{textblock}{3.2}(1,2.45)

      \includegraphics[scale=0.3]{\FundingLogoColorPicturePL}

    \end{textblock}


    \begin{textblock}{2.5}(3.7,2.5)

      \includegraphics[scale=0.2]{\JULogoColorPicturePL}

    \end{textblock}


    \begin{textblock}{2.5}(6,2.4)

      \includegraphics[scale=0.1]{\ZintegrUJLogoColorPicturePL}

    \end{textblock}


    \begin{textblock}{4.2}(8.4,2.6)

      \includegraphics[scale=0.3]{\EUSocialFundLogoColorPicturePL}

    \end{textblock}





    \begin{textblock}{11}(1,4)

      \begin{flushleft}

        \mdseries

        \footnotesize

        \RaggedRight

        \color{jFrametitleFGColor}

        Treść niniejszego wykładu jest udostępniona na~licencji
        Creative Commons (\textsc{cc}), z~uzna\-niem autorstwa
        (\textsc{by}) oraz udostępnianiem na tych samych warunkach
        (\textsc{sa}). Rysunki i~wy\-kresy zawarte w~wykładzie są
        autorstwa dr.~hab.~Pawła Węgrzyna et~al. i~są dostępne
        na tej samej licencji, o~ile nie wskazano inaczej.
        W~prezentacji wykorzystano temat Beamera Jagiellonian,
        oparty na~temacie Metropolis Matthiasa Vogelgesanga,
        dostępnym na licencji \LaTeX{} Project Public License~1.3c
        pod adresem: \colorhref{https://github.com/matze/mtheme}
        {https://github.com/matze/mtheme}.

        Projekt typograficzny: Iwona Grabska-Gradzińska;
        Skład: Kamil Ziemian \\
        Korekta: Wojciech Palacz;
        Modele: Dariusz Frymus, Kamil Nowakowski \\
        Rysunki i~wykresy: Kamil Ziemian, Paweł Węgrzyn, Wojciech Palacz \\
        Montaż: Agencja Filmowa Film \& Television Production~-- Zbigniew
        Masklak

      \end{flushleft}

    \end{textblock}

  \end{frame}
}





\newcommand{\GeometryThreeDTwoSpecialEndingSlidesVideoPL}[1]{%
  \begin{frame}[standout]

    \begin{textblock}{11}(1,0.7)

      \begin{flushleft}

        \mdseries

        \footnotesize

        \color{jFrametitleFGColor}

        Materiał powstał w ramach projektu współfinansowanego ze środków
        Unii Europejskiej w~ramach Europejskiego Funduszu Społecznego
        POWR.03.05.00-00-Z309/17-00.

      \end{flushleft}

    \end{textblock}





    \begin{textblock}{10}(0,2.2)

      \tikz \fill[color=jBackgroundStyleLight] (0,0) rectangle (12.8,-1.5);

    \end{textblock}


    \begin{textblock}{3.2}(1,2.45)

      \includegraphics[scale=0.3]{\FundingLogoColorPicturePL}

    \end{textblock}


    \begin{textblock}{2.5}(3.7,2.5)

      \includegraphics[scale=0.2]{\JULogoColorPicturePL}

    \end{textblock}


    \begin{textblock}{2.5}(6,2.4)

      \includegraphics[scale=0.1]{\ZintegrUJLogoColorPicturePL}

    \end{textblock}


    \begin{textblock}{4.2}(8.4,2.6)

      \includegraphics[scale=0.3]{\EUSocialFundLogoColorPicturePL}

    \end{textblock}





    \begin{textblock}{11}(1,4)

      \begin{flushleft}

        \mdseries

        \footnotesize

        \RaggedRight

        \color{jFrametitleFGColor}

        Treść niniejszego wykładu jest udostępniona na~licencji
        Creative Commons (\textsc{cc}), z~uzna\-niem autorstwa
        (\textsc{by}) oraz udostępnianiem na tych samych warunkach
        (\textsc{sa}). Rysunki i~wy\-kresy zawarte w~wykładzie są
        autorstwa dr.~hab.~Pawła Węgrzyna et~al. i~są dostępne
        na tej samej licencji, o~ile nie wskazano inaczej.
        W~prezentacji wykorzystano temat Beamera Jagiellonian,
        oparty na~temacie Metropolis Matthiasa Vogelgesanga,
        dostępnym na licencji \LaTeX{} Project Public License~1.3c
        pod adresem: \colorhref{https://github.com/matze/mtheme}
        {https://github.com/matze/mtheme}.

        Projekt typograficzny: Iwona Grabska-Gradzińska;
        Skład: Kamil Ziemian \\
        Korekta: Wojciech Palacz;
        Modele: Dariusz Frymus, Kamil Nowakowski \\
        Rysunki i~wykresy: Kamil Ziemian, Paweł Węgrzyn, Wojciech Palacz \\
        Montaż: Agencja Filmowa Film \& Television Production~-- Zbigniew
        Masklak

      \end{flushleft}

    \end{textblock}

  \end{frame}





  \begin{frame}[standout]


    \begingroup

    \color{jFrametitleFGColor}

    #1

    \endgroup

  \end{frame}
}










% ---------------------------------------
% Commands for lectures "Geometria 3D dla twórców gier wideo"
% English version
% ---------------------------------------
% \newcommand{\FundingLogoWhitePictureEN}
% {./PresentationPictures/CommonPictures/logotypFundusze_biale_bez_tla2.pdf}
\newcommand{\FundingLogoColorPictureEN}
{./PresentationPictures/CommonPictures/European_Funds_color_EN.pdf}
% \newcommand{\EULogoWhitePictureEN}
% {./PresentationPictures/CommonPictures/logotypUE_biale_bez_tla2.pdf}
\newcommand{\EUSocialFundLogoColorPictureEN}
{./PresentationPictures/CommonPictures/EU_Social_Fund_color_EN.pdf}
% \newcommand{\ZintegrUJLogoWhitePictureEN}
% {./PresentationPictures/CommonPictures/zintegruj-logo-white.pdf}
\newcommand{\ZintegrUJLogoColorPictureEN}
{./PresentationPictures/CommonPictures/ZintegrUJ_color.pdf}
\newcommand{\JULogoColorPictureEN}
{./JagiellonianPictures/LogoJU_EN/LogoJU_A_color.pdf}



\newcommand{\GeometryThreeDSpecialBeginningSlideEN}{%
  \begin{frame}[standout]

    \begin{textblock}{11}(1,0.7)

      \begin{flushleft}

        \mdseries

        \footnotesize

        \color{jFrametitleFGColor}

        This content was created as part of a project co-financed by the
        European Union within the framework of the European Social Fund
        POWR.03.05.00-00-Z309/17-00.

      \end{flushleft}

    \end{textblock}





    \begin{textblock}{10}(0,2.2)

      \tikz \fill[color=jBackgroundStyleLight] (0,0) rectangle (12.8,-1.5);

    \end{textblock}


    \begin{textblock}{3.2}(0.7,2.45)

      \includegraphics[scale=0.3]{\FundingLogoColorPictureEN}

    \end{textblock}


    \begin{textblock}{2.5}(4.15,2.5)

      \includegraphics[scale=0.2]{\JULogoColorPictureEN}

    \end{textblock}


    \begin{textblock}{2.5}(6.35,2.4)

      \includegraphics[scale=0.1]{\ZintegrUJLogoColorPictureEN}

    \end{textblock}


    \begin{textblock}{4.2}(8.4,2.6)

      \includegraphics[scale=0.3]{\EUSocialFundLogoColorPictureEN}

    \end{textblock}

  \end{frame}
}



\newcommand{\GeometryThreeDTwoSpecialBeginningSlidesEN}{%
  \begin{frame}[standout]

    \begin{textblock}{11}(1,0.7)

      \begin{flushleft}

        \mdseries

        \footnotesize

        \color{jFrametitleFGColor}

        This content was created as part of a project co-financed by the
        European Union within the framework of the European Social Fund
        POWR.03.05.00-00-Z309/17-00.

      \end{flushleft}

    \end{textblock}





    \begin{textblock}{10}(0,2.2)

      \tikz \fill[color=jBackgroundStyleLight] (0,0) rectangle (12.8,-1.5);

    \end{textblock}


    \begin{textblock}{3.2}(0.7,2.45)

      \includegraphics[scale=0.3]{\FundingLogoColorPictureEN}

    \end{textblock}


    \begin{textblock}{2.5}(4.15,2.5)

      \includegraphics[scale=0.2]{\JULogoColorPictureEN}

    \end{textblock}


    \begin{textblock}{2.5}(6.35,2.4)

      \includegraphics[scale=0.1]{\ZintegrUJLogoColorPictureEN}

    \end{textblock}


    \begin{textblock}{4.2}(8.4,2.6)

      \includegraphics[scale=0.3]{\EUSocialFundLogoColorPictureEN}

    \end{textblock}

  \end{frame}





  \TitleSlideWithPicture
}



\newcommand{\GeometryThreeDSpecialEndingSlideEN}{%
  \begin{frame}[standout]

    \begin{textblock}{11}(1,0.7)

      \begin{flushleft}

        \mdseries

        \footnotesize

        \color{jFrametitleFGColor}

        This content was created as part of a project co-financed by the
        European Union within the framework of the European Social Fund
        POWR.03.05.00-00-Z309/17-00.

      \end{flushleft}

    \end{textblock}





    \begin{textblock}{10}(0,2.2)

      \tikz \fill[color=jBackgroundStyleLight] (0,0) rectangle (12.8,-1.5);

    \end{textblock}


    \begin{textblock}{3.2}(0.7,2.45)

      \includegraphics[scale=0.3]{\FundingLogoColorPictureEN}

    \end{textblock}


    \begin{textblock}{2.5}(4.15,2.5)

      \includegraphics[scale=0.2]{\JULogoColorPictureEN}

    \end{textblock}


    \begin{textblock}{2.5}(6.35,2.4)

      \includegraphics[scale=0.1]{\ZintegrUJLogoColorPictureEN}

    \end{textblock}


    \begin{textblock}{4.2}(8.4,2.6)

      \includegraphics[scale=0.3]{\EUSocialFundLogoColorPictureEN}

    \end{textblock}





    \begin{textblock}{11}(1,4)

      \begin{flushleft}

        \mdseries

        \footnotesize

        \RaggedRight

        \color{jFrametitleFGColor}

        The content of this lecture is made available under a~Creative
        Commons licence (\textsc{cc}), giving the author the credits
        (\textsc{by}) and putting an obligation to share on the same terms
        (\textsc{sa}). Figures and diagrams included in the lecture are
        authored by Paweł Węgrzyn et~al., and are available under the same
        license unless indicated otherwise.\\ The presentation uses the
        Beamer Jagiellonian theme based on Matthias Vogelgesang’s
        Metropolis theme, available under license \LaTeX{} Project
        Public License~1.3c at: \colorhref{https://github.com/matze/mtheme}
        {https://github.com/matze/mtheme}.

        Typographic design: Iwona Grabska-Gradzińska \\
        \LaTeX{} Typesetting: Kamil Ziemian \\
        Proofreading: Wojciech Palacz,
        Monika Stawicka \\
        3D Models: Dariusz Frymus, Kamil Nowakowski \\
        Figures and charts: Kamil Ziemian, Paweł Węgrzyn, Wojciech Palacz

      \end{flushleft}

    \end{textblock}

  \end{frame}
}



\newcommand{\GeometryThreeDTwoSpecialEndingSlidesEN}[1]{%
  \begin{frame}[standout]


    \begin{textblock}{11}(1,0.7)

      \begin{flushleft}

        \mdseries

        \footnotesize

        \color{jFrametitleFGColor}

        This content was created as part of a project co-financed by the
        European Union within the framework of the European Social Fund
        POWR.03.05.00-00-Z309/17-00.

      \end{flushleft}

    \end{textblock}





    \begin{textblock}{10}(0,2.2)

      \tikz \fill[color=jBackgroundStyleLight] (0,0) rectangle (12.8,-1.5);

    \end{textblock}


    \begin{textblock}{3.2}(0.7,2.45)

      \includegraphics[scale=0.3]{\FundingLogoColorPictureEN}

    \end{textblock}


    \begin{textblock}{2.5}(4.15,2.5)

      \includegraphics[scale=0.2]{\JULogoColorPictureEN}

    \end{textblock}


    \begin{textblock}{2.5}(6.35,2.4)

      \includegraphics[scale=0.1]{\ZintegrUJLogoColorPictureEN}

    \end{textblock}


    \begin{textblock}{4.2}(8.4,2.6)

      \includegraphics[scale=0.3]{\EUSocialFundLogoColorPictureEN}

    \end{textblock}





    \begin{textblock}{11}(1,4)

      \begin{flushleft}

        \mdseries

        \footnotesize

        \RaggedRight

        \color{jFrametitleFGColor}

        The content of this lecture is made available under a~Creative
        Commons licence (\textsc{cc}), giving the author the credits
        (\textsc{by}) and putting an obligation to share on the same terms
        (\textsc{sa}). Figures and diagrams included in the lecture are
        authored by Paweł Węgrzyn et~al., and are available under the same
        license unless indicated otherwise.\\ The presentation uses the
        Beamer Jagiellonian theme based on Matthias Vogelgesang’s
        Metropolis theme, available under license \LaTeX{} Project
        Public License~1.3c at: \colorhref{https://github.com/matze/mtheme}
        {https://github.com/matze/mtheme}.

        Typographic design: Iwona Grabska-Gradzińska \\
        \LaTeX{} Typesetting: Kamil Ziemian \\
        Proofreading: Wojciech Palacz,
        Monika Stawicka \\
        3D Models: Dariusz Frymus, Kamil Nowakowski \\
        Figures and charts: Kamil Ziemian, Paweł Węgrzyn, Wojciech Palacz

      \end{flushleft}

    \end{textblock}

  \end{frame}





  \begin{frame}[standout]

    \begingroup

    \color{jFrametitleFGColor}

    #1

    \endgroup

  \end{frame}
}



\newcommand{\GeometryThreeDSpecialEndingSlideVideoVerOneEN}{%
  \begin{frame}[standout]

    \begin{textblock}{11}(1,0.7)

      \begin{flushleft}

        \mdseries

        \footnotesize

        \color{jFrametitleFGColor}

        This content was created as part of a project co-financed by the
        European Union within the framework of the European Social Fund
        POWR.03.05.00-00-Z309/17-00.

      \end{flushleft}

    \end{textblock}





    \begin{textblock}{10}(0,2.2)

      \tikz \fill[color=jBackgroundStyleLight] (0,0) rectangle (12.8,-1.5);

    \end{textblock}


    \begin{textblock}{3.2}(0.7,2.45)

      \includegraphics[scale=0.3]{\FundingLogoColorPictureEN}

    \end{textblock}


    \begin{textblock}{2.5}(4.15,2.5)

      \includegraphics[scale=0.2]{\JULogoColorPictureEN}

    \end{textblock}


    \begin{textblock}{2.5}(6.35,2.4)

      \includegraphics[scale=0.1]{\ZintegrUJLogoColorPictureEN}

    \end{textblock}


    \begin{textblock}{4.2}(8.4,2.6)

      \includegraphics[scale=0.3]{\EUSocialFundLogoColorPictureEN}

    \end{textblock}





    \begin{textblock}{11}(1,4)

      \begin{flushleft}

        \mdseries

        \footnotesize

        \RaggedRight

        \color{jFrametitleFGColor}

        The content of this lecture is made available under a Creative
        Commons licence (\textsc{cc}), giving the author the credits
        (\textsc{by}) and putting an obligation to share on the same terms
        (\textsc{sa}). Figures and diagrams included in the lecture are
        authored by Paweł Węgrzyn et~al., and are available under the same
        license unless indicated otherwise.\\ The presentation uses the
        Beamer Jagiellonian theme based on Matthias Vogelgesang’s
        Metropolis theme, available under license \LaTeX{} Project
        Public License~1.3c at: \colorhref{https://github.com/matze/mtheme}
        {https://github.com/matze/mtheme}.

        Typographic design: Iwona Grabska-Gradzińska;
        \LaTeX{} Typesetting: Kamil Ziemian \\
        Proofreading: Wojciech Palacz,
        Monika Stawicka \\
        3D Models: Dariusz Frymus, Kamil Nowakowski \\
        Figures and charts: Kamil Ziemian, Paweł Węgrzyn, Wojciech
        Palacz \\
        Film editing: Agencja Filmowa Film \& Television Production~--
        Zbigniew Masklak

      \end{flushleft}

    \end{textblock}

  \end{frame}
}



\newcommand{\GeometryThreeDSpecialEndingSlideVideoVerTwoEN}{%
  \begin{frame}[standout]

    \begin{textblock}{11}(1,0.7)

      \begin{flushleft}

        \mdseries

        \footnotesize

        \color{jFrametitleFGColor}

        This content was created as part of a project co-financed by the
        European Union within the framework of the European Social Fund
        POWR.03.05.00-00-Z309/17-00.

      \end{flushleft}

    \end{textblock}





    \begin{textblock}{10}(0,2.2)

      \tikz \fill[color=jBackgroundStyleLight] (0,0) rectangle (12.8,-1.5);

    \end{textblock}


    \begin{textblock}{3.2}(0.7,2.45)

      \includegraphics[scale=0.3]{\FundingLogoColorPictureEN}

    \end{textblock}


    \begin{textblock}{2.5}(4.15,2.5)

      \includegraphics[scale=0.2]{\JULogoColorPictureEN}

    \end{textblock}


    \begin{textblock}{2.5}(6.35,2.4)

      \includegraphics[scale=0.1]{\ZintegrUJLogoColorPictureEN}

    \end{textblock}


    \begin{textblock}{4.2}(8.4,2.6)

      \includegraphics[scale=0.3]{\EUSocialFundLogoColorPictureEN}

    \end{textblock}





    \begin{textblock}{11}(1,4)

      \begin{flushleft}

        \mdseries

        \footnotesize

        \RaggedRight

        \color{jFrametitleFGColor}

        The content of this lecture is made available under a Creative
        Commons licence (\textsc{cc}), giving the author the credits
        (\textsc{by}) and putting an obligation to share on the same terms
        (\textsc{sa}). Figures and diagrams included in the lecture are
        authored by Paweł Węgrzyn et~al., and are available under the same
        license unless indicated otherwise.\\ The presentation uses the
        Beamer Jagiellonian theme based on Matthias Vogelgesang’s
        Metropolis theme, available under license \LaTeX{} Project
        Public License~1.3c at: \colorhref{https://github.com/matze/mtheme}
        {https://github.com/matze/mtheme}.

        Typographic design: Iwona Grabska-Gradzińska;
        \LaTeX{} Typesetting: Kamil Ziemian \\
        Proofreading: Wojciech Palacz,
        Monika Stawicka \\
        3D Models: Dariusz Frymus, Kamil Nowakowski \\
        Figures and charts: Kamil Ziemian, Paweł Węgrzyn, Wojciech
        Palacz \\
        Film editing: IMAVI -- Joanna Kozakiewicz, Krzysztof Magda, Nikodem
        Frodyma

      \end{flushleft}

    \end{textblock}

  \end{frame}
}



\newcommand{\GeometryThreeDSpecialEndingSlideVideoVerThreeEN}{%
  \begin{frame}[standout]

    \begin{textblock}{11}(1,0.7)

      \begin{flushleft}

        \mdseries

        \footnotesize

        \color{jFrametitleFGColor}

        This content was created as part of a project co-financed by the
        European Union within the framework of the European Social Fund
        POWR.03.05.00-00-Z309/17-00.

      \end{flushleft}

    \end{textblock}





    \begin{textblock}{10}(0,2.2)

      \tikz \fill[color=jBackgroundStyleLight] (0,0) rectangle (12.8,-1.5);

    \end{textblock}


    \begin{textblock}{3.2}(0.7,2.45)

      \includegraphics[scale=0.3]{\FundingLogoColorPictureEN}

    \end{textblock}


    \begin{textblock}{2.5}(4.15,2.5)

      \includegraphics[scale=0.2]{\JULogoColorPictureEN}

    \end{textblock}


    \begin{textblock}{2.5}(6.35,2.4)

      \includegraphics[scale=0.1]{\ZintegrUJLogoColorPictureEN}

    \end{textblock}


    \begin{textblock}{4.2}(8.4,2.6)

      \includegraphics[scale=0.3]{\EUSocialFundLogoColorPictureEN}

    \end{textblock}





    \begin{textblock}{11}(1,4)

      \begin{flushleft}

        \mdseries

        \footnotesize

        \RaggedRight

        \color{jFrametitleFGColor}

        The content of this lecture is made available under a Creative
        Commons licence (\textsc{cc}), giving the author the credits
        (\textsc{by}) and putting an obligation to share on the same terms
        (\textsc{sa}). Figures and diagrams included in the lecture are
        authored by Paweł Węgrzyn et~al., and are available under the same
        license unless indicated otherwise.\\ The presentation uses the
        Beamer Jagiellonian theme based on Matthias Vogelgesang’s
        Metropolis theme, available under license \LaTeX{} Project
        Public License~1.3c at: \colorhref{https://github.com/matze/mtheme}
        {https://github.com/matze/mtheme}.

        Typographic design: Iwona Grabska-Gradzińska;
        \LaTeX{} Typesetting: Kamil Ziemian \\
        Proofreading: Wojciech Palacz,
        Monika Stawicka \\
        3D Models: Dariusz Frymus, Kamil Nowakowski \\
        Figures and charts: Kamil Ziemian, Paweł Węgrzyn, Wojciech
        Palacz \\
        Film editing: Agencja Filmowa Film \& Television Production~--
        Zbigniew Masklak \\
        Film editing: IMAVI -- Joanna Kozakiewicz, Krzysztof Magda, Nikodem
        Frodyma

      \end{flushleft}

    \end{textblock}

  \end{frame}
}



\newcommand{\GeometryThreeDTwoSpecialEndingSlidesVideoVerOneEN}[1]{%
  \begin{frame}[standout]

    \begin{textblock}{11}(1,0.7)

      \begin{flushleft}

        \mdseries

        \footnotesize

        \color{jFrametitleFGColor}

        This content was created as part of a project co-financed by the
        European Union within the framework of the European Social Fund
        POWR.03.05.00-00-Z309/17-00.

      \end{flushleft}

    \end{textblock}





    \begin{textblock}{10}(0,2.2)

      \tikz \fill[color=jBackgroundStyleLight] (0,0) rectangle (12.8,-1.5);

    \end{textblock}


    \begin{textblock}{3.2}(0.7,2.45)

      \includegraphics[scale=0.3]{\FundingLogoColorPictureEN}

    \end{textblock}


    \begin{textblock}{2.5}(4.15,2.5)

      \includegraphics[scale=0.2]{\JULogoColorPictureEN}

    \end{textblock}


    \begin{textblock}{2.5}(6.35,2.4)

      \includegraphics[scale=0.1]{\ZintegrUJLogoColorPictureEN}

    \end{textblock}


    \begin{textblock}{4.2}(8.4,2.6)

      \includegraphics[scale=0.3]{\EUSocialFundLogoColorPictureEN}

    \end{textblock}





    \begin{textblock}{11}(1,4)

      \begin{flushleft}

        \mdseries

        \footnotesize

        \RaggedRight

        \color{jFrametitleFGColor}

        The content of this lecture is made available under a Creative
        Commons licence (\textsc{cc}), giving the author the credits
        (\textsc{by}) and putting an obligation to share on the same terms
        (\textsc{sa}). Figures and diagrams included in the lecture are
        authored by Paweł Węgrzyn et~al., and are available under the same
        license unless indicated otherwise.\\ The presentation uses the
        Beamer Jagiellonian theme based on Matthias Vogelgesang’s
        Metropolis theme, available under license \LaTeX{} Project
        Public License~1.3c at: \colorhref{https://github.com/matze/mtheme}
        {https://github.com/matze/mtheme}.

        Typographic design: Iwona Grabska-Gradzińska;
        \LaTeX{} Typesetting: Kamil Ziemian \\
        Proofreading: Wojciech Palacz,
        Monika Stawicka \\
        3D Models: Dariusz Frymus, Kamil Nowakowski \\
        Figures and charts: Kamil Ziemian, Paweł Węgrzyn,
        Wojciech Palacz \\
        Film editing: Agencja Filmowa Film \& Television Production~--
        Zbigniew Masklak

      \end{flushleft}

    \end{textblock}

  \end{frame}





  \begin{frame}[standout]


    \begingroup

    \color{jFrametitleFGColor}

    #1

    \endgroup

  \end{frame}
}



\newcommand{\GeometryThreeDTwoSpecialEndingSlidesVideoVerTwoEN}[1]{%
  \begin{frame}[standout]

    \begin{textblock}{11}(1,0.7)

      \begin{flushleft}

        \mdseries

        \footnotesize

        \color{jFrametitleFGColor}

        This content was created as part of a project co-financed by the
        European Union within the framework of the European Social Fund
        POWR.03.05.00-00-Z309/17-00.

      \end{flushleft}

    \end{textblock}





    \begin{textblock}{10}(0,2.2)

      \tikz \fill[color=jBackgroundStyleLight] (0,0) rectangle (12.8,-1.5);

    \end{textblock}


    \begin{textblock}{3.2}(0.7,2.45)

      \includegraphics[scale=0.3]{\FundingLogoColorPictureEN}

    \end{textblock}


    \begin{textblock}{2.5}(4.15,2.5)

      \includegraphics[scale=0.2]{\JULogoColorPictureEN}

    \end{textblock}


    \begin{textblock}{2.5}(6.35,2.4)

      \includegraphics[scale=0.1]{\ZintegrUJLogoColorPictureEN}

    \end{textblock}


    \begin{textblock}{4.2}(8.4,2.6)

      \includegraphics[scale=0.3]{\EUSocialFundLogoColorPictureEN}

    \end{textblock}





    \begin{textblock}{11}(1,4)

      \begin{flushleft}

        \mdseries

        \footnotesize

        \RaggedRight

        \color{jFrametitleFGColor}

        The content of this lecture is made available under a Creative
        Commons licence (\textsc{cc}), giving the author the credits
        (\textsc{by}) and putting an obligation to share on the same terms
        (\textsc{sa}). Figures and diagrams included in the lecture are
        authored by Paweł Węgrzyn et~al., and are available under the same
        license unless indicated otherwise.\\ The presentation uses the
        Beamer Jagiellonian theme based on Matthias Vogelgesang’s
        Metropolis theme, available under license \LaTeX{} Project
        Public License~1.3c at: \colorhref{https://github.com/matze/mtheme}
        {https://github.com/matze/mtheme}.

        Typographic design: Iwona Grabska-Gradzińska;
        \LaTeX{} Typesetting: Kamil Ziemian \\
        Proofreading: Wojciech Palacz,
        Monika Stawicka \\
        3D Models: Dariusz Frymus, Kamil Nowakowski \\
        Figures and charts: Kamil Ziemian, Paweł Węgrzyn,
        Wojciech Palacz \\
        Film editing: IMAVI -- Joanna Kozakiewicz, Krzysztof Magda, Nikodem
        Frodyma

      \end{flushleft}

    \end{textblock}

  \end{frame}





  \begin{frame}[standout]


    \begingroup

    \color{jFrametitleFGColor}

    #1

    \endgroup

  \end{frame}
}



\newcommand{\GeometryThreeDTwoSpecialEndingSlidesVideoVerThreeEN}[1]{%
  \begin{frame}[standout]

    \begin{textblock}{11}(1,0.7)

      \begin{flushleft}

        \mdseries

        \footnotesize

        \color{jFrametitleFGColor}

        This content was created as part of a project co-financed by the
        European Union within the framework of the European Social Fund
        POWR.03.05.00-00-Z309/17-00.

      \end{flushleft}

    \end{textblock}





    \begin{textblock}{10}(0,2.2)

      \tikz \fill[color=jBackgroundStyleLight] (0,0) rectangle (12.8,-1.5);

    \end{textblock}


    \begin{textblock}{3.2}(0.7,2.45)

      \includegraphics[scale=0.3]{\FundingLogoColorPictureEN}

    \end{textblock}


    \begin{textblock}{2.5}(4.15,2.5)

      \includegraphics[scale=0.2]{\JULogoColorPictureEN}

    \end{textblock}


    \begin{textblock}{2.5}(6.35,2.4)

      \includegraphics[scale=0.1]{\ZintegrUJLogoColorPictureEN}

    \end{textblock}


    \begin{textblock}{4.2}(8.4,2.6)

      \includegraphics[scale=0.3]{\EUSocialFundLogoColorPictureEN}

    \end{textblock}





    \begin{textblock}{11}(1,4)

      \begin{flushleft}

        \mdseries

        \footnotesize

        \RaggedRight

        \color{jFrametitleFGColor}

        The content of this lecture is made available under a Creative
        Commons licence (\textsc{cc}), giving the author the credits
        (\textsc{by}) and putting an obligation to share on the same terms
        (\textsc{sa}). Figures and diagrams included in the lecture are
        authored by Paweł Węgrzyn et~al., and are available under the same
        license unless indicated otherwise. \\ The presentation uses the
        Beamer Jagiellonian theme based on Matthias Vogelgesang’s
        Metropolis theme, available under license \LaTeX{} Project
        Public License~1.3c at: \colorhref{https://github.com/matze/mtheme}
        {https://github.com/matze/mtheme}.

        Typographic design: Iwona Grabska-Gradzińska;
        \LaTeX{} Typesetting: Kamil Ziemian \\
        Proofreading: Leszek Hadasz, Wojciech Palacz,
        Monika Stawicka \\
        3D Models: Dariusz Frymus, Kamil Nowakowski \\
        Figures and charts: Kamil Ziemian, Paweł Węgrzyn,
        Wojciech Palacz \\
        Film editing: Agencja Filmowa Film \& Television Production~--
        Zbigniew Masklak \\
        Film editing: IMAVI -- Joanna Kozakiewicz, Krzysztof Magda, Nikodem
        Frodyma


      \end{flushleft}

    \end{textblock}

  \end{frame}





  \begin{frame}[standout]


    \begingroup

    \color{jFrametitleFGColor}

    #1

    \endgroup

  \end{frame}
}











% ---------------------------------------
% Packages, libraries and their configuration
% ---------------------------------------
\usetikzlibrary{positioning}
\usetikzlibrary{shadings}
\usetikzlibrary{shapes}





% ------------------------------
% Configuration for this particular presentation
% ------------------------------
\definecolor{jNormalTextForegroundBlueGrey}{HTML}{818F9B}
\definecolor{jNormalTextBackgroundDark}{HTML}{023159}
\definecolor{MathWhite}{HTML}{D6DEE5}
\definecolor{jMathTextForegroundGrey}{HTML}{4E4E4E}


% \tikzset{
%   sheetstyle/.style=[draw,rectangle,minimum heigth=4cm,text width=3cm,
%   fill=black!2,text centered,text color=jMathTextForegroundGrey]
% }
\tikzstyle{sheetstyle} = [rectangle,fill=black!2,minimum height=6em,text width=2.5em,text centered]

\tikzstyle{qubitplus} = [circle,minimum size=1.5cm,fill=gray]
\tikzstyle{qubitupperquater} = [circle,minimum size=1.5cm,top color=gray,
bottom color=white]
\tikzstyle{qubitlowerquater} = [circle,minimum size=1.5cm,top color=white,
bottom color=gray]










% ---------------------------------------------------------------------
\title{Wprowadzenie do programowania kwantowego, 02}

\author{Kamil Ziemian \\
 \texttt{kziemianfvt@gmail.com}}


\date{}
% ---------------------------------------------------------------------










% ####################################################################
% Początek dokumentu
\begin{document}
% ####################################################################





% Wyrównanie do lewej z łamaniem wyrazów

\RaggedRight





% ######################################
\maketitle % Tytuł całego tekstu
% ######################################










% ######################################
\begin{frame}
  \frametitle{????}


  \tableofcontents % Spis treści

\end{frame}
% ######################################










% ######################################
\section{Basic information}
% ######################################





% ##################
\begin{frame}
  \frametitle{Dlaczego programowanie kwantowe różnych~się \\
    od dotychczasowego?}


  Algorytm albo program komputerowy (w~języku C albo Python) możemy
  napisać na kartce papieru, nie o to nam jednak chodzi. Chcemy by
  ostatecznie został on wykonany przez jakiś komputer i~zrobił to
  czego od niego oczekujemy.

  To jak jest zbudowany i~jak działa komputer wpływa na to jak piszemy
  programy, choć rozwój komputerów sprawia, że~w~bardzo wielu
  przypadkach nie musimy już o~tym myśleć.

\end{frame}
% ##################





% ##################
\begin{frame}
  \frametitle{Dlaczego programowanie kwantowe różnych~się \\
    od dotychczasowego?}


  Przykładowo, kilkadziesiąt lat temu ze względu na niedoskonałość
  swojej budowy procesory relatywnie często~się myliły w~operacjach
  arytmetycznych i~potrafiły stwierdzić, że~$2 + 2 = 5$ (może
  przejaskrawiam). Z~tego powodu informatyków uczono odpowiednich
  algorytmów korekcyjnych i~sposobów pisania programów, by~wynajdywały
  oraz~poprawiały tego typu błędy.

  Kiedy jednak hardware stał się lepszy, szansa takich pomyłek zmalała
  do tego stopnia, że z~bardzo wielu % (może większości)
  programów studiów informatycznych usunięto naukę tych algorytmów
  i~ich implementacji.

  W~2018 roku pracownicy IBM podali, że ich komputery kwantowe zmagają
  się dokładnie z takim samym problemem i~potrzebują opracować
  analogiczne algorytmy korekcyjne, tylko dostosowane do kwantowych
  procesorów. To pokazuje na jakim etapie jesteśmy.

\end{frame}
% ##################





% ##################
\begin{frame}
  \frametitle{Dlaczego programowanie kwantowe różnych~się \\
    od dotychczasowego?}


  Przedmioty jakie widzimy gołym okiem mają zwykle $10^{ 23 }$ (mierzy
  to liczba Avogadra:
  $N_{ A } = 6.02\ldots \cdot 10^{ 23 } \si{.mol}^{ -1 }$. Jaka jest więc
  różnica między jednym atomem, a~zbiorowiskiem $10^{ 23 }$ atomów?

  Nie jest specjalnie trudne podniesienie przedmiotu o~masie
  $60 \si{.kg}$. Przedmiotu o~masie $10^{ 23 }$ większej nie możemy
  podnieść, bo nie ma jak. Jest to bowiem masa Ziemi, wszystkiego
  co~się na niej znajduje i~całej atmosfery (około
  $5.97 \cdot 10^{ 24 } \si{.kg}$). Nienależny się więc dziwić, że~są
  różnice między zachowaniem jednego atomu, a~przedmiotów wokół nas.
  Można, a~nawet należy~się dziwić, jak duże te różnice są.

  Mechanika Newtona wyjaśnię i~opisuje jak poruszają się samochody,
  długopisy, krzesła, etc. Analogiczna teoria dla małej ilości atomów
  nosi nazwę mechaniki kwantowej i~dzisiaj nie jest możliwe pracowanie
  z~komputerem kwantowym bez pewnego jej zrozumienia.

\end{frame}
% ##################





% ##################
\begin{frame}
  \frametitle{Dlaczego programowanie kwantowe różnych~się \\
  od dotychczasowego?}


  Komputer jest obiektem materialny, więc jego sposób działania jest
  związany z~prawami przyrody jakie rządzą materiałem z którego jest
  zbudowany. To niby oczywiste, ale często można o tym zapomnieć.

  Komputery które wszyscy znamy i~używamy, zbudowano z~takich
  materiałów i~w taki sposób, że~procesor działa zgodnie z~prawami
  fizyki klasycznej. Taki komputer będziemy określać jako komputer
  klasyczny, a jego część działającą w~oparciu o te prawa
  klasycznymi częściami. Np. klasyczny bit.

  Taki którego procesor zbudowano w~taki sposób, by procesor działa
  zgodnie z~prawami fizyki kwantowej, określamy jako komputer
  kwantowy (zgodnie z~przyjętymi już zwyczajami). Analogicznie
  nazywamy jego części.

\end{frame}
% ##################





% ##################
\begin{frame}
  \frametitle{Kwantowy hardware i~jego problemy}


  Jeśli mamy klasyczny bit o wartości $0$ albo $1$, to przy
  dzisiejszym stanie technologi, możemy przyjąć, że możemy go
  odczytywać nieskończoną ilość razy.

  Dla porównania, w 2018 roku na kwantowym komputerze odczyt
  kwantowego bitu można wykonać niewielką ilość razy, powiedzmy, że 10. Potem kwantowy bit ulegał rozbiciu (formalnie: de
  koherencji), informacja była tracona i należało uruchomić mechanizm
  jego odbudowy.

  Jest to jednak problem doskonałości hardware'u i możemy mieć
  nadzieję, że wraz z jego rozwojem będą minimalizowane.

\end{frame}
% ##################





% ##################
\begin{frame}
  \frametitle{Kwantowe programowanie i obliczenia}


  Nasz obecna wiedza fizyczna pozwala nam stwierdzić następujące
  rzeczy.

  Jeśli mamy poprawnie działający komputer klasyczny, i każmy mu 10
  razy dodać 1 do 1 to otrzymamy
  \begin{align*}
    2, 2, 2, 2, 2, 2, 2, 2, 2, 2.
  \end{align*}

  Jeśli mamy poprawnie działający komputer kwantowy, to otrzymamy
  zapewne jeden z poniższych rezultatów
  \begin{align*}
    &2, 1, 2, 2, 2, 3, 2, 2, 2, 2, \\
    &2, 2, 2, 2, 1, 2, 2, 2, 3, 2, \\
    &2, 2, 2, 2, 2, 2, 2, 2, 2, 2, \\
    &1, 1, 2, 2, 2, 2, 2, 2, 2, 2, \\
    &\vdots
  \end{align*}

\end{frame}
% ##################





% ##################
\begin{frame}
  \frametitle{Kwantowe programowanie i obliczenia}


  \begin{align*}
    &2, 1, 2, 2, 2, 3, 2, 2, 2, 2, \\
    &2, 2, 2, 2, 1, 2, 2, 2, 3, 2, \\
    &2, 2, 2, 2, 2, 2, 2, 2, 2, 2, \\
    &1, 1, 2, 2, 2, 2, 2, 2, 2, 2, \\
    &\vdots
  \end{align*}
  Błędne wyniki \alert{nie} są winną błędów złej konstrukcji
  kwantowego procesora. Według naszej obecnej wiedzy, jest to
  konsekwencja zbudowania procesora z materii ,,w stanie kwantowy''
  i każdy taki procesor musi się tak zachowywać.

  Inaczej mówiąc, to prawa fizyki dyktują to, że nawet idealny
  procesor kwantowy będzie od czasu do czasu dawał błędny wynik.

  Z drugiej strony widzimy, że zazwyczaj $1 + 1 = 2$ ;).

\end{frame}
% ##################





% ##################
\begin{frame}
  \frametitle{Trochę perspektywy}


  Jeśli ktoś miał magnetyczny dysk twardy o pojemności mierzonej w GB,
  to praktycznie na pewno był to kwantowy dysk twardy. Przejście od
  pojemności w MB do tej w GB jest związane z~budowaniem głowic
  czytników magnetycznych w~oparciu o~opisywane przez mechanikę
  kwantową zjawisko gigantycznego magnetooporu. Za jego odkrycie
  Albert Fert i Peter Gr\"{u}nberg otrzymali Nagrodę Nobla w 2007
  roku, zaś IBM wypuścił dyski komercyjne dyski w~tej technologii w
  1997 roku.

  Wielu ma nadzieję, że kwantowe procesory zrobią dla szybkości
  komputera to, co kwantowe dyski twarde zrobiły dla jego pojemności.

  Czemu więc nie przyjęła się nazwa kwantowy dysk twardy? Podejrzewam,
  że to decyzja działu marketingu.

\end{frame}
% ##################





% ##################
\begin{frame}
  \frametitle{Fizyka klasyczna (w uproszczeniu)}


  Stosunkowo najbliższa naszym codziennym doświadczeniom. Przyjmuje,
  że nie ma efektów kwantowych (o tym więcej później), a światło
  porusza się z nieskończoną prędkością. Oba te założenia są błędne,
  ale mimo tego ta część fizyki bardzo dobrze opisuje większość
  zjawisk widzianych gołym okiem. Most, mechaniczne części samochodów
  i samolotów, budujemy w oparciu o nią, a jeszcze się żaden nie
  rozbił z powodów przyjęcia skończonej prędkości światła.

  Do fizyki klasycznej należą między innymi.
  \begin{itemize}
    \RaggedRight

  \item mechanika Newtona,

  \item szkolna optyka,

  \item teoria sprężystości i ciał elastycznych (liny, struny
    instrumentów, etc.),

  \item hydrodynamika (nauka o ruchu cieczy),

  \item aerodynamika,

  \item termodynamika.

  \end{itemize}

\end{frame}
% ##################





% ##################
\begin{frame}
  \frametitle{Fizyka relatywistyczna (w uproszczeniu)}


  Przyjmuje, że nie ma efektów kwantowych, ale prędkość światła jest
  skończona. Stworzenie ogólnych jej podstaw przyniosło Albertowi
  Einsteinowi sławę, którą cieszy się do dziś.


  Należą do niej między innymi.
  \begin{itemize}
    \RaggedRight

  \item elektrodynamika Clerka Maxwella, a na tej niej bazuje nasza
    elektronika,

  \item szczególna teoria względności, Einstein wniósł w nią wielki
    wkład,

  \item ogólna teoria względności, stworzona głównie przez Einsteina,
    najlepsza znana teoria grawitacji.

  \end{itemize}

\end{frame}
% ##################





% ##################
\begin{frame}
  \frametitle{„Stany materii”}


  Każdy zna trzy stany skupienia wody: lód, ciesz i parę wodną.
  Podobnie można powiedzieć, że~otaczająca nas materia może być
  w~trzech „stanach skupienia”.

  \begin{itemize}
    \RaggedRight

  \item Stan kwantowy. Będziemy się nim dużo zajmowali.

  \item Stan klasyczny. W~nim jest większość przedmiotów wokół nas.

  \item Stan relatywistyczny. Kiedy prędkość obiektu jest zbliżona do
    prędkości światła: $c = 3 00 000 \, \frac{ \si{km} }{ \si{s} }$.

  \end{itemize}

  Tak jak z~wodą dany „stan skupienia” jest nie tyle własnością
  samej materii, co warunków w~których jest. Woda jest tą samą rzeczą
  w~-20\textcelsius{ } jaki i~w~200\textcelsius, choć wygląda
  i~zachowuje~się zupełnie inaczej.

  Tak samo otaczająca nas materia może przechodzić między tymi
  stanami, pozostając cały czas sobą, lecz zmieniając swoje własności.

\end{frame}
% ##################





% ##################
\begin{frame}
  \frametitle{Klasyczne i~kwantowe bity}


  Materiał z~której zbudowany jest normalny procesor znajduje~się
  w~klasycznym „stanie skupienia”, przez analogię będziemy więc
  mówili o~klasycznych komputerach, klasycznych bitach, etc.

  Materiał z~którego zbudowany jest kwantowy procesor znajduje~się
  w~kwantowym „stanie skupienia”, stąd nazwa.

  Kwantowy bit nazywamy \alert{qubitem} (ang. \emph{qubit} od
  \emph{QUantum BIT}), spotyka się też zapis „kubit”. Ze względów
  językowych próbowano wprowadzić „kwbit”, od „KWantowy BIT”, ale
  ta forma jest rzadko używana.

  Klasyczny bit może być tylko w~stanach reprezentujących jedną
  z~dwóch~wartości: 0, 1. By go reprezentować wystarczy moneta.

  Qubity są pod tym względem zupełnie inne. Oprócz 0 i 1, mogą być
  w~nieskończenie wielu innych stanach, które istnienie ujawnia się
  dopiero w~kwantowym stanie materii.

\end{frame}
% ##################





% ##################
\begin{frame}
  \frametitle{Hardware kwantowy}


  W~dzisiejszych komputerach kwantowych jeśli temperatura qubitu
  będzie większa niż -273\textcelsius{ } (dokładniej, 15 mK), to
  materiał z~którego jest wykonany przejdzie ze stanu kwantowego do
  stanu klasycznego, przez co stanie się on bezużyteczny. Trzeba wtedy
  uruchomić odpowiednią procedurę chłodzenia i~odbudowywania
  kwantowego „stanu skupienia” materiału. Temperatura nie jest
  niestety jedną rzeczą, która niszczy kwantową naturę qubitu.

  Zjawisko to w fizyce nosi nazwę dekoherencji, ja będą po prostu
  mówił, że qubit ulega zniszczeniu i~musi zostać odbudowany.

  Jeśli w~normalnym komputerze zapiszemy zdjęcie w~pamięci komputera,
  to możemy tworzące je bit odczytywać tyle razy ile chcemy.

  Obecnie informację z qubitu możemy odczytywać tylko niewielką liczbę
  razy, powiedzmy dla prostoty że 10. Wraz z~rozwoje technologii
  liczba ta będzie rosła.

\end{frame}
% ##################




% ##################
\begin{frame}
  \frametitle{Trzy rodzaje fizyki kwantowej}


  Obecnie ludzkość zna trzy główne wersje fizyki kwantowej.

  \begin{itemize}
    \RaggedRight

  \item \alert{Nierelatywistyczna mechanika kwantowa.} Przyjmuje, że
    prędkość światła jest nieskończona, dlatego jest w konflikcie ze
    szczególną teorią względności. Jest mimo tego bardzo użyteczna
    i~najlepiej zbadana ze wszystkich trzech. A biorąc pod uwagę jak
    wiele jej elementów wciąż jest niezrozumianych lub budzi
    kontrowersje i~zażarte dyskusje w środowisku naukowym, to tylko
    ilustruje ile o tym zjawiskach nie wiem. Wrócimy do tego potem.

  \item Relatywistyczna mechanika kwantowa. Dość ułomne połączenie fizyki
    kwantowej i~szczególnej teorii względności.

  \item Kwantowa teoria pola. Bardziej udane połączenie fizyki kwantowej
    i~szczególnej teorii względności. Jednak za słabo zrozumiana, by
    opracowywać odpowiednie komputery na jej podstawie.

  \end{itemize}

\end{frame}
% ##################





% ##################
\begin{frame}
  \frametitle{Mechanika kwantowa i~jej interpretacje}


  Mechanika kwantowa to teoria fizyczna na której opieramy nasze komputery
  kwantowe. Jest już dość starą teorią, za datę jej narodziny można uznać
  1925~r. Od tego czasu setki eksperymentów i~wynalazków (dyski
  twarde) potwierdziły, że~bardzo dobrze wyjaśnia i~opisuje świat.
  Pomimo tego wciąż na wiele pytań które ona niesie musimy
  odpowiadać „Nie wiem”.

  Przyczyną ten jest wiele. Do najważniejszych należy to, że
  eksperymenty które pozwoliłby nam rozwiązać pewne kwestie wciąż są
  zbyt trudne do wykonania. W~istocie są poważne osoby, które mówią,
  że~rozwój technologii napędzany przez prace nad komputerami
  kwantowymi oraz badania ich działania pozwoli nam wreszcie
  na wiele z~tych pytań odpowiedzieć.

  Niestety, wiele z~tych otwartych pytań dotyczy tego czym będziemy się
  zajmować. Musimy się więc pogodzić z~tym, że~pewnych rzecz nie rozumiemy.

\end{frame}
% ##################




% ##################
\begin{frame}
  \frametitle{Mechanika kwantowa i~jej interpretacje}


  Podano wiele możliwym rozwiązań tych problemów, po prostu nie umiemy
  stwierdzić które z~nich są błędne, a~które idą w~dobrą stronę.
  W~literaturze funkcjonują one pod nazwą „interpretacji mechaniki
  kwantowej”. Do najważniejszych z~nich należą:
  \begin{itemize}
    \RaggedRight

  \item interpretacja kopenhaska,

  \item teoria fali pilotującej De Broglie-Bohma,

  \item teorie kwantowej wiedzy (informacji),

  \item kwantowy bayesianism,

  \item interpretacja statystyczna (stochastyczna),

  \item logika kwantowa,

  \item teoria wieloświatów,

  \item kwantowy darwinizm.

  \end{itemize}

\end{frame}
% ##################





% ######################################
\section{Kwantowa kartka papieru}
% ######################################





% ##################
\begin{frame}
  \frametitle{Qubity są dziwne}


  Chyba każdy kto nie studiował mechaniki kwantowej i dla większości
  którzy to robili, qubity zachowują się po prostu dziwnie. Złośliwi
  mówią, że jeśli uważasz, że rozumiesz co tu się dzieje, to nie wiesz
  o czym mówisz.

  Dlatego teraz zajmiemy się prostym modelem bitu i qubitu w~postaci
  kartki papieru z 0 na jednej i 1 na drugiej stronie, którą nasz
  kolega kładzie na stole.

\end{frame}
% ##################





% ##################
\begin{frame}
  \frametitle{Klasyczna kartka papieru jako klasyczny bit}


  Nasz kolega Marian położył kartkę na stole. Podchodzimy do stołu i
  odczytujemy z kartki liczbę 0 albo 1. Wszystko jest tak proste, że
  nie trzeba nic tłumaczyć.
  % Nasz kolega chce nam przekazać informacje, że pewna liczba ma
  % wartość 0.
  % Za pomocą tej kartki papieru chcemy zapisać informacje, że pewna
  % liczba ma wartość 0.
  % Kładziemy kartkę na stole tak, by strona z 0 była na wierzchu.
  % Podchodzimy do stołu i~spoglądamy Jeśli potrzebujemy sobie
  % przypomnieć jaka jest wartość tej liczby, to patrzymy na tą kartkę
  % i odczytujemy z~niej liczbę 0.
  % Na kartkę możemy się popatrzeć dowolną ilość razy i kartka się nie
  % zużywa.





  \begin{tikzpicture}[auto]

    \node[sheetstyle] (zero) {\LARGE 0};

    \node[sheetstyle,right=of zero] (one) {\LARGE 1};

  \end{tikzpicture}

\end{frame}
% ##################





% ##################
\begin{frame}
  \frametitle{Kwantowa kartka papieru}


  Marian sobie tylko wiadomymi metodami zdobył kartkę papieru w stanie
  kwantowym i położył ją na stole. Kiedy pochodzimy do stołu i
  próbujemy odczytać co jest na kartce widzimy jedną z poniższych
  możliwości.





  \begin{tikzpicture}[node distance=0.5cm,auto]

    \node[sheetstyle] (zero) {\LARGE 0};

    \node[qubitlowerquater,right=of zero] (lowerquater) {};

    \node[qubitplus,right=of lowerquater] (qubitplus) {};

    \node[qubitupperquater,right=of qubitplus] (upperquater) {};

    \node[sheetstyle,right=of upperquater] (one) {\LARGE 1};

  \end{tikzpicture}

\end{frame}
% ##################





% ##################
\begin{frame}
  \frametitle{Kwantowa kartka papieru}


  \begin{tikzpicture}

    \node[qubitplus] (qubitplus 1) {};

    \node[sheetstyle,right=of qubitplus 1] (zero 1) {\Large 0};

    \node[qubitplus,below=of qubitplus 1] (qubitplus 2) {};

    \node[sheetstyle,right=of qubitplus 2] (zero 2) {\Large 1};

  \end{tikzpicture}

\end{frame}
% ##################









% ##################
\begin{frame}
  \frametitle{Kwantowa kartka papieru}


  \begin{tikzpicture}

    \node[qubitplus] (qubitplus 1) {};

    \node[sheetstyle,right=of qubitplus 1] (result 1 1) {\Large 0};

    \node[sheetstyle,right=of result 1 1] (result 1 2) {\Large 0};

    \node[sheetstyle,right=of result 1 2] (result 1 3) {\Large 1};

    \node[sheetstyle,right=of result 1 3] (result 1 4) {\Large 1};


    \node[qubitupperquater,below=of qubitplus 1] (qubitupperquater 1)
    {};

    \node[sheetstyle,right=of qubitupperquater 1] (result 2 1) {\Large
      0};

    \node[sheetstyle,right=of result 2 1] (result 2 2) {\Large 1};

    \node[sheetstyle,right=of result 2 2] (result 2 3) {\Large 1};

    \node[sheetstyle,right=of result 2 3] (result 2 4) {\Large 1};


    \node[qubitlowerquater,below=of qubitupperquater 1]
    (qubitlowerquater 1) {};

    \node[sheetstyle,right=of qubitlowerquater 1] (result 3 1) {\Large
      0};

    \node[sheetstyle,right=of result 3 1] (result 3 2) {\Large 0};

    \node[sheetstyle,right=of result 3 2] (result 3 3) {\Large 0};

    \node[sheetstyle,right=of result 3 3] (result 3 4) {\Large 1};

  \end{tikzpicture}

\end{frame}
% ##################





% ##################
\begin{frame}
  \frametitle{Kwantowa kartka papieru}


  \begin{tikzpicture}

    \node[sheetstyle] (zero) {\Large 0};

    \node[right=of zero] (ket 1) {\Large $| 0 \rangle =
      \begin{bmatrix}
        1 \\
        0
      \end{bmatrix}$};


    \node[sheetstyle,right=1cm of ket 1] (one) {\Large 1};

    \node[right=of one] (ket 2) {\Large $| 1 \rangle =
      \begin{bmatrix}
        0 \\
        1
      \end{bmatrix}$};


    \node[qubitplus,below=of zero] (qubitplus) {};

    \node[right=of qubitplus] (ket 2) {\Large
      $| + \rangle = \frac{ 1 }{ \sqrt{ 2 } } | 0 \rangle + \frac{ 1 }{ \sqrt{ 2 }
      } | 1 \rangle = \frac{ 1 }{ \sqrt{ 2 } }
      \begin{bmatrix}
        1 \\
        1
      \end{bmatrix}$};


    \node[qubitlowerquater,below=of qubitplus] (lowerquater) {};

    \node[right=of lowerquater] (ket 3) {\Large
      $| \psi \rangle = \frac{ \sqrt{ 3 } }{ 2 } | 0 \rangle + \frac{ 1 }{ 2 } | 1 \rangle
      = \frac{ 1 }{ \sqrt{ 2 } }
      \begin{bmatrix}
        \sqrt{3} \\
        1
      \end{bmatrix}$};



    % \node[sheetstyle,right=of result 1 1] (result 1 2) {\Large 0};

    % \node[sheetstyle,right=of result 1 2] (result 1 3) {\Large 1};

    % \node[sheetstyle,right=of result 1 3] (result 1 4) {\Large 1};


    % \node[qubitupperquater,below=of qubitplus 1] (qubitupperquater
    % 1) {};

    % \node[sheetstyle,right=of qubitupperquater 1] (result 2 1)
    % {\Large 0};

    % \node[sheetstyle,right=of result 2 1] (result 2 2) {\Large 1};

    % \node[sheetstyle,right=of result 2 2] (result 2 3) {\Large 1};

    % \node[sheetstyle,right=of result 2 3] (result 2 4) {\Large 1};


    % \node[qubitlowerquater,below=of qubitupperquater 1]
    % (qubitlowerquater 1) {};

    % \node[sheetstyle,right=of qubitlowerquater 1] (result 3 1)
    % {\Large 0};

    % \node[sheetstyle,right=of result 3 1] (result 3 2) {\Large 0};

    % \node[sheetstyle,right=of result 3 2] (result 3 3) {\Large 0};

    % \node[sheetstyle,right=of result 3 3] (result 3 4) {\Large 1};

  \end{tikzpicture}

\end{frame}
% ##################




% ##################
\begin{frame}
  \frametitle{Prawdopodobieństwo}


  Prawdopodobieństwo, że $| + \rangle$ skolapsuje do $| 0 \rangle$ jest równe
  kwadratowi modułu współczynnika przy $| 0 \rangle$.
  \begin{align*}
    &| + \rangle =
      \frac{ 1 }{ \sqrt{ 2 } } | 0 \rangle + \frac{ 1 }{ \sqrt{ 2 } } | 1 \rangle \\
    &P_{ | + \rangle }( | 0 \rangle ) = | \frac{ 1 }{ \sqrt{ 2 } } |^{ 2 } =
      \frac{ 1 }{ 2 }
  \end{align*}

  Aby to miało sens jako prawdopodobieństwa suma kwadratów modułów
  współczynników musi wynosić 1.
  \begin{align*}
    P_{ | + \rangle }( | 0 \rangle ) + P_{ | + \rangle }( | 1 \rangle ) =
    | \frac{ 1 }{ \sqrt{ 2 } } |^{ 2 } + | \frac{ 1 }{ \sqrt{ 2 } } |^{ 2 }
    = \frac{ 1 }{ 2 } + \frac{ 1 }{ 2 }
  \end{align*}

\end{frame}
% ##################





% ##################
\begin{frame}
  \frametitle{Bramki kwantowe}


  \begin{align*}
    &H =
    \frac{ 1 }{ \sqrt{ 2 } }
    \begin{bmatrix}
      \hphantom{-} 1 & \hphantom{-} 1 \\
      \hphantom{-} 1 & -1
    \end{bmatrix} \\
    &H | 0 \rangle =
      \frac{ 1 }{ \sqrt{ 2 } }
    \begin{bmatrix}
      \hphantom{-} 1 & \hphantom{-} 1 \\
      \hphantom{-} 1 & -1
    \end{bmatrix}
                       \begin{bmatrix}
                         1 \\
                         0
                       \end{bmatrix}
    =
    \frac{ 1 }{ \sqrt{ 2 } }
    \begin{bmatrix}
      1 \\
      1
    \end{bmatrix}
    =
    \frac{ 1 }{ \sqrt{ 2 } } | 0 \rangle + \frac{ 1 }{ \sqrt{ 2 } } | 1 \rangle
    = | + \rangle \\
        &H | 0 \rangle
      =
      \frac{ 1 }{ \sqrt{ 2 } }
    \begin{bmatrix}
      \hphantom{-} 1 & \hphantom{-} 1 \\
      \hphantom{-} 1 & -1
    \end{bmatrix}
                       \begin{bmatrix}
                         0 \\
                         1
                       \end{bmatrix}
    =
    \frac{ 1 }{ \sqrt{ 2 } }
    \begin{bmatrix}
      1 \\
      -1
    \end{bmatrix}
    =
    \frac{ 1 }{ \sqrt{ 2 } } | 0 \rangle - \frac{ 1 }{ \sqrt{ 2 } } | 1 \rangle
    = | - \rangle \\
    &X
      =
      \begin{bmatrix}
        0 & 1 \\
        1 & 0
      \end{bmatrix} \\
    &X | 0 \rangle = | 1 \rangle \quad X | 1 \rangle = | 0 \rangle
  \end{align*}


\end{frame}
% ##################





% ##################
\begin{frame}
  \frametitle{Iloczyn skalarny}


  \begin{align*}
    &z = a + ib \quad \bar{z} = a - ib, \quad i^{ 2 } = -1 \\
    &| \psi \rangle
      =
      \begin{bmatrix}
        \psi_{ 0 } \\
        \psi_{ 1 }
      \end{bmatrix}
    \quad
    | \varphi \rangle
    =
    \begin{bmatrix}
      \varphi_{ 0 } \\
      \varphi_{ 1 }
    \end{bmatrix} \\
    &( \psi, \varphi ) = \langle \psi | \varphi \rangle
      =
      \overline{ \psi_{ 0 } } \varphi_{ 0 } + \overline{ \psi_{ 1 } } \varphi_{ 1 }
  \end{align*}

  Prawdopodobieństwo, że zobaczymy wektor $| 0 \rangle$ pokolapsujący z wektora
  $| \psi \rangle$ wynosi
  \begin{align*}
    P_{ | \psi \rangle }( | 0 \rangle ) = | \langle 0 | \psi \rangle |^{ 2 } = | \langle \psi | 0 \rangle |^{ 2 }.
  \end{align*}
  Warunek, że suma wszystkich prawdopodobieństw wynosi 1 sprowadza się do
  tego, że
  \begin{align*}
    | \langle \psi | \psi \rangle |^{ 2 } = 1
  \end{align*}

\end{frame}
% ##################





% ##################
\begin{frame}
  \frametitle{Ogólnie}


  Stan układu kwantowego jest opisywany przez wektor z zespolonej
  przestrzeni Hilberta $\mathcal{H}$. W wypadku qubitu tą przestrzenią jest
  $\mathbb{C}^{ 2 }$. Wektor na taki nakładamy dwa ograniczenia.
  \begin{align*}
    &| \langle \psi | \psi \rangle |^{ 2 } = 1 \\
    &| \psi \rangle \equiv e^{ i \phi } | \psi \rangle, \quad \psi \in \mathbb{R}
  \end{align*}
  Znak $\equiv$ oznacza, że z punktu widzenia fizyki, wektory po obu stronach
  równości, są identyczne.

  Qubit
  \begin{align*}
    \begin{bmatrix}
      z \\
      w
    \end{bmatrix}
    \to
    \begin{bmatrix}
      | z | \\
      | w | e^{ i \phi }
    \end{bmatrix}
    \to
    \begin{bmatrix}
      \cos( \frac{ \theta }{ 2 } ) \\
      \sin( \frac{ \theta }{ 2 } ) e^{ i \phi }
    \end{bmatrix}
  \end{align*}
  Ostatnio przejście zachodzi dlatego, że $| z |^{ 2 } + | w |^{ 2 } = 1$.

\end{frame}
% ##################





% ######################################
\section{Zaczynamy podróż}
% ######################################





% ##################
\begin{frame}
  \frametitle{Co to jest komputery kwantowy?}


  Można powiedzieć, że~komputer kwantowy to komputer, który mam
  procesor nowego typu, który jest 1 000 razy szybszy niż te najlepsze
  standardowe procesory. My jednak potrzebujemy wiedzieć więcej
  i~dlatego musimy się bardziej zagłębić w~znaczenie słowa
  ,,kwantowy''.

  Z~codziennego doświadczenia wiemy jak zachowują się otaczające nas
  przedmioty, które są~zbudowane z~bardzo dużej ilości atomów. Przez
  dla nas codziennie ma znaczenie tylko zachowanie~się dużych skupisk
  atomów, nie mamy jednak żadnej intuicji jak zachowuje się jeden
  atom. Dzięki różnym narzędziom zrozumieliśmy, że~wówczas świat
  wygląda zupełnie inaczej, można wręcz rzecz, iż dziwacznie.

  \alert{Uwaga.} Wiele rzeczy będę mocno upraszczał.

\end{frame}
% ##################





% ##################
\begin{frame}
  \frametitle{Co to jest komputery kwantowy?}


  Przedmioty jakie widzimy gołym okiem mają zwykle $10^{ 23 }$ (mierzy
  to liczba Avogadra:
  $N_{ A } = 6.02\ldots \cdot 10^{ 23 } \si{.mol}^{ -1 }$. Jaka jest więc
  różnica między jednym atomem, a~zbiorowiskiem $10^{ 23 }$ atomów?

  Nie jest specjalnie trudne podniesienie przedmiotu o~masie
  $60 \si{.kg}$. Przedmiotu o~masie $10^{ 23 }$ większej nie możemy
  podnieść, bo nie ma jak. Jest to bowiem masa Ziemi, wszystkiego
  co~się na niej znajduje i~całej atmosfery (około
  $5.97 \cdot 10^{ 24 } \si{.kg}$). Nienależny się więc dziwić, że~są
  różnice między zachowaniem jednego atomu, a~przedmiotów wokół nas.
  Można, a~nawet należy~się dziwić, jak duże te różnice są.

  Mechanika Newtona wyjaśnię i~opisuje jak poruszają się samochody,
  długopisy, krzesła, etc. Analogiczna teoria dla małej ilości atomów
  nosi nazwę mechaniki kwantowej i~dzisiaj nie jest możliwe pracowanie
  z~komputerem kwantowym bez pewnego jej zrozumienia.

\end{frame}
% ##################





% ##################
\begin{frame}
  \frametitle{Co to jest komputery kwantowy?}


  Komputer kwantowy to komputer którego procesor zbudowany jest w~taki
  sposób, by~korzystał z~tych nieobserwowalnych przez nas normalnie
  własności materii opisywane przez mechanikę kwantową. Co pozwoli mu
  wykonywać obliczenia znacznie szybciej niż tym powszechnie dziś
  używanym.

  Jednak oznacza to, że musimy nauczyć się pracować z tymi bardzo
  dziwacznymi własnościami materii.

\end{frame}
% ##################





% ##################
\begin{frame}
  \frametitle{Trochę perspektywy}


  Przejście od magnetycznych dysków twardych o~pojemności mierzone
  w~MB do tej w GB było związane z~budowaniem ich z~wykorzystaniem
  kwantowego zjawiska gigantycznego magnetooporu. IBM wypuścił dyski
  komercyjne dyski korzystające z~niego w 1997 roku, zaś Albert Fert i
  Peter Gr\"{u}nberg otrzymali za jego odkrycie Nagrodę Nobla w~2007
  roku.

  Można mieć nadzieję, że~kwantowe procesory osiągną dla szybkości
  komputera coś podobnego do tego, co kwantowe dyski twarde zrobiły
  dla jego pojemności.

  Czemu więc nie przyjęła się nazwa kwantowy dysk twardy? Podejrzewam,
  że było to związane z marketingiem.

\end{frame}
% ##################





% ##################
\begin{frame}
  \frametitle{Czego należy się spodziewać po komputerach kwantowych?}


  Jeśli mamy poprawnie działający normalny komputer, i każemy mu 10
  razy wykonać działanie $1 + 1$ to otrzymamy
  \begin{align*}
    2, 2, 2, 2, 2, 2, 2, 2, 2, 2.
  \end{align*}

  Jeśli mamy poprawnie działający komputer kwantowy, to otrzymamy
  jeden z~możliwych rezultatów:
  \begin{align*}
    &2, 1, 2, 2, 2, 3, 2, 2, 2, 2, \\
    &2, 2, 2, 2, 1, 2, 2, 2, 3, 2, \\
    &2, 2, 2, 2, 2, 2, 2, 2, 2, 2, \\
    &1, 1, 2, 2, 2, 2, 2, 2, 2, 2, \\
    &0, 2, 1, 3, 2, 2, 2, 2, 2, 3, \\
    &\vdots
  \end{align*}

\end{frame}
% ##################





% ##################
\begin{frame}
  \frametitle{Kwantowe programowanie i obliczenia}


  \begin{align*}
    &2, 1, 2, 2, 2, 3, 2, 2, 2, 2, \\
    &2, 2, 2, 2, 1, 2, 2, 2, 3, 2, \\
    &2, 2, 2, 2, 2, 2, 2, 2, 2, 2, \\
    &1, 1, 2, 2, 2, 2, 2, 2, 2, 2, \\
    &\vdots
  \end{align*}
  Błędne wyniki \alert{nie} są winną błędów działania komputera
  kwantowego. Według naszej obecnej wiedzy, jeśli chcemy korzystać ze
  zjawisk kwantowych musimy zaakceptować to, że~część wyników jakie
  otrzymamy będzie błędna.

  Z drugiej strony widzimy, że najczęściej pojawia~się wynik
  $1 + 1 = 2$. Fizyka zjawisk kwantowych mówi, że~w~takim razie to $2$
  należy uznać za poprawny wynik i~w~tym przypadku jest to prawda.

\end{frame}
% ##################





% ##################
\begin{frame}
  \frametitle{Potrzebna w~dalszym ciągu wiedza}


  Wszędzie w~dalszym ciągu będziemy się odwoływać do następujących
  koncepcji matematycznych:
  \begin{itemize}
  \item rachunek prawdopodobieństwa,

  \item logika 0 i~1 (prawdy i~fałszu).

\end{itemize}
Jaka jest ich znajomość wśród was?

Potem pewnie będziemy musieli pomówić o~macierzach i~liczbach zespolonych.

\end{frame}
% ##################










% ######################################
\section{Bity, qubity i~procesory}
% ######################################



% ##################
\begin{frame}
  \frametitle{Bit}


  Bit jest wielkością która może przyjmować wartość 0 lub 1. Fizycznie
  bitem może być kartka papieru na której napiszemy cyfrę 0 lub 1, lub
  pewien układ elektroniczny w naszym komputerze.

  Historycznie ustalono, że komputery będą pracować nie w oparciu o
  bity ale o bajt, które są ciągiem $8 = 2^{ 3 }$ bitów. Np.
  $[ 0, 0, 0, 0, 0, 0, 0, 0 ]$, $[ 0, 0, 0, 0, 0, 0, 0, 1 ]$, etc.
  Ustawianie bitów w~ośmioelementowe bajty jest podyktowany niczym
  innym jak wygodą, istniały komputery których bajty miały długość 13
  bitów.

  Dzisiaj używamy operacji przerabiający mega bajty i giga bajty
  (oglądanie YouTubea) informacji, zupełnie nie myśląc o~tym jak
  komputer zarządza bajtami, co jest w RAMie, co na dysku twardym, co
  przetwarza procesor. I bardzo dobrze, że nie musimy o~tym myśleć.

  Jednak nasz poziom technologiczny wymusza na nas byśmy się ,,cofnęli
  w~rozwoju'' i~operowali na pojedynczych kwantowych bitach (o~nich
  później).

\end{frame}
% ##################





% ##################
\begin{frame}
  \frametitle{Bity i~procesor}


  Procesor pobiera dane w~postaci odpowiedniej ilości bajtów,
  następnie przetwarza je wykonując operacje na \alert{pojedynczych}
  bitach każdego bajtu, dając w wyniku odpowiednią ilość bajtów.

  Część procesora wykonująca ustaloną operację na bitach nazywamy
  \alert{bramką logiczną}.

  Okazuje, co jest dość niezwykłe, że aby zbudować procesor wykonujący
  dowolną operację na bajtach wystarczy byśmy umieli zbudować bramki
  logiczne: {????}.

\end{frame}
% ##################





% ######################################
\section{Fizyka, bity i~procesory}
% ######################################



% ##################
\begin{frame}
  \frametitle{,,Stany materii''}


  Każdy zna trzy stany skupienia wody: lód, ciesz i parę wodną.
  Podobnie można powiedzieć, że~otaczająca nas materia może być
  w~trzech ,,stanach skupienia''.

  \begin{itemize}
  \item Stan kwantowy. Będziemy się nim dużo zajmowali.

  \item Stan klasyczny. W~nim jest większość przedmiotów wokół nas.

  \item Stan relatywistyczny. Kiedy prędkość obiektu jest zbliżona do
    prędkości światła: $c = 3 00 000 \, \frac{ \si{km} }{ \si{s} }$.

  \end{itemize}

  Tak jak z~wodą dany ,,stan skupienia'' jest nie tyle własnością
  samej materii, co warunków w~których jest. Woda jest tą samą rzeczą
  w~-20\textcelsius{ } jaki i~w~200\textcelsius, choć wygląda
  i~zachowuje~się zupełnie inaczej.

  Tak samo otaczająca nas materia może przechodzić między tymi
  stanami, pozostając cały czas sobą, lecz zmieniając swoje własności.

\end{frame}
% ##################





% ##################
\begin{frame}
  \frametitle{Klasyczne i~kwantowe bity}


  Materiał z~której zbudowany jest normalny procesor znajduje~się
  w~klasycznym ,,stanie skupienia'', przez analogię będziemy więc
  mówili o~klasycznych komputerach, klasycznych bitach, etc.

  Materiał z~którego zbudowany jest kwantowy procesor znajduje~się
  w~kwantowym ,,stanie skupienia'', stąd nazwa.

  Kwantowy bit nazywamy \alert{qubitem} (ang. \emph{qubit} od
  \emph{QUantum BIT}), spotyka się też zapis ,,kubit''. Ze względów
  językowych próbowano wprowadzić ,,kwbit'', od ,,KWantowy BIT'', ale
  ta forma jest rzadko używana.

  Klasyczny bit może być tylko w~stanach reprezentujących jedną
  z~dwóch~wartości: 0, 1. By go reprezentować wystarczy moneta.

  Qubity są pod tym względem zupełnie inne. Oprócz 0 i 1, mogą być
  w~nieskończenie wielu innych stanach, które istnienie ujawnia się
  dopiero w~kwantowym stanie materii.

\end{frame}
% ##################





% ##################
\begin{frame}
  \frametitle{Hardware kwantowy}


  W~dzisiejszych komputerach kwantowych jeśli temperatura qubitu
  będzie większa niż -273\textcelsius{ } (dokładniej, 15 mK), to
  materiał z~którego jest wykonany przejdzie ze stanu kwantowego do
  stanu klasycznego, przez co stanie się on bezużyteczny. Trzeba wtedy
  uruchomić odpowiednią procedurę chłodzenia i~odbudowywania
  kwantowego ,,stanu skupienia'' materiału. Temperatura nie jest
  niestety jedną rzeczą, która niszczy kwantową naturę qubitu.

  Zjawisko to w fizyce nosi nazwę dekoherencji, ja będą po prostu
  mówił, że qubit ulega zniszczeniu i~musi zostać odbudowany.

  Jeśli w~normalnym komputerze zapiszemy zdjęcie w~pamięci komputera,
  to możemy tworzące je bit odczytywać tyle razy ile chcemy.

  Obecnie informację z qubitu możemy odczytywać tylko niewielką liczbę
  razy, powiedzmy dla prostoty że 10. Wraz z~rozwoje technologii
  liczba ta będzie rosła.

\end{frame}
% ##################





% % ##################
% \begin{frame}
%   \frametitle{Nierelatywistyczna mechanika kwantowa}


%   Teorią fizyczną na której bazują komputery kwantowe i~o~której
%   musimy coś powiedzieć jest nierelatywistyczna mechanika kwantowa.
%   Traktuje ono światło jako poruszające~się z~nieskończoną prędkością
%   (tak naprawdę jest to bardziej skomplikowane), co nie jest
%   jak wiemy prawdą. Teoria ta jest tym samym sprzeczna z~szczególną
%   teorią względności (Poincar\'{e}, Lorent, Einstein).

%   Istnieje sformułowanie mechaniki kwantowej, które jest zgodne
%   z~szczególną teorią względności. Niestety rozumiemy tą teorię zbyt
%   słabo, by móc za jej pomocą opisać komputer kwantowy.

%   Na szczęście nierelatywistyczna mechanika kwantowa wystarczy z
%   nawiązką do potrzeb ludzkości w tej kwestii.

% \end{frame}
% % ##################





% ##################
\begin{frame}
  \frametitle{Mechanika kwantowa i~jej interpretacje}


  Mechanika kwantowa to teoria fizyczna na której opieramy nasze komputery kwantowe. Jest już dość starą teorią, za datę jej narodziny można uznać
  1925~r. Od tego czasu setki eksperymentów i~wynalazków (dyski
  twarde) potwierdziły, że~bardzo dobrze wyjaśnia i~opisuje świat.
  Pomimo tego wciąż na wiele pytań które ona niesie musimy
  odpowiadać ,,Nie wiem''.

  Przyczyną ten jest wiele. Do najważniejszych należy to, że
  eksperymenty które pozwoliłby nam rozwiązać pewne kwestie wciąż są
  zbyt trudne do wykonania. W~istocie są poważne osoby, które mówią,
  że~rozwój technologii napędzany przez prace nad komputerami
  kwantowymi oraz badania ich działania pozwoli nam wreszcie
  na wiele z~tych pytań odpowiedzieć.

  Niestety, wiele z~tych otwartych pytań dotyczy tego czym będziemy się zajmować. Musimy się więc pogodzić z~tym, że~pewnych rzecz nie rozumiemy.

\end{frame}
% ##################




% ##################
\begin{frame}
  \frametitle{Mechanika kwantowa i~jej interpretacje}


  Podano wiele możliwym rozwiązań tych problemów, po prostu nie umiemy stwierdzić które z~nich są błędne, a~które idą w~dobrą stronę. W~literaturze funkcjonują one pod nazwą ,,interpretacji mechaniki kwantowej''. Do najważniejszych z~nich należą:
  \begin{itemize}
  \item interpretacja kopenhaska,

    \item teoria fali pilotującej De Broglie-Bohma,

  \item teorie kwantowej wiedzy (informacji),

  \item kwantowy bayesianism,

  \item interpretacja statystyczna (stochastyczna),

  \item logika kwantowa,

  \item teoria wieloświatów,

  \item kwantowy darwinizm.

  \end{itemize}

\end{frame}
% ##################





% ######################################
\section{Kwantowa kartka papieru}
% ######################################





% ##################
\begin{frame}
  \frametitle{Qubity są dziwne}


  Chyba każdy kto nie studiował mechaniki kwantowej i dla większości którzy to robili, qubity zachowują się po prostu dziwnie. Złośliwi mówią, że jeśli uważasz, że rozumiesz co tu się dzieje, to nie wiesz o czym mówisz.

  Dlatego teraz zajmiemy się prostym modelem bitu i qubitu w~postaci kartki papieru z 0 na jednej i 1 na drugiej stronie, którą nasz kolega kładzie na stole.

\end{frame}
% ##################





% ##################
\begin{frame}
  \frametitle{Klasyczna kartka papieru jako klasyczny bit}


  Nasz kolega Marian położył kartkę na stole. Podchodzimy do stołu i odczytujemy z kartki liczbę 0 albo 1. Wszystko jest tak proste, że nie trzeba nic tłumaczyć.
  % Nasz kolega chce nam przekazać informacje, że pewna liczba ma wartość 0.
  % Za pomocą tej kartki papieru chcemy zapisać informacje, że pewna liczba ma wartość 0.
  % Kładziemy kartkę na stole tak, by strona z 0 była na wierzchu. Podchodzimy do stołu i~spoglądamy  Jeśli potrzebujemy sobie przypomnieć jaka jest wartość tej liczby, to patrzymy na tą kartkę i odczytujemy z~niej liczbę 0.
  % Na kartkę możemy się popatrzeć dowolną ilość razy i kartka się nie zużywa.


  \begin{tikzpicture}[auto]
    \node[sheetstyle] (zero) {\LARGE 0};

    \node[sheetstyle,right=of zero] (one) {\LARGE 1};
  \end{tikzpicture}

\end{frame}
% ##################





% ##################
\begin{frame}
  \frametitle{Kwantowa kartka papieru}


  Marian sobie tylko wiadomymi metodami zdobył kartkę papieru w stanie kwantowym i położył ją na stole. Kiedy pochodzimy do stołu i próbujemy odczytać co jest na kartce widzimy jedną z poniższych możliwości.


    \begin{tikzpicture}[node distance=0.5cm,auto]
      \node[sheetstyle] (zero) {\LARGE 0};

      \node[qubitlowerquater,right=of zero] (lowerquater) {};

      \node[qubitplus,right=of lowerquater] (qubitplus) {};

      \node[qubitupperquater,right=of qubitplus] (upperquater) {};

    \node[sheetstyle,right=of upperquater] (one) {\LARGE 1};
  \end{tikzpicture}

\end{frame}
% ##################





% ##################
\begin{frame}
  \frametitle{Kwantowa kartka papieru}


  \begin{tikzpicture}
    \node[qubitplus] (qubitplus 1) {};

    \node[sheetstyle,right=of qubitplus 1] (zero 1) {\Large 0};

    \node[qubitplus,below=of qubitplus 1] (qubitplus 2) {};

    \node[sheetstyle,right=of qubitplus 2] (zero 2) {\Large 1};
  \end{tikzpicture}

\end{frame}
% ##################









% ##################
\begin{frame}
  \frametitle{Kwantowa kartka papieru}


  \begin{tikzpicture}
    \node[qubitplus] (qubitplus 1) {};

    \node[sheetstyle,right=of qubitplus 1] (result 1 1) {\Large 0};

    \node[sheetstyle,right=of result 1 1] (result 1 2) {\Large 0};

    \node[sheetstyle,right=of result 1 2] (result 1 3) {\Large 1};

    \node[sheetstyle,right=of result 1 3] (result 1 4) {\Large 1};


    \node[qubitupperquater,below=of qubitplus 1] (qubitupperquater 1) {};

    \node[sheetstyle,right=of qubitupperquater 1] (result 2 1) {\Large 0};

    \node[sheetstyle,right=of result 2 1] (result 2 2) {\Large 1};

    \node[sheetstyle,right=of result 2 2] (result 2 3) {\Large 1};

    \node[sheetstyle,right=of result 2 3] (result 2 4) {\Large 1};


    \node[qubitlowerquater,below=of qubitupperquater 1]
    (qubitlowerquater 1) {};

    \node[sheetstyle,right=of qubitlowerquater 1] (result 3 1) {\Large 0};

    \node[sheetstyle,right=of result 3 1] (result 3 2) {\Large 0};

    \node[sheetstyle,right=of result 3 2] (result 3 3) {\Large 0};

    \node[sheetstyle,right=of result 3 3] (result 3 4) {\Large 1};
  \end{tikzpicture}

\end{frame}
% ##################





% ##################
\begin{frame}
  \frametitle{Kwantowa kartka papieru}


  \begin{tikzpicture}
  \node[sheetstyle] (zero) {\Large 0};

  \node[right=of zero] (ket 1) {\Large $| 0 \rangle =
    \begin{bmatrix}
      1 \\
      0
    \end{bmatrix}$};


  \node[sheetstyle,right=1cm of ket 1] (one) {\Large 1};

  \node[right=of one] (ket 2) {\Large $| 1 \rangle =
    \begin{bmatrix}
      0 \\
      1
    \end{bmatrix}$};


  \node[qubitplus,below=of zero] (qubitplus) {};

  \node[right=of qubitplus] (ket 2) {\Large $| + \rangle
    = \frac{ 1 }{ \sqrt{ 2 } } | 0 \rangle + \frac{ 1 }{ \sqrt{ 2 } } | 1 \rangle
    = \frac{ 1 }{ \sqrt{ 2 } }
    \begin{bmatrix}
      1 \\
      1
    \end{bmatrix}$};


  \node[qubitlowerquater,below=of qubitplus] (lowerquater) {};

  \node[right=of lowerquater] (ket 3) {\Large $| \psi \rangle
    =
    \frac{ \sqrt{ 3 } }{ 2 } | 0 \rangle + \frac{ 1 }{ 2 } | 1 \rangle
    = \frac{ 1 }{ \sqrt{ 2 } }
    \begin{bmatrix}
      \sqrt{3} \\
      1
    \end{bmatrix}$};



    % \node[sheetstyle,right=of result 1 1] (result 1 2) {\Large 0};

    % \node[sheetstyle,right=of result 1 2] (result 1 3) {\Large 1};

    % \node[sheetstyle,right=of result 1 3] (result 1 4) {\Large 1};


    % \node[qubitupperquater,below=of qubitplus 1] (qubitupperquater 1) {};

    % \node[sheetstyle,right=of qubitupperquater 1] (result 2 1) {\Large 0};

    % \node[sheetstyle,right=of result 2 1] (result 2 2) {\Large 1};

    % \node[sheetstyle,right=of result 2 2] (result 2 3) {\Large 1};

    % \node[sheetstyle,right=of result 2 3] (result 2 4) {\Large 1};


    % \node[qubitlowerquater,below=of qubitupperquater 1]
    % (qubitlowerquater 1) {};

    % \node[sheetstyle,right=of qubitlowerquater 1] (result 3 1) {\Large 0};

    % \node[sheetstyle,right=of result 3 1] (result 3 2) {\Large 0};

    % \node[sheetstyle,right=of result 3 2] (result 3 3) {\Large 0};

    % \node[sheetstyle,right=of result 3 3] (result 3 4) {\Large 1};
  \end{tikzpicture}

\end{frame}
% ##################










% ######################################
\appendix
% ######################################





% ##################
\EndingSlide{Pytania? Dziękuję za uwagę.}
% ##################










% ####################################################################
% ####################################################################
% Bibliografia
\bibliographystyle{alpha}

\bibliography{Bibliography}{}





% ############################

% Koniec dokumentu
\end{document}
