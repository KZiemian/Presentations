% Autor: Kamil Ziemian

% --------------------------------------------------------------------
% Podstawowe ustawienia Beamera i używane pakiety
% --------------------------------------------------------------------
\RequirePackage[l2tabu, orthodox]{nag} % Wykrywa przestarzałe i niewłaściwe
% sposoby używania LaTeXa. Więcej jest w l2tabu English version.

\documentclass{beamer}  % Klasa dokumentu
\mode<presentation>  % Rodzaj tworzonych slajdów Beamera
\usetheme{Warsaw}  % Temat graficzny

\setbeamertemplate{headline}{}  % Usuwa nagłówek
\setbeamersize{text margin left=3mm}  % Wielkość lewego marginesu
\setbeamersize{text margin right=3mm}  % Wielkość prawego marginesu
\setbeamertemplate{navigation symbols}{}  % Usuwa ikony nawigacji w prawym
% dolnym rogu



\usepackage[polish]{babel}  % Tłumaczy na polski teksty automatyczne LaTeXa
% i pomaga z typografią.
\usepackage[MeX]{polski}  % Polonizacja LaTeXa, bez niej będzie pracował
% w języku angielskim.
\usepackage[utf8]{inputenc}  % Włączenie kodowania UTF-8, co daje dostęp
% do polskich znaków.
\usepackage{lmodern}  % Wprowadza fonty Latin Modern.
\usepackage[T1]{fontenc}  % Potrzebne do używania fontów Latin Modern.



% ----------------------------
% Pakiety napisane przez użytkownika.
% Mają być w tym samym katalogu to ten plik .tex
% ----------------------------
\usepackage{latexshortcuts}



% ----------------------------
% Pakiet "hyperref"
% Polecano by umieszczać go na końcu preambuły.
% ----------------------------
\usepackage{hyperref}  % Pozwala tworzyć hiperlinki i zamienia odwołania
% do bibliografii na hiperlinki.





% --------------------------------------------------------------------
\title[]{Star Wars, mythology \\
  and~psychoanalysis with a~thousand faces}
\author{Kamil Ziemian} 

% \\ \texttt{ziemniakzkosmosu@gmail.com}}

% \institute{II rok, fizyka teoretyczna, studia magisterskie.}

\date{12 May 2014}
% --------------------------------------------------------------------





% ####################################################################
% Początek dokumentu
\begin{document}
% ####################################################################

\maketitle


% ##########
\begin{frame}
  \frametitle{Introduction}

  % \begin{block}{}
  %   Connections of Star Wars and mythology are generally known to
  %   fans of this Universe.
  % \end{block}

  \begin{block}{I claim that is a connection between}
    \begin{figure}
      \centering

      \includegraphics[height=1.7in, width=1.4in]{Star-Wars-1.jpg}
      % \uncover<2>{
      \includegraphics[height=1.7in, width=1.2in]{Sigmund-Freud.jpg}
      \caption{\textbf{Star Wars}, Sigmund Freud (1856--1939)}
      % }
    \end{figure}
  \end{block}

\end{frame}
% ##########



% ##########
\begin{frame}
  \frametitle{Introduction}

  \begin{block}{And this connection is\ldots}
    \begin{figure}
      \centering

      \includegraphics[height=1.5in, width=2.3in]{Joseph-Campbell.jpg}
      \caption{Joseph Campbell (1904--1987)}
      American mythologist and psychoanalyst.
    \end{figure}
  \end{block}

\end{frame}
% ##########



% ##########
\begin{frame}
  \frametitle{Introduction}

  \begin{block}{Campbell's magnum opus}
    \begin{figure}
      \centering

      \includegraphics[height=1.7in,
      width=1.4in]{Hero-with-a-Thousand-Faces.jpg}
      % \uncover<2>{
      \includegraphics[height=1.7in,
      width=1.2in]{Bohater-o-Tysiacu-Twarzy.jpg}

      \caption{\textbf{The Hero with a Thousand Faces}}
      % }
      First published in United States in 1949 (picture on the
      left). \\
      On the right polish editon from 1997.
    \end{figure}
  \end{block}

\end{frame}
% ##########



% ##########
\begin{frame}
  \frametitle{Introduction}

  \begin{block}{But we need one more man in our story}
    \begin{figure}
      \centering

      \includegraphics[height=1.7in, width=1.4in]{CGJ}
      % \uncover<2>{
      \includegraphics[height=1.7in, width=1.4in]{CGJ1}
      \caption{Carl Gustav Jung (1875--1961)}
      % }
      The most important follower of Freud. Author of concepts:
      collective unconscious and archetypes.
    \end{figure}
  \end{block}

\end{frame}
% ##########



% ##########
\begin{frame}
  \frametitle{The adventure of Campbell begins\ldots}

  \begin{block}{The first words of the book}
    "The truths contained in religious doctrines are after all
    so~distorted and systematically disguised," writes Sigmund Freud,
    "that the mass of humanity cannot recognize them as
    truth." %The~case is similar to what happens when we tell a child that new-born babies are brought by the stork."
    (Sigmund Freud~\emph{The future of an illusion}, trans. by James
    Strachey) Page 9.
  \end{block}

  \begin{block}{Remark}
    % \begin{itemize}
    % \item
    Numbers of pages refers to polish edition [2].
    % \uncover<4>{
    % \item Strok is:
    %   \begin{figure}
    %     \centering \includegraphics[height=0.8in,
    %     width=1.0in]{Stroke}
    %   \end{figure}}
    % \end{itemize}
  \end{block}

\end{frame}
% ##########



% ##########
\begin{frame}
  \frametitle{The adventure of Campbell begins\ldots}

  \begin{block}{He continues}
    "It is the purpose of the present book to uncover some of the
    truths disguised for us under the figures of religion and
    mythology by~bringing together a multitude of~not-too-difficult
    examples and~letting the ancient meaning become apparent of
    itself." Page 9.
  \end{block}
  \pause

  \begin{block}{}
    "The old teachers knew what they were saying. Once we have learned
    to read again their symbolic language, it~requires no more than
    the talent of an anthologist to let their teaching be heard.
    But~first we must learn the grammar of the symbols, and~as a~key
    to this mystery I know of no better modern tool than
    psychoanalysis. Without regarding this as the last word on the
    subject, one can nevertheless permit it to serve as~an~approach."
    Page 9.
  \end{block}

\end{frame}
% ##########



% ##########
\begin{frame}
  \frametitle{The Monomyth}

  \begin{block}{Prolouge: The Monomyth}
    Name of this part of the book was taken from J. Joyce "Finnegan's
    wake", on which Campbell previously worked. It consist of four
    parts:
    \begin{itemize}
    \item[--] Myth and Dream,
    \item[--] Tragedy and Comedy,
    \item[--] The Hero and the God,
    \item[--] The World Navel.
    \end{itemize}
  \end{block}

\end{frame}
% ##########



% ##########
\begin{frame}
  \frametitle{The Monomyth}

  \begin{block}{}
    Campbell starting from saying, in slightly mischievous way, that
    whatever we:
    \begin{itemize}
    \item[--] hearing "mumbo jumbo of some red-eyed witch doctor
      of~the~Congo",
    \item[--] read sonnets of the mystic Lao-tse,
    \item[--] work on argument of Aquinas,
    \item[--] discover meaning of "a bizarre Eskimo fair' tale".
    \end{itemize}
    "it will be always the one, shape-shifting yet marvelously
    constant story that we find, together with a challengingly
    persistent suggestion of more remaining to be experienced than
    will ever be known or told." Page 19.
  \end{block}
  \pause

  \begin{block}{}
    "It would not be too much to say that myth is the secret opening
    through which the inexhaustible energies of the cosmos pour into
    human cultural manifestation." Page 19.
  \end{block}

\end{frame}
% ##########



% ##########
\begin{frame}
  \frametitle{The Monomyth}

  \begin{block}{But for the myths}
    "Most remarkable of all, however, are the revelations that have
    emerged from the mental clinic." Because based on it "Freud, Jung,
    and their followers have demonstrated irrefutably that the logic,
    the~heroes, and the deeds of myth survive into modern times." Page
    20.
  \end{block}
  \pause

  \begin{block}{}
    "The latest incarnation of~Oedipus, the continued romance
    of~Beauty and the Beast, stand this afternoon on the corner
    of~Forty-second Street and fifth Avenue, waiting for the traffic
    light to change." Page 20.
  \end{block}

  \begin{block}{I can only replay}
    \begin{figure}
      \centering \includegraphics[height=0.8in, width=1.0in]{Stroke}
      \begin{captation}
      \end{figure}}
  \end{block}

\end{frame}
% ##########



% ##########
\begin{frame}
  \frametitle{The adventure of the Hero begins\ldots}

  \begin{block}{}
    "The unfortunate father is the first radical intrusion of another
    order of reality into the beatitude of this earthly restatement
    of~ the~excellence of the situation within the womb; he,
    therefore, is~experienced primarily as an enemy." Page 21.
  \end{block}
  \pause

  \begin{block}{I can't say more than this picture}
    \begin{figure}
      \centering \includegraphics[height=1.7in, width=2.2in]{IamYour1}
    \end{figure}
  \end{block}

\end{frame}
% ##########



% ##########
\begin{frame}
  \frametitle{The Monomyth}

  \begin{block}{}
    "The unfortunate father is the first radical intrusion of another
    order of reality into the beatitude of this earthly restatement
    of~ the~excellence of the situation within the womb; he,
    therefore, is~experienced primarily as an enemy." Page 21.
  \end{block}
  \pause

  \begin{block}{I can't say more than this picture}
    \begin{figure}
      \centering

      \includegraphics[height=1.7in, width=2.2in]{IamYour1}
    \end{figure}
  \end{block}

\end{frame}
% ##########



% ##########
\begin{frame}
  \frametitle{The Monomyth}

  \begin{block}{}
    "What is difficult to leave, then, is~not~the womb but~the~phallus
    - unless, indeed, the life-weariness has~already seized the~heart,
    when it will be death that calls with the~promise of bliss that
    formerly was the~lure of love. Full circle from the tomb of the
    womb to~the~womb of the tomb, we come: an ambiguous, enigmatical
    incursion into a world of solid matter that is soon to~melt from
    us, like the substance of a dream." Pages 24 and 25.
  \end{block}

\end{frame}
% ##########



% ##########
\begin{frame}
  \frametitle{The adventure of the Hero begins\ldots}

  \begin{block}{General strucutre of monomyth}
    "The standard path of the~mythological adventure of the hero
    is~a~magnification of the formula represented in the rites of
    passage: separation - initiation - return: which might be named
    the nuclear unit of the monomyth." Page 34.
  \end{block}

  \begin{block}{American context}
    \begin{figure}
      \centering

      \includegraphics[height=1.7in, width=2.2in]{SWIa}
    \end{figure}
  \end{block}

\end{frame}
% ##########



% ##########
\begin{frame}
  \frametitle{The adventure of the Hero begins\ldots}

  \begin{block}{Departure. Part I. The call to adventure}
    "A blunder - apparently the merest chance - reveals an unsuspected
    world, and the individual is drawn into a relationship with forces
    that are not rightly understood." Page 50.
  \end{block}

  \begin{block}{}
    \begin{figure}
      \centering

      \includegraphics[height=1.5in, width=2.0in]{SWII}
      \includegraphics[height=1.5in, width=2.0in]{SWIII}
    \end{figure}
  \end{block}

\end{frame}
% ##########



% ##########
\begin{frame}
  \frametitle{The adventure of the Hero begins\ldots}

  \begin{block}{Departure. Part I: The call to adventure}
    "This first stage of the mythological journey—which we
    havedesignated the "call to adventure" —signifies that destiny has
    summoned the~hero and transferred his spiritual center of gravity
    from within the pale of his society to a zone unknown." Page 54.
  \end{block}

  \begin{block}{}
    \begin{figure}
      \centering

      \includegraphics[height=1.5in, width=2.0in]{SWIV}
      \includegraphics[height=1.5in, width=2.0in]{SWV}
    \end{figure}
  \end{block}

\end{frame}
% ##########



% ##########
\begin{frame}
  \frametitle{The adventure of the Hero begins\ldots}

  \begin{block}{Departure. Part II: Refusal of the Call}
    "Often in actual life, and not infrequently in the myths and
    popular
    tales, we encounter the dull case of the call unanswered;\ldots \\
    His flowering world becomes a wasteland of dry stones and his~life
    feels meaningless\ldots Whatever house he builds, it will be
    a~house of~death: a labyrinth of cyclopean walls to hide from him
    his~Minotaur." Page 55.
  \end{block}

  \begin{block}{}
    \begin{figure}
      \centering

      \includegraphics[height=1.5in, width=2.0in]{SWVIa}
      \includegraphics[height=1.5in, width=2.0in]{SWVII}
    \end{figure}
  \end{block}

\end{frame}
% ##########



% ##########
\begin{frame}
  \frametitle{The adventure of the Hero begins\ldots}

  \begin{block}{Departure. Part III: Supernatural Aid}
    "For those who have not refused the call, the first encounter of
    the hero-journey is with a protective figure (often a little old
    crone or~old man) who provides the adventurer with amulets against
    the dragon forces he is about to pass." Page 55.
  \end{block}

  \begin{block}{}
    \begin{figure}
      \centering

      \includegraphics[height=1.5in, width=2.0in]{SWVIII}
      \includegraphics[height=1.5in, width=2.0in]{SWIX}
    \end{figure}
  \end{block}

\end{frame}
% ##########



% ##########
\begin{frame}
  \frametitle{The End}

\begin{block}{How Campbell said}
  "Long long ago, when wishing still could lead to something\ldots"
  Page 49.
\end{block}

\begin{block}{Or George Lucas}
  \begin{figure}
    \centering

    \includegraphics[height=1.5in, width=2.0in]{SWX}
  \end{figure}
\end{block}

\end{frame}
% ##########



% ##########
\begin{frame}
  \frametitle{The End}
  \begin{center}
    \LARGE{Thank you!}
  \end{center}

\end{frame}
% ##########



% ##########
\begin{frame}

  \begin{block}{Bibliography}
    \begin{itemize}
    \item[1] Joseph Campbell, \emph{The Hero with a Thousand Faces},
      Princeton University Press 2004.
    \item[2] Joseph Campbell, \emph{Bohater o~tysiącu twarzy}, \\ Zysk
      i~Sk\dywiz a 1997.
    \item[3] A. Burzyńska, M. P. Markowski, \emph{Teorie literatury~XX
        wieku}, Znak 2006.
    \end{itemize}
  \end{block}

\end{frame}
% ##########





% ####################################################################
% ####################################################################
% Bibliografia
\bibliographystyle{alpha} \bibliography{Bibliography}{}


% ############################

% Koniec dokumentu
\end{document}
