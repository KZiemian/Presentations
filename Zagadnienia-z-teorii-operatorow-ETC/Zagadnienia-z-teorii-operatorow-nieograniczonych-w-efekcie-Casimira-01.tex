\RequirePackage[l2tabu, orthodox]{nag}
\documentclass{beamer} \mode<presentation>
\usepackage[utf8]{inputenc}% Pozwala pisać polskie znaki bezpośrednio.
\usepackage[polish]{babel}
\usepackage[MeX]{polski}
\usepackage{microtype}
\let\lll\undefined
\usepackage{amsmath}
\usepackage{amsfonts, amssymb, amscd, amsthm}
\usepackage{latexsym}
\usepackage{subfigure}
\usepackage{xcolor}
\usepackage{graphicx}% Pozwala wstawiać grafikę.
\usepackage{verbatim}
\usepackage{calrsfs}
\usepackage{hyperref}
% #####################################################################


% Ustawienia beamera
\usetheme{Warsaw} 
\usecolortheme{orchid}

\setbeamertemplate{navigation symbols}{}
\setbeamertemplate{headline}{}
\setbeamersize{text margin left=3mm}
\setbeamersize{text margin right=3mm}

% #####################################################################

\let\oldemptyset\emptyset
\let\emptyset\varnothing


\newcommand{\ld}{\ldots}

% Slashe w tekście
\newcommand{\tb}{\textbf}
\newcommand{\tbs}{\textbackslash}

% Podstawowe oznaczenia matematyczne
\newcommand{\fr}{\frac}
\newcommand{\tfr}{\tfrac}
\newcommand{\tr}{\textrm}

% Oznaczenia ,,nad i pod''. Wymyśl lepszą nazwę
\newcommand{\ul}{\underline}
\newcommand{\ol}{\overline}
\newcommand{\wh}{\widehat}
\newcommand{\wt}{\widetilde}

% Trzcionki matematyczne
\newcommand{\mr}{\mathrm}
\newcommand{\mb}{\mathbb}
\newcommand{\mc}{\mathcal}
\newcommand{\mf}{\mathfrak}
\newcommand{\mbf}{\mathbf}

% Strzałki
\newcommand{\ra}{\rightarrow}
\newcommand{\Ra}{\Rightarrow}

\newcommand{\wtw}{wtedy i~tylko wtedy}

% Litery greckie
\newcommand{\al}{\alpha}
\newcommand{\be}{\beta}
\newcommand{\ga}{\gamma}
\newcommand{\del}{\delta}
\newcommand{\Del}{\Delta}
\newcommand{\la}{\lambda}
\newcommand{\eps}{\epsilon}
\newcommand{\veps}{\varepsilon}
\newcommand{\vp}{\varphi}
\newcommand{\om}{\omega}
\newcommand{\Om}{\Omega}
\newcommand{\si}{\sigma}
\newcommand{\Si}{\Sigma}

% Standardowe oznaczenia literowe
\newcommand{\N}{\mb{N}}
\newcommand{\R}{\mb{R}}
\newcommand{\C}{\mb{C}}
\newcommand{\D}{\mc{D}}
\let\H\undefined
\newcommand{\H}{\mc{H}}
\let\L\undefined
\newcommand{\L}{\mc{L}}
\newcommand{\Rn}{\R^{ n }}
\newcommand{\Rc}{\mc{R}}
\newcommand{\Cc}{\mc{C}}

% Mniej używane oznaczenia literowe
\newcommand{\B}{\mc{B}}
\newcommand{\Oc}{\mc{O}}
\newcommand{\Rp}{\R_{ + }}

% Teoria mnogości
\newcommand{\es}{\emptyset}
\newcommand{\sset}{\subset}
\newcommand{\setm}{\setminus}
\newcommand{\ti}{\times}

% Algebra
\newcommand{\Real}{\mf{Re}}
\newcommand{\Imag}{\mf{Im}}
\newcommand{\ot}{\otimes}
\newcommand{\Tr}{\mr{Tr}}

% Granice
\newcommand{\Lim}{\lim\limits}
\newcommand{\Liminf}{\ul{\lim}}
\newcommand{\Limsup}{\ol{\lim}}


% Sumy
\newcommand{\Sum}{\sum\limits}

% Różniczkowanie i pochodne
\newcommand{\pr}{\partial}
\newcommand{\de}{\mr{d}}
\newcommand{\dd}[3]{\fr{ \de^{ #1 } { #2 } }{ \de { #3 }^{ #1 } }}
\newcommand{\pd}[3]{\fr{ \pr^{ #1 } { #2 } }{ \pr { #3 }^{ #1 } }}

% Całki
\newcommand{\Int}{\int\limits}
\newcommand{\IntA}[1]{\Int_{ -\infty }^{ +\infty } \de #1 \;}
\newcommand{\IntB}[1]{\int_{ \R } \de #1 \;}
\newcommand{\IntC}[1]{\Int_{ 0 }^{ +\infty } \de #1 \;}
\newcommand{\IntCa}[2]{ \Int #1 \, \de#2 } % Juz z calkowaniem
\newcommand{\IntFi}[2]{ \Int \de#1 \, #2 } % Bardziej Fizyczna notacja;)
\newcommand{\IntWie}[3]{ \int_{ #1 } \de^{ #2 }#3 \; } % Wielowymiarowa.
\newcommand{\IntDo}[5]{ \Int_{ #1 } \de^{ #2 }#3\, \de^{ #4 } #5 \; }
\newcommand{\IntL}[3]{ \int\limits_{ { #1 } }^{ { #2 } } \d#3 \; }

% Powszechnie używane symbole 
\newcommand{\Cinfty}{\Cc^{ \infty }}

% Analiza funkcjonalna
\newcommand{\da}{\dagger}

% Różne oznaczenia z analizy
\newcommand{\dx}{\de x}

% Brakety
\newcommand{\lket}{\langle}
\newcommand{\rket}{\rangle}

% Wartość bezwzględna i normy
\providecommand{\absj}[1]{\lvert #1 \rvert}
\providecommand{\absd}[1]{\left| #1 \right|}
\newcommand{\norm}[1]{\left|\left| #1 \right|\right|}

% Oznaczenia wektorów
\newcommand{\veca}{\vec{ a }}
\newcommand{\vecb}{\vec{ b }}
\newcommand{\vecx}{\vec{ x }}
\newcommand{\vecp}{\vec{ p }}
\newcommand{\vecq}{\vec{ q }}


% Do wystąpień
\newcommand{\red}[1]{{\color{red} #1}}
\newcommand{\colorlink}[1]{\href{#1}{\color{orange} link}}

% Do tej konkretnej prezentacji
\newcommand{\DA}{\D( A )}
\newcommand{\g}{\wh{ g }}
\newcommand{\gl}{g_{ \la }}



% ####################################################################
\title[]{Zgadnienia teorii operatorów nieograniczonych \\
  w~efekcie Casimira}
\subtitle{Seminarium Zakładu Teorii Pola}
\author{Kamil Ziemian \\
  \texttt{kziemianfvt@gmail.com} }
\institute{Uniwersytet Jagielloński w Krakowie}
\date[24 marca 2017]{24 marca 2017}
% #####################################################################




% #####################################################################
\begin{document}
% #####################################################################




\begin{frame}
\titlepage
\end{frame}



\begin{frame}
\tableofcontents
\end{frame}



% ####################################################################
\section{Efekt Casimira dla swobodnego pola skalarnego}


\begin{frame}
  \frametitle{Przedstawienie problemu}


  \begin{block}{Podstawowe informacje}
    Dynamika swobodnego pola skalarnego dane jest przez równanie
    \begin{equation}
      \label{eq:EC01}
      ( \pr^{ 2 }_{ t } + h^{ 2 }) \phi( \vecx, t ) = 0, \quad h^{ 2 } = -\Del.
    \end{equation}
    Położenie ciał makroskopowych opisują parametry oznaczane zbiorczo
    jako $a$. Wprowadzamy operator:
    \begin{equation}
      \label{eq:EC02}
      h_{ a }^{ 2 } = h^{ 2 } + V_{ a } = -\Del + V_{ a }.
    \end{equation}
    $V_{ a }$ opisuje wpływ ciał makroskopowych na pole $\phi$.
    Równanie ewolucji pola z~uwzględnieniem efektów od~wprowadzonych
    ciał, ma następującą postać:
    \begin{equation}
      \label{eq:EC03}
      ( \pr^{ 2 }_{ t } + h_{ a }^{ 2 }) \phi( \vecx, t ) = 0.
    \end{equation}
  \end{block}

\end{frame}



\begin{frame}
  \frametitle{Problem fizyczny}

  \begin{block}{Dwa punkty materialne}
    Chcemy zbadać jak będzie wyglądał efekt Casimira, to pomysł
    dr.~Damskiego, gdy wprowadzimy do układu dwie cząstki punktowe. \\
    Oznacza to, że~\red{chcielibyśmy czegoś takiego jak potencjał}
    \begin{equation}
      \label{eq:EC04}
      V( \vecx ) = -c \del( \vecx - \vecx_{ 1 } ) - c \del( \vecx
      - \vecx_{ 2 } ),
      \quad c > 0.
    \end{equation}
    Z~nie do końca zrozumiałych dla mnie względów, w~dalszej analizie
    efektu Casimira, które nie będę tu omawiał, okazuje~się, że~musimy
    pracować z~pseudopotencjałem, który odpowiada słabo
    przyciągającemu potencjałowi $-c\del( \vecx - \vecx_{ 1 })$.
  \end{block}
  
\end{frame}



\begin{frame}
  \frametitle{Operator $V_{ a }$}

  \begin{block}{Jawna postać}
    Przyjęliśmy, że~operator ten, jest operatorem rzutowym.
    \begin{equation}
      \label{eq:EC05}
      V_{ \veca } = \si( g ) ( | U_{ \vecb } g \rket \lket U_{ \vecb } g |
      + | U_{ -\vecb } g \rket \lket U_{ -\vecb } g | ), \quad
      \vecb = \fr{ \veca }{ 2 }.
    \end{equation}
    $g( \vecx )$ jest sferycznie symetryczną funkcją zespoloną klasy
    $\Cinfty$.
    \begin{equation}
      \label{eq:EC06}
      (U_{ \vecb }g)( \vecx ) = g( \vecx - \vecb ).
    \end{equation}
    Czyli
    \begin{equation}
      \label{eq:EC07}
      | U_{ \vecb } g \rket \lket U_{ \vecb } g | f \rket
      = | U_{ \vecb } g \rket \left( \IntWie{ \R^{ 3 } }{ 3 }{ x }
        \ol{ g }( \vecx - \vecb ) f( \vecx ) \right).
    \end{equation}
  \end{block}

  \begin{block}{Skalowanie}
    Potencjał ,,dwóch $\del$'' otrzymujemy w~granicy $\la \ra 0$, gdy
    \begin{equation}
      \label{eq:EC08}
      g( \vecx - \vecb ) \ra g_{ \la }( \vecx - \vecb )
      = \la^{ -\ga } g( \tfr{ \vecx - \vecb }{ \la } ).
    \end{equation}
  \end{block}

\end{frame}



\begin{frame}
  \frametitle{Operator $V_{ a }$}

  \begin{block}{$\si( g )$}
    \begin{equation}
      \label{eq:EC09}
      V_{ \veca } = \si( g ) ( | U_{ \vecb } g \rket \lket U_{ \vecb } g |
      + | U_{ -\vecb } g \rket \lket U_{ -\vecb } g | ).
    \end{equation}
    Nie całkiem trywialna analiza pokazała, że~aby otrzymać
    interesujące oddziaływania w~granicy $\la \ra 0$, musimy przyjąć,
    że~$\si$ jest funkcjonałem na~funkcjach $g$.
    \begin{align}
      \label{eq:EC10}
      \g( \absj{ \vecp } ) = \g( \vecp ) &=  \fr{ 1 }{ \sqrt{ 2 \pi }^{ 2 } }
                                           \IntWie{ \R^{ 3 } }{ 3 }{ x } g( \vecx )
                                           e^{ -i \vecx \cdot \vecp }, \\
      M_{ p } &= \absj{ \g( p ) }^{ 2 }, \\
      \si^{ -1 }( g ) = \fr{ 1 }{ \si( g ) } &= -\al M_{ 0 }
                                               - 2 \pi \IntA{ p } M_{ p }.
    \end{align}
    Sytuacja ta wydaje mi~się dość nietypowa.
  \end{block}
  
\end{frame}



\begin{frame}
  \frametitle{Siła Casimira}

  \begin{block}{Podstawowe wzory fizyczne}
    Przypomnijmy
    \begin{align}
      \label{eq:EC11}
      h_{ a }^{ 2 } &= h^{ 2 } + V_{ a } = -\Del + V_{ a } \\
      V_{ \veca } = \si( g ) ( | U_{ \vecb } g \rket & \lket U_{ \vecb } g |
                                                       + | U_{ -\vecb } g \rket \lket U_{ -\vecb } g | ), \quad
                                                       \vecb = \fr{ \veca }{ 2 }.
    \end{align}
    Energia układu pola skalarnego $\phi$ w~obecności ciała
    makroskopowych dana jest wzorem, który może wydawać~się zupełnie
    niespodziewany.
    \begin{equation}
      \label{eq:EC12}
      \mc{E}_{ a } = (\Om_{ a }, H \Om_{ a } ) = \fr{ 1 }{ 4 } \Tr[ ( h_{ a } - h )
      h_{ a }^{ -1 } ( h_{ a } - h ) ],
    \end{equation}
    gdzie $\Om_{ a }$ to stan próżni dla operatora $h_{ a }^{ 2 }$.
    Jeżeli wykonamy podstawienie $g \ra \gl$, to powyższy wzór będzie
    zależał również od~$\la$.
  \end{block}
  
\end{frame}



% #####################################################################
\section{Pytania matematyczne}
% #####################################################################



\begin{frame}
  \frametitle{Podstawowe pytania matematyczne}

  \begin{block}{Operatory liniowe}
    Niech $\D( A ) \sset \H$ będzie podprzestrzenią liniową.
    \tb{Operator liniowy} to~liniowe odwzorowanie $A: \D( A ) \ra \H$.
    $\D( A )$ nazywamy \tb{dziedziną operatora}.
  \end{block}

  \begin{block}{Czy}
    \begin{enumerate}
    \item $h_{ a }^{ 2 } = -\Del + V_{ a }$ jest samosprzężony?
    \item Czy jego widmo nie zawiera części ,,niefizycznej''?
    \end{enumerate}
    Innych pytań, również matematycznych jest wiele, ale~nim nie
    będę~się zajmował.
  \end{block}

\end{frame}



% #####################################################################
\section{Problem dziedziny}
% #####################################################################



\begin{frame}
  \frametitle{Uwagi i~przypomnienie}

  \begin{block}{Ważne}
    Cały czas będziemy pracowali w~zespolonej przestrzeni Hilberta
    $\H$.
  \end{block}

  \begin{block}{Zbiór gęsty}
    W~przestrzeni metrycznej $X$ zbiór $D$ jest gęsty jeśli
    \begin{equation}
      \label{eq:EC}
      \forall x \in X, \exists \; a_{ n } \in D, \; \Lim_{ n \ra +\infty}
      a_{ n } = x.
    \end{equation}
    Ogólniejsze sformułowanie to $\ol{ D } = X$.
  \end{block}

  \begin{block}{Przypadek niekończenie wymiarowy}
    Jeżeli przestrzeń wektorowo\dywiz topologiczna, w~szczególności
    przestrzeń Hilberta, jest nieskończenie wymiarowa, to mogą istnieć
    gęste podprzestrzenie liniowe. \tb{Przykład.} Na mocy twierdzenia
    Weierstrassa o~aproksymacji wielomianami, podprzestrzeń
    wielomianów jest gęsta w przestrzeni funkcji
    $\Cc( [ a, b ], \R )$, jednak nie każda funkcja ciągła na odcinku
    jest wielomianem.
  \end{block}
  
\end{frame}



\begin{frame}
  \frametitle{Operatory ograniczone}

  \begin{block}{Operatory ograniczone}
    Operator jest ograniczony jeśli istnieje stała $C > 0$ taka, że
    \begin{equation}
      \label{eq:5}
      \forall \vp \in \D(A), \; \norm{ A \vp } \leq C \norm{ \vp }.
    \end{equation}
    Wtedy
    \begin{equation}
      \label{eq:3}
      \norm{ A } = \inf\{ C \, | \, \norm{ A \vp } \leq C \norm{ \vp } \},
    \end{equation}
    nazywamy normą operatora.
  \end{block}

  \begin{block}{Widmo samosprzężonego operatora ograniczonego}
    Widmo takie operatora zawiera~się w~odcinku
    $[ -\norm{ A }, +\norm{ A }]$. \\
  \end{block}

  \begin{block}{Wniosek}
    Większość powszechnie znanych i~lubianych operatorów nie może być
    ograniczona.
  \end{block}
  
\end{frame}



\begin{frame}
  \frametitle{Operatory nieograniczone}

  \begin{block}{Operatory nieograniczone}
    Jeśli operator $A$ o~dziedzinie $\D( A )$ nie jest ograniczony,
    to~nazywamy go~nieograniczonym (ang. unbounded). Inaczej, \tb{nie
      istnieje} stała $C > 0$, \\
    taka by zachodziło
    \begin{equation}
      \label{eq:5}
      \forall \vp \in \D(A), \; \norm{ A \vp } \leq C \norm{ \vp }.
    \end{equation}
    Inna charakterystyka jest następująca. \tb{Istnieje} ciąg
    $D( A ) \ni \vp_{ n } \ra 0$ \\
    taki, że
    \begin{equation}
      \label{eq:4}
      \Lim_{ n \ra +\infty } A \vp_{ n } \neq 0.
    \end{equation}
    W~istocie ciąg ten można wybrać tak, że
    \begin{equation}
      \label{eq:4}
      \Lim_{ n \ra +\infty } \norm{ A \vp_{ n } } = +\infty.
    \end{equation}
  \end{block}
  
\end{frame}



\begin{frame}
  \frametitle{Sedno wielu subtelności}

  \begin{block}{Twierdzenie Hellingera\dywiz Toeplitza}
    Jeśli $\D( A ) = \H$ i~operator $A$ jest samosprzężony, to jest
    ograniczony.
  \end{block}

  \begin{block}{Konsekwencje}
    Nieograniczony operator samosprzężony może być określony najwyżej
    na gęstej dziedzinie $\D( A )$. Ponieważ może zachodzić
    $\D( A ) \cap \D( B ) = \{ 0 \}$, choć obie dziedziny są gęste,
    większość wzorów z~podręczników do~mechaniki kwantowej zostaje
    poddana w~matematyczną wątpliwość. Żeby nie powiedzieć, że~zostaje
    zredukowana do czysto formalnych manipulacji, nonsensownych
    z~punktu widzenia matematyki.
  \end{block}

  \begin{block}{Konkrente problemy}
    \begin{enumerate}
    \item Zmiana dziedziny $\D( A )$ może zmienić widmo operatora,
      a~tym samym jego sens fizyczny.
    \item $A + B$ ma sens tylko na zbiorze $\D( A ) \cap \D( B )$.
    \end{enumerate}
  \end{block}
  
\end{frame}



\begin{frame}
  \frametitle{Sedno wielu subtelności}

  \begin{block}{Konkrente problemy}
    \begin{enumerate}
    \item Zmiana dziedziny $\D( A )$ może zmienić widmo operatora,
      a~tym samym jego sens fizyczny.
    \item $A + B$ ma sens tylko na zbiorze $\D( A ) \cap \D( B )$.
    \item $[ A, B ] = AB - BA$ ma sens tylko dla takiego $\vp$,
      że~$B \vp \in \D( A )$ i~$A \vp \in \D( B )$.
    \item $[ \wh{ x }, \wh{ p } ] = i \hbar I$, można udowodnić,
      że~operatory ograniczone nie mogą mieć takiej relacji komutacji.
      Z~drugiej strony, co ta relacja niby znaczy? Wszak trzeba
      ustalić dziedzinę prawej strony.
    \item Operator ewolucji
      \begin{equation}
        \label{eq:6}
        e^{ -\fr{ i }{ \hbar } \wh{ H } t } = I + \fr{ 1 }{ 1! }
        \left( -\fr{ i }{ \hbar } t \right) \wh{ H } + \fr{ 1 }{ 2! }
        \left( -\fr{ i }{ \hbar } t \right)^{ 2 } \wh{ H }^{ 2 } + \ld
        + \fr{ 1 }{ n! } \left( -\fr{ i }{ \hbar } t \right)^{ n }
        \wh{ H }^{ n } + \ld
      \end{equation}
      N\dywiz ty wyraz ma sens tylko na $\D( \wh{ H }^{ n } )$, więc
      ten szereg może być zbieżny tylko dla $0$.
    \end{enumerate}
  \end{block}
  
\end{frame}


\begin{frame}
  \frametitle{Problem głębszy}


  \begin{block}{Konkretne problemy --~relacje komutacji}
    Przykład z~książki Schiffa, \S 1.3, równanie (3.2). Rozpatrzmy
    operator współrzędnej kątowej $\wh{ \vp }$ i $\wh{ L }_{ z }$:
    \begin{align}
      \label{eq:8}
      \wh{ \vp } \psi( r, \vp, \theta ) &= \vp \psi( r, \vp, \theta ) \\
      \wh{ L }_{ z } &= -i \hbar \pd{}{}{ \vp }.
    \end{align}
    Od razu mamy:
    \begin{equation}
      \label{eq:9}
      [ \wh{ \vp }, \wh{ L }_{ z } ] = i \hbar I.
    \end{equation}
    Analogicznie więc jak dla położenia i~pędu musi być:
    \begin{equation}
      \label{eq:10}
      \si_{ \vp } \si_{ L_{ z } } \geq \fr{ \hbar }{ 2 }.
    \end{equation}
  \end{block}
  
\end{frame}



\begin{frame}
  \frametitle{Problem głębszy}

  \begin{block}{Problem}
    Którego ani ja, ani Shift nie zauważył, a~Pawłowi Duchowi zajęło
    to 15~sekund. Po pierwsze byłoby dziwne, gdyby
    $\si_{ \vp } > 2 \pi$. Po drugie, skoro istnieje zupełnie normalny
    stan własny $\wh{ L }_{ z }$, to dla niego mamy
    \begin{equation}
      \label{eq:11}
      \si_{ \vp } \si_{ L_{ z } } = \si_{ \vp } 0 = 0 \geq \fr{ \hbar }{ 2 }.
    \end{equation}
  \end{block}

  \begin{block}{Rozwiązanie problemu}
    Dla operatorów nieograniczonych, nie można tak po prostu liczyć
    komutatorów, bo trzeba ustalić odpowiednie dziedziny. Andrzej
    Herdegen i~Piotr Ziobro poświęcili temu zagadnieniu pracę
    ,,Generalized uncertainty relations'', arXiv:~1511.06589,
    \colorlink{https://arxiv.org/abs/1511.06589}, którą polecam
    zainteresowanym.
  \end{block}

  \begin{block}{Więcej tego typu zagadek}
    Można znaleźć w~F. Gieres ,,Mathematical surprises and~Dirac's
    formalism in~quantum mechanics'', arXiv:~9907069,
    \colorlink{https://arxiv.org/abs/quant-ph/9907069}.
  \end{block}

\end{frame}



\begin{frame}
  \frametitle{Operatory samosprzężone}

  \begin{block}{Szczęśliwie}
    Fizyka matematyczna wysiłkiem takich ludzi jak John von Neumann,
    Tosio Kato, Marshall Stone i~wielu innych, potrafi sobie poradzić
    z~większością tych wyzwań przypadku najważniejszy modeli
    i~problemów nierelatywistycznej mechaniki kwantowej, a~niekiedy
    nawet w~bardziej skomplikowanych. \\
    Więcej w~bibliografii.
  \end{block}

  \begin{block}{Uwaga}
    Od tej pory zakładamy, że~dziedziny wszystkich operatorów~są gęste
    w~$\H$.
  \end{block}

  \begin{block}{Operatory symetryczne}
    Operator nieograniczony $A$ jest symetrycznych, jeśli
    \begin{equation}
      \label{eq:7}
      \forall \vp, \psi \in \DA, \; ( \vp, A\psi ) = ( A\vp, \psi ).
    \end{equation}
  \end{block}

\end{frame}



\begin{frame}
  \frametitle{Operatory samosprzężone}

  \begin{block}{Operator sprzężony}
    Niech dany będzie operator $A$ z~gęstą dziedziną $\DA$. Oznaczmy
    przez $\D( A^{ * } )$ zbiór wektorów $\vp$ takich,
    że~odwzorowanie:
    \begin{equation}
      \label{eq:12}
      \DA \ni \psi \ra ( \vp, A\psi ),
    \end{equation}
    jest ciągłe na~$\DA$. Używając twierdzenia o~rozszerzaniu
    operatorów ograniczonych i~lematu Risza, dostajemy, że~istnieje
    wektor $\eta$ taki, iż
    \begin{equation}
      \label{eq:13}
      \forall \psi \in \DA, \; ( \vp, A\psi ) = ( \eta, \psi ).
    \end{equation}
    Przypomnienie z~algebry liniowej. Definiujemy operator $A^{ * }$
    jako
    \begin{equation}
      \label{eq:14}
      A^{ * } \vp = \eta.
    \end{equation}
    Jest to operator liniowy o~dziedzinie $\D( A^{ * } )$, sprzężony
    do~$A$.
  \end{block}

\end{frame}



\begin{frame}
  \frametitle{Operatory samosprzężone}

  \begin{block}{Operatory samosprzężone}
    Operator $A$ jest samosprzężony jeśli:
    \begin{enumerate}
    \item $\DA = \D( A^{ * } )$.
    \item
      $\forall \vp, \psi \in \DA, \; ( \vp, A\psi ) = ( A\vp, \psi )$.
    \end{enumerate}
    Oba warunki~są równie ważne. Choć ze świecą szukać pierwszego
    w~podręcznikach do~mechaniki kwantowej.
  \end{block}

  \begin{block}{Przypomnienie} % C-x C-+
    Czy możemy stwierdzić, że operator
    \begin{equation}
      \label{eq:15}
      h_{ a }^{ 2 } = -\Del + V_{ a },
    \end{equation}
    jest samosprzężony, jeśli wiemy, że~laplasjan jest samosprzężony?
  \end{block}
  
\end{frame}



\begin{frame}
  \frametitle{Twierdzenie Kato\dywiz Rellicha}

  \begin{block}{$H$\dywiz ograniczoność}
    Niech $H$ i~$A$, będą operatorami o~gęstych dziedzinach. Jeżeli
    \begin{enumerate}
    \item $\D( H ) \sset \DA$.
    \item Istnieją stałe $a, b > 0$ takie, że
      \begin{equation}
        \label{eq:16}
        \forall \vp \in \D( H ), \; \norm{ A\vp } \leq a \norm{ H \vp }
        + b \norm{ \vp }.
      \end{equation}
    \end{enumerate}
    Wówczas $A$ nazywamy $H$\dywiz ograniczony. Liczbę $\inf a$
    nazywamy $H$\dywiz granicą operatora $A$.
  \end{block}

  \begin{block}{Wniosek} % C-x C-+
    Dla operatora ograniczonego $A$ można przyjąć $a = 0$
    i~$b = \norm{ A }$. Operator ograniczony jest więc $H$\dywiz
    ograniczony dla każdego operatora $H$ z~$H$\dywiz granicą równą 0.
    Detale pominę.
  \end{block}
  
\end{frame}




\begin{frame}
  \frametitle{Twierdzenie Kato\dywiz Rellicha}

  \begin{block}{Uproszczone twierdzenie Kato\dywiz Rellicha}
    Załóżmy, że~operator $H_{ 0 }$ jest samosprzężony, a~$V$ jest
    symetryczny, $H$\dywiz ograniczony z~$H$\dywiz granicą mniejszą od
    1. Wtedy $H = H_{ 0 } + V$ jest samosprzężony na $\D( H )$
  \end{block}

  \begin{block}{Przypadek badanego efektu Casimira}
    \begin{align}
      \label{eq:19}
      h_{ a }^{ 2 } &= -\Del + V_{ a }, \\
      V_{ \veca } = \si( g ) ( | U_{ \vecb } &g \rket \lket U_{ \vecb } g |
                                               + | U_{ -\vecb } g \rket \lket U_{ -\vecb } g | ).
    \end{align}
    $V_{ a }$ jest operatorem skończonego rzędu, tym bardziej jest
    ograniczony. \\
    $h_{ a }^{ 2 }$ jest więc samosprzężony na~$\D( -\Del )$.
  \end{block}
  
\end{frame}



% #####################################################################
\section{Analiza spektralna widma ciągłego}
% #####################################################################



\begin{frame}
  \frametitle{Klasyfikacja widma operatora samosprzężonego}

  \begin{block}{Widmo operatora samosprzężonego $\si( A )$}
    To nie tylko zbiór $\si( A ) \sset \R$, ale to także pewna miara
    $\mu_{ A }$ na tym zbiorze. Dokładniej: dodatnia, $\si$\dywiz
    skończona miara borelowska na~tym zbiorze. Czyli
    \begin{itemize}
    \item[--] Jest określona na zbiorach borelowskich
      $\B( \si( A ) )$;
    \item[--] $\mu_{ A } : \B( \si( A ) ) \ra \Rp$;
    \item[--]
      $\si( A ) = \bigcup\limits_{ i = 1 }^{ \infty } \Oc_{ i }$,
      $\mu_{ A }( \Oc_{ i } ) < +\infty$;
    \end{itemize}
  \end{block}

  \begin{block}{Standardowa mądrość mówi,~że}
    Widmo dzieli~się na dyskretne oraz ciągłe. I~tyle. W~fizyce
    matematycznej jest~trudniej/ciekawiej. Widmo ciągłe należy jeszcze
    podzielić na część absolutnie ciągłą względem miary Lebesgue'a
    i~część osobliwą względem tej miary.
  \end{block}

\end{frame}



\begin{frame}
  \frametitle{Klasyfikacja widma operatora samosprzężonego}

  \begin{block}{Standardowa mądrość mówi,~że}
    Widmo dzieli~się na dyskretne oraz ciągłe. I~tyle. W~fizyce
    matematycznej jest~trudniej/ciekawiej. Widmo ciągłe należy jeszcze
    podzielić na część absolutnie ciągłą względem miary Lebesgue'a
    i~część osobliwą względem tej~miary.
  \end{block}
  
  \begin{block}{Szczególny przypadek twierdzenia Lebesgue'a
      o~rozkładzie miary}
    Dowolna miara $\si$\dywiz skończona na~$\R$ ma rozkład
    $\mu_{ A } = \mu_{ pp } + \mu_{ ac } + \mu_{ s }$.
    \begin{itemize}
    \item[--] $\mu_{ pp }( S ) = \Sum_{ x \in S } \mu_{ pp }( x )$,
      $S \sset \si( A )$. Istnieje przeliczalna ilość punktów o~mierze
      różnej od~0, $\mu_{ pp }( x_{ i } ) \neq 0$. Zbiór który nie ma
      zawiera tych punktów ma miarę~0.
    \item[--] $\mu_{ ac }( x ) = f( x ) \, \de x$, gdzie $\de x$ jest
      miarą Lebesgue'a na~prostej. Nie chcę wchodzić w~szczegóły.
    \item[--] $\mu_{ s }$. Jeżeli $\mu_{ s }( S ) \neq 0$, to miara
      Lebesgue'a tego zbioru jest równa 0. Analogicznie w~drugą
      stronę.
    \end{itemize}
    
  \end{block}

\end{frame}



\begin{frame}
  \frametitle{Klasyfikacja widma operatora samosprzężonego}

  \begin{block}{Sens widma}
    \begin{itemize}
    \item[--] $\mu_{ pp }$, (ang. pure points) --~stany związane.
    \item[--] $\mu_{ ac }$, (ang. absolutely continous) --~stany
      rozproszeniowe. % Sprawdź angielski.
    \item[--] $\mu_{ s }$, (ang. singular) --~standardowo uważa~się,
      że nie mają sensu fizycznego. Dobra teoria powinna być go
      pozbawiona.
    \end{itemize}
    
  \end{block}

  \begin{block}{Twr. XIII 20, Reed, Simon, \emph{Analysis~of Operators}}
    Niech $G( w^{ 2 } ) = ( w^{ 2 } I - H )^{ -1 }$, będzie rezolwentą
    operatora~$H$ i~niech $(a , b)$ będzie ograniczonym odcinkiem.
    Jeśli istnieje taki $\veps > 0$, $\eta > 0$, $p > 1$ i~gęsty
    podzbiór~$\D$
    \begin{equation}
      \label{eq:17}
      \sup_{ 0 < \veps < \eta } \Int_{ a }^{ b }
      \absj{ \Imag( \vp, G( x + i \veps ) \vp ) }^{ p } \, \dx < +\infty,
      \quad \forall \vp \in \D,
    \end{equation}
    to~na~odcinku $( a, b )$ jest tylko widmo absolutnie ciągłe.
  \end{block}
  
\end{frame}



\begin{frame}
  \frametitle{Widmo ciągłe w~efekcie Casimira}

  \begin{block}{Ogólnie}
    Widmo $-\Del$ składa~się tylko z~widma absolutnie ciągłego, które
    jest równe $[ 0, +\infty )$. Z~twierdzenia Weyla wiemy,
    że~$\si( h_{ a }^{ 2 } )_{ c } \sset [ 0, +\infty )$.
  \end{block}

  \begin{block}{Idea}
    \begin{equation}
      \label{eq:17}
      \sup_{ 0 < \veps < \eta } \Int_{ a }^{ b }
      \absj{ \Imag( \vp, G( x + i \veps ) \vp ) }^{ p } \, \dx < +\infty,
      \quad \forall \vp \in \D,
    \end{equation}
    Prof. Herdegen zaproponował, żeby znaleźć pokazać, żeby dla~danego
    odcinka $( a, b )$ i każdego $\veps$ z~przedziału $( 0, \eta )$
    wyraz $( \vp, G( x + i \veps ) \vp )$ jest ograniczony.
  \end{block}

\end{frame}



\begin{frame}
  \frametitle{Widmo ciągłe w~efekcie Casimira}

  \begin{block}{Rezolwenta $h_{ a }^{ 2 }$}
    \begin{align}
      \label{eq:18}
      G( w^{ 2 } ) &= G_{ 0 }( w^{ 2 } ) + G_{ V }( w^{ 2 } ) \\
      G_{ 0 }( w^{ 2 } ) | \vecp \rket &= \fr{ 1 }{ w^{ 2 } - \vecp^{ 2 } }.
    \end{align}
    Należy więc pokazać, że~dla wszystkich $\vp$ z~pewnego gęstego
    zbioru $\D$ i~na pasku $( a, b ) + i \veps$,
    $\veps \in ( 0, \eta )$, ograniczone~są jednocześnie iloczyny
    skalarne $( \vp, G_{ 0 }( x + i \veps ) \vp )$
    i~$( \vp, G_{ V }( x + i \veps ) \vp )$
  \end{block}

\end{frame}



\begin{frame}
  \frametitle{Widmo ciągłe w~efekcie Casimira}

  \begin{block}{Rezolwenta $G_{ V }( w^{ 2 } )$}
    Jest do obliczenia za pomocą wzorów z~teorii rozpraszania. Jak~się
    okazuje
    \begin{equation}
      \label{eq:20}
      \begin{split}
        ( f, G_{ V } f ) = &\left( \IntWie{}{ 3 }{ p } \ol{ f( \vecp )
          } \g( \vecp ) \fr{ 1 }{ w^{ 2 } - \vecp^{ 2 } } ( \exp( -i
          \vecb \cdot \vecp ) \; \exp( -i \vecb \cdot \vecp )
          ) \right) \\
        & T( w^{ 2 } ) \left( \IntWie{}{ 3 }{ q } f( \vecq ) \ol{ \g(
            \vecq ) } \fr{ 1 }{ w^{ 2 } - \vecq^{ 2 } } ( \exp( -i
          \vecb \cdot \vecq ) \; \exp( -i \vecb \cdot \vecq ) )^{ \da
          } \right).
      \end{split}
    \end{equation}
    $T( w^{ 2 } )$ jest macierzą $2 \ti 2$.
  \end{block}

  \begin{block}{Taktyka dowodu}
    Dla rezolwenty laplasjanu $( f, G_{ 0 }( k^{ 2 } + i \veps ) f)$
    jest ograniczony w~paskach, gdy~$f$ należy do~gęstego
    w~$L^{ 2 }( \R^{ 3 }, \de^{ 3 } x )$ zbioru funkcji klasy
    $\Cinfty$ o~nośniku zwartym. Trzeb pokazać, że~na tym samym
    zbiorze jest ograniczony ten sam iloczyn skalarny dla
    $G_{ V }( w^{ 2 } )$.
  \end{block}
  
\end{frame}



\begin{frame}
  \frametitle{Widmo ciągłe w~efekcie Casimira}

  \begin{block}{Ostatnie wyniki}
    Dotyczą pokazania, że~wyrazy typu
    \begin{equation}
      \label{eq:21}
      \IntWie{}{ 3 }{ p } \ol{ f( \vecp ) } \fr{ 1 }{ w^{ 2 }
        - \vecp^{ 2 } },
    \end{equation}
    są~ograniczone na rozpatrywanych paskach.
  \end{block}

  \begin{block}{Obliczenia}
    \begin{align}
      \label{eq:22}
      \IntWie{}{ 3 }{ p } \ol{ f( \vecp ) } \fr{ 1 }{ w^{ 2 }
      - \vecp^{ 2 } } &= 2 \pi \IntA{ p } \fr{ p^{ 2 } }{ w^{ 2 }
                        - p^{ 2 } } N( p ) \\
      N( p ) &= \Int_{ S^{ 2 } } \de \Om \; f( \vecp ).
    \end{align}
    $N( p )$ jest funkcją klasy $\Cinfty( \R )$.
  \end{block}
  
\end{frame}



\begin{frame}
  \frametitle{Widmo ciągłe w~efekcie Casimira}

  \begin{block}{Przekształcenia}
    Podstawmy $k^{ 2 } + i \veps$
    \begin{equation}
      \label{eq:22}
      \begin{split}
        \IntA{ p } \fr{ p^{ 2 } }{ k^{ 2 } - p^{ 2 } + i \veps } N( p
        ) &= \IntA{ p } N( p ) \fr{ k^{ 2 } + i \veps }{ k^{ 2 } - p^{
            2 } + i \veps } - \IntA{ p } N( p ).
      \end{split}
    \end{equation}
    Trzeba teraz pokazać, że~wyrażenie to~jest ograniczone, gdy
    $k^{ 2 } \in ( a^{ 2 }, b^{ 2 })$, $a > 0$, $b > 0$.
  \end{block}

\end{frame}



\begin{frame}[fragile]
  \frametitle{Standardowe operacja z~rachunku dystrybucji}

  \begin{block}{}
    \begin{equation}
      \label{eq:23}
      \begin{split}
        & \IntA{ p } N( p ) \fr{ k^{ 2 } + i \veps }{ k^{ 2 } - p^{ 2
          } + i \veps } = N( k ) \IntA{ p } \fr{ k^{ 2 } + i \veps}
        { k^{ 2 } - p^{ 2 } + i \veps } \\
        & + ( k^{ 2 } + i \veps ) \IntA{ p } \fr{ N( p ) - N( k ) } {
          k^{ 2 } - p^{ 2 } }
        \fr{ k^{ 2 } - p^{ 2 } }{ k^{ 2 } - p^{ 2 } + i \veps } \\
        &= -i \pi \sqrt{ k^{ 2 } + i \veps } N( k ) + ( k^{ 2 } + i
        \veps ) \IntA{ p } \fr{ N( p ) - N( k ) }{ k^{ 2 } - p^{ 2 } }
        \fr{ k^{ 2 } - p^{ 2 } }{ k^{ 2 } - p^{ 2 } + i \veps }.
      \end{split}
    \end{equation}
    $( N( p ) - N( k ) )/( k^{ 2 } - p^{ 2 } )$ jest funkcją gładką,
    więc wyrażenie wyżej jest ograniczone jeśli tylko $k^{ 2 } > 0$.
    Biorąc $k^{ 2 } \in ( a^{ 2 }, b^{ 2 } )$, dla coraz mniejszych
    $a > 0$, możemy ,,wyczerpać'' zbiór $( 0, +\infty )$ i~wraz
    z~podobnie uzyskaną wiedzą o~macierzy $T( w^{ 2 } )$ pozwala
    wykazać, że~nie zawiera widma osobliwego. Podsumowując
    $\si_{ s }( h_{ a }^{ 2 } ) = \emptyset$.
  \end{block}

\end{frame}



% \begin{frame}[fragile]
%   \frametitle{Wróćmy do~tekstu}

%   \begin{block}{Część główna}
%     Zawiera tekst, który (po obrobieniu przez \LaTeX a) znajdzie~się
%     w~gotowym dokumencie.
%     \begin{verbatim}
% \begin{document}



% Hello World!



% \end{document}
% \end{verbatim}
%   \end{block}

% \end{frame}



% \begin{frame}
%   \frametitle{Białe znaki w~\LaTeX u}

%   \begin{block}{}
%     Białe znaki to \tb{spacja}, \tb{tab} i~\tb{enter}. \LaTeX{}
%     traktuje jedną spację i~100 spacji tak samo: jako jeden odstęp.
%     Tab traktuje jako spację. Przejście do nowej linii,
%     \red{traktuje jako jeden odstęp.} Jedną pustą linię traktuje
%     jako komendę, że~w~tym miejscu ma {\color{red} skończyć akapit}
%     (częsty błąd początkujących), zaś~100 pustych linii jak jedną
%     pustą linię.
%   \end{block}

% \end{frame}



% \begin{frame}
%   \frametitle{Białe znaki w~\LaTeX u}

%   \begin{block}{Organizacja tekstu za pomocą białych znaków}
%     Plik dobrze napisany, to plik który łatwiej~się poprawia.
%     A~każdy musi go~poprawiać.
%     \begin{itemize}
%     \item[--] W~preambule białe znaki nie mają znaczenie, co nie
%       znaczy, że~nie należy ich używać. Wręcz przeciwnie!
%     \item[--] Ponieważ przejście do~nowej to tyle samo co spacja,
%       warto dzielić tekst na~krótsze linie (ja stosuje standard 79
%       znaków). Dzięki temu nie musimy scrolować tekstu by~go
%       zmienić,
%       wszystko jest na ekranie. \\
%       Poza tym komentarze lepiej działają, ale~o~tym za chwilę.
%     \item[--] Ja wyznaję zasadę, że~dwie ważne części pliku
%       źródłowego powinien oddzielać odstęp 3~pustych linii.
%     \item[--] Pomiędzy mniej ważnymi fragmentami powinny
%       znajdować~się 2~lub~1 puste linie. Przy założeniu,
%       że~przejście do~nowego akapitu nie psuje wyglądu tekstu.
%     \end{itemize}
%   \end{block}
  
% \end{frame}



% \begin{frame}[fragile]
%   \frametitle{Przykład}

%   \begin{block}{Stąd taki dziki wygląd tego pliku}
    
%   \begin{verbatim}
% \RequirePackage[l2tabu, orthodox]{nag}
% \documentclass{article}



% \begin{document}



% Hello World!



% \end{document}
% \end{verbatim}
%   \end{block}

% \end{frame}



% \begin{frame}
%   \frametitle{Białe znaki w~\LaTeX u}

%   \begin{block}{Organizacja tekstu za pomocą białych znaków}
%     Plik dobrze napisany, to plik który łatwiej~się poprawia.
%     A~każdy musi go~poprawiać.
%     \begin{itemize}
%     \item[--] Ponieważ przejście do~nowej to tyle samo co spacja,
%       warto dzielić tekst na~krótsze linie (ja stosuje standard 79
%       znaków). Dzięki temu nie musimy scrolować tekstu by~go
%       zmienić,
%       wszystko jest na ekranie. \\
%       Poza tym komentarze lepiej działają, ale~o~tym za chwilę.
%     \item[--] Ja wyznaję zasadę, że~dwie ważne części pliku
%       źródłowego powinien oddzielać odstęp 3~pustych linii.
%     \item[--] Pomiędzy mniej ważnymi fragmentami powinny
%       znajdować~się 2~lub~1 puste linie. Przy założeniu,
%       że~przejście do~nowego akapitu nie psuje wyglądu tekstu.
%     \end{itemize}
%   \end{block}

%   \begin{block}{Ważne}
%     Nie musicie tego robić tak jak ja, jednak gorąco polecam przyjąć
%     jakiś standard. To~niezmiernie ułatwia pracę z~plikiem
%     źródłowym.
%   \end{block}
  
% \end{frame}



% \begin{frame}[fragile]
%   \frametitle{Polskie znaki}

%   \begin{block}{Na razie pewnie nie da~się ich pisać}
%     \LaTeX{} został stworzony do pracy z~językiem angielskim, stąd
%     aby~używać wygodnie znaków polskich, należy rozszerzyć jego
%     możliwości.
%   \end{block}

%   \begin{block}{Mądrość starożytnych, znaleziona w~bibliotece
%     dr.~Google}
%     Powiada że~należy tak zmodyfikować preambułę
% \begin{verbatim}
% \documentclass[a4paper,11pt]{article}
% \usepackage[utf8]{inputenc}
% \usepackage[polish]{babel}
% \usepackage[MeX]{polski}
% \end{verbatim}
%   \end{block}

%   \begin{block}{\tbs usepackage}
%     Dosłownie ,,użyj paczki''. Importuje paczkę, która rozszerza
%     możliwości standardowego \LaTeX a.
%   \end{block}
  
% \end{frame}



% \begin{frame}[fragile]
%   \frametitle{Co warto zawsze mieć w~preambule?}

%   \begin{block}{Według mnie to}
% \begin{verbatim}
% \let\lll\undefined
% \usepackage{amsmath, amsfonts, amssymb, amscd, amsthm}
% \usepackage{calrsfs}
% \usepackage{xcolor}
% \end{verbatim}
%   \end{block}
  
% \end{frame}



% \begin{frame}[fragile]
%   \frametitle{Co warto zawsze mieć w~preambule?}

%   \begin{block}{}
% \begin{verbatim}
% \usepackage{vmargin}
% % ------------------------------------------------------
% % MARGINS
% % ------------------------------------------------------
% \setmarginsrb
% { 0.7in} % left margin
% { 0.6in} % top margin
% { 0.7in} % right margin
% { 0.8in} % bottom margin
% {  20pt} % head height
% {0.25in} % head sep
% {   9pt} % foot height
% { 0.3in} % foot sep

% \usepackage{hyperref}
% \usepackage{cleveref}
% \end{verbatim}
%   \end{block}
  
% \end{frame}



% \begin{frame}[fragile]
%   \frametitle{Polecenia w~części głównej}

%   \begin{block}{Składnia}
%     Polecenie zaczyna~się ,,\tbs'' zaś kończy \tb{spacją} lub
%     nawiasem wąsatym ,,\}''. Spacja w~pierwszym przypadku \tb{nie
%     oznacza} odstępu, lecz koniec polecenia.
%   \end{block}
  
%   \begin{block}{Wpiszcie taki tekst}
%     \verb+\TeX i~\LaTeX to \emph{nie} jest to samo.+
%   \end{block}

%   \begin{block}{Dlaczego?}
%     WYWINWYG =~to czego chcesz, to nie jest to co dostajesz.
%     Ponieważ spacja po \verb+\TeX+ nie oznacza odstępu, tylko koniec
%     polecenia \LaTeX{} rozumiem, że~odstępu ma tam nie być. Danie
%     dwóch spacji nic nie da, bo dla niego 1 spacja = $N$ spacji =
%     koniec komendy. Są różne szkoły radzenia sobie z~tym:
%     \begin{itemize}
%     \item[--] \verb+\TeX{} i~\LaTeX{} to+;
%     \item[--] \verb+{\TeX} i~{\LaTeX} to+.
%     \end{itemize}
%   \end{block}

% \end{frame}



% \begin{frame}[fragile]
%   \frametitle{Wdowy i~sieroty}

%   \begin{block}{Po co ta $\sim$?}
%     Przyimki takie jak ,,a'', ,,w'', ,,u'', ,,z'', czy też zaimek
%     zwrotny ,,się'' \\
%     źle wyglądają na końcu lub początku linii, bo~nie mają sensu bez
%     słów które~są zaraz obok. Podobno określa~się je jako wdowy
%     i~sieroty. \LaTeX{} sam decyduje jak rozłożyć graficznie słowa
%     ,,na~kartce'', więc często produkuje takie kwiatki. Aby temu
%     zapobiec, należy użyć \tb{twardej spacji}. Od zwykłe spacji
%     różni~się tym, że~\LaTeX{} zrobi wszystko, by~te słowa
%     znalazły~się w~jednej linii, oznacza ją właśnie \verb+~+.
%   \end{block}

%   \begin{block}{Porównanie}
%     \begin{itemize}
%     \item[--] \tb{Poprawnie:} \verb+zrobi~się+.
%     \item[--] \tb{Niepoprawnie:} \verb+zrobi się+.
%     \end{itemize}
%   \end{block}

%   \begin{block}{Dłuższe wyrażenia}
%     Jeśli chcemy by jakiś ciąg słów nigdy nie został rozbity między
%     dwie linie, to~piszemy np.~\verb+\mbox{coś tam, coś tam}+.
%   \end{block}
  
% \end{frame}



% \begin{frame}[fragile]
%   \frametitle{\% --~nasz najlepszy przyjaciel w~\LaTeX u}

%   \begin{block}{O~co chodzi z~tym~\%?}
%     Jak każdy dobry język programowania, \LaTeX{} posiada
%     komentarze. Wszystko od znaku \%, do końca linii jest zupełnie
%     przez niego ignorowane. Jeśli chcemy mieć znak procenta
%     w~tekście trzeba napisać \verb+\%+.
%   \end{block}

%   \begin{block}{Podstawowe rady}
%     \begin{itemize}
%     \item[--] Jeżeli dopiero zaczynasz pisać w~\LaTeX u i~uznałeś,
%       że~chcesz usunąć tekst z~pliku źródłowego, to~najlepiej go
%       wykomentuj. Zaraz dojdziesz do~wniosku, że~by ci~się przydał.
%     \item[--] Komentarz to najlepszy debugger na świecie, dzięki
%       znanej już Platonowi, to jest bodaj w~dialogu ,,Sofista'',
%       metodzie bisekcji. Za~sekundę do~tego wrócimy.
%     \end{itemize}
%   \end{block}

%   \begin{block}{Skorzystaj z~możliwości swojego środowiska}
%     Każde środowisko ma skrót klawiszowy odpowiedzialne
%     za~wykomentowanie i~odkomentowanie dowolnego bloku tekstu.
%     Dla~\TeX Makera: Ctrl-T, Ctrl-U.
%   \end{block}
  
% \end{frame}



% \begin{frame}[fragile]
%   \frametitle{Plik~się nie kompiluje}

%   \begin{block}{Początkujący powinien powiesić to nad łóżkiem}
%     Skróty klawiszowe do~komentowania i~odkomentowania bloków
%     tekstu. Dla \TeX Makera:
%     \begin{itemize}
%     \item[--] wykomentuj --~Ctrl-T;
%     \item[--] odkomentuj --~Ctrl-U.
%     \end{itemize}
%     Skoro już przy tym jesteśmy, to~trzeba powiedzieć o~trybie
%     matematycznym.
%   \end{block}

%   \begin{block}{Jeśli mamy błąd kompilacji i~nie wiesz jaki}
%     \begin{enumerate}
%     \item Kompilator podaje nam linię w~której jest błąd, ale
%       często~się myli w~zakresie 10~linii. Jeśli linie~są relatywnie
%       krótkie, to~jest większa szansa, że~poda poprawną.
%     \item Odkomentowujemy tekst, gdzie kompilator wskazuje błąd.
%     \item Kompilujemy jeszcze raz. Jeśli błąd jest wciąż obecny, to
%       znaczy, że~błędów jest więcej albo~wykomentowaliśmy zły
%       fragment.
%     \item Powtarzamy krok 2, aż~kompilator zadziała.
%     \end{enumerate}
%   \end{block}
  
% \end{frame}



% \begin{frame}[fragile]
%   \frametitle{Plik~się nie kompiluje}

%   \begin{block}{Jeśli mamy błąd kompilacji i~nie wiesz jaki}
%     \begin{enumerate}
%     \item Kompilator podaje nam linię w~której jest błąd, ale
%       często~się myli w~zakresie 10~linii. Jeśli linie~są relatywnie
%       krótkie, to~jest większa szansa, że~poda poprawną.
%     \item Odkomentowujemy tekst, gdzie kompilator wskazuje błąd.
%     \item Kompilujemy jeszcze raz. Jeśli błąd jest wciąż obecny, to
%       znaczy, że~błędów jest więcej albo~zakomentowaliśmy zły
%       fragment.
%     \item Powtarzamy krok 2, aż~kompilator zadziała.
%     \item Przyglądamy~się krótkiemu fragmentowi zakomentowanego
%       tekstu, jeśli~znajdziemy w~nim błąd to go poprawiamy, jeśli
%       nie to go~odkomentowujemy.
%     \item Jeśli zadziałał, to powtarzamy krok 5, jeśli nie, to
%       znaczy, że~przeoczyliśmy błąd i~musimy dokładnie
%       przeanalizować odkomentowany fragment.
%     \item Stosujemy tę procedurę, aż~wszystko działa.
%     \end{enumerate}
%   \end{block}

% \end{frame}



% \begin{frame}[fragile]
%   \frametitle{Warto robić wcięcia}

%   \begin{block}{Dlaczego warto skracać linie?}
%     Relatywnie krótkie linie w~pliku źródłowym pozwalają pracować na
%     małych fragmentach tekstu, a~znacznie łatwiej znaleźć błąd
%     w~krótkim fragmencie tekstu niż w~długim.
%   \end{block}
  
%   \begin{block}{Przykład trybu matematycznego}
%     \begin{equation}
%       \label{eq:2}
%       \eps > 0, \quad \veps > 0.
%     \end{equation}
%     Aby to otrzymać piszemy
% \begin{verbatim}
% \begin{equation}
%   \label{eq:1}
%   \epsilon > 0, \quad \varepsilon > 0.
% \end{equation}
% \end{verbatim}
%     Wcięcia bardzo ułatwiają później pracę z~\LaTeX em, dobre
%     środowisko zrobi je za was. A~o~trybie matematyczny, więcej
%     opowie wam Wojtek.
%   \end{block}
    
% \end{frame}



% \begin{frame}[fragile]
%   \frametitle{Tryb matematyczny}

%   \begin{block}{Chcemy mieć}
%     \begin{equation}
%       \label{eq:1}
%       \pd{ 2 }{ u( x, t ) }{ x } - \fr{ 1 }{ c^{ 2 } } \pd{ 2 }{ u(
%         x, t ) }{ t } = 0.
%     \end{equation}
%     Będzie trochę roboty, ale~potem pokażemy jak zmniejszyć jej
%     ilość do~rozsądnych rozmiarów.
%   \end{block}

%   \begin{block}{Słownik}
%     \begin{itemize}
%     \item[--] potęga --~\verb+a^{ 2 }+ = $a^{ 2 }$;
%     \item[--] ułamek --~\verb+\frac{ 1 }{ 2 }+ = $\fr{ 1 }{ 2 }$;
%     \item[--] pochodna cząstkowa --~\verb+\partial+ = $\partial$.
%     \end{itemize}
%   \end{block}

% \end{frame}



% \begin{frame}[fragile]
%   \frametitle{Tryb matematyczny}

%   \begin{block}{Musimy napisać}
% \begin{verbatim}
% \begin{equation}
%   \label{eq:1}
%   \frac{ \partial^{ 2 } u( x, t ) }{ \partial x^{ 2 } }
%   - \frac{ 1 }{ c^{ 2 } } \frac{ \partial^{ 2 } u( x, t ) }
%   { \partial t^{ 2 } }  = 0
% \end{equation}
% \end{verbatim}
%     Większość białych znaków jest tu niepotrzebna, dodałem je dla
%     własnej wygody. Tak samo podział na linie. W~trybie
%     matematycznym {\color{red} nie wolno} zostawiać pustych linii!!!
%   \end{block}

% \end{frame}



% \begin{frame}[fragile]
%   \frametitle{Tryb matematyczny}

%   \begin{block}{Jeśli mamy błąd kompilacji i~nie wiesz jaki}
%     \begin{enumerate}
%     \item Kompilator podaje nam linię w~której jest błąd, ale
%       często~się myli o~jakieś 10~linii. Jeśli linie~są relatywnie
%       krótkie, to~jest większa szansa, że~poda poprawną.
%     \item Odkomentowujemy tekst, gdzie kompilator wskazuje błąd.
%     \item Kompilujemy jeszcze raz. Jeśli błąd jest wciąż obecny, to
%       znaczy, że~błędów jest więcej albo ~zakomentowaliśmy zły
%       fragment.
%     \item Powtarzamy krok 2, aż~kompilator zadziała.
%     \item Przyglądamy~się krótkiemu fragmentowi zakomentowanego
%       tekstu, jeśli~znajdziemy w~nim błąd to go poprawiamy, jeśli
%       nie to go~odkomentowujemy.
%     \item Jeśli zadziałał, to powtarzamy krok 5, jeśli nie, to
%       znaczy, że~przeoczyliśmy błąd i~musimy dokładnie
%       przeanalizować odkomentowany fragment.
%     \item Stosujemy tę procedurę, aż~wszystko działa.
%     \end{enumerate}
%   \end{block}

% \end{frame}



% \begin{frame}[fragile]
%   \frametitle{Uprośćmy sobie życie}

%   \begin{block}{Dlaczego warto skracać linie?}
%     Relatywnie krótkie linie w~pliku źródłowym pozwalają pracować na
%     małych fragmentach tekstu, a~znacznie łatwiej znaleźć błąd
%     w~krótkiej części tekstu niż w~długiej.
%   \end{block}
  
%   \begin{block}{Przy odrobinie sprytu życie jest prostsze}
%     Chcemy mieć % (trochę to brzydkie, ale~kod prosty)
%     \begin{equation}
%       \label{eq:2}
%       \eps > 0, \veps > 0.
%     \end{equation}
%     Aby to uzyskać trzeba napisać
% \begin{verbatim}
% \begin{equation}
%   \label{eq:2}
%   \epsilon > 0, \varepsilon > 0.
% \end{equation}
% \end{verbatim}
%     Nie ma potrzeby tyle pisać.
%   \end{block}
    
% \end{frame}



% \begin{frame}[fragile]
%   \frametitle{Uprośćmy sobie życie}

%   \begin{block}{\tbs newcommand}
%     Wstawiamy do preambuły, najlepiej po ostatnim
%     \verb+\usepackage+, linię
% \begin{verbatim}
% \newcommand{\eps}{\epsilon}
% \end{verbatim}
%     Jeżeli dobrze rozumiem, a~to rozumowanie jeszcze mnie nie
%     zawiodło, urządzenie zwane preprocesorem zamieni w~tekście ,,na
%     chama'' każde wystąpienie \verb+\eps+ na \verb+\epsilon+.
%     Analogicznie
% \begin{verbatim}
% \newcommand{\veps}{\varepsilon}
% \end{verbatim}
%     Od razu lepiej:).
%   \end{block}

% \end{frame}



% \begin{frame}[fragile]
%   \frametitle{Uprośćmy sobie życie}

%   \begin{block}{Pochodna cząstkowa}
%     Wstawiamy do~preambuły
% \begin{verbatim}
% \newcommand{\pd}[3]{\frac{ \partial^{ #1 } { #2 } }
% { \partial { #3 }^{ #1 } }}
% \end{verbatim}
%     Lepiej nie rozdzielać, choć można, tej komendy na dwie linie jak
%     powyżej, zrobiłem to, aby tekst ładnie mieścił~się na~slajdzie.
%     Nawiasy wąsate wokół ,,\#liczba'', nie~są konieczne, ale~bez
%     nich \LaTeX{} czasem protestuje.
%   \end{block}

%   \begin{block}{Piszemy}
% \begin{verbatim}
% \pd{ 2 }{ u( x, t ) }{ x } 
% - \frac{ 1 }{ c^{ 2 } } \pd{ 2 }{ u( x, t ) }{ t } = 0
% \end{verbatim}
%     Która wersja jest prostsza?
%   \end{block}
  
% \end{frame}



% \begin{frame}[fragile]
%   \frametitle{Uprośćmy sobie życie}

%   \begin{block}{Objaśnienie}
% \begin{verbatim}
% \newcommand{\nazwa-komendy}[ilość-argumentów]
% {treść komendy #1 #2 #3,...}
% \end{verbatim}
%     Nazwa komendy to chyba jasne. Jeśli komenda ma nie przyjmować
%     żadnych argumentów, jak komenda \verb+\LaTeX+, to pomijamy
%     nawias kwadratowy. Jeśli ma przyjmować np.~5 argumentów, to
%     piszemy [5]. W~treści komendy \LaTeX{} podstawi ,,na chama''
%     pierwszy argument za ,,\#1'', drugi za~,,\#2'', etc.
%   \end{block}

%   \begin{block}{}
%     Tutaj też lepiej nie rozbijać \verb+\newcommand+ na~dwie linie.
%   \end{block}

%   \begin{block}{Użycie komendy}
% \begin{verbatim}
% \nazwa-komendy{pierwszy-argument}{drugi-argument}...
% \end{verbatim}
%   \end{block}

% \end{frame}



% \begin{frame}[fragile]
%   \frametitle{Uprośćmy sobie życie}

%   \begin{block}{Czy rozumiecie już jak działa}
% \begin{verbatim}
% \newcommand{\pd}[3]{\frac{ \partial^{ #1 } { #2 } }
% { \partial { #3 }^{ #1 } } }
% \pd{ 2 }{ u( x, t ) }{ x } 
% \end{verbatim}
%     ta komenda?
%   \end{block}

%   \begin{block}{}
%     {\color{red} Ważne}, \verb+\pd{ }{ u( x, t ) }{ x }+ też działa.
%   \end{block}

%   \begin{block}{Inny przykład}
%     \verb+\newcommand{\sizeOne}{8pt}+ --~definicja rozmiaru
%     czcionki.
%   \end{block}

%   \begin{block}{Osobiste doświadczenie}
%     Potrafię mieć 40 \verb+\newcommand+ w~jednym pliku źródłowym,
%     tak bardzo upraszczają mi życie. Np.~jeśli chcę mieć znak $\ra$,
%     to po co mam pisać \verb+\rightarrow+, kiedy mogę \verb+\ra+?
%   \end{block}
  
% \end{frame}



% \begin{frame}
%   \frametitle{Nie pytaj co możesz zrobić dla swojego środowiska}

%   \begin{block}{To środowisko ma zrobić coś dla ciebie :)}
%     Podstawowe skróty i~polecenia, na przykładzie \TeX Makera.
%     \begin{itemize}
%     \item[--] Ctrl-T --~zakomentuj blok tekstu;
%     \item[--] Ctrl-U --~odkomentuj blok tekstu;
%     \item[--] Ctrl-$>$ - wcięcie bloku;
%     \item[--] Ctrl-$<$ - usunięcie wcięcia bloku;
%     \item[--] Ctrl-R - zastąp;
%     \item[--] Ctrl-F - znajdź;
%     \item[--] Ctrl-M - znajdź następny;
%     \item[--] Ctrl-G - przejdź do linii;
%     \item[--] Narzędzia $\ra$ Wyczyść.
%     \end{itemize}
%     Sprawdź jakie są w~twoim i~ich używaj :).
%   \end{block}
  
% \end{frame}


% \begin{frame}
%   \frametitle{Co warto zrobić dalej?}

%   \begin{block}{Rady}
%     \begin{itemize}
%     \item Posłuchać następnych wystąpień :).
%     \item Przeczytać \emph{Nie za~krótkie wprowadzenie do~systemu
%       \LaTeX a}.
%     \item Nauczyć~się Bib\TeX a (chyba, że~jest już jakieś lepsze
%       rozwiązanie).
%     \item Pogooglować i~poeksperymentować.
%     \end{itemize}
%   \end{block}

% \end{frame}


\begin{frame}
  \frametitle{Koniec}

  \begin{center}
    {\LARGE Dziękuję}
  \end{center}
  
\end{frame}



\begin{frame}
  \frametitle{Literatura}

  \begin{block}{Podstawy uczynienia nierelatywistycznej mechaniki
      kwantowej sensowną}
    \begin{itemize}
    \item[--] M. Reed, B. Simon, \emph{Methods of Modern Mathematical
        Physics. \\
        Vol.~I: Functional Analysis}.
    \item[--] M. Reed, B. Simon, \emph{Methods of Modern Mathematical
        Physics. \\
        Vol.~II: Fourier Analysis, Self-Adjointness}.
    \item[--] M. Reed, B. Simon, \emph{Methods of Modern Mathematical
        Physics. \\
        Vol.~III: Scattering Theory}. Książka już przestarzała, należy
      ją uzupełnić, np.~następną pozycją.
    \item[--] J. Dereziński, Ch. Gerard, \emph{Scattering theory~of
        classical and quantum N-particle systems},
      \colorlink{http://www.fuw.edu.pl/~derezins/}.
    \item[--] M. Reed, B. Simon, \emph{Methods of Modern Mathematical
        Physics. \\
        Vol.~IV: Analysis of Operators}. Dr~Mariusz Hynek nazwał tę
      pozycję ,,najlepszą książką popularnonaukową do~analizy
      funkcjonalnej''.
    \end{itemize}
  \end{block}

\end{frame}



\begin{frame}
  \frametitle{Literatura}

  \begin{block}{Podstawy uczynienia nierelatywistycznej mechaniki
      kwantowej sensowną}
    \begin{itemize}
    \item[--] M. Reed, B. Simon, \emph{Methods of Modern Mathematical
        Physics. \\
        Vol.~IV: Analysis~of Operators}. Dr Mariusz Hynek nazwał tę
      pozycję ,,najlepszą książką popularnonaukową do analizy
      funkcjonalnej''.
    \item[--] M. Grabowksi, R. S. Ingarden, \emph{Mechanika kwantowa.
        Ujęcie w~przestrzeni Hilberta}. W~dużej mierze to tłumaczenie
      wybranych fragmentów Reeda, Simona na~polski.
    \end{itemize}
  \end{block}

  \begin{block}{Literatura dodatkowa}
    \begin{itemize}
    \item[--] Andrzej Herdegen, Piotr Ziobro, \emph{Generalized
        uncertainty relations},
      \colorlink{https://arxiv.org/abs/1511.06589}.
    \item[--] F.~Gieres, \emph{Mathematical surprises and Dirac's
        formalism in quantum mechanics},
      \colorlink{https://arxiv.org/abs/quant-ph/9907069}.
    \item[--] Walter Thirring, \emph{Fizyka matematyczna. Tom 3:
        Mechanika kwantowa atomów i~cząstek}.
    \item[--] Walter Thirring, \emph{Fizyka matematyczna. Tom 4:
        Mechanika kwantowa wielkich układów}.
    \end{itemize}
  \end{block}
  
\end{frame}



\end{document}