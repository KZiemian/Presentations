\documentclass{beamer}
\mode<presentation>
%\usepackage{beamerthemesplit}
\usepackage{epsfig}
\usetheme{Warsaw}
%\usecolortheme{albatross}
%\usecolortheme{beetle}
%\usecolortheme{crane}
%\usecolortheme{dolphin}
%\usecolortheme{dove}
%\usecolortheme{fly}
%\usecolortheme{seagull}
%\usecolortheme{lily}
\usecolortheme{orchid}
%\usecolortheme{whale}
%\usecolortheme{seahorse}
\usepackage[]{graphicx}
\usepackage{tensor}
\usepackage{subfigure}%Pozwala dzielić figury na podfigury.
\usepackage[polish]{babel}%Tłumaczy na polski teksty automatyczne LaTeXa i pomaga z typografią.
\usepackage{amsfonts}%Czcionki matematyczne od American Mathematic Society.
\usepackage{amsmath}%Dalsze wsparcie od AMS. Więc tego, co najlepsze w LaTeX, czyli trybu matematycznego.
\usepackage[plmath,OT4,MeX]{polski}%Polska notacja we wzorach matematycznych. Ładne polskie czcionki i więcej cudzysłowów. Pełna polonizacja.
\usepackage[utf8]{inputenc}%Pozwala pisać polskie znaki bezpośrednio.
\usepackage{latexsym}%
\usepackage{indentfirst}%Sprawia, że jest wcięcie w pierwszym akapicie.
\usepackage{textcomp}%Pakiet z dziwnymi symbolami.
\usepackage{xy}%Pozwala rysować grafy.
\frenchspacing%Wyłącza duże odstępy na końcu zdania. Podobno pakiet polski robi to samo, ale zostawić nie zaszkodzi.


\title[Estetyka edytowanego tekstu]{Estetyka edytowanego tekstu}
%\subtitle{Czyli fizyka matematyczna spotyka związki komutacji}
\author{Kamil Ziemian} 
%\\ \texttt{ziemniakzkosmosu@gmail.com}}
%\institute{II rok, fizyka teoretyczna, studia magisterskie.}
\date[15.12.2012]{15 grudnia 2012 r.}



\begin{document}

\begin{frame}
\titlepage
\end{frame}

\begin{frame}
\frametitle{Plan}
\tableofcontents
\end{frame}



\section{Wstęp}



\begin{frame}
\frametitle{Jak już wiecie, \LaTeX { to}}
\pause


\begin{block}{}
Wspaniałe narzędzie do pisania tekstu, wzorów, tworzenia prezentacji et ct.,
które potrafi świetnie zarządzać tekstem \linebreak bez bezpośredniej ingerencji piszącego.
\end{block}
\pause

\begin{block}{Uwaga}
\pause
Co nie \newline
znaczy, że nie da się z niego źle \linebreak korzystać.
\pause
I tworzyć teksty jakich się nie powinno w ogóle pisać, a tym bardziej przedstawiać.
\end{block}
\pause

\begin{block}
Trzeba też pamiętać o drobnych pułapkach, jakie \LaTeX zastawia.
\end{block}

\end{frame}



\section{Podstawowe uwagi i wskazówki}

\begin{frame}
  \frametitle{Akapity}


\begin{block}{Akapity}
  Jedną z podstawowych jednostek tekstu jest akapit. Według zasad
  układu tekstu akapit powinien być wyznaczony przez jedną główną
  myśl, bądź w przypadku prozy zwartą część narracji.
\end{block}
\pause

\begin{block}{Możliwe błędy}
  Nie należy więc jednej większej myśli.
\end{block}
\pause

\begin{block}{Możliwe błędy}
  Przedstawiać przez dwa akapity.
\end{block}

\begin{block}{}
  Ani poświęcać jednego akapitu dwóm lub więcej zagadnieniom.
\end{block}

\end{frame}



\begin{frame}
  \frametitle{Zakończenia lini}


\begin{block}{}
  \LaTeX{ } z reguły świetnie organizuje tekst, który się do niego
  wpisuje, lecz potrafi mieć problem z pewnymi zwrotami, które nie
  powinny kończyć wierszy.

\end{block}
\pause

\begin{block}{Przykłady}
  W języku polskim są to w szczególnosci ,,w'', ,,z'', ,,a'', ,,i'',
  ,,że'', \linebreak et ct. Wynika to z tego, że nie posiadają one
  sensu jako autonomiczne wyrażenia, lecz są uzależnione od słów po
  nich następujących.
\end{block}


\end{frame}



\begin{frame}
  \frametitle{Zakończenia lini}
  \pause

\begin{block}{Rozwiązania}
  Można się przed tym zabezpieczyć używając twardych spacji,\linebreak
  bądź opcji \textbackslash mbox\{\}, aby związać dane wyrażenie z
  występującym po nim słowem.
\end{block}
\pause

\begin{block}{}
  Jeżeli zdarzy się, że nie da to dobrego rezultatu, to można zawsze
  użyć opcji \textbackslash newline i \textbackslash linebreak.
\end{block}


\end{frame}



\begin{frame}
  \frametitle{Dzielenia zdań}
  \pause

\begin{block}{Problem}
  Jeżeli wiersz kończy się ,,.'' lub ,, , '' , to zdanie zostało
  poprawnie podzielone. W innych przypadkach należy jednak uważać.
\end{block}
\pause

\begin{block}{}
  Należy unikać sytuacji, gdy w danym wersie znajduje się tylko
  początkowe słowo danego zdania. Nie jest to jednak reguła. \pause
  Podobnie jest z pojedynczym wyrazem kończącym zdanie.
\end{block}
\pause

\begin{block}{}
  Bezwzględnie należy za to unikać, aby w wersie znajdował się
  pojedynczy symbol.
\end{block}


\end{frame}



\section{O pisaniu samego ,,kodu''}


\begin{frame}{O pisaniu samego ,,kodu''}
\pause

\begin{block}{Na pierwszy rzut oka}
Dla osoby pierwszy raz stykającej się z \LaTeX em, może on przywieść na myśl pisanie kodu w jakimś programie. Jest to
prawdą o tyle,\linebreak że tekst wpisany w \LaTeX u może się wskutek błędów nie skompilować, a szukanie błędów może być
uciążliwe. 
\end{block}
\pause

\begin{block}{}
Dlatego myślę, że warto rozważyć, jak w programowaniu, pewne standardy pracy.
\end{block}

\end{frame}



\begin{frame}{O pisaniu samego ,,kodu''}
\pause

\begin{block}{Moje propozycje}

\begin{itemize}
\pause
\item[--] \LaTeX { }ignoruje większość białych znaków.
\pause
\item[--] Korzystaj z tego i nie twórz zwartych bloków tekstu.
\pause
\item[--] Staraj się z ich pomocą wyróżniać każdą logiczną jednostkę tekstu.
\pause
\item[--] Wzory matematyczne potrafią być prawdziwą kopalnią błędów, zarówno kompilacji, jak i rzeczowych. 
\pause
\item[--]Staraj się je zaznaczać w tekście i nie pozwól, by stały się tylko gęstym zbiorem znaków.
\pause
\item[--] Jakkolwiek to jest bardzo trudne, postaraj się choć raz przeczytać je ze zrozumieniem.
\end{itemize}

\end{block}

\end{frame}
%\begin{frame}{Przestrzenie liniowo - topologiczne}
%
%
%\begin{block}{Przykłady}
%\begin{itemize}
%\pause
%\item[--]Zbiór wielomianów określonych na odcinku $[0, 1]$ z normą $||f||_{ 0 } = \sup_{ x \in [0, 1] } |f(x)| $.
%\pause
%\item[--]Zbiór funkcji ciągłych określonych na odcinku $[0, 1]$ z normą $||f||_{ 1 } = \sup_{ x \in [0, 1] } |f(x)| $.
%Przestrzeń tą oznaczamy $\mathcal{ C } [0, 1]$.
%\pause
%\item[--]Przestrzeń $L^{ 2 }( \mathbb{ R }^{ n }, dx) $. Jest to najczęściej używana przestrzeń w nierelatywistycznej mechanice kwantowej.
%\end{itemize}
%
%\end{block}
%
%\end{frame}
%
%
%
%\section[Postulaty MK]{Postulaty mechaniki kwantowej}
%
%
%
%\begin{frame}
%\frametitle{Postulaty mechaniki kwantowej}
%\pause
%
%
%\begin{block}{}
%Postulaty te pochodzą z książki Stevena Weinberga ,,Teoria pól kwantowych. Tom I Podstawy.''
%\end{block}
%\pause
%
%\begin{block}{Postulat I}
%Stany fizyczne są reprezentowane przez promienie w przestrzeni Hilberta.
%\end{block}
%\pause
%
%\begin{block}{Postulat II}
%Obserwable są reprezentowane przez operatory samosprzężone. Stan reprezentowany przez promień $\mathcal{ R }(\psi)$ ma określoną wartość $\alpha$ oberwabli reprezentowanej przez operator $A$, jeżeli 
%$$A \psi = \alpha \psi \textrm{.}$$
%\end{block}
%
%\end{frame}
%
%
%\begin{frame}
%\frametitle{Postulaty mechaniki kwantowej}
%
%
%\begin{block}{Postulat III}
%Jeśli układ jest w stanie reprezentowanym przez promień $\mathcal{ R }$ i przeprowadzane jest doświadczenie sprawdzające, czy jest on w jednym ze stanów reprezentowanych przez wzajemnie ortogonalne promienie $\mathcal{ R }_{ 1 }, \mathcal{ R }_{ 2 }, \ldots$, to prawdopodobieństwo znalezienia go w stanie reprezentowanym przez $\mathcal{ R }_{ n }$ wynosi:
%$$\mathcal{ P }(\mathcal{ R } \rightarrow \mathcal{ R }_{ n }) = |(\psi, \psi_{ n })|^{ 2 } \textrm{.}$$
%\end{block}
%
%\end{frame}
%
%
%
%\section[FM a MK]{Fizyka matematyczna a mechanika kwantowa}
%
%
%
%\begin{frame}
%\frametitle{Fizyka matematyczna a mechanika kwantowa}
%
%
%\begin{block}{Pytanie}
%Co można zakwestionować w tych postulatach?
%\end{block}
%\pause
%\begin{block}{Z punktu widzenia matematyki\ldots}
%\begin{itemize}
%\pause
%\item[--]\ldots czy przestrzeń Hilberta ma jakąś dodatkową strukturę?
%\pause
%\item[--]\ldots jakimi dokładnie operatorami są obserwable?
%\end{itemize}
%
%\end{block}
%
%\end{frame}
%
%
%\begin{frame}
%\frametitle{Operatory samosprzężone}
%
%
%\begin{block}{}
%W przestrzeni Hilberta istnieją trzy podstawowe klasy operatorów, mających diametralnie różne własności. Musimy więc ustalić, do której z tych klas należą operatory reprezentujące oberwable.
%\end{block}
%\pause
%
%\begin{block}{Klasy operatorów}
%\begin{itemize}
%\item[--]Operatory zwarte.
%\item[--]Operatory ograniczone.
%\item[--]Operatory nieograniczone.
%\end{itemize}
%
%\end{block}
%
%\end{frame}
%
%
%\begin{frame}{Operatory samosprzężone}
%
%
%\begin{block}{Uwagii}
%\begin{itemize}
%\pause
%\item[--]Dla operatora ograniczonego $A$ można zdefiniować normę operatora $|| A || < \infty$ jako najmniejszą liczbę spełniającą:
%$$|| A x || \leq || A || \; || x ||, \forall x \in \mathcal{ H } \textrm{.}$$
%\pause
%\item[--]Operatory zwarte są ograniczone.
%\pause
%\item[--]Matematycznie trzeba inaczej traktować wartości własne i widmo ciągłe.
%\end{itemize}
%
%\end{block}
%
%\end{frame}
%
%
%\begin{frame}{Operatory zwarte}
%
%
%\begin{block}{Twierdzenie Riesza o rozkładzie spektralnym}
%\pause
%\emph{Widmo samosprzężonego operatora zwartego składa się z samych wartości własnych, zawartych w odcinku $[ -|| A ||, || A ||]$. Zbiór wartośći własnych jest skończony lub można go ustawić w ciąg $\lambda_{ 1 } \geq \lambda_{ 2 } \geq \lambda_{ 3 } \geq \ldots \rightarrow 0$. Ponadto, może z wyjątkiem $\lambda = 0$, wszystkie podprzestrzenie własne mają wymiar skończony.} 
%
%\end{block}
%\end{frame}
%
%
%\begin{frame}
%
%
%\begin{block}{Wnioski}
%\begin{itemize}
%\pause
%\item[--]Operatory zwarte samosprzężone są najbliższymi krewnymi macierzy hermitowskich w przypadku, gdy wymiar $\mathcal{ H }$ jest nieskończony.
%\pause
%\item[--]Działanie operatora zwartego $\mathcal{ L }$ można w poprawnie matematyczny sposób zapisać jako:
%$$\mathcal{ L } | V \rangle = \sum_{ \lambda } \lambda a_{\lambda} | \lambda \rangle \textrm{.}$$
%\pause
%\item[--]Absolutnie nie nadają się jako kandydaci na obeserwable takie jak: $\hat{ P }, \, \hat{ X }, \, a^{ \dagger }, \, a $, et ct.
%\end{itemize}
%
%\end{block}
%\pause
%
%\begin{block}{Dlaczego?}
%\pause
%Ograniczoność widma operatorów zwartych nie zgadza się z dobrze znanymi własnościami tych obserwabli.
%
%\end{block}
%
%\end{frame}
%
%
%\begin{frame}{Czy w takim razie operatory zwarte nie mają tu zastosowań?}
%\pause
%
%
%\begin{block}{Kontrprzykład}
%\pause
%Załóżmy, że nie wiemy w jakim konkretnie stanie znaduje się nasz układ, wiemy jednak, że z prawdopodobieństwem $p_{ i }$ zanjduje się w stanie $\psi_{i}$. Ponieważ $p_{ i }$ są nieujemne i sumowalne do jedności, operator dany wzorem
%$$\rho = \sum_{ i = 0 }^{ \infty } p_ { i }\; (\psi, \cdot) \psi$$
%jest poprawnie określony i jest operatorem zwartym.
%\end{block}
%\pause
%
%\begin{block}{}
%Z tego względu teoria operatorów okazała się wartościowym podejściem do problemów kwantowej fizyki statystycznej.
%\end{block}
%
%\end{frame}
%
%\begin{frame}{Operatory nieograniczone}
%\pause
%
%
%\begin{block}{}
%Ponieważ dla ograniczonych operatorów samosprzężonych istnieje analogiczne twierdzenie o lokalizacji widma na odcinku $[ -|| A ||, || A ||]$, pozostaje nam zająć sie operatorami nieograniczonymi.
%\end{block}
%\pause
%
%\begin{block}{Rozwiązanie}
%Rzeczywiście okazuje się, że większość operatorów znanych z mechaniki kwantowej, takich jak $\hat{ P }, \, \hat{ X }, \, a^{ \dagger }, \, a $ et ct., okazuje się być samosprzężonymi operatorami nieograniczonymi.
%\end{block}
%\pause
%
%\begin{block}{}
%Z punktu widzenia fizyki matematycznej to krótkie stwierdzenie zawiera w sobie ogromną ilość skomplikowanej matematyki i nietrywialnych/ciekawych/irytujących problemów do rozwiązania.
%\end{block}
%
%\end{frame}
%
%
%\begin{frame}{Źródło problemów}
%\pause
%
%
%\begin{block}{Twierdzenie Hellingera - Toeplitza}
%Operator samosprzężony określony na całej przestrzeni Hilberta jest ograniczony.
%\end{block}
%\pause
%
%\begin{block}{Konsekwencje}
%Operator reprezentujący najbardziej podstawowe obserwable możemy określić jedynie na podprzestrzeniach wektorowych, które są gęste w $\mathcal{ H }$. Podprzestrzenie takie będziemy oznaczać jako $\mathcal{ D }$, zaś dziedznę obserwabli $A$ przez $\mathcal{ D }( A )$.
%\end{block}
%
%\begin{block}{Własności gęstych poprzestrzeni}
%W przetrzeni skończenie wymiarowej takie podprzestrzenie nie istnieją. Gdy zaś istnieją, potrafią mieć bardzo nieoczekiwane i wyjątkowe własności. W szczególności nie można z nich wnioskować o własnościa całej przestrzeni.
%\end{block}
%
%\end{frame}
%
%
%\begin{frame}{Podprzestrzenie gęste}
%\pause
%
%
%\begin{block}{Własności gęstych poprzestrzeni}
%W przetrzeni skończenie wymiarowej takie podprzestrzenie nie istnieją. Gdy zaś istnieją, potrafią mieć bardzo nieoczekiwane i wyjątkowe własności.
%\end{block}
%\pause
%
%\begin{block}{Przykład z przestrzeni Banacha}
%W przestrzeni Banacha $\mathcal{ C } [0, 1]$ z normą $||f|| = \sup_{ x \in [0, 1] } |f(x)| $, wielomiany tworzą podprzestrzeń gęstą na mocy klasycznego twierdzenia Weierstrassa o aproksymacji funkcji ciągłych.
%\end{block}
%\pause
%
%\begin{block}{}
%Z drugiej strony Banach udowodnił, że większość fukcji ciągłych nie posiada pochodnych w żadnym punkcie swojej dziedziny. Widzimy więc, że z faktu, iż możemy jakąś funkcję z dowolną dokładnością przybliżyć funkcją analityczną (wielomianem!), nie wynika nawet jej różniczkowalność.
%\end{block}
%
%\end{frame}
%
%
%\begin{frame}{Problemy w mechanice kwantowej}
%\pause
%
%
%\begin{block}{Podstawowe problemy}
%\begin{itemize}
%\item[--]Korzystając z reguł Borna, nic nie wiem o dziedzinach danych obserwabli.
%\pause
%\item[--]Może się zdarzyć, że $D( A ) \cap D( B ) = {0}$, więc suma dwóch obserwabli $A$ i $B$, może mieć sens tylko w działaniu na wektor zerowy.
%\pause
%\item[--]W szczególności nic nie wiem o tym, na jakim zbiorze mają być spełnione relacje komutacji $[\hat{ X }, \hat{ P }] = i \hbar$. Co więcej, można pokazać, że na odpowiednio sensownej dziedzinie nie istnieją operatory ograniczone realizujące te związki.
%\pause
%\item[--]Widmo takiego operatora zależy od wyboru dziedziny.
%\item[--]Tak samo jak samosprzężoność.
%\end{itemize}
%\end{block}
%
%\end{frame}
%
%
%\begin{frame}
%
%
%\begin{block}{Ciekawsze problemy}
%\begin{itemize}
%\item[--]Załóżmy, że mamy standardowy problem, w którym hamiltonian ma postać $H = H_{ 0 } + V$, gdzie $V$ jest odpowiednią poprawką i znamy $\mathcal{ D }(H_{ 0 })$. Dodanie takiej poprawki może sprawić, że $\mathcal{ D }(H) \neq \mathcal{ D }(H_{ 0 })$.
%\pause
%\item[--]Jeżeli hamiltonian (jak to zwykle bywa) jest nieograniczony, to równanie Schr\"{o}dingera
%$$i \hbar \partial_{ t } | \psi \rangle = H | \psi \rangle$$
%jest dla wielu wektorów po prostu pozbawione sensu.
%\end{itemize}
%\end{block}
%
%\begin{block}{Osiągnięcia}
%W ramach operatorowego podejścia do mechaniki kwantowej wiele z tych problemów udało się w dużym stopniu rozwiązać.
%\end{block}
%
%\end{frame}
%
%
%
%\section[]{Podejście algebraiczne}
%
%
%
%\begin{frame}{Podejście algebraiczne}
%
%
%\begin{block}{Podsumowanie teorii operatorów}
%Zapoczątkowana przez John von Neumanna teoria operatorów nieogranczonych stanowi do dziś najpotężniejsze podejście do uzyskiwania matematycznych rozwiązań problemów mechaniki kwantowej. Z innych jej osiągnięć warto wspomnieć rozwiniętą przez T. Kato teorię zaburzeń.
%\end{block}
%
%\begin{block}{Alternatywne podejście}
%Formalizm ten jest jednak jak widać dość delikatny. Pytanie, czy można znaleźć formalizm, który jest bardziej stabilny.
%Poprawną odpowiedź na to pytanie zapoczątkował pomysł wielkiego XX w. matematyka H. Weyla.
%\end{block}
%
%\end{frame}
%
%\begin{frame}{Podejście algebraiczne}
%
%
%\begin{block}{Związki Weyla}
%Weyl, wspierając się na poprzednich pracach von Neumanna, zauważył, że korzystając rachunku oparatorowego i funkcji ograniczonej możemy zbudować operatory ograniczone z operatorów $ \hat{ X } $ i $ \hat{ P } $. Korzystając z funkcji ograniczonych $\exp(i p x)$, związki komutacji przechodzą w następujące \emph{związki Weyla} dla operatrów ograniczonych:
%\pause
%$$e^{ i \alpha \hat{ X } } e^{ i \beta \hat{ P } } e^{ -i \alpha \hat{ X } } = e^{ i \beta (\hat{ P } - \alpha) } \textrm{.}$$
%
%\end{block}
%\pause
%
%\begin{block}{}
%Można pokazać, że tak zdefiniowane operatory zwierają tyle samo informacji co pierwotne $\hat{ X }$ i $\hat{ P }$, które można otrzymać z form eksponencjalnych poprzez różniczkowanie.
%\end{block}
%
%\end{frame}
%
%
%\begin{frame}{Podejście algebraiczne}
%
%
%\begin{block}{}
%Sytuacja ta wraz z równoważnością różnych realizacji mechaniki kwantowej, takich jak reprezentacja położeniowa, pędowa, czy obsadzeń w przypadku oscylatora, zasugerowało następujące podejście.
%\end{block}
%\pause
%
%\begin{block}{Algebraiczna mechanika kwantowa}
%Podstawowym obiektem jest odpowiednia algebra $\mathcal{ A }$ (*-algebra\linebreak lub $C^{*}$), stany zaś odpowiadają funkcjonałom liniowym na $\mathcal{ A }$. Przy wybraniu pewnej reprezentacji elementy algebry przechodzą w operatory na przestrzeni Hilbert, a stany w wektory tej przestrzeni. Reprezentacje równoważne odpowiadają tej samej sytuacji fizycznej.
%\end{block}
%
%\end{frame}
%
%
%\begin{frame}{AQFT}
%
%
%\begin{block}{AQFT}
%To podejście zostało zastosowane do QFT m.in. prze Haaga, Kastlera, Wightmana i Gardinga, w rezulatcie czego otrzymano teorie zwaną algebraiczną kwantową teorią pola. 
%\end{block}
%\pause
%
%\begin{block}{}
%Przedstawione tu sformułowanie mechaniki kwantowej jest chyba najbardziej udaną teorią we współczesnej fizyce matematycznej. Jednak w przypadku QFT sytuacja nie jest już tak dobra.
%\end{block}
%\pause
%
%\end{frame}
%
%
%\begin{frame}{Sukcesy i porażki}
%
%
%\begin{block}{Sukcesy i porażki}
%\begin{itemize}
%\pause
%\item[--]Nie istnieje pełne algebraiczne sformułownie jakiejkolwiek rzeczywistej teorii pola.
%\pause
%\item[--]Istnieją modele konkretnych realnych efektów.
%\pause
%\item[--]Bliski związek z teorią grup.
%\pause
%\item[--]Kwantyzacja pola Diraca.
%\pause
%\item[--]Reguły nadwyboru.
%\pause
%\item[--]Brak nieskończoności.
%\pause
%\item[--]Brak nieskończoności.
%\end{itemize}
%\end{block}
%
%\end{frame}
%
%%\begin{frame}{I na ostatek}
%%\begin{block}{Najważniejszy filozof drugiej połowy XX w.}
%%\pause
%%\begin{figure}
%%\centering
%%\includegraphics[height=1.6in, width=1.2in]{JD1}
%%\includegraphics[height=1.6in, width=1.8in]{JD3}
%%\includegraphics[height=1.6in, width=1.2in]{JD2}
%%\caption{Jacques Derrida (1930 - 2004).}
%%\end{figure}
%%\end{block}
%%\end{frame}
%%\begin{thebibliography}{99}
%%\bibitem{bm}[TeorieLiteratury XX wieku]
%%
%%\end{thebibliography}



\end{document}