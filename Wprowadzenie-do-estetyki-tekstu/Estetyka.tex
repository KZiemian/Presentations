% Autor: Kamil Ziemian

% --------------------------------------------------------------------
% Podstawowe ustawienia Beamera i używane pakiety
% --------------------------------------------------------------------
\RequirePackage[l2tabu, orthodox]{nag} % Wykrywa przestarzałe i niewłaściwe
% sposoby używania LaTeXa. Więcej jest w l2tabu English version.

\documentclass{beamer}  % Klasa dokumentu
\mode<presentation>  % Rodzaj tworzonych slajdów Beamera
\usetheme{Warsaw}  % Temat graficzny

\setbeamertemplate{headline}{}  % Usuwa nagłówek
\setbeamersize{text margin left=3mm}  % Wielkość lewego marginesu
\setbeamersize{text margin right=3mm}  % Wielkość prawego marginesu
\setbeamertemplate{navigation symbols}{}  % Usuwa ikony nawigacji w prawym
% dolnym rogu



\usepackage[polish]{babel}  % Tłumaczy na polski teksty automatyczne LaTeXa
% i pomaga z typografią.
\usepackage[MeX]{polski}  % Polonizacja LaTeXa, bez niej będzie pracował
% w języku angielskim.
\usepackage[utf8]{inputenc}  % Włączenie kodowania UTF-8, co daje dostęp
% do polskich znaków.
\usepackage{lmodern}  % Wprowadza fonty Latin Modern.
\usepackage[T1]{fontenc}  % Potrzebne do używania fontów Latin Modern.



% ----------------------------
% Pakiety napisane przez użytkownika.
% Mają być w tym samym katalogu to ten plik .tex
% ----------------------------
\usepackage{latexshortcuts}



% ----------------------------
% Pakiet "hyperref"
% Polecano by umieszczać go na końcu preambuły.
% ----------------------------
\usepackage{hyperref}  % Pozwala tworzyć hiperlinki i zamienia odwołania
% do bibliografii na hiperlinki.





% --------------------------------------------------------------------
\title[Estetyka edytowanego tekstu]{Warsztaty \LaTeX a 2014 \\ Estetyka edytowanego tekstu}
\author{Kamil Ziemian \\
 \texttt{kziemianfvt@gmail.com}}

% \institute{II rok, fizyka teoretyczna, studia magisterskie}

\date[29.11.2014]{29 listopada 2014 r.}
% --------------------------------------------------------------------





% ####################################################################
% Początek dokumentu
\begin{document}
% ####################################################################



% ######################################
\begin{frame}
  \titlepage % Tytuł całego tekstu
\end{frame}
% ######################################

% \begin{frame}
%   \frametitle{Plan}
%   \tableofcontents
% \end{frame}





% ##########
\begin{frame}
  \frametitle{Najważniejszy punkt wystąpienia}

  \begin{block}{}
    Człowiek ma tendencje do~tego by się wstydzić, że~popełnił błąd
    językowy mówiąc rzeczy dobre i~chlubić~się tym, iż~mówi piękną
    polszczyzną o~rzeczach złych. W~to szaleństwo nie~wolno wpaść.
  \end{block}

  \begin{block}{W szczególności}
    \begin{itemize}
    \item[--] Najważniejsza jest wartość merytoryczna tego co
      napisane.
    \item[--] Tekst zrozumiały, to tekst dobrze napisany.
    \item[--] Tekst dobrze napisany, można zwykle wygładzić i~uczynić
      ładniejszym.
    \item[--] Przesadnie dbanie o~estetykę to często wynik tego,
      że~ktoś nie~wie co jest ważne.
    \end{itemize}
  \end{block}

\end{frame}
% ##########



% ##########
\begin{frame}
  \frametitle{Najważniejszy punkt wystąpienia}

  \begin{block}{W tym momencie}
    Powinniście~się zastanowić czy nie warto wyjść.
  \end{block}

\end{frame}
% ##########



% ##########
\begin{frame}
  \frametitle{O czym warto wiedzieć}

  \begin{block}{\LaTeX{} to}
    To narzędzie do pisania książek, artykułów, tworzenia \linebreak
    prezentacji etc., dające prawie nieograniczone możliwości,
    \linebreak mogące przy tym świetnie zarządzać tekstem bez
    ingerencji piszącego.
  \end{block}

  \begin{block}{Jednak}
    Ponieważ nie \\
    widać od razu jaki tekst~się ostatecznie \linebreak otrzyma,
    dowolnie wielki błąd może piszącemu umknąć.
  \end{block}

  \begin{block}
    Trzeba też pamiętać o drobnych pułapkach, jakie \LaTeX{} zastawia.
  \end{block}

\end{frame}
% ##########



% ##########
\begin{frame}
  \frametitle{Podstawowe błędy}

  \begin{block}{Akapit}
    Robimy zostawiając jedną pustą linię.
  \end{block}

  \begin{block}{Ważne}
    Domyślne ustawienia \LaTeX{}~są dostosowane do reguł składu
    anglosaskiego, należy więc je zmienić na stosowane w~Polsce.
  \end{block}

  \begin{block}{Bardzo ważne}
    Istnieją specjalne klasy, która uwzględnia polskie reguły składu:
    \texttt{mwart}, \texttt{mwrep}, \texttt{mwbk}. Informacje o~niej
    można znaleźć na~stronie
    \colorlink{http://www.mimuw.edu.pl/~wolinski/mwcls.html}{http://www.mimuw.edu.pl/~wolinski/mwcls.html}.
  \end{block}

\end{frame}
% ##########



% ##########
\begin{frame}
  \frametitle{Podstawowe błędy}

  \begin{block}{Podstawowe reguły składu}
    \begin{itemize}
    \item[--] Duże marginesy % --~\verb+\usepackage{fullpage}+.
    \item[--] Wcieńcia w~pierwszym akapicie
      % --~\verb+\usepackage{indentfirst}+.
      % \item[--] Rozmiar pewnych odstępów -- \textbackslash
      %   frenchspacing.
    \end{itemize}
  \end{block}

\end{frame}
% ##########



% ##########
\begin{frame}[fragile]
  \frametitle{Podstawowe błędy}

  \begin{block}{Tryb matematyczny}
    \begin{itemize}
    \item[--] Zmienne zawsze należy pisać jako ,,$z$'' (\texttt{\$ z
        \$}), a~nie~,,z''.
    \item[--] Jednostki należy pisać jako ,,$\si{kg}$'' (\tbs
      si\{kg\}), a~nie~\emph{kg}. Aby ta komenda działała należy
      wstawić do preambuły \tbs usepackage\{siunitx\}.
    \item[--] \LaTeX{} daje możliwość definiowania własnych komend:
      \tbs newcommand\{\}\{\}. Jest to szczególnie przydatne w~trybie
      matematycznym. By dostać $\dd{ 2 }{ x }{ t }$ wystarczy napisać
      \linebreak \verb+\dd{ 2 }{ x }{ t }+, a~nie
\begin{verbatim}
 \frac{ \mathrm{d}^{ 2 } x }{ \mathrm{d} t^{ 2 } }.
\end{verbatim}.
    \end{itemize}
  \end{block}

\end{frame}
% ##########



% ##########
\begin{frame}
  \frametitle{Akapity}

  \begin{block}{Akapity}
    Jedną z podstawowych jednostek tekstu jest akapit. Pojedynczy
    akapit powinien być wyznaczony przez jedną zwartą myśl, bądź temat
    który omawia. W~przypadku literatury, przez fragment narracji.
  \end{block}
  \pause

  \begin{block}{Możliwe błędy}
    Nie należy więc jednej konkretnej myśli
  \end{block}
  \pause

  \begin{block}{Możliwe błędy}
    ciągną przez dwa akapity.
  \end{block}

  \begin{block}{}
    Ani wstawiać zbyt wiele do jednego akapitu.
  \end{block}

\end{frame}
% ##########



% ##########
\begin{frame}
  \frametitle{Zakończenie linii}

  \begin{block}{}
    \LaTeX{ } z reguły świetnie organizuje tekst, lecz często źle
    umieszcza ,,w'', ,,z'', ,,a'', ,,i'', ,,że'' etc. Wyrazy te
    nie~powinny się znajdować na końca linii, bo mają one sens tylko
    w~połączeniu ze słowami występującymi za~nimi.
  \end{block}

  \begin{block}{Rozwiązania}
    Należy zawsze wiązać tez słowa z następującymi po nich twardą
    spacją \textasciitilde .
    \begin{itemize}
    \item[--] i tyle.
    \item[--] i\textasciitilde tyle.
    \item[--] \textbackslash mbox\{i tyle\}.
    \end{itemize}
  \end{block}

  \begin{block}{}
    Jeżeli zdarzy się, że nie da to dobrego rezultatu, to można zawsze
    użyć opcji \textbackslash newline i \textbackslash linebreak.
  \end{block}

\end{frame}
% ##########



% ##########
\begin{frame}
  \frametitle{Dzielenia zdań}

  \begin{block}{Zasady}
    \begin{itemize}
    \item[--] Jeżeli wiersz kończy się ,,.'' lub ,, , '' , to zdanie
      zostało poprawnie podzielone. W innych przypadkach warto jednak
      uważać.
    \item[--] Należy unikać sytuacji, gdy w danym wersie znajduje się
      tylko początkowe słowo danego zdania. Nie jest to jednak reguła.
    \item[--] Podobnie jest z pojedynczym wyrazem kończącym zdanie.
    \end{itemize}
  \end{block}
  \pause

  \begin{block}{}
    Poważniejszym problem to wiersz zawierający pojedynczy
    znak. \\
    $a$
  \end{block}

\end{frame}
% ##########


% \section{O pisaniu samego ,,kodu''}

% ##########
\begin{frame}{O pisaniu samego ,,kodu''}

  \begin{block}{Na pierwszy rzut oka}
    Dla osoby pierwszy raz stykającej się z \LaTeX em, może on
    przywieść na myśl pisanie kodu w jakimś języku programowania.
    \linebreak Jest to prawdą o tyle, że jeśli popełnimy błąd plik
    \LaTeX a może się nie skompilować i nie dostaniemy pliku
    z~właściwym tekstem. \linebreak Zaś znalezienie błędu jest zwykle
    dość trudne.
  \end{block}

  \begin{block}{}
    Dlatego myślę, że warto przyjąć, jak w programowaniu, pewne
    standardy pracy.
  \end{block}

\end{frame}
% ##########



% ##########
\begin{frame}[fragile]{Przykłady}

  \begin{block}{Chcemy by było}
    \begin{equation}
      \begin{split}
        (R^{*}_{\xi}\omega)(V) \,\; &= \,\; (i^{-1}_{G}\circ d_{p}(\ad_{\xi^{-1}}\circ\rho_{p}^{-1}))(V^{|}) \\
        &=
        (i^{-1}_{G}\circ(d_{e}\ad_{\xi^{-1}})\circ(d_{e}\rho_{p})^{-1})(V^{|}).
      \end{split}
    \end{equation}
  \end{block}
  \pause

  \begin{block}{A widzimy}
\begin{verbatim}
(R^{*}_{\xi}\omega)(V) \,\; &= \,\; (i^{-1}_{G}
\circ d_{p}(\ad_{\xi^{-1}}\circ\rho_{p}^{-1}))(V^{|}) 
\\ &= (i^{-1}_{G}\circ(d_{e}\ad_{\xi^{-1}})
\circ(d_{e}\rho_{p})^{-1})(V^{|}).
\end{verbatim}
  \end{block}

\end{frame}
% ##########



% ##########
\begin{frame}{O pisaniu samego ,,kodu''}

  \begin{block}{Moje prywatne propozycje}
    \LaTeX{} ignoruje większość białych znaków. Warto z~tego
    skorzystać by stworzyć ,,kod'' który~się tobie dobrze czyta.
  \end{block}

  \begin{block}{Potęga komentarzy}
    \LaTeX{} zwykle informuje nas gdzie jest błąd przerywający
    kompilacje. Jeśli nie jesteś pewien gdzie się konkretnie
    znajduje\linebreak od komentuj podejrzany fragment i zobacz czy
    teraz działa.
  \end{block}

\end{frame}
% ##########


% ##########
\begin{frame}{O pisaniu samego ,,kodu''}

  \begin{block}{O tworzeniu książki z pomocą \LaTeX a}
    Książka w naturalny sposób dzieli się na rozdziały. Dlatego,
    równie naturalne, jest podzielenie samego tekstu na pliki
    zawierające tylko pojedyncze rozdziały i dopiero komputer łączy je
    w jeden plik pdf. Jest to jednak temat na osobne wystąpienie.
  \end{block}

\end{frame}
% ##########



% ##########
\begin{frame}{Na zakończenie}

  \begin{block}{Trzy ważne rzeczy o których nie będzie mowy}
    \begin{itemize}
    \item[--] Gramatyka.
    \item[--] Ortografia.
    \item[--] Interpunkcja.
    \end{itemize}
  \end{block}

  \begin{block}{Trzy prawdy o komputerach}
    \begin{itemize}
    \item[--] Odpowiedź na twój problem jest w internecie.
    \item[--] Często nie da się jej zrozumieć.
    \item[--] Zawsze można poprosić o~pomoc kogoś mądrzejszego.
    \end{itemize}
  \end{block}

\end{frame}
% ##########



% ##########
\begin{frame}{Na zakończenie}

  \begin{block}{Dla zainteresowanych}
    Więcej informacji na temat budowania dobrego tekstu można znaleźć
    w~,,Architekturze książki'' Andrzeja Tomaszewskiego.
    % Chciałbym podziękować Aleksandrze Brambor za uświadomienie mnie
    % o
    % jej istnieniu.
  \end{block}
  \pause

  \begin{center}
    \begin{LARGE}
      DZIĘKUJE.
    \end{LARGE}
  \end{center}

\end{frame}
% ##########





% ####################################################################
% ####################################################################
% Bibliografia
\bibliographystyle{alpha} \bibliography{Bibliography}{}


% ############################

% Koniec dokumentu
\end{document}
