\documentclass{beamer}
\mode<presentation>
%\usepackage{beamerthemesplit}
\usepackage{epsfig}
\usetheme{Warsaw}
%\usecolortheme{albatross}
%\usecolortheme{beetle}
%\usecolortheme{crane}
%\usecolortheme{dolphin}
%\usecolortheme{dove}
%\usecolortheme{fly}
%\usecolortheme{seagull}
%\usecolortheme{lily}
\usecolortheme{orchid}
%\usecolortheme{whale}
%\usecolortheme{seahorse}
\usepackage[]{graphicx}
\usepackage{tensor}
\usepackage{subfigure}%Pozwala dzielić figury na podfigury.
\usepackage[polish]{babel}%Tłumaczy na polski teksty automatyczne LaTeXa i pomaga z typografią.
\usepackage{amsfonts}%Czcionki matematyczne od American Mathematic Society.
\usepackage{amsmath}%Dalsze wsparcie od AMS. Więc tego, co najlepsze w LaTeX, czyli trybu matematycznego.
\usepackage[plmath,OT4,MeX]{polski}%Polska notacja we wzorach matematycznych. Ładne polskie czcionki i więcej cudzysłowów. Pełna polonizacja.
\usepackage[utf8]{inputenc}%Pozwala pisać polskie znaki bezpośrednio.
\usepackage{latexsym}%
\usepackage{indentfirst}%Sprawia, że jest wcięcie w pierwszym akapicie.
\usepackage{textcomp}%Pakiet z dziwnymi symbolami.
\usepackage{xy}%Pozwala rysować grafy.
\newcommand{\sgm}{\textrm{sgm}}
\newcommand{\ad}{\textrm{ad}}
\newcommand{\Ad}{\textrm{Ad}}
\newcommand{\cy}{\textrm{cykl}}
\frenchspacing%Wyłącza duże odstępy na końcu zdania. Podobno pakiet polski robi to samo, ale zostawić nie zaszkodzi.


\title[Estetyka edytowanego tekstu]{Estetyka edytowanego tekstu}
%\subtitle{Czyli fizyka matematyczna spotyka związki komutacji}
\author{Kamil Ziemian \newline $\langle N | \widehat{ K } | F \rangle$} 
%\\ \texttt{ziemniakzkosmosu@gmail.com}}
%\institute{II rok, fizyka teoretyczna, studia magisterskie.}
\date[15.12.2012]{15 grudnia 2012 r.}



\begin{document}

\begin{frame}
\titlepage
\end{frame}

\begin{frame}
\frametitle{Plan}
\tableofcontents
\end{frame}



\section{Wstęp}



\begin{frame}
\frametitle{Jak już wiecie, \LaTeX { to}}
\pause


\begin{block}{}
Wspaniałe narzędzie do pisania tekstu, wzorów, tworzenia prezentacji et ct.,
które potrafi świetnie zarządzać tekstem \linebreak bez bezpośredniej ingerencji piszącego.
\end{block}
\pause

\begin{block}{Uwaga}
\pause
Co nie \newline
znaczy, że nie da się z niego źle \linebreak korzystać.
\pause
I tworzyć teksty jakich się nie powinno w ogóle pisać, a tym bardziej przedstawiać.
\end{block}
\pause

\begin{block}
Trzeba też pamiętać o drobnych pułapkach, jakie \LaTeX zastawia.
\end{block}

\end{frame}



\section{Podstawowe uwagi i wskazówki}

\begin{frame}
\frametitle{Akapity}


\begin{block}{Akapity}
Jedną z podstawowych jednostek tekstu jest akapit. Według zasad układu tekstu akapit powinien być wyznaczony przez jedną
główną myśl, bądź w przypadku prozy zwartą część narracji. 
\end{block}
\pause

\begin{block}{Możliwe błędy}
Nie należy więc jednej większej myśli.
\end{block}
\pause

\begin{block}{Możliwe błędy}
Przedstawiać przez dwa akapity.
\end{block}

\begin{block}{}
Ani poświęcać jednego akapitu dwóm lub więcej zagadnieniom.
\end{block}

\end{frame}



\begin{frame}
\frametitle{Zakończenia lini}


\begin{block}{}
\LaTeX{ } z reguły świetnie organizuje tekst, który się do niego wpisuje, lecz
potrafi mieć problem z pewnymi zwrotami, które nie powinny kończyć wierszy.

\end{block}
\pause

\begin{block}{Przykłady}
W języku polskim są to w szczególnosci ,,w'', ,,z'', ,,a'', ,,i'', ,,że'', \linebreak et ct. Wynika to z tego, że nie posiadają one sensu jako autonomiczne wyrażenia, lecz są uzależnione od słów po nich następujących.
\end{block}


\end{frame}



\begin{frame}
\frametitle{Zakończenia lini}
\pause

\begin{block}{Rozwiązania}
Można się przed tym zabezpieczyć używając twardych spacji,\linebreak bądź opcji \textbackslash mbox\{\}, aby związać dane wyrażenie z występującym po nim słowem.
\end{block}
\pause

\begin{block}{}
Jeżeli zdarzy się, że nie da to dobrego rezultatu, to można zawsze użyć opcji \textbackslash newline i \textbackslash linebreak.
\end{block}


\end{frame}



\begin{frame}
\frametitle{Dzielenia zdań}
\pause

\begin{block}{Problem}
Jeżeli wiersz kończy się ,,.'' lub ,, , '' , to zdanie zostało poprawnie podzielone. W innych przypadkach należy jednak uważać.
\end{block}
\pause

\begin{block}{}
Należy unikać sytuacji, gdy w danym wersie znajduje się tylko początkowe słowo danego zdania. Nie jest to jednak reguła.
\pause
Podobnie jest z pojedynczym wyrazem kończącym zdanie.
\end{block}
\pause

\begin{block}{}
Bezwzględnie należy za to unikać, aby w wersie znajdował się pojedynczy symbol.
\end{block}


\end{frame}



\section{O pisaniu samego ,,kodu''}


\begin{frame}{O pisaniu samego ,,kodu''}
\pause

\begin{block}{Na pierwszy rzut oka}
Dla osoby pierwszy raz stykającej się z \LaTeX em, może on przywieść na myśl pisanie kodu w jakimś programie. Jest to
prawdą o tyle,\linebreak że tekst wpisany w \LaTeX u może się wskutek błędów nie skompilować, a szukanie błędów może być
uciążliwe. 
\end{block}

\begin{block}{}
Dlatego myślę, że warto rozważyć, jak w programowaniu, pewne standardy pracy.
\end{block}

\end{frame}



\begin{frame}{O pisaniu samego ,,kodu''}
\pause

\begin{block}{Moje prywatne propozycje}

\begin{itemize}
\pause
\item[--] \LaTeX { }ignoruje większość białych znaków.
\pause
\item[--] Korzystaj z tego i nie twórz zwartych bloków tekstu.
\pause
\item[--] Staraj się z ich pomocą wyróżniać każdą logiczną jednostkę tekstu.
\pause
\item[--] Wzory matematyczne potrafią być prawdziwą kopalnią błędów, zarówno kompilacji jak i rzeczowych. 
\pause
\item[--]Staraj się je zaznaczać w tekście i nie pozwól, by stały się tylko gęstym zbiorem znaków.
\pause
\item[--] Jakkolwiek to jest bardzo trudne, postaraj się choć raz przeczytać je ze zrozumieniem.
\end{itemize}

\end{block}

\end{frame}



%\begin{frame}{Przykład}
%
%
%\begin{block}{Takie równanie z geometrii}
%\begin{*equation}
%\begin{split}
%(R^{*}_{\xi}\omega)(V) \,\; &= \,\; (i^{-1}_{G}\circ d_{p}(\ad_{\xi^{-1}}\circ\rho_{p}^{-1}))(V^{|})\textrm{.}
%\end{split}
%\end{equation}
%\end{block}
%
%\end{frame}

\begin{frame}{Trzy rzeczy na zakończenie}


\begin{block}{których sam nie umiem}
\begin{itemize}
\item[--] Gramatyka.
\item[--] Ortografia.
\item[--] Interpunkcja.
\end{itemize}
\end{block}

\end{frame}


\end{document}