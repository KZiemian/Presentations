\documentclass{beamer} \mode<presentation>
% \usepackage{beamerthemesplit}
\usepackage{epsfig} 
\usetheme{Warsaw} 
\usecolortheme{orchid}
% \usecolortheme{whale} \usecolortheme{seahorse}
\usepackage{verbatim}
\usepackage{subfigure}%Pozwala dzielić figury na podfigury.
\usepackage[polish]{babel}% Tłumaczy na polski teksty automatyczne LaTeXa i pomaga z typografią.
\usepackage[plmath,OT4,MeX]{polski}% Polska notacja we wzorach matematycznych. Ładne polskie
\usepackage[T1]{fontenc}% Pozwala pisać znaki diakrytyczne z języków innych niż polski.
\usepackage[utf8]{inputenc}% Pozwala pisać polskie znaki bezpośrednio.
\usepackage{indentfirst}% Sprawia, że jest wcięcie w pierwszym akapicie.
\frenchspacing% Wyłącza duże odstępy na końcu zdania. Pakiet babel polski robi to samo, ale to jest zabezpieczenie jakibym chciał przestać go używać.
% \usepackage{fullpage}% Mniejszse marginesy.
\usepackage{amsfonts}% Czcionki matematyczne od American Mathematic Society.
\usepackage{amsmath}% Dalsze wsparcie od AMS. Więc tego, co najlepsze w LaTeX, czyli trybu %matematycznego.
\usepackage{amscd}% Jeszcze wsparcie od AMS.
\usepackage{latexsym}% Więcej symboli.
\usepackage{textcomp}% Pakiet z dziwnymi symbolami.
\usepackage{xy}% Pozwala rysować grafy.
\usepackage{tensor}% Pozwala prosto używać notacji tensorowej. Albo nawet pięknej notacji %tensorowej:).
\usepackage{graphicx}% Pozwala wstawiać grafikę.
% \usepackage{url}% Pozwala pisać ładnie znak ~.
\newcommand{\de}{\mathrm{d}}


\title[Terrarium]{NKFowy Grupa Miłośników Python
  Terrarium} %Spotkanie pierwsze oby nie ostatnie.
\subtitle{Spotkanie pierwsze. Oby nie ostatnie.}  \author{Kamil
  Ziemian}
% \\ \texttt{ziemniakzkosmosu@gmail.com}} \institute{II rok, fizyka
% teoretyczna, studia magisterskie.}
\date[27 III 2014]{27 marca 2014 r.}



\begin{document}

\begin{frame}
\titlepage
\end{frame}

\begin{frame}
\frametitle{Cel i plan}

\begin{block}{Cel spotkań}
Wspólna nauka języka programowania Python i jego zastosowań do rozwiązywania problemów naukowych symbolicznie i numerycznie, analizy i wizualizacji danych etc.
\end{block}

\begin{block}{W planie}
\begin{itemize}
\item[--] Podstawy języka i podstawowe wiadomości z informatyki.
\item[--] IPython.
\item[--] NumPy/SciPy.
\item[--] MathPlotLib.
\item[--] Zadania.
\item[--] Wasze pomysły.
\item[--] Józef Sokołowski.
\end{itemize}
\end{block}
\end{frame}

\begin{frame}
\frametitle{Przywitanie}
\begin{block}{Przywitanie}
\begin{figure}
\centering
\includegraphics[height=1.8in, width=1.3in]{GvR2}
\includegraphics[height=1.8in, width=1.2in]{GvR}
\caption{Guido van Rossum (1956 - ), holender.}
\end{figure}
\end{block}
\end{frame}

\begin{frame}
\frametitle{Dlaczego Python?}

\begin{block}{Python}
\begin{itemize}
\item[--] Jest bardzo prostym językiem:
\item[-] ma dobrą i przejrzystą składnie;
\item[-] zawiera zaawansowane i dobrze dobrane typy danych.
\item[--] Jest interpretowany.
\item[--] Posiada OGROMNY zasób bibliotek.
\item[--] Powszechnie stosowany.
\item[--] Dynamicznie rozwijany.
\item[--] Wieloparadygmatowy.
\end{itemize}
\end{block}

\begin{block}{Problemy}
\begin{itemize}
\item[--] Jest wolny. Najprawdopodobniej.
\end{itemize}
\end{block}
\end{frame}

\begin{frame}
\frametitle{Kilka ważnych informacji}

\begin{block}{Uwagi o Python}
\begin{itemize}
\item[--] Jest dwustandardowy: 2.7x, 3.x (3.3.x). Wersje te nie są wstecznie kompatybilne.
\item[--] ,,Rzeczy niebezpieczne mają być utrudnione, lecz nie zabronione.''
\item[--] Doświadczenie z nauki można łatwo przenieść na wiele innych języków.
\end{itemize}
\end{block}

\begin{block}{Uwagi o Terrarium}
\begin{itemize}
\item[--] Będzie trochę matematyki.
\item[--] Prowadzący spotkania nie jest najlepszy. Za to ma oddane wsparcie.
\end{itemize}
\end{block}
\end{frame}

\begin{frame}
\frametitle{Ilustracja}

\begin{block}{Python}
def factorial(n):\\
\hspace{5 mm}if n > 1:\\
\hspace{10 mm}return n*factorial(n-1)\\
\hspace{5 mm}else:\\
\hspace{10 mm}return 1\\
\vspace{8pt}
print factorial(4)
\end{block}

\end{frame}

\begin{frame}[fragile]
\frametitle{Ilustracja}

\begin{block}{C}
\begin{verbatim}
int factorial(int n)
{
  if (n > 1) return n*factorial(n - 1);
  else return 1;
}

int main()
{
  printf("%i\n", factorial(4));
  return 0;
}
\end{verbatim}
\end{block}

\end{frame}

\begin{frame}[fragile]
\frametitle{Ilustracja}

\begin{block}{Common Lisp}
\begin{verbatim}
(defun factorial (x)
  (if (> x 1)
    (* x (factorial (- x 1)))
    1))

(factorial 4)
\end{verbatim}
\end{block}

\end{frame}

\begin{frame}
\frametitle{Krótka lista innych bibliotek/rozszerzeń}

\begin{block}{}
\begin{itemize}
\item[--] PyFeyn. (Nie udało mi się uruchomić.)
\item[--] Biopython.
\item[--] Astropysics.
\item[--] PsychoPy.
\end{itemize}
\end{block}

\end{frame}

\begin{frame}
\frametitle{Materiały}

\begin{block}{}
\begin{itemize}
\item[--] Oficjalna strona Pythona: http://www.python.org/
\item[--] Stowarzyszenie Polska Grupa Użytkowników Pythona: http://pl.python.org
\item[--] Kurs z MIT ,,Introduction to Computer Science and Programming''. Specjalne podziękowania dla Johna Guttaga, Erica Grimsona i MIT.
\item[--] Zanurkuj w Pythonie.
\item[--] A. B. Downey ,,Think Python'', ,,Think Complexity'' http://greenteapress.com
\end{itemize}
\end{block}

\end{frame}

\begin{frame}
\frametitle{Materiały}

\begin{block}{}
\begin{itemize}
\item[--] M. Newman ,,Computational Physics with Python'' http://www-personal.umich.edu/~mejn/computational-physics/
\item[--] E. Ayars ,,Computational Physics with Python''
\item[--] Oficjalna grupa dyskusyjna: comp.lang.python
\item[--] Cała masa innych.
\end{itemize}
\end{block}

\end{frame}

\begin{frame}
\frametitle{}

\begin{center}
\LARGE{DZIĘKUJE.}
\end{center}

\end{frame}


\end{document}