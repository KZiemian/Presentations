% ---------------------------------------------------------------------
% Basic configuration of Beamera and Jagiellonian
% ---------------------------------------------------------------------
\RequirePackage[l2tabu, orthodox]{nag}



\ifx\PresentationStyle\notset
\def\PresentationStyle{dark}
\fi



\documentclass[10pt,t]{beamer}
\mode<presentation>
\usetheme[style=\PresentationStyle,logoLang=Latin,logoColor=monochromaticJUwhite,JUlogotitle=yes]{jagiellonian}



% ---------------------------------------
% Configuration files of Jagiellonian loceted in catalog preambule
% ---------------------------------------
% Configuration for polish language
% Need description
\usepackage[polish]{babel}
% Need description
\usepackage[MeX]{polski}



% ------------------------------
% Better support of polish chars in technical parts of PDF
% ------------------------------
\hypersetup{pdfencoding=auto,psdextra}

% Package "textpos" give as enviroment "textblock" which is very usefull in
% arranging text on slides.

% This is standard configuration of "textpos"
\usepackage[overlay,absolute]{textpos}

% If you need to see bounds of "textblock's" comment line above and uncomment
% one below.

% Caution! When showboxes option is on significant ammunt of space is add
% to the top of textblock and as such, everyting put in them gone down.
% We need to check how to remove this bug.

% \usepackage[showboxes,overlay,absolute]{textpos}



% Setting scale length for package "textpos"
\setlength{\TPHorizModule}{10mm}
\setlength{\TPVertModule}{\TPHorizModule}


% ---------------------------------------
% TikZ
% ---------------------------------------
% Importing TikZ libraries
\usetikzlibrary{arrows.meta}
\usetikzlibrary{positioning}





% % Configuration package "bm" that need for making bold symbols
% \newcommand{\bmmax}{0}
% \newcommand{\hmmax}{0}
% \usepackage{bm}




% ---------------------------------------
% Packages for scientific texts
% ---------------------------------------
% \let\lll\undefined  % Sometimes you must use this line to allow
% "amsmath" package to works with packages with packages for polish
% languge imported
% /preambul/LanguageSettings/JagiellonianPolishLanguageSettings.tex.
% This comments (probably) removes polish letter Ł.
\usepackage{amsmath}  % Packages from American Mathematical Society (AMS)
\usepackage{amssymb}
\usepackage{amscd}
\usepackage{amsthm}
\usepackage{siunitx}  % Package for typsetting SI units.
\usepackage{upgreek}  % Better looking greek letters.
% Example of using upgreek: pi = \uppi


\usepackage{calrsfs}  % Zmienia czcionkę kaligraficzną w \mathcal
% na ładniejszą. Może w innych miejscach robi to samo, ale o tym nic
% nie wiem.










% ---------------------------------------
% Packages written for lectures "Geometria 3D dla twórców gier wideo"
% ---------------------------------------
% \usepackage{./ProgramowanieSymulacjiFizykiPaczki/ProgramowanieSymulacjiFizyki}
% \usepackage{./ProgramowanieSymulacjiFizykiPaczki/ProgramowanieSymulacjiFizykiIndeksy}
% \usepackage{./ProgramowanieSymulacjiFizykiPaczki/ProgramowanieSymulacjiFizykiTikZStyle}





% !!!!!!!!!!!!!!!!!!!!!!!!!!!!!!
% !!!!!!!!!!!!!!!!!!!!!!!!!!!!!!
% EVIL STUFF
\if\JUlogotitle1
\edef\LogoJUPath{LogoJU_\JUlogoLang/LogoJU_\JUlogoShape_\JUlogoColor.pdf}
\titlegraphic{\hfill\includegraphics[scale=0.22]
{./JagiellonianPictures/\LogoJUPath}}
\fi
% ---------------------------------------
% Commands for handling colors
% ---------------------------------------


% Command for setting normal text color for some text in math modestyle
% Text color depend on used style of Jagiellonian

% Beamer version of command
\newcommand{\TextWithNormalTextColor}[1]{%
  {\color{jNormalTextFGColor}
    \setbeamercolor{math text}{fg=jNormalTextFGColor} {#1}}
}

% Article and similar classes version of command
% \newcommand{\TextWithNormalTextColor}[1]{%
%   {\color{jNormalTextsFGColor} {#1}}
% }



% Beamer version of command
\newcommand{\NormalTextInMathMode}[1]{%
  {\color{jNormalTextFGColor}
    \setbeamercolor{math text}{fg=jNormalTextFGColor} \text{#1}}
}


% Article and similar classes version of command
% \newcommand{\NormalTextInMathMode}[1]{%
%   {\color{jNormalTextsFGColor} \text{#1}}
% }




% Command that sets color of some mathematical text to the same color
% that has normal text in header (?)

% Beamer version of the command
\newcommand{\MathTextFrametitleFGColor}[1]{%
  {\color{jFrametitleFGColor}
    \setbeamercolor{math text}{fg=jFrametitleFGColor} #1}
}

% Article and similar classes version of the command
% \newcommand{\MathTextWhiteColor}[1]{{\color{jFrametitleFGColor} #1}}





% Command for setting color of alert text for some text in math modestyle

% Beamer version of the command
\newcommand{\MathTextAlertColor}[1]{%
  {\color{jOrange} \setbeamercolor{math text}{fg=jOrange} #1}
}

% Article and similar classes version of the command
% \newcommand{\MathTextAlertColor}[1]{{\color{jOrange} #1}}





% Command that allow you to sets chosen color as the color of some text into
% math mode. Due to some nuances in the way that Beamer handle colors
% it not work in all cases. We hope that in the future we will improve it.

% Beamer version of the command
\newcommand{\SetMathTextsColor}[2]{%
  {\color{#1} \setbeamercolor{math text}{fg=#1} #2}
}


% Article and similar classes version of the command
% \newcommand{\SetMathTextColor}[2]{{\color{#1} #2}}










% ---------------------------------------
% Commands for setting background pictures for some slides
% ---------------------------------------
\newcommand{\TitleBackgroundPicture}
{./PresentationPictures/CommonPictures/Cute_dragon_BG_dark.png}
\newcommand{\SectionBackgroundPicture}
{./PresentationPictures/CommonPictures/Cute_dragon_small_BG_light.png}



\newcommand{\TitleSlideWithPicture}{
  \begingroup

  \usebackgroundtemplate{ % \hspace*{-11.5em}
    \includegraphics[height=\paperheight]{\TitleBackgroundPicture}}

  \maketitle

  \endgroup
}





\newcommand{\SectionSlideWithPicture}[1]{%
  \begingroup

  \usebackgroundtemplate{ % \hspace*{-11.5em}
    \includegraphics[height=\paperheight]{\SectionBackgroundPicture}}

  \setbeamercolor{titlelike}{fg=normal text.fg}

  \section{#1}

  \endgroup
}





\newcommand{\EndingSlide}[1]{%
  \begin{frame}[standout]

    \begingroup

    \color{jFrametitleFGColor}

    #1

    \endgroup

  \end{frame}
}










% ---------------------------------------
% Packages, libraries and their configuration
% ---------------------------------------





% ---------------------------------------
% Configuration for this particular presentation
% ---------------------------------------
  \LetLtxMacro{\oldsqrt}{\sqrt}
\def\sqrt{\mathpalette\DHLhksqrt}
\def\DHLhksqrt#1#2{%
\setbox0=\hbox{$#1\oldsqrt{#2\,}$}\dimen0=\ht0
\advance\dimen0-0.2\ht0
\setbox2=\hbox{\vrule height\ht0 depth -\dimen0}%
{\box0\lower0.4pt\box2}}










% ---------------------------------------------------------------------
\title{Jak pisać w~\LaTeX u}

\author{Kamil Ziemian}


% \\ \texttt{ziemniakzkosmosu@gmail.com} }

\date[21 XI 2015]{21 listopada 2015 r.}
% ---------------------------------------------------------------------










% ####################################################################
% Początek dokumentu
\begin{document}
% ####################################################################





% Wyrównanie do lewej z łamaniem wyrazów

\RaggedRight





% ######################################
\maketitle % Tytuł całego tekstu
% ######################################





% ######################################
\begin{frame}
  \frametitle{Plan prezentacji}


  \tableofcontents % Spis treści

\end{frame}
% ######################################










% ##################
\begin{frame}
  \frametitle{\LaTeX{} w~ogólności}


  % #############
  \begin{figure}

    \centering

    \includegraphics[scale=0.4]
    {./PresentationPictures/Programing_language.jpg}


    \caption{Cała prawda o~\LaTeX u.
      \href{http://www.buzzfeed.com/lukelewis/28-things-only-developers-will-find-funny}{Link
        (tu kliknij)}}

  \end{figure}
  % #############

\end{frame}
% ##################





% ##################
\begin{frame}
  \frametitle{Filozofia}


  \noindent Trudno się dziwić, że~tak się pogrążyłem w marnościach
  i~odchodziłem daleko od Ciebie, skoro jako wzory do naśladowania
  przedstawiano mi ludzi, którzy wstydzili się jak hańby tego,
  że~opowiadając o dobrych swoich czynach, popełnili błąd gramatyczny
  albo~użyli wyrażeń prowincjonalnych, a~dumnie kroczyli w obłoku
  pochwały, jeśli o~swoich niegodziwych pasjach mówili zdaniami
  zaokrąglonymi, błyszczącymi obfitą ornamentyką.

  Św. Augustyn \textit{Wyznania}, księga I, 16

\end{frame}
% ##################










% ######################################
\section{Rady dla początkujących}
% ######################################



% ##################
\begin{frame}
  \frametitle{Podstawy}


  Częste początkowe (i~nie tylko) błędy

  \begin{itemize}
    \RaggedRight

  \item {\Large \color{red} Nie czytanie tekstu który~się
      napisało!}

  \item Nowy akapit tworzymy, nie~przez przejście do nowej
    linii, lecz przez wstawienie co~najmniej jednej pustej.

  \item W~trybie matematyczny pusta linia wywołuje błąd przy
    kompilacji.

  \item Środowisko które zostało otwarte np. equation, math,
    etc., należy {\color{red} natychmiast zamknąć}. Z~nawiasami to
    samo.

  \item Brak organizacji „kodu”.

  \item Nie korzystanie z~wszystkich opcji jakie daje \LaTeX.

  \item Ani z~wszystkich fajnych pakietów.

  \end{itemize}



  Resztę powinno dać się {\color{blue} wygooglować}.

\end{frame}
% ##################





% ##################
\begin{frame}[fragile]
  \frametitle{Estetyczne drobiazgi\ldots}


  \ldots o~które~się przesadnie czepiamy
  \begin{itemize}
    \RaggedRight

  \item Wyrazów takich jak „i”,~„a”,~„że”,~etc., nie należy
    zostawiać na~końcu linii, tak samo jak nie powinno~się
    na~początku linii umieszczać~,,się''. Najlepszym sposobem
    radzenia sobie z~tym jest używanie twardej spacji \verb+~+,
    która nie~pozwoli podzielić wyrazów nią~oddzielonych. Np.
    „\verb+z~tym+”, zamiast „z tym”.

  \item Podobnie warto postępować z~tytułami typu: „Pan”,
    „Pani”, „dr.”, etc.

  \item Nazwy najpopularniejszych funkcji typu $\exp$, $\sin$
    etc. powinny być pisane czcionką drukowaną jak powyżej. Źle
    wygląda np.~$exp$. Dla większości funkcji wystarczy wpisać \textbackslash
    \textit{nazwa\_{}funkcji} np. \verb+\exp+.

  \end{itemize}



  Resztę powinno dać się {\color{red} wygooglować}.

\end{frame}
% ##################





% ##################
\begin{frame}[fragile]
  \frametitle{Estetyczne drobiazgi\ldots}


  \ldots o~które~mogą przyczepić~się na pracowni
  \begin{itemize}
    \RaggedRight

  \item Jednostki fizyczne powinny być pisane czcionką drukowaną,
    a~nie~standardową dla trybu matematycznego kursywą.
    Czyli $\si{kg}$, a~nie $kg$.

  \item Pomiędzy liczbą~a~jednostką, tak jak między jednostkami,
    powinien być odstęp. Porównaj $9 \si{kgm}$ z~$9 \, \si{kg.m}$.

  \end{itemize}



  Zawsze można~się o~coś przyczepić. Istnieją zasady co do stosowania „-”
  i~„--”. Ja zawsze sprawdzam je na stronach 21 i~22
  \href{ftp://ftp.gust.org.pl/pub/CTAN/info/lshort/polish/lshort2e.pdf}
  {\textit{Nie za krótkiego wprowadzenia do \LaTeX a} (tu kliknij)}.

  Resztę powinno dać się {\color{yellow} wygooglować}.

\end{frame}
% ##################





% ##################
\begin{frame}[fragile]
  \frametitle{\{\} może mieć znaczenie}

  Porównaj \\
  „\verb+\LaTeX i~\TeX są+” daje „\LaTeX i~\TeX są”; \\
  „\verb+\LaTeX{} i~\TeX{} są+” daje „\LaTeX{} i~\TeX{} są”; \\
  „\verb+\LaTeX{} i~\TeX~są+'' daje ,,\LaTeX{} i~\TeX~są”.


\end{frame}
% ##################










% ######################################
\section{Fajne pakiety na zachętę}
% ######################################



% ##################
\begin{frame}[fragile]
  \frametitle{Pakiet siunitx, czyli na wszystko jest kilka paczek}


  Możliwości \\
  \verb+\num{.123}+ $\to \num{.123}$; \\
  \verb+\num{123.45}+ $\to \num{123.45}$; \\
  \verb+\num{3.45d-4}+ $\to \num{3.45d-4}$; \\
  \verb+\num{-e10}+ $\to \num{-e10}$; \\
  \verb+\si{\kilo \gram \per \second}+ $\to \si{\kilo \gram \per \second}$; \\
  \verb+\si{kg.m/s^2}+ $\to \si{kg.m/s^{ 2 }}$; \\
  \verb+\SI[mode = text]{1.38d-23}{J.K^{-1}}+
  $\to \SI[mode = text]{1.38d-23}{J.K^{-1}}$. \verb+9 \si{.kg}+
  $\to 9 \si{.kg}$.

  \href{ftp://ftp.dante.de/tex-archive/macros/latex/exptl/siunitx/siunitx.pdf}
  {Tu kliknij po więcej}

  Resztę powinno dać się {\color{magenta} wygooglować}.

\end{frame}
% ##################





% ##################
\begin{frame}[fragile]
  \frametitle{Pakiet hyperref}


  Tu jest równanie
  \begin{equation}
    \label{eq:Jak-pisac-01}
    f( x ) = \exp( -x^{ 2 } ).
  \end{equation}

  Hyperref
  \begin{itemize}
    \RaggedRight

  \item W beamerze hyperlink nie~jest widoczny, w~innych plikach
    jest on otoczony kolorową ramką.

  \item Automatycznie linkowane~są też odniesienia do bibliografii.

  \item Wypisując \verb+\href{+\textit{Link}\verb+}{To się wyświetli}+
    tworzymy link do strony internetowej.

  \end{itemize}



  % A~tu odwołanie do równania: \eqref{eq:Jak-pisac-01}.

  % Resztę powinno dać się {\color{green} wygooglować}.

\end{frame}
% ##################






% ##################
\begin{frame}
  \frametitle{Vmargin -- kontrola marginesów}


  Rozmiar marginesów. Standard anglosaski ma bardzo duże marginesy, co
  zwykle oznacza marnowanie sporej części kartki. Można je zmniejszyć za
  pomocą pakietu \textit{fullpage}, jednak znacznie lepszym rozwiązaniem
  jest pakiet \textit{vmargin}.

\end{frame}
% ##################





% ##################
\begin{frame}[fragile]
  \frametitle{Kontrola marginesów}


  Trzeba jednak mieć wyczucie jednostek typograficznych
\begin{verbatim}
\usepackage{vmargin}
% ---------------------------------------------------
% MARGINS
% ---------------------------------------------------
\setmarginsrb
{ 0.7in}  % left margin
{ 0.6in}  % top margin
{ 0.7in}  % right margin
{ 0.8in}  % bottom margin
{  20pt}  % head height
{0.25in}  % head sep
{   9pt}  % foot height
{ 0.3in}  % foot sep
\end{verbatim}

\end{frame}
% ##################





% ##########
\begin{frame}[fragile]
  \frametitle{Ciekawostka: verse}

  Wpisujemy

  \begingroup

  \tiny

\begin{verbatim}
\settowidth{\versewidth}{Haniebnym marnotrawstwem sił ducha i ciała}

\begin{verse}[\versewidth]
  \poemlines{20}

  Haniebnym marnotrawstwem sił ducha i ciała \\
  Jest żądza, gdy osiąga cel; przedtem zaś skłania \\
  Do kłamstwa, morderstw, podłości -- zaborcza, zuchwała, \\
  Dzika, sroga, niegodna krztyny zaufania; \\
  Ledwie rozkosz przyniesie, już czyni ją wstrętem; \\
  Ledwie w obłędnych łowach dopadnie zwierzyny -- \\
  Z obłędną nienawiścią widzi w niej przynętę, \\
  Co szaleństwem skaziła jej plany i czyny: \\
  Burzy się, gdy chce posiąść i gdy już posiada; \\
  Sama szalona, chętkom folguje szalonym; \\
  Jęk szczęścia, ledwie zabrzmiał, przemienia w jęk „Biada!”, \\
  Marzenie w jednej chwili czyni snem prześnionym. \\
  \vin Wszystko to wiemy; lecz choć wiemy, wciąż nas wściekła \\
  \vin Żądza, raj obiecując, wiedzie w otchłań piekła.
\end{verse}

W. Shakespeare \textit{Sonet 129}, tłumaczył S. Barańczak
\end{verbatim}

  \endgroup

\end{frame}
% ##################










% ######################################
\section{Praca z~„kodem”}
% ######################################



% ##################
\begin{frame}
  \frametitle{Przykładowy slajd}


  Uwagi o~Pythonie
  \begin{itemize}
    \RaggedRight

  \item Jest dwustandardowy: 2.7x, 3.x (3.3.x). Wersje te nie są
    wstecznie kompatybilne.

  \item „Rzeczy niebezpieczne mają być utrudnione, lecz nie
    zabronione.”

  \item Doświadczenie z nauki można łatwo przenieść na wiele
    innych języków.

  \end{itemize}



  Uwagi o Terrarium
  \begin{itemize}
    \RaggedRight

  \item Będzie trochę matematyki.

  \item Prowadzący spotkania nie jest najlepszy. Za to ma oddane
    wsparcie.

  \end{itemize}

\end{frame}
% ##################





% ##########
\begin{frame}[fragile]
  \frametitle{Dobrych wcięć nigdy nie~za~wiele}


  „Kod” poprzedniego slajdu może wyglądać tak

  \begingroup

  \tiny

\begin{verbatim}
\begin{frame}
\frametitle{Przykładowy slajd}
Uwagi o Python
\begin{itemize}
\RaggedRight
\item Jest dwustandardowy: 2.7x, 3.x (3.3.x). Wersje te nie są
wstecznie kompatybilne.
\item „Rzeczy niebezpieczne mają być utrudnione, lecz nie
zabronione.”
\item Doświadczenie z nauki można łatwo przenieść na wiele
innych języków.
\end{itemize}

Uwagi o Terrarium
\begin{itemize}
\item Będzie trochę matematyki.
\item Prowadzący spotkania nie jest najlepszy. Za to ma oddane wsparcie.
\end{itemize}
\end{verbatim}

\endgroup

\end{frame}
% ##################





% ##################
\begin{frame}[fragile]
  \frametitle{Dobrych wcięć nigdy nie~za~wiele}


  Albo tak

  \begingroup

  \tiny

\begin{verbatim}
 \begin{frame}
   \frametitle{Przykładowy slajd}


   Uwagi o Python
   \begin{itemize}
     \RaggedRight

   \item Jest dwustandardowy: 2.7x, 3.x (3.3.x). Wersje te nie
     są wstecznie kompatybilne.

   \item „Rzeczy niebezpieczne mają być utrudnione, lecz nie
     zabronione.”

   \item Doświadczenie z nauki można łatwo przenieść na wiele
     innych języków.

   \end{itemize}

   Uwagi o Terrarium
   \begin{itemize}
     \RaggedRight

   \item Będzie trochę matematyki.

   \item Prowadzący spotkania nie jest najlepszy. Za to ma oddane wsparcie.

   \end{itemize}
\end{verbatim}

\endgroup



  Edytor może pomóc \\
  -- Wcięcie „kodu” (dla \TeX Makera): Ctrl-$>$; \\
  -- cofnięcie „kodu”: Ctrl-$<$.

\end{frame}
% ##################





% ##################
\begin{frame}
  \frametitle{Podstawowa technika usuwania błędów kompilacji}


  Czytelny „kod”. Pozwala zorientować~się co gdzie jest. Ponieważ często
  nie umiemy od razu zlokalizować zbyt źródła najlepiej jest wykomentować
  podejrzane partie „kodu”, aż~kompilator ruszy. Gdy już wiemy gdzie jest
  błąd, jego znalezienie jest prostsze.

  Edytor może pomóc \\
  W~\TeX Makerze: \\
  -- komentarz: Ctrl-T; \\
  -- odkomentowanie: Ctrl-U.

\end{frame}
% ##################





% ##################
\begin{frame}[fragile]
  \frametitle{\LaTeX{} i~\TeX{} to również języki programowania}


  I~czasem trzeba użyć fragmentu kodu
  \begin{itemize}
    \RaggedRight

  \item Pozbyć~się z~tej prezentacji niepotrzebnych dodatków
    beamera, które zajmują potrzebne miejsce.
    \begin{itemize}
      \RaggedRight

    \item[1)] Niepotrzebny spis treści na górze:
      \verb+\setbeamertemplate{navigation symbols}{}+

    \item[2)] Przez nikogo nie używane ikonki na dole:
      \verb+\setbeamertemplate{navigation symbols}{}+.

    \end{itemize}

    Można było użyć stylu „Cracow”, ale~to rozwiązanie było
    prostsze.

  \item Zwiększyć odstęp między wierszami w~tabeli:
    \verb+\renewcommand{\arraystretch}{1.2}+. Wartość~1.0
    to~standardowy ostęp.

  \end{itemize}

\end{frame}
% ##################





% ##################
\begin{frame}[fragile]
  \frametitle{\LaTeX{} i~\TeX{} to również języki programowania}


  I~czasem trzeba użyć fragmentu kodu
  \begin{itemize}
    \RaggedRight

  \item Przy korzystaniu z~pakietów AMS (American Mathematical Society)
    i~pakietów polonizacyjnych może~się pojawić błąd: \\
    {\color{red} Error}: \verb+\lll+ already defined. \\
    Dwa pakiety definiują dwie komendy o~tej samej nazwie. Aby rozwiązać
    ten problem wystarczy oddefiniować \verb+\lll+ przez wstawienie
    do preambuły: \\
    \vspace{3pt}
    \verb+\let\lll\undefined+ \\
    \verb+\usepackage{Druga_która_definiuje_\lll}+.

  \end{itemize}



  Na szczęście. Korzystanie z~komend zwykle jest niepotrzebne.

\end{frame}
% ##################





% ##################
\begin{frame}[fragile]
  \frametitle{\textbackslash newcommand to fantastyczna rzecz}


  Moja opinia. Używam jej do wszystkiego i~wciąż wydaje mi~się, że~mogę
  sobie jeszcze bardziej uprościć życie, używając jej jeszcze więcej.

  Stary przykład. USUNIĘTY.

\end{frame}
% ##################





% ##################
\begin{frame}
  \frametitle{???}


  Częste początkowe (i~nie tylko) błędy
  \begin{itemize}
    \RaggedRight

  \item {\Large \color{red} Nie czytanie tekstu który~się napisało!}

  \item Nowy akapit tworzymy, nie~przez przejście do nowej linii, lecz
    przez wstawienie co~najmniej jednej pustej.

  \item W~trybie matematyczny pusta linia wywołuje błąd przy kompilacji.

  \item Brak organizacji „kodu”.

  \item Nie korzystanie z~wszystkich opcji jakie daje \LaTeX.

  \item Ani z~wszystkich fajnych pakietów.

  \end{itemize}

\end{frame}
% ##################





% ##################
\begin{frame}[fragile]
  \frametitle{Co jeszcze można wygooglować?}


  „Kod” strony
\begin{verbatim}
Częste początkowe (i~nie tylko)
\begin{itemize}
  \RaggedRight

\item {\Large \color{red} Nie czytanie napisało!}

\item Nowy akapit tworzymy, nie~przez.

\item W~trybie matematyczny pusta linia kompilacji.

\item Brak organizacji „kodu”.

\item Nie korzystanie z~wszystkich opcji.

\item Ani z~wszystkich fajnych pakietów.

\end{itemize}
\end{verbatim}

\end{frame}
% ##################





% ##################
\begin{frame}[fragile]
  \frametitle{Co jeszcze można wygooglować?}


  Na przykład to
\begin{verbatim}
\LetLtxMacro{\oldsqrt}{\sqrt}
\def\sqrt{\mathpalette\DHLhksqrt}
\def\DHLhksqrt#1#2{%
\setbox0=\hbox{$#1\oldsqrt{#2\,}$}\dimen0=\ht0
\advance\dimen0-0.2\ht0
\setbox2=\hbox{\vrule height\ht0 depth -\dimen0}%
{\box0\lower0.4pt\box2}}
\end{verbatim}

  Wtedy {\Large \verb+\oldsqrt{a}+ $\to \oldsqrt{a}$; \\
    \verb+\sqrt{a}+ $\to \sqrt{a}$.} \\
  Pewnie można ładniej, ale~nie chciało mi~się szukać.

\end{frame}
% ##################










% ######################################
\appendix
% ######################################





% ######################################
\EndingSlide{\LaTeX{} nie~jest trudny. ;) \\
  Dziękuję! Pytania?}
% ######################################










% ####################################################################
% ####################################################################
% Bibliografia
\bibliographystyle{plalpha}

\bibliography{Bibliography}{}





% ############################

% Koniec dokumentu
\end{document}