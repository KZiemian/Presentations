% ---------------------------------------------------------------------
% Basic configuration of Beamera and Jagiellonian
% ---------------------------------------------------------------------
\RequirePackage[l2tabu, orthodox]{nag}



\ifx\PresentationStyle\notset
\def\PresentationStyle{dark}
\fi



\documentclass[10pt,t]{beamer}
\mode<presentation>
\usetheme[style=\PresentationStyle,logoLang=Latin,logoColor=monochromaticJUwhite,JUlogotitle=yes]{jagiellonian}



% ---------------------------------------
% Configuration files of Jagiellonian loceted in catalog preambule
% ---------------------------------------
% Configuration for polish language
% Need description
\usepackage[polish]{babel}
% Need description
\usepackage[MeX]{polski}



% ------------------------------
% Better support of polish chars in technical parts of PDF
% ------------------------------
\hypersetup{pdfencoding=auto,psdextra}

% Package "textpos" give as enviroment "textblock" which is very usefull in
% arranging text on slides.

% This is standard configuration of "textpos"
\usepackage[overlay,absolute]{textpos}

% If you need to see bounds of "textblock's" comment line above and uncomment
% one below.

% Caution! When showboxes option is on significant ammunt of space is add
% to the top of textblock and as such, everyting put in them gone down.
% We need to check how to remove this bug.

% \usepackage[showboxes,overlay,absolute]{textpos}



% Setting scale length for package "textpos"
\setlength{\TPHorizModule}{10mm}
\setlength{\TPVertModule}{\TPHorizModule}


% ---------------------------------------
% TikZ
% ---------------------------------------
% Importing TikZ libraries
\usetikzlibrary{arrows.meta}
\usetikzlibrary{positioning}





% % Configuration package "bm" that need for making bold symbols
% \newcommand{\bmmax}{0}
% \newcommand{\hmmax}{0}
% \usepackage{bm}




% ---------------------------------------
% Packages for scientific texts
% ---------------------------------------
% \let\lll\undefined  % Sometimes you must use this line to allow
% "amsmath" package to works with packages with packages for polish
% languge imported
% /preambul/LanguageSettings/JagiellonianPolishLanguageSettings.tex.
% This comments (probably) removes polish letter Ł.
\usepackage{amsmath}  % Packages from American Mathematical Society (AMS)
\usepackage{amssymb}
\usepackage{amscd}
\usepackage{amsthm}
\usepackage{siunitx}  % Package for typsetting SI units.
\usepackage{upgreek}  % Better looking greek letters.
% Example of using upgreek: pi = \uppi


\usepackage{calrsfs}  % Zmienia czcionkę kaligraficzną w \mathcal
% na ładniejszą. Może w innych miejscach robi to samo, ale o tym nic
% nie wiem.










% ---------------------------------------
% Packages written for lectures "Geometria 3D dla twórców gier wideo"
% ---------------------------------------
% \usepackage{./ProgramowanieSymulacjiFizykiPaczki/ProgramowanieSymulacjiFizyki}
% \usepackage{./ProgramowanieSymulacjiFizykiPaczki/ProgramowanieSymulacjiFizykiIndeksy}
% \usepackage{./ProgramowanieSymulacjiFizykiPaczki/ProgramowanieSymulacjiFizykiTikZStyle}





% !!!!!!!!!!!!!!!!!!!!!!!!!!!!!!
% !!!!!!!!!!!!!!!!!!!!!!!!!!!!!!
% EVIL STUFF
\if\JUlogotitle1
\edef\LogoJUPath{LogoJU_\JUlogoLang/LogoJU_\JUlogoShape_\JUlogoColor.pdf}
\titlegraphic{\hfill\includegraphics[scale=0.22]
{./JagiellonianPictures/\LogoJUPath}}
\fi
% ---------------------------------------
% Commands for handling colors
% ---------------------------------------


% Command for setting normal text color for some text in math modestyle
% Text color depend on used style of Jagiellonian

% Beamer version of command
\newcommand{\TextWithNormalTextColor}[1]{%
  {\color{jNormalTextFGColor}
    \setbeamercolor{math text}{fg=jNormalTextFGColor} {#1}}
}

% Article and similar classes version of command
% \newcommand{\TextWithNormalTextColor}[1]{%
%   {\color{jNormalTextsFGColor} {#1}}
% }



% Beamer version of command
\newcommand{\NormalTextInMathMode}[1]{%
  {\color{jNormalTextFGColor}
    \setbeamercolor{math text}{fg=jNormalTextFGColor} \text{#1}}
}


% Article and similar classes version of command
% \newcommand{\NormalTextInMathMode}[1]{%
%   {\color{jNormalTextsFGColor} \text{#1}}
% }




% Command that sets color of some mathematical text to the same color
% that has normal text in header (?)

% Beamer version of the command
\newcommand{\MathTextFrametitleFGColor}[1]{%
  {\color{jFrametitleFGColor}
    \setbeamercolor{math text}{fg=jFrametitleFGColor} #1}
}

% Article and similar classes version of the command
% \newcommand{\MathTextWhiteColor}[1]{{\color{jFrametitleFGColor} #1}}





% Command for setting color of alert text for some text in math modestyle

% Beamer version of the command
\newcommand{\MathTextAlertColor}[1]{%
  {\color{jOrange} \setbeamercolor{math text}{fg=jOrange} #1}
}

% Article and similar classes version of the command
% \newcommand{\MathTextAlertColor}[1]{{\color{jOrange} #1}}





% Command that allow you to sets chosen color as the color of some text into
% math mode. Due to some nuances in the way that Beamer handle colors
% it not work in all cases. We hope that in the future we will improve it.

% Beamer version of the command
\newcommand{\SetMathTextColor}[2]{%
  {\color{#1} \setbeamercolor{math text}{fg=#1} #2}
}


% Article and similar classes version of the command
% \newcommand{\SetMathTextColor}[2]{{\color{#1} #2}}










% ---------------------------------------
% Commands for few special slides
% ---------------------------------------
\newcommand{\EndingSlide}[1]{%
  \begin{frame}[standout]

    \begingroup

    \color{jFrametitleFGColor}

    #1

    \endgroup

  \end{frame}
}










% ---------------------------------------
% Commands for setting background pictures for some slides
% ---------------------------------------
\newcommand{\TitleBackgroundPicture}
{./JagiellonianPictures/Backgrounds/LajkonikDark.png}
\newcommand{\SectionBackgroundPicture}
{./JagiellonianPictures/Backgrounds/LajkonikLight.png}



\newcommand{\TitleSlideWithPicture}{%
  \begingroup

  \usebackgroundtemplate{%
    \includegraphics[height=\paperheight]{\TitleBackgroundPicture}}

  \maketitle

  \endgroup
}





\newcommand{\SectionSlideWithPicture}[1]{%
  \begingroup

  \usebackgroundtemplate{%
    \includegraphics[height=\paperheight]{\SectionBackgroundPicture}}

  \setbeamercolor{titlelike}{fg=normal text.fg}

  \section{#1}

  \endgroup
}










% ---------------------------------------
% Commands for lectures "Geometria 3D dla twórców gier wideo"
% Polish version
% ---------------------------------------
% Komendy teraz wykomentowane były potrzebne, gdy loga były na niebieskim
% tle, nie na białym. A są na białym bo tego chcieli w biurze projektu.
% \newcommand{\FundingLogoWhitePicturePL}
% {./PresentationPictures/CommonPictures/logotypFundusze_biale_bez_tla2.pdf}
\newcommand{\FundingLogoColorPicturePL}
{./PresentationPictures/CommonPictures/European_Funds_color_PL.pdf}
% \newcommand{\EULogoWhitePicturePL}
% {./PresentationPictures/CommonPictures/logotypUE_biale_bez_tla2.pdf}
\newcommand{\EUSocialFundLogoColorPicturePL}
{./PresentationPictures/CommonPictures/EU_Social_Fund_color_PL.pdf}
% \newcommand{\ZintegrUJLogoWhitePicturePL}
% {./PresentationPictures/CommonPictures/zintegruj-logo-white.pdf}
\newcommand{\ZintegrUJLogoColorPicturePL}
{./PresentationPictures/CommonPictures/ZintegrUJ_color.pdf}
\newcommand{\JULogoColorPicturePL}
{./JagiellonianPictures/LogoJU_PL/LogoJU_A_color.pdf}





\newcommand{\GeometryThreeDSpecialBeginningSlidePL}{%
  \begin{frame}[standout]

    \begin{textblock}{11}(1,0.7)

      \begin{flushleft}

        \mdseries

        \footnotesize

        \color{jFrametitleFGColor}

        Materiał powstał w ramach projektu współfinansowanego ze środków
        Unii Europejskiej w ramach Europejskiego Funduszu Społecznego
        POWR.03.05.00-00-Z309/17-00.

      \end{flushleft}

    \end{textblock}





    \begin{textblock}{10}(0,2.2)

      \tikz \fill[color=jBackgroundStyleLight] (0,0) rectangle (12.8,-1.5);

    \end{textblock}


    \begin{textblock}{3.2}(1,2.45)

      \includegraphics[scale=0.3]{\FundingLogoColorPicturePL}

    \end{textblock}


    \begin{textblock}{2.5}(3.7,2.5)

      \includegraphics[scale=0.2]{\JULogoColorPicturePL}

    \end{textblock}


    \begin{textblock}{2.5}(6,2.4)

      \includegraphics[scale=0.1]{\ZintegrUJLogoColorPicturePL}

    \end{textblock}


    \begin{textblock}{4.2}(8.4,2.6)

      \includegraphics[scale=0.3]{\EUSocialFundLogoColorPicturePL}

    \end{textblock}

  \end{frame}
}



\newcommand{\GeometryThreeDTwoSpecialBeginningSlidesPL}{%
  \begin{frame}[standout]

    \begin{textblock}{11}(1,0.7)

      \begin{flushleft}

        \mdseries

        \footnotesize

        \color{jFrametitleFGColor}

        Materiał powstał w ramach projektu współfinansowanego ze środków
        Unii Europejskiej w ramach Europejskiego Funduszu Społecznego
        POWR.03.05.00-00-Z309/17-00.

      \end{flushleft}

    \end{textblock}





    \begin{textblock}{10}(0,2.2)

      \tikz \fill[color=jBackgroundStyleLight] (0,0) rectangle (12.8,-1.5);

    \end{textblock}


    \begin{textblock}{3.2}(1,2.45)

      \includegraphics[scale=0.3]{\FundingLogoColorPicturePL}

    \end{textblock}


    \begin{textblock}{2.5}(3.7,2.5)

      \includegraphics[scale=0.2]{\JULogoColorPicturePL}

    \end{textblock}


    \begin{textblock}{2.5}(6,2.4)

      \includegraphics[scale=0.1]{\ZintegrUJLogoColorPicturePL}

    \end{textblock}


    \begin{textblock}{4.2}(8.4,2.6)

      \includegraphics[scale=0.3]{\EUSocialFundLogoColorPicturePL}

    \end{textblock}

  \end{frame}





  \TitleSlideWithPicture
}



\newcommand{\GeometryThreeDSpecialEndingSlidePL}{%
  \begin{frame}[standout]

    \begin{textblock}{11}(1,0.7)

      \begin{flushleft}

        \mdseries

        \footnotesize

        \color{jFrametitleFGColor}

        Materiał powstał w ramach projektu współfinansowanego ze środków
        Unii Europejskiej w~ramach Europejskiego Funduszu Społecznego
        POWR.03.05.00-00-Z309/17-00.

      \end{flushleft}

    \end{textblock}





    \begin{textblock}{10}(0,2.2)

      \tikz \fill[color=jBackgroundStyleLight] (0,0) rectangle (12.8,-1.5);

    \end{textblock}


    \begin{textblock}{3.2}(1,2.45)

      \includegraphics[scale=0.3]{\FundingLogoColorPicturePL}

    \end{textblock}


    \begin{textblock}{2.5}(3.7,2.5)

      \includegraphics[scale=0.2]{\JULogoColorPicturePL}

    \end{textblock}


    \begin{textblock}{2.5}(6,2.4)

      \includegraphics[scale=0.1]{\ZintegrUJLogoColorPicturePL}

    \end{textblock}


    \begin{textblock}{4.2}(8.4,2.6)

      \includegraphics[scale=0.3]{\EUSocialFundLogoColorPicturePL}

    \end{textblock}





    \begin{textblock}{11}(1,4)

      \begin{flushleft}

        \mdseries

        \footnotesize

        \RaggedRight

        \color{jFrametitleFGColor}

        Treść niniejszego wykładu jest udostępniona na~licencji
        Creative Commons (\textsc{cc}), z~uzna\-niem autorstwa
        (\textsc{by}) oraz udostępnianiem na tych samych warunkach
        (\textsc{sa}). Rysunki i~wy\-kresy zawarte w~wykładzie są
        autorstwa dr.~hab.~Pawła Węgrzyna et~al. i~są dostępne
        na tej samej licencji, o~ile nie wskazano inaczej.
        W~prezentacji wykorzystano temat Beamera Jagiellonian,
        oparty na~temacie Metropolis Matthiasa Vogelgesanga,
        dostępnym na licencji \LaTeX{} Project Public License~1.3c
        pod adresem: \colorhref{https://github.com/matze/mtheme}
        {https://github.com/matze/mtheme}.

        Projekt typograficzny: Iwona Grabska-Gradzińska \\
        Skład: Kamil Ziemian;
        Korekta: Wojciech Palacz \\
        Modele: Dariusz Frymus, Kamil Nowakowski \\
        Rysunki i~wykresy: Kamil Ziemian, Paweł Węgrzyn, Wojciech Palacz

      \end{flushleft}

    \end{textblock}

  \end{frame}
}



\newcommand{\GeometryThreeDTwoSpecialEndingSlidesPL}[1]{%
  \begin{frame}[standout]


    \begin{textblock}{11}(1,0.7)

      \begin{flushleft}

        \mdseries

        \footnotesize

        \color{jFrametitleFGColor}

        Materiał powstał w ramach projektu współfinansowanego ze środków
        Unii Europejskiej w~ramach Europejskiego Funduszu Społecznego
        POWR.03.05.00-00-Z309/17-00.

      \end{flushleft}

    \end{textblock}





    \begin{textblock}{10}(0,2.2)

      \tikz \fill[color=jBackgroundStyleLight] (0,0) rectangle (12.8,-1.5);

    \end{textblock}


    \begin{textblock}{3.2}(1,2.45)

      \includegraphics[scale=0.3]{\FundingLogoColorPicturePL}

    \end{textblock}


    \begin{textblock}{2.5}(3.7,2.5)

      \includegraphics[scale=0.2]{\JULogoColorPicturePL}

    \end{textblock}


    \begin{textblock}{2.5}(6,2.4)

      \includegraphics[scale=0.1]{\ZintegrUJLogoColorPicturePL}

    \end{textblock}


    \begin{textblock}{4.2}(8.4,2.6)

      \includegraphics[scale=0.3]{\EUSocialFundLogoColorPicturePL}

    \end{textblock}





    \begin{textblock}{11}(1,4)

      \begin{flushleft}

        \mdseries

        \footnotesize

        \RaggedRight

        \color{jFrametitleFGColor}

        Treść niniejszego wykładu jest udostępniona na~licencji
        Creative Commons (\textsc{cc}), z~uzna\-niem autorstwa
        (\textsc{by}) oraz udostępnianiem na tych samych warunkach
        (\textsc{sa}). Rysunki i~wy\-kresy zawarte w~wykładzie są
        autorstwa dr.~hab.~Pawła Węgrzyna et~al. i~są dostępne
        na tej samej licencji, o~ile nie wskazano inaczej.
        W~prezentacji wykorzystano temat Beamera Jagiellonian,
        oparty na~temacie Metropolis Matthiasa Vogelgesanga,
        dostępnym na licencji \LaTeX{} Project Public License~1.3c
        pod adresem: \colorhref{https://github.com/matze/mtheme}
        {https://github.com/matze/mtheme}.

        Projekt typograficzny: Iwona Grabska-Gradzińska \\
        Skład: Kamil Ziemian;
        Korekta: Wojciech Palacz \\
        Modele: Dariusz Frymus, Kamil Nowakowski \\
        Rysunki i~wykresy: Kamil Ziemian, Paweł Węgrzyn, Wojciech Palacz

      \end{flushleft}

    \end{textblock}

  \end{frame}





  \begin{frame}[standout]

    \begingroup

    \color{jFrametitleFGColor}

    #1

    \endgroup

  \end{frame}
}



\newcommand{\GeometryThreeDSpecialEndingSlideVideoPL}{%
  \begin{frame}[standout]

    \begin{textblock}{11}(1,0.7)

      \begin{flushleft}

        \mdseries

        \footnotesize

        \color{jFrametitleFGColor}

        Materiał powstał w ramach projektu współfinansowanego ze środków
        Unii Europejskiej w~ramach Europejskiego Funduszu Społecznego
        POWR.03.05.00-00-Z309/17-00.

      \end{flushleft}

    \end{textblock}





    \begin{textblock}{10}(0,2.2)

      \tikz \fill[color=jBackgroundStyleLight] (0,0) rectangle (12.8,-1.5);

    \end{textblock}


    \begin{textblock}{3.2}(1,2.45)

      \includegraphics[scale=0.3]{\FundingLogoColorPicturePL}

    \end{textblock}


    \begin{textblock}{2.5}(3.7,2.5)

      \includegraphics[scale=0.2]{\JULogoColorPicturePL}

    \end{textblock}


    \begin{textblock}{2.5}(6,2.4)

      \includegraphics[scale=0.1]{\ZintegrUJLogoColorPicturePL}

    \end{textblock}


    \begin{textblock}{4.2}(8.4,2.6)

      \includegraphics[scale=0.3]{\EUSocialFundLogoColorPicturePL}

    \end{textblock}





    \begin{textblock}{11}(1,4)

      \begin{flushleft}

        \mdseries

        \footnotesize

        \RaggedRight

        \color{jFrametitleFGColor}

        Treść niniejszego wykładu jest udostępniona na~licencji
        Creative Commons (\textsc{cc}), z~uzna\-niem autorstwa
        (\textsc{by}) oraz udostępnianiem na tych samych warunkach
        (\textsc{sa}). Rysunki i~wy\-kresy zawarte w~wykładzie są
        autorstwa dr.~hab.~Pawła Węgrzyna et~al. i~są dostępne
        na tej samej licencji, o~ile nie wskazano inaczej.
        W~prezentacji wykorzystano temat Beamera Jagiellonian,
        oparty na~temacie Metropolis Matthiasa Vogelgesanga,
        dostępnym na licencji \LaTeX{} Project Public License~1.3c
        pod adresem: \colorhref{https://github.com/matze/mtheme}
        {https://github.com/matze/mtheme}.

        Projekt typograficzny: Iwona Grabska-Gradzińska;
        Skład: Kamil Ziemian \\
        Korekta: Wojciech Palacz;
        Modele: Dariusz Frymus, Kamil Nowakowski \\
        Rysunki i~wykresy: Kamil Ziemian, Paweł Węgrzyn, Wojciech Palacz \\
        Montaż: Agencja Filmowa Film \& Television Production~-- Zbigniew
        Masklak

      \end{flushleft}

    \end{textblock}

  \end{frame}
}





\newcommand{\GeometryThreeDTwoSpecialEndingSlidesVideoPL}[1]{%
  \begin{frame}[standout]

    \begin{textblock}{11}(1,0.7)

      \begin{flushleft}

        \mdseries

        \footnotesize

        \color{jFrametitleFGColor}

        Materiał powstał w ramach projektu współfinansowanego ze środków
        Unii Europejskiej w~ramach Europejskiego Funduszu Społecznego
        POWR.03.05.00-00-Z309/17-00.

      \end{flushleft}

    \end{textblock}





    \begin{textblock}{10}(0,2.2)

      \tikz \fill[color=jBackgroundStyleLight] (0,0) rectangle (12.8,-1.5);

    \end{textblock}


    \begin{textblock}{3.2}(1,2.45)

      \includegraphics[scale=0.3]{\FundingLogoColorPicturePL}

    \end{textblock}


    \begin{textblock}{2.5}(3.7,2.5)

      \includegraphics[scale=0.2]{\JULogoColorPicturePL}

    \end{textblock}


    \begin{textblock}{2.5}(6,2.4)

      \includegraphics[scale=0.1]{\ZintegrUJLogoColorPicturePL}

    \end{textblock}


    \begin{textblock}{4.2}(8.4,2.6)

      \includegraphics[scale=0.3]{\EUSocialFundLogoColorPicturePL}

    \end{textblock}





    \begin{textblock}{11}(1,4)

      \begin{flushleft}

        \mdseries

        \footnotesize

        \RaggedRight

        \color{jFrametitleFGColor}

        Treść niniejszego wykładu jest udostępniona na~licencji
        Creative Commons (\textsc{cc}), z~uzna\-niem autorstwa
        (\textsc{by}) oraz udostępnianiem na tych samych warunkach
        (\textsc{sa}). Rysunki i~wy\-kresy zawarte w~wykładzie są
        autorstwa dr.~hab.~Pawła Węgrzyna et~al. i~są dostępne
        na tej samej licencji, o~ile nie wskazano inaczej.
        W~prezentacji wykorzystano temat Beamera Jagiellonian,
        oparty na~temacie Metropolis Matthiasa Vogelgesanga,
        dostępnym na licencji \LaTeX{} Project Public License~1.3c
        pod adresem: \colorhref{https://github.com/matze/mtheme}
        {https://github.com/matze/mtheme}.

        Projekt typograficzny: Iwona Grabska-Gradzińska;
        Skład: Kamil Ziemian \\
        Korekta: Wojciech Palacz;
        Modele: Dariusz Frymus, Kamil Nowakowski \\
        Rysunki i~wykresy: Kamil Ziemian, Paweł Węgrzyn, Wojciech Palacz \\
        Montaż: Agencja Filmowa Film \& Television Production~-- Zbigniew
        Masklak

      \end{flushleft}

    \end{textblock}

  \end{frame}





  \begin{frame}[standout]


    \begingroup

    \color{jFrametitleFGColor}

    #1

    \endgroup

  \end{frame}
}










% ---------------------------------------
% Commands for lectures "Geometria 3D dla twórców gier wideo"
% English version
% ---------------------------------------
% \newcommand{\FundingLogoWhitePictureEN}
% {./PresentationPictures/CommonPictures/logotypFundusze_biale_bez_tla2.pdf}
\newcommand{\FundingLogoColorPictureEN}
{./PresentationPictures/CommonPictures/European_Funds_color_EN.pdf}
% \newcommand{\EULogoWhitePictureEN}
% {./PresentationPictures/CommonPictures/logotypUE_biale_bez_tla2.pdf}
\newcommand{\EUSocialFundLogoColorPictureEN}
{./PresentationPictures/CommonPictures/EU_Social_Fund_color_EN.pdf}
% \newcommand{\ZintegrUJLogoWhitePictureEN}
% {./PresentationPictures/CommonPictures/zintegruj-logo-white.pdf}
\newcommand{\ZintegrUJLogoColorPictureEN}
{./PresentationPictures/CommonPictures/ZintegrUJ_color.pdf}
\newcommand{\JULogoColorPictureEN}
{./JagiellonianPictures/LogoJU_EN/LogoJU_A_color.pdf}



\newcommand{\GeometryThreeDSpecialBeginningSlideEN}{%
  \begin{frame}[standout]

    \begin{textblock}{11}(1,0.7)

      \begin{flushleft}

        \mdseries

        \footnotesize

        \color{jFrametitleFGColor}

        This content was created as part of a project co-financed by the
        European Union within the framework of the European Social Fund
        POWR.03.05.00-00-Z309/17-00.

      \end{flushleft}

    \end{textblock}





    \begin{textblock}{10}(0,2.2)

      \tikz \fill[color=jBackgroundStyleLight] (0,0) rectangle (12.8,-1.5);

    \end{textblock}


    \begin{textblock}{3.2}(0.7,2.45)

      \includegraphics[scale=0.3]{\FundingLogoColorPictureEN}

    \end{textblock}


    \begin{textblock}{2.5}(4.15,2.5)

      \includegraphics[scale=0.2]{\JULogoColorPictureEN}

    \end{textblock}


    \begin{textblock}{2.5}(6.35,2.4)

      \includegraphics[scale=0.1]{\ZintegrUJLogoColorPictureEN}

    \end{textblock}


    \begin{textblock}{4.2}(8.4,2.6)

      \includegraphics[scale=0.3]{\EUSocialFundLogoColorPictureEN}

    \end{textblock}

  \end{frame}
}



\newcommand{\GeometryThreeDTwoSpecialBeginningSlidesEN}{%
  \begin{frame}[standout]

    \begin{textblock}{11}(1,0.7)

      \begin{flushleft}

        \mdseries

        \footnotesize

        \color{jFrametitleFGColor}

        This content was created as part of a project co-financed by the
        European Union within the framework of the European Social Fund
        POWR.03.05.00-00-Z309/17-00.

      \end{flushleft}

    \end{textblock}





    \begin{textblock}{10}(0,2.2)

      \tikz \fill[color=jBackgroundStyleLight] (0,0) rectangle (12.8,-1.5);

    \end{textblock}


    \begin{textblock}{3.2}(0.7,2.45)

      \includegraphics[scale=0.3]{\FundingLogoColorPictureEN}

    \end{textblock}


    \begin{textblock}{2.5}(4.15,2.5)

      \includegraphics[scale=0.2]{\JULogoColorPictureEN}

    \end{textblock}


    \begin{textblock}{2.5}(6.35,2.4)

      \includegraphics[scale=0.1]{\ZintegrUJLogoColorPictureEN}

    \end{textblock}


    \begin{textblock}{4.2}(8.4,2.6)

      \includegraphics[scale=0.3]{\EUSocialFundLogoColorPictureEN}

    \end{textblock}

  \end{frame}





  \TitleSlideWithPicture
}



\newcommand{\GeometryThreeDSpecialEndingSlideEN}{%
  \begin{frame}[standout]

    \begin{textblock}{11}(1,0.7)

      \begin{flushleft}

        \mdseries

        \footnotesize

        \color{jFrametitleFGColor}

        This content was created as part of a project co-financed by the
        European Union within the framework of the European Social Fund
        POWR.03.05.00-00-Z309/17-00.

      \end{flushleft}

    \end{textblock}





    \begin{textblock}{10}(0,2.2)

      \tikz \fill[color=jBackgroundStyleLight] (0,0) rectangle (12.8,-1.5);

    \end{textblock}


    \begin{textblock}{3.2}(0.7,2.45)

      \includegraphics[scale=0.3]{\FundingLogoColorPictureEN}

    \end{textblock}


    \begin{textblock}{2.5}(4.15,2.5)

      \includegraphics[scale=0.2]{\JULogoColorPictureEN}

    \end{textblock}


    \begin{textblock}{2.5}(6.35,2.4)

      \includegraphics[scale=0.1]{\ZintegrUJLogoColorPictureEN}

    \end{textblock}


    \begin{textblock}{4.2}(8.4,2.6)

      \includegraphics[scale=0.3]{\EUSocialFundLogoColorPictureEN}

    \end{textblock}





    \begin{textblock}{11}(1,4)

      \begin{flushleft}

        \mdseries

        \footnotesize

        \RaggedRight

        \color{jFrametitleFGColor}

        The content of this lecture is made available under a~Creative
        Commons licence (\textsc{cc}), giving the author the credits
        (\textsc{by}) and putting an obligation to share on the same terms
        (\textsc{sa}). Figures and diagrams included in the lecture are
        authored by Paweł Węgrzyn et~al., and are available under the same
        license unless indicated otherwise.\\ The presentation uses the
        Beamer Jagiellonian theme based on Matthias Vogelgesang’s
        Metropolis theme, available under license \LaTeX{} Project
        Public License~1.3c at: \colorhref{https://github.com/matze/mtheme}
        {https://github.com/matze/mtheme}.

        Typographic design: Iwona Grabska-Gradzińska \\
        \LaTeX{} Typesetting: Kamil Ziemian \\
        Proofreading: Wojciech Palacz,
        Monika Stawicka \\
        3D Models: Dariusz Frymus, Kamil Nowakowski \\
        Figures and charts: Kamil Ziemian, Paweł Węgrzyn, Wojciech Palacz

      \end{flushleft}

    \end{textblock}

  \end{frame}
}



\newcommand{\GeometryThreeDTwoSpecialEndingSlidesEN}[1]{%
  \begin{frame}[standout]


    \begin{textblock}{11}(1,0.7)

      \begin{flushleft}

        \mdseries

        \footnotesize

        \color{jFrametitleFGColor}

        This content was created as part of a project co-financed by the
        European Union within the framework of the European Social Fund
        POWR.03.05.00-00-Z309/17-00.

      \end{flushleft}

    \end{textblock}





    \begin{textblock}{10}(0,2.2)

      \tikz \fill[color=jBackgroundStyleLight] (0,0) rectangle (12.8,-1.5);

    \end{textblock}


    \begin{textblock}{3.2}(0.7,2.45)

      \includegraphics[scale=0.3]{\FundingLogoColorPictureEN}

    \end{textblock}


    \begin{textblock}{2.5}(4.15,2.5)

      \includegraphics[scale=0.2]{\JULogoColorPictureEN}

    \end{textblock}


    \begin{textblock}{2.5}(6.35,2.4)

      \includegraphics[scale=0.1]{\ZintegrUJLogoColorPictureEN}

    \end{textblock}


    \begin{textblock}{4.2}(8.4,2.6)

      \includegraphics[scale=0.3]{\EUSocialFundLogoColorPictureEN}

    \end{textblock}





    \begin{textblock}{11}(1,4)

      \begin{flushleft}

        \mdseries

        \footnotesize

        \RaggedRight

        \color{jFrametitleFGColor}

        The content of this lecture is made available under a~Creative
        Commons licence (\textsc{cc}), giving the author the credits
        (\textsc{by}) and putting an obligation to share on the same terms
        (\textsc{sa}). Figures and diagrams included in the lecture are
        authored by Paweł Węgrzyn et~al., and are available under the same
        license unless indicated otherwise.\\ The presentation uses the
        Beamer Jagiellonian theme based on Matthias Vogelgesang’s
        Metropolis theme, available under license \LaTeX{} Project
        Public License~1.3c at: \colorhref{https://github.com/matze/mtheme}
        {https://github.com/matze/mtheme}.

        Typographic design: Iwona Grabska-Gradzińska \\
        \LaTeX{} Typesetting: Kamil Ziemian \\
        Proofreading: Wojciech Palacz,
        Monika Stawicka \\
        3D Models: Dariusz Frymus, Kamil Nowakowski \\
        Figures and charts: Kamil Ziemian, Paweł Węgrzyn, Wojciech Palacz

      \end{flushleft}

    \end{textblock}

  \end{frame}





  \begin{frame}[standout]

    \begingroup

    \color{jFrametitleFGColor}

    #1

    \endgroup

  \end{frame}
}



\newcommand{\GeometryThreeDSpecialEndingSlideVideoVerOneEN}{%
  \begin{frame}[standout]

    \begin{textblock}{11}(1,0.7)

      \begin{flushleft}

        \mdseries

        \footnotesize

        \color{jFrametitleFGColor}

        This content was created as part of a project co-financed by the
        European Union within the framework of the European Social Fund
        POWR.03.05.00-00-Z309/17-00.

      \end{flushleft}

    \end{textblock}





    \begin{textblock}{10}(0,2.2)

      \tikz \fill[color=jBackgroundStyleLight] (0,0) rectangle (12.8,-1.5);

    \end{textblock}


    \begin{textblock}{3.2}(0.7,2.45)

      \includegraphics[scale=0.3]{\FundingLogoColorPictureEN}

    \end{textblock}


    \begin{textblock}{2.5}(4.15,2.5)

      \includegraphics[scale=0.2]{\JULogoColorPictureEN}

    \end{textblock}


    \begin{textblock}{2.5}(6.35,2.4)

      \includegraphics[scale=0.1]{\ZintegrUJLogoColorPictureEN}

    \end{textblock}


    \begin{textblock}{4.2}(8.4,2.6)

      \includegraphics[scale=0.3]{\EUSocialFundLogoColorPictureEN}

    \end{textblock}





    \begin{textblock}{11}(1,4)

      \begin{flushleft}

        \mdseries

        \footnotesize

        \RaggedRight

        \color{jFrametitleFGColor}

        The content of this lecture is made available under a Creative
        Commons licence (\textsc{cc}), giving the author the credits
        (\textsc{by}) and putting an obligation to share on the same terms
        (\textsc{sa}). Figures and diagrams included in the lecture are
        authored by Paweł Węgrzyn et~al., and are available under the same
        license unless indicated otherwise.\\ The presentation uses the
        Beamer Jagiellonian theme based on Matthias Vogelgesang’s
        Metropolis theme, available under license \LaTeX{} Project
        Public License~1.3c at: \colorhref{https://github.com/matze/mtheme}
        {https://github.com/matze/mtheme}.

        Typographic design: Iwona Grabska-Gradzińska;
        \LaTeX{} Typesetting: Kamil Ziemian \\
        Proofreading: Wojciech Palacz,
        Monika Stawicka \\
        3D Models: Dariusz Frymus, Kamil Nowakowski \\
        Figures and charts: Kamil Ziemian, Paweł Węgrzyn, Wojciech
        Palacz \\
        Film editing: Agencja Filmowa Film \& Television Production~--
        Zbigniew Masklak

      \end{flushleft}

    \end{textblock}

  \end{frame}
}



\newcommand{\GeometryThreeDSpecialEndingSlideVideoVerTwoEN}{%
  \begin{frame}[standout]

    \begin{textblock}{11}(1,0.7)

      \begin{flushleft}

        \mdseries

        \footnotesize

        \color{jFrametitleFGColor}

        This content was created as part of a project co-financed by the
        European Union within the framework of the European Social Fund
        POWR.03.05.00-00-Z309/17-00.

      \end{flushleft}

    \end{textblock}





    \begin{textblock}{10}(0,2.2)

      \tikz \fill[color=jBackgroundStyleLight] (0,0) rectangle (12.8,-1.5);

    \end{textblock}


    \begin{textblock}{3.2}(0.7,2.45)

      \includegraphics[scale=0.3]{\FundingLogoColorPictureEN}

    \end{textblock}


    \begin{textblock}{2.5}(4.15,2.5)

      \includegraphics[scale=0.2]{\JULogoColorPictureEN}

    \end{textblock}


    \begin{textblock}{2.5}(6.35,2.4)

      \includegraphics[scale=0.1]{\ZintegrUJLogoColorPictureEN}

    \end{textblock}


    \begin{textblock}{4.2}(8.4,2.6)

      \includegraphics[scale=0.3]{\EUSocialFundLogoColorPictureEN}

    \end{textblock}





    \begin{textblock}{11}(1,4)

      \begin{flushleft}

        \mdseries

        \footnotesize

        \RaggedRight

        \color{jFrametitleFGColor}

        The content of this lecture is made available under a Creative
        Commons licence (\textsc{cc}), giving the author the credits
        (\textsc{by}) and putting an obligation to share on the same terms
        (\textsc{sa}). Figures and diagrams included in the lecture are
        authored by Paweł Węgrzyn et~al., and are available under the same
        license unless indicated otherwise.\\ The presentation uses the
        Beamer Jagiellonian theme based on Matthias Vogelgesang’s
        Metropolis theme, available under license \LaTeX{} Project
        Public License~1.3c at: \colorhref{https://github.com/matze/mtheme}
        {https://github.com/matze/mtheme}.

        Typographic design: Iwona Grabska-Gradzińska;
        \LaTeX{} Typesetting: Kamil Ziemian \\
        Proofreading: Wojciech Palacz,
        Monika Stawicka \\
        3D Models: Dariusz Frymus, Kamil Nowakowski \\
        Figures and charts: Kamil Ziemian, Paweł Węgrzyn, Wojciech
        Palacz \\
        Film editing: IMAVI -- Joanna Kozakiewicz, Krzysztof Magda, Nikodem
        Frodyma

      \end{flushleft}

    \end{textblock}

  \end{frame}
}



\newcommand{\GeometryThreeDSpecialEndingSlideVideoVerThreeEN}{%
  \begin{frame}[standout]

    \begin{textblock}{11}(1,0.7)

      \begin{flushleft}

        \mdseries

        \footnotesize

        \color{jFrametitleFGColor}

        This content was created as part of a project co-financed by the
        European Union within the framework of the European Social Fund
        POWR.03.05.00-00-Z309/17-00.

      \end{flushleft}

    \end{textblock}





    \begin{textblock}{10}(0,2.2)

      \tikz \fill[color=jBackgroundStyleLight] (0,0) rectangle (12.8,-1.5);

    \end{textblock}


    \begin{textblock}{3.2}(0.7,2.45)

      \includegraphics[scale=0.3]{\FundingLogoColorPictureEN}

    \end{textblock}


    \begin{textblock}{2.5}(4.15,2.5)

      \includegraphics[scale=0.2]{\JULogoColorPictureEN}

    \end{textblock}


    \begin{textblock}{2.5}(6.35,2.4)

      \includegraphics[scale=0.1]{\ZintegrUJLogoColorPictureEN}

    \end{textblock}


    \begin{textblock}{4.2}(8.4,2.6)

      \includegraphics[scale=0.3]{\EUSocialFundLogoColorPictureEN}

    \end{textblock}





    \begin{textblock}{11}(1,4)

      \begin{flushleft}

        \mdseries

        \footnotesize

        \RaggedRight

        \color{jFrametitleFGColor}

        The content of this lecture is made available under a Creative
        Commons licence (\textsc{cc}), giving the author the credits
        (\textsc{by}) and putting an obligation to share on the same terms
        (\textsc{sa}). Figures and diagrams included in the lecture are
        authored by Paweł Węgrzyn et~al., and are available under the same
        license unless indicated otherwise.\\ The presentation uses the
        Beamer Jagiellonian theme based on Matthias Vogelgesang’s
        Metropolis theme, available under license \LaTeX{} Project
        Public License~1.3c at: \colorhref{https://github.com/matze/mtheme}
        {https://github.com/matze/mtheme}.

        Typographic design: Iwona Grabska-Gradzińska;
        \LaTeX{} Typesetting: Kamil Ziemian \\
        Proofreading: Wojciech Palacz,
        Monika Stawicka \\
        3D Models: Dariusz Frymus, Kamil Nowakowski \\
        Figures and charts: Kamil Ziemian, Paweł Węgrzyn, Wojciech
        Palacz \\
        Film editing: Agencja Filmowa Film \& Television Production~--
        Zbigniew Masklak \\
        Film editing: IMAVI -- Joanna Kozakiewicz, Krzysztof Magda, Nikodem
        Frodyma

      \end{flushleft}

    \end{textblock}

  \end{frame}
}



\newcommand{\GeometryThreeDTwoSpecialEndingSlidesVideoVerOneEN}[1]{%
  \begin{frame}[standout]

    \begin{textblock}{11}(1,0.7)

      \begin{flushleft}

        \mdseries

        \footnotesize

        \color{jFrametitleFGColor}

        This content was created as part of a project co-financed by the
        European Union within the framework of the European Social Fund
        POWR.03.05.00-00-Z309/17-00.

      \end{flushleft}

    \end{textblock}





    \begin{textblock}{10}(0,2.2)

      \tikz \fill[color=jBackgroundStyleLight] (0,0) rectangle (12.8,-1.5);

    \end{textblock}


    \begin{textblock}{3.2}(0.7,2.45)

      \includegraphics[scale=0.3]{\FundingLogoColorPictureEN}

    \end{textblock}


    \begin{textblock}{2.5}(4.15,2.5)

      \includegraphics[scale=0.2]{\JULogoColorPictureEN}

    \end{textblock}


    \begin{textblock}{2.5}(6.35,2.4)

      \includegraphics[scale=0.1]{\ZintegrUJLogoColorPictureEN}

    \end{textblock}


    \begin{textblock}{4.2}(8.4,2.6)

      \includegraphics[scale=0.3]{\EUSocialFundLogoColorPictureEN}

    \end{textblock}





    \begin{textblock}{11}(1,4)

      \begin{flushleft}

        \mdseries

        \footnotesize

        \RaggedRight

        \color{jFrametitleFGColor}

        The content of this lecture is made available under a Creative
        Commons licence (\textsc{cc}), giving the author the credits
        (\textsc{by}) and putting an obligation to share on the same terms
        (\textsc{sa}). Figures and diagrams included in the lecture are
        authored by Paweł Węgrzyn et~al., and are available under the same
        license unless indicated otherwise.\\ The presentation uses the
        Beamer Jagiellonian theme based on Matthias Vogelgesang’s
        Metropolis theme, available under license \LaTeX{} Project
        Public License~1.3c at: \colorhref{https://github.com/matze/mtheme}
        {https://github.com/matze/mtheme}.

        Typographic design: Iwona Grabska-Gradzińska;
        \LaTeX{} Typesetting: Kamil Ziemian \\
        Proofreading: Wojciech Palacz,
        Monika Stawicka \\
        3D Models: Dariusz Frymus, Kamil Nowakowski \\
        Figures and charts: Kamil Ziemian, Paweł Węgrzyn,
        Wojciech Palacz \\
        Film editing: Agencja Filmowa Film \& Television Production~--
        Zbigniew Masklak

      \end{flushleft}

    \end{textblock}

  \end{frame}





  \begin{frame}[standout]


    \begingroup

    \color{jFrametitleFGColor}

    #1

    \endgroup

  \end{frame}
}



\newcommand{\GeometryThreeDTwoSpecialEndingSlidesVideoVerTwoEN}[1]{%
  \begin{frame}[standout]

    \begin{textblock}{11}(1,0.7)

      \begin{flushleft}

        \mdseries

        \footnotesize

        \color{jFrametitleFGColor}

        This content was created as part of a project co-financed by the
        European Union within the framework of the European Social Fund
        POWR.03.05.00-00-Z309/17-00.

      \end{flushleft}

    \end{textblock}





    \begin{textblock}{10}(0,2.2)

      \tikz \fill[color=jBackgroundStyleLight] (0,0) rectangle (12.8,-1.5);

    \end{textblock}


    \begin{textblock}{3.2}(0.7,2.45)

      \includegraphics[scale=0.3]{\FundingLogoColorPictureEN}

    \end{textblock}


    \begin{textblock}{2.5}(4.15,2.5)

      \includegraphics[scale=0.2]{\JULogoColorPictureEN}

    \end{textblock}


    \begin{textblock}{2.5}(6.35,2.4)

      \includegraphics[scale=0.1]{\ZintegrUJLogoColorPictureEN}

    \end{textblock}


    \begin{textblock}{4.2}(8.4,2.6)

      \includegraphics[scale=0.3]{\EUSocialFundLogoColorPictureEN}

    \end{textblock}





    \begin{textblock}{11}(1,4)

      \begin{flushleft}

        \mdseries

        \footnotesize

        \RaggedRight

        \color{jFrametitleFGColor}

        The content of this lecture is made available under a Creative
        Commons licence (\textsc{cc}), giving the author the credits
        (\textsc{by}) and putting an obligation to share on the same terms
        (\textsc{sa}). Figures and diagrams included in the lecture are
        authored by Paweł Węgrzyn et~al., and are available under the same
        license unless indicated otherwise.\\ The presentation uses the
        Beamer Jagiellonian theme based on Matthias Vogelgesang’s
        Metropolis theme, available under license \LaTeX{} Project
        Public License~1.3c at: \colorhref{https://github.com/matze/mtheme}
        {https://github.com/matze/mtheme}.

        Typographic design: Iwona Grabska-Gradzińska;
        \LaTeX{} Typesetting: Kamil Ziemian \\
        Proofreading: Wojciech Palacz,
        Monika Stawicka \\
        3D Models: Dariusz Frymus, Kamil Nowakowski \\
        Figures and charts: Kamil Ziemian, Paweł Węgrzyn,
        Wojciech Palacz \\
        Film editing: IMAVI -- Joanna Kozakiewicz, Krzysztof Magda, Nikodem
        Frodyma

      \end{flushleft}

    \end{textblock}

  \end{frame}





  \begin{frame}[standout]


    \begingroup

    \color{jFrametitleFGColor}

    #1

    \endgroup

  \end{frame}
}



\newcommand{\GeometryThreeDTwoSpecialEndingSlidesVideoVerThreeEN}[1]{%
  \begin{frame}[standout]

    \begin{textblock}{11}(1,0.7)

      \begin{flushleft}

        \mdseries

        \footnotesize

        \color{jFrametitleFGColor}

        This content was created as part of a project co-financed by the
        European Union within the framework of the European Social Fund
        POWR.03.05.00-00-Z309/17-00.

      \end{flushleft}

    \end{textblock}





    \begin{textblock}{10}(0,2.2)

      \tikz \fill[color=jBackgroundStyleLight] (0,0) rectangle (12.8,-1.5);

    \end{textblock}


    \begin{textblock}{3.2}(0.7,2.45)

      \includegraphics[scale=0.3]{\FundingLogoColorPictureEN}

    \end{textblock}


    \begin{textblock}{2.5}(4.15,2.5)

      \includegraphics[scale=0.2]{\JULogoColorPictureEN}

    \end{textblock}


    \begin{textblock}{2.5}(6.35,2.4)

      \includegraphics[scale=0.1]{\ZintegrUJLogoColorPictureEN}

    \end{textblock}


    \begin{textblock}{4.2}(8.4,2.6)

      \includegraphics[scale=0.3]{\EUSocialFundLogoColorPictureEN}

    \end{textblock}





    \begin{textblock}{11}(1,4)

      \begin{flushleft}

        \mdseries

        \footnotesize

        \RaggedRight

        \color{jFrametitleFGColor}

        The content of this lecture is made available under a Creative
        Commons licence (\textsc{cc}), giving the author the credits
        (\textsc{by}) and putting an obligation to share on the same terms
        (\textsc{sa}). Figures and diagrams included in the lecture are
        authored by Paweł Węgrzyn et~al., and are available under the same
        license unless indicated otherwise. \\ The presentation uses the
        Beamer Jagiellonian theme based on Matthias Vogelgesang’s
        Metropolis theme, available under license \LaTeX{} Project
        Public License~1.3c at: \colorhref{https://github.com/matze/mtheme}
        {https://github.com/matze/mtheme}.

        Typographic design: Iwona Grabska-Gradzińska;
        \LaTeX{} Typesetting: Kamil Ziemian \\
        Proofreading: Leszek Hadasz, Wojciech Palacz,
        Monika Stawicka \\
        3D Models: Dariusz Frymus, Kamil Nowakowski \\
        Figures and charts: Kamil Ziemian, Paweł Węgrzyn,
        Wojciech Palacz \\
        Film editing: Agencja Filmowa Film \& Television Production~--
        Zbigniew Masklak \\
        Film editing: IMAVI -- Joanna Kozakiewicz, Krzysztof Magda, Nikodem
        Frodyma


      \end{flushleft}

    \end{textblock}

  \end{frame}





  \begin{frame}[standout]


    \begingroup

    \color{jFrametitleFGColor}

    #1

    \endgroup

  \end{frame}
}











% ---------------------------------------
% Packages, libraries and their configuration
% ---------------------------------------
% \usepackage{graphicx}% Pozwala wstawiać grafikę.
\usepackage{verse}
% \usepackage{verbatim}
% \usepackage{xy} % Pozwala rysować grafy.
\usepackage{tikz-feynman}
\usepackage{chemfig}
% \usepackage{skak}
% \usepackage{lmodern}





% ---------------------------------------
% Configuration for this particular presentation
% ---------------------------------------










% ---------------------------------------------------------------------
\title{Wstęp do \LaTeX} %Spotkanie pierwsze oby nie ostatnie.
\subtitle{Kilka uwag o~jego używaniu}

\author{Kamil Ziemian \\
  \texttt{kziemianfvt@gmail.com}}


\institute{Uniwersytet Jagielloński w Krakowie}
\date[10 XII 2016]{10 grudnia 2016}
% ---------------------------------------------------------------------










% ####################################################################
% Początek dokumentu
\begin{document}
% ####################################################################





% Wyrównanie do lewej z łamaniem wyrazów

\RaggedRight





% ######################################
\maketitle % Tytuł całego tekstu
% ######################################





% ######################################
\section{\TeX{} i~\LaTeX}
% ######################################



% ##################
\begin{frame}
  \frametitle{Za i~przeciw używania \LaTeX a}


  Przeciw
  \begin{enumerate}
    \RaggedRight

  \item Mój kolega miał wygłosić seminarium na temat
    \textit{O~wyższości pisania w~Wordzie nad \LaTeX em}; niestety
    pozostało ono w~sferze planów, więc konkretnych argumentów nie
    znam.

  \item Aby móc go używać, trzeba~się trochę rzeczy nauczyć.

  \item Na początku pliki bez przerwy nie będą ci~się działać.

  \item A~jak już~zadziałają, to nie będą zawierać tego co chciałeś.

  \item Rozdziela formę od treści.

  \end{enumerate}

\end{frame}
% ##################





% ##################
\begin{frame}
  \frametitle{Za i~przeciw używania \LaTeX a}


  Za (ta prezentacja jest zrobiona w~\LaTeX u)
  \begin{enumerate}
    \RaggedRight

  \item Jest naprawdę ładny.

  \item Kiedy~się go nauczysz, większość rzeczy zrobi za ciebie.

  \item Możesz w~nim zmienić praktycznie wszystko co chcesz.

  \item Ponieważ ktoś zapewne już zrobił to za ciebie, więc musisz
    tylko użyć {\color{blue} doktora Google}.

  \item Przy odrobinie sprytu, znacznie upraszcza życie.

  \item Rozdziela formę od treści.

  \end{enumerate}

\end{frame}
% ##################





% ##################
\begin{frame}
  \frametitle{Komiks na dobry początek}


  \begin{figure}

    \centering

    \includegraphics[height=4cm]
    {./PresentationPictures/Programing_languages.jpg}


    \caption{Cała prawda o~\LaTeX u,
      \colorhref{http://www.buzzfeed.com/lukelewis/28-things-only-developers-will-find-funny}{http://www.buzzfeed.com/lukelewis/28-things-only-developers-will-find-funny} (kliknij w ten link) }

  \end{figure}

\end{frame}
% ##################





% ##################
\begin{frame}
  \frametitle{Co można zrobić w~\LaTeX u?}


  Na~przykład
  \begin{itemize}
    \RaggedRight

  \item Tą prezentację.

  \item Pracę licencjacką, magisterską, doktorską, etc.

  \item Artykuły, eseje, felietony.

  \item Sprawozdania na pracownię.

  \item Książkę.

  \item Erratę do~książki ;).

  \item Grafikę matematyczną:

    \begin{center}

      \setlength{\unitlength}{0.65mm}

      \begin{picture}
        (60,40)
        \put(30,20){\vector(1,0){30}}
        \put(30,20){\vector(4,1){20}} \put(30,20){\vector(3,1){25}}
        \put(30,20){\vector(2,1){30}} \put(30,20){\vector(1,2){10}}
        \thicklines \put(30,20){\vector(-4,1){30}}
        \put(30,20){\vector(-1,4){5}} \thinlines
        \put(30,20){\vector(-1,-1){5}}
        \put(30,20){\vector(-1,-4){5}}
      \end{picture}

    \end{center}

  \end{itemize}

\end{frame}
% ##################





% ##################
\begin{frame}
  \frametitle{Co można robić w~\LaTeX u?}


  Na~przykład
  \begin{itemize}
    \RaggedRight

  \item Grafy Feynmana (choć to sprawa niebanalna):

    \begin{center}
      \feynmandiagram [horizontal=a to b]
      { i1 -- [fermion] a -- [fermion] i2, a -- [photon] b, f1 --
        [fermion] b -- [fermion] f2, };
    \end{center}

  \item Wzory chemiczne:

    \begin{center}

      \chemfig{A*5(-B=C-D-E=)}

    \end{center}

  \item Jak~się postarać, to można pisać po~hebrajsku
    (hieroglify też~się da).

  \item Ogólnie rzecz biorąc wszystko, tylko trzeba
    {\color{purple} pogooglować}.

  \end{itemize}

\end{frame}
% ##################





% ##################
\begin{frame}
  \frametitle{Kolena edycja sporu NKFu o~poezję}



  \settowidth{\versewidth}{Zarówno, gdyś spełniona, jak i
    zawiedziona!}{

    \tiny

    \begin{verse}[\versewidth]
      \setlength{\vrightskip}{-10em} \poemlines{30}

      C o w l e y \\
      \vin Nadziejo, której kruchy żywot kona \\
      Zarówno, gdyś spełniona, jak i zawiedziona! \\
      Ty, którą Zło i Dobro niszczy w równej mierze, \\
      A Los wiecznie na rogi alternatyw bierze. \\
      \vin Czczym cieniem jesteś: doszczętnie niknąca \\
      \vin Tak w noc najgłębszą, jak przy blasku słońca. \\
      \vin Ze wszelkich błogosławieństw, które Los nam rodzi, \\
      \vin\vin Żadne cię nie nagrodzi. \\
      Jeśli sądzić po skutkach wszystko, co istnieje, \\
      Najbardziej beznadziejne są ludzkie nadzieje. \\!

      C r a s h a w \\
      \vin Nadziejo! Tyś jest Niebios długiem, Ziemi darem, \\
      Nie zaistniałych rzeczy realnych wymiarem. \\
      Najpewniejsza, choć najmniej uchwytna! Zaiste, \\
      Dzięki tobie Nic nasze jest tak wyraziste. \\
      \vin Chmuro z ognia, i cieniem i światłem będąca, \\
      \vin Tyś naszym życiem w śmierci, w nocy -- blaskiem słońca. \\
      \vin Ze wszelakich nieszczęść, które Los w nadmiarze płodzi, \\
      \vin\vin Żadne ci nie zaszkodzi. \\
      Los na twój widok chowa swoje tępe rogi, \\
      Jak słabowity księżyc w czas poranku błogi. \\!

    \end{verse}
  }



  O~nadziei: dialog w pytaniach i~odpowiedziach pomiędzy
  A.~Cowleyem i~R.~Crashawem, fragment (tłumaczył Stanisław
  Barańczak)

\end{frame}
% ##################





% ##################
\begin{frame}
  \frametitle{Odrobina historii}


  Dzieło Mistrza. W~1977 roku Donald E.~Knuth, legenda informatyki,
  stwierdził,
  że~aby w~pełni wykorzystać potencjał, coraz szerzej stosownego
  wtedy składu cyfrowego oraz przeciwdziałać obniżającej~się jakości
  typograficznej (wprowadzenie nowej lepszej i~prostszej technologi
  zwykle obniża jakość wykonania), stworzył własny język
  programowania i~program do~składu tekstu. Temu programowi nadał
  nazwę \TeX. Kolejne wersje \TeX a~są zbieżne do $\pi$, obecna to
  3.14159265 (to typowe dla Knutha).


  O wymowie. \TeX{} to zapisana w~alfabecie łacińskim pierwsza sylaba
  ($\tau \varepsilon \chi$) słowa $\tau \varepsilon \chi \nu \eta$ (techne), od~którego pochodzi słowo technika
  i~technologia. Po grecku to oznacza czynności takie jak budowa statków,
  kowalstwo, malowanie, rzeźbienie czy hodowla koni. $\pi o \iota \eta \sigma \iota \zeta$,
  \textit{poiesis} = gr.~robić, jest czymś innym, może dlatego, że~jak
  powiada Platon w~pierwszym przykładzie fanowskiego dziennikarstwa
  muzycznego, dialogu „Ion”, poeci nic nie umieją.

\end{frame}
% ##################





% ##################
\begin{frame}
  \frametitle{Odrobina historia}


  Przybywa Lampart.
  Zapewne dlatego, że~pisanie w~samym \TeX u jest zbyt skomplikowane
  jak~na~tworzenie plików tekstowych, na początku lat 80 XX~wieku,
  Leslie Lamport tworzy nakładkę na~\TeX a, która zajmuje~się za~nas
  większością \TeX nicznych szczegółów (teraz możecie odetchnąć
  z~ulgą). Otrzymany w~ten sposób system składu nosi nazwę \LaTeX,
  obecna wersja to~\LaTeX{} 2$_{ \varepsilon }$.

  Wymowa. \textit{La} jest od Lamporta, stąd powinno~się mówić
  \textit{la-tech} ewentualnie \textit{lej\dywiz tech}. Sam Donald Knuth
  nie wiedział jak popranie należy wymawiać nazwę. \LaTeX{}
  2$_{ \varepsilon }$ czyta się jako \textit{la-tech dwa~i}.

\end{frame}
% ##################





% ##################
\begin{frame}
  \frametitle{Odrobina historia}


  Hello World!

  \begin{figure}

    \centering

    \includegraphics[height=4cm]
    {./PresentationPictures/Donald_Ervin_Knuth.jpg}
    \includegraphics[height=4cm]{./PresentationPictures/Leslie_Lamport.jpg}


    \caption{Donald Ervin Knuth (ur. 1938), Leslie Lamport
      (ur. 1941)}

  \end{figure}

\end{frame}
% ##################





% ##################
\begin{frame}
  \frametitle{Zanim zaczniemy}


  Typy procesorów tekstu (brr, co za groźne nazwa)
  \begin{itemize}
    \RaggedRight

  \item WYSIWYG = \textit{what you see is what you get} (to co
    widzisz, jest tym co~dostaniesz). Przykłady: MS Word,
    OpenOffice.

  \item WYSIWYM = \textit{what you see is what you mean} (to co
    widzisz, to jest to o~czym myślałeś). Przykłady: \LaTeX.

  \end{itemize}



  Tak naprawdę
  jeśli zaczynasz pisać w~\LaTeX u, to jest to raczej WYWINWYG =
  \textit{what you want is not what you get} (to czego chciałeś, to
  nie jest to co dostałeś). Ale~od tego~są te~warsztaty, aby~ułatwić
  wam przejście tego trudnego początkowego etapu. Jak w~takim
  wypadku wygląda plika \LaTeX a?

\end{frame}
% ##################










% ######################################
\section{Jak pisać w~\LaTeX u?}
% ######################################



% ##################
\begin{frame}
  \frametitle{Kilka rad na~dobry początek}


  Mądrości programistów.
  Tworzenie tekstu w~\LaTeX u to trochę jak programowanie, więc
  warto pamiętać o~kilku mądrościach programistów.
  \begin{itemize}
    \RaggedRight

  \item Najtrudniejsza rzecz to skompilować pierwszy tekst do
    PDFa. Potem jest już z~górki.

  \item Aby nauczyć~się pisać w~\LaTeX u, trzeba pisać w~\LaTeX u.

  \item Twój główny wróg na samym początku to literówki,
    np.~\textbf{\textbackslash beign} zamiast \textbf{\textbackslash
      begin}. Nie zniechęcaj~się tym.

  \item To jak „kody” \LaTeX a wygląda ma znaczenie. Pisz go
    tak, by w~jakimś stopniu oddawał logikę tekstu.

  \end{itemize}

  Przykład. Niech będzie nim ten slajd.

\end{frame}
% ##################





% ##################
\begin{frame}
  \frametitle{Kilka rad na~dobry początek}


  Mądrości programistów
  \begin{itemize}
    \RaggedRight

  \item Naucz~się pisać bezwzrokowo, to wcale nie jest trudne.

  \item Dobre środowisko to skarb. Poznaj jego podstawowe skróty
    klawiszowe i~możliwości.

  \end{itemize}

  Polecane środowiska
  \begin{enumerate}
    \RaggedRight

  \item \TeX Maker;

  \item Kile (warto połączyć z~następnym);

  \item Share\LaTeX.com,
    \colorhref{https://www.sharelatex.com/}{https://www.sharelatex.com/};

  \item WinEdit;

  \item OverLeaf.com,
    \colorhref{https://www.overleaf.com/}{https://www.overleaf.com/};

  \item Vim z~\LaTeX suite i~Vim-\LaTeX;

  \item Lyx;

  \item Emacs z~AUC\TeX-em.

  \end{enumerate}

\end{frame}
% ##################





% ##################
\begin{frame}[fragile]
  \frametitle{Nasz pierwszy tekst}


  Plik źródłowy.
  W~edytorze tekstu (MS Word, OpenOffice~się nie nadają) należy
  utworzyć plik z~rozszerzeniem „.tex”, tak zwany plik źródłowy.
  Niech nosi nazwę „Nasz-pierwszy-tekst-w-LaTeXu.tex”. Niektóre
  dystrybucje \LaTeX a nie zaakceptują pliku „Nasz pierwszy tekst
  w~LaTeXu.tex”.

  Kompilacja.
  Aby powstał gotowy dokument, najlepiej w~formacie PDF, musimy
  skompilować (groźne słowo związane z~takimi rzeczami jak C/C++
  i~Java, ale~nie musicie~się tego bać) plik źródłowy. Zwykle to~się
  nam nie uda i~kompilator wyrzuci komunikat o~błędach.

  W~dobrze skonfigurowany środowisku.
  Wystarczy wcisnąć jeden klawisz, aby~skompilować plik i~wieloma
  rzeczami nie trzeba~się martwić. W~\TeX Makerze jest to~F1.

\end{frame}
% ##################





% ##################
\begin{frame}[fragile]
  \frametitle{Nasz pierwszy tekst}


  Klasyczna szkoła
\begin{verbatim}
\documentclass{article}



\begin{document}



Hello World!



\end{document}
\end{verbatim}

\end{frame}
% ##################





% ##################
\begin{frame}[fragile]
  \frametitle{Nasz pierwszy tekst}


  Nowa szkoła (w~pierwszej linii jest L2TABU)
\begin{verbatim}
\RequirePackage[l2tabu, orthodox]{nag}
\documentclass{article}



\begin{document}



Hello World!



\end{document}
\end{verbatim}

\end{frame}
% ##################





% ##################
\begin{frame}
  \frametitle{Wynik kompilacji}


  Pliki generowane.
  \LaTeX{} tworząc plik PDFa produkuje masę plików z~rozszerzeniami
  „.aux”, „.bbl”, „.blg”, „.log”, „.nav”, na~szczęście
  w~większości przypadków nie musimy~się nimi przejmować.

  Ale czasem. Rzadko, bo rzadko, ale~się zdarza, że~choć wszystko w~pliku
  źródłowym jest~dobrze, to~z~powodu błędu w~którymś z~tych
  dodatkowych plików, kompilacja nie może dojść do skutku. Wtedy
  trzeba usunąć je wszystkie i~skompilować jeszcze raz plik
  źródłowy.

  \textbf{Uwaga.}
  Pamiętajcie by~przy okazji nie usunąć pliku z~rozszerzeniem
  \textbf{„.tex”!!!} Dobre środowisko ma wbudowaną opcję, usunięcia
  wszystkich zbędnych plików za nas. Np.~w~\TeX Makerze mamy: Narzędzia
  $\to$ Wyczyść.

\end{frame}
% ##################





% ##################
\begin{frame}[fragile]
  \frametitle{Nasz pierwszy tekst}


  Podział pliku źródłowego
  \begin{enumerate}
    \RaggedRight

  \item Preambuła.

  \item Część główna.

  \end{enumerate}



  Preambuła.
  Preambuła zaczyna~się w~pierwszej linii pliku, zaś kończy na
  {\color{purple} \verb+\begin{document}+}. Zawiera informacje,
    które służą \LaTeX owi stworzenia gotowego tekstu, rozszerzają
    jego standardowe możliwości (na chyba nieograniczoną liczbę sposobów)
    oraz komendy które sami zdefiniowaliśmy.



\begin{verbatim}
\RequirePackage[l2tabu, orthodox]{nag}
\documentclass{article}

\begin{document}
\end{verbatim}


\end{frame}
% ##################





% ##################
\begin{frame}[fragile]
  \frametitle{Co robi ta preambuła?}


  Linia po linii.
  \begin{enumerate}
    \RaggedRight

  \item \verb+\RequirePackage[l2tabu, orthodox]{nag}+~-- na razie
    najlepiej potraktować to jako magię dla zaawansowanych.
    W~procesie kompilacji wykrywa przestarzała lub niepotrzebne
    paczki, które od~czasów starożytnych mistrzów \LaTeX a kopiujemy
    do~swojej preambuły oraz~błędnie stosowane komendy \LaTeX a,
    wszystko to~według standardów opisanych w~\textit{An essential guide
      to \LaTeX$2_{ \varepsilon }$ usage} Marka Trettina (popularnie
    l2tabu,
    \colorhref{ftp://sunsite.icm.edu.pl/pub/CTAN/info/l2tabu/english/l2tabuen.pdf}{ftp://sunsite.icm.edu.pl/pub/CTAN/info/l2tabu/english/l2tabuen.pdf}).
    Np.~w~tej prezentacji \textbf{nag} wykrył, że~paczka „epsfig” jest
    przestarzała i~należy ją zmienić na „graphicx”.

  \item \verb+\documentclass{article}+~-- definiuje klasę dokumentu
    (co za odkrycie), czyli najbardziej ogólne reguły tworzenia
    danego dokumentu przez \LaTeX a. Podstawowe, klasyczne klasy
    \LaTeX a to \textbf{article}, \textbf{report}, \textbf{book},
    \textbf{letter}, plus niekończenie wiele dodatkowych, a~jeśli jesteś
    odpowiednio dobry, możesz napisać też swoją.

    Ja używam \textbf{article} w~99.99\% przypadków.

  \end{enumerate}

\end{frame}
% ##################





% ##################
\begin{frame}[fragile]
  \frametitle{Co robi ta preambuła?}


  Polecenia \LaTeX a.
  \LaTeX, jak przystało na język programowania, posiada odpowiednie
  polecenia, które zawsze zaczynają~się od znaku „\textbackslash”
  (backslash, „w-tył-ciach”). Co zaczyna~się symbolem
  „\textbackslash” jest poleceniem \LaTeX a. Widzimy więc,
  że~poleceniami~są \verb+\RequirePackage[l2tabu, orthodox]{nag}+
  i~\verb+\documentclass{article}+.



  Polecenia w~preambule. Najbardziej podstawowe wyglądają tak

  \begin{center}

    \verb+\nazwa-polecenia+[\textit{argumenty-dodatkowe}]
    \{\textit{argumenty-obowiązkowe}\}

  \end{center}

  Argumenty dodatkowe należy oddzielać przecinkami i~można podawać
  w~dowolnej kolejności.

\end{frame}
% ##################





% ##################
\begin{frame}[fragile]
  \frametitle{Pierwsza przeróbka preambuły}


  Polecenia w~preambule.
  Najbardziej podstawowe wyglądają tak

  \begin{center}

    \verb+\nazwa-polecenia+[\textit{argumenty-dodatkowe}]
    \{\textit{argumenty-obowiązkowe}\}

  \end{center}

  Argumenty dodatkowe należy oddzielać przecinkami i~można podawać
  w~dowolnej kolejności.

  Zróbmy coś takiego. \verb+\documentclass[a4paper, 12pt]{article}+
  Widzicie różnicę? Pt = point, jednak ze standardowych jednostek
  składu tekstu, równa 1/72 cala. Cal to 2.54 centymetra, więc
  w~bardziej zrozumiałych jednostkach 1 pt = 3.52 mm.

\end{frame}
% ##################





% ##################
\begin{frame}[fragile]
  \frametitle{Wróćmy do~tekstu}


  Część główna. Zawiera tekst, który (po obrobieniu przez \LaTeX a)
  znajdzie~się w~gotowym dokumencie.

\begin{verbatim}
\begin{document}



Hello World!



\end{document}
\end{verbatim}

\end{frame}
% ##################





% ##################
\begin{frame}
  \frametitle{Białe znaki w~\LaTeX u}


  Białe znaki to \textbf{spacja}, \textbf{tab} i~\textbf{enter}. \LaTeX{}
  traktuje jedną spację i~100 spacji tak samo: jako jeden odstęp.
  Tab traktuje jako spację. Przejście do nowej linii, {\color{red} traktuje
    jako jeden odstęp.} Jedną pustą linię traktuje jako komendę,
  że~w~tym miejscu ma {\color{red} skończyć akapit} (częsty błąd
  początkujących), zaś~100 pustych linii jak jedną pustą linię.

\end{frame}
% ##################





% ##################
\begin{frame}
  \frametitle{Białe znaki w~\LaTeX u}


  Organizacja tekstu za pomocą białych znaków.
  Plik dobrze napisany, to plik który łatwiej~się poprawia. A~każdy
  musi go~poprawiać.
  \begin{itemize}
    \RaggedRight

  \item W~preambule białe znaki nie mają znaczenie, co nie
    znaczy, że~nie należy ich używać. Wręcz przeciwnie!

  \item Ponieważ przejście do~nowej to tyle samo co spacja,
    warto dzielić tekst na~krótsze linie (ja stosuje standard 79
    znaków). Dzięki temu nie musimy scrolować tekstu by~go zmienić,
    wszystko jest na ekranie.
    Poza tym komentarze lepiej działają, ale~o~tym za chwilę.

  \item Ja wyznaję zasadę, że~dwie ważne części pliku źródłowego
    powinien oddzielać odstęp 3~pustych linii.

  \item Pomiędzy mniej ważnymi fragmentami powinny znajdować~się
    2~lub~1 puste linie. Przy założeniu, że~przejście do~nowego
    akapitu nie psuje wyglądu tekstu.

  \end{itemize}

\end{frame}
% ##################





% ##################
\begin{frame}[fragile]
  \frametitle{Przykład}


  Stąd taki dziki wygląd tego pliku

\begin{verbatim}
\RequirePackage[l2tabu, orthodox]{nag}
\documentclass{article}



\begin{document}



Hello World!



\end{document}
\end{verbatim}

\end{frame}
% ##################





% ##################
\begin{frame}
  \frametitle{Białe znaki w~\LaTeX u}


  Organizacja tekstu za pomocą białych znaków.
  Plik dobrze napisany, to plik który łatwiej~się poprawia. A~każdy
  musi go~poprawiać.
  \begin{itemize}
    \RaggedRight

  \item Ponieważ przejście do~nowej to tyle samo co spacja,
    warto dzielić tekst na~krótsze linie (ja stosuje standard 79
    znaków). Dzięki temu nie musimy scrolować tekstu by~go zmienić,
    wszystko jest na ekranie.
    Poza tym komentarze lepiej działają, ale~o~tym za chwilę.

  \item Ja wyznaję zasadę, że~dwie ważne części pliku źródłowego
    powinien oddzielać odstęp 3~pustych linii.

  \item Pomiędzy mniej ważnymi fragmentami powinny znajdować~się
    2~lub~1 puste linie. Przy założeniu, że~przejście do~nowego
    akapitu nie psuje wyglądu tekstu.

  \end{itemize}



  Ważne. Nie musicie tego robić tak jak ja, jednak gorąco polecam przyjąć
  jakiś standard. To~niezmiernie ułatwia pracę z~plikiem źródłowym.

\end{frame}
% ##################





% ##################
\begin{frame}[fragile]
  \frametitle{Polskie znaki}


  Na razie pewnie nie da~się ich pisać. \LaTeX{} został stworzony do pracy
  z~językiem angielskim, stąd aby~używać wygodnie znaków polskich, należy
  rozszerzyć jego możliwości.

  Mądrość starożytnych, znaleziona w~bibliotece dr.~Google. Powiada
  że~należy tak zmodyfikować preambułę



\begin{verbatim}
\documentclass[a4paper,11pt]{article}
\usepackage[utf8]{inputenc}
\usepackage[polish]{babel}
\usepackage[MeX]{polski}
\end{verbatim}



  \textbackslash usepackage. Dosłownie „użyj paczki”. Importuje paczkę,
  która rozszerza możliwości bazowego \LaTeX a.

\end{frame}
% ##################





% ##################
\begin{frame}[fragile]
  \frametitle{Co warto zawsze mieć w~preambule?}


  Według mnie to
\begin{verbatim}
\let\lll\undefined
\usepackage{amsmath, amsfonts, amssymb, amscd, amsthm}
\usepackage{calrsfs}
\usepackage{xcolor}
\end{verbatim}

\end{frame}
% ##################





% ##################
\begin{frame}[fragile]
  \frametitle{Co warto zawsze mieć w~preambule?}


\begin{verbatim}
\usepackage{vmargin}
% ------------------------------------------------------
% MARGINS
% ------------------------------------------------------
\setmarginsrb
{ 0.7in} % left margin
{ 0.6in} % top margin
{ 0.7in} % right margin
{ 0.8in} % bottom margin
{  20pt} % head height
{0.25in} % head sep
{   9pt} % foot height
{ 0.3in} % foot sep

\usepackage{hyperref}
\usepackage{cleveref}
\end{verbatim}

\end{frame}
% ##################





% ##################
\begin{frame}[fragile]
  \frametitle{Polecenia w~części głównej}


  Składnia. Polecenie zaczyna~się „\textbackslash” zaś kończy
  \textbf{spacją} lub nawiasem wąsatym „\}”. Spacja w~pierwszym przypadku
  \textbf{nie oznacza} odstępu, lecz koniec polecenia.

  Wpiszcie taki tekst \verb+\TeX i~\LaTeX to \textit{nie} jest to samo.+
  Dlaczego?

  WYWINWYG~= to czego chcesz, to nie jest to co dostajesz. Ponieważ
  spacja po \verb+\TeX+ nie oznacza odstępu, tylko koniec polecenia
  \LaTeX{} rozumiem, że~odstępu ma tam nie być. Danie dwóch spacji
  nic nie da, bo dla niego 1 spacja = $N$ spacji = koniec komendy.
  Są różne szkoły radzenia sobie z~tym:
  \begin{itemize}

  \item[--] \verb+\TeX{} i~\LaTeX{} to+;

  \item[--] \verb+{\TeX} i~{\LaTeX} to+.

  \end{itemize}

\end{frame}
% ##################





% ##################
\begin{frame}[fragile]
  \frametitle{Wdowy i~sieroty}


  Po co ta $\sim$? Przyimki takie jak „a”, „w”, „u”, „z”, czy też zaimek
  zwrotny „się” źle wyglądają na końcu lub początku linii, bo~nie mają
  sensu bez słów które~są zaraz obok. Podobno określa~się je jako wdowy
  i~sieroty. \LaTeX{} sam decyduje jak rozłożyć graficznie słowa
  „na~kartce”, więc często produkuje takie kwiatki. Aby temu zapobiec,
  należy użyć \textbf{twardej spacji}. Od zwykłe spacji
  różni~się tym, że~\LaTeX{} zrobi wszystko, by~te słowa
  znalazły~się w~jednej linii, oznacza ją właśnie \verb+~+.

  Porównanie
  \begin{itemize}

  \item[--] \textbf{Poprawnie:} \verb+zrobi~się+.

  \item[--] \textbf{Niepoprawnie:} \verb+zrobi się+.

  \end{itemize}

  Dłuższe wyrażenia. Jeśli chcemy by jakiś ciąg słów nigdy nie został
  rozbity między dwie linie, to~piszemy np.~\verb+\mbox{coś tam, coś tam}+.

\end{frame}
% ##################





% ##################
\begin{frame}[fragile]
  \frametitle{\% --~nasz najlepszy przyjaciel w~\LaTeX u}


  O~co chodzi z~tym~\%? Jak każdy dobry język programowania, \LaTeX{}
  posiada komentarze. Wszystko od znaku \%, do końca linii jest zupełnie
  przez niego ignorowane. Jeśli chcemy mieć znak procenta w~tekście trzeba
  napisać \verb+\%+.

  Podstawowe rady
  \begin{itemize}

  \item Jeżeli dopiero zaczynasz pisać w~\LaTeX u i~uznałeś,
    że~chcesz usunąć tekst z~pliku źródłowego, to~najlepiej go
    wykomentuj. Zaraz dojdziesz do~wniosku, że~by ci~się przydał.

  \item Komentarz to najlepszy debugger na świecie, dzięki
    znanej już Platonowi, to jest bodaj w~dialogu \textit{Sofista},
    metodzie bisekcji. Za~sekundę do~tego wrócimy.

  \end{itemize}

  Skorzystaj z~możliwości swojego środowiska.
  Każde środowisko ma skrót klawiszowy odpowiedzialne
  za~wykomentowanie i~odkomentowanie dowolnego bloku tekstu.
  Dla~\TeX Makera: Ctrl-T, Ctrl-U.

\end{frame}
% ##################





% ##################
\begin{frame}[fragile]
  \frametitle{Plik~się nie kompiluje}


  Początkujący powinien powiesić to nad łóżkiem. Skróty klawiszowe
  do~komentowania i~odkomentowania bloków tekstu. Dla \TeX Makera:
  \begin{itemize}

  \item wykomentuj~-- Ctrl-T;

  \item odkomentuj~-- Ctrl-U.

  \end{itemize}

  Skoro już przy tym jesteśmy, to~trzeba powiedzieć o~trybie
  matematycznym.

  Jeśli mamy błąd kompilacji i~nie wiesz jaki
  \begin{enumerate}

  \item Kompilator podaje nam linię w~której jest błąd, ale
    często~się myli w~zakresie 10~linii. Jeśli linie~są relatywnie
    krótkie, to~jest większa szansa, że~poda poprawną.

  \item Odkomentowujemy tekst, gdzie kompilator wskazuje błąd.

  \item Kompilujemy jeszcze raz. Jeśli błąd jest wciąż obecny, to
    znaczy, że~błędów jest więcej albo~wykomentowaliśmy zły
    fragment.

  \item Powtarzamy krok 2, aż~kompilator zadziała.

  \end{enumerate}

\end{frame}
% ##################





% ##################
\begin{frame}[fragile]
  \frametitle{Plik~się nie kompiluje}


  Jeśli mamy błąd kompilacji i~nie wiesz jaki
  \begin{enumerate}

  \item Kompilator podaje nam linię w~której jest błąd, ale
    często~się myli w~zakresie 10~linii. Jeśli linie~są relatywnie
    krótkie, to~jest większa szansa, że~poda poprawną.

  \item Odkomentowujemy tekst, gdzie kompilator wskazuje błąd.

  \item Kompilujemy jeszcze raz. Jeśli błąd jest wciąż obecny, to
    znaczy, że~błędów jest więcej albo~zakomentowaliśmy zły fragment.

  \item Powtarzamy krok 2, aż~kompilator zadziała.

  \item Przyglądamy~się krótkiemu fragmentowi zakomentowanego
    tekstu, jeśli~znajdziemy w~nim błąd to go poprawiamy, jeśli nie
    to go~odkomentowujemy.

  \item Jeśli zadziałał, to powtarzamy krok 5, jeśli nie, to znaczy,
    że~przeoczyliśmy błąd i~musimy dokładnie przeanalizować
    odkomentowany fragment.

  \item Stosujemy tę procedurę, aż~wszystko działa.

  \end{enumerate}

\end{frame}
% ##################





% ##################
\begin{frame}[fragile]
  \frametitle{Warto robić wcięcia}

  Dlaczego warto skracać linie? Relatywnie krótkie linie w~pliku źródłowym
  pozwalają pracować na małych fragmentach tekstu, a~znacznie łatwiej
  znaleźć błąd w~krótkim fragmencie tekstu niż w~długim.

  Przykład trybu matematycznego
  \begin{equation}
    \label{eq:Wstep-do-LaTeXa-01}
    \epsilon > 0, \quad \varepsilon > 0.
  \end{equation}

  Aby to otrzymać piszemy
\begin{verbatim}
\begin{equation}
  \label{eq:Wstep-do-LaTeXa-01}
  \epsilon > 0, \quad \varepsilon > 0.
\end{equation}
\end{verbatim}

  Wcięcia bardzo ułatwiają później pracę z~\LaTeX em, dobre
  środowisko zrobi je za was. A~o~trybie matematyczny, więcej opowie
  wam Wojtek.

\end{frame}
% ##################





% ##################
\begin{frame}[fragile]
  \frametitle{Tryb matematyczny}


  Chcemy mieć
  \begin{equation}
    \label{eq:Wstep-do-LaTeXa-02}
    \frac{ \partial^{ 2 } u( x, t ) }{ \partial x^{ 2 } }
    - \frac{ 1 }{ c^{ 2 } }
    \frac{ \partial^{ 2 } u( x, t ) }{ \partial t^{ 2 } }
    = 0.
  \end{equation}
  Będzie trochę roboty, ale~potem pokażemy jak zmniejszyć jej
  ilość do~rozsądnych rozmiarów.

  Słownik
  \begin{itemize}

  \item potęga --~\verb+a^{ 2 }+ = $a^{ 2 }$;

  \item ułamek --~\verb+\frac{ 1 }{ 2 }+ = $\frac{ 1 }{ 2 }$;

  \item pochodna cząstkowa --~\verb+\partial+ = $\partial$.

  \end{itemize}

\end{frame}
% ##################





% ##################
\begin{frame}[fragile]
  \frametitle{Tryb matematyczny}


  Musimy napisać
\begin{verbatim}
\begin{equation}
  \label{eq:Wstep-do-LaTeXa-02}
  \frac{ \partial^{ 2 } u( x, t ) }{ \partial x^{ 2 } }
  - \frac{ 1 }{ c^{ 2 } } \frac{ \partial^{ 2 } u( x, t ) }
  { \partial t^{ 2 } }  = 0
\end{equation}
\end{verbatim}

  Większość białych znaków jest tu niepotrzebna, dodałem je dla
  własnej wygody. Tak samo podział na linie. W~trybie
  matematycznym {\color{red} nie wolno} zostawiać pustych linii!!!

\end{frame}
% ##################





% ##################
\begin{frame}[fragile]
  \frametitle{Tryb matematyczny}


  Jeśli mamy błąd kompilacji i~nie wiesz jaki.
  \begin{enumerate}

  \item Kompilator podaje nam linię w~której jest błąd, ale
    często~się myli o~jakieś 10~linii. Jeśli linie~są relatywnie
    krótkie, to~jest większa szansa, że~poda poprawną.

  \item Odkomentowujemy tekst, gdzie kompilator wskazuje błąd.

  \item Kompilujemy jeszcze raz. Jeśli błąd jest wciąż obecny, to
    znaczy, że~błędów jest więcej albo ~zakomentowaliśmy zły
    fragment.

  \item Powtarzamy krok 2, aż~kompilator zadziała.

  \item Przyglądamy~się krótkiemu fragmentowi zakomentowanego
    tekstu, jeśli~znajdziemy w~nim błąd to go poprawiamy, jeśli
    nie to go~odkomentowujemy.

  \item Jeśli zadziałał, to powtarzamy krok 5, jeśli nie, to
    znaczy, że~przeoczyliśmy błąd i~musimy dokładnie
    przeanalizować odkomentowany fragment.

  \item Stosujemy tę procedurę, aż~wszystko działa.

  \end{enumerate}

\end{frame}
% ##################





% ##################
\begin{frame}[fragile]
  \frametitle{Uprośćmy sobie życie}


  Dlaczego warto skracać linie?
  Relatywnie krótkie linie w~pliku źródłowym pozwalają pracować na
  małych fragmentach tekstu, a~znacznie łatwiej znaleźć błąd w~krótkiej
  części tekstu niż w~długiej.


  Przy odrobinie sprytu życie jest prostsze
  Chcemy mieć % (trochę to brzydkie, ale~kod prosty)
  \begin{equation}
    \label{eq:Wstep-do-LaTeXa-03}
    \epsilon > 0, \varepsilon > 0.
  \end{equation}
  Aby to uzyskać trzeba napisać
\begin{verbatim}
\begin{equation}
  \label{eq:Wstep-do-LaTeXa-03}
  \epsilon > 0, \varepsilon > 0.
\end{equation}
\end{verbatim}

  Nie ma potrzeby tyle pisać.

\end{frame}
% ##################





% ##################
\begin{frame}[fragile]
  \frametitle{Uprośćmy sobie życie}


  \textbackslash newcommand. Wstawiamy do preambuły, najlepiej po ostatnim
  \verb+\usepackage+, linię
\begin{verbatim}
\newcommand{\eps}{\epsilon}
\end{verbatim}

  Jeżeli dobrze rozumiem, a~to rozumowanie jeszcze mnie nie
  zawiodło, urządzenie zwane preprocesorem zamieni w~tekście „na
  chama” każde wystąpienie \verb+\eps+ na \verb+\epsilon+.

  Analogicznie
\begin{verbatim}
\newcommand{\veps}{\varepsilon}
\end{verbatim}

  Od razu lepiej:).

\end{frame}
% ##################





% ##################
\begin{frame}[fragile]
  \frametitle{Uprośćmy sobie życie}


  Pochodna cząstkowa. Wstawiamy do~preambuły
\begin{verbatim}
\newcommand{\pd}[3]{\frac{ \partial^{ #1 } { #2 } }
{ \partial { #3 }^{ #1 } }}
\end{verbatim}

  Lepiej nie rozdzielać, choć można, tej komendy na dwie linie jak
  powyżej, zrobiłem to, aby tekst ładnie mieścił~się na~slajdzie.
  Nawiasy wąsate wokół „\#liczba”, nie~są konieczne, ale~bez
  nich \LaTeX{} czasem protestuje.

  Piszemy
\begin{verbatim}
\pd{ 2 }{ u( x, t ) }{ x }
- \frac{ 1 }{ c^{ 2 } } \pd{ 2 }{ u( x, t ) }{ t } = 0
\end{verbatim}
  Która wersja jest prostsza?

\end{frame}
% ##################





% ##################
\begin{frame}[fragile]
  \frametitle{Uprośćmy sobie życie}


  Objaśnienie
\begin{verbatim}
\newcommand{\nazwa-komendy}[ilość-argumentów]
{treść komendy #1 #2 #3,...}
\end{verbatim}

  Nazwa komendy to chyba jasne. Jeśli komenda ma nie przyjmować
  żadnych argumentów, jak komenda \verb+\LaTeX+, to pomijamy
  nawias kwadratowy. Jeśli ma przyjmować np.~5 argumentów, to
  piszemy [5]. W~treści komendy \LaTeX{} podstawi „na chama”
  pierwszy argument za „\#1”, drugi za~„\#2”, etc.

  Tutaj też lepiej nie rozbijać \verb+\newcommand+ na~dwie linie.

  Użycie komendy
\begin{verbatim}
\nazwa-komendy{pierwszy-argument}{drugi-argument}...
\end{verbatim}

\end{frame}
% ##################





% ##################
\begin{frame}[fragile]
  \frametitle{Uprośćmy sobie życie}


  Czy rozumiecie już jak działa
\begin{verbatim}
\newcommand{\pd}[3]{\frac{ \partial^{ #1 } { #2 } }
{ \partial { #3 }^{ #1 } } }
\pd{ 2 }{ u( x, t ) }{ x }
\end{verbatim}
  ta komenda?

  {\color{red} Ważne}, \verb+\pd{ }{ u( x, t ) }{ x }+ też działa.

  Inny przykład. \verb+\newcommand{\sizeOne}{8pt}+~-- definicja rozmiaru
  czcionki.

  Osobiste doświadczenie.
  Potrafię mieć 40 \verb+\newcommand+ w~jednym pliku źródłowym,
  tak bardzo upraszczają mi życie. Np.~jeśli chcę mieć znak $\to$,
  to po co mam pisać \verb+\rightarrow+, kiedy mogę \verb+\ra+?

\end{frame}
% ##################





% ##################
\begin{frame}
  \frametitle{Nie pytaj co możesz zrobić dla swojego środowiska}


  To środowisko ma zrobić coś dla ciebie :) Podstawowe skróty i~polecenia,
  na przykładzie \TeX Makera.
  \begin{itemize}

  \item Ctrl-T~-- zakomentuj blok tekstu;

  \item Ctrl-U~-- odkomentuj blok tekstu;

  \item Ctrl-$>$~-- wcięcie bloku;

  \item Ctrl-$<$~-- usunięcie wcięcia bloku;

  \item Ctrl-R~-- zastąp;

  \item Ctrl-F~-- znajdź;

  \item Ctrl-M~-- znajdź następny;

  \item Ctrl-G~-- przejdź do linii;

  \item Narzędzia $\to$ Wyczyść.
  \end{itemize}

  Sprawdź jakie są w~twoim i~ich używaj :).

\end{frame}
% ##################





% ##################
\begin{frame}
  \frametitle{Co warto zrobić dalej?}


  Rady
  \begin{itemize}

  \item Posłuchać następnych wystąpień :).

  \item Przeczytać \textit{Nie za~krótkie wprowadzenie do~systemu
      \LaTeX a}.

  \item Nauczyć~się Bib\TeX a (chyba, że~jest już jakieś lepsze
    rozwiązanie).

  \item Pogooglować i~poeksperymentować.

  \end{itemize}

\end{frame}
% ##################





% ##################
\begin{frame}
  \frametitle{Literatura}


  Podstawowa
  \begin{itemize}

  \item Tobias Oetiker et.al, \textit{Nie za~krótkie wprowadzenie
      do~systemu \LaTeX{}~2$_{ \varepsilon }$''},
    \colorhref{http://texdoc.net/texmf-dist/doc/latex/lshort-polish/lshort2e.pdf}{http://texdoc.net/texmf-dist/doc/latex/lshort-polish/lshort2e.pdf}.

  \item \textit{\LaTeX. Wikibooks},
    \colorhref{https://en.wikibooks.org/wiki/LaTeX}.

  \item \emph{Share\LaTeX{} guides},
    \colorhref{https://www.sharelatex.com/learn/Main_Page}.

  \item Włodzimierz Macewicz, Stanisław Wawrykiewicz,
    \textit{Wirtualna Akademia \TeX owa},
    \colorhref{http://www.gust.org.pl/projects/wirtualna-akademia-texowa}
    {http://www.gust.org.pl/projects/wirtualna-akademia-texowa}.

  \item Scott Pakin, \textit{The Comprehensive \LaTeX{} Symbol
      List},
    \colorhref{http://tug.ctan.org/info/symbols/comprehensive/symbols-a4.pdf}
    {http://tug.ctan.org/info/symbols/comprehensive/symbols-a4.pdf}.

  \end{itemize}

  Zaawansowana (sam jej nie rozumiem).
  \begin{itemize}

  \item Christian Feuers\"{a}nger, \textit{Notes On Programming in
      \TeX},
    \colorhref{http://pgfplots.sourceforge.net/TeX-programming-notes.pdf}
    {http://pgfplots.sourceforge.net/TeX-programming-notes.pdf}.

  \item Mark Trettin, \textit{An~essential guide to
      \LaTeX{}~2$_{ \varepsilon }$ usage},
    \colorhref{ftp://sunsite.icm.edu.pl/pub/CTAN/info/l2tabu/english/l2tabuen.pdf}{ftp://sunsite.icm.edu.pl/pub/CTAN/info/l2tabu/english/l2tabuen.pdf}.

  \end{itemize}

\end{frame}
% ##################





% ######################################
\EndingSlide{Teraz cała przygoda z~\LaTeX em przed wami :)!\\
  \vspace{1em} Dziękuję.}
% ######################################





% ############################

% Koniec dokumentu
\end{document}
