% ------------------------------
% Textpos (TEXT POSition) -- pakiet do dobrego zarządzania ułożeniem
% obiektów na „kartce” tekstu
% ------------------------------
% Potrzebne do slajdu końcowego
% \usepackage[showboxes,overlay,absolute]{textpos}
\usepackage[overlay,absolute]{textpos}
\setlength{\TPHorizModule}{10mm}
\setlength{\TPVertModule}{\TPHorizModule}
% \textblockorigin{10mm}{10mm}





% ------------------------------
% Paczki potrzebne w danym projekcie
% ------------------------------
\newcommand{\bmmax}{0}  % Górne ograniczenie na ilość "alfabetów
% matematycznych", które może wykorzystać paczka "bm" do tworzenia
% "Bold symbols". Zapobiega przekroczenia standardowej maksymalnej
% ilości jakie dopuszcza TeX (i pdfLaTeX), wynoszącej 16.
\newcommand{\hmmax}{0}
\usepackage{bm}  % Bold Math symbols, obecnie preferowany sposób.




% ------------------------------
% Pakiety do tekstów z nauk przyrodniczych
% ------------------------------
% \let\lll\undefined  % Amsmath gryzie się z pakietami do języka
% % polskiego, bo oba definiują komendę \lll. Aby rozwiązać ten problem
% % oddefiniowuję tę komendę, ale może tym samym pozbywam się dużego Ł.
\usepackage{amsmath}  % Podstawowe wsparcie od American
% Mathematical Society (w skrócie AMS)
% \usepackage{amsfonts, amssymb, amscd, amsthm}  % Dalsze wsparcie od AMS
\usepackage{amssymb, amscd, amsthm}  % Dalsze wsparcie od AMS
\usepackage{siunitx}  % Do prostszego pisania jednostek fizycznych
\usepackage{upgreek}  % Ładniejsze greckie litery
% Przykładowa składnia: pi = \uppi
\usepackage{icomma} % Przecinek w trybie matematyczny ma właściwy odstęp



% ------------------------------
% TikZ
% ------------------------------
% Włączenie konkretnych bibliotek pakietu TikZ
\usetikzlibrary{arrows.meta}
\usetikzlibrary{angles}
\usetikzlibrary{3d}
\usetikzlibrary{positioning}
\usetikzlibrary{intersections}
\usetikzlibrary{decorations.markings}
\usetikzlibrary{calc}






% ------------------------------
% Komendy i ustawienia do wykładów „Geometria 3D dla twórców gier wideo”
% ------------------------------
\newcommand{\colorhref}[2]{\href{#1}{\color{jDarkOrange} #2}}
% Potrzebna w komendach poniżej

\newcommand{\titlebackground}{./pictures/SzablonTytBluedark.png}
% Dla trybu Dark
% \newcommand{\sectionbackground}{./pictures/SzablonTytBluedark.png}
% Dla trybu Light
\newcommand{\sectionbackground}{./pictures/SzablonTytLight.png}



\newcommand{\jagielloniantitlepage}{
  \begingroup

  \setbeamercolor{background canvas}{bg=JagiellonianBlue}

  \maketitle

  \endgroup
}

\newcommand{\jagielloniantitlepagewithpicture}{
  \begingroup
  \usebackgroundtemplate{ \hspace*{-11.5em}
    \includegraphics[height=\paperheight]{\titlebackground}}

  \maketitle

  \endgroup
}


\newcommand{\jagielloniansectionwithpicture}[1]{%
  \begingroup
  \usebackgroundtemplate{ \hspace*{-11.5em}
    \includegraphics[height=\paperheight]{\sectionbackground}}

  \setbeamercolor{titlelike}{fg=normal text.fg}

  \section{#1}

  \endgroup }



\newcommand{\jagiellonianendslide}[1]{%
  \begin{frame}[standout]


    \begin{textblock}{11}(1,0.7)

      \begin{flushleft}
        \mdseries

        \footnotesize

        \color{jStrongWhite}

        Treść niniejszego wykładu jest udostępniona na~licencji
        Creative Commons (\textsc{cc}), z~uzna- \\
        niem autorstwa (\textsc{by}) oraz udostępnianiem na tych samych
        warunkach (\textsc{sa}). Rysunki i~wykresy zawarte w~wykładzie są
        autorstwa dr.~hab.~Pawła Węgrzyna et~al. i~są dostępne na tej
        samej licencji, o~ile nie wskazano inaczej. W~prezentacji
        wykorzystano temat Beamera Jagiellonian, oparty na~temacie
        Metropolis Matthiasa Vogelgesanga, dostępnym na licencji
        \LaTeX{ } Project Public License~1.3c pod adresem:
        \colorhref{https://github.com/matze/mtheme}
        {https://github.com/matze/mtheme}.

        Projekt typograficzny: Iwona Grabska-Gradzińska \\
        Skład: Kamil Ziemian \\
        Projekt graficzny i~wykonanie grafik: Dariusz Frymus \\
        Rysunki i~wykresy ilustrujące treści merytoryczne,
        Ti\emph{k}Z: Kamil Ziemian, Paweł Węgrzyn \\
        Rysunki i~wykresy ilustrujące treści merytoryczne,
        Mathematica: Paweł Węgrzyn

        Materiały powstały w ramach projektu ZintegrUJ -- Kompleksowego
        Programu Rozwoju Uniwersytetu Jagiellońskiego. Projekt
        współfinansowany ze~środków Unii Europejskiej w~ramach
        Europejskiego Funduszu Społecznego.

      \end{flushleft}

    \end{textblock}




    \begin{textblock}{3.2}(1,7.83)

      \includegraphics[scale=0.3]
      {./pictures/logotypFundusze_biale_bez_tla2.pdf}

    \end{textblock}



    \begin{textblock}{4.2}(7.4,7.9)

      \includegraphics[scale=0.3]{./pictures/logotypUE_biale_bez_tla2.pdf}

    \end{textblock}

    % \begin{textblock}{4}(7.3,7.8)
    %   \includegraphics{./pictures/CC-by-sa.png}
    % \end{textblock}

  \end{frame}





  \begin{frame}[standout]


    \begingroup

    \color{jStrongWhite}

    #1

    \endgroup

  \end{frame}
}
% ##################





% ############################ Komendy Beamera i LaTeX

% Zmiana koloru fontu matematycznego
\newcommand{\beamermathcolor}[2]{{\color{#1}\setbeamercolor{math text}
    {fg=#1} #2}}

% Zmiana koloru fontu na foreground BEAMER-koloru „normal text”
\newcommand{\normaltextcolor}[1]{{\color{normal text.fg}{#1}}}

% Zmiana koloru fontu matematycznego na jMathTextForegorundWhite
\newcommand{\beamermathcolorwhite}[1]{{\color{jStrongWhite}
    \setbeamercolor{math text}{fg=jStrongWhite} #1}}

\newcommand{\alertmath}[1]{{\color{jDarkOrange} \setbeamercolor{math
      text}{fg=jDarkOrange} #1}}








% ------------------------------
% Pakiety napisane do wykładów z  „Geometrii 3D dla twórców gier
% komputerowych”
% ------------------------------
% \usepackage{./Geometria3DPaczki/Geometria3D}
% \usepackage{./Geometria3DPaczki/Geometria3DKolory}
% \usepackage{./Geometria3DPaczki/Geometria3DIndeksy}
