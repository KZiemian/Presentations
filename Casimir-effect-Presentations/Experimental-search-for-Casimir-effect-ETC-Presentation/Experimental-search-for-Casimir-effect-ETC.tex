% ------------------------------------------------------------------------------------------------------------------
% Basic configuration of Beamera class and Jagiellonian theme
% ------------------------------------------------------------------------------------------------------------------
\RequirePackage[l2tabu, orthodox]{nag}



\ifx\PresentationStyle\notset
  \def\PresentationStyle{light}
\fi



% Options: t -- align frame text to the top
\documentclass[10pt,t]{beamer}
\mode<presentation>
\usetheme[style=\PresentationStyle]{jagiellonian}





% ------------------------------------------------------------------------------------
% Procesing configuration files of Jagiellonian theme located in
% the directory "preambule"
% ------------------------------------------------------------------------------------
% Configuration for polish language
% Need description
\usepackage[english]{babel}





% % ------------------------------
% % Better support of polish chars in technical parts of PDF
% % ------------------------------
% \hypersetup{pdfencoding=auto,psdextra}

% Package "textpos" give as enviroment "textblock" which is very usefull in
% arranging text on slides.

% This is standard configuration of "textpos"
\usepackage[overlay,absolute]{textpos}

% If you need to see bounds of "textblock's" comment line above and uncomment
% one below.

% Caution! When showboxes option is on significant ammunt of space is add
% to the top of textblock and as such, everyting put in them gone down.
% We need to check how to remove this bug.

% \usepackage[showboxes,overlay,absolute]{textpos}



% Setting scale length for package "textpos"
\setlength{\TPHorizModule}{10mm}
\setlength{\TPVertModule}{\TPHorizModule}


% ---------------------------------------
% TikZ
% ---------------------------------------
% Importing TikZ libraries
\usetikzlibrary{arrows.meta}
\usetikzlibrary{positioning}





% % Configuration package "bm" that need for making bold symbols
% \newcommand{\bmmax}{0}
% \newcommand{\hmmax}{0}
% \usepackage{bm}




% ---------------------------------------
% Packages for scientific texts
% ---------------------------------------
% \let\lll\undefined  % Sometimes you must use this line to allow
% "amsmath" package to works with packages with packages for polish
% languge imported
% /preambul/LanguageSettings/JagiellonianPolishLanguageSettings.tex.
% This comments (probably) removes polish letter Ł.
\usepackage{amsmath}  % Packages from American Mathematical Society (AMS)
\usepackage{amssymb}
\usepackage{amscd}
\usepackage{amsthm}
\usepackage{siunitx}  % Package for typsetting SI units.
\usepackage{upgreek}  % Better looking greek letters.
% Example of using upgreek: pi = \uppi


\usepackage{calrsfs}  % Zmienia czcionkę kaligraficzną w \mathcal
% na ładniejszą. Może w innych miejscach robi to samo, ale o tym nic
% nie wiem.










% ---------------------------------------
% Packages written for lectures "Geometria 3D dla twórców gier wideo"
% ---------------------------------------
% \usepackage{./ProgramowanieSymulacjiFizykiPaczki/ProgramowanieSymulacjiFizyki}
% \usepackage{./ProgramowanieSymulacjiFizykiPaczki/ProgramowanieSymulacjiFizykiIndeksy}
% \usepackage{./ProgramowanieSymulacjiFizykiPaczki/ProgramowanieSymulacjiFizykiTikZStyle}





% !!!!!!!!!!!!!!!!!!!!!!!!!!!!!!
% !!!!!!!!!!!!!!!!!!!!!!!!!!!!!!
% EVIL STUFF
\if\JUlogotitle1
\edef\LogoJUPath{LogoJU_\JUlogoLang/LogoJU_\JUlogoShape_\JUlogoColor.pdf}
\titlegraphic{\hfill\includegraphics[scale=0.22]
{./JagiellonianPictures/\LogoJUPath}}
\fi
% ---------------------------------------
% Commands for handling colors
% ---------------------------------------


% Command for setting normal text color for some text in math modestyle
% Text color depend on used style of Jagiellonian

% Beamer version of command
\newcommand{\TextWithNormalTextColor}[1]{%
  {\color{jNormalTextFGColor}
    \setbeamercolor{math text}{fg=jNormalTextFGColor} {#1}}
}

% Article and similar classes version of command
% \newcommand{\TextWithNormalTextColor}[1]{%
%   {\color{jNormalTextsFGColor} {#1}}
% }



% Beamer version of command
\newcommand{\NormalTextInMathMode}[1]{%
  {\color{jNormalTextFGColor}
    \setbeamercolor{math text}{fg=jNormalTextFGColor} \text{#1}}
}


% Article and similar classes version of command
% \newcommand{\NormalTextInMathMode}[1]{%
%   {\color{jNormalTextsFGColor} \text{#1}}
% }




% Command that sets color of some mathematical text to the same color
% that has normal text in header (?)

% Beamer version of the command
\newcommand{\MathTextFrametitleFGColor}[1]{%
  {\color{jFrametitleFGColor}
    \setbeamercolor{math text}{fg=jFrametitleFGColor} #1}
}

% Article and similar classes version of the command
% \newcommand{\MathTextWhiteColor}[1]{{\color{jFrametitleFGColor} #1}}





% Command for setting color of alert text for some text in math modestyle

% Beamer version of the command
\newcommand{\MathTextAlertColor}[1]{%
  {\color{jOrange} \setbeamercolor{math text}{fg=jOrange} #1}
}

% Article and similar classes version of the command
% \newcommand{\MathTextAlertColor}[1]{{\color{jOrange} #1}}





% Command that allow you to sets chosen color as the color of some text into
% math mode. Due to some nuances in the way that Beamer handle colors
% it not work in all cases. We hope that in the future we will improve it.

% Beamer version of the command
\newcommand{\SetMathTextColor}[2]{%
  {\color{#1} \setbeamercolor{math text}{fg=#1} #2}
}


% Article and similar classes version of the command
% \newcommand{\SetMathTextColor}[2]{{\color{#1} #2}}










% ---------------------------------------
% Commands for few special slides
% ---------------------------------------
\newcommand{\EndingSlide}[1]{%
  \begin{frame}[standout]

    \begingroup

    \color{jFrametitleFGColor}

    #1

    \endgroup

  \end{frame}
}










% ---------------------------------------
% Commands for setting background pictures for some slides
% ---------------------------------------
\newcommand{\TitleBackgroundPicture}
{./JagiellonianPictures/Backgrounds/LajkonikDark.png}
\newcommand{\SectionBackgroundPicture}
{./JagiellonianPictures/Backgrounds/LajkonikLight.png}



\newcommand{\TitleSlideWithPicture}{%
  \begingroup

  \usebackgroundtemplate{%
    \includegraphics[height=\paperheight]{\TitleBackgroundPicture}}

  \maketitle

  \endgroup
}





\newcommand{\SectionSlideWithPicture}[1]{%
  \begingroup

  \usebackgroundtemplate{%
    \includegraphics[height=\paperheight]{\SectionBackgroundPicture}}

  \setbeamercolor{titlelike}{fg=normal text.fg}

  \section{#1}

  \endgroup
}










% ---------------------------------------
% Commands for lectures "Geometria 3D dla twórców gier wideo"
% Polish version
% ---------------------------------------
% Komendy teraz wykomentowane były potrzebne, gdy loga były na niebieskim
% tle, nie na białym. A są na białym bo tego chcieli w biurze projektu.
% \newcommand{\FundingLogoWhitePicturePL}
% {./PresentationPictures/CommonPictures/logotypFundusze_biale_bez_tla2.pdf}
\newcommand{\FundingLogoColorPicturePL}
{./PresentationPictures/CommonPictures/European_Funds_color_PL.pdf}
% \newcommand{\EULogoWhitePicturePL}
% {./PresentationPictures/CommonPictures/logotypUE_biale_bez_tla2.pdf}
\newcommand{\EUSocialFundLogoColorPicturePL}
{./PresentationPictures/CommonPictures/EU_Social_Fund_color_PL.pdf}
% \newcommand{\ZintegrUJLogoWhitePicturePL}
% {./PresentationPictures/CommonPictures/zintegruj-logo-white.pdf}
\newcommand{\ZintegrUJLogoColorPicturePL}
{./PresentationPictures/CommonPictures/ZintegrUJ_color.pdf}
\newcommand{\JULogoColorPicturePL}
{./JagiellonianPictures/LogoJU_PL/LogoJU_A_color.pdf}





\newcommand{\GeometryThreeDSpecialBeginningSlidePL}{%
  \begin{frame}[standout]

    \begin{textblock}{11}(1,0.7)

      \begin{flushleft}

        \mdseries

        \footnotesize

        \color{jFrametitleFGColor}

        Materiał powstał w ramach projektu współfinansowanego ze środków
        Unii Europejskiej w ramach Europejskiego Funduszu Społecznego
        POWR.03.05.00-00-Z309/17-00.

      \end{flushleft}

    \end{textblock}





    \begin{textblock}{10}(0,2.2)

      \tikz \fill[color=jBackgroundStyleLight] (0,0) rectangle (12.8,-1.5);

    \end{textblock}


    \begin{textblock}{3.2}(1,2.45)

      \includegraphics[scale=0.3]{\FundingLogoColorPicturePL}

    \end{textblock}


    \begin{textblock}{2.5}(3.7,2.5)

      \includegraphics[scale=0.2]{\JULogoColorPicturePL}

    \end{textblock}


    \begin{textblock}{2.5}(6,2.4)

      \includegraphics[scale=0.1]{\ZintegrUJLogoColorPicturePL}

    \end{textblock}


    \begin{textblock}{4.2}(8.4,2.6)

      \includegraphics[scale=0.3]{\EUSocialFundLogoColorPicturePL}

    \end{textblock}

  \end{frame}
}



\newcommand{\GeometryThreeDTwoSpecialBeginningSlidesPL}{%
  \begin{frame}[standout]

    \begin{textblock}{11}(1,0.7)

      \begin{flushleft}

        \mdseries

        \footnotesize

        \color{jFrametitleFGColor}

        Materiał powstał w ramach projektu współfinansowanego ze środków
        Unii Europejskiej w ramach Europejskiego Funduszu Społecznego
        POWR.03.05.00-00-Z309/17-00.

      \end{flushleft}

    \end{textblock}





    \begin{textblock}{10}(0,2.2)

      \tikz \fill[color=jBackgroundStyleLight] (0,0) rectangle (12.8,-1.5);

    \end{textblock}


    \begin{textblock}{3.2}(1,2.45)

      \includegraphics[scale=0.3]{\FundingLogoColorPicturePL}

    \end{textblock}


    \begin{textblock}{2.5}(3.7,2.5)

      \includegraphics[scale=0.2]{\JULogoColorPicturePL}

    \end{textblock}


    \begin{textblock}{2.5}(6,2.4)

      \includegraphics[scale=0.1]{\ZintegrUJLogoColorPicturePL}

    \end{textblock}


    \begin{textblock}{4.2}(8.4,2.6)

      \includegraphics[scale=0.3]{\EUSocialFundLogoColorPicturePL}

    \end{textblock}

  \end{frame}





  \TitleSlideWithPicture
}



\newcommand{\GeometryThreeDSpecialEndingSlidePL}{%
  \begin{frame}[standout]

    \begin{textblock}{11}(1,0.7)

      \begin{flushleft}

        \mdseries

        \footnotesize

        \color{jFrametitleFGColor}

        Materiał powstał w ramach projektu współfinansowanego ze środków
        Unii Europejskiej w~ramach Europejskiego Funduszu Społecznego
        POWR.03.05.00-00-Z309/17-00.

      \end{flushleft}

    \end{textblock}





    \begin{textblock}{10}(0,2.2)

      \tikz \fill[color=jBackgroundStyleLight] (0,0) rectangle (12.8,-1.5);

    \end{textblock}


    \begin{textblock}{3.2}(1,2.45)

      \includegraphics[scale=0.3]{\FundingLogoColorPicturePL}

    \end{textblock}


    \begin{textblock}{2.5}(3.7,2.5)

      \includegraphics[scale=0.2]{\JULogoColorPicturePL}

    \end{textblock}


    \begin{textblock}{2.5}(6,2.4)

      \includegraphics[scale=0.1]{\ZintegrUJLogoColorPicturePL}

    \end{textblock}


    \begin{textblock}{4.2}(8.4,2.6)

      \includegraphics[scale=0.3]{\EUSocialFundLogoColorPicturePL}

    \end{textblock}





    \begin{textblock}{11}(1,4)

      \begin{flushleft}

        \mdseries

        \footnotesize

        \RaggedRight

        \color{jFrametitleFGColor}

        Treść niniejszego wykładu jest udostępniona na~licencji
        Creative Commons (\textsc{cc}), z~uzna\-niem autorstwa
        (\textsc{by}) oraz udostępnianiem na tych samych warunkach
        (\textsc{sa}). Rysunki i~wy\-kresy zawarte w~wykładzie są
        autorstwa dr.~hab.~Pawła Węgrzyna et~al. i~są dostępne
        na tej samej licencji, o~ile nie wskazano inaczej.
        W~prezentacji wykorzystano temat Beamera Jagiellonian,
        oparty na~temacie Metropolis Matthiasa Vogelgesanga,
        dostępnym na licencji \LaTeX{} Project Public License~1.3c
        pod adresem: \colorhref{https://github.com/matze/mtheme}
        {https://github.com/matze/mtheme}.

        Projekt typograficzny: Iwona Grabska-Gradzińska \\
        Skład: Kamil Ziemian;
        Korekta: Wojciech Palacz \\
        Modele: Dariusz Frymus, Kamil Nowakowski \\
        Rysunki i~wykresy: Kamil Ziemian, Paweł Węgrzyn, Wojciech Palacz

      \end{flushleft}

    \end{textblock}

  \end{frame}
}



\newcommand{\GeometryThreeDTwoSpecialEndingSlidesPL}[1]{%
  \begin{frame}[standout]


    \begin{textblock}{11}(1,0.7)

      \begin{flushleft}

        \mdseries

        \footnotesize

        \color{jFrametitleFGColor}

        Materiał powstał w ramach projektu współfinansowanego ze środków
        Unii Europejskiej w~ramach Europejskiego Funduszu Społecznego
        POWR.03.05.00-00-Z309/17-00.

      \end{flushleft}

    \end{textblock}





    \begin{textblock}{10}(0,2.2)

      \tikz \fill[color=jBackgroundStyleLight] (0,0) rectangle (12.8,-1.5);

    \end{textblock}


    \begin{textblock}{3.2}(1,2.45)

      \includegraphics[scale=0.3]{\FundingLogoColorPicturePL}

    \end{textblock}


    \begin{textblock}{2.5}(3.7,2.5)

      \includegraphics[scale=0.2]{\JULogoColorPicturePL}

    \end{textblock}


    \begin{textblock}{2.5}(6,2.4)

      \includegraphics[scale=0.1]{\ZintegrUJLogoColorPicturePL}

    \end{textblock}


    \begin{textblock}{4.2}(8.4,2.6)

      \includegraphics[scale=0.3]{\EUSocialFundLogoColorPicturePL}

    \end{textblock}





    \begin{textblock}{11}(1,4)

      \begin{flushleft}

        \mdseries

        \footnotesize

        \RaggedRight

        \color{jFrametitleFGColor}

        Treść niniejszego wykładu jest udostępniona na~licencji
        Creative Commons (\textsc{cc}), z~uzna\-niem autorstwa
        (\textsc{by}) oraz udostępnianiem na tych samych warunkach
        (\textsc{sa}). Rysunki i~wy\-kresy zawarte w~wykładzie są
        autorstwa dr.~hab.~Pawła Węgrzyna et~al. i~są dostępne
        na tej samej licencji, o~ile nie wskazano inaczej.
        W~prezentacji wykorzystano temat Beamera Jagiellonian,
        oparty na~temacie Metropolis Matthiasa Vogelgesanga,
        dostępnym na licencji \LaTeX{} Project Public License~1.3c
        pod adresem: \colorhref{https://github.com/matze/mtheme}
        {https://github.com/matze/mtheme}.

        Projekt typograficzny: Iwona Grabska-Gradzińska \\
        Skład: Kamil Ziemian;
        Korekta: Wojciech Palacz \\
        Modele: Dariusz Frymus, Kamil Nowakowski \\
        Rysunki i~wykresy: Kamil Ziemian, Paweł Węgrzyn, Wojciech Palacz

      \end{flushleft}

    \end{textblock}

  \end{frame}





  \begin{frame}[standout]

    \begingroup

    \color{jFrametitleFGColor}

    #1

    \endgroup

  \end{frame}
}



\newcommand{\GeometryThreeDSpecialEndingSlideVideoPL}{%
  \begin{frame}[standout]

    \begin{textblock}{11}(1,0.7)

      \begin{flushleft}

        \mdseries

        \footnotesize

        \color{jFrametitleFGColor}

        Materiał powstał w ramach projektu współfinansowanego ze środków
        Unii Europejskiej w~ramach Europejskiego Funduszu Społecznego
        POWR.03.05.00-00-Z309/17-00.

      \end{flushleft}

    \end{textblock}





    \begin{textblock}{10}(0,2.2)

      \tikz \fill[color=jBackgroundStyleLight] (0,0) rectangle (12.8,-1.5);

    \end{textblock}


    \begin{textblock}{3.2}(1,2.45)

      \includegraphics[scale=0.3]{\FundingLogoColorPicturePL}

    \end{textblock}


    \begin{textblock}{2.5}(3.7,2.5)

      \includegraphics[scale=0.2]{\JULogoColorPicturePL}

    \end{textblock}


    \begin{textblock}{2.5}(6,2.4)

      \includegraphics[scale=0.1]{\ZintegrUJLogoColorPicturePL}

    \end{textblock}


    \begin{textblock}{4.2}(8.4,2.6)

      \includegraphics[scale=0.3]{\EUSocialFundLogoColorPicturePL}

    \end{textblock}





    \begin{textblock}{11}(1,4)

      \begin{flushleft}

        \mdseries

        \footnotesize

        \RaggedRight

        \color{jFrametitleFGColor}

        Treść niniejszego wykładu jest udostępniona na~licencji
        Creative Commons (\textsc{cc}), z~uzna\-niem autorstwa
        (\textsc{by}) oraz udostępnianiem na tych samych warunkach
        (\textsc{sa}). Rysunki i~wy\-kresy zawarte w~wykładzie są
        autorstwa dr.~hab.~Pawła Węgrzyna et~al. i~są dostępne
        na tej samej licencji, o~ile nie wskazano inaczej.
        W~prezentacji wykorzystano temat Beamera Jagiellonian,
        oparty na~temacie Metropolis Matthiasa Vogelgesanga,
        dostępnym na licencji \LaTeX{} Project Public License~1.3c
        pod adresem: \colorhref{https://github.com/matze/mtheme}
        {https://github.com/matze/mtheme}.

        Projekt typograficzny: Iwona Grabska-Gradzińska;
        Skład: Kamil Ziemian \\
        Korekta: Wojciech Palacz;
        Modele: Dariusz Frymus, Kamil Nowakowski \\
        Rysunki i~wykresy: Kamil Ziemian, Paweł Węgrzyn, Wojciech Palacz \\
        Montaż: Agencja Filmowa Film \& Television Production~-- Zbigniew
        Masklak

      \end{flushleft}

    \end{textblock}

  \end{frame}
}





\newcommand{\GeometryThreeDTwoSpecialEndingSlidesVideoPL}[1]{%
  \begin{frame}[standout]

    \begin{textblock}{11}(1,0.7)

      \begin{flushleft}

        \mdseries

        \footnotesize

        \color{jFrametitleFGColor}

        Materiał powstał w ramach projektu współfinansowanego ze środków
        Unii Europejskiej w~ramach Europejskiego Funduszu Społecznego
        POWR.03.05.00-00-Z309/17-00.

      \end{flushleft}

    \end{textblock}





    \begin{textblock}{10}(0,2.2)

      \tikz \fill[color=jBackgroundStyleLight] (0,0) rectangle (12.8,-1.5);

    \end{textblock}


    \begin{textblock}{3.2}(1,2.45)

      \includegraphics[scale=0.3]{\FundingLogoColorPicturePL}

    \end{textblock}


    \begin{textblock}{2.5}(3.7,2.5)

      \includegraphics[scale=0.2]{\JULogoColorPicturePL}

    \end{textblock}


    \begin{textblock}{2.5}(6,2.4)

      \includegraphics[scale=0.1]{\ZintegrUJLogoColorPicturePL}

    \end{textblock}


    \begin{textblock}{4.2}(8.4,2.6)

      \includegraphics[scale=0.3]{\EUSocialFundLogoColorPicturePL}

    \end{textblock}





    \begin{textblock}{11}(1,4)

      \begin{flushleft}

        \mdseries

        \footnotesize

        \RaggedRight

        \color{jFrametitleFGColor}

        Treść niniejszego wykładu jest udostępniona na~licencji
        Creative Commons (\textsc{cc}), z~uzna\-niem autorstwa
        (\textsc{by}) oraz udostępnianiem na tych samych warunkach
        (\textsc{sa}). Rysunki i~wy\-kresy zawarte w~wykładzie są
        autorstwa dr.~hab.~Pawła Węgrzyna et~al. i~są dostępne
        na tej samej licencji, o~ile nie wskazano inaczej.
        W~prezentacji wykorzystano temat Beamera Jagiellonian,
        oparty na~temacie Metropolis Matthiasa Vogelgesanga,
        dostępnym na licencji \LaTeX{} Project Public License~1.3c
        pod adresem: \colorhref{https://github.com/matze/mtheme}
        {https://github.com/matze/mtheme}.

        Projekt typograficzny: Iwona Grabska-Gradzińska;
        Skład: Kamil Ziemian \\
        Korekta: Wojciech Palacz;
        Modele: Dariusz Frymus, Kamil Nowakowski \\
        Rysunki i~wykresy: Kamil Ziemian, Paweł Węgrzyn, Wojciech Palacz \\
        Montaż: Agencja Filmowa Film \& Television Production~-- Zbigniew
        Masklak

      \end{flushleft}

    \end{textblock}

  \end{frame}





  \begin{frame}[standout]


    \begingroup

    \color{jFrametitleFGColor}

    #1

    \endgroup

  \end{frame}
}










% ---------------------------------------
% Commands for lectures "Geometria 3D dla twórców gier wideo"
% English version
% ---------------------------------------
% \newcommand{\FundingLogoWhitePictureEN}
% {./PresentationPictures/CommonPictures/logotypFundusze_biale_bez_tla2.pdf}
\newcommand{\FundingLogoColorPictureEN}
{./PresentationPictures/CommonPictures/European_Funds_color_EN.pdf}
% \newcommand{\EULogoWhitePictureEN}
% {./PresentationPictures/CommonPictures/logotypUE_biale_bez_tla2.pdf}
\newcommand{\EUSocialFundLogoColorPictureEN}
{./PresentationPictures/CommonPictures/EU_Social_Fund_color_EN.pdf}
% \newcommand{\ZintegrUJLogoWhitePictureEN}
% {./PresentationPictures/CommonPictures/zintegruj-logo-white.pdf}
\newcommand{\ZintegrUJLogoColorPictureEN}
{./PresentationPictures/CommonPictures/ZintegrUJ_color.pdf}
\newcommand{\JULogoColorPictureEN}
{./JagiellonianPictures/LogoJU_EN/LogoJU_A_color.pdf}



\newcommand{\GeometryThreeDSpecialBeginningSlideEN}{%
  \begin{frame}[standout]

    \begin{textblock}{11}(1,0.7)

      \begin{flushleft}

        \mdseries

        \footnotesize

        \color{jFrametitleFGColor}

        This content was created as part of a project co-financed by the
        European Union within the framework of the European Social Fund
        POWR.03.05.00-00-Z309/17-00.

      \end{flushleft}

    \end{textblock}





    \begin{textblock}{10}(0,2.2)

      \tikz \fill[color=jBackgroundStyleLight] (0,0) rectangle (12.8,-1.5);

    \end{textblock}


    \begin{textblock}{3.2}(0.7,2.45)

      \includegraphics[scale=0.3]{\FundingLogoColorPictureEN}

    \end{textblock}


    \begin{textblock}{2.5}(4.15,2.5)

      \includegraphics[scale=0.2]{\JULogoColorPictureEN}

    \end{textblock}


    \begin{textblock}{2.5}(6.35,2.4)

      \includegraphics[scale=0.1]{\ZintegrUJLogoColorPictureEN}

    \end{textblock}


    \begin{textblock}{4.2}(8.4,2.6)

      \includegraphics[scale=0.3]{\EUSocialFundLogoColorPictureEN}

    \end{textblock}

  \end{frame}
}



\newcommand{\GeometryThreeDTwoSpecialBeginningSlidesEN}{%
  \begin{frame}[standout]

    \begin{textblock}{11}(1,0.7)

      \begin{flushleft}

        \mdseries

        \footnotesize

        \color{jFrametitleFGColor}

        This content was created as part of a project co-financed by the
        European Union within the framework of the European Social Fund
        POWR.03.05.00-00-Z309/17-00.

      \end{flushleft}

    \end{textblock}





    \begin{textblock}{10}(0,2.2)

      \tikz \fill[color=jBackgroundStyleLight] (0,0) rectangle (12.8,-1.5);

    \end{textblock}


    \begin{textblock}{3.2}(0.7,2.45)

      \includegraphics[scale=0.3]{\FundingLogoColorPictureEN}

    \end{textblock}


    \begin{textblock}{2.5}(4.15,2.5)

      \includegraphics[scale=0.2]{\JULogoColorPictureEN}

    \end{textblock}


    \begin{textblock}{2.5}(6.35,2.4)

      \includegraphics[scale=0.1]{\ZintegrUJLogoColorPictureEN}

    \end{textblock}


    \begin{textblock}{4.2}(8.4,2.6)

      \includegraphics[scale=0.3]{\EUSocialFundLogoColorPictureEN}

    \end{textblock}

  \end{frame}





  \TitleSlideWithPicture
}



\newcommand{\GeometryThreeDSpecialEndingSlideEN}{%
  \begin{frame}[standout]

    \begin{textblock}{11}(1,0.7)

      \begin{flushleft}

        \mdseries

        \footnotesize

        \color{jFrametitleFGColor}

        This content was created as part of a project co-financed by the
        European Union within the framework of the European Social Fund
        POWR.03.05.00-00-Z309/17-00.

      \end{flushleft}

    \end{textblock}





    \begin{textblock}{10}(0,2.2)

      \tikz \fill[color=jBackgroundStyleLight] (0,0) rectangle (12.8,-1.5);

    \end{textblock}


    \begin{textblock}{3.2}(0.7,2.45)

      \includegraphics[scale=0.3]{\FundingLogoColorPictureEN}

    \end{textblock}


    \begin{textblock}{2.5}(4.15,2.5)

      \includegraphics[scale=0.2]{\JULogoColorPictureEN}

    \end{textblock}


    \begin{textblock}{2.5}(6.35,2.4)

      \includegraphics[scale=0.1]{\ZintegrUJLogoColorPictureEN}

    \end{textblock}


    \begin{textblock}{4.2}(8.4,2.6)

      \includegraphics[scale=0.3]{\EUSocialFundLogoColorPictureEN}

    \end{textblock}





    \begin{textblock}{11}(1,4)

      \begin{flushleft}

        \mdseries

        \footnotesize

        \RaggedRight

        \color{jFrametitleFGColor}

        The content of this lecture is made available under a~Creative
        Commons licence (\textsc{cc}), giving the author the credits
        (\textsc{by}) and putting an obligation to share on the same terms
        (\textsc{sa}). Figures and diagrams included in the lecture are
        authored by Paweł Węgrzyn et~al., and are available under the same
        license unless indicated otherwise.\\ The presentation uses the
        Beamer Jagiellonian theme based on Matthias Vogelgesang’s
        Metropolis theme, available under license \LaTeX{} Project
        Public License~1.3c at: \colorhref{https://github.com/matze/mtheme}
        {https://github.com/matze/mtheme}.

        Typographic design: Iwona Grabska-Gradzińska \\
        \LaTeX{} Typesetting: Kamil Ziemian \\
        Proofreading: Wojciech Palacz,
        Monika Stawicka \\
        3D Models: Dariusz Frymus, Kamil Nowakowski \\
        Figures and charts: Kamil Ziemian, Paweł Węgrzyn, Wojciech Palacz

      \end{flushleft}

    \end{textblock}

  \end{frame}
}



\newcommand{\GeometryThreeDTwoSpecialEndingSlidesEN}[1]{%
  \begin{frame}[standout]


    \begin{textblock}{11}(1,0.7)

      \begin{flushleft}

        \mdseries

        \footnotesize

        \color{jFrametitleFGColor}

        This content was created as part of a project co-financed by the
        European Union within the framework of the European Social Fund
        POWR.03.05.00-00-Z309/17-00.

      \end{flushleft}

    \end{textblock}





    \begin{textblock}{10}(0,2.2)

      \tikz \fill[color=jBackgroundStyleLight] (0,0) rectangle (12.8,-1.5);

    \end{textblock}


    \begin{textblock}{3.2}(0.7,2.45)

      \includegraphics[scale=0.3]{\FundingLogoColorPictureEN}

    \end{textblock}


    \begin{textblock}{2.5}(4.15,2.5)

      \includegraphics[scale=0.2]{\JULogoColorPictureEN}

    \end{textblock}


    \begin{textblock}{2.5}(6.35,2.4)

      \includegraphics[scale=0.1]{\ZintegrUJLogoColorPictureEN}

    \end{textblock}


    \begin{textblock}{4.2}(8.4,2.6)

      \includegraphics[scale=0.3]{\EUSocialFundLogoColorPictureEN}

    \end{textblock}





    \begin{textblock}{11}(1,4)

      \begin{flushleft}

        \mdseries

        \footnotesize

        \RaggedRight

        \color{jFrametitleFGColor}

        The content of this lecture is made available under a~Creative
        Commons licence (\textsc{cc}), giving the author the credits
        (\textsc{by}) and putting an obligation to share on the same terms
        (\textsc{sa}). Figures and diagrams included in the lecture are
        authored by Paweł Węgrzyn et~al., and are available under the same
        license unless indicated otherwise.\\ The presentation uses the
        Beamer Jagiellonian theme based on Matthias Vogelgesang’s
        Metropolis theme, available under license \LaTeX{} Project
        Public License~1.3c at: \colorhref{https://github.com/matze/mtheme}
        {https://github.com/matze/mtheme}.

        Typographic design: Iwona Grabska-Gradzińska \\
        \LaTeX{} Typesetting: Kamil Ziemian \\
        Proofreading: Wojciech Palacz,
        Monika Stawicka \\
        3D Models: Dariusz Frymus, Kamil Nowakowski \\
        Figures and charts: Kamil Ziemian, Paweł Węgrzyn, Wojciech Palacz

      \end{flushleft}

    \end{textblock}

  \end{frame}





  \begin{frame}[standout]

    \begingroup

    \color{jFrametitleFGColor}

    #1

    \endgroup

  \end{frame}
}



\newcommand{\GeometryThreeDSpecialEndingSlideVideoVerOneEN}{%
  \begin{frame}[standout]

    \begin{textblock}{11}(1,0.7)

      \begin{flushleft}

        \mdseries

        \footnotesize

        \color{jFrametitleFGColor}

        This content was created as part of a project co-financed by the
        European Union within the framework of the European Social Fund
        POWR.03.05.00-00-Z309/17-00.

      \end{flushleft}

    \end{textblock}





    \begin{textblock}{10}(0,2.2)

      \tikz \fill[color=jBackgroundStyleLight] (0,0) rectangle (12.8,-1.5);

    \end{textblock}


    \begin{textblock}{3.2}(0.7,2.45)

      \includegraphics[scale=0.3]{\FundingLogoColorPictureEN}

    \end{textblock}


    \begin{textblock}{2.5}(4.15,2.5)

      \includegraphics[scale=0.2]{\JULogoColorPictureEN}

    \end{textblock}


    \begin{textblock}{2.5}(6.35,2.4)

      \includegraphics[scale=0.1]{\ZintegrUJLogoColorPictureEN}

    \end{textblock}


    \begin{textblock}{4.2}(8.4,2.6)

      \includegraphics[scale=0.3]{\EUSocialFundLogoColorPictureEN}

    \end{textblock}





    \begin{textblock}{11}(1,4)

      \begin{flushleft}

        \mdseries

        \footnotesize

        \RaggedRight

        \color{jFrametitleFGColor}

        The content of this lecture is made available under a Creative
        Commons licence (\textsc{cc}), giving the author the credits
        (\textsc{by}) and putting an obligation to share on the same terms
        (\textsc{sa}). Figures and diagrams included in the lecture are
        authored by Paweł Węgrzyn et~al., and are available under the same
        license unless indicated otherwise.\\ The presentation uses the
        Beamer Jagiellonian theme based on Matthias Vogelgesang’s
        Metropolis theme, available under license \LaTeX{} Project
        Public License~1.3c at: \colorhref{https://github.com/matze/mtheme}
        {https://github.com/matze/mtheme}.

        Typographic design: Iwona Grabska-Gradzińska;
        \LaTeX{} Typesetting: Kamil Ziemian \\
        Proofreading: Wojciech Palacz,
        Monika Stawicka \\
        3D Models: Dariusz Frymus, Kamil Nowakowski \\
        Figures and charts: Kamil Ziemian, Paweł Węgrzyn, Wojciech
        Palacz \\
        Film editing: Agencja Filmowa Film \& Television Production~--
        Zbigniew Masklak

      \end{flushleft}

    \end{textblock}

  \end{frame}
}



\newcommand{\GeometryThreeDSpecialEndingSlideVideoVerTwoEN}{%
  \begin{frame}[standout]

    \begin{textblock}{11}(1,0.7)

      \begin{flushleft}

        \mdseries

        \footnotesize

        \color{jFrametitleFGColor}

        This content was created as part of a project co-financed by the
        European Union within the framework of the European Social Fund
        POWR.03.05.00-00-Z309/17-00.

      \end{flushleft}

    \end{textblock}





    \begin{textblock}{10}(0,2.2)

      \tikz \fill[color=jBackgroundStyleLight] (0,0) rectangle (12.8,-1.5);

    \end{textblock}


    \begin{textblock}{3.2}(0.7,2.45)

      \includegraphics[scale=0.3]{\FundingLogoColorPictureEN}

    \end{textblock}


    \begin{textblock}{2.5}(4.15,2.5)

      \includegraphics[scale=0.2]{\JULogoColorPictureEN}

    \end{textblock}


    \begin{textblock}{2.5}(6.35,2.4)

      \includegraphics[scale=0.1]{\ZintegrUJLogoColorPictureEN}

    \end{textblock}


    \begin{textblock}{4.2}(8.4,2.6)

      \includegraphics[scale=0.3]{\EUSocialFundLogoColorPictureEN}

    \end{textblock}





    \begin{textblock}{11}(1,4)

      \begin{flushleft}

        \mdseries

        \footnotesize

        \RaggedRight

        \color{jFrametitleFGColor}

        The content of this lecture is made available under a Creative
        Commons licence (\textsc{cc}), giving the author the credits
        (\textsc{by}) and putting an obligation to share on the same terms
        (\textsc{sa}). Figures and diagrams included in the lecture are
        authored by Paweł Węgrzyn et~al., and are available under the same
        license unless indicated otherwise.\\ The presentation uses the
        Beamer Jagiellonian theme based on Matthias Vogelgesang’s
        Metropolis theme, available under license \LaTeX{} Project
        Public License~1.3c at: \colorhref{https://github.com/matze/mtheme}
        {https://github.com/matze/mtheme}.

        Typographic design: Iwona Grabska-Gradzińska;
        \LaTeX{} Typesetting: Kamil Ziemian \\
        Proofreading: Wojciech Palacz,
        Monika Stawicka \\
        3D Models: Dariusz Frymus, Kamil Nowakowski \\
        Figures and charts: Kamil Ziemian, Paweł Węgrzyn, Wojciech
        Palacz \\
        Film editing: IMAVI -- Joanna Kozakiewicz, Krzysztof Magda, Nikodem
        Frodyma

      \end{flushleft}

    \end{textblock}

  \end{frame}
}



\newcommand{\GeometryThreeDSpecialEndingSlideVideoVerThreeEN}{%
  \begin{frame}[standout]

    \begin{textblock}{11}(1,0.7)

      \begin{flushleft}

        \mdseries

        \footnotesize

        \color{jFrametitleFGColor}

        This content was created as part of a project co-financed by the
        European Union within the framework of the European Social Fund
        POWR.03.05.00-00-Z309/17-00.

      \end{flushleft}

    \end{textblock}





    \begin{textblock}{10}(0,2.2)

      \tikz \fill[color=jBackgroundStyleLight] (0,0) rectangle (12.8,-1.5);

    \end{textblock}


    \begin{textblock}{3.2}(0.7,2.45)

      \includegraphics[scale=0.3]{\FundingLogoColorPictureEN}

    \end{textblock}


    \begin{textblock}{2.5}(4.15,2.5)

      \includegraphics[scale=0.2]{\JULogoColorPictureEN}

    \end{textblock}


    \begin{textblock}{2.5}(6.35,2.4)

      \includegraphics[scale=0.1]{\ZintegrUJLogoColorPictureEN}

    \end{textblock}


    \begin{textblock}{4.2}(8.4,2.6)

      \includegraphics[scale=0.3]{\EUSocialFundLogoColorPictureEN}

    \end{textblock}





    \begin{textblock}{11}(1,4)

      \begin{flushleft}

        \mdseries

        \footnotesize

        \RaggedRight

        \color{jFrametitleFGColor}

        The content of this lecture is made available under a Creative
        Commons licence (\textsc{cc}), giving the author the credits
        (\textsc{by}) and putting an obligation to share on the same terms
        (\textsc{sa}). Figures and diagrams included in the lecture are
        authored by Paweł Węgrzyn et~al., and are available under the same
        license unless indicated otherwise.\\ The presentation uses the
        Beamer Jagiellonian theme based on Matthias Vogelgesang’s
        Metropolis theme, available under license \LaTeX{} Project
        Public License~1.3c at: \colorhref{https://github.com/matze/mtheme}
        {https://github.com/matze/mtheme}.

        Typographic design: Iwona Grabska-Gradzińska;
        \LaTeX{} Typesetting: Kamil Ziemian \\
        Proofreading: Wojciech Palacz,
        Monika Stawicka \\
        3D Models: Dariusz Frymus, Kamil Nowakowski \\
        Figures and charts: Kamil Ziemian, Paweł Węgrzyn, Wojciech
        Palacz \\
        Film editing: Agencja Filmowa Film \& Television Production~--
        Zbigniew Masklak \\
        Film editing: IMAVI -- Joanna Kozakiewicz, Krzysztof Magda, Nikodem
        Frodyma

      \end{flushleft}

    \end{textblock}

  \end{frame}
}



\newcommand{\GeometryThreeDTwoSpecialEndingSlidesVideoVerOneEN}[1]{%
  \begin{frame}[standout]

    \begin{textblock}{11}(1,0.7)

      \begin{flushleft}

        \mdseries

        \footnotesize

        \color{jFrametitleFGColor}

        This content was created as part of a project co-financed by the
        European Union within the framework of the European Social Fund
        POWR.03.05.00-00-Z309/17-00.

      \end{flushleft}

    \end{textblock}





    \begin{textblock}{10}(0,2.2)

      \tikz \fill[color=jBackgroundStyleLight] (0,0) rectangle (12.8,-1.5);

    \end{textblock}


    \begin{textblock}{3.2}(0.7,2.45)

      \includegraphics[scale=0.3]{\FundingLogoColorPictureEN}

    \end{textblock}


    \begin{textblock}{2.5}(4.15,2.5)

      \includegraphics[scale=0.2]{\JULogoColorPictureEN}

    \end{textblock}


    \begin{textblock}{2.5}(6.35,2.4)

      \includegraphics[scale=0.1]{\ZintegrUJLogoColorPictureEN}

    \end{textblock}


    \begin{textblock}{4.2}(8.4,2.6)

      \includegraphics[scale=0.3]{\EUSocialFundLogoColorPictureEN}

    \end{textblock}





    \begin{textblock}{11}(1,4)

      \begin{flushleft}

        \mdseries

        \footnotesize

        \RaggedRight

        \color{jFrametitleFGColor}

        The content of this lecture is made available under a Creative
        Commons licence (\textsc{cc}), giving the author the credits
        (\textsc{by}) and putting an obligation to share on the same terms
        (\textsc{sa}). Figures and diagrams included in the lecture are
        authored by Paweł Węgrzyn et~al., and are available under the same
        license unless indicated otherwise.\\ The presentation uses the
        Beamer Jagiellonian theme based on Matthias Vogelgesang’s
        Metropolis theme, available under license \LaTeX{} Project
        Public License~1.3c at: \colorhref{https://github.com/matze/mtheme}
        {https://github.com/matze/mtheme}.

        Typographic design: Iwona Grabska-Gradzińska;
        \LaTeX{} Typesetting: Kamil Ziemian \\
        Proofreading: Wojciech Palacz,
        Monika Stawicka \\
        3D Models: Dariusz Frymus, Kamil Nowakowski \\
        Figures and charts: Kamil Ziemian, Paweł Węgrzyn,
        Wojciech Palacz \\
        Film editing: Agencja Filmowa Film \& Television Production~--
        Zbigniew Masklak

      \end{flushleft}

    \end{textblock}

  \end{frame}





  \begin{frame}[standout]


    \begingroup

    \color{jFrametitleFGColor}

    #1

    \endgroup

  \end{frame}
}



\newcommand{\GeometryThreeDTwoSpecialEndingSlidesVideoVerTwoEN}[1]{%
  \begin{frame}[standout]

    \begin{textblock}{11}(1,0.7)

      \begin{flushleft}

        \mdseries

        \footnotesize

        \color{jFrametitleFGColor}

        This content was created as part of a project co-financed by the
        European Union within the framework of the European Social Fund
        POWR.03.05.00-00-Z309/17-00.

      \end{flushleft}

    \end{textblock}





    \begin{textblock}{10}(0,2.2)

      \tikz \fill[color=jBackgroundStyleLight] (0,0) rectangle (12.8,-1.5);

    \end{textblock}


    \begin{textblock}{3.2}(0.7,2.45)

      \includegraphics[scale=0.3]{\FundingLogoColorPictureEN}

    \end{textblock}


    \begin{textblock}{2.5}(4.15,2.5)

      \includegraphics[scale=0.2]{\JULogoColorPictureEN}

    \end{textblock}


    \begin{textblock}{2.5}(6.35,2.4)

      \includegraphics[scale=0.1]{\ZintegrUJLogoColorPictureEN}

    \end{textblock}


    \begin{textblock}{4.2}(8.4,2.6)

      \includegraphics[scale=0.3]{\EUSocialFundLogoColorPictureEN}

    \end{textblock}





    \begin{textblock}{11}(1,4)

      \begin{flushleft}

        \mdseries

        \footnotesize

        \RaggedRight

        \color{jFrametitleFGColor}

        The content of this lecture is made available under a Creative
        Commons licence (\textsc{cc}), giving the author the credits
        (\textsc{by}) and putting an obligation to share on the same terms
        (\textsc{sa}). Figures and diagrams included in the lecture are
        authored by Paweł Węgrzyn et~al., and are available under the same
        license unless indicated otherwise.\\ The presentation uses the
        Beamer Jagiellonian theme based on Matthias Vogelgesang’s
        Metropolis theme, available under license \LaTeX{} Project
        Public License~1.3c at: \colorhref{https://github.com/matze/mtheme}
        {https://github.com/matze/mtheme}.

        Typographic design: Iwona Grabska-Gradzińska;
        \LaTeX{} Typesetting: Kamil Ziemian \\
        Proofreading: Wojciech Palacz,
        Monika Stawicka \\
        3D Models: Dariusz Frymus, Kamil Nowakowski \\
        Figures and charts: Kamil Ziemian, Paweł Węgrzyn,
        Wojciech Palacz \\
        Film editing: IMAVI -- Joanna Kozakiewicz, Krzysztof Magda, Nikodem
        Frodyma

      \end{flushleft}

    \end{textblock}

  \end{frame}





  \begin{frame}[standout]


    \begingroup

    \color{jFrametitleFGColor}

    #1

    \endgroup

  \end{frame}
}



\newcommand{\GeometryThreeDTwoSpecialEndingSlidesVideoVerThreeEN}[1]{%
  \begin{frame}[standout]

    \begin{textblock}{11}(1,0.7)

      \begin{flushleft}

        \mdseries

        \footnotesize

        \color{jFrametitleFGColor}

        This content was created as part of a project co-financed by the
        European Union within the framework of the European Social Fund
        POWR.03.05.00-00-Z309/17-00.

      \end{flushleft}

    \end{textblock}





    \begin{textblock}{10}(0,2.2)

      \tikz \fill[color=jBackgroundStyleLight] (0,0) rectangle (12.8,-1.5);

    \end{textblock}


    \begin{textblock}{3.2}(0.7,2.45)

      \includegraphics[scale=0.3]{\FundingLogoColorPictureEN}

    \end{textblock}


    \begin{textblock}{2.5}(4.15,2.5)

      \includegraphics[scale=0.2]{\JULogoColorPictureEN}

    \end{textblock}


    \begin{textblock}{2.5}(6.35,2.4)

      \includegraphics[scale=0.1]{\ZintegrUJLogoColorPictureEN}

    \end{textblock}


    \begin{textblock}{4.2}(8.4,2.6)

      \includegraphics[scale=0.3]{\EUSocialFundLogoColorPictureEN}

    \end{textblock}





    \begin{textblock}{11}(1,4)

      \begin{flushleft}

        \mdseries

        \footnotesize

        \RaggedRight

        \color{jFrametitleFGColor}

        The content of this lecture is made available under a Creative
        Commons licence (\textsc{cc}), giving the author the credits
        (\textsc{by}) and putting an obligation to share on the same terms
        (\textsc{sa}). Figures and diagrams included in the lecture are
        authored by Paweł Węgrzyn et~al., and are available under the same
        license unless indicated otherwise. \\ The presentation uses the
        Beamer Jagiellonian theme based on Matthias Vogelgesang’s
        Metropolis theme, available under license \LaTeX{} Project
        Public License~1.3c at: \colorhref{https://github.com/matze/mtheme}
        {https://github.com/matze/mtheme}.

        Typographic design: Iwona Grabska-Gradzińska;
        \LaTeX{} Typesetting: Kamil Ziemian \\
        Proofreading: Leszek Hadasz, Wojciech Palacz,
        Monika Stawicka \\
        3D Models: Dariusz Frymus, Kamil Nowakowski \\
        Figures and charts: Kamil Ziemian, Paweł Węgrzyn,
        Wojciech Palacz \\
        Film editing: Agencja Filmowa Film \& Television Production~--
        Zbigniew Masklak \\
        Film editing: IMAVI -- Joanna Kozakiewicz, Krzysztof Magda, Nikodem
        Frodyma


      \end{flushleft}

    \end{textblock}

  \end{frame}





  \begin{frame}[standout]


    \begingroup

    \color{jFrametitleFGColor}

    #1

    \endgroup

  \end{frame}
}











% ------------------------------------------------------------------------------------
% Importing packages, libraries and setting their configuration
% ------------------------------------------------------------------------------------

% ------------------------------------------------------
% Packages for scientific papers
% ------------------------------------------------------
% Switching off \lll symbol, that I guess is representing letter ``Ł''.
% It collide with `amsmath' package's command with the same name
% \let\lll\undefined
% Basic package from American Mathematical Society (AMS)
% \usepackage[intlimits]{amsmath}
% Equations are numbered separately in every section.
% \numberwithin{equation}{section}





% ------------------------------------------------------
% BibLaTeX
% ------------------------------------------------------
% Package biblatex, with biber as its backend, allow us to handle
% bibliography entries that use Unicode symbols outside ASCII.
\usepackage[
language=polish,
backend=biber,
style=alphabetic,
url=false,
eprint=true,
]{biblatex}

\addbibresource{Experimental-search-for-Casimir-effect-ETC-Bibliography.bib}












% ------------------------------------------------------
% Local packages
% ------------------------------------------------------
% Special configuration for this particular presentation
\usepackage{./Local-packages/local-settings}

% Package containing various command useful for working with a text
\usepackage{./Local-packages/general-commands}

% Package containing commands and other code useful for working with
% mathematical text
\usepackage{./Local-packages/math-commands}

\usepackage{./Local-packages/PGF-TikZ-Arrows-styles}

\usepackage{./Local-packages/PGF-TikZ-Local-pics}

\usepackage{./Local-packages/PGF-TikZ-Functions-pics}

\definecolor{GraySpecial}{HTML}{BCBCBC}
\definecolor{DarkRedSpecial}{HTML}{991C1C}

\definecolor{BrownOne}{HTML}{9B8D7A}

\definecolor{BrownTwo}{HTML}{B67C58}

\definecolor{GraySpecialTwo}{HTML}{695E5E}

\definecolor{YellowSpecialOne}{HTML}{CE7E00}







% ------------------------------------------------------------------------------------------------------------------
\title{Experimental search for Casimir effect}
\subtitle{An~overview}

\author{Kamil Ziemian \\
  \email}


% \institute{Uniwersytet Jagielloński w~Krakowie}

\date[19 XI 2024]{Seminary of Particle Physics Theory Departament,
  19 November 2024}
% ------------------------------------------------------------------------------------------------------------------










% ####################################################################
% Beginning of the document
\begin{document}
% ####################################################################





% ######################################
% Number of chars: 62k+, 73k+, 25k+,
% Text is adjusted to the left and words are broken at the end of the line.
\RaggedRight
% ######################################





% ######################################
\maketitle
% ######################################





% ######################################
\begin{frame}
  \frametitle{Table of contents}


  % Table of contents
  \tableofcontents

\end{frame}
% ######################################
















% ######################################
\section{First theoretical works}
% ######################################



% ##################
\begin{frame}
  \frametitle{Casimir paper from 1948}


  In the year 1948 Hendrik Brught Gerhard Casimir publish three pages
  long article \textit{On~the~Attraction Between Two Perfectly Conducting
    Plates} \parencite{Casimir-On-the-Attraction-Between-ETC-Pub-1948}, in
  which he predicted that two electrical neutral plates put inside the cubic
  cavity will attract each other due to strange relation between ``zero-point
  energy'' of the~electromagnetic field with different boundary conditions.

  As it is well known, see for example a~book by Peskin and Schroeder
  \parencite{Peskin-Schroeder-An-Introduction-to-Quantum-Field-Theory-Pub-1995},
  procedure known as canonical quantisation applied to classical
  electromagnetic field, produce an ``infinite constant'' in the~expression
  for energy, which Casimir called ``the zero-point energy''. His claim
  was that since boundary conditions change ``value of the~infinite
  constant'', their difference should manifest itself as finite force.
  Casimir choose as these two situation a~free field and field with a~strict
  Dirichlet boundary conditions.

\end{frame}
% ##################





% ##################
\begin{frame}
  \frametitle{Casimir view of zero-point energy}


  \textit{In both cases the expressions $\frac{ 1 }{ 2 } \sum \hbar \, \omega$ where
    the~summation extendes over all possible resonance frequencies~of
    the~cavities are divergent and devoid~of physical meaning but
    the~\textit{difference} between these sums in the two situations,
    $\frac{ 1 }{ 2 } ( \sum \hbar \omega )_{ \text{I} } -
    \frac{ 1 }{ 2 } ( \sum \hbar \omega )_{ \text{II} }$, will be shown to have
    a~well defined value and this value will be~interpreted as
    the~interaction between the~plate and the~$xy$ face.} (Cf. p.~793 in
  \parencite{Casimir-On-the-Attraction-Between-ETC-Pub-1948}.)

  To obtain fine result from
  $\delta E = \frac{ 1 }{ 2 } ( \sum \hbar \omega )_{ \text{I} } -
  \frac{ 1 }{ 2 } ( \sum \hbar \omega )_{ \text{II} }$ Casimir needed to introduce
  cutoff function{ } $f$ and cutoff parameter $k_{ \HorSpaceThree m }$. To
  justified their presence he
  recounts that when frequency of electromagnetic
  waves increase, metal objects became more and more transparent to them.
  By expanding such ``improved'' expression into a series and taking only
  zero order term, he arrives at now famous formula:
  \begin{equation}
    \label{eq:First-theoretical-works-01}
    \frac{ \delta E }{ L^{ 2 } } =
    -\hbar c \frac{ \pi^{ 2 } }{ 720 } \frac{ 1 }{ a^{ 3 } }.
  \end{equation}

\end{frame}
% ##################





% ##################
\begin{frame}
  \frametitle{Number of citations of Casimir 1948 paper}


  \begin{figure}

    \label{fig:Casimir-paper-citations}

    \centering


    \includegraphics[scale=0.6]
    {./Presentation-pictures/Casimir-paper-citations.png}

    \caption{Inspire HEP graph of numbers of citation of Casimir 1948 paper
      per year, \colorhref{https://inspirehep.net/literature/24990}
      {https://inspirehep.net/literature/24990}.}


  \end{figure}

\end{frame}
% ##################





% ##################
\begin{frame}
  \frametitle{Lifshitz theory}


  In the $1955$ E.M. Lifshitz publish paper on in which show how to compute
  Casimir effect taking into account temperature and properties of the
  material of the~plates. Another results of this paper was the proof, that
  he can do physics without Landau. I~used English version of this article
  published in $1956$
  as~\textit{The~theory~of molecular attractive forces between solids}
  \parencite{Lifshitz-The-theory-of-molecular-ETC-Pub-1956}.

  % Outside giving us one proof, that he can do physics without Landau, more
  % importantly he shown us how to compute Casimir effect taking into
  % account some properties of real materials and effect of $T > 0$.

  Since the majority of the~experiments compare their results with
  prediction of some variants of Lifshitz theory, we need to presents it
  outlines for the~system of two plates. Presentation offered here is based
  on article by Klimchitskaya and~Mostepanenko
  \parencite{Klimchitskaya-Mostepanenko-Current-status-ETC-Pub-2022}.
  As~a~caveat we need to add, that in his original work, Lifshitz
  considered not two plates~of finite width, but two infinite half-spaces
  filed with by material medium, but from our point of view this isn't
  very important.

\end{frame}
% ##################




% ##################
\begin{frame}
  \frametitle{Lifshitz theory}


  In Lifshitz approach dielectric permittivity $\varepsilon( \omega )$ and magnetic
  permeability $\mu( \omega )$ are functions of electromagnetic waves frequency
  and can also depend on~$T$. We introduce auxiliary notation

  \vspace{-1em}



  \begin{equation}
    \label{eq:First-theoretical-works-02}
    k( \omega, k_{ \HorSpaceOne \bot } ) =
    \sqrt{ ( k_{ \HorSpaceOne \bot } )^{ 2 } -
      \varepsilon( \omega ) \mu( \omega ) \frac{ \omega^{ 2 } }{ c^{ 2 } } }, \quad
    q( \omega, k_{ \HorSpaceOne \bot } ) =
    \sqrt{ ( k_{ \HorSpaceOne \bot } )^{ 2 } -
      \frac{ \omega^{ 2 } }{ c^{ 2 } } },
  \end{equation}

  \vspace{-1em}



  where
  $k_{ \HorSpaceOne \bot } = \sqrt{ k_{ \HorSpaceOne x }^{ \HorSpaceTen 2 } +
    k_{ \HorSpaceOne y }^{ \HorSpaceTen 2 } }$ is the length of the component of
  the wave vector $k$ laying in the plane of the~plates.

  We now define transverse electric (\textsc{te}) and transverse magnetic
  (\textsc{tm}) reflection coefficients as
  ($q \equiv q( \omega, k_{ \HorSpaceOne \bot } )$):

  \vspace{-1.5em}



  \begin{subequations}

    \begin{align}
      \label{eq:First-theoretical-works-03-A}
      r_{ \, \text{TE} }( \omega, k_{ \HorSpaceOne \bot } )
      &=
        \frac{ \mu( \omega ) q - k( \omega, k_{ \HorSpaceOne \bot } ) }
        { \mu( \omega ) q + k( \omega, k_{ \bot } ) }, \\[0.5em]
      \label{eq:First-theoretical-works-03-B}
      r_{ \, \text{TM} }( \omega, k_{ \HorSpaceOne \bot } )
      &=
        \frac{ \varepsilon( \omega ) q - k( \omega, k_{ \HorSpaceOne \bot } ) }
        { \varepsilon( \omega ) q + k( \omega, k_{ \HorSpaceOne \bot } ) }.
    \end{align}

  \end{subequations}

\end{frame}
% ##################





% ##################
\begin{frame}
  \frametitle{Lifshitz formula for Casimir force}

  \vspace{-0.75em}


  Force per unit area is now given by
  \begin{equation}
    \label{eq:First-theoretical-works-04}
    \begin{split}
      F_{ a }( T )
      &=
        -\frac{ \hbar }{ 2 \pi^{ 2 } }
        \int\limits_{ 0 }^{ \infty } dk_{ \HorSpaceOne \bot } \, k_{ \HorSpaceOne \bot }
        \int\limits_{ 0 }^{ \infty } d \omega \, \coth\left( \frac{ \hbar \omega }{
        2 k_{ \HorSpaceThree \text{B} } \HorSpaceOne T } \right) \times \\[0.5em]
      &\hspace{2.5em}
        \times \, \Imag\left\{ \frac{ q( \omega, k_{ \HorSpaceOne \bot } ) }{
        r_{ \, \text{TM} }( \omega, k_{ \bot } )^{ -2 } e^{ \HorSpaceOne 2 a q } - 1 } +
        \frac{ q( \omega, k_{ \HorSpaceOne \bot } ) }{
        r_{ \, \text{TE} }( \omega, k_{ \bot } )^{ -2 } e^{ \HorSpaceOne 2 a q } - 1 }
        \right\}.
    \end{split}
  \end{equation}
  To simplify this formula Lifshitz himself replace integrals over
  $[ 0, + \infty )$, by integrals over imaginary axis. Today such procedure is
  called a~Matsubara formalism. This and other result of Lifshitz theory we
  must leave outside scope of this talk.

  What need to be noted, that to compute Casimir force using above formula,
  we need empirical data of how material of plates interact with
  electromagnetic field in broad range of frequencies. We also need some
  theoretical model of function $\varepsilon( \omega )$.

\end{frame}
% ##################





% ##################
\begin{frame}
  \frametitle{Plasma and Drude model}


  Two most widely used in the~field of Casimir effect models of $\varepsilon( \omega )$
  are well know from condensed matter physic plasma model
  $\varepsilon_{ \HorSpaceTwo \text{p} }( \omega )$ and Drude model
  $\varepsilon_{ \HorSpaceThree \text{D} }( \omega )$:
  \begin{equation}
    \label{eq:First-theoretical-works-05}
    \varepsilon_{ \HorSpaceTwo \text{p} }( \omega ) =
    1 - \frac{ \omega_{ \HorSpaceTwo \text{p} }^{ \, 2 } }{ \omega^{ 2 } }, \qquad
    \varepsilon_{ \HorSpaceThree \text{D} }( \omega ) =
    1 - \frac{ \omega_{ \HorSpaceTwo \text{p} }^{ \, 2 } }{ \omega [ \omega + i \gamma( T ) ] }.
  \end{equation}
  We go back to this topic at the end of the talk.

\end{frame}
% ##################










% ######################################
\section{Sparnaay experiment from 1958}
% ######################################


% ##################
\begin{frame}
  \frametitle{First attempt of measuring Casimir force}


  We first need to note that according to original Casimir result
  \eqref{eq:First-theoretical-works-01},
  force between two bodies of surface area $1 \, \si{cm}^{ 2 }$
  put at the distance $1 \, \mu\si{m}$ is of the order of magnitude
  \begin{equation}
    \label{eq:Sparnaay-experiment-from-ETC-01}
    F_{ a } \sim 10^{ -7 } \, \si{N}.
  \end{equation}
  We can immediately see that such effect wouldn't be easy to measure.
  Casimir himself end his $1948$ paper writing „Althrough the effect is
  small, an experimental confirmation seems not unfeasible and might be of
  a~certain interst.”
  (p.~$795$~\parencite{Casimir-On-the-Attraction-Between-ETC-Pub-1948}),
  which can serve as very fitting description of the problem.

  Results of first such experimental attempt to measure this effect were
  published by M.J. Sparnaay in $1958$ paper \textit{Measurements of
    attractive forces between flat plates}
  \parencite{Sparnaay-Measurments-of-attractive-forces-ETC-Pub-1958}.
  This experiment can be described as important failure.

\end{frame}
% ##################





% ##################
\begin{frame}
  \frametitle{Sparnaay results}


  Sparnaay measured force acting on three pairs of metal plates keep at the
  distance of few $\mu\si{m}$ from each other and compared experimental data
  with original Casimir formula \eqref{eq:First-theoretical-works-01}. For
  the pairs chromium-chromium and chromium-steel plates he observed
  attractive force,
  while for pair aluminium-aluminium plates he find presence of repulsive
  force. It is clear that such results cannot be deemed as conclusive.

  What is of the most importance to us, is a list of weak points of the
  experiment, that Sparnaay provided for us and which had a big impact on
  future research. According to his the biggest problem of his method was
  a~big hysteresis of the experimental system, which was result of sharp
  edges of plates and presence of the spring scale attached to them.
  Second, he
  wasn't able to eliminate difference of electrostatic potentials between
  plates. According to his estimates $\Delta V = 17 \, \si{mV}$ is big enough to
  make measurement of Casimir force impossible.

\end{frame}
% ##################





% ##################
\begin{frame}
  \frametitle{Problems of Sparnaay experiments}


  Another problem comes from the presence of dust particles as large
  as~$2 \, \mu\si{m}$, which present can lead to difference in electrostatic
  potential big enough to overcome Casimir force. Despite various tries,
  Sparnaay wasn't able this eliminate them form experimental devices.

  We also should noted, that since distance between plates was of the order
  of $\mu\si{m}$, dust particle of size $2 \, \mu\si{m}$ is in
  comparisons a~very big object. Their presence also prevent the metal
  plates of touching each other, with minimal distance between them
  recorded in the experiment was $0.2 \, \mu\si{m}$.

  The last thing we need to noted is the fact, that keeping plates parallel
  to each other was also a~very hard task and distance between them can
  change by $10\%$ of the total value depending on the place in which it
  was measured.

\end{frame}
% ##################










% ######################################
\section{van Blokland and Overbeek experiment from 1978}
% ######################################



% ##################
\begin{frame}
  \frametitle{Advances after 1958}


  The~time period between Sparnaay paper from $1958$ and publication of
  thic by P.~van Blokland and J. Overbeek from $1978$ shouldn't be
  considered a~barren field, but we need to largely glossed over it.
  We should
  noted, that due to problems encountered by Sparnaay in the following
  years different geometrical configurations were tested and new
  non-conducting materials were used.

  Sphere-plate (lens-plate) configuration has clear advantage over two
  plates, because problem of putting objects parallel to each other is
  mostly eliminated. I wrote ``mostly'', because producing spherical
  object~of the radius~of $10 \, \si{cm}$ and preventing it from big
  deformation, can be surprisingly hard thing to do. In very important
  experiment of S.~Lamoreaux from $1997$, only after dismalting experiment
  set, previously measured radius $( 11.3 \pm 0.1 ) \, \si{cm}$ was found to
  be in reality equal to $( 12.5 \pm 0.3 ) \, \si{cm}$
  \parencite{Lamoreaux-Demonstration-of-the-Casimir-Force-ETC-Pub-1997},
  \parencite{Lamoreaux-Erratum-Demonstration-of-the-Casimir-ETC-Pub-1998}.
  Regardless of that, to this day, this is the most used experimental
  configuration.

\end{frame}
% ##################





% ##################
\begin{frame}
  \frametitle{Sphere-plate configuration}


  \begin{figure}

    \centering


    \begin{tikzpicture}

      % Distance between sphere and plate
      \draw[double axis arrow,thin] (4.55,0.15) -- (4.55,3.3);

      \node at (4.75,1.722) {$a$};



      % % Distance between pair of plates
      % \draw[double axis arrow,thin] (0.65,0.15) -- (0.65,4.605);
      % % 2.225
      % \node at (0.85,2.375) {$z$};



      % Fragment of the sphere
      \fill[color=GraySpecial,opacity=0.6] (0,5.2)
      arc (225:315:6.467) -- cycle;



      % Plate
      \fill[color=GraySpecial] (0,0) rectangle (9.1,0.15);



      % Dashed lines
      % \draw[dashed,opacity=0.7] (0,0) -- (0,4.68);

      % \draw[dashed,opacity=0.7] (1.3,0) -- (1.3,4.68);

      % \draw[dashed,opacity=0.7] (2.6,0) -- (2.6,3.868);

      % \draw[dashed,opacity=0.7] (3.9,0) -- (3.9,3.445);

      % \draw[dashed,opacity=0.7] (5.2,0) -- (5.2,3.348);

      % \draw[dashed,opacity=0.7] (6.5,0) -- (6.5,3.868);

      % \draw[dashed,opacity=0.7] (7.8,0) -- (7.8,4.68);

      % \draw[dashed,opacity=0.7] (9.1,0) -- (9.1,4.68);



      % % Plates of PFA
      % \pic at (0.65,4.68) {plate PFA};

      % \pic at (1.95,3.868) {plate PFA};

      % \pic at (3.25,3.445) {plate PFA};

      % \pic at (4.55,3.348) {plate PFA};

      % \pic at (5.85,3.445) {plate PFA};

      % \pic at (7.15,3.868) {plate PFA};

      % \pic at (8.45,4.68) {plate PFA};

    \end{tikzpicture}

    \caption{Sphere-plate system.}


    \label{fig:}


  \end{figure}

\end{frame}
% ##################





% ##################
\begin{frame}
  \frametitle{van Blokland and Overbeek experiment}


  At this moment we should point out, that according to theoretical
  computation Casimir force is very strongly geometrical and distance
  dependent. If we compute it, using standard methods, for the system of
  single sphere with massive scalar field, we will find that it tries to
  shrink sphere at small distance and expand it at large.

  P.~van Bloklanda and J.~Overbeeka, published they results in $1978$ paper
  \textit{van der Waals forces between objects covered with a~chromium
    layer}
  \parencite{Blokland-Overbeek-van-der-Waals-Forces-between-ETC-Pub-1978}.
  They investigated sphere-plate configuration, when distance between them
  was in the range $132\text{-}670 \, \si{nm}$. Both object were covered by
  the layer of chromium. Unfortunately, formation of chromium oxides on the
  surface couldn't be prevented and this contribution to them whole effect
  must be included in theoretical computations.

\end{frame}
% ##################





% ##################
\begin{frame}
  \frametitle{Procedura indukcyjna Epsteina-Glasera}


  Another problem was caused by the fact that chromium has two absorption
  bands at the $\lambda = 600 \, \si{nm}$. To simplify model they were replaced
  with single absorption band. It was estimated that this single band
  contributed around $40\%$ (!) of measured effect. Even after such
  simplifications computation based on Lifshitz theory wasn't achievable and
  use of some empirical data and numerical approximations was needed.

  Also, the electric potential difference of $20 \, \si{mV}$ cannot be
  eliminated, so appropriate compensating voltage was  applied
  to the~system.

  At the end, with accuracy of $50\%$ of the measured effect, particular
  theoretical predictions were in agreement with measured data. For this
  reason, this experiment is considered to be the first experimental
  confirmation~of existence of Casimir effect. At the same time, it
  contribution to the understanding of \textsc{qed} was quite small.

\end{frame}
% ##################










% ######################################
\section{G.~Bressi et al. experiment from 2002}
% ######################################



% ##################
\begin{frame}
  \frametitle{The main idea}

  \vspace{-0.5em}


  According to my knowledge, the most accuratte experimental results for
  the system of two plates were published in the paper by G.~Bressi,
  G.~Carugno, R.~Onofrio and G.~Ruoso \textit{Measurement of the Casimir
    Force between Parallel Metallic Surfaces} from $2002$
  \parencite{Bressi-et-al-Measurement-of-the-Casimir-Force-ETC-Pub-2002}.

  In the first step G.~Bressi et al. rewrote Casimir formula as
  \begin{equation}
    \label{eq:G-Bressi-et-al-ETC-01}
    F_{ a } =
    -\frac{ K_{ \HorSpaceTwo \text{C} } }{ d^{ 4 } }, \qquad
    K_{ \HorSpaceTwo \text{C} } = \frac{ \pi h c }{ 480 } \approx
    1.3 \cdot 10^{ -27 } \, \si{N.m^{ 2 }}.
  \end{equation}
  We will use symbol $d$ for the distance between plates, to be consistent
  with the original paper. The aim of their experiment is to measure
  experimental value of physical constant from this equation, that they
  denote as $K_{ \HorSpaceTwo \text{C}, \, \text{exp} }$.

  In the experiment they use two plates. First one was called
  \textit{resonator} and had dimensions
  $1.9 \, \si{cm} \times 1.2 \, \si{mm} \times 47 \, \mu\si{m}$.
  Second was called \textit{source} and had dimensions
  $1.9 \, \si{cm} \times 1.2 \, \si{mm} \times 0.5 \, \si{mm}$. Both were covered
  with $50 \, \si{nm}$ layer of chromium.

\end{frame}
% ##################





% ##################
\begin{frame}
  \frametitle{Sketch of experimental settings}


  \begin{figure}

    \label{aaa:bbb}


    \centering

    \begin{tikzpicture}


      \fill[color=BrownOne] (5.75,1.85) -- (5.75,3.15) -- (1.45,3.15) --
      (1.45,1.85) -- cycle;

      \fill[color=BrownOne] (1.5,4) -- (1,4) -- (1,1) -- (1.5,1) -- cycle;

      \fill[color=brown] (1,3.75) -- (0.84,3.75) -- (0.84,1.25) --
      (1,1.25) -- cycle;

      \node at (-1,1) {Resonator};



      \draw[color=black,line width=0.7] (1.7,3.15) -- (2,1.85);

      \draw[color=black,line width=0.7] (2.3,3.15) -- (2.6,1.85);

      \draw[color=black,line width=0.7] (2.9,3.15) -- (3.2,1.85);

      \draw[color=black,line width=0.7] (3.5,3.15) -- (3.8,1.85);

      \draw[color=black,line width=0.7] (4.1,3.15) -- (4.4,1.85);

      \draw[color=black,line width=0.7] (4.7,3.15) -- (5,1.85);

      \draw[color=black,line width=0.7] (5.3,3.15) -- (5.6,1.85);


      \fill[color=gray] (-5,-0.5) -- (5,-0.5) -- (5,5) -- (4.5,5) --
      (4.5,0) -- (-5,0) -- cycle;






      % Source
      \fill[color=brown] (-0.08,0) rectangle (-0.01,2.5);


      \node at (-0.45,4.3) {Source};

      \draw[pointing arrow] (-0.5,4) -- (0.75,2.8);



      \fill[color=GraySpecialTwo] (-4,0) -- (-2.8,0) -- (-2.8,1.65) --
      (-3.5,1.65) -- (-3.5,2.25) -- (-2.8,2.25) -- (-2.8,2.7) -- (-4,2.7) --
      cycle;

      \node at (-3.3,3) {Inteferometer};


      \fill[color=YellowSpecialOne] (-3.55,1.65) -- (-2.8,1.65) --
      (-2.8,1.75) -- (-3.45,1.75) -- (-3.45,2.25) -- (-2.8,2.25) --
      (-2.8,2.35) -- (-3.55,2.35) -- cycle;




      \fill[color=DarkRedSpecial] (-3.45,1.75) rectangle (-3.15,2.25);

      %
      \pic[scale=0.5] at (-3.15,2) {sin graph small amplitude 3};

    \end{tikzpicture}

    \caption{Sketch of G.~Bressi et al. experimental settings.}

  \end{figure}

\end{frame}
% ##################





% ##################
\begin{frame}
  \frametitle{Method of measurement}


  Using fibre optic interferometer the resonance frequency $\nu$ of
  the lowest torsional mode of resonator was measured. In the paper
  written in addition with A.~Galvani and F. Veronesem
  \parencite{Bressi-et-al-Experimental-studies-of-macroscopic-ETC-Pub-2001},
  authors derived the the following equation:
  \begin{equation}
    \label{eq:G-Bressi-et-al-ETC-02}
    \Delta \HorSpaceOne \nu( d )^{ 2 } =
    \nu^{ \HorSpaceOne 2 } - \nu_{ \HorSpaceTwo 0 }^{ \; 2 } =
    -C_{ \HorSpaceTwo \text{el} } \frac{ V_{ \text{r} }^{ \, 2 } }{
      d^{ \HorSpaceThree 3 } } -
    \frac{ C_{ \HorSpaceTwo \text{Cas} } }{ d^{ \HorSpaceFour 5 } }.
  \end{equation}
  We now need to explain some notation. First,
  $\nu_{ \HorSpaceFour 0 } = 138.275 \, \si{Hz}$ is resonance frequency of
  free resonator, $S$ is area of overlap of two plates, $V_{ \text{r} }$ is
  residual voltage and $m_{ \HorSpaceThree \text{eff} }$ is effective mass of
  the lowest torsion mode. The two constants $C_{ \HorSpaceTwo \text{el} }$
  and $C_{ \HorSpaceTwo \text{Cas} }$ are defined as
  \begin{equation}
    \label{eq:G-Bressi-et-al-ETC-03}
    C_{ \HorSpaceTwo \text{el} } =
    \frac{ \varepsilon_{ \HorSpaceTwo 0 } \, S }{ 4 \pi^{ 2 }
      m_{ \HorSpaceThree \text{eff} } }, \qquad
    C_{ \HorSpaceTwo \text{Cas} } =
    \frac{ K_{ \HorSpaceTwo \text{C}, \, \text{exp} } \, S }{ \pi^{ 2 }
      m_{ \HorSpaceThree \text{eff} } }.
  \end{equation}
  The part of the equation \eqref{eq:G-Bressi-et-al-ETC-02}
  proportional to $C_{ \HorSpaceTwo \text{el} }$ is called electrostatic term.

\end{frame}
% ##################





% ##################
\begin{frame}
  \frametitle{Experiment}


  Physical constant in Casimir equation can now be computed as
  \begin{equation}
    \label{eq:G-Bressi-et-al-ETC-04}
    K_{ \HorSpaceTwo \text{C}, \, \text{exp} } =
    \frac{ \varepsilon_{ 0 } }{ 4 } \frac{ C_{ \HorSpaceTwo \text{Cas} } }{
      C_{ \HorSpaceTwo \text{el} } }.
  \end{equation}

  \vspace{-1em}



  We can now take another look on the main equation.
  \begin{equation}
    \label{eq:G-Bressi-et-al-ETC-05}
    \Delta \HorSpaceOne \nu( d )^{ 2 } =
    \nu^{ \HorSpaceOne 2 } - \nu_{ \HorSpaceTwo 0 }^{ \; 2 } =
    -C_{ \HorSpaceTwo \text{el} } \frac{ V_{ \text{r} }^{ \, 2 } }{
      d^{ \HorSpaceThree 3 } } -
    \frac{ C_{ \HorSpaceTwo \text{Cas} } }{ d^{ \HorSpaceFour 5 } }.
  \end{equation}
  To find values of physical constants present in it, three series~of
  measurements with different values of $V_{ \text{r} }$ were performed.
  In each series distance between plates was changing from
  $0.5 \, \mu\si{m}$ to $3 \, \mu\si{m}$. After that four series of
  measurements, from which results electrostatic term was subtracted, was
  used to estimate the value of $C_{ \HorSpaceTwo \text{Cas} }$. This procedure
  give us, according to the authors, following results:
  \begin{align}
    \label{eq:G-Bressi-et-al-ETC-06-A}
    C_{ \HorSpaceTwo \text{el} }
    &=
      ( 4.24 \pm 0.11 ) \cdot 10^{ -13 } \, \si{Hz}^{ 2 } \, \si{m}^{ 3 }, \\
    \label{eq:G-Bressi-et-al-ETC-06-B}
    C_{ \HorSpaceTwo \text{Cas} }
    &=
      ( 2.34 \pm 0.34 ) \cdot 10^{ -28 } \, \si{Hz}^{ 2 } \, \si{m}^{ 5 }.
    \end{align}

\end{frame}
% ##################





% ##################
\begin{frame}
  \frametitle{Results}


  Easy computation give us
  \begin{equation}
    \label{eq:Zarys-badan-ETC-14}
    K_{ \HorSpaceTwo \text{C}, \, \text{exp} } =
    ( 1.22 \pm 0.18 ) \cdot 10^{ -27 } \, \si{N.m}^{ 2 }.
  \end{equation}
  This agree withing experimental uncertainty with theoretical
  value $K_{ \HorSpaceTwo \text{C} } = \frac{ \pi h c }{ 480 } \approx
  1.3 \cdot 10^{ -27 } \, \si{N.m^{ 2 }}$.

  We also should noted that for $d = 0.5 \, \mu\si{m}$ we have
  \begin{equation}
    \label{eq:Herdegens-approach-01}
    \frac{ C_{ \HorSpaceTwo \text{el} } }{ d^{ 3 } } \approx
    3.39 \cdot 10^{ 6 } \, \si{Hz}, \qquad
    \frac{ C_{ \HorSpaceTwo \text{Cas} } }{ d^{ 5 } } \approx
    7.49 \cdot 10^{ 3 } \, \si{Hz},
  \end{equation}
  while for $d = 3 \, \mu\si{m}$
  \begin{equation}
    \label{eq:Herdegens-approach-01}
    \frac{ C_{ \HorSpaceTwo \text{el} } }{ d^{ 3 } } \approx
    1.57 \cdot 10^{ 4 } \, \si{Hz}, \qquad
    \frac{ C_{ \HorSpaceTwo \text{Cas} } }{ d^{ 5 } } \approx
    9.63 \cdot 10^{ -1 } \, \si{Hz}.
  \end{equation}

\end{frame}
% ##################















% ######################################
\section{Few words about measurements of Casimir effect in
  XXI century}
% ######################################



% ##################
\begin{frame}
  \frametitle{Other configurations and possible future}


  While sphere-plate configuration remains the staple of the field of
  experimental Casimir effect, in the~time after year $2000$ a variety of
  different geometrical configurations was tested. For example G.~Biamonte,
  D. L\'{o}pez and R.S. Decca measured this effect in system of sphere
  with rotating plate, periodically covered by strips of different types of
  metals
  \parencite{Biomonte-Lopez-Decca-Isoelectronic-determination-ETC-Pub-2016}.
  Unfortunately, I~don't know of any completed measurement of two plate
  system after the publication of paper G.~Bressi et al. in $2002$.

  In this context the \textsc{cannex} project
  (The Casimir And Non-Newtonian force EXperiment) should be mentioned
  \parencite{Sedmik-Pitschmann-Next-Generation-Design-ETC-Pub-2021}. It aim
  is to measure discussed effect in two plate system in and out thermal
  equilibrium, with precision that provide new bounds on axion and
  axion-like dark matter, among other things. It is a~lofty and very
  desirable goal, but challenges before it are quite daunting. At this
  moment, according to my knowledge, no result of this experiment are
  available in literature.

\end{frame}
% ##################





% ##################
\begin{frame}
  \frametitle{Problem of $\MathTextFrametitleFGColor{\varepsilon( \omega )}$ model}


  As was said before to computer Casimir effect using Lifshitz theory, we
  need a~model of the function $\varepsilon( \omega )$ and the two most used are defined
  as:
  \begin{equation}
    \label{eq:First-theoretical-works-05}
    \varepsilon_{ \HorSpaceTwo \text{p} }( \omega ) =
    1 - \frac{ \omega_{ \HorSpaceTwo \text{p} }^{ \, 2 } }{ \omega^{ 2 } }, \qquad
    \varepsilon_{ \HorSpaceThree \text{D} }( \omega ) =
    1 - \frac{ \omega_{ \HorSpaceTwo \text{p} }^{ \, 2 } }{ \omega [ \omega + i \gamma( T ) ] }.
  \end{equation}
  They are called plasma and Drude model.

  From the time around year $2005$ following start to appear. According to
  condense matter physics, Drude model should be more realistic, but
  measurements of Casimir force consistently excluded it, in the favor of
  plasma model.
  Since Casimir force measurements are very complicated, they should not be
  used for validation of permittivity models, but this is still a~big
  problem.
  Debate about its source spans around two decades and
  according to my knowledge, at current moment it isn't yet solved.

\end{frame}
% ##################







% ##################
\begin{frame}
  \frametitle{The end}

  \vspace{7em}


  \begin{center}

    \Large

    Thank you. \\
    Are there any questions?

  \end{center}

\end{frame}
% ##################






% % ######################################
% \section{Procedura indukcyjna Epsteina-Glasera}
% % ######################################



% ####################################################################
\appendix
% ####################################################################



% ##################
\begin{frame}
  \frametitle{Proximity force approximation}


\begin{figure}

  \centering


  \begin{tikzpicture}

    % Distance between sphere and plate
    \draw[double axis arrow,thin] (4.55,0.15) -- (4.55,3.25);

    \node at (4.75,1.722) {$a$};



    % Distance between pair of plates
    \draw[double axis arrow,thin] (0.65,0.15) -- (0.65,4.605);
    % 2.225
    \node at (0.85,2.375) {$z$};



    % Fragment of the sphere
    \fill[color=GraySpecial,opacity=0.6] (0,5.2)
    arc (225:315:6.467) -- cycle;



    % Plate
    \fill[color=GraySpecial] (0,0) rectangle (9.1,0.15);



    % Dashed lines
    \draw[dashed,opacity=0.7] (0,0) -- (0,4.68);

    \draw[dashed,opacity=0.7] (1.3,0) -- (1.3,4.68);

    \draw[dashed,opacity=0.7] (2.6,0) -- (2.6,3.868);

    \draw[dashed,opacity=0.7] (3.9,0) -- (3.9,3.445);

    \draw[dashed,opacity=0.7] (5.2,0) -- (5.2,3.348);

    \draw[dashed,opacity=0.7] (6.5,0) -- (6.5,3.868);

    \draw[dashed,opacity=0.7] (7.8,0) -- (7.8,4.68);

    \draw[dashed,opacity=0.7] (9.1,0) -- (9.1,4.68);



    % Plates of PFA
    \pic at (0.65,4.68) {plate PFA};

    \pic at (1.95,3.868) {plate PFA};

    \pic at (3.25,3.445) {plate PFA};

    \pic at (4.55,3.348) {plate PFA};

    \pic at (5.85,3.445) {plate PFA};

    \pic at (7.15,3.868) {plate PFA};

    \pic at (8.45,4.68) {plate PFA};

  \end{tikzpicture}

  \caption{\textsc{pfa} method for sphere-plate system.}


  \label{fig:Ilustration-of-PFA}

\end{figure}


%   \begin{itemize}
%     \RaggedRight

%   \item K. J. Keller, \textit{Dimensional Regularization in~Position
%       Space and~a~Forest Formula for Regularized Epstein\dywiz Glaser
%       Renormalization}, arXiv:~1006.2148v1.

%   \item H. Epstein, V. Glaser, \textit{The~role~of locality
%       in~perturbation theory}, Ann. Inst. H. Poincar\'{e} A
%     \textbf{19} (1973) 211.

%   \item R. Brunetti, K. Fredenhagen [BF00], \textit{Microlocal
%       Analysis and~Interacting Quantum Field Theories: Renormalization
%       on~Physical Backgrounds}, Commun.Math.Phys, \textbf{208} (2000)
%     623-661, arXiv:~9903.028.

%   \item G. Scharf, \textit{Finite quantum electrodynamics}, Springer,
%     1995.

%   \item R. Brunetti, K. Fredenhagen, \textit{Quantum Field
%       Theory\linebreak on~Curved Backgrounds}, Proceedings of the
%     Kompaktkurs \emph{Quantenfeldtheorie auf gekruemmten
%       Raumzeiten}
%     held at~Universitaet Potsdam, Germany, in~8--12.10.2007,
%     arXiv:
%     0901.2063.

%   \item M. D\"{u}etsch, K. Fredenhagen, K. J.
%     Keller, K.~Rejzner, \textit{Dimensional Regularization in~Position
%       Space, and~a Forest Formula for Epstein-Glaser Renormalization},
%     arXiv: 1311.5424, [DFKR13].

%   \end{itemize}

\end{frame}
% ##################


% ##################



% ##################






























% ####################################################################
% ####################################################################
% Bibliography

\printbibliography





% ############################
% End of the document

\end{document}
