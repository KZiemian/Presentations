% ------------------------------------------------------------------------------------------------------------------
% Basic configuration of Beamera class and Jagiellonian theme
% ------------------------------------------------------------------------------------------------------------------
\RequirePackage[l2tabu, orthodox]{nag}



\ifx\PresentationStyle\notset
  \def\PresentationStyle{light}
\fi



% Options: t -- align frame text to the top
\documentclass[10pt,t]{beamer}
\mode<presentation>
\usetheme[style=\PresentationStyle]{jagiellonian}





% ------------------------------------------------------------------------------------
% Procesing configuration files of Jagiellonian theme located in
% the directory "preambule"
% ------------------------------------------------------------------------------------
% Configuration for polish language
% Need description
\usepackage[english]{babel}





% % ------------------------------
% % Better support of polish chars in technical parts of PDF
% % ------------------------------
% \hypersetup{pdfencoding=auto,psdextra}

% Package "textpos" give as enviroment "textblock" which is very usefull in
% arranging text on slides.

% This is standard configuration of "textpos"
\usepackage[overlay,absolute]{textpos}

% If you need to see bounds of "textblock's" comment line above and uncomment
% one below.

% Caution! When showboxes option is on significant ammunt of space is add
% to the top of textblock and as such, everyting put in them gone down.
% We need to check how to remove this bug.

% \usepackage[showboxes,overlay,absolute]{textpos}



% Setting scale length for package "textpos"
\setlength{\TPHorizModule}{10mm}
\setlength{\TPVertModule}{\TPHorizModule}


% ---------------------------------------
% TikZ
% ---------------------------------------
% Importing TikZ libraries
\usetikzlibrary{arrows.meta}
\usetikzlibrary{positioning}





% % Configuration package "bm" that need for making bold symbols
% \newcommand{\bmmax}{0}
% \newcommand{\hmmax}{0}
% \usepackage{bm}




% ---------------------------------------
% Packages for scientific texts
% ---------------------------------------
% \let\lll\undefined  % Sometimes you must use this line to allow
% "amsmath" package to works with packages with packages for polish
% languge imported
% /preambul/LanguageSettings/JagiellonianPolishLanguageSettings.tex.
% This comments (probably) removes polish letter Ł.
\usepackage{amsmath}  % Packages from American Mathematical Society (AMS)
\usepackage{amssymb}
\usepackage{amscd}
\usepackage{amsthm}
\usepackage{siunitx}  % Package for typsetting SI units.
\usepackage{upgreek}  % Better looking greek letters.
% Example of using upgreek: pi = \uppi


\usepackage{calrsfs}  % Zmienia czcionkę kaligraficzną w \mathcal
% na ładniejszą. Może w innych miejscach robi to samo, ale o tym nic
% nie wiem.










% ---------------------------------------
% Packages written for lectures "Geometria 3D dla twórców gier wideo"
% ---------------------------------------
% \usepackage{./ProgramowanieSymulacjiFizykiPaczki/ProgramowanieSymulacjiFizyki}
% \usepackage{./ProgramowanieSymulacjiFizykiPaczki/ProgramowanieSymulacjiFizykiIndeksy}
% \usepackage{./ProgramowanieSymulacjiFizykiPaczki/ProgramowanieSymulacjiFizykiTikZStyle}





% !!!!!!!!!!!!!!!!!!!!!!!!!!!!!!
% !!!!!!!!!!!!!!!!!!!!!!!!!!!!!!
% EVIL STUFF
\if\JUlogotitle1
\edef\LogoJUPath{LogoJU_\JUlogoLang/LogoJU_\JUlogoShape_\JUlogoColor.pdf}
\titlegraphic{\hfill\includegraphics[scale=0.22]
{./JagiellonianPictures/\LogoJUPath}}
\fi
% ---------------------------------------
% Commands for handling colors
% ---------------------------------------


% Command for setting normal text color for some text in math modestyle
% Text color depend on used style of Jagiellonian

% Beamer version of command
\newcommand{\TextWithNormalTextColor}[1]{%
  {\color{jNormalTextFGColor}
    \setbeamercolor{math text}{fg=jNormalTextFGColor} {#1}}
}

% Article and similar classes version of command
% \newcommand{\TextWithNormalTextColor}[1]{%
%   {\color{jNormalTextsFGColor} {#1}}
% }



% Beamer version of command
\newcommand{\NormalTextInMathMode}[1]{%
  {\color{jNormalTextFGColor}
    \setbeamercolor{math text}{fg=jNormalTextFGColor} \text{#1}}
}


% Article and similar classes version of command
% \newcommand{\NormalTextInMathMode}[1]{%
%   {\color{jNormalTextsFGColor} \text{#1}}
% }




% Command that sets color of some mathematical text to the same color
% that has normal text in header (?)

% Beamer version of the command
\newcommand{\MathTextFrametitleFGColor}[1]{%
  {\color{jFrametitleFGColor}
    \setbeamercolor{math text}{fg=jFrametitleFGColor} #1}
}

% Article and similar classes version of the command
% \newcommand{\MathTextWhiteColor}[1]{{\color{jFrametitleFGColor} #1}}





% Command for setting color of alert text for some text in math modestyle

% Beamer version of the command
\newcommand{\MathTextAlertColor}[1]{%
  {\color{jOrange} \setbeamercolor{math text}{fg=jOrange} #1}
}

% Article and similar classes version of the command
% \newcommand{\MathTextAlertColor}[1]{{\color{jOrange} #1}}





% Command that allow you to sets chosen color as the color of some text into
% math mode. Due to some nuances in the way that Beamer handle colors
% it not work in all cases. We hope that in the future we will improve it.

% Beamer version of the command
\newcommand{\SetMathTextColor}[2]{%
  {\color{#1} \setbeamercolor{math text}{fg=#1} #2}
}


% Article and similar classes version of the command
% \newcommand{\SetMathTextColor}[2]{{\color{#1} #2}}










% ---------------------------------------
% Commands for few special slides
% ---------------------------------------
\newcommand{\EndingSlide}[1]{%
  \begin{frame}[standout]

    \begingroup

    \color{jFrametitleFGColor}

    #1

    \endgroup

  \end{frame}
}










% ---------------------------------------
% Commands for setting background pictures for some slides
% ---------------------------------------
\newcommand{\TitleBackgroundPicture}
{./JagiellonianPictures/Backgrounds/LajkonikDark.png}
\newcommand{\SectionBackgroundPicture}
{./JagiellonianPictures/Backgrounds/LajkonikLight.png}



\newcommand{\TitleSlideWithPicture}{%
  \begingroup

  \usebackgroundtemplate{%
    \includegraphics[height=\paperheight]{\TitleBackgroundPicture}}

  \maketitle

  \endgroup
}





\newcommand{\SectionSlideWithPicture}[1]{%
  \begingroup

  \usebackgroundtemplate{%
    \includegraphics[height=\paperheight]{\SectionBackgroundPicture}}

  \setbeamercolor{titlelike}{fg=normal text.fg}

  \section{#1}

  \endgroup
}










% ---------------------------------------
% Commands for lectures "Geometria 3D dla twórców gier wideo"
% Polish version
% ---------------------------------------
% Komendy teraz wykomentowane były potrzebne, gdy loga były na niebieskim
% tle, nie na białym. A są na białym bo tego chcieli w biurze projektu.
% \newcommand{\FundingLogoWhitePicturePL}
% {./PresentationPictures/CommonPictures/logotypFundusze_biale_bez_tla2.pdf}
\newcommand{\FundingLogoColorPicturePL}
{./PresentationPictures/CommonPictures/European_Funds_color_PL.pdf}
% \newcommand{\EULogoWhitePicturePL}
% {./PresentationPictures/CommonPictures/logotypUE_biale_bez_tla2.pdf}
\newcommand{\EUSocialFundLogoColorPicturePL}
{./PresentationPictures/CommonPictures/EU_Social_Fund_color_PL.pdf}
% \newcommand{\ZintegrUJLogoWhitePicturePL}
% {./PresentationPictures/CommonPictures/zintegruj-logo-white.pdf}
\newcommand{\ZintegrUJLogoColorPicturePL}
{./PresentationPictures/CommonPictures/ZintegrUJ_color.pdf}
\newcommand{\JULogoColorPicturePL}
{./JagiellonianPictures/LogoJU_PL/LogoJU_A_color.pdf}





\newcommand{\GeometryThreeDSpecialBeginningSlidePL}{%
  \begin{frame}[standout]

    \begin{textblock}{11}(1,0.7)

      \begin{flushleft}

        \mdseries

        \footnotesize

        \color{jFrametitleFGColor}

        Materiał powstał w ramach projektu współfinansowanego ze środków
        Unii Europejskiej w ramach Europejskiego Funduszu Społecznego
        POWR.03.05.00-00-Z309/17-00.

      \end{flushleft}

    \end{textblock}





    \begin{textblock}{10}(0,2.2)

      \tikz \fill[color=jBackgroundStyleLight] (0,0) rectangle (12.8,-1.5);

    \end{textblock}


    \begin{textblock}{3.2}(1,2.45)

      \includegraphics[scale=0.3]{\FundingLogoColorPicturePL}

    \end{textblock}


    \begin{textblock}{2.5}(3.7,2.5)

      \includegraphics[scale=0.2]{\JULogoColorPicturePL}

    \end{textblock}


    \begin{textblock}{2.5}(6,2.4)

      \includegraphics[scale=0.1]{\ZintegrUJLogoColorPicturePL}

    \end{textblock}


    \begin{textblock}{4.2}(8.4,2.6)

      \includegraphics[scale=0.3]{\EUSocialFundLogoColorPicturePL}

    \end{textblock}

  \end{frame}
}



\newcommand{\GeometryThreeDTwoSpecialBeginningSlidesPL}{%
  \begin{frame}[standout]

    \begin{textblock}{11}(1,0.7)

      \begin{flushleft}

        \mdseries

        \footnotesize

        \color{jFrametitleFGColor}

        Materiał powstał w ramach projektu współfinansowanego ze środków
        Unii Europejskiej w ramach Europejskiego Funduszu Społecznego
        POWR.03.05.00-00-Z309/17-00.

      \end{flushleft}

    \end{textblock}





    \begin{textblock}{10}(0,2.2)

      \tikz \fill[color=jBackgroundStyleLight] (0,0) rectangle (12.8,-1.5);

    \end{textblock}


    \begin{textblock}{3.2}(1,2.45)

      \includegraphics[scale=0.3]{\FundingLogoColorPicturePL}

    \end{textblock}


    \begin{textblock}{2.5}(3.7,2.5)

      \includegraphics[scale=0.2]{\JULogoColorPicturePL}

    \end{textblock}


    \begin{textblock}{2.5}(6,2.4)

      \includegraphics[scale=0.1]{\ZintegrUJLogoColorPicturePL}

    \end{textblock}


    \begin{textblock}{4.2}(8.4,2.6)

      \includegraphics[scale=0.3]{\EUSocialFundLogoColorPicturePL}

    \end{textblock}

  \end{frame}





  \TitleSlideWithPicture
}



\newcommand{\GeometryThreeDSpecialEndingSlidePL}{%
  \begin{frame}[standout]

    \begin{textblock}{11}(1,0.7)

      \begin{flushleft}

        \mdseries

        \footnotesize

        \color{jFrametitleFGColor}

        Materiał powstał w ramach projektu współfinansowanego ze środków
        Unii Europejskiej w~ramach Europejskiego Funduszu Społecznego
        POWR.03.05.00-00-Z309/17-00.

      \end{flushleft}

    \end{textblock}





    \begin{textblock}{10}(0,2.2)

      \tikz \fill[color=jBackgroundStyleLight] (0,0) rectangle (12.8,-1.5);

    \end{textblock}


    \begin{textblock}{3.2}(1,2.45)

      \includegraphics[scale=0.3]{\FundingLogoColorPicturePL}

    \end{textblock}


    \begin{textblock}{2.5}(3.7,2.5)

      \includegraphics[scale=0.2]{\JULogoColorPicturePL}

    \end{textblock}


    \begin{textblock}{2.5}(6,2.4)

      \includegraphics[scale=0.1]{\ZintegrUJLogoColorPicturePL}

    \end{textblock}


    \begin{textblock}{4.2}(8.4,2.6)

      \includegraphics[scale=0.3]{\EUSocialFundLogoColorPicturePL}

    \end{textblock}





    \begin{textblock}{11}(1,4)

      \begin{flushleft}

        \mdseries

        \footnotesize

        \RaggedRight

        \color{jFrametitleFGColor}

        Treść niniejszego wykładu jest udostępniona na~licencji
        Creative Commons (\textsc{cc}), z~uzna\-niem autorstwa
        (\textsc{by}) oraz udostępnianiem na tych samych warunkach
        (\textsc{sa}). Rysunki i~wy\-kresy zawarte w~wykładzie są
        autorstwa dr.~hab.~Pawła Węgrzyna et~al. i~są dostępne
        na tej samej licencji, o~ile nie wskazano inaczej.
        W~prezentacji wykorzystano temat Beamera Jagiellonian,
        oparty na~temacie Metropolis Matthiasa Vogelgesanga,
        dostępnym na licencji \LaTeX{} Project Public License~1.3c
        pod adresem: \colorhref{https://github.com/matze/mtheme}
        {https://github.com/matze/mtheme}.

        Projekt typograficzny: Iwona Grabska-Gradzińska \\
        Skład: Kamil Ziemian;
        Korekta: Wojciech Palacz \\
        Modele: Dariusz Frymus, Kamil Nowakowski \\
        Rysunki i~wykresy: Kamil Ziemian, Paweł Węgrzyn, Wojciech Palacz

      \end{flushleft}

    \end{textblock}

  \end{frame}
}



\newcommand{\GeometryThreeDTwoSpecialEndingSlidesPL}[1]{%
  \begin{frame}[standout]


    \begin{textblock}{11}(1,0.7)

      \begin{flushleft}

        \mdseries

        \footnotesize

        \color{jFrametitleFGColor}

        Materiał powstał w ramach projektu współfinansowanego ze środków
        Unii Europejskiej w~ramach Europejskiego Funduszu Społecznego
        POWR.03.05.00-00-Z309/17-00.

      \end{flushleft}

    \end{textblock}





    \begin{textblock}{10}(0,2.2)

      \tikz \fill[color=jBackgroundStyleLight] (0,0) rectangle (12.8,-1.5);

    \end{textblock}


    \begin{textblock}{3.2}(1,2.45)

      \includegraphics[scale=0.3]{\FundingLogoColorPicturePL}

    \end{textblock}


    \begin{textblock}{2.5}(3.7,2.5)

      \includegraphics[scale=0.2]{\JULogoColorPicturePL}

    \end{textblock}


    \begin{textblock}{2.5}(6,2.4)

      \includegraphics[scale=0.1]{\ZintegrUJLogoColorPicturePL}

    \end{textblock}


    \begin{textblock}{4.2}(8.4,2.6)

      \includegraphics[scale=0.3]{\EUSocialFundLogoColorPicturePL}

    \end{textblock}





    \begin{textblock}{11}(1,4)

      \begin{flushleft}

        \mdseries

        \footnotesize

        \RaggedRight

        \color{jFrametitleFGColor}

        Treść niniejszego wykładu jest udostępniona na~licencji
        Creative Commons (\textsc{cc}), z~uzna\-niem autorstwa
        (\textsc{by}) oraz udostępnianiem na tych samych warunkach
        (\textsc{sa}). Rysunki i~wy\-kresy zawarte w~wykładzie są
        autorstwa dr.~hab.~Pawła Węgrzyna et~al. i~są dostępne
        na tej samej licencji, o~ile nie wskazano inaczej.
        W~prezentacji wykorzystano temat Beamera Jagiellonian,
        oparty na~temacie Metropolis Matthiasa Vogelgesanga,
        dostępnym na licencji \LaTeX{} Project Public License~1.3c
        pod adresem: \colorhref{https://github.com/matze/mtheme}
        {https://github.com/matze/mtheme}.

        Projekt typograficzny: Iwona Grabska-Gradzińska \\
        Skład: Kamil Ziemian;
        Korekta: Wojciech Palacz \\
        Modele: Dariusz Frymus, Kamil Nowakowski \\
        Rysunki i~wykresy: Kamil Ziemian, Paweł Węgrzyn, Wojciech Palacz

      \end{flushleft}

    \end{textblock}

  \end{frame}





  \begin{frame}[standout]

    \begingroup

    \color{jFrametitleFGColor}

    #1

    \endgroup

  \end{frame}
}



\newcommand{\GeometryThreeDSpecialEndingSlideVideoPL}{%
  \begin{frame}[standout]

    \begin{textblock}{11}(1,0.7)

      \begin{flushleft}

        \mdseries

        \footnotesize

        \color{jFrametitleFGColor}

        Materiał powstał w ramach projektu współfinansowanego ze środków
        Unii Europejskiej w~ramach Europejskiego Funduszu Społecznego
        POWR.03.05.00-00-Z309/17-00.

      \end{flushleft}

    \end{textblock}





    \begin{textblock}{10}(0,2.2)

      \tikz \fill[color=jBackgroundStyleLight] (0,0) rectangle (12.8,-1.5);

    \end{textblock}


    \begin{textblock}{3.2}(1,2.45)

      \includegraphics[scale=0.3]{\FundingLogoColorPicturePL}

    \end{textblock}


    \begin{textblock}{2.5}(3.7,2.5)

      \includegraphics[scale=0.2]{\JULogoColorPicturePL}

    \end{textblock}


    \begin{textblock}{2.5}(6,2.4)

      \includegraphics[scale=0.1]{\ZintegrUJLogoColorPicturePL}

    \end{textblock}


    \begin{textblock}{4.2}(8.4,2.6)

      \includegraphics[scale=0.3]{\EUSocialFundLogoColorPicturePL}

    \end{textblock}





    \begin{textblock}{11}(1,4)

      \begin{flushleft}

        \mdseries

        \footnotesize

        \RaggedRight

        \color{jFrametitleFGColor}

        Treść niniejszego wykładu jest udostępniona na~licencji
        Creative Commons (\textsc{cc}), z~uzna\-niem autorstwa
        (\textsc{by}) oraz udostępnianiem na tych samych warunkach
        (\textsc{sa}). Rysunki i~wy\-kresy zawarte w~wykładzie są
        autorstwa dr.~hab.~Pawła Węgrzyna et~al. i~są dostępne
        na tej samej licencji, o~ile nie wskazano inaczej.
        W~prezentacji wykorzystano temat Beamera Jagiellonian,
        oparty na~temacie Metropolis Matthiasa Vogelgesanga,
        dostępnym na licencji \LaTeX{} Project Public License~1.3c
        pod adresem: \colorhref{https://github.com/matze/mtheme}
        {https://github.com/matze/mtheme}.

        Projekt typograficzny: Iwona Grabska-Gradzińska;
        Skład: Kamil Ziemian \\
        Korekta: Wojciech Palacz;
        Modele: Dariusz Frymus, Kamil Nowakowski \\
        Rysunki i~wykresy: Kamil Ziemian, Paweł Węgrzyn, Wojciech Palacz \\
        Montaż: Agencja Filmowa Film \& Television Production~-- Zbigniew
        Masklak

      \end{flushleft}

    \end{textblock}

  \end{frame}
}





\newcommand{\GeometryThreeDTwoSpecialEndingSlidesVideoPL}[1]{%
  \begin{frame}[standout]

    \begin{textblock}{11}(1,0.7)

      \begin{flushleft}

        \mdseries

        \footnotesize

        \color{jFrametitleFGColor}

        Materiał powstał w ramach projektu współfinansowanego ze środków
        Unii Europejskiej w~ramach Europejskiego Funduszu Społecznego
        POWR.03.05.00-00-Z309/17-00.

      \end{flushleft}

    \end{textblock}





    \begin{textblock}{10}(0,2.2)

      \tikz \fill[color=jBackgroundStyleLight] (0,0) rectangle (12.8,-1.5);

    \end{textblock}


    \begin{textblock}{3.2}(1,2.45)

      \includegraphics[scale=0.3]{\FundingLogoColorPicturePL}

    \end{textblock}


    \begin{textblock}{2.5}(3.7,2.5)

      \includegraphics[scale=0.2]{\JULogoColorPicturePL}

    \end{textblock}


    \begin{textblock}{2.5}(6,2.4)

      \includegraphics[scale=0.1]{\ZintegrUJLogoColorPicturePL}

    \end{textblock}


    \begin{textblock}{4.2}(8.4,2.6)

      \includegraphics[scale=0.3]{\EUSocialFundLogoColorPicturePL}

    \end{textblock}





    \begin{textblock}{11}(1,4)

      \begin{flushleft}

        \mdseries

        \footnotesize

        \RaggedRight

        \color{jFrametitleFGColor}

        Treść niniejszego wykładu jest udostępniona na~licencji
        Creative Commons (\textsc{cc}), z~uzna\-niem autorstwa
        (\textsc{by}) oraz udostępnianiem na tych samych warunkach
        (\textsc{sa}). Rysunki i~wy\-kresy zawarte w~wykładzie są
        autorstwa dr.~hab.~Pawła Węgrzyna et~al. i~są dostępne
        na tej samej licencji, o~ile nie wskazano inaczej.
        W~prezentacji wykorzystano temat Beamera Jagiellonian,
        oparty na~temacie Metropolis Matthiasa Vogelgesanga,
        dostępnym na licencji \LaTeX{} Project Public License~1.3c
        pod adresem: \colorhref{https://github.com/matze/mtheme}
        {https://github.com/matze/mtheme}.

        Projekt typograficzny: Iwona Grabska-Gradzińska;
        Skład: Kamil Ziemian \\
        Korekta: Wojciech Palacz;
        Modele: Dariusz Frymus, Kamil Nowakowski \\
        Rysunki i~wykresy: Kamil Ziemian, Paweł Węgrzyn, Wojciech Palacz \\
        Montaż: Agencja Filmowa Film \& Television Production~-- Zbigniew
        Masklak

      \end{flushleft}

    \end{textblock}

  \end{frame}





  \begin{frame}[standout]


    \begingroup

    \color{jFrametitleFGColor}

    #1

    \endgroup

  \end{frame}
}










% ---------------------------------------
% Commands for lectures "Geometria 3D dla twórców gier wideo"
% English version
% ---------------------------------------
% \newcommand{\FundingLogoWhitePictureEN}
% {./PresentationPictures/CommonPictures/logotypFundusze_biale_bez_tla2.pdf}
\newcommand{\FundingLogoColorPictureEN}
{./PresentationPictures/CommonPictures/European_Funds_color_EN.pdf}
% \newcommand{\EULogoWhitePictureEN}
% {./PresentationPictures/CommonPictures/logotypUE_biale_bez_tla2.pdf}
\newcommand{\EUSocialFundLogoColorPictureEN}
{./PresentationPictures/CommonPictures/EU_Social_Fund_color_EN.pdf}
% \newcommand{\ZintegrUJLogoWhitePictureEN}
% {./PresentationPictures/CommonPictures/zintegruj-logo-white.pdf}
\newcommand{\ZintegrUJLogoColorPictureEN}
{./PresentationPictures/CommonPictures/ZintegrUJ_color.pdf}
\newcommand{\JULogoColorPictureEN}
{./JagiellonianPictures/LogoJU_EN/LogoJU_A_color.pdf}



\newcommand{\GeometryThreeDSpecialBeginningSlideEN}{%
  \begin{frame}[standout]

    \begin{textblock}{11}(1,0.7)

      \begin{flushleft}

        \mdseries

        \footnotesize

        \color{jFrametitleFGColor}

        This content was created as part of a project co-financed by the
        European Union within the framework of the European Social Fund
        POWR.03.05.00-00-Z309/17-00.

      \end{flushleft}

    \end{textblock}





    \begin{textblock}{10}(0,2.2)

      \tikz \fill[color=jBackgroundStyleLight] (0,0) rectangle (12.8,-1.5);

    \end{textblock}


    \begin{textblock}{3.2}(0.7,2.45)

      \includegraphics[scale=0.3]{\FundingLogoColorPictureEN}

    \end{textblock}


    \begin{textblock}{2.5}(4.15,2.5)

      \includegraphics[scale=0.2]{\JULogoColorPictureEN}

    \end{textblock}


    \begin{textblock}{2.5}(6.35,2.4)

      \includegraphics[scale=0.1]{\ZintegrUJLogoColorPictureEN}

    \end{textblock}


    \begin{textblock}{4.2}(8.4,2.6)

      \includegraphics[scale=0.3]{\EUSocialFundLogoColorPictureEN}

    \end{textblock}

  \end{frame}
}



\newcommand{\GeometryThreeDTwoSpecialBeginningSlidesEN}{%
  \begin{frame}[standout]

    \begin{textblock}{11}(1,0.7)

      \begin{flushleft}

        \mdseries

        \footnotesize

        \color{jFrametitleFGColor}

        This content was created as part of a project co-financed by the
        European Union within the framework of the European Social Fund
        POWR.03.05.00-00-Z309/17-00.

      \end{flushleft}

    \end{textblock}





    \begin{textblock}{10}(0,2.2)

      \tikz \fill[color=jBackgroundStyleLight] (0,0) rectangle (12.8,-1.5);

    \end{textblock}


    \begin{textblock}{3.2}(0.7,2.45)

      \includegraphics[scale=0.3]{\FundingLogoColorPictureEN}

    \end{textblock}


    \begin{textblock}{2.5}(4.15,2.5)

      \includegraphics[scale=0.2]{\JULogoColorPictureEN}

    \end{textblock}


    \begin{textblock}{2.5}(6.35,2.4)

      \includegraphics[scale=0.1]{\ZintegrUJLogoColorPictureEN}

    \end{textblock}


    \begin{textblock}{4.2}(8.4,2.6)

      \includegraphics[scale=0.3]{\EUSocialFundLogoColorPictureEN}

    \end{textblock}

  \end{frame}





  \TitleSlideWithPicture
}



\newcommand{\GeometryThreeDSpecialEndingSlideEN}{%
  \begin{frame}[standout]

    \begin{textblock}{11}(1,0.7)

      \begin{flushleft}

        \mdseries

        \footnotesize

        \color{jFrametitleFGColor}

        This content was created as part of a project co-financed by the
        European Union within the framework of the European Social Fund
        POWR.03.05.00-00-Z309/17-00.

      \end{flushleft}

    \end{textblock}





    \begin{textblock}{10}(0,2.2)

      \tikz \fill[color=jBackgroundStyleLight] (0,0) rectangle (12.8,-1.5);

    \end{textblock}


    \begin{textblock}{3.2}(0.7,2.45)

      \includegraphics[scale=0.3]{\FundingLogoColorPictureEN}

    \end{textblock}


    \begin{textblock}{2.5}(4.15,2.5)

      \includegraphics[scale=0.2]{\JULogoColorPictureEN}

    \end{textblock}


    \begin{textblock}{2.5}(6.35,2.4)

      \includegraphics[scale=0.1]{\ZintegrUJLogoColorPictureEN}

    \end{textblock}


    \begin{textblock}{4.2}(8.4,2.6)

      \includegraphics[scale=0.3]{\EUSocialFundLogoColorPictureEN}

    \end{textblock}





    \begin{textblock}{11}(1,4)

      \begin{flushleft}

        \mdseries

        \footnotesize

        \RaggedRight

        \color{jFrametitleFGColor}

        The content of this lecture is made available under a~Creative
        Commons licence (\textsc{cc}), giving the author the credits
        (\textsc{by}) and putting an obligation to share on the same terms
        (\textsc{sa}). Figures and diagrams included in the lecture are
        authored by Paweł Węgrzyn et~al., and are available under the same
        license unless indicated otherwise.\\ The presentation uses the
        Beamer Jagiellonian theme based on Matthias Vogelgesang’s
        Metropolis theme, available under license \LaTeX{} Project
        Public License~1.3c at: \colorhref{https://github.com/matze/mtheme}
        {https://github.com/matze/mtheme}.

        Typographic design: Iwona Grabska-Gradzińska \\
        \LaTeX{} Typesetting: Kamil Ziemian \\
        Proofreading: Wojciech Palacz,
        Monika Stawicka \\
        3D Models: Dariusz Frymus, Kamil Nowakowski \\
        Figures and charts: Kamil Ziemian, Paweł Węgrzyn, Wojciech Palacz

      \end{flushleft}

    \end{textblock}

  \end{frame}
}



\newcommand{\GeometryThreeDTwoSpecialEndingSlidesEN}[1]{%
  \begin{frame}[standout]


    \begin{textblock}{11}(1,0.7)

      \begin{flushleft}

        \mdseries

        \footnotesize

        \color{jFrametitleFGColor}

        This content was created as part of a project co-financed by the
        European Union within the framework of the European Social Fund
        POWR.03.05.00-00-Z309/17-00.

      \end{flushleft}

    \end{textblock}





    \begin{textblock}{10}(0,2.2)

      \tikz \fill[color=jBackgroundStyleLight] (0,0) rectangle (12.8,-1.5);

    \end{textblock}


    \begin{textblock}{3.2}(0.7,2.45)

      \includegraphics[scale=0.3]{\FundingLogoColorPictureEN}

    \end{textblock}


    \begin{textblock}{2.5}(4.15,2.5)

      \includegraphics[scale=0.2]{\JULogoColorPictureEN}

    \end{textblock}


    \begin{textblock}{2.5}(6.35,2.4)

      \includegraphics[scale=0.1]{\ZintegrUJLogoColorPictureEN}

    \end{textblock}


    \begin{textblock}{4.2}(8.4,2.6)

      \includegraphics[scale=0.3]{\EUSocialFundLogoColorPictureEN}

    \end{textblock}





    \begin{textblock}{11}(1,4)

      \begin{flushleft}

        \mdseries

        \footnotesize

        \RaggedRight

        \color{jFrametitleFGColor}

        The content of this lecture is made available under a~Creative
        Commons licence (\textsc{cc}), giving the author the credits
        (\textsc{by}) and putting an obligation to share on the same terms
        (\textsc{sa}). Figures and diagrams included in the lecture are
        authored by Paweł Węgrzyn et~al., and are available under the same
        license unless indicated otherwise.\\ The presentation uses the
        Beamer Jagiellonian theme based on Matthias Vogelgesang’s
        Metropolis theme, available under license \LaTeX{} Project
        Public License~1.3c at: \colorhref{https://github.com/matze/mtheme}
        {https://github.com/matze/mtheme}.

        Typographic design: Iwona Grabska-Gradzińska \\
        \LaTeX{} Typesetting: Kamil Ziemian \\
        Proofreading: Wojciech Palacz,
        Monika Stawicka \\
        3D Models: Dariusz Frymus, Kamil Nowakowski \\
        Figures and charts: Kamil Ziemian, Paweł Węgrzyn, Wojciech Palacz

      \end{flushleft}

    \end{textblock}

  \end{frame}





  \begin{frame}[standout]

    \begingroup

    \color{jFrametitleFGColor}

    #1

    \endgroup

  \end{frame}
}



\newcommand{\GeometryThreeDSpecialEndingSlideVideoVerOneEN}{%
  \begin{frame}[standout]

    \begin{textblock}{11}(1,0.7)

      \begin{flushleft}

        \mdseries

        \footnotesize

        \color{jFrametitleFGColor}

        This content was created as part of a project co-financed by the
        European Union within the framework of the European Social Fund
        POWR.03.05.00-00-Z309/17-00.

      \end{flushleft}

    \end{textblock}





    \begin{textblock}{10}(0,2.2)

      \tikz \fill[color=jBackgroundStyleLight] (0,0) rectangle (12.8,-1.5);

    \end{textblock}


    \begin{textblock}{3.2}(0.7,2.45)

      \includegraphics[scale=0.3]{\FundingLogoColorPictureEN}

    \end{textblock}


    \begin{textblock}{2.5}(4.15,2.5)

      \includegraphics[scale=0.2]{\JULogoColorPictureEN}

    \end{textblock}


    \begin{textblock}{2.5}(6.35,2.4)

      \includegraphics[scale=0.1]{\ZintegrUJLogoColorPictureEN}

    \end{textblock}


    \begin{textblock}{4.2}(8.4,2.6)

      \includegraphics[scale=0.3]{\EUSocialFundLogoColorPictureEN}

    \end{textblock}





    \begin{textblock}{11}(1,4)

      \begin{flushleft}

        \mdseries

        \footnotesize

        \RaggedRight

        \color{jFrametitleFGColor}

        The content of this lecture is made available under a Creative
        Commons licence (\textsc{cc}), giving the author the credits
        (\textsc{by}) and putting an obligation to share on the same terms
        (\textsc{sa}). Figures and diagrams included in the lecture are
        authored by Paweł Węgrzyn et~al., and are available under the same
        license unless indicated otherwise.\\ The presentation uses the
        Beamer Jagiellonian theme based on Matthias Vogelgesang’s
        Metropolis theme, available under license \LaTeX{} Project
        Public License~1.3c at: \colorhref{https://github.com/matze/mtheme}
        {https://github.com/matze/mtheme}.

        Typographic design: Iwona Grabska-Gradzińska;
        \LaTeX{} Typesetting: Kamil Ziemian \\
        Proofreading: Wojciech Palacz,
        Monika Stawicka \\
        3D Models: Dariusz Frymus, Kamil Nowakowski \\
        Figures and charts: Kamil Ziemian, Paweł Węgrzyn, Wojciech
        Palacz \\
        Film editing: Agencja Filmowa Film \& Television Production~--
        Zbigniew Masklak

      \end{flushleft}

    \end{textblock}

  \end{frame}
}



\newcommand{\GeometryThreeDSpecialEndingSlideVideoVerTwoEN}{%
  \begin{frame}[standout]

    \begin{textblock}{11}(1,0.7)

      \begin{flushleft}

        \mdseries

        \footnotesize

        \color{jFrametitleFGColor}

        This content was created as part of a project co-financed by the
        European Union within the framework of the European Social Fund
        POWR.03.05.00-00-Z309/17-00.

      \end{flushleft}

    \end{textblock}





    \begin{textblock}{10}(0,2.2)

      \tikz \fill[color=jBackgroundStyleLight] (0,0) rectangle (12.8,-1.5);

    \end{textblock}


    \begin{textblock}{3.2}(0.7,2.45)

      \includegraphics[scale=0.3]{\FundingLogoColorPictureEN}

    \end{textblock}


    \begin{textblock}{2.5}(4.15,2.5)

      \includegraphics[scale=0.2]{\JULogoColorPictureEN}

    \end{textblock}


    \begin{textblock}{2.5}(6.35,2.4)

      \includegraphics[scale=0.1]{\ZintegrUJLogoColorPictureEN}

    \end{textblock}


    \begin{textblock}{4.2}(8.4,2.6)

      \includegraphics[scale=0.3]{\EUSocialFundLogoColorPictureEN}

    \end{textblock}





    \begin{textblock}{11}(1,4)

      \begin{flushleft}

        \mdseries

        \footnotesize

        \RaggedRight

        \color{jFrametitleFGColor}

        The content of this lecture is made available under a Creative
        Commons licence (\textsc{cc}), giving the author the credits
        (\textsc{by}) and putting an obligation to share on the same terms
        (\textsc{sa}). Figures and diagrams included in the lecture are
        authored by Paweł Węgrzyn et~al., and are available under the same
        license unless indicated otherwise.\\ The presentation uses the
        Beamer Jagiellonian theme based on Matthias Vogelgesang’s
        Metropolis theme, available under license \LaTeX{} Project
        Public License~1.3c at: \colorhref{https://github.com/matze/mtheme}
        {https://github.com/matze/mtheme}.

        Typographic design: Iwona Grabska-Gradzińska;
        \LaTeX{} Typesetting: Kamil Ziemian \\
        Proofreading: Wojciech Palacz,
        Monika Stawicka \\
        3D Models: Dariusz Frymus, Kamil Nowakowski \\
        Figures and charts: Kamil Ziemian, Paweł Węgrzyn, Wojciech
        Palacz \\
        Film editing: IMAVI -- Joanna Kozakiewicz, Krzysztof Magda, Nikodem
        Frodyma

      \end{flushleft}

    \end{textblock}

  \end{frame}
}



\newcommand{\GeometryThreeDSpecialEndingSlideVideoVerThreeEN}{%
  \begin{frame}[standout]

    \begin{textblock}{11}(1,0.7)

      \begin{flushleft}

        \mdseries

        \footnotesize

        \color{jFrametitleFGColor}

        This content was created as part of a project co-financed by the
        European Union within the framework of the European Social Fund
        POWR.03.05.00-00-Z309/17-00.

      \end{flushleft}

    \end{textblock}





    \begin{textblock}{10}(0,2.2)

      \tikz \fill[color=jBackgroundStyleLight] (0,0) rectangle (12.8,-1.5);

    \end{textblock}


    \begin{textblock}{3.2}(0.7,2.45)

      \includegraphics[scale=0.3]{\FundingLogoColorPictureEN}

    \end{textblock}


    \begin{textblock}{2.5}(4.15,2.5)

      \includegraphics[scale=0.2]{\JULogoColorPictureEN}

    \end{textblock}


    \begin{textblock}{2.5}(6.35,2.4)

      \includegraphics[scale=0.1]{\ZintegrUJLogoColorPictureEN}

    \end{textblock}


    \begin{textblock}{4.2}(8.4,2.6)

      \includegraphics[scale=0.3]{\EUSocialFundLogoColorPictureEN}

    \end{textblock}





    \begin{textblock}{11}(1,4)

      \begin{flushleft}

        \mdseries

        \footnotesize

        \RaggedRight

        \color{jFrametitleFGColor}

        The content of this lecture is made available under a Creative
        Commons licence (\textsc{cc}), giving the author the credits
        (\textsc{by}) and putting an obligation to share on the same terms
        (\textsc{sa}). Figures and diagrams included in the lecture are
        authored by Paweł Węgrzyn et~al., and are available under the same
        license unless indicated otherwise.\\ The presentation uses the
        Beamer Jagiellonian theme based on Matthias Vogelgesang’s
        Metropolis theme, available under license \LaTeX{} Project
        Public License~1.3c at: \colorhref{https://github.com/matze/mtheme}
        {https://github.com/matze/mtheme}.

        Typographic design: Iwona Grabska-Gradzińska;
        \LaTeX{} Typesetting: Kamil Ziemian \\
        Proofreading: Wojciech Palacz,
        Monika Stawicka \\
        3D Models: Dariusz Frymus, Kamil Nowakowski \\
        Figures and charts: Kamil Ziemian, Paweł Węgrzyn, Wojciech
        Palacz \\
        Film editing: Agencja Filmowa Film \& Television Production~--
        Zbigniew Masklak \\
        Film editing: IMAVI -- Joanna Kozakiewicz, Krzysztof Magda, Nikodem
        Frodyma

      \end{flushleft}

    \end{textblock}

  \end{frame}
}



\newcommand{\GeometryThreeDTwoSpecialEndingSlidesVideoVerOneEN}[1]{%
  \begin{frame}[standout]

    \begin{textblock}{11}(1,0.7)

      \begin{flushleft}

        \mdseries

        \footnotesize

        \color{jFrametitleFGColor}

        This content was created as part of a project co-financed by the
        European Union within the framework of the European Social Fund
        POWR.03.05.00-00-Z309/17-00.

      \end{flushleft}

    \end{textblock}





    \begin{textblock}{10}(0,2.2)

      \tikz \fill[color=jBackgroundStyleLight] (0,0) rectangle (12.8,-1.5);

    \end{textblock}


    \begin{textblock}{3.2}(0.7,2.45)

      \includegraphics[scale=0.3]{\FundingLogoColorPictureEN}

    \end{textblock}


    \begin{textblock}{2.5}(4.15,2.5)

      \includegraphics[scale=0.2]{\JULogoColorPictureEN}

    \end{textblock}


    \begin{textblock}{2.5}(6.35,2.4)

      \includegraphics[scale=0.1]{\ZintegrUJLogoColorPictureEN}

    \end{textblock}


    \begin{textblock}{4.2}(8.4,2.6)

      \includegraphics[scale=0.3]{\EUSocialFundLogoColorPictureEN}

    \end{textblock}





    \begin{textblock}{11}(1,4)

      \begin{flushleft}

        \mdseries

        \footnotesize

        \RaggedRight

        \color{jFrametitleFGColor}

        The content of this lecture is made available under a Creative
        Commons licence (\textsc{cc}), giving the author the credits
        (\textsc{by}) and putting an obligation to share on the same terms
        (\textsc{sa}). Figures and diagrams included in the lecture are
        authored by Paweł Węgrzyn et~al., and are available under the same
        license unless indicated otherwise.\\ The presentation uses the
        Beamer Jagiellonian theme based on Matthias Vogelgesang’s
        Metropolis theme, available under license \LaTeX{} Project
        Public License~1.3c at: \colorhref{https://github.com/matze/mtheme}
        {https://github.com/matze/mtheme}.

        Typographic design: Iwona Grabska-Gradzińska;
        \LaTeX{} Typesetting: Kamil Ziemian \\
        Proofreading: Wojciech Palacz,
        Monika Stawicka \\
        3D Models: Dariusz Frymus, Kamil Nowakowski \\
        Figures and charts: Kamil Ziemian, Paweł Węgrzyn,
        Wojciech Palacz \\
        Film editing: Agencja Filmowa Film \& Television Production~--
        Zbigniew Masklak

      \end{flushleft}

    \end{textblock}

  \end{frame}





  \begin{frame}[standout]


    \begingroup

    \color{jFrametitleFGColor}

    #1

    \endgroup

  \end{frame}
}



\newcommand{\GeometryThreeDTwoSpecialEndingSlidesVideoVerTwoEN}[1]{%
  \begin{frame}[standout]

    \begin{textblock}{11}(1,0.7)

      \begin{flushleft}

        \mdseries

        \footnotesize

        \color{jFrametitleFGColor}

        This content was created as part of a project co-financed by the
        European Union within the framework of the European Social Fund
        POWR.03.05.00-00-Z309/17-00.

      \end{flushleft}

    \end{textblock}





    \begin{textblock}{10}(0,2.2)

      \tikz \fill[color=jBackgroundStyleLight] (0,0) rectangle (12.8,-1.5);

    \end{textblock}


    \begin{textblock}{3.2}(0.7,2.45)

      \includegraphics[scale=0.3]{\FundingLogoColorPictureEN}

    \end{textblock}


    \begin{textblock}{2.5}(4.15,2.5)

      \includegraphics[scale=0.2]{\JULogoColorPictureEN}

    \end{textblock}


    \begin{textblock}{2.5}(6.35,2.4)

      \includegraphics[scale=0.1]{\ZintegrUJLogoColorPictureEN}

    \end{textblock}


    \begin{textblock}{4.2}(8.4,2.6)

      \includegraphics[scale=0.3]{\EUSocialFundLogoColorPictureEN}

    \end{textblock}





    \begin{textblock}{11}(1,4)

      \begin{flushleft}

        \mdseries

        \footnotesize

        \RaggedRight

        \color{jFrametitleFGColor}

        The content of this lecture is made available under a Creative
        Commons licence (\textsc{cc}), giving the author the credits
        (\textsc{by}) and putting an obligation to share on the same terms
        (\textsc{sa}). Figures and diagrams included in the lecture are
        authored by Paweł Węgrzyn et~al., and are available under the same
        license unless indicated otherwise.\\ The presentation uses the
        Beamer Jagiellonian theme based on Matthias Vogelgesang’s
        Metropolis theme, available under license \LaTeX{} Project
        Public License~1.3c at: \colorhref{https://github.com/matze/mtheme}
        {https://github.com/matze/mtheme}.

        Typographic design: Iwona Grabska-Gradzińska;
        \LaTeX{} Typesetting: Kamil Ziemian \\
        Proofreading: Wojciech Palacz,
        Monika Stawicka \\
        3D Models: Dariusz Frymus, Kamil Nowakowski \\
        Figures and charts: Kamil Ziemian, Paweł Węgrzyn,
        Wojciech Palacz \\
        Film editing: IMAVI -- Joanna Kozakiewicz, Krzysztof Magda, Nikodem
        Frodyma

      \end{flushleft}

    \end{textblock}

  \end{frame}





  \begin{frame}[standout]


    \begingroup

    \color{jFrametitleFGColor}

    #1

    \endgroup

  \end{frame}
}



\newcommand{\GeometryThreeDTwoSpecialEndingSlidesVideoVerThreeEN}[1]{%
  \begin{frame}[standout]

    \begin{textblock}{11}(1,0.7)

      \begin{flushleft}

        \mdseries

        \footnotesize

        \color{jFrametitleFGColor}

        This content was created as part of a project co-financed by the
        European Union within the framework of the European Social Fund
        POWR.03.05.00-00-Z309/17-00.

      \end{flushleft}

    \end{textblock}





    \begin{textblock}{10}(0,2.2)

      \tikz \fill[color=jBackgroundStyleLight] (0,0) rectangle (12.8,-1.5);

    \end{textblock}


    \begin{textblock}{3.2}(0.7,2.45)

      \includegraphics[scale=0.3]{\FundingLogoColorPictureEN}

    \end{textblock}


    \begin{textblock}{2.5}(4.15,2.5)

      \includegraphics[scale=0.2]{\JULogoColorPictureEN}

    \end{textblock}


    \begin{textblock}{2.5}(6.35,2.4)

      \includegraphics[scale=0.1]{\ZintegrUJLogoColorPictureEN}

    \end{textblock}


    \begin{textblock}{4.2}(8.4,2.6)

      \includegraphics[scale=0.3]{\EUSocialFundLogoColorPictureEN}

    \end{textblock}





    \begin{textblock}{11}(1,4)

      \begin{flushleft}

        \mdseries

        \footnotesize

        \RaggedRight

        \color{jFrametitleFGColor}

        The content of this lecture is made available under a Creative
        Commons licence (\textsc{cc}), giving the author the credits
        (\textsc{by}) and putting an obligation to share on the same terms
        (\textsc{sa}). Figures and diagrams included in the lecture are
        authored by Paweł Węgrzyn et~al., and are available under the same
        license unless indicated otherwise. \\ The presentation uses the
        Beamer Jagiellonian theme based on Matthias Vogelgesang’s
        Metropolis theme, available under license \LaTeX{} Project
        Public License~1.3c at: \colorhref{https://github.com/matze/mtheme}
        {https://github.com/matze/mtheme}.

        Typographic design: Iwona Grabska-Gradzińska;
        \LaTeX{} Typesetting: Kamil Ziemian \\
        Proofreading: Leszek Hadasz, Wojciech Palacz,
        Monika Stawicka \\
        3D Models: Dariusz Frymus, Kamil Nowakowski \\
        Figures and charts: Kamil Ziemian, Paweł Węgrzyn,
        Wojciech Palacz \\
        Film editing: Agencja Filmowa Film \& Television Production~--
        Zbigniew Masklak \\
        Film editing: IMAVI -- Joanna Kozakiewicz, Krzysztof Magda, Nikodem
        Frodyma


      \end{flushleft}

    \end{textblock}

  \end{frame}





  \begin{frame}[standout]


    \begingroup

    \color{jFrametitleFGColor}

    #1

    \endgroup

  \end{frame}
}











% ------------------------------------------------------------------------------------
% Importing packages, libraries and setting their configuration
% ------------------------------------------------------------------------------------





% ------------------------------------------------------
% BibLaTeX
% ------------------------------------------------------
% Package biblatex, with biber as its backend, allow us to handle
% bibliography entries that use Unicode symbols outside ASCII.
\usepackage[
language=polish,
backend=biber,
style=alphabetic,
url=false,
eprint=true,
]{biblatex}

\addbibresource{Herdegens-approach-and-two-quasi-delta-ETC-Bibliography.bib}





% ------------------------------------------------------
% Wonderful package PGF/TikZ
% ------------------------------------------------------

% Node and pics for drawing charts
\usepackage{./Local-packages/PGF-TikZ-Chart-nodes-and-pics}

% Styles for arrows
\usepackage{./Local-packages/PGF-TikZ-Arrows-styles}

% Pic for drawing functions
\usepackage{./Local-packages/PGF-TikZ-Functions-pics}






% ------------------------------------------------------
% Local packages
% ------------------------------------------------------
% Special configuration for this particular presentation
\usepackage{./Local-packages/local-settings}

% Package containing various command useful for working with a text
\usepackage{./Local-packages/general-commands}

% Package containing commands and other code useful for working with
% mathematical text
\usepackage{./Local-packages/math-commands}










% ------------------------------------------------------------------------------------------------------------------
\title{Herdegen's approach to Casimir effect in application to
  the~two quasi-delta system}

\author{Kamil Ziemian \\
  \texttt{kziemianfvt@gmail.com}}


% \institute{Uniwersytet Jagielloński w~Krakowie}

\date[13 X 2024]{Seminar of Field Theory Departament,
  13 October 2024}
% ------------------------------------------------------------------------------------------------------------------










% ####################################################################
% Beginning of the document
\begin{document}
% ####################################################################





% ######################################
% Text is adjusted to the left and words are broken at the end of the line.
\RaggedRight
% ######################################





% ######################################
\maketitle
% ######################################





% ######################################
\begin{frame}
  \frametitle{Table of contents}


  \tableofcontents % Spis treści

\end{frame}
% ######################################










% ######################################
\section{Short history of Casimir effect}
% ######################################



% ##################
\begin{frame}
  \frametitle{Casimir paper from 1948}


  In the year 1948 Hendrik Brugt Gerhard Casimir ($1909\text{-}2000$)
  published a three pages long article \textit{On~the~Attraction
    Between Two Perfectly Conducting Plates}
  \parencite{Casimir-On-the-Attraction-Between-ETC-Pub-1948}, in
  which he predicted that two electrical neutral plates put inside the
  cubic cavity will attract each other due to the strange relation between
  ``zero-point energy'' of the~electromagnetic field obeying different
  boundary conditions.

  As it is well known, see for example book by Peskin and Schroeder
  \parencite{Peskin-Schroeder-An-Introduction-to-Quantum-Field-Theory-Pub-1995},
  procedure known as canonical quantisation applied to classical
  electromagnetic field produces ``infinite constant'' in the~expression
  for energy, which Casimir call ``zero-point energy''. Casimir's claim
  was that since boundary conditions change ``value of infinite constant'',
  the difference of two such ``infinite numbers'' should manifest itself
  as finite force.

\end{frame}
% ##################





% ##################
\begin{frame}
  \frametitle{Zero-point energies used by Casimir}


  Casimir first step was to consider zero-point energy of field for empty
  cavity, that is expressed by formula:
  \begin{equation}
    \label{eq:Short-history-ETC-01}
    \frac{ 1 }{ 2 } \left( \sum \hbar \omega \right)_{ \text{II} } =
    \frac{ \hbar c a \HorSpaceFive L^{ 2 } }{ 2 \pi^{ 2 } }
    \int\limits_{ 0 }^{ \infty } \int\limits_{ 0 }^{ \infty } \kappa
    \sqrt{ k_{ \HorSpaceOne z }^{ \,\, 2 } + \kappa^{ 2 } } \, d\kappa \,
    dk_{ \HorSpaceTwo z }.
  \end{equation}
  Notation used here is taken from the original paper.

  On the other hand, zero-point energy in the situation when
  two parallel plates, with Dirichlet boundary conditions, are
  inserted at the distance $a$ from each other is given by:
  \begin{equation}
    \label{eq:Short-history-ETC-02}
    \frac{ 1 }{ 2 } \left( \sum \hbar \omega \right)_{ \text{I} } =
    \frac{ \hbar c \HorSpaceFive L^{ 2 } }{ 2 \pi }
    \sum_{ n = ( 0 ), \, 1 }^{ \infty } \, \int\limits_{ 0 }^{ \infty } \kappa
    \sqrt{ \left( n^{ 2 } \frac{ \pi^{ 2 } }{ a^{ 2 } } \right)
      + \kappa^{ 2 } } \, d\kappa.
  \end{equation}

\end{frame}
% ##################





% ##################
\begin{frame}
  \frametitle{Introduction of cutoff}


  To obtain finite result from
  $\delta E = \frac{ 1 }{ 2 } ( \sum \hbar \omega )_{ \text{I} } -
  \frac{ 1 }{ 2 } ( \sum \hbar \omega )_{ \text{II} }$ Casimir needs to introduce
  cutoff function $f$ and parameter $k_{ \HorSpaceThree m }$.
  After doing that regularized energy difference of zero-point energies is
  given by
  \begin{equation}
    \label{eq:Short-history-ETC-03}
    \begin{split}
      \delta E
      &=
        \frac{ \hbar c \HorSpaceFive L^{ 2 } \pi^{ 2 } }{ 4 a^{ 3 } } \!
        \left( \, \sum_{ ( 0 ), \, 1 }^{ \infty } \,
        \int\limits_{ 0 }^{ \infty } \! d\kappa \, \sqrt{ n^{ 2 } \! + \! u } \,
        f\!\left( \tfrac{ \pi \sqrt{ n^{ 2 } + u } }{ a k_{ \HorSpaceTwo m } }
        \right) \right. \, - \\
      &\left. \;\;\;
        - \int\limits_{ 0 }^{ \infty } \! \int\limits_{ 0 }^{ \infty } \! du \, dn \,
        \sqrt{ n^{ \, 2 } \! + \! u } \,
        f\!\left( \tfrac{ \pi \sqrt{ n^{ 2 } + u } }{
        a k_{ \HorSpaceTwo m } } \right) \right).
    \end{split}
  \end{equation}

  By expanding this expression into a series and taking only zero order term
  Casimir arrives at formula:
  \begin{equation}
    \label{eq:Short-history-ETC-04}
    \frac{ \delta E }{ L^{ 2 } } =
    -\hbar c \frac{ \pi^{ 2 } }{ 720 } \frac{ 1 }{ a^{ 3 } },
  \end{equation}
  which gives us now a famous expression for force, proportional
  to~$a^{ -4 }$.

\end{frame}
% ##################





% ##################
\begin{frame}
  \frametitle{Review of Casimir approach}


  We need to stress that in his article Casimir provides physical
  justification for presence of function $f$ and parameter $k_{ m }$.
  Namely that metal is becoming more transparent to electromagnetic
  radiation, when energy of radiation grows.

  The Casimir method of computation is still present in current literature
  in more or less refined form. I~don't believe this can be viewed
  as satisfactory in $2024$.

  First of all, its use of zero-point energy, which according to
  original~(!) Casimir paper is itself nonphysical, is hard to explain.
  It is worth noticing that for over $700$ pages of Peskin and Schroeder
  book, zero-point energy is considered as physically irrelevant, to
  suddenly be deemed physically important in the context of Casimir
  effect. The authors seem to be dissatisfied by this state of affairs.

\end{frame}
% ##################





% ##################
\begin{frame}
  \frametitle{Review of Casimir approach}


  Second, that it does not introduce physical properties of materials at the
  beginning, but as an auxiliary tool for removing ill defined mathematical
  quantities. For both these reasons, alternative approaches should be
  welcome.

\end{frame}
% ##################





% ##################
\begin{frame}
  \frametitle{Number of citations of Casimir 1948 paper}


  \begin{figure}

    \label{fig:Casimir-paper-citations}

    \centering


    \includegraphics[scale=0.6]
    {./Presentation-pictures/Casimir-paper-citations.png}

    \caption{Inspire HEP graph of numbers of citation of Casimir 1948 paper
      per year, \colorhref{https://inspirehep.net/literature/24990}
      {https://inspirehep.net/literature/24990}.}


  \end{figure}

\end{frame}
% ##################










% ######################################
\section{Overview of Herdegen's approach to Casimir effect}
% ######################################



% ##################
\begin{frame}
  \frametitle{Introduction to Herdegen's approach}


  Approach to Casimir force that we will be discussing, was first announced
  in short paper of Andrzej Herdegen \textit{No-nonsens Casimir force}
  \parencite{Herdegen-Nononsens-Casimir-force-Pub-2001} and fully presented
  in the two part article \textit{Quantum backreaction (Casimir) effect}
  \parencite{Herdegen-Quantum-backreaction-ETC-Part-I-Pub-2005},
  \parencite{Herdegen-Quantum-backreaction-ETC-Part-II-Pub-2006}. Casimir
  effect is interpreted here as backreaction of quantum system
  to adiabatic changes in the macroscopic part of the system and apparatus
  of algebraic quantum field theory (\textsc{aqft}) is used to allow
  rigorous analysis of physical properties of it.

  We need to pay special attention to how the Casimir effect is understood
  in this formalism. According to it, this effect is present in the system
  having the quantum part $Q$ and macroscopic part $M$. In the standard
  example quantum part is quantum electromagnetic field and macroscopic
  contains two metal, electric neutral plates. When $M$ moves
  \alert{adiabatically} there is backreaction on the part of subsystem $Q$,
  that gives rise to particular force.

\end{frame}
% ##################





% ##################
\begin{frame}
  \frametitle{Elements of Herdegen's approach}


  According to my reading of experimental literature on the subject,
  this assumption of adiabatic movement of macroscopic subsystem fits
  well with the way how true experiments are done.

  In \textsc{aqft} quantum system is described by $C^{ * }$ algebra
  $\Acal$, representing physical quantities that can be measured inside it.
  This algebra is attached to region of space time, where measurement of
  this quantities can take place. Herdegen's paper
  \parencite{Herdegen-Quantum-backreaction-ETC-Part-I-Pub-2005}
  presents a set of axioms tailored for discussion of Casimir effect.
  Here we can only mention a few of them.

  We start from the assumption that the quantum subsystem $Q$ is defined by
  its algebra $\Acal$ and representation $\pi$ of elements of this algebra as
  bounded operators on Hilbert space $\Hcal$:
  \begin{equation}
    \label{eq:Overview-of-Herdegens-approach-ETC-01}
    \Acal \ni A \mapsto \pi( A ) \in \Bcal( \Hcal ).
  \end{equation}

\end{frame}
% ##################






% ##################
\begin{frame}
  \frametitle{Algebraic nature of observables}


  Algebra $\Acal$ is localized on hyperplane $t = \const$, which due to
  assumption of adiabatic changes inside the system, pose no problem to
  the~formalism.

  Time evolution on algebra $\Acal$ is given by family of automorphism
  $\alpha_{ \HorSpaceThree t }$ and translates to unitary evolution on $\Hcal$:
  \begin{equation}
    \label{eq:Overview-of-Herdegens-approach-ETC-02}
    \pi( \alpha_{ \HorSpaceThree t } \, A ) \mapsto U( t ) \, \pi( A ) \, U^{ * }( t ), \qquad
    U( t ) = \exp( i t H ).
  \end{equation}
  This relation defines the~hamiltonian of the free quantum system.

  About the macroscopic system $M$ we assume that it changes its states
  adiabatically, due to some external conditions. The backreaction of
  quantum subsystem $Q$ to this movement results in transfer of energy to
  $M$, which manifests itself as Casimir force.

\end{frame}
% ##################





% ##################
\begin{frame}
  \frametitle{AQFT suited for Casimir effect}


  Let~$a$ be a collective name for parameters describing subsystem~$M$.
  For two plates~$a$ reduces to one number, distance of the plates.
  \textsc{aqft} scheme requires that algebra $\Acal$ cannot change when
  part $M$ is introduced to the free system $Q$ and representation $\pi$
  can only change into equivalent one.

  By $H_{ a }$ we will denote the hamiltonian of subsystem $Q$ interacting
  with $M$ in position~$a$. We assume that for every allowed $a$ exists
  non degenerate eigenstate of $H_{ a }$:
  \begin{equation}
    \label{eq:Overview-of-Herdegens-approach-ETC-03}
    H_{ a } \HorSpaceFive \psi_{ \HorSpaceOne a } =
    E_{ \HorSpaceOne a } \HorSpaceFive \psi_{ \HorSpaceOne a }.
  \end{equation}
  In particular, in the presence of $M$ the ground state of the subsystem
  $Q$ becomes a~function of $a$. We will denote it by $\Omega_{ a }$.

\end{frame}
% ##################





% ##################
\begin{frame}
  \frametitle{AQFT suited for Casimir effect}


  We now switch to Schr\"{o}dinger picture of interaction. If movement of
  $M$ is given by function $a( t )$, then according to adiabatic
  approximation we have the following evolution of physical state:
  \begin{equation}
    \label{eq:Overview-of-Herdegens-approach-ETC-04}
    \psi( t ) =
    e^{ \HorSpaceOne i \varphi( t ) } \HorSpaceOne \psi_{ \HorSpaceOne a( t ) }, \qquad
    \psi( 0 ) = \psi_{ \HorSpaceOne a( 0 ) }.
  \end{equation}
  The~expectation value of operator $B$ in the time $t$ is now given by
  \begin{equation}
    \label{eq:Overview-of-Herdegens-approach-ETC-05}
    \langle B \rangle_{ t } =
    ( \psi_{ \HorSpaceOne a( t ) }, B \HorSpaceOne \psi_{ \HorSpaceOne a( t ) } ).
  \end{equation}
  It should be noticed that it doesn't depends on function $\varphi( t )$.

\end{frame}
% ##################





% ##################
\begin{frame}
  \frametitle{AQFT suited for Casimir effect}


  Casimir energy is now defined as expectation value of  hamiltonian of
  \alert{free} subsystem $Q$ in the ground state $\Omega_{ a }$ of interacting
  system $Q\text{-}M$:
  \begin{equation}
    \label{eq:Overview-of-Herdegens-approach-ETC-06}
    \Ecal_{ a } = ( \Omega_{ a }, H \Omega_{ a } ).
  \end{equation}
  This choice may seems controversial, but as Herdegen demonstrated in
  \parencite{Herdegen-Quantum-backreaction-ETC-Part-I-Pub-2005}, this is
  the only unambiguous energy operator present in such settings. We also
  don't see reasons to consider energy that isn't explicitly defined as
  expectation value of self-adjoint operator, even when such approaches to
  Casimir effect are present in the literature, cf. paper by A.~Scardicchio
  about Casimir effect and Dirac delta potentials
  \parencite{Scardicchio-Casimir-dynamics-ETC-2005}.

  Casimir force is now defined in the~standard way:
  \begin{equation}
    \label{eq:Overview-of-Herdegens-approach-ETC-07}
    \Fcal_{ a } = -\frac{ \partial \Ecal_{ a } }{ \partial a }.
  \end{equation}

\end{frame}
% ##################










% ######################################
\section{Construction of quasi-free system and expressions
  for physical quantities}
% ######################################



% ##################
\begin{frame}
  \frametitle{Quasi-free system}


  By \textbf{quasi-free system} we understand system with symplectic space,
  time evolution on it given by family of symplectic transformations, which
  also has specified quantum version. This quantum version must be
  represented in Fock space by canonical variable, that are linear
  combinations of operators $a$ and $a^{ \dagger }$. We also requires that
  quantum hamiltonian is hermitian form in $a$ and $a^{ \dagger }$.

  Andrzej Herdegen and Mariusz Stopa in paper \textit{Global vs~local
    Casimir effect} \parencite{Herdegen-Stopa-Global-vs-local-ETC-2010}
  presented following way of treating Casimir effect. We start from
  free quantum scalar field obeying equation:
  \begin{equation}
    \label{eq:Construction-of-quasi-free-system-ETC-01}
    \left( \partial_{ t }^{ 2 } + h^{ 2 } \right) \phi( t, \vecx ) = 0, \qquad
    h^{ 2 } = -\Delta.
  \end{equation}
  The presence of macroscopic bodies is modeled by perturbing operator
  $h^{ 2 }$ by adding to it operator of finite rank $V_{ a }$:
  \begin{equation}
    \label{eq:Epstein-Glaser-24-A}
    h_{ a }^{ 2 } = h^{ 2 } + V_{ a }.
  \end{equation}

\end{frame}
% ##################





% ##################
\begin{frame}
  \frametitle{Construction of quasi-free system}


  \begin{equation}
    \label{eq:Construction-of-quasi-free-system-ETC-02}
    h_{ a }^{ 2 } = h^{ 2 } + V_{ a }.
  \end{equation}
  This equation defines a model of subsystem $Q$ interacting with $M$.
  We should stress that all effects caused by subsystem $M$ are contained
  in finite-rank operator $V_{ a }$.

  In the next step we need to take square root of operators $h^{ 2 }$
  and $h_{ a }^{ 2 }$, which means that they need to be positive. This
  requirement is source of few subtleties in the problem.

  Assuming that square roots $h$ and $h_{ a }$ of $h^{ 2 }$ and $h_{ a }^{ 2 }$
  exists, we can now use known for a~long time procedure to construct
  quasi-free theory. One reference in which it could be found is Herdegen
  paper \parencite{Herdegen-Quantum-backreaction-ETC-Part-I-Pub-2005}.

\end{frame}
% ##################





% ##################
\begin{frame}
  \frametitle{Number of particles and Casimir energy}


  In paper \parencite{Herdegen-Quantum-backreaction-ETC-Part-I-Pub-2005}
  Herdegen used quite complicated decomposition of field into creation
  and annihilation operators to prove two fundamental relations. First, that
  number of quanta of \alert{free} field created when subsystem $M$ in
  position $a$ is introduced is given by
  \begin{equation}
    \label{eq:Construction-of-quasi-free-system-ETC-03}
    \Ncal_{ a }^{ \vphantom{1} } =
    \frac{ 1 }{ 4 } \Tr\big[ \HorSpaceOne h^{ -1/2 } (
    h_{ \HorSpaceOne a }^{ \vphantom{1} } - h ) h_{ \HorSpaceOne a }^{ -1 }
    ( h_{ a }^{ \vphantom{1} } - h ) h^{ -1/2 } \HorSpaceTwo \big].
  \end{equation}
  Using the same method he proved that
  \begin{equation}
    \label{eq:Construction-of-quasi-free-system-ETC-04}
    \Ecal_{ \HorSpaceTwo a } =
    ( \GroundStateAOne, H \HorSpaceOne \GroundStateAOne ) =
    \frac{ 1 }{ 4 }
    \Tr\big[ ( h_{ \HorSpaceOne a }^{ \vphantom{1} } - h )
    h_{ \HorSpaceOne a }^{ -1 }
    ( h_{ \HorSpaceOne a }^{ \vphantom{1} } - h ) \big].
  \end{equation}
  These two equations are crucial to the rest of the talk.

\end{frame}
% ##################





% ##################
\begin{frame}
  \frametitle{Number of particles and Casimir energy}

  \vspace{-2em}


  \begin{subequations}

    \begin{align}
      \label{eq:Construction-of-quasi-free-system-ETC-05-A}
      \Ncal_{ a }^{ \vphantom{1} }
      &=
        \frac{ 1 }{ 4 } \Tr\big[ \HorSpaceOne h^{ -1/2 } (
        h_{ \HorSpaceOne a }^{ \vphantom{1} } - h )
        h_{ \HorSpaceOne a }^{ -1 }
        ( h_{ a }^{ \vphantom{1} } - h ) h^{ -1/2 } \HorSpaceTwo \big]
        < +\infty, \\
      \label{eq:Construction-of-quasi-free-system-ETC-05-B}
      \Ecal_{ \HorSpaceTwo a }
      &=
        ( \GroundStateAOne, H \HorSpaceOne \GroundStateAOne ) =
        \frac{ 1 }{ 4 }
        \Tr\big[ ( h_{ \HorSpaceOne a }^{ \vphantom{1} } - h )
        h_{ \HorSpaceOne a }^{ -1 }
        ( h_{ \HorSpaceOne a }^{ \vphantom{1} } - h ) \big] < +\infty.
    \end{align}

  \end{subequations}

  \vspace{-1.5em}



  Finitness of number of quanta is fundamental from both physical and
  mathematical point of view. It guarantees that various representations
  of elements underplaying algebra $\Acal$ as operators in Hilbert space are
  equivalent. This allow us to compare results for different values of $a$
  in meaningful way. Requirements of finitness of Casimir energy is natural
  assumption of physical nature.

  Another result of paper
  \parencite{Herdegen-Quantum-backreaction-ETC-Part-I-Pub-2005}
  is that energy density defined by standard energy-momentum tensor is
  given by
  \begin{equation}
    \label{eq:Construction-of-quasi-free-system-ETC-06}
    T_{ a }( \varphi, \psi ) =
    \frac{ 1 }{ 4 } \big( \varphi, ( h_{ a }^{ \vphantom{1} } - h ) \psi \big) +
    \frac{ 1 }{ 4 } \big( \nabla \varphi, ( h_{ a }^{ -1 } -
    h^{ -1 }_{ \vphantom{a} } ) \nabla \psi \big).
  \end{equation}
  Unfortunately, this will be marginal topic in this talk.

\end{frame}
% ##################










% ######################################
\section{Scalar field and two quasi-deltas}
% ######################################



% ##################
\begin{frame}
  \frametitle{Formulation of the problem}


  Again following Herdegen and Stopa
  \parencite{Herdegen-Stopa-Global-vs-local-ETC-2010}, we start with set of
  equations
  \begin{equation}
    \label{eq:Scalar-field-and-two-quasi-deltas-01}
    \left( \partial_{ t }^{ 2 } + h_{ \veca }^{ 2 } \right) \phi( t, \vecx ) = 0, \qquad
    h_{ \veca }^{ 2 } = -\Delta + V_{ \veca }.
  \end{equation}
  We want to represented model in which scalar field interacts with two
  spherically symmetric bodies, which centers are separated by
  vector~$\veca$.

  Following their footsteps we introduce quasi-potential in the form
  \begin{equation}
    \label{eq:Scalar-field-and-two-quasi-deltas-02}
    V_{ \veca }( \vecx, \vecy \HorSpaceThree ) =
    \sigma( g ) \! \left[ g\!\left( \vecx - \vecb \HorSpaceFive \right) \,
      \overline{ g\!\left( \vecy - \vecb \HorSpaceFive \right) }
      + g\!\left( \vecx + \vecb \HorSpaceFive \right) \,
      \overline{ g\!\left( \vecy + \vecb \HorSpaceFive \right) } \,
    \right], \;
    \vecb = \frac{ \veca }{ 2 }.
  \end{equation}
  To be succinct I~presented quasi-potential in the position representation.
  We assumed that function $g( \vecx )$ is spherical symmetric, smooth
  ($C^{ \infty }( \Rbb^{ 3 } )$), with compact support contained in ball of
  radius $L / 2$. Value of $L$
  should be macroscopic, but its particular numeric value is not very
  important at this level of considerations. The real valued functional
  $\sigma( g )$ will be explained latter.

\end{frame}
% ##################





% ##################
\begin{frame}
  \frametitle{Formulation of the problem}


  For technical reason we can only solve our problem for function
  $g( \vecx )$ whose values are non negative:
  \begin{equation}
    \label{eq:Scalar-field-and-two-quasi-deltas-03}
    g( \vecx ) \geq 0.
  \end{equation}

  \vspace{-2em}



  Quasi-potential is projection on the two dimensional subspace spanned by
  functions $g\!\left( \vecx \pm \vecb \HorSpaceFive \right)$.

  Herdegen and Stopa consider two parallel plates system, which due to
  symmetry
  within the plane of plates, reduce to one dimensional problem.
  Because our model, due to rotational symmetry along distinguished axis, is
  also at some level one dimensional, we can use similar methods as
  they before. Still, there are very important differences in behavior
  of both systems.

\end{frame}
% ##################





% ##################
\begin{frame}
  \frametitle{Symbolic sketch of quasi-potential}


  Below we can see simple, symbolic to some extent, sketch of
  quasi-potential, with parameter $L = 2$ in appropriate unit system.
  It shows possible shapes of functions defining two macroscopic bodies.

  \vspace{5em}





  \begin{figure}

    \label{fig:Symbolic-sketch-of-quasi-potential}


    \centering

    \begin{tikzpicture}

      % Smooth functions witch compact support
      \pic at (-3,0) {smooth function with compact support fill 1};

      \pic at (3,0) {smooth function with compact support fill 1};





      % x axis
      \draw[axis arrow] (-4.5,0) -- (5,0);

      \pic at (5,0) {x mark for horizontal axis 1};


      % y axis
      \draw[axis arrow] (0,-0.2) -- (0,2.3);

      \pic at (0,2.3) {V mark for vertical axis 1};

      \node[right] at (0,-0.2) {$0$};





      % Thicks on x axis
      \pic at (-4,0) {tick x axis thin};

      \node[below] at (-4,0) {$-4$};


      \pic at (-3,0) {tick x axis thin};

      \node[below] at (-3,0) {$-3$};


      \pic at (-2,0) {tick x axis thin};

      \node[below] at (-2,0) {$-2$};


      \pic at (-1,0) {tick x axis thin};

      \node[below] at (-1,0) {$-1$};


      \pic at (1,0) {tick x axis thin};

      \node[number below x axis] at (1,0) {$1$};


      \pic at (2,0) {tick x axis thin};

      \node[number below x axis] at (2,0) {$2$};


      \pic at (3,0) {tick x axis thin};

      \node[number below x axis] at (3,0) {$3$};


      \pic at (4,0) {tick x axis thin 1};

      \node[number below x axis] at (4,0) {$4$};


      % \pic at (5,0) {tick x axis thin};

      % \node[below] at (5,0) {$5$};





      % Ticks on y axis
      \pic at (0,0.5) {tick y axis thin 1};

      \node[node scale small 2,left] at (0,0.5) {$0.5$};


      \pic at (0,1) {tick y axis thin};

      \node[left] at (0,1) {$1$};


      \pic at (0,1.5) {tick y axis thin 1};

      \node[node scale small 2,left] at (0,1.5) {$1.5$};

    \end{tikzpicture}

    \caption{Symbolic sketch of quasi-potential $V_{ \veca }$}


  \end{figure}

\end{frame}
% ##################





% ##################
\begin{frame}
  \frametitle{Few basic properties}


  We need to assume that distance between center of objects
  $a = \absOne{ \veca }$ is bigger than their dimeter $L$.

  Since our quasi-potential $V_{ a }$ is operator of finite order,
  Kato-Rellich and Kato-Rosenblum theorems guaranties that $h_{ a }^{ 2 }$ is
  self-adjoint operator and that absolutely continuous spectrum of
  $h^{ 2 }$ and $h_{ a }^{ 2 }$ are unitary equivalent. After some
  computation we can prove that both $h^{ 2 }$ and $h_{ a }^{ 2 }$ have only
  absolutely continuous spectrum equal to $[ 0, +\infty )$, which opens
  the~way to use the scattering theory methods to solve this problem. To
  describe derived results we need to introduce more functions.

\end{frame}
% ##################





% ##################
\begin{frame}
  \frametitle{Functions used in computations}


  Because function $g( \vecx )$ has spherical symmetry, we can replace it
  by function of one real variable, that we will be denoting simply
  by $g( x )$. Simple calculation shows that Fourier transform of
  $g( \vecx )$,
  which is also spherically symmetric, can be expressed as
  \begin{equation}
    \label{eq:Scalar-field-and-two-quasi-deltas-04}
    \gHat( p ) = \gHat( \vecp \HorSpaceFour ) =
    \frac{ 1 }{ \sqrt{ 2 \pi } }
    \int\limits_{ -\infty }^{ +\infty } dx \, \frac{ \sin( p x ) }{ p x }
    \HorSpaceTwo x^{ 2 } g( x ).
  \end{equation}
  We again replaced function on $\Rbb^{ 3 }$ by function on $\Rbb$.

  We now define a bunch of functions.

  \vspace{-2em}



  \begin{subequations}

    \begin{align}
      \label{eq:Scalar-field-and-two-quasi-deltas-06-A}
      M_{ p }
      &= \gHat( p )^{ 2 }, \\
      \label{eq:Scalar-field-and-two-quasi-deltas-06-B}
      \checkMx
      &=
        \frac{ 1 }{ \sqrt{ 2 \pi } }
        \int\limits_{ -\infty }^{ +\infty } dp \, e^{ i p x } M_{ p }.
    \end{align}

  \end{subequations}

\end{frame}
% ##################




% ##################
\begin{frame}
  \frametitle{Functions used in computations}

  \vspace{-2em}


  \begin{subequations}
    \begin{align}
      \label{eq:Scalar-field-and-two-quasi-deltas-07-A}
      \chi( g )
      &=
        \frac{ 2 \pi }{ \alpha }
        \int\limits_{ -\infty }^{ +\infty } dp \, M_{ p } =
        \sqrt{ 2 \pi }^{ 3 } \frac{ \checkMzero }{ \alpha }, \\
      \label{eq:Scalar-field-and-two-quasi-deltas-07-B}
      \sigma( g )^{ -1 }
      &=
        -\alpha ( M_{ 0 } + \chi( g ) ), \\
      \label{eq:Scalar-field-and-two-quasi-deltas-07-C}
      h( x )
      &=
        \frac{ \sin( x ) }{ x }.
    \end{align}
  \end{subequations}
  Form of the $\sigma( g )$ shown above is motivated by problem of rescaling
  of our model. Parameter $\alpha$ is well known from theory of $\delta$ potentials
  and describe scattering length in such systems. This is relevant
  since when we shrink sizes of our bodies to the point, in the limit
  our problem becomes that of two $\delta$ system.

\end{frame}
% ##################





% ##################
\begin{frame}
  \frametitle{Formulas for number of particles and energy}


  We need a few more functions. We must omit few nontrivial
  steps, so please believe me that this notation make sens.

  \vspace{-2em}



  \begin{subequations}

    \begin{align}
      \label{eq:Scalar-field-and-two-quasi-deltas-08-A}
      t( i k )
      &=
        \alpha + 2\pi \int\limits_{ -\infty }^{ +\infty } dp \,
        \frac{ k^{ 2 } }{ k^{ 2 } + p^{ 2 } } M_{ p } \\
      \label{eq:Scalar-field-and-two-quasi-deltas-08-B}
      u( i k )
      &=
        2\pi^{ 2 } M_{ i k } \frac{ e^{ -ka } }{ a }.
    \end{align}

  \end{subequations}

  \vspace{-1em}



  Now, using in the intermediate steps some methods of scattering theory
  and few other tricks, we can write the number of particles and Casimir
  energy using one equation:
  \begin{equation}
    \label{eq:Scalar-field-and-two-quasi-deltas-09}
    \Pcal_{ \tau }( a ) =
    16 \Real \int\limits_{ \Rbb_{ + }^{ \, 2 } } dk \, dp \,
    \frac{ p^{ \HorSpaceOne 2 - \tau } }{ ( p + i k )^{ 2 } } M_{ \HorSpaceOne p }
    \frac{ t( i k ) + h( p a ) u( i k ) }{ t( i k )^{ 2 } - u( i k )^{ 2 } }.
  \end{equation}
  For $\tau = 0$ we get energy and $\tau = 1$ number of created quanta.

\end{frame}
% ##################





% ##################
\begin{frame}
  \frametitle{Interaction energy}

  \vspace{-1em}


  \begin{equation}
    \label{eq:Scalar-field-and-two-quasi-deltas-10}
    \Pcal_{ \tau }( a ) =
    16 \Real \int\limits_{ \Rbb_{ + }^{ \, 2 } } dk \, dp \,
    \frac{ p^{ \HorSpaceOne 2 - \tau } }{ ( p + i k )^{ 2 } } M_{ \HorSpaceOne p }
    \frac{ t( i k ) + h( p a ) u( i k ) }{ t( i k )^{ 2 } - u( i k )^{ 2 } }.
  \end{equation}
  It is can be shown that both integrals in above equation are convergent.

  We now fix our attention on the energy. By taking limit $a \nearrow +\infty$
  we identify self-energy of spherical bodies, i.e. energy of field
  clustering around two bodies. After subtracting it from full Casimir
  energy, we arrive at interaction energy:
  \begin{equation}
    \label{eq:Scalar-field-and-two-quasi-deltas-11}
    \EInt( a ) =
    16 \, \Real \int\limits_{ \Rbb_{ + }^{ \, 2 } } dp \, dk \,
    \frac{ p^{ \HorSpaceOne 2 } }{ ( p + i k )^{ 2 } } \, M_{ \HorSpaceOne p } \,
    \frac{ u( i k )^{ 2 } + h( p a ) t( i k ) u( i k ) }{
      t( i k ) [ t( i k )^{ 2 } - u( i k )^{ 2 } ] }.
  \end{equation}

\end{frame}
% ##################










% ######################################
\section{Rescaled version of the model}
% ######################################


% ##################
\begin{frame}
  \frametitle{Why rescaled model is useful?}

  \vspace{-2em}


  \begin{equation}
    \label{eq:Rescaled-version-of-the-model-01}
    \EInt( a ) =
    16 \, \Real \int\limits_{ \Rbb_{ + }^{ \, 2 } } dp \, dk \,
    \frac{ p^{ \HorSpaceOne 2 } }{ ( p + i k )^{ 2 } } \, M_{ \HorSpaceOne p } \,
    \frac{ u( i k )^{ 2 } + h( p a ) t( i k ) u( i k ) }{
      t( i k ) [ t( i k )^{ 2 } - u( i k )^{ 2 } ] }.
  \end{equation}
  Since this formula is quite to hard to understand when bodies have finite
  sizes, we analyzed limit case of rescaled model.

  Rescaled family of models is defined by replacing $g( \vecx )$ by family
  of functions
  \begin{equation}
    \label{eq:Rescaled-version-of-the-model-02}
    g_{ \lambda }( \vecx ) =
    \lambda^{ -3 } g\left( \tfrac{ \vecx }{ \lambda } \right), \qquad
    \lambda \in ( 0, 1 ].
  \end{equation}
  It should be noted that function $g_{ \lambda }( \vecx )$, $\lambda$ fixed, have the
  same properties that we demand from function $g( \vecx )$, so all
  previous results hold in this case.

\end{frame}
% ##################





% ##################
\begin{frame}
  \frametitle{Taking the limit}


  Since we are only interesting in asymptotic case, we make two assumptions,
  that greatly simplify computations:
  \begin{equation}
    \label{eq:Rescaled-version-of-the-model-03}
    L = \frac{ 2 \pi^{ 2 } }{ \alpha }, \qquad
    \lambda \in \left( 0, \tfrac{ 1 }{ 2 } \right).
  \end{equation}
  First of these assumptions is the reason why equations in the following
  part of the talk \alert{don't} contain parameter~$L$.

  We now need to introduce natural, dimensionless variable $\gamma$:
  \begin{equation}
    \label{eq:Rescaled-version-of-the-model-04}
    \gamma := \frac{ \alpha a }{ 2 \pi^{ 2 } }.
  \end{equation}
  We can prove that when
  \begin{equation}
    \label{eq:Rescaled-version-of-the-model-05}
    \gamma > 1,
  \end{equation}
  operator $h_{ a, \, \lambda }^{ 2 }$ is non negative and if this isn't true,
  there is \alert{possibility} of presence of bounded state with
  negative energy.

\end{frame}
% ##################





% ##################
\begin{frame}
  \frametitle{Taking the limit}


  When $\lambda \searrow 0$ our model converges in strong resolvent sense to scalar
  field interacting with two $\delta$-s. This model is well known in literature
  and is in great details discussed in monograph by S. Albeverio et al.
  \textit{Solvable Models in~Quantum Mechanics}
  \parencite{Albeverio-et-al-Solvable-Models-in-Quantum-Mechanics-Pub-1988}.

  In this book it is proven that if $\gamma \leq 1$ in the system of two strict
  Dirac $\delta$-s there \alert{is} at least one bound state to the negative
  energy. We weren't able to show existence of such states for $\lambda \neq 0$,
  when we have only ``smoothed $\delta$-s'', so we don't understand full
  implications of condition \eqref{eq:Rescaled-version-of-the-model-05}.
  At this moment, we don't think this particular problem is worth future
  research in the context of Casimir effect.

\end{frame}
% ##################





% ##################
\begin{frame}
  \frametitle{Asymptotic expansion of Casimir energy}

  % \vspace{-2em}


  Our main result is asymptotic expansion of interaction energy:
  \begin{equation}
    \label{eq:Rescaled-version-of-the-model-04}
    \begin{split}
      &\EInt( a, \lambda ) = \\
      &=
        \frac{ 2 \alpha }{ \pi^{ 3 } } \! \Bigg[ \frac{ \chi }{ \lambda } \!
        \int\limits_{ 0 }^{ +\infty } \!
        \frac{ e^{ -2 l } \, dl }{ ( \gamma + l )
        \big[ ( \gamma + l )^{ 2 } - e^{ -2 l } \big] } \! + \!
        \frac{ b_{ \HorSpaceOne 1 } \chi }{ \gamma } \!
        \int\limits_{ 0 }^{ +\infty } \!
        \frac{ l^{ 2 } \big[ 3 ( \gamma + l )^{ 2 } e^{ -2 l } -
        e^{ -4 l } \big] }
        { ( \gamma + l )^{ 2 }
        \big[ ( \gamma + l )^{ 2 } - e^{ -2 l } \big]^{ 2 } } dl
        \, - \\[0.5em]
      &- \! \frac{ 2 }{ \gamma } \! \int\limits_{ 0 }^{ +\infty } \!
        \frac{ l e^{ -2 l }\, dl }{ ( \gamma + l )
        \big[ ( \gamma + l )^{ 2 } - e^{ -2 l } \big] } \! + \!
        \frac{ 1 }{ \gamma } \!
        \int\limits_{ 0 }^{ +\infty } \! \frac{ ( 1 - l ) e^{ -2 l } }{
        ( \gamma + l )^{ 2 } - e^{ -2 l } } \, dl + R( a, \lambda ) \Bigg],
    \end{split}
  \end{equation}
  Function $\chi( g ) > 0$ was defined in equation
  \eqref{eq:Scalar-field-and-two-quasi-deltas-07-A}. Parameter
  $b_{ 1 }$ is given by explicit, but quite convoluted integrals of function
  $g( \vecx )$ (notice that it does not dependent on $\lambda$) and
  $0 < b_{ 1 } \leq \alpha^{ -1 }$.

  You should notice that this formula is \alert{model dependent}. Two terms
  in expansion ``remember'' how we approximated $\delta$-s interactions.

\end{frame}
% ##################





% ##################
\begin{frame}
  \frametitle{Asymptotic expansion of Casimir energy}

  \vspace{-2em}


  \begin{equation}
    \label{eq:Rescaled-version-of-the-model-05}
    \begin{split}
      &\EInt( a, \lambda ) = \\
      &=
        \frac{ 2 \alpha }{ \pi^{ 3 } } \! \Bigg[ \frac{ \chi }{ \lambda } \!
        \int\limits_{ 0 }^{ +\infty } \!
        \frac{ e^{ -2 l } \, dl }{ ( \gamma + l )
        \big[ ( \gamma + l )^{ 2 } - e^{ -2 l } \big] } \! + \!
        \frac{ b_{ \HorSpaceOne 1 } \chi }{ \gamma } \!
        \int\limits_{ 0 }^{ +\infty } \!
        \frac{ l^{ 2 } \big[ 3 ( \gamma + l )^{ 2 } e^{ -2 l } -
        e^{ -4 l } \big] }
        { ( \gamma + l )^{ 2 }
        \big[ ( \gamma + l )^{ 2 } - e^{ -2 l } \big]^{ 2 } } dl
        \, - \\[0.5em]
      &- \! \frac{ 2 }{ \gamma } \! \int\limits_{ 0 }^{ +\infty } \!
        \frac{ l e^{ -2 l }\, dl }{ ( \gamma + l )
        \big[ ( \gamma + l )^{ 2 } - e^{ -2 l } \big] } \! + \!
        \frac{ 1 }{ \gamma } \!
        \int\limits_{ 0 }^{ +\infty } \! \frac{ ( 1 - l ) e^{ -2 l } }{
        ( \gamma + l )^{ 2 } - e^{ -2 l } } \, dl + R( a, \lambda ) \Bigg],
    \end{split}
  \end{equation}
  To analyse this formula we were forced to use combination of analytical
  and numeric methods. With very high degree of confidence we established
  that it predicts repulsive force that very quickly vanishes with the
  distance. This is in striking contrast with previous works of A.
  Sccardicchio \parencite{Scardicchio-Casimir-dynamics-ETC-2005}, who used
  version of zero-point energy approach to derive for this system force
  that according to him is \alert{universal} and \alert{attractive}.

\end{frame}
% ##################





% ##################
\begin{frame}
  \frametitle{Asymptotic expansion of Casimir energy}

  \vspace{-2.5em}


  \begin{equation}
    \label{eq:Rescaled-version-of-the-model-06}
    \begin{split}
      &\EInt( a, \lambda ) = \\
      &=
        \frac{ 2 \alpha }{ \pi^{ 3 } } \! \Bigg[ \frac{ \chi }{ \lambda } \!
        \int\limits_{ 0 }^{ +\infty } \!
        \frac{ e^{ -2 l } \, dl }{ ( \gamma + l )
        \big[ ( \gamma + l )^{ 2 } - e^{ -2 l } \big] } \! + \!
        \frac{ b_{ \HorSpaceOne 1 } \chi }{ \gamma } \!
        \int\limits_{ 0 }^{ +\infty } \!
        \frac{ l^{ 2 } \big[ 3 ( \gamma + l )^{ 2 } e^{ -2 l } -
        e^{ -4 l } \big] }
        { ( \gamma + l )^{ 2 }
        \big[ ( \gamma + l )^{ 2 } - e^{ -2 l } \big]^{ 2 } } dl
        \, - \\[0.5em]
      &- \! \frac{ 2 }{ \gamma } \! \int\limits_{ 0 }^{ +\infty } \!
        \frac{ l e^{ -2 l }\, dl }{ ( \gamma + l )
        \big[ ( \gamma + l )^{ 2 } - e^{ -2 l } \big] } \! + \!
        \frac{ 1 }{ \gamma } \!
        \int\limits_{ 0 }^{ +\infty } \! \frac{ ( 1 - l ) e^{ -2 l } }{
        ( \gamma + l )^{ 2 } - e^{ -2 l } } \, dl + R( a, \lambda ) \Bigg],
    \end{split}
  \end{equation}
  The four integrals in equation above, from left two right, all taken
  with sign plus, we denoted as $I_{ 1 }$, $I_{ 2 }$, $I_{ 3 }$ and $I_{ 4 }$.
  At the next slides we plot results of numerical computations of Casimir
  energy for which we assumed:
  \begin{equation}
    \label{eq:Rescaled-version-of-the-model-07}
    \frac{ \chi }{ \lambda } = 10.0, \qquad
    \chi b_{ 1 } = 1.0.
  \end{equation}

\end{frame}
% ##################





% ##################
\begin{frame}
  \frametitle{Graph of four components of interaction energy}


  \begin{figure}

    \label{fig:Terms-of-asymptotic-expansion}

    \centering


    \includegraphics[scale=0.525]
    {./Presentation-pictures/Terms\_of\_asymptotic\_expansion\_01.png}

    \caption{Plot of four integrands from equation
      \eqref{eq:Rescaled-version-of-the-model-05}, taken with plus sign.}


  \end{figure}

\end{frame}
% ##################





% ##################
\begin{frame}
  \frametitle{Graph of Casimir interaction energy}


  \begin{figure}

    \label{fig:Asymptotic-expansion-of-Casimir-energy}

    \centering


    \includegraphics[scale=0.525]
    {./Presentation-pictures/Casimir\_energy\_asymptotic\_expansion\_01.png}

    \caption{Plot of asymptotic expansion of Casimir energy computed in our
      approach.}


  \end{figure}

\end{frame}
% ##################





% ##################
\begin{frame}
  \frametitle{Sccardicchio prediction for Casimir energy}


  \begin{figure}

    \label{fig:Casimir-energy-Scardicchio}

    \centering


    \includegraphics[scale=0.525]
    {./Presentation-pictures/Casimir\_energy\_Scardicchio\_01.png}

    \caption{Casimir energy for two delta system computed by
      A.~Sccardicchio.}


  \end{figure}

\end{frame}
% ##################





% ##################
\begin{frame}
  \frametitle{Local energy density}


  We also computed local energy density. Interaction part of it
  in the limit $\lambda \searrow 0$ is shown on the next slide (not the shortest
  formula that you can imagine). More precisely, it is a regular
  distribution on the set $\Rbb^{ 3 } \setminus \{ -\veca / 2, \veca / 2\}$.

  One should note that local density of interaction energy is model
  \alert{independent}, while global energy is heavily model depended.

\end{frame}
% ##################





% ##################
\begin{frame}
  \frametitle{Local energy density}

  \vspace{-1em}


  \begin{equation}
    \label{eq:Rescaled-version-of-the-model-08}
    \begin{split}
      \EIntDenLim
      &( \veca, \vecx \HorSpaceOne ) =
        \frac{ 1 }{ 8\pi^{ 2 } }
        \int\limits_{ 0 }^{ +\infty } dl \,
        \frac{ e^{ -2 l } }{ ( \gamma + l ) [ ( \gamma + l )^{ 2 } - e^{ -2 l } ] } \, \times
      \\[0.5em]
      &\hspace{1em}
        \times \Bigg[ \frac{ e^{ -2 l \absTwo{ \vecx + \vecb } / a } }{
        \absTwo{ \vecx + \vecb }^{ 4 } }
        \left( 1 + 2 l \frac{ \absTwo{ \vecx + \vecb } }{ a } \right) +
        \frac{ e^{ -2 l \absTwo{ \vecx - \vecb } / a } }{
        \absTwo{ \vecx - \vecb }^{ 4 } } \left( 1 +
        2 l \frac{ \absTwo{ \vecx - \vecb } }{ a } \right) \Bigg] \, -
      \\[0.5em]
      &- \frac{ 1 }{ 4 \pi^{ 2 } } \int\limits_{ 0 }^{ +\infty } dl \,
        \frac{ e^{ -l } }{ ( \gamma + l )^{ 2 } - e^{ -2 l } }
        \frac{ e^{ -l ( \absOne{ \vecx + \vecb } + |\,
        \vecx - \vecb\, | ) / a } }{ |\, \vecx + \vecb\, | \absTwo{ \vecx -
        \vecb } } \, \times \\[0.5em]
      &\hspace{3.5em}
        \times \Bigg[ \frac{ l^{ 2 } }{ a^{ 2 } } \Bigg( 1 - \frac{ (
        \vecx + \veca ) \cdot ( \vecx - \veca ) }{ |\, \vecx
        + \vecb\, | \absTwo{ \vecx - \vecb } } \Bigg) \, - \\[0.5em]
      &\hspace{5.25em}
        - \frac{ ( \vecx + \vecb \HorSpaceSix ) \cdot
        ( \vecx - \vecb \HorSpaceSix ) }{
        \absOne{ \vecx + \vecb } \HorSpaceFive
        \absOne{ \vecx - \vecb } }
        \frac{ 1 + l ( | \vecx
        + \vecb | + | \vecx - \vecb | ) / a }{ | \vecx +
        \vecb | \, \absOne{ \vecx - \vecb } } \Bigg].
    \end{split}
  \end{equation}

\end{frame}
% ##################





% ##################
\begin{frame}
  \frametitle{Local energy density}


  As a byproduct of our computation we find
  local density of self-energy of single $\delta$, that is identical to that
  found by very different methods by Davide Fermi and Livio Pizzocchero.
  Their
  results were published in series of papers, for us the most important one
  is \textit{Local Casimir Effect for~a~Scalar Field in~Presence~of a~Point
    Impurity}
  \parencite{Fermi-Pizzocchero-Local-Casimir-Effect-for-a-Scalar-ETC-2018}.
  This local energy density depends on parameter $\alpha$, as it should be,
  but outside of that is model independent. Again, our computation also
  shows that \alert{global} self-energy of single $\delta$ is divergent in model
  dependent way.

  This confirms what Herdegen and Stopa found before. Namely, that relation
  between Casimir energy and local energy density is not straightforward at
  all and computation need to be done with grate care. And that local energy
  density is more ``predicable'' than global one. We believe that for
  over-idealized, simple models, like these with Dirchlet boundary
  condition, we should abandon hope of computing global energy from local
  energy density.

\end{frame}
% ##################










% ######################################
\section{Closing remarks}
% ######################################


% ##################
\begin{frame}
  \frametitle{What is a~next step?}


  The most natural way forward now seems to be analysis of Casimir effect
  for system of scalar field and single sphere. Basic analysis of it was
  already done in our master thesis, where main stumbling block was
  difficulty in proving
  that limit $\lambda \searrow 0$ could be taken. For fixed model at least
  numerical results should be in our reach, but we shouldn't count our
  chicken before they hatch.

\end{frame}
% ##################





% ##################
\begin{frame}
  \frametitle{Closing information}


  Results for two quasi-delta system was first published in our paper
  \textit{Algebraic Approach to Casimir Force Between Two $\delta$-like
    Potentials} \parencite{Ziemian-Algebraic-Approach-ETC-2021},
  with correction of a travial mistake of sign in
  \parencite{Ziemian-Correction-to-Algebraic-Approach-ETC-2023}.

  Numerical computation was made using packages \texttt{scipy} and
  \texttt{numpy}. Code is publicly available on GitHub: \\
  \colorhref{https://github.com/KZiemian/Code-for-scientific-papers}
  {https://github.com/KZiemian/Code-for-scientific-papers} \\
  in directory \\
  \texttt{Algebraic-Approach-to-Casimir-Force-Between-\ldots} \\
  You can also ask me for it directly at \email.

\end{frame}
% ##################





% ##################
\begin{frame}
  \frametitle{The end}

  \vspace{7em}


  \begin{center}

    \Large

    Thank you. \\
    Are there any questions?

  \end{center}

\end{frame}
% ##################










% ####################################################################
% ####################################################################
% Bibliography

\printbibliography





% ############################
% End of the document

\end{document}
