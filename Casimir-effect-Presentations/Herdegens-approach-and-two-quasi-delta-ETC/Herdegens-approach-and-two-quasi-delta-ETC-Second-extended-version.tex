% ------------------------------------------------------------------------------------------------------------------
% Basic configuration of Beamera class and Jagiellonian theme
% ------------------------------------------------------------------------------------------------------------------
\RequirePackage[l2tabu, orthodox]{nag}



\ifx\PresentationStyle\notset
  \def\PresentationStyle{light}
\fi



% Options: t -- align frame text to the top
\documentclass[10pt,t]{beamer}
\mode<presentation>
\usetheme[style=\PresentationStyle]{jagiellonian}





% ------------------------------------------------------------------------------------
% Procesing configuration files of Jagiellonian theme located in
% the directory "preambule"
% ------------------------------------------------------------------------------------
% Configuration for polish language
% Need description
\usepackage[english]{babel}





% % ------------------------------
% % Better support of polish chars in technical parts of PDF
% % ------------------------------
% \hypersetup{pdfencoding=auto,psdextra}

% Package "textpos" give as enviroment "textblock" which is very usefull in
% arranging text on slides.

% This is standard configuration of "textpos"
\usepackage[overlay,absolute]{textpos}

% If you need to see bounds of "textblock's" comment line above and uncomment
% one below.

% Caution! When showboxes option is on significant ammunt of space is add
% to the top of textblock and as such, everyting put in them gone down.
% We need to check how to remove this bug.

% \usepackage[showboxes,overlay,absolute]{textpos}



% Setting scale length for package "textpos"
\setlength{\TPHorizModule}{10mm}
\setlength{\TPVertModule}{\TPHorizModule}


% ---------------------------------------
% TikZ
% ---------------------------------------
% Importing TikZ libraries
\usetikzlibrary{arrows.meta}
\usetikzlibrary{positioning}





% % Configuration package "bm" that need for making bold symbols
% \newcommand{\bmmax}{0}
% \newcommand{\hmmax}{0}
% \usepackage{bm}




% ---------------------------------------
% Packages for scientific texts
% ---------------------------------------
% \let\lll\undefined  % Sometimes you must use this line to allow
% "amsmath" package to works with packages with packages for polish
% languge imported
% /preambul/LanguageSettings/JagiellonianPolishLanguageSettings.tex.
% This comments (probably) removes polish letter Ł.
\usepackage{amsmath}  % Packages from American Mathematical Society (AMS)
\usepackage{amssymb}
\usepackage{amscd}
\usepackage{amsthm}
\usepackage{siunitx}  % Package for typsetting SI units.
\usepackage{upgreek}  % Better looking greek letters.
% Example of using upgreek: pi = \uppi


\usepackage{calrsfs}  % Zmienia czcionkę kaligraficzną w \mathcal
% na ładniejszą. Może w innych miejscach robi to samo, ale o tym nic
% nie wiem.










% ---------------------------------------
% Packages written for lectures "Geometria 3D dla twórców gier wideo"
% ---------------------------------------
% \usepackage{./ProgramowanieSymulacjiFizykiPaczki/ProgramowanieSymulacjiFizyki}
% \usepackage{./ProgramowanieSymulacjiFizykiPaczki/ProgramowanieSymulacjiFizykiIndeksy}
% \usepackage{./ProgramowanieSymulacjiFizykiPaczki/ProgramowanieSymulacjiFizykiTikZStyle}





% !!!!!!!!!!!!!!!!!!!!!!!!!!!!!!
% !!!!!!!!!!!!!!!!!!!!!!!!!!!!!!
% EVIL STUFF
\if\JUlogotitle1
\edef\LogoJUPath{LogoJU_\JUlogoLang/LogoJU_\JUlogoShape_\JUlogoColor.pdf}
\titlegraphic{\hfill\includegraphics[scale=0.22]
{./JagiellonianPictures/\LogoJUPath}}
\fi
% ---------------------------------------
% Commands for handling colors
% ---------------------------------------


% Command for setting normal text color for some text in math modestyle
% Text color depend on used style of Jagiellonian

% Beamer version of command
\newcommand{\TextWithNormalTextColor}[1]{%
  {\color{jNormalTextFGColor}
    \setbeamercolor{math text}{fg=jNormalTextFGColor} {#1}}
}

% Article and similar classes version of command
% \newcommand{\TextWithNormalTextColor}[1]{%
%   {\color{jNormalTextsFGColor} {#1}}
% }



% Beamer version of command
\newcommand{\NormalTextInMathMode}[1]{%
  {\color{jNormalTextFGColor}
    \setbeamercolor{math text}{fg=jNormalTextFGColor} \text{#1}}
}


% Article and similar classes version of command
% \newcommand{\NormalTextInMathMode}[1]{%
%   {\color{jNormalTextsFGColor} \text{#1}}
% }




% Command that sets color of some mathematical text to the same color
% that has normal text in header (?)

% Beamer version of the command
\newcommand{\MathTextFrametitleFGColor}[1]{%
  {\color{jFrametitleFGColor}
    \setbeamercolor{math text}{fg=jFrametitleFGColor} #1}
}

% Article and similar classes version of the command
% \newcommand{\MathTextWhiteColor}[1]{{\color{jFrametitleFGColor} #1}}





% Command for setting color of alert text for some text in math modestyle

% Beamer version of the command
\newcommand{\MathTextAlertColor}[1]{%
  {\color{jOrange} \setbeamercolor{math text}{fg=jOrange} #1}
}

% Article and similar classes version of the command
% \newcommand{\MathTextAlertColor}[1]{{\color{jOrange} #1}}





% Command that allow you to sets chosen color as the color of some text into
% math mode. Due to some nuances in the way that Beamer handle colors
% it not work in all cases. We hope that in the future we will improve it.

% Beamer version of the command
\newcommand{\SetMathTextColor}[2]{%
  {\color{#1} \setbeamercolor{math text}{fg=#1} #2}
}


% Article and similar classes version of the command
% \newcommand{\SetMathTextColor}[2]{{\color{#1} #2}}










% ---------------------------------------
% Commands for few special slides
% ---------------------------------------
\newcommand{\EndingSlide}[1]{%
  \begin{frame}[standout]

    \begingroup

    \color{jFrametitleFGColor}

    #1

    \endgroup

  \end{frame}
}










% ---------------------------------------
% Commands for setting background pictures for some slides
% ---------------------------------------
\newcommand{\TitleBackgroundPicture}
{./JagiellonianPictures/Backgrounds/LajkonikDark.png}
\newcommand{\SectionBackgroundPicture}
{./JagiellonianPictures/Backgrounds/LajkonikLight.png}



\newcommand{\TitleSlideWithPicture}{%
  \begingroup

  \usebackgroundtemplate{%
    \includegraphics[height=\paperheight]{\TitleBackgroundPicture}}

  \maketitle

  \endgroup
}





\newcommand{\SectionSlideWithPicture}[1]{%
  \begingroup

  \usebackgroundtemplate{%
    \includegraphics[height=\paperheight]{\SectionBackgroundPicture}}

  \setbeamercolor{titlelike}{fg=normal text.fg}

  \section{#1}

  \endgroup
}










% ---------------------------------------
% Commands for lectures "Geometria 3D dla twórców gier wideo"
% Polish version
% ---------------------------------------
% Komendy teraz wykomentowane były potrzebne, gdy loga były na niebieskim
% tle, nie na białym. A są na białym bo tego chcieli w biurze projektu.
% \newcommand{\FundingLogoWhitePicturePL}
% {./PresentationPictures/CommonPictures/logotypFundusze_biale_bez_tla2.pdf}
\newcommand{\FundingLogoColorPicturePL}
{./PresentationPictures/CommonPictures/European_Funds_color_PL.pdf}
% \newcommand{\EULogoWhitePicturePL}
% {./PresentationPictures/CommonPictures/logotypUE_biale_bez_tla2.pdf}
\newcommand{\EUSocialFundLogoColorPicturePL}
{./PresentationPictures/CommonPictures/EU_Social_Fund_color_PL.pdf}
% \newcommand{\ZintegrUJLogoWhitePicturePL}
% {./PresentationPictures/CommonPictures/zintegruj-logo-white.pdf}
\newcommand{\ZintegrUJLogoColorPicturePL}
{./PresentationPictures/CommonPictures/ZintegrUJ_color.pdf}
\newcommand{\JULogoColorPicturePL}
{./JagiellonianPictures/LogoJU_PL/LogoJU_A_color.pdf}





\newcommand{\GeometryThreeDSpecialBeginningSlidePL}{%
  \begin{frame}[standout]

    \begin{textblock}{11}(1,0.7)

      \begin{flushleft}

        \mdseries

        \footnotesize

        \color{jFrametitleFGColor}

        Materiał powstał w ramach projektu współfinansowanego ze środków
        Unii Europejskiej w ramach Europejskiego Funduszu Społecznego
        POWR.03.05.00-00-Z309/17-00.

      \end{flushleft}

    \end{textblock}





    \begin{textblock}{10}(0,2.2)

      \tikz \fill[color=jBackgroundStyleLight] (0,0) rectangle (12.8,-1.5);

    \end{textblock}


    \begin{textblock}{3.2}(1,2.45)

      \includegraphics[scale=0.3]{\FundingLogoColorPicturePL}

    \end{textblock}


    \begin{textblock}{2.5}(3.7,2.5)

      \includegraphics[scale=0.2]{\JULogoColorPicturePL}

    \end{textblock}


    \begin{textblock}{2.5}(6,2.4)

      \includegraphics[scale=0.1]{\ZintegrUJLogoColorPicturePL}

    \end{textblock}


    \begin{textblock}{4.2}(8.4,2.6)

      \includegraphics[scale=0.3]{\EUSocialFundLogoColorPicturePL}

    \end{textblock}

  \end{frame}
}



\newcommand{\GeometryThreeDTwoSpecialBeginningSlidesPL}{%
  \begin{frame}[standout]

    \begin{textblock}{11}(1,0.7)

      \begin{flushleft}

        \mdseries

        \footnotesize

        \color{jFrametitleFGColor}

        Materiał powstał w ramach projektu współfinansowanego ze środków
        Unii Europejskiej w ramach Europejskiego Funduszu Społecznego
        POWR.03.05.00-00-Z309/17-00.

      \end{flushleft}

    \end{textblock}





    \begin{textblock}{10}(0,2.2)

      \tikz \fill[color=jBackgroundStyleLight] (0,0) rectangle (12.8,-1.5);

    \end{textblock}


    \begin{textblock}{3.2}(1,2.45)

      \includegraphics[scale=0.3]{\FundingLogoColorPicturePL}

    \end{textblock}


    \begin{textblock}{2.5}(3.7,2.5)

      \includegraphics[scale=0.2]{\JULogoColorPicturePL}

    \end{textblock}


    \begin{textblock}{2.5}(6,2.4)

      \includegraphics[scale=0.1]{\ZintegrUJLogoColorPicturePL}

    \end{textblock}


    \begin{textblock}{4.2}(8.4,2.6)

      \includegraphics[scale=0.3]{\EUSocialFundLogoColorPicturePL}

    \end{textblock}

  \end{frame}





  \TitleSlideWithPicture
}



\newcommand{\GeometryThreeDSpecialEndingSlidePL}{%
  \begin{frame}[standout]

    \begin{textblock}{11}(1,0.7)

      \begin{flushleft}

        \mdseries

        \footnotesize

        \color{jFrametitleFGColor}

        Materiał powstał w ramach projektu współfinansowanego ze środków
        Unii Europejskiej w~ramach Europejskiego Funduszu Społecznego
        POWR.03.05.00-00-Z309/17-00.

      \end{flushleft}

    \end{textblock}





    \begin{textblock}{10}(0,2.2)

      \tikz \fill[color=jBackgroundStyleLight] (0,0) rectangle (12.8,-1.5);

    \end{textblock}


    \begin{textblock}{3.2}(1,2.45)

      \includegraphics[scale=0.3]{\FundingLogoColorPicturePL}

    \end{textblock}


    \begin{textblock}{2.5}(3.7,2.5)

      \includegraphics[scale=0.2]{\JULogoColorPicturePL}

    \end{textblock}


    \begin{textblock}{2.5}(6,2.4)

      \includegraphics[scale=0.1]{\ZintegrUJLogoColorPicturePL}

    \end{textblock}


    \begin{textblock}{4.2}(8.4,2.6)

      \includegraphics[scale=0.3]{\EUSocialFundLogoColorPicturePL}

    \end{textblock}





    \begin{textblock}{11}(1,4)

      \begin{flushleft}

        \mdseries

        \footnotesize

        \RaggedRight

        \color{jFrametitleFGColor}

        Treść niniejszego wykładu jest udostępniona na~licencji
        Creative Commons (\textsc{cc}), z~uzna\-niem autorstwa
        (\textsc{by}) oraz udostępnianiem na tych samych warunkach
        (\textsc{sa}). Rysunki i~wy\-kresy zawarte w~wykładzie są
        autorstwa dr.~hab.~Pawła Węgrzyna et~al. i~są dostępne
        na tej samej licencji, o~ile nie wskazano inaczej.
        W~prezentacji wykorzystano temat Beamera Jagiellonian,
        oparty na~temacie Metropolis Matthiasa Vogelgesanga,
        dostępnym na licencji \LaTeX{} Project Public License~1.3c
        pod adresem: \colorhref{https://github.com/matze/mtheme}
        {https://github.com/matze/mtheme}.

        Projekt typograficzny: Iwona Grabska-Gradzińska \\
        Skład: Kamil Ziemian;
        Korekta: Wojciech Palacz \\
        Modele: Dariusz Frymus, Kamil Nowakowski \\
        Rysunki i~wykresy: Kamil Ziemian, Paweł Węgrzyn, Wojciech Palacz

      \end{flushleft}

    \end{textblock}

  \end{frame}
}



\newcommand{\GeometryThreeDTwoSpecialEndingSlidesPL}[1]{%
  \begin{frame}[standout]


    \begin{textblock}{11}(1,0.7)

      \begin{flushleft}

        \mdseries

        \footnotesize

        \color{jFrametitleFGColor}

        Materiał powstał w ramach projektu współfinansowanego ze środków
        Unii Europejskiej w~ramach Europejskiego Funduszu Społecznego
        POWR.03.05.00-00-Z309/17-00.

      \end{flushleft}

    \end{textblock}





    \begin{textblock}{10}(0,2.2)

      \tikz \fill[color=jBackgroundStyleLight] (0,0) rectangle (12.8,-1.5);

    \end{textblock}


    \begin{textblock}{3.2}(1,2.45)

      \includegraphics[scale=0.3]{\FundingLogoColorPicturePL}

    \end{textblock}


    \begin{textblock}{2.5}(3.7,2.5)

      \includegraphics[scale=0.2]{\JULogoColorPicturePL}

    \end{textblock}


    \begin{textblock}{2.5}(6,2.4)

      \includegraphics[scale=0.1]{\ZintegrUJLogoColorPicturePL}

    \end{textblock}


    \begin{textblock}{4.2}(8.4,2.6)

      \includegraphics[scale=0.3]{\EUSocialFundLogoColorPicturePL}

    \end{textblock}





    \begin{textblock}{11}(1,4)

      \begin{flushleft}

        \mdseries

        \footnotesize

        \RaggedRight

        \color{jFrametitleFGColor}

        Treść niniejszego wykładu jest udostępniona na~licencji
        Creative Commons (\textsc{cc}), z~uzna\-niem autorstwa
        (\textsc{by}) oraz udostępnianiem na tych samych warunkach
        (\textsc{sa}). Rysunki i~wy\-kresy zawarte w~wykładzie są
        autorstwa dr.~hab.~Pawła Węgrzyna et~al. i~są dostępne
        na tej samej licencji, o~ile nie wskazano inaczej.
        W~prezentacji wykorzystano temat Beamera Jagiellonian,
        oparty na~temacie Metropolis Matthiasa Vogelgesanga,
        dostępnym na licencji \LaTeX{} Project Public License~1.3c
        pod adresem: \colorhref{https://github.com/matze/mtheme}
        {https://github.com/matze/mtheme}.

        Projekt typograficzny: Iwona Grabska-Gradzińska \\
        Skład: Kamil Ziemian;
        Korekta: Wojciech Palacz \\
        Modele: Dariusz Frymus, Kamil Nowakowski \\
        Rysunki i~wykresy: Kamil Ziemian, Paweł Węgrzyn, Wojciech Palacz

      \end{flushleft}

    \end{textblock}

  \end{frame}





  \begin{frame}[standout]

    \begingroup

    \color{jFrametitleFGColor}

    #1

    \endgroup

  \end{frame}
}



\newcommand{\GeometryThreeDSpecialEndingSlideVideoPL}{%
  \begin{frame}[standout]

    \begin{textblock}{11}(1,0.7)

      \begin{flushleft}

        \mdseries

        \footnotesize

        \color{jFrametitleFGColor}

        Materiał powstał w ramach projektu współfinansowanego ze środków
        Unii Europejskiej w~ramach Europejskiego Funduszu Społecznego
        POWR.03.05.00-00-Z309/17-00.

      \end{flushleft}

    \end{textblock}





    \begin{textblock}{10}(0,2.2)

      \tikz \fill[color=jBackgroundStyleLight] (0,0) rectangle (12.8,-1.5);

    \end{textblock}


    \begin{textblock}{3.2}(1,2.45)

      \includegraphics[scale=0.3]{\FundingLogoColorPicturePL}

    \end{textblock}


    \begin{textblock}{2.5}(3.7,2.5)

      \includegraphics[scale=0.2]{\JULogoColorPicturePL}

    \end{textblock}


    \begin{textblock}{2.5}(6,2.4)

      \includegraphics[scale=0.1]{\ZintegrUJLogoColorPicturePL}

    \end{textblock}


    \begin{textblock}{4.2}(8.4,2.6)

      \includegraphics[scale=0.3]{\EUSocialFundLogoColorPicturePL}

    \end{textblock}





    \begin{textblock}{11}(1,4)

      \begin{flushleft}

        \mdseries

        \footnotesize

        \RaggedRight

        \color{jFrametitleFGColor}

        Treść niniejszego wykładu jest udostępniona na~licencji
        Creative Commons (\textsc{cc}), z~uzna\-niem autorstwa
        (\textsc{by}) oraz udostępnianiem na tych samych warunkach
        (\textsc{sa}). Rysunki i~wy\-kresy zawarte w~wykładzie są
        autorstwa dr.~hab.~Pawła Węgrzyna et~al. i~są dostępne
        na tej samej licencji, o~ile nie wskazano inaczej.
        W~prezentacji wykorzystano temat Beamera Jagiellonian,
        oparty na~temacie Metropolis Matthiasa Vogelgesanga,
        dostępnym na licencji \LaTeX{} Project Public License~1.3c
        pod adresem: \colorhref{https://github.com/matze/mtheme}
        {https://github.com/matze/mtheme}.

        Projekt typograficzny: Iwona Grabska-Gradzińska;
        Skład: Kamil Ziemian \\
        Korekta: Wojciech Palacz;
        Modele: Dariusz Frymus, Kamil Nowakowski \\
        Rysunki i~wykresy: Kamil Ziemian, Paweł Węgrzyn, Wojciech Palacz \\
        Montaż: Agencja Filmowa Film \& Television Production~-- Zbigniew
        Masklak

      \end{flushleft}

    \end{textblock}

  \end{frame}
}





\newcommand{\GeometryThreeDTwoSpecialEndingSlidesVideoPL}[1]{%
  \begin{frame}[standout]

    \begin{textblock}{11}(1,0.7)

      \begin{flushleft}

        \mdseries

        \footnotesize

        \color{jFrametitleFGColor}

        Materiał powstał w ramach projektu współfinansowanego ze środków
        Unii Europejskiej w~ramach Europejskiego Funduszu Społecznego
        POWR.03.05.00-00-Z309/17-00.

      \end{flushleft}

    \end{textblock}





    \begin{textblock}{10}(0,2.2)

      \tikz \fill[color=jBackgroundStyleLight] (0,0) rectangle (12.8,-1.5);

    \end{textblock}


    \begin{textblock}{3.2}(1,2.45)

      \includegraphics[scale=0.3]{\FundingLogoColorPicturePL}

    \end{textblock}


    \begin{textblock}{2.5}(3.7,2.5)

      \includegraphics[scale=0.2]{\JULogoColorPicturePL}

    \end{textblock}


    \begin{textblock}{2.5}(6,2.4)

      \includegraphics[scale=0.1]{\ZintegrUJLogoColorPicturePL}

    \end{textblock}


    \begin{textblock}{4.2}(8.4,2.6)

      \includegraphics[scale=0.3]{\EUSocialFundLogoColorPicturePL}

    \end{textblock}





    \begin{textblock}{11}(1,4)

      \begin{flushleft}

        \mdseries

        \footnotesize

        \RaggedRight

        \color{jFrametitleFGColor}

        Treść niniejszego wykładu jest udostępniona na~licencji
        Creative Commons (\textsc{cc}), z~uzna\-niem autorstwa
        (\textsc{by}) oraz udostępnianiem na tych samych warunkach
        (\textsc{sa}). Rysunki i~wy\-kresy zawarte w~wykładzie są
        autorstwa dr.~hab.~Pawła Węgrzyna et~al. i~są dostępne
        na tej samej licencji, o~ile nie wskazano inaczej.
        W~prezentacji wykorzystano temat Beamera Jagiellonian,
        oparty na~temacie Metropolis Matthiasa Vogelgesanga,
        dostępnym na licencji \LaTeX{} Project Public License~1.3c
        pod adresem: \colorhref{https://github.com/matze/mtheme}
        {https://github.com/matze/mtheme}.

        Projekt typograficzny: Iwona Grabska-Gradzińska;
        Skład: Kamil Ziemian \\
        Korekta: Wojciech Palacz;
        Modele: Dariusz Frymus, Kamil Nowakowski \\
        Rysunki i~wykresy: Kamil Ziemian, Paweł Węgrzyn, Wojciech Palacz \\
        Montaż: Agencja Filmowa Film \& Television Production~-- Zbigniew
        Masklak

      \end{flushleft}

    \end{textblock}

  \end{frame}





  \begin{frame}[standout]


    \begingroup

    \color{jFrametitleFGColor}

    #1

    \endgroup

  \end{frame}
}










% ---------------------------------------
% Commands for lectures "Geometria 3D dla twórców gier wideo"
% English version
% ---------------------------------------
% \newcommand{\FundingLogoWhitePictureEN}
% {./PresentationPictures/CommonPictures/logotypFundusze_biale_bez_tla2.pdf}
\newcommand{\FundingLogoColorPictureEN}
{./PresentationPictures/CommonPictures/European_Funds_color_EN.pdf}
% \newcommand{\EULogoWhitePictureEN}
% {./PresentationPictures/CommonPictures/logotypUE_biale_bez_tla2.pdf}
\newcommand{\EUSocialFundLogoColorPictureEN}
{./PresentationPictures/CommonPictures/EU_Social_Fund_color_EN.pdf}
% \newcommand{\ZintegrUJLogoWhitePictureEN}
% {./PresentationPictures/CommonPictures/zintegruj-logo-white.pdf}
\newcommand{\ZintegrUJLogoColorPictureEN}
{./PresentationPictures/CommonPictures/ZintegrUJ_color.pdf}
\newcommand{\JULogoColorPictureEN}
{./JagiellonianPictures/LogoJU_EN/LogoJU_A_color.pdf}



\newcommand{\GeometryThreeDSpecialBeginningSlideEN}{%
  \begin{frame}[standout]

    \begin{textblock}{11}(1,0.7)

      \begin{flushleft}

        \mdseries

        \footnotesize

        \color{jFrametitleFGColor}

        This content was created as part of a project co-financed by the
        European Union within the framework of the European Social Fund
        POWR.03.05.00-00-Z309/17-00.

      \end{flushleft}

    \end{textblock}





    \begin{textblock}{10}(0,2.2)

      \tikz \fill[color=jBackgroundStyleLight] (0,0) rectangle (12.8,-1.5);

    \end{textblock}


    \begin{textblock}{3.2}(0.7,2.45)

      \includegraphics[scale=0.3]{\FundingLogoColorPictureEN}

    \end{textblock}


    \begin{textblock}{2.5}(4.15,2.5)

      \includegraphics[scale=0.2]{\JULogoColorPictureEN}

    \end{textblock}


    \begin{textblock}{2.5}(6.35,2.4)

      \includegraphics[scale=0.1]{\ZintegrUJLogoColorPictureEN}

    \end{textblock}


    \begin{textblock}{4.2}(8.4,2.6)

      \includegraphics[scale=0.3]{\EUSocialFundLogoColorPictureEN}

    \end{textblock}

  \end{frame}
}



\newcommand{\GeometryThreeDTwoSpecialBeginningSlidesEN}{%
  \begin{frame}[standout]

    \begin{textblock}{11}(1,0.7)

      \begin{flushleft}

        \mdseries

        \footnotesize

        \color{jFrametitleFGColor}

        This content was created as part of a project co-financed by the
        European Union within the framework of the European Social Fund
        POWR.03.05.00-00-Z309/17-00.

      \end{flushleft}

    \end{textblock}





    \begin{textblock}{10}(0,2.2)

      \tikz \fill[color=jBackgroundStyleLight] (0,0) rectangle (12.8,-1.5);

    \end{textblock}


    \begin{textblock}{3.2}(0.7,2.45)

      \includegraphics[scale=0.3]{\FundingLogoColorPictureEN}

    \end{textblock}


    \begin{textblock}{2.5}(4.15,2.5)

      \includegraphics[scale=0.2]{\JULogoColorPictureEN}

    \end{textblock}


    \begin{textblock}{2.5}(6.35,2.4)

      \includegraphics[scale=0.1]{\ZintegrUJLogoColorPictureEN}

    \end{textblock}


    \begin{textblock}{4.2}(8.4,2.6)

      \includegraphics[scale=0.3]{\EUSocialFundLogoColorPictureEN}

    \end{textblock}

  \end{frame}





  \TitleSlideWithPicture
}



\newcommand{\GeometryThreeDSpecialEndingSlideEN}{%
  \begin{frame}[standout]

    \begin{textblock}{11}(1,0.7)

      \begin{flushleft}

        \mdseries

        \footnotesize

        \color{jFrametitleFGColor}

        This content was created as part of a project co-financed by the
        European Union within the framework of the European Social Fund
        POWR.03.05.00-00-Z309/17-00.

      \end{flushleft}

    \end{textblock}





    \begin{textblock}{10}(0,2.2)

      \tikz \fill[color=jBackgroundStyleLight] (0,0) rectangle (12.8,-1.5);

    \end{textblock}


    \begin{textblock}{3.2}(0.7,2.45)

      \includegraphics[scale=0.3]{\FundingLogoColorPictureEN}

    \end{textblock}


    \begin{textblock}{2.5}(4.15,2.5)

      \includegraphics[scale=0.2]{\JULogoColorPictureEN}

    \end{textblock}


    \begin{textblock}{2.5}(6.35,2.4)

      \includegraphics[scale=0.1]{\ZintegrUJLogoColorPictureEN}

    \end{textblock}


    \begin{textblock}{4.2}(8.4,2.6)

      \includegraphics[scale=0.3]{\EUSocialFundLogoColorPictureEN}

    \end{textblock}





    \begin{textblock}{11}(1,4)

      \begin{flushleft}

        \mdseries

        \footnotesize

        \RaggedRight

        \color{jFrametitleFGColor}

        The content of this lecture is made available under a~Creative
        Commons licence (\textsc{cc}), giving the author the credits
        (\textsc{by}) and putting an obligation to share on the same terms
        (\textsc{sa}). Figures and diagrams included in the lecture are
        authored by Paweł Węgrzyn et~al., and are available under the same
        license unless indicated otherwise.\\ The presentation uses the
        Beamer Jagiellonian theme based on Matthias Vogelgesang’s
        Metropolis theme, available under license \LaTeX{} Project
        Public License~1.3c at: \colorhref{https://github.com/matze/mtheme}
        {https://github.com/matze/mtheme}.

        Typographic design: Iwona Grabska-Gradzińska \\
        \LaTeX{} Typesetting: Kamil Ziemian \\
        Proofreading: Wojciech Palacz,
        Monika Stawicka \\
        3D Models: Dariusz Frymus, Kamil Nowakowski \\
        Figures and charts: Kamil Ziemian, Paweł Węgrzyn, Wojciech Palacz

      \end{flushleft}

    \end{textblock}

  \end{frame}
}



\newcommand{\GeometryThreeDTwoSpecialEndingSlidesEN}[1]{%
  \begin{frame}[standout]


    \begin{textblock}{11}(1,0.7)

      \begin{flushleft}

        \mdseries

        \footnotesize

        \color{jFrametitleFGColor}

        This content was created as part of a project co-financed by the
        European Union within the framework of the European Social Fund
        POWR.03.05.00-00-Z309/17-00.

      \end{flushleft}

    \end{textblock}





    \begin{textblock}{10}(0,2.2)

      \tikz \fill[color=jBackgroundStyleLight] (0,0) rectangle (12.8,-1.5);

    \end{textblock}


    \begin{textblock}{3.2}(0.7,2.45)

      \includegraphics[scale=0.3]{\FundingLogoColorPictureEN}

    \end{textblock}


    \begin{textblock}{2.5}(4.15,2.5)

      \includegraphics[scale=0.2]{\JULogoColorPictureEN}

    \end{textblock}


    \begin{textblock}{2.5}(6.35,2.4)

      \includegraphics[scale=0.1]{\ZintegrUJLogoColorPictureEN}

    \end{textblock}


    \begin{textblock}{4.2}(8.4,2.6)

      \includegraphics[scale=0.3]{\EUSocialFundLogoColorPictureEN}

    \end{textblock}





    \begin{textblock}{11}(1,4)

      \begin{flushleft}

        \mdseries

        \footnotesize

        \RaggedRight

        \color{jFrametitleFGColor}

        The content of this lecture is made available under a~Creative
        Commons licence (\textsc{cc}), giving the author the credits
        (\textsc{by}) and putting an obligation to share on the same terms
        (\textsc{sa}). Figures and diagrams included in the lecture are
        authored by Paweł Węgrzyn et~al., and are available under the same
        license unless indicated otherwise.\\ The presentation uses the
        Beamer Jagiellonian theme based on Matthias Vogelgesang’s
        Metropolis theme, available under license \LaTeX{} Project
        Public License~1.3c at: \colorhref{https://github.com/matze/mtheme}
        {https://github.com/matze/mtheme}.

        Typographic design: Iwona Grabska-Gradzińska \\
        \LaTeX{} Typesetting: Kamil Ziemian \\
        Proofreading: Wojciech Palacz,
        Monika Stawicka \\
        3D Models: Dariusz Frymus, Kamil Nowakowski \\
        Figures and charts: Kamil Ziemian, Paweł Węgrzyn, Wojciech Palacz

      \end{flushleft}

    \end{textblock}

  \end{frame}





  \begin{frame}[standout]

    \begingroup

    \color{jFrametitleFGColor}

    #1

    \endgroup

  \end{frame}
}



\newcommand{\GeometryThreeDSpecialEndingSlideVideoVerOneEN}{%
  \begin{frame}[standout]

    \begin{textblock}{11}(1,0.7)

      \begin{flushleft}

        \mdseries

        \footnotesize

        \color{jFrametitleFGColor}

        This content was created as part of a project co-financed by the
        European Union within the framework of the European Social Fund
        POWR.03.05.00-00-Z309/17-00.

      \end{flushleft}

    \end{textblock}





    \begin{textblock}{10}(0,2.2)

      \tikz \fill[color=jBackgroundStyleLight] (0,0) rectangle (12.8,-1.5);

    \end{textblock}


    \begin{textblock}{3.2}(0.7,2.45)

      \includegraphics[scale=0.3]{\FundingLogoColorPictureEN}

    \end{textblock}


    \begin{textblock}{2.5}(4.15,2.5)

      \includegraphics[scale=0.2]{\JULogoColorPictureEN}

    \end{textblock}


    \begin{textblock}{2.5}(6.35,2.4)

      \includegraphics[scale=0.1]{\ZintegrUJLogoColorPictureEN}

    \end{textblock}


    \begin{textblock}{4.2}(8.4,2.6)

      \includegraphics[scale=0.3]{\EUSocialFundLogoColorPictureEN}

    \end{textblock}





    \begin{textblock}{11}(1,4)

      \begin{flushleft}

        \mdseries

        \footnotesize

        \RaggedRight

        \color{jFrametitleFGColor}

        The content of this lecture is made available under a Creative
        Commons licence (\textsc{cc}), giving the author the credits
        (\textsc{by}) and putting an obligation to share on the same terms
        (\textsc{sa}). Figures and diagrams included in the lecture are
        authored by Paweł Węgrzyn et~al., and are available under the same
        license unless indicated otherwise.\\ The presentation uses the
        Beamer Jagiellonian theme based on Matthias Vogelgesang’s
        Metropolis theme, available under license \LaTeX{} Project
        Public License~1.3c at: \colorhref{https://github.com/matze/mtheme}
        {https://github.com/matze/mtheme}.

        Typographic design: Iwona Grabska-Gradzińska;
        \LaTeX{} Typesetting: Kamil Ziemian \\
        Proofreading: Wojciech Palacz,
        Monika Stawicka \\
        3D Models: Dariusz Frymus, Kamil Nowakowski \\
        Figures and charts: Kamil Ziemian, Paweł Węgrzyn, Wojciech
        Palacz \\
        Film editing: Agencja Filmowa Film \& Television Production~--
        Zbigniew Masklak

      \end{flushleft}

    \end{textblock}

  \end{frame}
}



\newcommand{\GeometryThreeDSpecialEndingSlideVideoVerTwoEN}{%
  \begin{frame}[standout]

    \begin{textblock}{11}(1,0.7)

      \begin{flushleft}

        \mdseries

        \footnotesize

        \color{jFrametitleFGColor}

        This content was created as part of a project co-financed by the
        European Union within the framework of the European Social Fund
        POWR.03.05.00-00-Z309/17-00.

      \end{flushleft}

    \end{textblock}





    \begin{textblock}{10}(0,2.2)

      \tikz \fill[color=jBackgroundStyleLight] (0,0) rectangle (12.8,-1.5);

    \end{textblock}


    \begin{textblock}{3.2}(0.7,2.45)

      \includegraphics[scale=0.3]{\FundingLogoColorPictureEN}

    \end{textblock}


    \begin{textblock}{2.5}(4.15,2.5)

      \includegraphics[scale=0.2]{\JULogoColorPictureEN}

    \end{textblock}


    \begin{textblock}{2.5}(6.35,2.4)

      \includegraphics[scale=0.1]{\ZintegrUJLogoColorPictureEN}

    \end{textblock}


    \begin{textblock}{4.2}(8.4,2.6)

      \includegraphics[scale=0.3]{\EUSocialFundLogoColorPictureEN}

    \end{textblock}





    \begin{textblock}{11}(1,4)

      \begin{flushleft}

        \mdseries

        \footnotesize

        \RaggedRight

        \color{jFrametitleFGColor}

        The content of this lecture is made available under a Creative
        Commons licence (\textsc{cc}), giving the author the credits
        (\textsc{by}) and putting an obligation to share on the same terms
        (\textsc{sa}). Figures and diagrams included in the lecture are
        authored by Paweł Węgrzyn et~al., and are available under the same
        license unless indicated otherwise.\\ The presentation uses the
        Beamer Jagiellonian theme based on Matthias Vogelgesang’s
        Metropolis theme, available under license \LaTeX{} Project
        Public License~1.3c at: \colorhref{https://github.com/matze/mtheme}
        {https://github.com/matze/mtheme}.

        Typographic design: Iwona Grabska-Gradzińska;
        \LaTeX{} Typesetting: Kamil Ziemian \\
        Proofreading: Wojciech Palacz,
        Monika Stawicka \\
        3D Models: Dariusz Frymus, Kamil Nowakowski \\
        Figures and charts: Kamil Ziemian, Paweł Węgrzyn, Wojciech
        Palacz \\
        Film editing: IMAVI -- Joanna Kozakiewicz, Krzysztof Magda, Nikodem
        Frodyma

      \end{flushleft}

    \end{textblock}

  \end{frame}
}



\newcommand{\GeometryThreeDSpecialEndingSlideVideoVerThreeEN}{%
  \begin{frame}[standout]

    \begin{textblock}{11}(1,0.7)

      \begin{flushleft}

        \mdseries

        \footnotesize

        \color{jFrametitleFGColor}

        This content was created as part of a project co-financed by the
        European Union within the framework of the European Social Fund
        POWR.03.05.00-00-Z309/17-00.

      \end{flushleft}

    \end{textblock}





    \begin{textblock}{10}(0,2.2)

      \tikz \fill[color=jBackgroundStyleLight] (0,0) rectangle (12.8,-1.5);

    \end{textblock}


    \begin{textblock}{3.2}(0.7,2.45)

      \includegraphics[scale=0.3]{\FundingLogoColorPictureEN}

    \end{textblock}


    \begin{textblock}{2.5}(4.15,2.5)

      \includegraphics[scale=0.2]{\JULogoColorPictureEN}

    \end{textblock}


    \begin{textblock}{2.5}(6.35,2.4)

      \includegraphics[scale=0.1]{\ZintegrUJLogoColorPictureEN}

    \end{textblock}


    \begin{textblock}{4.2}(8.4,2.6)

      \includegraphics[scale=0.3]{\EUSocialFundLogoColorPictureEN}

    \end{textblock}





    \begin{textblock}{11}(1,4)

      \begin{flushleft}

        \mdseries

        \footnotesize

        \RaggedRight

        \color{jFrametitleFGColor}

        The content of this lecture is made available under a Creative
        Commons licence (\textsc{cc}), giving the author the credits
        (\textsc{by}) and putting an obligation to share on the same terms
        (\textsc{sa}). Figures and diagrams included in the lecture are
        authored by Paweł Węgrzyn et~al., and are available under the same
        license unless indicated otherwise.\\ The presentation uses the
        Beamer Jagiellonian theme based on Matthias Vogelgesang’s
        Metropolis theme, available under license \LaTeX{} Project
        Public License~1.3c at: \colorhref{https://github.com/matze/mtheme}
        {https://github.com/matze/mtheme}.

        Typographic design: Iwona Grabska-Gradzińska;
        \LaTeX{} Typesetting: Kamil Ziemian \\
        Proofreading: Wojciech Palacz,
        Monika Stawicka \\
        3D Models: Dariusz Frymus, Kamil Nowakowski \\
        Figures and charts: Kamil Ziemian, Paweł Węgrzyn, Wojciech
        Palacz \\
        Film editing: Agencja Filmowa Film \& Television Production~--
        Zbigniew Masklak \\
        Film editing: IMAVI -- Joanna Kozakiewicz, Krzysztof Magda, Nikodem
        Frodyma

      \end{flushleft}

    \end{textblock}

  \end{frame}
}



\newcommand{\GeometryThreeDTwoSpecialEndingSlidesVideoVerOneEN}[1]{%
  \begin{frame}[standout]

    \begin{textblock}{11}(1,0.7)

      \begin{flushleft}

        \mdseries

        \footnotesize

        \color{jFrametitleFGColor}

        This content was created as part of a project co-financed by the
        European Union within the framework of the European Social Fund
        POWR.03.05.00-00-Z309/17-00.

      \end{flushleft}

    \end{textblock}





    \begin{textblock}{10}(0,2.2)

      \tikz \fill[color=jBackgroundStyleLight] (0,0) rectangle (12.8,-1.5);

    \end{textblock}


    \begin{textblock}{3.2}(0.7,2.45)

      \includegraphics[scale=0.3]{\FundingLogoColorPictureEN}

    \end{textblock}


    \begin{textblock}{2.5}(4.15,2.5)

      \includegraphics[scale=0.2]{\JULogoColorPictureEN}

    \end{textblock}


    \begin{textblock}{2.5}(6.35,2.4)

      \includegraphics[scale=0.1]{\ZintegrUJLogoColorPictureEN}

    \end{textblock}


    \begin{textblock}{4.2}(8.4,2.6)

      \includegraphics[scale=0.3]{\EUSocialFundLogoColorPictureEN}

    \end{textblock}





    \begin{textblock}{11}(1,4)

      \begin{flushleft}

        \mdseries

        \footnotesize

        \RaggedRight

        \color{jFrametitleFGColor}

        The content of this lecture is made available under a Creative
        Commons licence (\textsc{cc}), giving the author the credits
        (\textsc{by}) and putting an obligation to share on the same terms
        (\textsc{sa}). Figures and diagrams included in the lecture are
        authored by Paweł Węgrzyn et~al., and are available under the same
        license unless indicated otherwise.\\ The presentation uses the
        Beamer Jagiellonian theme based on Matthias Vogelgesang’s
        Metropolis theme, available under license \LaTeX{} Project
        Public License~1.3c at: \colorhref{https://github.com/matze/mtheme}
        {https://github.com/matze/mtheme}.

        Typographic design: Iwona Grabska-Gradzińska;
        \LaTeX{} Typesetting: Kamil Ziemian \\
        Proofreading: Wojciech Palacz,
        Monika Stawicka \\
        3D Models: Dariusz Frymus, Kamil Nowakowski \\
        Figures and charts: Kamil Ziemian, Paweł Węgrzyn,
        Wojciech Palacz \\
        Film editing: Agencja Filmowa Film \& Television Production~--
        Zbigniew Masklak

      \end{flushleft}

    \end{textblock}

  \end{frame}





  \begin{frame}[standout]


    \begingroup

    \color{jFrametitleFGColor}

    #1

    \endgroup

  \end{frame}
}



\newcommand{\GeometryThreeDTwoSpecialEndingSlidesVideoVerTwoEN}[1]{%
  \begin{frame}[standout]

    \begin{textblock}{11}(1,0.7)

      \begin{flushleft}

        \mdseries

        \footnotesize

        \color{jFrametitleFGColor}

        This content was created as part of a project co-financed by the
        European Union within the framework of the European Social Fund
        POWR.03.05.00-00-Z309/17-00.

      \end{flushleft}

    \end{textblock}





    \begin{textblock}{10}(0,2.2)

      \tikz \fill[color=jBackgroundStyleLight] (0,0) rectangle (12.8,-1.5);

    \end{textblock}


    \begin{textblock}{3.2}(0.7,2.45)

      \includegraphics[scale=0.3]{\FundingLogoColorPictureEN}

    \end{textblock}


    \begin{textblock}{2.5}(4.15,2.5)

      \includegraphics[scale=0.2]{\JULogoColorPictureEN}

    \end{textblock}


    \begin{textblock}{2.5}(6.35,2.4)

      \includegraphics[scale=0.1]{\ZintegrUJLogoColorPictureEN}

    \end{textblock}


    \begin{textblock}{4.2}(8.4,2.6)

      \includegraphics[scale=0.3]{\EUSocialFundLogoColorPictureEN}

    \end{textblock}





    \begin{textblock}{11}(1,4)

      \begin{flushleft}

        \mdseries

        \footnotesize

        \RaggedRight

        \color{jFrametitleFGColor}

        The content of this lecture is made available under a Creative
        Commons licence (\textsc{cc}), giving the author the credits
        (\textsc{by}) and putting an obligation to share on the same terms
        (\textsc{sa}). Figures and diagrams included in the lecture are
        authored by Paweł Węgrzyn et~al., and are available under the same
        license unless indicated otherwise.\\ The presentation uses the
        Beamer Jagiellonian theme based on Matthias Vogelgesang’s
        Metropolis theme, available under license \LaTeX{} Project
        Public License~1.3c at: \colorhref{https://github.com/matze/mtheme}
        {https://github.com/matze/mtheme}.

        Typographic design: Iwona Grabska-Gradzińska;
        \LaTeX{} Typesetting: Kamil Ziemian \\
        Proofreading: Wojciech Palacz,
        Monika Stawicka \\
        3D Models: Dariusz Frymus, Kamil Nowakowski \\
        Figures and charts: Kamil Ziemian, Paweł Węgrzyn,
        Wojciech Palacz \\
        Film editing: IMAVI -- Joanna Kozakiewicz, Krzysztof Magda, Nikodem
        Frodyma

      \end{flushleft}

    \end{textblock}

  \end{frame}





  \begin{frame}[standout]


    \begingroup

    \color{jFrametitleFGColor}

    #1

    \endgroup

  \end{frame}
}



\newcommand{\GeometryThreeDTwoSpecialEndingSlidesVideoVerThreeEN}[1]{%
  \begin{frame}[standout]

    \begin{textblock}{11}(1,0.7)

      \begin{flushleft}

        \mdseries

        \footnotesize

        \color{jFrametitleFGColor}

        This content was created as part of a project co-financed by the
        European Union within the framework of the European Social Fund
        POWR.03.05.00-00-Z309/17-00.

      \end{flushleft}

    \end{textblock}





    \begin{textblock}{10}(0,2.2)

      \tikz \fill[color=jBackgroundStyleLight] (0,0) rectangle (12.8,-1.5);

    \end{textblock}


    \begin{textblock}{3.2}(0.7,2.45)

      \includegraphics[scale=0.3]{\FundingLogoColorPictureEN}

    \end{textblock}


    \begin{textblock}{2.5}(4.15,2.5)

      \includegraphics[scale=0.2]{\JULogoColorPictureEN}

    \end{textblock}


    \begin{textblock}{2.5}(6.35,2.4)

      \includegraphics[scale=0.1]{\ZintegrUJLogoColorPictureEN}

    \end{textblock}


    \begin{textblock}{4.2}(8.4,2.6)

      \includegraphics[scale=0.3]{\EUSocialFundLogoColorPictureEN}

    \end{textblock}





    \begin{textblock}{11}(1,4)

      \begin{flushleft}

        \mdseries

        \footnotesize

        \RaggedRight

        \color{jFrametitleFGColor}

        The content of this lecture is made available under a Creative
        Commons licence (\textsc{cc}), giving the author the credits
        (\textsc{by}) and putting an obligation to share on the same terms
        (\textsc{sa}). Figures and diagrams included in the lecture are
        authored by Paweł Węgrzyn et~al., and are available under the same
        license unless indicated otherwise. \\ The presentation uses the
        Beamer Jagiellonian theme based on Matthias Vogelgesang’s
        Metropolis theme, available under license \LaTeX{} Project
        Public License~1.3c at: \colorhref{https://github.com/matze/mtheme}
        {https://github.com/matze/mtheme}.

        Typographic design: Iwona Grabska-Gradzińska;
        \LaTeX{} Typesetting: Kamil Ziemian \\
        Proofreading: Leszek Hadasz, Wojciech Palacz,
        Monika Stawicka \\
        3D Models: Dariusz Frymus, Kamil Nowakowski \\
        Figures and charts: Kamil Ziemian, Paweł Węgrzyn,
        Wojciech Palacz \\
        Film editing: Agencja Filmowa Film \& Television Production~--
        Zbigniew Masklak \\
        Film editing: IMAVI -- Joanna Kozakiewicz, Krzysztof Magda, Nikodem
        Frodyma


      \end{flushleft}

    \end{textblock}

  \end{frame}





  \begin{frame}[standout]


    \begingroup

    \color{jFrametitleFGColor}

    #1

    \endgroup

  \end{frame}
}











% ------------------------------------------------------------------------------------
% Importing packages, libraries and setting their configuration
% ------------------------------------------------------------------------------------

% ------------------------------------------------------
% Packages for scientific papers
% ------------------------------------------------------
% Switching off \lll symbol, that I guess is representing letter ``Ł''.
% It collide with `amsmath' package's command with the same name
% \let\lll\undefined
% Basic package from American Mathematical Society (AMS)
% \usepackage[intlimits]{amsmath}
% Equations are numbered separately in every section.
% \numberwithin{equation}{section}





% ------------------------------------------------------
% BibLaTeX
% ------------------------------------------------------
% Package biblatex, with biber as its backend, allow us to handle
% bibliography entries that use Unicode symbols outside ASCII.
\usepackage[
language=polish,
backend=biber,
style=alphabetic,
url=false,
eprint=true,
]{biblatex}

\addbibresource{Herdegens-approach-and-two-quasi-delta-ETC-Bibliography.bib}





% ------------------------------------------------------
% Wonderful package PGF/TikZ
% ------------------------------------------------------

% Node and pics for drawing charts
\usepackage{./Local-packages/PGF-TikZ-Chart-nodes-and-pics}

% Styles for arrows
\usepackage{./Local-packages/PGF-TikZ-Arrows-styles}

% Pic for drawing functions
\usepackage{./Local-packages/PGF-TikZ-Functions-pics}






% ------------------------------------------------------
% Local packages
% ------------------------------------------------------
% Special configuration for this particular presentation
\usepackage{./Local-packages/local-settings}

% Package containing various command useful for working with a text
\usepackage{./Local-packages/general-commands}

% Package containing commands and other code useful for working with
% mathematical text
\usepackage{./Local-packages/math-commands}










% ------------------------------------------------------------------------------------------------------------------
\title{Herdegen's approach to Casimir effect in application to
  the~two quasi-delta system}

\author{Kamil Ziemian \\
  \texttt{kziemianfvt@gmail.com}}


% \institute{Uniwersytet Jagielloński w~Krakowie}

\date[13 X 2024]{Seminar of Field Theory Departament,
  13 October 2024}
% ------------------------------------------------------------------------------------------------------------------










% ####################################################################
% Beginning of the document
\begin{document}
% ####################################################################





% ######################################
% Number of chars: 62k+, 73k+, 25k+, 51k+,
% Text is adjusted to the left and words are broken at the end of the line.
\RaggedRight
% ######################################





% ######################################
\maketitle
% ######################################





% ######################################
\begin{frame}
  \frametitle{Table of contents}


  \tableofcontents % Spis treści

\end{frame}
% ######################################










% % ######################################
% \section{General information about the talk}
% % ######################################


% % ##################
% \begin{frame}
%   \frametitle{General information about the talk}


%   If you have any questions about the topic, please interrupt me during
%   the~talk. I~will be also grateful for suggestions how this talk can be
%   improved and for pointing out any mistakes found in it.

%   In this presentation I will often gloss over more technical details
%   and assumptions. To be more concise I will also simplified some
%   historical facts. My hope is that it allow us to presented
%   general picture of problem in a clear and easy manner.

% \end{frame}
% % ##################










% ######################################
\section{Short history of Casimir effect}
% ######################################



% ##################
\begin{frame}
  \frametitle{Casimir paper from 1948}


  In the year 1948 Hendrik Brugt Gerhard Casimir published a~only three
  pages long article \textit{On~the~Attraction Between Two Perfectly
    Conducting Plates}
  \parencite{Casimir-On-the-Attraction-Between-ETC-Pub-1948}, in
  which he predicted that two electrical neutral plates put inside the
  cubic cavity will attract each other due to the strange relation between
  ``zero-point energy'' of the~electromagnetic field and different
  boundary conditions.

  As it is well known, see for example book by Peskin and Schroeder
  \parencite{Peskin-Schroeder-An-Introduction-to-Quantum-Field-Theory-Pub-1995},
  procedure known as canonical quantisation applied to classical
  electromagnetic field produces ``infinite constant'' in the~expression
  for energy, which he refers to as ``zero-point energy''. Casimir's claims
  was that since boundary conditions change ``value of infinite constant'',
  their difference should manifest itself as finite force.

\end{frame}
% ##################





% ##################
\begin{frame}
  \frametitle{Zero-point energies used by Casimir}


  Casimir considered zero-point energy of empty cavity, that is expressed
  by formula:
  \begin{equation}
    \label{eq:Short-history-ETC-01}
    % \begin{split}
    \frac{ 1 }{ 2 } \left( \sum \hbar \omega \right)_{ \text{II} } =
    \frac{ \hbar c a \HorSpaceFive L^{ 2 } }{ 2 \pi^{ 2 } }
    \int\limits_{ 0 }^{ \infty } \int\limits_{ 0 }^{ \infty } \kappa
    \sqrt{ k_{ \HorSpaceOne z }^{ \,\, 2 } + \kappa^{ 2 } } \, d\kappa \,
    dk_{ \HorSpaceTwo z }.
    % \end{split}
  \end{equation}
  Notation used here is taken from the original paper.

  On the other hand, zero-point energy in the situation when
  two parallel plates, with Dirichlet boundary conditions, are
  inserted at the distance $a$ from each other is given by:
  \begin{equation}
    \label{eq:Short-history-ETC-02}
    \frac{ 1 }{ 2 } \left( \sum \hbar \omega \right)_{ \text{I} } =
    \frac{ \hbar c \HorSpaceFive L^{ 2 } }{ 2 \pi }
    \sum_{ n = ( 0 ), \, 1 }^{ \infty } \, \int\limits_{ 0 }^{ \infty } \kappa
    \sqrt{ \left( n^{ 2 } \frac{ \pi^{ 2 } }{ a^{ 2 } } \right)
      + \kappa^{ 2 } } \, d\kappa.
  \end{equation}

\end{frame}
% ##################





% ##################
\begin{frame}
  \frametitle{Introduction of cutoff}


  To obtain finite result from
  $\delta E = \frac{ 1 }{ 2 } ( \sum \hbar \omega )_{ \text{I} } -
  \frac{ 1 }{ 2 } ( \sum \hbar \omega )_{ \text{II} }$ Casimir needs to introduce
  cutoff function $f$ and parameter $k_{ \HorSpaceThree m }$.
  After doing that regularized energy difference of zero-point energies is
  given by
  \begin{equation}
    \label{eq:Short-history-ETC-03}
    \begin{split}
      \delta E
      &=
        \frac{ \hbar c \HorSpaceFive L^{ 2 } \pi^{ 2 } }{ 4 a^{ 3 } } \!
        \left( \, \sum_{ ( 0 ), \, 1 }^{ \infty } \,
        \int\limits_{ 0 }^{ \infty } \! d\kappa \, \sqrt{ n^{ 2 } \! + \! u } \,
        f\!\left( \tfrac{ \pi \sqrt{ n^{ 2 } + u } }{ a k_{ \HorSpaceTwo m } }
        \right) \right. \, - \\
      &\left. \;\;\;
        - \int\limits_{ 0 }^{ \infty } \! \int\limits_{ 0 }^{ \infty } \! du \, dn \,
        \sqrt{ n^{ \, 2 } \! + \! u } \,
        f\!\left( \tfrac{ \pi \sqrt{ n^{ 2 } + u } }{
        a k_{ \HorSpaceTwo m } } \right) \right).
    \end{split}
  \end{equation}

  By expanding this expression into a series and taking only zero order term
  Casimir arrives at formula:
  \begin{equation}
    \label{eq:Short-history-ETC-04}
    \frac{ \delta E }{ L^{ 2 } } =
    -\hbar c \frac{ \pi^{ 2 } }{ 720 } \frac{ 1 }{ a^{ 3 } },
  \end{equation}
  which gives us now a famous expression for force, proportional
  to~$a^{ -4 }$.

\end{frame}
% ##################





% ##################
\begin{frame}
  \frametitle{Review of Casimir approach}


  The Casimir method of computation is still present in current literature
  in more or less refined form. I~don't believe this can be view
  as satisfactory in $2024$.

  First of all, its use of zero-point energy, which according to
  original~(!) Casimir paper is itself nonphysical, is hard to explain.
  It is worth noticing that for over $700$ pages of Peskin and Schroeder
  book, zero-point energy is considered as physically irrelevant, to
  suddenly be deemed physically important in the context of Casimir
  effect. The authors seem to be dissatisfied by this state of affairs.

  Second, it does not introduce physical properties of materials at the
  beginning, but as an auxiliary tool for removing ill defined mathematical
  quantities. For both these reasons, alternative approaches should be
  welcome.

\end{frame}
% ##################





% ##################
\begin{frame}
  \frametitle{Number of citations of Casimir 1948 paper}


  \begin{figure}

    \label{fig:aaa}

    \centering


    \includegraphics[scale=0.6]
    {./Presentation-pictures/Casimir-paper-citations.png}

    \caption{Inspire HEP graph of numbers of citation of Casimir 1948 paper
      per year, \colorhref{https://inspirehep.net/literature/24990}
      {https://inspirehep.net/literature/24990}}


  \end{figure}

\end{frame}
% ##################










% ######################################
\section{Overview of Herdegen's approach to Casimir effect}
% ######################################



% ##################
\begin{frame}
  \frametitle{Introduction to Herdegen's approach}


  Approach to Casimir force that we will be discussing, was first announced
  in short paper of Andrzej Herdegen \textit{No-nonsens Casimir force}
  \parencite{Herdegen-Nononsens-Casimir-force-Pub-2001} and fully presented
  in the two part article \textit{Quantum backreaction (Casimir) effect}
  \parencite{Herdegen-Quantum-backreaction-ETC-Part-I-Pub-2005},
  \parencite{Herdegen-Quantum-backreaction-ETC-Part-II-Pub-2006}. Casimir
  effect in this approach is interpreted as backreaction of quantum system
  to adiabatic changes in macroscopic system and apparatus of algebraic
  quantum field theory (\textsc{aqft}) is used to allow rigorous analysis
  of physical properties of the system.

  We need to pay special attention to how the Casimir effect is interpreted
  in this formalism. According to it, this effect is present in the system
  having the quantum part $Q$ and macroscopic part $M$. In the standard
  example quantum part is quantum electromagnetic field and macroscopic
  contains two metal, electric neutral plates. When $M$ moves
  \alert{adiabatically} there is backreaction on the part of subsystem $Q$,
  that gives rise to particular force.

\end{frame}
% ##################





% ##################
\begin{frame}
  \frametitle{Elements of Herdegen's approach}


  According to my reading of experimental literature on the subject,
  this assumption of adiabatic movement of macroscopic subsystem fits
  well with the way how true experiments are done.

  We now want to give a~short summary of algebraic quantum field theory.
  The~two main pillars of it can be summarized in a~somewhat simplified
  ways as follows.
  \begin{itemize}

  \item For quantum observables the most important relations are of
    algebraic nature.

  \item Physical measurements are localized in bounded regions of
    spacetime.

  \end{itemize}
  Due to time constraints we can only sketch how this quite sophisticated
  formalism works. We first try to explain, what is the idea laying behind
  the~first point.

\end{frame}
% ##################






% ##################
\begin{frame}
  \frametitle{Algebraic nature of observables}


  In standard curse of non-relativistic quantum mechanics, assuming
  one dimensional problem, the position and momentum operators are given as
  \begin{equation}
    \label{eq:Overview-of-Herdegens-approach-ETC-01}
    X = x, \quad
    P = -i \, \hbar \frac{ d }{ dx },
  \end{equation}
  and both of these operators act on a~space of physical states that is
  equal to $L^{ 2 }( \Rbb, d\mu( x ) )$. From this, glossing over some
  mathematical problems, we can derive relation
  \begin{equation}
    \label{eq:Overview-of-Herdegens-approach-ETC-02}
    [ X, P ] = i \, \hbar \, \id.
  \end{equation}
  In \textsc{aqft} we interpreted equation
  \eqref{eq:Overview-of-Herdegens-approach-ETC-02} as being fundamental
  relation between observables of algebraic nature. More precisely, given
  physical system, we attach to it $C^{ * }$ algebra $\Acal$, representing
  physical quantities that can be measured inside it.

\end{frame}
% ##################





% ##################
\begin{frame}
  \frametitle{Algebraic nature of observables}


  For technical reasons, we replace relation $[ X, P ] = i \, \hbar \, \id$
  with it Weyl form:
  \begin{equation}
    \label{eq:Overview-of-Herdegens-approach-ETC-03}
    e^{ \, i P s } e^{ \, i X t } = e^{ \, i X t } e^{ \, i P s } e^{ \, i s t }.
  \end{equation}
  % We can assume that our algebra $\Acal$ has a~neutral element $\UnitAlg$.
  While element of algebra $\Acal$ represents measurable physical
  quantities, the states are represented by class of the elements of space
  dual to $\Acal$: $\omega \in \Acal^{ \text{d} }$. Such $\omega$ to be considered state,
  must fulfill two requirements:
  \begin{equation}
    \label{eq:Overview-of-Herdegens-approach-ETC-04}
    \omega( A^{ * } A ) \geq 0, \; \forall A \in \Acal, \quad
    \omega( \UnitAlg ) = 1.
  \end{equation}

  \vspace{-2em}



  We can now talk about localization of physical events in the Minkowski
  spacetime.

\end{frame}
% ##################





% ##################
\begin{frame}
  \frametitle{Picture of Minkowski spacetime}


  \begin{figure}

    \centering


    \begin{tikzpicture}

      % x axis
      \draw[axis arrow] (-5,0) -- (5,0);

      \pic at (5,0) {x mark for horizontal axis 1};


      % t axis
      \draw[axis arrow] (0,-2.5) -- (0,4);

      \pic at (0,4) {t mark for vertical axis 1};



      % Light cone
      \fill[color=blue,opacity=0.4] (0,0) -- (3.5,3.5) -- (-3.5,3.5) -- cycle;

      \fill[color=blue,opacity=0.4] (0,0) -- (2.2,-2.2) -- (-2.2,-2.2) --
      cycle;

      \draw[dashed] (2.2,-2.2) -- (-3.5,3.5);

      \draw[dashed] (-2.2,-2.2) -- (3.5,3.5);



      % Region number 1
      \draw[dashed] plot[smooth cycle]
      coordinates { (1.2,0) (3,-1) (3.5,1.4) (2.5,1.2) };

      \node[scale=1.3] at (2.3,0.3) {$\Ocal_{ 1 }$};


      % Region number 2
      \draw[dashed] plot[smooth cycle]
      coordinates { (0.8,0) (2.9,-1.5) (4.2,1.8) (1,3) };

      \node[scale=1.3] at (2.7,1.8) {$\Ocal_{ 2 }$};


      % Region number 3
      \draw[dashed] plot[smooth cycle]
      coordinates { (-1,-0.2) (-2,1.2) (-3,0.3) (-2,-1) };

      \node[scale=1.3] at (-2,0.5) {$\Ocal_{ 3 }$};


      % Event reaching form region 3 to region 1
      \draw[dashed,color=black] (-1.3,0) -- (2,3.3);

      \fill[color=brown] (-1.3,0) circle [radius=0.09];

      \node[below] at (-1.3,-0.05) {$E$};

    \end{tikzpicture}

    \caption{Picture of spacetime}


  \end{figure}

\end{frame}
% ##################





% ##################
\begin{frame}
  \frametitle{Localization of algebras}


  Previously we were just talking about algebras, but in \textsc{aqft}
  every algebra is localized in a~particular open region of spacetime
  with compact closure. We will denotes the algebra of region $\Ocal_{ 1 }$
  as $\Acal( \Ocal_{ 1 } )$. This algebra represents all quantities
  that can be measured in such regions. You can imagine that such
  represents experiment that take finite space and work for finite time.

  Family of all such localized algebras we will call \textbf{net of
    algebras}. Since regions from previous picture obeys relation
  $\Ocal_{ 1 } \subset \Ocal_{ 2 }$ we require that
  \begin{equation}
    \label{eq:Overview-of-Herdegens-approach-ETC-05}
    \Acal( \Ocal_{ 1 } ) \subset \Acal( \Ocal_{ 2 } ),
  \end{equation}
  which has natural interpretation. It simply says that every quantity
  that can be measured in particular region is also measurable in any
  region that extends the first one.

\end{frame}
% ##################





% ##################
\begin{frame}
  \frametitle{Causality and problems with construction}


  Analogously, since there is no causally link between $\Ocal_{ 1 }$ and
  $\Ocal_{ 3 }$, we require that measurements of all physical quantities
  are independent. Algebraically this mean that
  \begin{equation}
    \label{eq:Overview-of-Herdegens-approach-ETC-06}
    [ A, B ] = 0, \quad
    \forall A \in \Acal( \Ocal_{ 1 } ), B \in \Acal( \Ocal_{ 3 } ).
  \end{equation}
  This and other requirements that we won't mention, that nets of algebras
  need to obey, should be viewed as axioms in the spirit of axiomatic
  quantum field theory.

  Construction of a~model fulfilling any such scheme is notoriously hard,
  see \parencite{Summers-Prespective-on-Constructive-ETC-Ver-2016} for the
  account. To my knowledge at this moment no
  model of quantum field theory in the $1 + 3$ dimension interacting with
  itself or other quantum fields was explicitly constructed. Fortunately,
  theory of quantum field weakly interacting with classical object
  were constructed and are general enough to allow us to handle Casimir
  effect in a variety of cases.

\end{frame}
% ##################





% ##################
\begin{frame}
  \frametitle{AQFT suited for Casimir effect}


  Herdegen's paper
  \parencite{Herdegen-Quantum-backreaction-ETC-Part-I-Pub-2005}
  presents a set of axioms tailored for discussion of Casimir effect.
  Here we can only mention a few of them.

  We start from the assumption that the quantum subsystem $Q$ is defined by
  its algebra $\Acal$ and representation $\pi$ of this algebra of bounded
  operators on Hilbert space $\Hcal$.
  \begin{equation}
    \label{eq:Overview-of-Herdegens-approach-ETC-07}
    \Acal \ni A \mapsto \pi( A ) \in \Bcal( \Hcal ).
  \end{equation}
  Algebra $\Acal$ is localized on hyperplane $t = \const$, which due to
  assumption of adiabatic changes inside the system, pose no problem to
  the~formalism.

  Time evolution on algebra $\Acal$ is given by family of automorphism
  $\alpha_{ \HorSpaceThree t }$ and translates to unitary evolution on $\Hcal$:
  \begin{equation}
    \label{eq:Overview-of-Herdegens-approach-ETC-07}
    \pi( \alpha_{ \HorSpaceThree t } \, A ) \mapsto U( t ) \, \pi( A ) \, U^{ * }( t ), \qquad
    U( t ) = \exp( i t H ).
  \end{equation}
  This relation defines the~hamiltonian of the free quantum system.

\end{frame}
% ##################





% ##################
\begin{frame}
  \frametitle{AQFT suited for Casimir effect}


  About the macroscopic system $M$ we assume that it changes its states
  adiabatically, due to some external conditions. The backreaction of
  quantum subsystem $Q$ to this movement results in transfer of energy to
  $M$, which manifests itself as Casimir force.

  Let~$a$ be a collective name for parameters describing subsystem~$M$. For
  two plates~$a$ reduce to one number, distance of the plates.
  \textsc{aqft} scheme requires that algebra $\Acal$ cannot change when
  part $M$ is introduced to the free system $Q$ and representation $\pi$
  can only change into equivalent one.

  By $H_{ a }$ we will denote the hamiltonian of subsystem $Q$ interacting
  with $M$ in position~$a$. We assume that for every allowed $a$ exists
  non degenerate eigenstate of $H_{ a }$:
  \begin{equation}
    \label{eq:Overview-of-Herdegens-approach-ETC-07}
    H_{ a } \HorSpaceFive \psi_{ \HorSpaceOne a } =
    E_{ \HorSpaceOne a } \HorSpaceFive \psi_{ \HorSpaceOne a }.
  \end{equation}

\end{frame}
% ##################





% ##################
\begin{frame}
  \frametitle{AQFT suited for Casimir effect}


  In particular, in the presence of $M$ the ground state of the subsystem
  $Q$ became function of $a$. We will denote it by $\Omega_{ a }$.

  We now switch to Schr\"{o}dinger picture of interaction. If movement of
  $M$ is given by function $a( t )$, then according to adiabatic
  approximation we have following evolution of physical state:
  \begin{equation}
    \label{eq:Formalizm-algebraiczny-dla-efektu-Casimira-12}
    \psi( t ) =
    e^{ \HorSpaceOne i \varphi( t ) } \HorSpaceOne \psi_{ \HorSpaceOne a( t ) }, \qquad
    \psi( 0 ) = \psi_{ \HorSpaceOne a( 0 ) }.
  \end{equation}
  The~expectation value of operator $B$ in the time $t$ is now given by
  \begin{equation}
    \label{eq:Epstein-Glaser-24-A}
    \langle B \rangle_{ t } =
    ( \psi_{ \HorSpaceOne a( t ) }, B \HorSpaceOne \psi_{ \HorSpaceOne a( t ) } ).
  \end{equation}
  It should be notice that it doesn't depends on function $\varphi( t )$.

\end{frame}
% ##################





% ##################
\begin{frame}
  \frametitle{AQFT suited for Casimir effect}


  Casimir energy is now defined as exception value of \alert{free}
  hamiltonian of subsystem $Q$ in the ground state $\Omega_{ a }$ of interacting
  system $Q\text{-}M$:
  \begin{equation}
    \label{eq:Epstein-Glaser-24-A}
    \Ecal_{ a } = ( \Omega_{ a }, H \Omega_{ a } ).
  \end{equation}
  This choice may seems controversial, but as Herdegen demonstrated in
  \parencite{Herdegen-Quantum-backreaction-ETC-Part-I-Pub-2005}, this is
  the only unambiguous energy operator present in such settings. We also
  don't see reasons to consider energy that isn't explicitly defined as
  expectation value of self-adjoint operator, even when such approaches to
  Casimir effect are present in the literature, cf. paper by Scardicchio
  about Casimir effect and Dirac delta potentials
  \parencite{Scardicchio-Casimir-dynamics-ETC-2005}.

  Casimir force is now defined in the~standard way:
  \begin{equation}
    \label{eq:Epstein-Glaser-24-A}
    \Fcal_{ a } = -\frac{ \partial \Ecal_{ a } }{ \partial a }.
  \end{equation}

\end{frame}
% ##################










% ######################################
\section{Construction of quasi-free system and expressions
  for physical quantities}
% ######################################



% ##################
\begin{frame}
  \frametitle{Quasi-free system}


  By \textbf{quasi-free system} we understand system with symplectic space,
  time evolution on it given by family of symplectic transformations, which
  also has specified quantum version. This quantum version must be
  represented in Fock space by canonical variable, that are linear
  combination of operators $a$ and $a^{ \dagger }$. We also require that
  quantum hamiltonian is hermitian form in $a$ and $a^{ \dagger }$.

  Andrzej Herdegen and Mariusz Stopa in paper \textit{Global vs~local
    Casimir effect} \parencite{Herdegen-Stopa-Global-vs-local-ETC-2010}
  presented following way of treating Casimir effect. We start from
  free quantum scalar field obeying equation:
  \begin{equation}
    \label{eq:Epstein-Glaser-24-A}
    \left( \partial_{ t }^{ 2 } + h^{ 2 } \right) \phi( t, \vecx ) = 0, \qquad
    h^{ 2 } = -\Delta.
  \end{equation}
  The presence of macroscopic bodies is modeled by perturbing operator
  $h^{ 2 }$ by adding to it operator of finite rank $V_{ a }$:
  \begin{equation}
    \label{eq:Epstein-Glaser-24-A}
    h_{ a }^{ 2 } = h^{ 2 } + V_{ a }.
  \end{equation}

\end{frame}
% ##################





% ##################
\begin{frame}
  \frametitle{Construction of quasi-free system}


  \begin{equation}
    \label{eq:Epstein-Glaser-24-A}
    h_{ a }^{ 2 } = h^{ 2 } + V_{ a }.
  \end{equation}
  This equation define model of subsystem $Q$ interacting with $M$.
  We should stress that all effects caused by subsystem $M$ are contained
  in finite-rank operator $V_{ a }$.

  In the next step we need to take square root of operators $h^{ 2 }$
  and $h_{ a }^{ 2 }$, which means that they need to be positive. This
  requirement is source of few subtleties in the problem.

  Assuming that square roots $h$ and $h_{ a }$ of $h^{ 2 }$ and $h_{ a }^{ 2 }$
  exists, we can now use known for a~long time procedure to construct
  quasi-free theory. One reference in which it could be find is Herdegen
  paper \parencite{Herdegen-Quantum-backreaction-ETC-Part-I-Pub-2005}.

\end{frame}
% ##################





% % ##################
% \begin{frame}
%   \frametitle{Construction of quasi-free system}
% !!!!!!!!!!!!!!!!!!!!!!!!!!!!!!
%   \vspace{-0.5em}


%   \begin{figure}

%     \centering


%     \begin{tikzpicture}

%       \node (Rcal-h) at (0,0) {$( \Rcal, h )$};


%       \node (Lcal-sigma) at (-2,-0.8) {$( \Lcal, \sigma )$};

%       \draw[pointing arrow thin 1, dashed] (Rcal-h) -- (Lcal-sigma);


%       \node (LcalTilde-sigma) at (-2,-2) {$( \LcalTilde, \sigma )$};

%       \draw[pointing arrow thin 1, dashed] (Lcal-sigma) -- (LcalTilde-sigma);


%       \node (LcalHat-sigmaHat) at (-2,-3.4) {$( \LcalHat, \sigmaHat )$};

%       \draw[pointing arrow thin 1, dashed] (LcalTilde-sigma) --
%       (LcalHat-sigmaHat);





%       \node (Kcal) at (1.8,-2) {$\Kcal$};

%       \draw[pointing arrow thin 1, dashed] (Rcal-h) -- (Kcal);


%       \draw[pointing arrow thin 1] (LcalTilde-sigma) -- (Kcal);

%       \node (j) at (0.1,-1.7) {$j( V )$};


%       \draw[pointing arrow thin 1] (LcalHat-sigmaHat)
%       .. controls (-1,-3.5) and (1,-3.2) .. (Kcal);

%       \node (jHat) at (0.1,-2.8) {$\jHat( V )$};



%       \node (Weyl-algebra) at (-2,-5.2) {$W( \LcalHat \, \, )$};

%       \draw[pointing arrow thin 1] (LcalHat-sigmaHat) -- (Weyl-algebra);

%       \node (Weyl-opeartor) at (-2.5,-4.3) {$W( V )$};


%       \node (H-Fock) at (1.8,-5.2) {$\HFock( \Kcal )$};

%       \draw[pointing arrow thin 1] (Kcal) -- (H-Fock);

%       \node (W0-f) at (1.2,-3.6) {$W_{ 0 }( \HorSpaceNine f )$};

%       \draw[pointing arrow thin 1,dashed] (Kcal)
%       .. controls (2.7,-2.9) and (2.6,-4.3) .. (2.1,-4.9);



%       \draw[pointing arrow thin 1] (Weyl-algebra) -- (H-Fock);

%       \node at (0,-4.95) {$\pi_{ \; \jHat }$};

%     \end{tikzpicture}

%     \caption{Scheme of construction of the quasi-free theory. Dashed arrow
%       means that one space is constructed from the other. Continuous arrow
%       means that we have a function from one space to another.}


%     \label{fig:Scheme-of-construction}


%   \end{figure}

% \end{frame}
% % ##################





% ##################
\begin{frame}
  \frametitle{Number of particles and Casimir energy}


  In paper \parencite{Herdegen-Quantum-backreaction-ETC-Part-I-Pub-2005}
  Herdegen used quite complicated decomposition of field into to creation
  and annihilation operators to prove two fundamental relations. First, that
  number of quanta of \alert{free} field created when subsystem $M$ in
  position $a$ is introduced is given by
  \begin{equation}
    \label{eq:Uklad-quasi-swobodny-w-obecnosci-ETC-21}
    \Ncal_{ a }^{ \vphantom{1} } :=
    \frac{ 1 }{ 4 } \Tr\big[ \HorSpaceOne h^{ -1/2 } (
    h_{ \HorSpaceOne a }^{ \vphantom{1} } - h ) h_{ \HorSpaceOne a }^{ -1 }
    ( h_{ a }^{ \vphantom{1} } - h ) h^{ -1/2 } \HorSpaceTwo \big].
  \end{equation}
  Using same method he proved that
  \begin{equation}
    \label{eq:Epstein-Glaser-24-A}
    \Ecal_{ \HorSpaceTwo a } :=
    ( \GroundStateAOne, H \HorSpaceOne \GroundStateAOne ) =
    \frac{ 1 }{ 4 }
    \Tr\big[ ( h_{ \HorSpaceOne a }^{ \vphantom{1} } - h )
    h_{ \HorSpaceOne a }^{ -1 }
    ( h_{ \HorSpaceOne a }^{ \vphantom{1} } - h ) \big].
  \end{equation}
  This two equation are crucial to the rest of the talk.

\end{frame}
% ##################





% ##################
\begin{frame}
  \frametitle{Number of particles and Casimir energy}

  \vspace{-2em}


  \begin{align}
    \label{eq:Uklad-quasi-swobodny-w-obecnosci-ETC-21}
    \Ncal_{ a }^{ \vphantom{1} }
    &=
      \frac{ 1 }{ 4 } \Tr\big[ \HorSpaceOne h^{ -1/2 } (
      h_{ \HorSpaceOne a }^{ \vphantom{1} } - h )
      h_{ \HorSpaceOne a }^{ -1 }
      ( h_{ a }^{ \vphantom{1} } - h ) h^{ -1/2 } \HorSpaceTwo \big]
      < +\infty, \\
    \label{eq:Epstein-Glaser-24-A}
    \Ecal_{ \HorSpaceTwo a }
    &=
      ( \GroundStateAOne, H \HorSpaceOne \GroundStateAOne ) =
      \frac{ 1 }{ 4 }
      \Tr\big[ ( h_{ \HorSpaceOne a }^{ \vphantom{1} } - h )
      h_{ \HorSpaceOne a }^{ -1 }
      ( h_{ \HorSpaceOne a }^{ \vphantom{1} } - h ) \big] < +\infty.
  \end{align}
  Finitness of number of quanta is fundamental from both physical and
  mathematical point of view. It guarantee that various representations
  of elements underplaying algebra $\Acal$ as operators in Hilbert space are
  equivalent. This allow us to compare results for different values of $a$
  in meaningful way. Requirements of finitness of Casimir energy is natural
  assumption of physical nature.

  Another result of paper
  \parencite{Herdegen-Quantum-backreaction-ETC-Part-I-Pub-2005}
  is that energy density defined by standard energy-momentum tensor is
  given by
  \begin{equation}
    \label{eq:Lokalna-gestosc-energii-ETC-06}
    T_{ a }( \varphi, \psi ) =
    \frac{ 1 }{ 4 } \big( \varphi, ( h_{ a }^{ \vphantom{1} } - h ) \psi \big) +
    \frac{ 1 }{ 4 } \big( \nabla \varphi, ( h_{ a }^{ -1 } -
    h^{ -1 }_{ \vphantom{a} } ) \nabla \psi \big).
  \end{equation}
  Unfortunately, this will be marginal topic in this talk.

\end{frame}
% ##################










% ######################################
\section{Scalar field and two quasi-delta}
% ######################################



% ##################
\begin{frame}
  \frametitle{Formulation of the problem}


  Again following Herdegen and Stopa
  \parencite{Herdegen-Stopa-Global-vs-local-ETC-2010}, we start with set of
  equations
  \begin{equation}
    \label{eq:Epstein-Glaser-24-A}
    \left( \partial_{ t }^{ 2 } + h_{ \veca }^{ 2 } \right) \phi( t, \vecx ) = 0, \qquad
    h_{ \veca }^{ 2 } = -\Delta + V_{ \veca }.
  \end{equation}
  We want to represented model in which scalar field interact with two
  spherical symmetric bodies, which centers are separated by vector~$\veca$.

  Again following their footsteps we introduce quasi-potential in the form
  \begin{equation}
    \label{eq:Model-ukladu-z-dwoma-ETC-08}
    V_{ \veca }( \vecx, \vecy \HorSpaceThree ) =
    \sigma( g ) \left[ g\!\left( \vecx - \vecb \HorSpaceFive \right) \,
      \overline{ g\!\left( \vecy - \vecb \HorSpaceFive \right) }
      + g\!\left( \vecx + \vecb \HorSpaceFive \right) \,
      \overline{ g\!\left( \vecy + \vecb \HorSpaceFive \right) } \,
    \right].
  \end{equation}
  To be succinct I~presented quasi-potential in the position representation.
  We assumed that function $g( \vecx )$ is smooth ($C^{ \infty }( \Rbb^{ 3 } )$),
  with compact support contained in ball of radius $L$. Value of $L$
  should be macroscopic, but it particular numeric value is not very
  important at this level of considerations. The real valued function
  $\sigma( g )$ will be explained latter.

\end{frame}
% ##################





% ##################
\begin{frame}
  \frametitle{Formulation of the problem}


  For technical reason we can only solve our problem for function whose
  values are non negative:
  \begin{equation}
    \label{eq:Epstein-Glaser-24-A}
    g( \vecx ) \geq 0.
  \end{equation}

  \vspace{-1em}



  Quasi-potential is projection on the two dimensional subspace spanned by
  functions $g\!\left( \vecx \pm \vecb \HorSpaceFive \right)$.

  Herdegen and Stopa consider two plates system, which due to symmetry
  within the plane of plates, reduce to one dimensional problem.
  Because our model, due to rotational symmetry along one axis, is
  also at some level one dimensional, we can use many the same methods as
  they before. Still, there are very important differences in behavior
  of both systems.

\end{frame}
% ##################





% ##################
\begin{frame}
  \frametitle{Symbolic sketch of quasi-potential}


  Below we can see simple, symbolic to some extent, sketch of
  quasi-potential, with parameter $L = 2$ in appropriate unit system.
  It show shape of function making one particular object.

  \vspace{5em}





  \begin{figure}

    \label{aaa:bbb}


    \centering

    \begin{tikzpicture}

      % Smooth functions witch compact support
      \pic at (-3,0) {smooth function with compact support fill 1};

      \pic at (3,0) {smooth function with compact support fill 1};





      % x axis
      \draw[axis arrow] (-4.5,0) -- (5,0);

      \pic at (5,0) {x mark for horizontal axis 1};


      % y axis
      \draw[axis arrow] (0,-0.2) -- (0,2.3);

      \pic at (0,2.3) {V mark for vertical axis 1};

      \node[right] at (0,-0.2) {$0$};





      % Thicks on x axis
      \pic at (-4,0) {tick x axis thin};

      \node[below] at (-4,0) {$-4$};


      \pic at (-3,0) {tick x axis thin};

      \node[below] at (-3,0) {$-3$};


      \pic at (-2,0) {tick x axis thin};

      \node[below] at (-2,0) {$-2$};


      \pic at (-1,0) {tick x axis thin};

      \node[below] at (-1,0) {$-1$};


      \pic at (1,0) {tick x axis thin};

      \node[number below x axis] at (1,0) {$1$};


      \pic at (2,0) {tick x axis thin};

      \node[number below x axis] at (2,0) {$2$};


      \pic at (3,0) {tick x axis thin};

      \node[number below x axis] at (3,0) {$3$};


      \pic at (4,0) {tick x axis thin 1};

      \node[number below x axis] at (4,0) {$4$};


      % \pic at (5,0) {tick x axis thin};

      % \node[below] at (5,0) {$5$};





      % Ticks on y axis
      \pic at (0,0.5) {tick y axis thin 1};

      \node[node scale small 2,left] at (0,0.5) {$0.5$};


      \pic at (0,1) {tick y axis thin};

      \node[left] at (0,1) {$1$};


      \pic at (0,1.5) {tick y axis thin 1};

      \node[node scale small 2,left] at (0,1.5) {$1.5$};

    \end{tikzpicture}

    \caption{Symbolic sketch of quasi-potential $V_{ \veca }$}


  \end{figure}

\end{frame}
% ##################





% ##################
\begin{frame}
  \frametitle{Few basic properties}


  We need to assume that distance between center of objects
  $a = \absOne{ \veca }$, need be bigger than upper bound of it diameter,
  which is given by parameter~$L$.

  Since our quasi-potential $V_{ a }$ is operator of finite order,
  Kato-Rellich and Kato-Rosenblum guaranties that $h_{ a }^{ 2 }$ is
  self-adjoint operator and that absolutely continuous spectrum of
  $h^{ 2 }$ and $h_{ a }^{ 2 }$ are unitary equivalent. After some
  computation we can prove that both $h^{ 2 }$ and $h_{ a }^{ 2 }$ have only
  absolutely continuous spectrum equal to $[ 0, +\infty )$, which open
  the~way to use scattering theory method to solve this problem. But,
  to show how to do it, we need introduce more functions.

\end{frame}
% ##################





% ##################
\begin{frame}
  \frametitle{Construction of quasi-free system}


  Because function $g( \vecx )$ has spherical symmetry, we can replace it
  by function of one real variable, that we will be denoting simply
  by $g( x )$. Simple calculation shows that Fourier transform $g( \vecx )$,
  which is also spherically symmetric, can be expressed as
  \begin{equation}
    \label{eq:Model-ukladu-z-dwoma-ETC-09}
    \gHat( p ) = \gHat( \vecp \HorSpaceFour ) =
    \frac{ 1 }{ \sqrt{ 2 \pi } }
    \int_{ -\infty }^{ +\infty } dx \, \frac{ \sin( p x ) }{ p x }
    \HorSpaceTwo x^{ 2 } g( x ).
  \end{equation}
  We again replaced function on $\Rbb^{ 3 }$ by function on $\Rbb$.

  We now define a bunch of functions.
  \begin{align}
    \label{eq:Epstein-Glaser-24-A}
    M_{ p }
    &= \gHat( p )^{ 2 }, \\
    \checkMx
    &=
      \frac{ 1 }{ \sqrt{ 2 \pi } }
      \int\limits_{ -\infty }^{ +\infty } dp \, e^{ i p x } M_{ p }.
  \end{align}

\end{frame}
% ##################




% ##################
\begin{frame}
  \frametitle{Computation of number of particles and energy}

  \vspace{-2em}


  \begin{align}
    \chi( g )
    &=
      \frac{ 2 \pi }{ \alpha }
      \int\limits_{ -\infty }^{ +\infty } dp \, M_{ p } =
      \sqrt{ 2 \pi }^{ 3 } \frac{ \checkMzero }{ \alpha }, \\
    \sigma( g )^{ -1 }
    &=
      -\alpha ( M_{ 0 } + \chi( g ) ), \\
    h( x )
    &=
      \frac{ \sin( x ) }{ x }.
  \end{align}
  Form of the $\sigma( g )$ shown above is motivated by problem of rescaling
  of our model. Parameter $\alpha$ is well know from theory of $\delta$ potentials
  and describe scattering length in such system. This is relevant
  since when we shrink size of our bodies to the point, in the limit
  our problem became this of two $\delta$ system.




  % Unfortunately we need few more functions. I must omit few nontrivial
  % steps, so please believe me that this notation have sens.
  % \begin{align}
  %   \label{eq:Epstein-Glaser-24-A}
  %   t( i k )
  %   &=
  %     \alpha + 2\pi \int_{ -\infty }^{ +\infty } dp \, \frac{ k^{ 2 } }{ k^{ 2 } + p^{ 2 } }
  %     M_{ p } \\
  %   u( i k )
  %   &=
  %     2\pi^{ 2 } M_{ i k } \frac{ e^{ -ka } }{ a }.
  % \end{align}

\end{frame}
% ##################





% ##################
\begin{frame}
  \frametitle{Computation of number of particles and energy}


  % We now define bunch functions
  % \begin{align}
  %   \label{eq:Epstein-Glaser-24-A}
  %   M_{ p }
  %   &= \gHat( p )^{ 2 }, \\
  %   \checkMx
  %   &=
  %     \frac{ 1 }{ \sqrt{ 2 \pi } }
  %     \int\limits_{ -\infty }^{ +\infty } dp \, e^{ i p x } M_{ p }, \\
  %   \chi( g )
  %   &=
  %     \frac{ 2 \pi }{ \alpha }
  %     \int\limits_{ -\infty }^{ +\infty } dp M_{ p } =
  %     \sqrt{ 2 \pi }^{ 3 } \frac{ \checkMzero }{ \alpha }, \\
  %   \sigma( g )^{ -1 }
  %   &=
  %     -\alpha ( M_{ 0 } + \chi( g ) ).
  % \end{align}
  % Form of the $\sigma( g )$ shown above is motivated by problem of recycling our
  % model.

  Unfortunately we need few more functions. I must omit few nontrivial
  steps, so please believe me that this notation make sens.
  \begin{align}
    \label{eq:Epstein-Glaser-24-A}
    t( i k )
    &=
      \alpha + 2\pi \int\limits_{ -\infty }^{ +\infty } dp \,
      \frac{ k^{ 2 } }{ k^{ 2 } + p^{ 2 } } M_{ p } \\
    u( i k )
    &=
      2\pi^{ 2 } M_{ i k } \frac{ e^{ -ka } }{ a }.
  \end{align}

  \vspace{-1em}



  Now, using in the intermediate steps some methods of scattering theory
  and few other tricks, we can wrote number of particles and Casimir
  energy using one equation:
  \begin{equation}
    \label{eq:Energia-i-liczba-czastek-dla-modelu-ETC-31}
    \Pcal_{ \tau }( a ) =
    16 \Real \int\limits_{ \Rbb_{ + }^{ \, 2 } } dk \, dp \,
    \frac{ p^{ \HorSpaceOne 2 - \tau } }{ ( p + i k )^{ 2 } } M_{ \HorSpaceOne p }
    \frac{ t( i k ) + h( p a ) u( i k ) }{ t( i k )^{ 2 } - u( i k )^{ 2 } }.
  \end{equation}
  For $\tau = 0$ we get energy and $\tau = 1$ number of created particles.

\end{frame}
% ##################





% ##################
\begin{frame}
  \frametitle{Computation of number of particles and energy}

  \vspace{-1em}


  \begin{equation}
    \label{eq:Energia-i-liczba-czastek-dla-modelu-ETC-31}
    \Pcal_{ \tau }( a ) =
    16 \Real \int\limits_{ \Rbb_{ + }^{ \, 2 } } dk \, dp \,
    \frac{ p^{ \HorSpaceOne 2 - \tau } }{ ( p + i k )^{ 2 } } M_{ \HorSpaceOne p }
    \frac{ t( i k ) + h( p a ) u( i k ) }{ t( i k )^{ 2 } - u( i k )^{ 2 } }.
  \end{equation}
  It is can be shown that both integrals in above equation are convergent.

  We now fix our attention to the energy. By taking limit $a \nearrow +\infty$
  we identify self-energy of plates, i.e. energy of field clustering around
  two bodies. After subtracting it from full Casimir energy, we arrive
  at interaction energy:
  \begin{equation}
    \label{eq:Energia-i-liczba-czastek-dla-modelu-ETC-34}
    \EInt( a ) =
    16 \, \Real \int\limits_{ \Rbb_{ + }^{ \, 2 } } dp \, dk \,
    \frac{ p^{ \HorSpaceOne 2 } }{ ( p + i k )^{ 2 } } \, M_{ \HorSpaceOne p } \,
    \frac{ u( i k )^{ 2 } + h( p a ) t( i k ) u( i k ) }{
      t( i k ) [ t( i k )^{ 2 } - u( i k )^{ 2 } ] }.
  \end{equation}

\end{frame}
% ##################










% ######################################
\section{Rescalling of the model}
% ######################################


% ##################
\begin{frame}
  \frametitle{Computation of number of particles and energy}


  \begin{equation}
    \label{eq:Energia-i-liczba-czastek-dla-modelu-ETC-34}
    \EInt( a ) =
    16 \, \Real \int\limits_{ \Rbb_{ + }^{ \, 2 } } dp \, dk \,
    \frac{ p^{ \HorSpaceOne 2 } }{ ( p + i k )^{ 2 } } \, M_{ \HorSpaceOne p } \,
    \frac{ u( i k )^{ 2 } + h( p a ) t( i k ) u( i k ) }{
      t( i k ) [ t( i k )^{ 2 } - u( i k )^{ 2 } ] }.
  \end{equation}
  Since this formula is quite to hard to understand when bodies have finite
  sizes, we analyzed limit case of rescaled model.

  Rescaled family of models is defined by replacing $g( \vecx )$ by family
  of functions
  \begin{equation}
    \label{eq:Energia-i-liczba-czastek-dla-modelu-ETC-34}
    g_{ \lambda }( \vecx ) =
    \lambda^{ -3 } g\left( \tfrac{ \vecx }{ \lambda } \right), \qquad
    \lambda \in ( 0, 1 ].
  \end{equation}
  It should be noted that function $g_{ \lambda }( \vecx )$, $\lambda$ fixed, have the
  same properties that we demand from function $g( \vecx )$, so all
  previous results folds in this case.

\end{frame}
% ##################





% ##################
\begin{frame}
  \frametitle{Computation of number of particles and energy}


  Since we are only interesting in limit case, we make two assumptions,
  that greatly simplify computations:
  \begin{equation}
    \label{eq:Energia-i-liczba-czastek-dla-modelu-ETC-34}
    L = \frac{ 2 \pi^{ 2 } }{ \alpha } \quad
    \lambda \in \left( 0, \tfrac{ 1 }{ 2 } \right).
  \end{equation}
  First of these assumptions is the reason why equations in the following
  part of the talk \alert{don't} contains parameter~$L$.

  When $\lambda \searrow 0$ our model converge in strong resolvent sens to scalar field
  interacting with two $\delta$-s. This model is well known in literature, see
  monograph by S. Albeverio et al. \parencite{}.


  % \begin{equation}
  %   \label{eq:Energia-Casimira-dla-rodziny-ETC-09}
  %   \begin{split}
  %     &\EInt( a, \lambda ) = \\
  %     &=
  %       \frac{ 2 \alpha }{ \pi^{ 3 } } \! \Bigg[ \frac{ \chi }{ \lambda } \!
  %       \int\limits_{ 0 }^{ +\infty } \!
  %       \frac{ e^{ -2 l } \, dl }{ ( \gamma + l )
  %       \big[ ( \gamma + l )^{ 2 } - e^{ -2 l } \big] } \! + \!
  %       \frac{ b_{ \HorSpaceOne 1 } \chi }{ \gamma } \!
  %       \int\limits_{ 0 }^{ +\infty } \!
  %       \frac{ l^{ 2 } \big[ 3 ( \gamma + l )^{ 2 } e^{ -2 l } -
  %       e^{ -4 l } \big] }
  %       { ( \gamma + l )^{ 2 }
  %       \big[ ( \gamma + l )^{ 2 } - e^{ -2 l } \big]^{ 2 } } dl
  %       \, - \\[0.5em]
  %     &- \! \frac{ 2 }{ \gamma } \! \int\limits_{ 0 }^{ +\infty } \!
  %       \frac{ l e^{ -2 l }\, dl }{ ( \gamma + l )
  %       \big[ ( \gamma + l )^{ 2 } - e^{ -2 l } \big] } \! + \!
  %       \frac{ 1 }{ \gamma } \!
  %       \int\limits_{ 0 }^{ +\infty } \! \frac{ ( 1 - l ) e^{ -2 l } }{
  %       ( \gamma + l )^{ 2 } - e^{ -2 l } } dl + R( a, \lambda ) \Bigg],
  %   \end{split}
  % \end{equation}
  % where $\gamma = ( \alpha a ) / ( 2 \pi^{ 2 } ) > 1$. Condition $\gamma > 1$ allow us to
  % construct quasi-free theory for $\lambda \in ( 0, 1 ]$.

\end{frame}
% ##################





% ##################
\begin{frame}
  \frametitle{Computation of number of particles and energy}

  % \vspace{-2em}


  Our main result is asymptotic expansion of interaction energy:
  \begin{equation}
    \label{eq:Energia-Casimira-dla-rodziny-ETC-09}
    \begin{split}
      &\EInt( a, \lambda ) = \\
      &=
        \frac{ 2 \alpha }{ \pi^{ 3 } } \! \Bigg[ \frac{ \chi }{ \lambda } \!
        \int\limits_{ 0 }^{ +\infty } \!
        \frac{ e^{ -2 l } \, dl }{ ( \gamma + l )
        \big[ ( \gamma + l )^{ 2 } - e^{ -2 l } \big] } \! + \!
        \frac{ b_{ \HorSpaceOne 1 } \chi }{ \gamma } \!
        \int\limits_{ 0 }^{ +\infty } \!
        \frac{ l^{ 2 } \big[ 3 ( \gamma + l )^{ 2 } e^{ -2 l } -
        e^{ -4 l } \big] }
        { ( \gamma + l )^{ 2 }
        \big[ ( \gamma + l )^{ 2 } - e^{ -2 l } \big]^{ 2 } } dl
        \, - \\[0.5em]
      &- \! \frac{ 2 }{ \gamma } \! \int\limits_{ 0 }^{ +\infty } \!
        \frac{ l e^{ -2 l }\, dl }{ ( \gamma + l )
        \big[ ( \gamma + l )^{ 2 } - e^{ -2 l } \big] } \! + \!
        \frac{ 1 }{ \gamma } \!
        \int\limits_{ 0 }^{ +\infty } \! \frac{ ( 1 - l ) e^{ -2 l } }{
        ( \gamma + l )^{ 2 } - e^{ -2 l } } dl + R( a, \lambda ) \Bigg],
    \end{split}
  \end{equation}
  Function $\chi( g ) > 0$ was defined in equation \eqref{}. Parameter
  $b_{ 1 }$ is give by explicit, but quite convoluted integrals of function
  $g( \vecx )$ (notice that it is not dependent on $\lambda$) and fulfill
  inequality $0 < b_{ 1 } \leq \alpha^{ -1 }$.

  You should notice that this formula is \alert{model dependent}. Two terms
  in expansion ``remember'' how we approximated $\delta$-s interactions.

\end{frame}
% ##################





% ##################
\begin{frame}
  \frametitle{Computation of number of particles and energy}

  \vspace{-2em}


  \begin{equation}
    \label{eq:Energia-Casimira-dla-rodziny-ETC-09}
    \begin{split}
      &\EInt( a, \lambda ) = \\
      &=
        \frac{ 2 \alpha }{ \pi^{ 3 } } \! \Bigg[ \frac{ \chi }{ \lambda } \!
        \int\limits_{ 0 }^{ +\infty } \!
        \frac{ e^{ -2 l } \, dl }{ ( \gamma + l )
        \big[ ( \gamma + l )^{ 2 } - e^{ -2 l } \big] } \! + \!
        \frac{ b_{ \HorSpaceOne 1 } \chi }{ \gamma } \!
        \int\limits_{ 0 }^{ +\infty } \!
        \frac{ l^{ 2 } \big[ 3 ( \gamma + l )^{ 2 } e^{ -2 l } -
        e^{ -4 l } \big] }
        { ( \gamma + l )^{ 2 }
        \big[ ( \gamma + l )^{ 2 } - e^{ -2 l } \big]^{ 2 } } dl
        \, - \\[0.5em]
      &- \! \frac{ 2 }{ \gamma } \! \int\limits_{ 0 }^{ +\infty } \!
        \frac{ l e^{ -2 l }\, dl }{ ( \gamma + l )
        \big[ ( \gamma + l )^{ 2 } - e^{ -2 l } \big] } \! + \!
        \frac{ 1 }{ \gamma } \!
        \int\limits_{ 0 }^{ +\infty } \! \frac{ ( 1 - l ) e^{ -2 l } }{
        ( \gamma + l )^{ 2 } - e^{ -2 l } } dl + R( a, \lambda ) \Bigg],
    \end{split}
  \end{equation}
  To analyse formula \eqref{} we used combination of analytical and numeric
  methods. With very high degree of probability we established that \eqref{}
  predicts repulsive force that very quickly vanish with the distance.
  This is in strict contrast with previous work of A. Sccardicchio, who
  using version of zero-point energy derive \alert{universal} and
  \alert{attractive} Casimir force in such system.


  % Function $\chi( g ) > 0$ was defined in equation \eqref{}. Parameter
  % $b_{ 1 }$ is give by explicit, but quite convoluted integrals of function
  % $g( \vecx )$ (notice independence of parameter $\lambda$) and
  % fulfill inequality $0 < b_{ 1 } \leq \alpha^{ -1 }$.

\end{frame}
% ##################





% ##################
\begin{frame}
  \frametitle{Computation of number of particles and energy}

  \vspace{-2em}


  \begin{equation}
    \label{eq:Energia-Casimira-dla-rodziny-ETC-09}
    \begin{split}
      &\EInt( a, \lambda ) = \\
      &=
        \frac{ 2 \alpha }{ \pi^{ 3 } } \! \Bigg[ \frac{ \chi }{ \lambda } \!
        \int\limits_{ 0 }^{ +\infty } \!
        \frac{ e^{ -2 l } \, dl }{ ( \gamma + l )
        \big[ ( \gamma + l )^{ 2 } - e^{ -2 l } \big] } \! + \!
        \frac{ b_{ \HorSpaceOne 1 } \chi }{ \gamma } \!
        \int\limits_{ 0 }^{ +\infty } \!
        \frac{ l^{ 2 } \big[ 3 ( \gamma + l )^{ 2 } e^{ -2 l } -
        e^{ -4 l } \big] }
        { ( \gamma + l )^{ 2 }
        \big[ ( \gamma + l )^{ 2 } - e^{ -2 l } \big]^{ 2 } } dl
        \, - \\[0.5em]
      &- \! \frac{ 2 }{ \gamma } \! \int\limits_{ 0 }^{ +\infty } \!
        \frac{ l e^{ -2 l }\, dl }{ ( \gamma + l )
        \big[ ( \gamma + l )^{ 2 } - e^{ -2 l } \big] } \! + \!
        \frac{ 1 }{ \gamma } \!
        \int\limits_{ 0 }^{ +\infty } \! \frac{ ( 1 - l ) e^{ -2 l } }{
        ( \gamma + l )^{ 2 } - e^{ -2 l } } dl + R( a, \lambda ) \Bigg],
    \end{split}
  \end{equation}
  The four integrands in equation above, from left two right, all taken
  with sign plus, we denoted as $I_{ 1 }$, $I_{ 2 }$, $I_{ 3 }$ and $I_{ 4 }$.
  At the next slides we plot results of numerical computations of Casimir
  energy for which we assumed:
  \begin{equation}
    \frac{ \chi }{ \lambda } = 10.0, \qquad
    \chi b_{ 1 } = 1.0.
  \end{equation}

\end{frame}
% ##################





% ##################
\begin{frame}
  \frametitle{Computation of number of particles and energy}


  \begin{figure}

    \label{fig:aaa}

    \centering


    \includegraphics[scale=0.525]
    {./Presentation-pictures/Terms\_of\_asymptotic\_expansion\_01.png}

    \caption{Plot of four integrands from equation \eqref{}, taken with
      plus sign}


  \end{figure}

\end{frame}
% ##################





% ##################
\begin{frame}
  \frametitle{Computation of number of particles and energy}


  \begin{figure}

    \label{fig:aaa}

    \centering


    \includegraphics[scale=0.525]
    {./Presentation-pictures/Casimir\_energy\_asymptotic\_expansion\_01.png}

    \caption{Plot of asymptotic expansion of Casimir energy}


  \end{figure}

\end{frame}
% ##################





% ##################
\begin{frame}
  \frametitle{Computation of number of particles and energy}


  \begin{figure}

    \label{fig:aaa}

    \centering


    \includegraphics[scale=0.525]
    {./Presentation-pictures/Casimir\_energy\_Scardicchio\_01.png}

    \caption{Casimir energy for two delta system computed by
      A.~Sccardicchio}


  \end{figure}

\end{frame}
% ##################





% ##################
\begin{frame}
  \frametitle{Computation of number of particles and energy}


  We also computed local energy density. Interaction part of it
  in the limit $\lambda \searrow 0$ is shown on the next slide (not the shortest
  formula that you can imagine). More precisely, it is a regular
  distribution on the set on the set $\Rbb^{ 3 } \setminus \{ -\veca / 2, \veca / 2\}$
  and this equation show this part of it.
  % \begin{equation}
  %   \label{eq:Lokalna-gestosc-energii-w-granicy-ETC-14} % OK
  %   \begin{split}
  %     \EIntDenLim
  %     &( \veca, \vecx \HorSpaceOne ) =
  %       \frac{ 1 }{ 8\pi^{ 2 } }
  %       \int_{ 0 }^{ +\infty } dl \,
  %       \frac{ e^{ -2 l } }{ ( \gamma + l ) [ ( \gamma + l )^{ 2 } - e^{ -2 l } ] } \, \times
  %     \\[0.5em]
  %     &\hspace{1em}
  %       \times \Bigg[ \frac{ e^{ -2 l \absTwo{ \vecx + \vecb } / a } }{
  %       \absTwo{ \vecx + \vecb }^{ 4 } }
  %       \left( 1 + 2 l \frac{ \absTwo{ \vecx + \vecb } }{ a } \right) +
  %       \frac{ e^{ -2 l \absTwo{ \vecx - \vecb } / a } }{
  %       \absTwo{ \vecx - \vecb }^{ 4 } } \left( 1 +
  %       2 l \frac{ \absTwo{ \vecx - \vecb } }{ a } \right) \Bigg] \, -
  %     \\[0.5em]
  %     &- \frac{ 1 }{ 4 \pi^{ 2 } } \int_{ 0 }^{ +\infty } dl \,
  %       \frac{ e^{ -l } }{ ( \gamma + l )^{ 2 } - e^{ -2 l } }
  %       \frac{ e^{ -l ( \absOne{ \vecx + \vecb } + |\,
  %       \vecx - \vecb\, | ) / a } }{ |\, \vecx + \vecb\, | \absTwo{ \vecx -
  %       \vecb } } \, \times \\[0.5em]
  %     &\hspace{3.5em}
  %       \times \Bigg[ \frac{ l^{ 2 } }{ a^{ 2 } } \Bigg( 1 - \frac{ (
  %       \vecx + \veca ) \cdot ( \vecx - \veca ) }{ |\, \vecx
  %       + \vecb\, | \absTwo{ \vecx - \vecb } } \Bigg) \, - \\[0.5em]
  %     &\hspace{5.25em}
  %       - \frac{ ( \vecx + \vecb \HorSpaceSix ) \cdot
  %       ( \vecx - \vecb \HorSpaceSix ) }{
  %       \absOne{ \vecx + \vecb } \HorSpaceFive
  %       \absOne{ \vecx - \vecb } }
  %       \frac{ 1 + l ( | \vecx
  %       + \vecb | + | \vecx - \vecb | ) / a }{ | \vecx +
  %       \vecb | \, \absOne{ \vecx - \vecb } } \Bigg].
  %   \end{split}
  % \end{equation}

  One should noted that local density of interaction energy is model
  \alert{independent}, which preclude computation of global energy by
  integration of it local density. We interpret it, roughly, as result of
  nontrivial part of energy belonging to it singular support
  $\{ -\veca / 2, \veca / 2 \}$.

\end{frame}
% ##################





% ##################
\begin{frame}
  \frametitle{Computation of number of particles and energy}

  \vspace{-1em}


  % We also computed local energy density. Interaction part of it
  % in the limit $\lambda \searrow 0$ is shown on the next slide (not the shortest
  % formula that you can imagine). More precisely, it is a regular
  % distribution on the set on the set $\Rbb^{ 3 } \setminus \{ -\veca / 2, \veca / 2\}$
  % and this equation show this part of it.
  \begin{equation}
    \label{eq:Lokalna-gestosc-energii-w-granicy-ETC-14} % OK
    \begin{split}
      \EIntDenLim
      &( \veca, \vecx \HorSpaceOne ) =
        \frac{ 1 }{ 8\pi^{ 2 } }
        \int_{ 0 }^{ +\infty } dl \,
        \frac{ e^{ -2 l } }{ ( \gamma + l ) [ ( \gamma + l )^{ 2 } - e^{ -2 l } ] } \, \times
      \\[0.5em]
      &\hspace{1em}
        \times \Bigg[ \frac{ e^{ -2 l \absTwo{ \vecx + \vecb } / a } }{
        \absTwo{ \vecx + \vecb }^{ 4 } }
        \left( 1 + 2 l \frac{ \absTwo{ \vecx + \vecb } }{ a } \right) +
        \frac{ e^{ -2 l \absTwo{ \vecx - \vecb } / a } }{
        \absTwo{ \vecx - \vecb }^{ 4 } } \left( 1 +
        2 l \frac{ \absTwo{ \vecx - \vecb } }{ a } \right) \Bigg] \, -
      \\[0.5em]
      &- \frac{ 1 }{ 4 \pi^{ 2 } } \int_{ 0 }^{ +\infty } dl \,
        \frac{ e^{ -l } }{ ( \gamma + l )^{ 2 } - e^{ -2 l } }
        \frac{ e^{ -l ( \absOne{ \vecx + \vecb } + |\,
        \vecx - \vecb\, | ) / a } }{ |\, \vecx + \vecb\, | \absTwo{ \vecx -
        \vecb } } \, \times \\[0.5em]
      &\hspace{3.5em}
        \times \Bigg[ \frac{ l^{ 2 } }{ a^{ 2 } } \Bigg( 1 - \frac{ (
        \vecx + \veca ) \cdot ( \vecx - \veca ) }{ |\, \vecx
        + \vecb\, | \absTwo{ \vecx - \vecb } } \Bigg) \, - \\[0.5em]
      &\hspace{5.25em}
        - \frac{ ( \vecx + \vecb \HorSpaceSix ) \cdot
        ( \vecx - \vecb \HorSpaceSix ) }{
        \absOne{ \vecx + \vecb } \HorSpaceFive
        \absOne{ \vecx - \vecb } }
        \frac{ 1 + l ( | \vecx
        + \vecb | + | \vecx - \vecb | ) / a }{ | \vecx +
        \vecb | \, \absOne{ \vecx - \vecb } } \Bigg].
    \end{split}
  \end{equation}

\end{frame}
% ##################





% ##################
\begin{frame}
  \frametitle{Computation of number of particles and energy}


  We should also note, that as a byproduct of our computation we find
  self-energy of single $\delta$, that is identical to that find by very
  different methods by L. Pizzochero and D. Fermi \parencite{}.

  This confirms what Herdegen and Stopa found before. Namely, that relation
  between Casimir energy and local energy density is not straightforward at
  all and computation need to be done with grate care. And that local energy
  density is more ``predicable'' than global one.

\end{frame}
% ##################





% ##################
\begin{frame}
  \frametitle{What next?}


  The most natural step now is to try compute Casimir effect for system
  of scalar field and single sphere. Basic analysis of it was already done
  in our master thesis, where main stumbling block was lack of prove
  that limit $\lambda \searrow 0$ could be taken. For fixed model at least numerical
  results should be in our reach, but we shouldn't count our chicken
  before they hatch.

\end{frame}
% ##################





% ##################
\begin{frame}
  \frametitle{References}




\end{frame}
% ##################



% % ##################
% \begin{frame}
%   \frametitle{Computation of number of particles and energy}


  % We also computed local energy density. Interaction part of it
  % in the limit $\lambda \searrow 0$ is shown on the next slide (not the shortest
  % formula that you can imagine). More precisely, it is a regular
  % distribution on the set on the set $\Rbb^{ 3 } \setminus \{ -\veca / 2, \veca / 2\}$
  % and this equation show this part of it.
  % \begin{equation}
  %   \label{eq:Lokalna-gestosc-energii-w-granicy-ETC-14} % OK
  %   \begin{split}
  %     \EIntDenLim
  %     &( \veca, \vecx \HorSpaceOne ) =
  %       \frac{ 1 }{ 8\pi^{ 2 } }
  %       \int_{ 0 }^{ +\infty } dl \,
  %       \frac{ e^{ -2 l } }{ ( \gamma + l ) [ ( \gamma + l )^{ 2 } - e^{ -2 l } ] } \, \times
  %     \\[0.5em]
  %     &\hspace{1em}
  %       \times \Bigg[ \frac{ e^{ -2 l \absTwo{ \vecx + \vecb } / a } }{
  %       \absTwo{ \vecx + \vecb }^{ 4 } }
  %       \left( 1 + 2 l \frac{ \absTwo{ \vecx + \vecb } }{ a } \right) +
  %       \frac{ e^{ -2 l \absTwo{ \vecx - \vecb } / a } }{
  %       \absTwo{ \vecx - \vecb }^{ 4 } } \left( 1 +
  %       2 l \frac{ \absTwo{ \vecx - \vecb } }{ a } \right) \Bigg] \, -
  %     \\[0.5em]
  %     &- \frac{ 1 }{ 4 \pi^{ 2 } } \int_{ 0 }^{ +\infty } dl \,
  %       \frac{ e^{ -l } }{ ( \gamma + l )^{ 2 } - e^{ -2 l } }
  %       \frac{ e^{ -l ( \absOne{ \vecx + \vecb } + |\,
  %       \vecx - \vecb\, | ) / a } }{ |\, \vecx + \vecb\, | \absTwo{ \vecx -
  %       \vecb } } \, \times \\[0.5em]
  %     &\hspace{3.5em}
  %       \times \Bigg[ \frac{ l^{ 2 } }{ a^{ 2 } } \Bigg( 1 - \frac{ (
  %       \vecx + \veca ) \cdot ( \vecx - \veca ) }{ |\, \vecx
  %       + \vecb\, | \absTwo{ \vecx - \vecb } } \Bigg) \, - \\[0.5em]
  %     &\hspace{5.25em}
  %       - \frac{ ( \vecx + \vecb \HorSpaceSix ) \cdot
  %       ( \vecx - \vecb \HorSpaceSix ) }{
  %       \absOne{ \vecx + \vecb } \HorSpaceFive
  %       \absOne{ \vecx - \vecb } }
  %       \frac{ 1 + l ( | \vecx
  %       + \vecb | + | \vecx - \vecb | ) / a }{ | \vecx +
  %       \vecb | \, \absOne{ \vecx - \vecb } } \Bigg].
  %   \end{split}
  % \end{equation}

% \end{frame}
% % ##################



% % ##################
% \begin{frame}
%   \frametitle{Twierdzenie o~przedłużaniu dystrybucji}


%   \begin{figure}

%     \label{aaa:bbb}


%     \centering

%     \begin{tikzpicture}


%       \fill[color=blue] (5.75,1.85) -- (5.75,3.15) -- (1.5,3.15) --
%       (1.5,1.85) -- cycle;

%       \fill[color=green] (1.5,4) -- (1,4) -- (1,1) -- (1.5,1) -- cycle;

%       \fill[color=brown] (1,3.75) -- (0.84,3.75) -- (0.84,1.25) --
%       (1,1.25) -- cycle;


%       \draw[color=black,line width=0.7] (1.7,3.15) -- (2,1.85);

%       \draw[color=black,line width=0.7] (2.3,3.15) -- (2.6,1.85);

%       \draw[color=black,line width=0.7] (2.9,3.15) -- (3.2,1.85);

%       \draw[color=black,line width=0.7] (3.5,3.15) -- (3.8,1.85);

%       \draw[color=black,line width=0.7] (4.1,3.15) -- (4.4,1.85);

%       \draw[color=black,line width=0.7] (4.7,3.15) -- (5,1.85);

%       \draw[color=black,line width=0.7] (5.3,3.15) -- (5.6,1.85);


%       \fill[color=gray] (-5,-0.5) -- (5,-0.5) -- (5,5) -- (4.5,5) --
%       (4.5,0) -- (-5,0) -- cycle;







%       \fill[color=brown] (-0.08,0) rectangle (0.08,2.5);

%       % % y axis
%       % \draw[axis arrow] (0,-1.3) -- (0,1.5);

%       % % x axis
%       % \draw[axis arrow] (-0.3,0) -- (10,0);


%       % \node[number on the right and below] at (0,0) {$0$};



%       % % Thicks on x axis
%       % \pic at (0.5,0) {tick x axis thin 1};

%       % \node[number below x axis] at (0.5,0) {$0.5$};


%       % \pic at (1,0) {tick x axis thin};

%       % \node[number below x axis] at (1,0) {$1$};


%       % \pic at (1.5,0) {tick x axis thin 1};

%       % \node[number below x axis] at (1.5,0) {$1.5$};


%       % \pic at (2,0) {tick x axis thin};

%       % \pic at (2.5,0) {tick x axis thin};

%       % \pic at (3,0) {tick x axis thin};


%       % % Ticks on y axis
%       % \pic at (0,-1) {tick y axis thin};

%       % \node[number on the left of y axis] at (0,-1) {$-1$};


%       % \pic at (0,-0.5) {tick y axis thin 1};

%       % \node[number on the left of y axis] at (0,-0.5) {$-0.5$};


%       % \pic at (0,0.5) {tick y axis thin 1};

%       % \pic at (0,1) {tick y axis thin};




%       % % Graph of sin function
%       % \draw[color=blue] (0,0) -- (0.05,0.049) -- (0.1,0.099) -- (0.15,0.149) --
%       % (0.2,0.198) -- (0.25,0.247) -- (0.3,0.295) -- (0.35,0.342) --
%       % (0.4,0.389) -- (0.45,0.434) -- (0.5,0.479) -- (0.55,0.522) --
%       % (0.6,0.564) -- (0.65,0.605) -- (0.7,0.644) -- (0.75,0.681) --
%       % (0.8,0.717) -- (0.85,0.751) -- (0.9,0.783) -- (0.95,0.813) --
%       % (1,0.841) -- (1.05,0.867) -- (1.1,0.891) -- (1.15,0.912) -- (1.2,0.932) --
%       % (1.25,0.948) -- (1.3,0.963) -- (1.35,0.975) -- (1.4,0.985) --
%       % (1.45,0.992) -- (1.5,0.997) -- (1.55,0.999) -- (1.6,0.999) --
%       % (1.65,0.996) -- (1.7,0.991) -- (1.75,0.983) -- (1.8,0.973) --
%       % (1.85,0.961) -- (1.9,0.946) -- (1.95,0.928) -- (2.0,0.909) --
%       % (2.05,0.887) -- (2.1,0.863) -- (2.15,0.836) -- (2.2,0.808) --
%       % (2.25,0.778) -- (2.3,0.745) -- (2.35,0.711) -- (2.4,0.675) --
%       % (2.45,0.637) -- (2.5,0.598) -- (2.55,0.557) -- (2.6,0.515) --
%       % (2.65,0.472) -- (2.7,0.427) -- (2.75,0.381) -- (2.8,0.334) --
%       % (2.85,0.287) -- (2.9,0.239) -- (2.95,0.19) -- (3,0.141) --
%       % (3.05,0.091) -- (3.1,0.041) -- (10,0);


%     \end{tikzpicture}

%     \caption{Funkcja $\sin$}

%   \end{figure}

% \end{frame}
% % ##################











% ##################
\begin{frame}
  \frametitle{The end}

  \vspace{7em}


  \begin{center}

    \Large

    Thank you. \\
    Are there any questions?

  \end{center}

\end{frame}
% ##################

























































% ####################################################################
% ####################################################################
% Bibliography

\printbibliography





% ############################
% End of the document

\end{document}
