% ------------------------------------------------------------------------------------------------------------------
% Basic configuration of Beamera class and Jagiellonian theme
% ------------------------------------------------------------------------------------------------------------------
\RequirePackage[l2tabu, orthodox]{nag}



\ifx\PresentationStyle\notset
  \def\PresentationStyle{dark}
\fi



% Options: t -- align frame text to the top
\documentclass[10pt,t]{beamer}
\mode<presentation>
\usetheme[style=\PresentationStyle]{jagiellonian}





% ------------------------------------------------------------------------------------
% Procesing configuration files of Jagiellonian theme located in
% the directory "preambule"
% ------------------------------------------------------------------------------------
% Configuration for polish language
% Need description
\usepackage[english]{babel}





% % ------------------------------
% % Better support of polish chars in technical parts of PDF
% % ------------------------------
% \hypersetup{pdfencoding=auto,psdextra}

% Package "textpos" give as enviroment "textblock" which is very usefull in
% arranging text on slides.

% This is standard configuration of "textpos"
\usepackage[overlay,absolute]{textpos}

% If you need to see bounds of "textblock's" comment line above and uncomment
% one below.

% Caution! When showboxes option is on significant ammunt of space is add
% to the top of textblock and as such, everyting put in them gone down.
% We need to check how to remove this bug.

% \usepackage[showboxes,overlay,absolute]{textpos}



% Setting scale length for package "textpos"
\setlength{\TPHorizModule}{10mm}
\setlength{\TPVertModule}{\TPHorizModule}


% ---------------------------------------
% TikZ
% ---------------------------------------
% Importing TikZ libraries
\usetikzlibrary{arrows.meta}
\usetikzlibrary{positioning}





% % Configuration package "bm" that need for making bold symbols
% \newcommand{\bmmax}{0}
% \newcommand{\hmmax}{0}
% \usepackage{bm}




% ---------------------------------------
% Packages for scientific texts
% ---------------------------------------
% \let\lll\undefined  % Sometimes you must use this line to allow
% "amsmath" package to works with packages with packages for polish
% languge imported
% /preambul/LanguageSettings/JagiellonianPolishLanguageSettings.tex.
% This comments (probably) removes polish letter Ł.
\usepackage{amsmath}  % Packages from American Mathematical Society (AMS)
\usepackage{amssymb}
\usepackage{amscd}
\usepackage{amsthm}
\usepackage{siunitx}  % Package for typsetting SI units.
\usepackage{upgreek}  % Better looking greek letters.
% Example of using upgreek: pi = \uppi


\usepackage{calrsfs}  % Zmienia czcionkę kaligraficzną w \mathcal
% na ładniejszą. Może w innych miejscach robi to samo, ale o tym nic
% nie wiem.










% ---------------------------------------
% Packages written for lectures "Geometria 3D dla twórców gier wideo"
% ---------------------------------------
% \usepackage{./ProgramowanieSymulacjiFizykiPaczki/ProgramowanieSymulacjiFizyki}
% \usepackage{./ProgramowanieSymulacjiFizykiPaczki/ProgramowanieSymulacjiFizykiIndeksy}
% \usepackage{./ProgramowanieSymulacjiFizykiPaczki/ProgramowanieSymulacjiFizykiTikZStyle}





% !!!!!!!!!!!!!!!!!!!!!!!!!!!!!!
% !!!!!!!!!!!!!!!!!!!!!!!!!!!!!!
% EVIL STUFF
\if\JUlogotitle1
\edef\LogoJUPath{LogoJU_\JUlogoLang/LogoJU_\JUlogoShape_\JUlogoColor.pdf}
\titlegraphic{\hfill\includegraphics[scale=0.22]
{./JagiellonianPictures/\LogoJUPath}}
\fi
% ---------------------------------------
% Commands for handling colors
% ---------------------------------------


% Command for setting normal text color for some text in math modestyle
% Text color depend on used style of Jagiellonian

% Beamer version of command
\newcommand{\TextWithNormalTextColor}[1]{%
  {\color{jNormalTextFGColor}
    \setbeamercolor{math text}{fg=jNormalTextFGColor} {#1}}
}

% Article and similar classes version of command
% \newcommand{\TextWithNormalTextColor}[1]{%
%   {\color{jNormalTextsFGColor} {#1}}
% }



% Beamer version of command
\newcommand{\NormalTextInMathMode}[1]{%
  {\color{jNormalTextFGColor}
    \setbeamercolor{math text}{fg=jNormalTextFGColor} \text{#1}}
}


% Article and similar classes version of command
% \newcommand{\NormalTextInMathMode}[1]{%
%   {\color{jNormalTextsFGColor} \text{#1}}
% }




% Command that sets color of some mathematical text to the same color
% that has normal text in header (?)

% Beamer version of the command
\newcommand{\MathTextFrametitleFGColor}[1]{%
  {\color{jFrametitleFGColor}
    \setbeamercolor{math text}{fg=jFrametitleFGColor} #1}
}

% Article and similar classes version of the command
% \newcommand{\MathTextWhiteColor}[1]{{\color{jFrametitleFGColor} #1}}





% Command for setting color of alert text for some text in math modestyle

% Beamer version of the command
\newcommand{\MathTextAlertColor}[1]{%
  {\color{jOrange} \setbeamercolor{math text}{fg=jOrange} #1}
}

% Article and similar classes version of the command
% \newcommand{\MathTextAlertColor}[1]{{\color{jOrange} #1}}





% Command that allow you to sets chosen color as the color of some text into
% math mode. Due to some nuances in the way that Beamer handle colors
% it not work in all cases. We hope that in the future we will improve it.

% Beamer version of the command
\newcommand{\SetMathTextsColor}[2]{%
  {\color{#1} \setbeamercolor{math text}{fg=#1} #2}
}


% Article and similar classes version of the command
% \newcommand{\SetMathTextColor}[2]{{\color{#1} #2}}










% ---------------------------------------
% Commands for setting background pictures for some slides
% ---------------------------------------
\newcommand{\TitleBackgroundPicture}
{./PresentationPictures/CommonPictures/Cute_dragon_BG_dark.png}
\newcommand{\SectionBackgroundPicture}
{./PresentationPictures/CommonPictures/Cute_dragon_small_BG_light.png}



\newcommand{\TitleSlideWithPicture}{
  \begingroup

  \usebackgroundtemplate{ % \hspace*{-11.5em}
    \includegraphics[height=\paperheight]{\TitleBackgroundPicture}}

  \maketitle

  \endgroup
}





\newcommand{\SectionSlideWithPicture}[1]{%
  \begingroup

  \usebackgroundtemplate{ % \hspace*{-11.5em}
    \includegraphics[height=\paperheight]{\SectionBackgroundPicture}}

  \setbeamercolor{titlelike}{fg=normal text.fg}

  \section{#1}

  \endgroup
}





\newcommand{\EndingSlide}[1]{%
  \begin{frame}[standout]

    \begingroup

    \color{jFrametitleFGColor}

    #1

    \endgroup

  \end{frame}
}










% ------------------------------------------------------------------------------------
% Importing packages, libraries and setting their configuration
% ------------------------------------------------------------------------------------

% ------------------------------------------------------
% Packages for scientific papers
% ------------------------------------------------------
% Switching off \lll symbol, that I guess is representing letter ``Ł''.
% It collide with `amsmath' package's command with the same name
% \let\lll\undefined
% Basic package from American Mathematical Society (AMS)
% \usepackage[intlimits]{amsmath}
% Equations are numbered separately in every section.
% \numberwithin{equation}{section}





% ------------------------------------------------------
% BibLaTeX
% ------------------------------------------------------
% Package biblatex, with biber as its backend, allow us to handle
% bibliography entries that use Unicode symbols outside ASCII.
\usepackage[
language=polish,
backend=biber,
style=alphabetic,
url=false,
eprint=true,
]{biblatex}

\addbibresource{Herdegens-approach-and-two-quasi-delta-ETC-Bibliography.bib}





% ------------------------------------------------------
% Wonderful package PGF/TikZ
% ------------------------------------------------------

% Styles for arrows
\usepackage{./Local-packages/PGF-TikZ-Arrows-styles}

% Node and pics for drawing charts
\usepackage{./Local-packages/PGF-TikZ-Chart-nodes-and-pics}





% ------------------------------------------------------
% Local packages
% ------------------------------------------------------
% Special configuration for this particular presentation
\usepackage{./Local-packages/local-settings}

% Package containing various command useful for working with a text
\usepackage{./Local-packages/general-commands}

% Package containing commands and other code useful for working with
% mathematical text
\usepackage{./Local-packages/math-commands}










% ------------------------------------------------------------------------------------------------------------------
\title{Heredegen's approach to Casimir effect in application the
  two quasi-delta system}

\author{Kamil Ziemian \\
  \texttt{kziemianfvt@gmail.com}}


% \institute{Uniwersytet Jagielloński w~Krakowie}

\date[13 X 2024]{Seminary of Field Theory Departament,
  13 October 2024}
% ------------------------------------------------------------------------------------------------------------------










% ####################################################################
% Beginning of the document
\begin{document}
% ####################################################################





% ######################################
% Text is adjusted to the left and words are broken at the end of the line.
% Number of chars: 62k+, 73k+, 25k+,
\RaggedRight
% ######################################





% ######################################
\maketitle
% ######################################





% ######################################
\begin{frame}
  \frametitle{Table of contents}


  \tableofcontents % Spis treści

\end{frame}
% ######################################










% % ######################################
% \section{General information about the talk}
% % ######################################


% ##################
\begin{frame}
  \frametitle{General information about the talk}


  If you have any question about the topic, please interrupt me during
  the~talk. I~will be also grateful for suggestion how this talk can be
  improved and for pointing any mistakes found in it.

\end{frame}
% ##################










% ######################################
\section{Short history of Casimir effect}
% ######################################



% ##################
\begin{frame}
  \frametitle{Casimir paper from 1948}


  In the year 1948 Hendrik Brught Gerhard Casimir publish only three pages
  long article \textit{On~the~Attraction Between Two Perfectly Conducting
    Plates} \parencite{Casimir-On-the-Attraction-Between-ETC-Pub-1948}, in which
  he predicted that two electrical neutral plates put inside the cubic cavity
  will attract each other due to strange relation between ``zero-point
  energy'' of the~electromagnetic field with different boundary conditions.

  As it is well known, see for example book by Peskin and Schroeder
  \parencite{Peskin-Schroeder-An-Introduction-to-Quantum-Field-Theory-Pub-1995},
  procedure known as canonical quantisation applied to classical
  electromagnetic field, produce ``infinite constant'' in the~expression
  for energy, which his refer to as ``zero-point energy''. Casimir claims
  was that since boundary conditions change ``value of infinite constant'',
  their difference should manifest itself as finite force.

\end{frame}
% ##################





% ##################
\begin{frame}
  \frametitle{Zero-point energies used by Casimir}


  Casimir considered zero-point energy of empty cavity, that is expressed
  by formula:
  \begin{equation}
    \label{eq:Herdegens-approach-01}
    % \begin{split}
    \frac{ 1 }{ 2 } \left( \sum \hbar \omega \right)_{ \text{II} } =
    \frac{ \hbar c a \HorSpaceFive L^{ 2 } }{ 2 \pi^{ 2 } }
    \int\limits_{ 0 }^{ \infty } \int\limits_{ 0 }^{ \infty } \kappa
    \sqrt{ k_{ \HorSpaceOne z }^{ \,\, 2 } + \kappa^{ 2 } } \, d\kappa \,
    dk_{ \HorSpaceTwo z }.
    % \end{split}
  \end{equation}
  Notation used here is taken from original paper.

  On the other hand, zero-point energy in the situation when additional
  plate, that produce Dirichlet boundary conditions, is inserted at the
  distance $a$ is given by:
  \begin{equation}
    \label{eq:Herdegens-approach-02}
    \frac{ 1 }{ 2 } \left( \sum \hbar \omega \right)_{ \text{I} } =
    \frac{ \hbar c \HorSpaceFive L^{ 2 } }{ 2 \pi }
    \sum_{ n = ( 0 ), \, 1 }^{ \infty } \, \int\limits_{ 0 }^{ \infty } \kappa
    \sqrt{ \left( n^{ 2 } \frac{ \pi^{ 2 } }{ a^{ 2 } } \right)
      + \kappa^{ 2 } } \, d\kappa.
  \end{equation}

\end{frame}
% ##################





% ##################
\begin{frame}
  \frametitle{Introduction of cutoff}


  To obtain finite result from
  $\delta E = \frac{ 1 }{ 2 } ( \sum \hbar \omega )_{ \text{I} } -
  \frac{ 1 }{ 2 } ( \sum \hbar \omega )_{ \text{II} }$ Casimir need to introduce
  cutoff function $f$, with cutoff parameter $k_{ \HorSpaceThree m }$.
  After doing that regularized energy difference of zero-point energies is
  given by
  \begin{equation}
    \label{eq:Oryginalna-praca-ETC-05}
    \begin{split}
      \delta E
      &=
        \frac{ \hbar c \HorSpaceFive L^{ 2 } \pi^{ 2 } }{ 4 a^{ 3 } } \!
        \left( \, \sum_{ ( 0 ), \, 1 }^{ \infty } \,
        \int\limits_{ 0 }^{ \infty } \! d\kappa \, \sqrt{ n^{ 2 } \! + \! u } \,
        f\!\left( \tfrac{ \pi \sqrt{ n^{ 2 } + u } }{ a k_{ \HorSpaceTwo m } }
        \right) \right. \, - \\
      &\left. \;\;\;
        - \int\limits_{ 0 }^{ \infty } \! \int\limits_{ 0 }^{ \infty } \! du \, dn \,
        \sqrt{ n^{ \, 2 } \! + \! u } \,
        f\!\left( \tfrac{ \pi \sqrt{ n^{ 2 } + u } }{
        a k_{ \HorSpaceTwo m } } \right) \right).
    \end{split}
  \end{equation}

  By expanding this expression into a series and taking only zero order term
  Casimir arrives at now famous formula:
  \begin{equation}
    \label{eq:Oryginalna-praca-ETC-06}
    \frac{ \delta E }{ L^{ 2 } } =
    -\hbar c \frac{ \pi^{ 2 } }{ 720 } \frac{ 1 }{ a^{ 3 } }.
  \end{equation}

\end{frame}
% ##################





% ##################
\begin{frame}
  \frametitle{Review of Casimir approach}


  Casimir method of computation is still present in current literature in
  more or less refined form. In my humble opinion, it cannot be judge as
  satisfactory in $2024$.

  First of all, it use of zero-point energy, which according to
  original~(!) Casimir paper is itself nonphysical, is hard to explain.
  It worth noticing that for over $700$ pages of Peskin and Schroeder book,
  zero-point energy is considered as physically irrelevant, to suddenly be
  considered physically important in the context of Casimir effect.
  The authors seems to be dissatisfied by this state of affair.

  Second, it not introduce physical properties of materials at the beginning,
  but as auxiliary tool for removing ill defined mathematical quantities.

  For both these reasons, alternative approaches should be welcome.

\end{frame}
% ##################










% ######################################
\section{Overview of Herdegen's approach to Casimir effect}
% ######################################



% ##################
\begin{frame}
  \frametitle{Introduction to Herdegen's approach}


  Approach to Casimir force that we will be discussing, was first announced
  in short paper of Andrzej Herdegen \textit{No-nonsens Casimir force}
  \parencite{Herdegen-Nononsens-Casimir-force-Pub-2001} and fully presented
  in the two part article \textit{Quantum backreaction (Casimir) effect}
  \parencite{Herdegen-Quantum-backreaction-ETC-Part-I-Pub-2005},
  \parencite{Herdegen-Quantum-backreaction-ETC-Part-II-Pub-2006}. Casimir
  effect in this approach is interpreted as backreaction of quantum system
  to adiabatic changes in macroscopic system and apparatus of algebraic
  quantum field theory (\textsc{aqft}) is used to allow us for rigorous
  analysis of physical properties of the system.

  We need to put special attention to how Casimir effect is interpreted
  in this formalism. According to it, this effect is present in the system
  having quantum part $Q$ and macroscopic part $M$. In the standard example
  quantum part is quantum electromagnetic field and macroscopic contains
  two meta, electric neutral plates. When $M$ change \alert{adiabaticaly}
  there is backreaction on the part of subsystem $Q$, that give rise to
  particular force.

\end{frame}
% ##################





% ##################
\begin{frame}
  \frametitle{Element of Herdegen's approach}


  According to my reading of experimental literature on the subject,
  this assumption of adiabatic movement of macroscopic subsystem fits
  well with the way how true experiments are done.

  We now want to give short summary of algebraic quantum field theory.
  The~two main pillars of it can be summarized in somewhat simplified way as
  follow.
  \begin{itemize}

  \item For quantum observables the most important relations are of
    algebraic nature.

  \item Physical measurements are localized in bounded regions in of
    spacetime.

  \end{itemize}
  Due to time constrains we can only sketch how this quite sophisticated
  formalism works. We first try to explain, what is the idea laying behind
  the~first point.

\end{frame}
% ##################






% ##################
\begin{frame}
  \frametitle{Algebraic nature of observables}


  In standard curse of non-relativistic quantum mechanics, assuming
  one dimensional problem, the position and
  momentum operators are given as,
  \begin{equation}
    \label{eq:Overview-of-Herdegens-approach-ETC-01}
    X = x, \quad
    P = -i \, \hbar \frac{ d }{ dx },
  \end{equation}
  and both of this operator act on space of physical state that is equal
  to $L^{ 2 }( \Rbb, d\mu( x ) )$. From this we can, glossing over some
  mathematical problems, derive relation
  \begin{equation}
    \label{eq:Overview-of-Herdegens-approach-ETC-02}
    [ X, P ] = i \, \hbar \, \id.
  \end{equation}
  In \textsc{aqft} we interpreted equation
  \eqref{eq:Overview-of-Herdegens-approach-ETC-02} as being fundamental
  relation between observables of algebraic nature. More precisely, given
  physical system, we attach to it $C^{ * }$ algebra $\Acal$, representing
  physical quantities that can be measured inside it.

\end{frame}
% ##################





% ##################
\begin{frame}
  \frametitle{Algebraic nature of observables}


  For technical reasons, we replace relation $[ X, P ] = i \, \hbar \, \id$
  with it Weyl form:
  \begin{equation}
    \label{eq:Overview-of-Herdegens-approach-ETC-03}
    e^{ i P s } e^{ i X t } = e^{ i X t } e^{ i P s } e^{ i s t }.
  \end{equation}

  We can assume that our algebra $\Acal$ have a~neutral element $\UnitAlg$.
  While element of algebra $\Acal$ represents measurable physical
  quantities, the states are represented by class of the elements of space
  dual to $\Acal$: $\omega \in \Acal^{ \text{d} }$. Such $\omega$ to be considered state,
  must fulfill two requirements:
  \begin{equation}
    \label{eq:Overview-of-Herdegens-approach-ETC-04}
    \omega( A^{ * } A ) \geq 0, \; \forall A \in \Acal, \quad
    \omega( \UnitAlg ) = 1.
  \end{equation}

  We can now talk about localization of physical event in the Minkowski
  spacetime.

\end{frame}
% ##################





% ##################
\begin{frame}
  \frametitle{Picture of Minkowski spacetime}


  \begin{figure}

    \centering


    \begin{tikzpicture}

      % x axis
      \draw[axis arrow] (-5,0) -- (5,0);

      \pic at (5,0) {x mark for horizontal axis 1};


      % t axis
      \draw[axis arrow] (0,-2.5) -- (0,4);

      \pic at (0,4) {t mark for vertical axis 1};



      % Light cone
      \fill[color=blue,opacity=0.4] (0,0) -- (3.5,3.5) -- (-3.5,3.5) -- cycle;

      \fill[color=blue,opacity=0.4] (0,0) -- (2.2,-2.2) -- (-2.2,-2.2) --
      cycle;

      \draw[dashed] (2.2,-2.2) -- (-3.5,3.5);

      \draw[dashed] (-2.2,-2.2) -- (3.5,3.5);



      % Region number 1
      \draw[dashed] plot[smooth cycle]
      coordinates { (1.2,0) (3,-1) (3.5,1.4) (2.5,1.2) };

      \node[scale=1.3] at (2.3,0.3) {$\Ocal_{ 1 }$};


      % Region number 2
      \draw[dashed] plot[smooth cycle]
      coordinates { (0.8,0) (2.9,-1.5) (4.2,1.8) (1,3) };

      \node[scale=1.3] at (2.7,1.8) {$\Ocal_{ 2 }$};


      % Region number 3
      \draw[dashed] plot[smooth cycle]
      coordinates { (-1,-0.2) (-2,1.2) (-3,0.3) (-2,-1) };

      \node[scale=1.3] at (-2,0.5) {$\Ocal_{ 3 }$};

    \end{tikzpicture}

    \caption{Picture of spacetime with few bounded regions marked.}


  \end{figure}

\end{frame}
% ##################





% ##################
\begin{frame}
  \frametitle{Localization of algebras}


  Previously we will just talking about algebras, but in \textsc{aqft}
  every algebra is localized in particular open region of spacetime
  with compact closure. We will denote algebra of region $\Ocal_{ 1 }$
  as $\Acal( \Ocal_{ 1 } )$. This algebra represents all quantities
  that can be measured in such region. You can imagine that it represent
  size of experimental aperture and span of time when it's working.

  Family of all such localized algebras we will call \textbf{net of
    algebras}. Since regions from previous picture obeys relation
  $\Ocal_{ 1 } \subset \Ocal_{ 2 }$ we require that
  \begin{equation}
    \label{eq:Overview-of-Herdegens-approach-ETC-05}
    \Acal( \Ocal_{ 1 } ) \subset \Acal( \Ocal_{ 2 } ),
  \end{equation}
  which have natural interpretation. It simple says that every quantity
  that can be measured in particular region is also measurable in any
  region that extend the first one.

\end{frame}
% ##################





% ##################
\begin{frame}
  \frametitle{Causality and problems with construction}


  Analogously, since there is not causal link between $\Ocal_{ 1 }$ and
  $\Ocal_{ 3 }$, we require that measurements of all physical quantities
  are independent. Algebraically this mean that
  \begin{equation}
    \label{eq:Overview-of-Herdegens-approach-ETC-06}
    [ A, B ] = 0, \quad
    \forall A \in \Acal( \Ocal_{ 1 } ), B \in \Acal( \Ocal_{ 3 } ).
  \end{equation}
  This and other requirements that we won't mention, that nets of algebras
  need to obey, should be viewed as axioms in the spirit of axiomatic
  quantum field theory.

  Construction of model fulfilling any such scheme is notoriously hard, see
  \parencite{Summers-Prespective-on-Constructive-ETC-Ver-2016} for the
  account. According to my knowledge at this moment no
  model of quantum field theory in $1 + 3$ dimension interacting with
  itself or other quantum field was explicit constructed. Fortunately,
  theory of quantum field weakly interacting with classical object
  were constructed and are general enough to allow us to handle Casimir
  effect in variety of cases.

\end{frame}
% ##################





% ##################
\begin{frame}
  \frametitle{AQFT for Casimir effect}


  Herdegen's paper
  \parencite{Herdegen-Quantum-backreaction-ETC-Part-I-Pub-2005}
  constrains set of axioms tailored for discussion of Casimir effect.
  Here we can only mention few of them.

  We start from assumption that quantum subsystem $Q$ is defined by its
  algebra $\Acal$ and representation $\pi$ of this algebra of bounded
  operators on Hilbert space $\Hcal$.
  \begin{equation}
    \label{eq:Overview-of-Herdegens-approach-ETC-07}
    \Acal \ni A \mapsto \pi( A ) \in \Bcal( \Acal ).
  \end{equation}
  Algebra $\Acal$ is localized on hyperplane $t = \const$, which due to
  assumption of adiabatic changes inside system, pose no problem to
  the~formalism.

  Time evolution on algebra $\Acal$ is given by family of automorphism
  $\alpha_{ \HorSpaceThree t }$ and translate to unitary evolution on $\Hcal$:
  \begin{equation}
    \label{eq:Overview-of-Herdegens-approach-ETC-07}
    \pi( \alpha_{ \HorSpaceThree t } \, A ) \mapsto U( t ) \pi( A ) U^{ * }( t ), \quad
    U( t ) = \exp( i t H ).
  \end{equation}
  This relation define hamiltonian operator of the free quantum system.

\end{frame}
% ##################





% ##################
\begin{frame}
  \frametitle{????}


  \begin{subequations}
    \begin{align}
      \label{eq:Epstein-Glaser-24-A}
      &A_{ \Gamma }( g _{ 1 } + g_{ 2 } ) =
        A_{ \Gamma_{ 1 } }( g _{ 1 } ) A_{ \Gamma_{ 2 } }( g_{ 2 } ), \\
      \label{eq:Epstein-Glaser-24-B}
      &\dot{ A }_{ \Gamma }' =
        \dot{ A }_{ \Gamma }
        + \sum\limits_{ \alpha \leq \textrm{div}( u ) } C_{ \alpha } \delta^{ ( \alpha ) }, \quad
        \sd( A_{ \Gamma } ) > d.
    \end{align}
  \end{subequations}

  \textbf{Dwa wierzchołki}
  \begin{equation}
    \label{eq:Epstein-Glaser-25}
    A_{ \Gamma }^{ \mu,\, \xi } =
    \Delta_{ F }^{ m,\, \mu,\, \xi }( x - y, y - x ) \Delta_{ F }^{ m,\, \mu,\, \xi }( x - y, y - x ),
  \end{equation}
  dystrybucja ta~jest określona
  na~$\Dcal( \Rbb^{ 2 d } \setminus \{ ( x, x ) \} )$. Używając
  niezmienniczości translacyjnej problemu dostajemy dystrybucję
  $\widetilde{ A }_{ \Gamma }^{ \mu,\, \xi }$ określoną
  na~$\Dcal( \Rbb^{ d } \setminus \{ 0 \} )$.

  \textbf{Renormalizacja}
  \begin{equation}
    \label{eq:Epstein-Glaser-26}
    \widetilde{ A }_{ \Gamma }^{ \mu } =
    \lim\limits_{ \xi \to 0 } ( \Delta_{ F }^{ m,\, \mu,\, \xi } )^{ 2 } ( 2 ( x - y ), 0 )
    + \sum_{ \alpha } C_{ \alpha } \delta^{ ( \alpha ) }.
  \end{equation}

\end{frame}
% ##################





% % ##################
% \begin{frame}
%   \frametitle{Procedura indukcyjna Epsteina-Glasera}


%   \textbf{Warunek przyczynowości}
%   \begin{subequations}
%     \begin{align}
%       \label{eq:Epstein-Glaser-27-A}
%       &A_{ \Gamma }( g _{ 1 } + g_{ 2 } ) =
%         A_{ \Gamma_{ 1 } }( g _{ 1 } ) A_{ \Gamma_{ 2 } }( g_{ 2 } ), \\
%       \label{eq:Epstein-Glaser-27-B}
%       &\dot{ A }_{ \Gamma }' =
%         \dot{ A }_{ \Gamma }
%         + \sum\limits_{ \alpha \leq \textrm{div}( u ) } C_{ \alpha } \delta^{ ( \alpha ) }, \quad
%         \sd( A_{ \Gamma } ) > d.
%     \end{align}
%   \end{subequations}


%   \textbf{Trzy wierzchołki.}
%   Umiemy już zetknąć dwa wierzchołki, niezdefiniowana jest operacja na
%   trzech. Jednak warunek przyczynowości którym rozbijamy graf na
%   iloczyn dwu- i~jednowierzchołkowego, prowadzi to do redukcji
%   dystrybucji na $\Dcal( \Rbb^{ 3 d } \setminus \{ ( x, x, x ) \} )$ do
%   dystrybucji na $\Dcal( \Rbb^{ 2 d } \setminus \{ 0 \} )$.

% \end{frame}
% % ##################





% % ##################
% \begin{frame}
%   \frametitle{Procedura indukcyjna Epsteina-Glasera}


%   \begin{subequations}
%     \begin{align}
%       \label{eq:Epstein-Glaser-28-A}
%       &A_{ \Gamma }( g _{ 1 } + g_{ 2 } ) =
%         A_{ \Gamma_{ 1 } }( g _{ 1 } ) A_{ \Gamma_{ 2 } }( g_{ 2 } ), \\
%       \label{eq:Epstein-Glaser-28-B}
%       &\dot{ A }_{ \Gamma }' =
%         \dot{ A }_{ \Gamma }
%         + \sum\limits_{ \alpha \leq \textrm{div}( u ) } C_{ \alpha } \delta^{ ( \alpha ) }, \quad
%         \sd( A_{ \Gamma } ) > d.
%     \end{align}
%   \end{subequations}

%   \textbf{Scaling expansion (Hollands, Wald)} \\
%   \begin{equation}
%     \label{eq:Epstein-Glaser-29}
%     \prod_{ l = 1 }^{ k } \Delta_{ F }^{ m,\, \mu,\, \xi }( x_{ i }, y_{ i } ) =
%     \sum_{ s_{ k } = 0 }^{ \infty } ( m^{ 2 } )^{ s_{ k } }
%     \Delta_{ F }^{ \vec{ s }_{ k }, \mu, \xi }( x_{ 1 }, y_{ 1 }, \ldots,
%     x_{ k }, y_{ k } ),
%   \end{equation}
%   przy czym,
%   \begin{equation}
%     \label{eq:Epstein-Glaser-30}
%     \sd( \Delta_{ F }^{ \vec{ s }_{ k },\, \mu,\, \xi } ) =
%     d + \Real( \xi ) - 2 - 2 s_{ k }.
%   \end{equation}

%   \textbf{Grupa renormalizacji} \\
%   Swoboda dobierania stałych $C_{ \alpha }$ przy
%   $\delta^{ ( \alpha ) }$ jest opisana grupą renormalizacji.

% \end{frame}
% % ##################


% % ##################
% \begin{frame}
%   \frametitle{Twierdzenie o~przedłużaniu dystrybucji}


%   \begin{figure}

%     \label{aaa:bbb}


%     \centering

%     \begin{tikzpicture}


%       \fill[color=blue] (5.75,1.85) -- (5.75,3.15) -- (1.5,3.15) --
%       (1.5,1.85) -- cycle;

%       \fill[color=green] (1.5,4) -- (1,4) -- (1,1) -- (1.5,1) -- cycle;

%       \fill[color=brown] (1,3.75) -- (0.84,3.75) -- (0.84,1.25) --
%       (1,1.25) -- cycle;


%       \draw[color=black,line width=0.7] (1.7,3.15) -- (2,1.85);

%       \draw[color=black,line width=0.7] (2.3,3.15) -- (2.6,1.85);

%       \draw[color=black,line width=0.7] (2.9,3.15) -- (3.2,1.85);

%       \draw[color=black,line width=0.7] (3.5,3.15) -- (3.8,1.85);

%       \draw[color=black,line width=0.7] (4.1,3.15) -- (4.4,1.85);

%       \draw[color=black,line width=0.7] (4.7,3.15) -- (5,1.85);

%       \draw[color=black,line width=0.7] (5.3,3.15) -- (5.6,1.85);


%       \fill[color=gray] (-5,-0.5) -- (5,-0.5) -- (5,5) -- (4.5,5) --
%       (4.5,0) -- (-5,0) -- cycle;







%       \fill[color=brown] (-0.08,0) rectangle (0.08,2.5);

%       % % y axis
%       % \draw[axis arrow] (0,-1.3) -- (0,1.5);

%       % % x axis
%       % \draw[axis arrow] (-0.3,0) -- (10,0);


%       % \node[number on the right and below] at (0,0) {$0$};



%       % % Thicks on x axis
%       % \pic at (0.5,0) {tick x axis thin 1};

%       % \node[number below x axis] at (0.5,0) {$0.5$};


%       % \pic at (1,0) {tick x axis thin};

%       % \node[number below x axis] at (1,0) {$1$};


%       % \pic at (1.5,0) {tick x axis thin 1};

%       % \node[number below x axis] at (1.5,0) {$1.5$};


%       % \pic at (2,0) {tick x axis thin};

%       % \pic at (2.5,0) {tick x axis thin};

%       % \pic at (3,0) {tick x axis thin};


%       % % Ticks on y axis
%       % \pic at (0,-1) {tick y axis thin};

%       % \node[number on the left of y axis] at (0,-1) {$-1$};


%       % \pic at (0,-0.5) {tick y axis thin 1};

%       % \node[number on the left of y axis] at (0,-0.5) {$-0.5$};


%       % \pic at (0,0.5) {tick y axis thin 1};

%       % \pic at (0,1) {tick y axis thin};




%       % % Graph of sin function
%       % \draw[color=blue] (0,0) -- (0.05,0.049) -- (0.1,0.099) -- (0.15,0.149) --
%       % (0.2,0.198) -- (0.25,0.247) -- (0.3,0.295) -- (0.35,0.342) --
%       % (0.4,0.389) -- (0.45,0.434) -- (0.5,0.479) -- (0.55,0.522) --
%       % (0.6,0.564) -- (0.65,0.605) -- (0.7,0.644) -- (0.75,0.681) --
%       % (0.8,0.717) -- (0.85,0.751) -- (0.9,0.783) -- (0.95,0.813) --
%       % (1,0.841) -- (1.05,0.867) -- (1.1,0.891) -- (1.15,0.912) -- (1.2,0.932) --
%       % (1.25,0.948) -- (1.3,0.963) -- (1.35,0.975) -- (1.4,0.985) --
%       % (1.45,0.992) -- (1.5,0.997) -- (1.55,0.999) -- (1.6,0.999) --
%       % (1.65,0.996) -- (1.7,0.991) -- (1.75,0.983) -- (1.8,0.973) --
%       % (1.85,0.961) -- (1.9,0.946) -- (1.95,0.928) -- (2.0,0.909) --
%       % (2.05,0.887) -- (2.1,0.863) -- (2.15,0.836) -- (2.2,0.808) --
%       % (2.25,0.778) -- (2.3,0.745) -- (2.35,0.711) -- (2.4,0.675) --
%       % (2.45,0.637) -- (2.5,0.598) -- (2.55,0.557) -- (2.6,0.515) --
%       % (2.65,0.472) -- (2.7,0.427) -- (2.75,0.381) -- (2.8,0.334) --
%       % (2.85,0.287) -- (2.9,0.239) -- (2.95,0.19) -- (3,0.141) --
%       % (3.05,0.091) -- (3.1,0.041) -- (10,0);


%     \end{tikzpicture}

%     \caption{Funkcja $\sin$}

%   \end{figure}

% \end{frame}
% % ##################
































% % ######################################
% \section{Regularyzacja wymiarowa w~przestrzeni położeń}
% % ######################################













% % ######################################
% \section{Procedura indukcyjna Epsteina-Glasera}
% % ######################################









% % ##################
% \begin{frame}
%   \frametitle{Procedura indukcyjna Epsteina-Glasera}


%   Od~tej chwili interesujące nas grafy $\Gamma$ zawierają tylko wierzchołki
%   i~linie je łączące.

%   \textbf{Warunek przyczynowości.}
%   Jeżeli wielkość zależna od~grafu
%   $A_{ \Gamma }( g _{ 1 } + g_{ 2 } )$, zależy od~funkcji takich,
%   że~nośniki $g_{ i }$~są przyczynowo odseparowane, to~musi zachodzić:
%   \begin{equation}
%     \label{eq:Epstein-Glaser-23}
%     A_{ \Gamma }( g _{ 1 } + g_{ 2 } ) =
%     A_{ \Gamma_{ 1 } }( g _{ 1 } ) A_{ \Gamma_{ 2 } }( g_{ 2 } ),
%   \end{equation}
%   jako mnożenie liczb.

% \end{frame}
% % ##################






















% ####################################################################
% ####################################################################
% Bibliography

\printbibliography





% ############################
% End of the document

\end{document}
