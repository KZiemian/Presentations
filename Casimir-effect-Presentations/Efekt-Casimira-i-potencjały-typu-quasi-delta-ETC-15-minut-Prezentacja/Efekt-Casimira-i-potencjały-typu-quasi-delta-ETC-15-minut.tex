% ------------------------------------------------------------------------------------------------------------------
% Basic configuration of Beamera class and Jagiellonian theme
% ------------------------------------------------------------------------------------------------------------------
\RequirePackage[l2tabu, orthodox]{nag}



\ifx\PresentationStyle\notset
  \def\PresentationStyle{light}
\fi



% Options: t -- align frame text to the top
\documentclass[10pt,t]{beamer}
\mode<presentation>
\usetheme[style=\PresentationStyle]{jagiellonian}





% ------------------------------------------------------------------------------------
% Procesing configuration files of Jagiellonian theme located in
% the directory "preambule"
% ------------------------------------------------------------------------------------
% Configuration for polish language
% Need description
\usepackage[polish]{babel}
% Need description
\usepackage[MeX]{polski}



% ------------------------------
% Better support of polish chars in technical parts of PDF
% ------------------------------
\hypersetup{pdfencoding=auto,psdextra}

% Package "textpos" give as enviroment "textblock" which is very usefull in
% arranging text on slides.

% This is standard configuration of "textpos"
\usepackage[overlay,absolute]{textpos}

% If you need to see bounds of "textblock's" comment line above and uncomment
% one below.

% Caution! When showboxes option is on significant ammunt of space is add
% to the top of textblock and as such, everyting put in them gone down.
% We need to check how to remove this bug.

% \usepackage[showboxes,overlay,absolute]{textpos}



% Setting scale length for package "textpos"
\setlength{\TPHorizModule}{10mm}
\setlength{\TPVertModule}{\TPHorizModule}


% ---------------------------------------
% TikZ
% ---------------------------------------
% Importing TikZ libraries
\usetikzlibrary{arrows.meta}
\usetikzlibrary{positioning}





% % Configuration package "bm" that need for making bold symbols
% \newcommand{\bmmax}{0}
% \newcommand{\hmmax}{0}
% \usepackage{bm}




% ---------------------------------------
% Packages for scientific texts
% ---------------------------------------
% \let\lll\undefined  % Sometimes you must use this line to allow
% "amsmath" package to works with packages with packages for polish
% languge imported
% /preambul/LanguageSettings/JagiellonianPolishLanguageSettings.tex.
% This comments (probably) removes polish letter Ł.
\usepackage{amsmath}  % Packages from American Mathematical Society (AMS)
\usepackage{amssymb}
\usepackage{amscd}
\usepackage{amsthm}
\usepackage{siunitx}  % Package for typsetting SI units.
\usepackage{upgreek}  % Better looking greek letters.
% Example of using upgreek: pi = \uppi


\usepackage{calrsfs}  % Zmienia czcionkę kaligraficzną w \mathcal
% na ładniejszą. Może w innych miejscach robi to samo, ale o tym nic
% nie wiem.










% ---------------------------------------
% Packages written for lectures "Geometria 3D dla twórców gier wideo"
% ---------------------------------------
% \usepackage{./ProgramowanieSymulacjiFizykiPaczki/ProgramowanieSymulacjiFizyki}
% \usepackage{./ProgramowanieSymulacjiFizykiPaczki/ProgramowanieSymulacjiFizykiIndeksy}
% \usepackage{./ProgramowanieSymulacjiFizykiPaczki/ProgramowanieSymulacjiFizykiTikZStyle}





% !!!!!!!!!!!!!!!!!!!!!!!!!!!!!!
% !!!!!!!!!!!!!!!!!!!!!!!!!!!!!!
% EVIL STUFF
\if\JUlogotitle1
\edef\LogoJUPath{LogoJU_\JUlogoLang/LogoJU_\JUlogoShape_\JUlogoColor.pdf}
\titlegraphic{\hfill\includegraphics[scale=0.22]
{./JagiellonianPictures/\LogoJUPath}}
\fi
% ---------------------------------------
% Commands for handling colors
% ---------------------------------------


% Command for setting normal text color for some text in math modestyle
% Text color depend on used style of Jagiellonian

% Beamer version of command
\newcommand{\TextWithNormalTextColor}[1]{%
  {\color{jNormalTextFGColor}
    \setbeamercolor{math text}{fg=jNormalTextFGColor} {#1}}
}

% Article and similar classes version of command
% \newcommand{\TextWithNormalTextColor}[1]{%
%   {\color{jNormalTextsFGColor} {#1}}
% }



% Beamer version of command
\newcommand{\NormalTextInMathMode}[1]{%
  {\color{jNormalTextFGColor}
    \setbeamercolor{math text}{fg=jNormalTextFGColor} \text{#1}}
}


% Article and similar classes version of command
% \newcommand{\NormalTextInMathMode}[1]{%
%   {\color{jNormalTextsFGColor} \text{#1}}
% }




% Command that sets color of some mathematical text to the same color
% that has normal text in header (?)

% Beamer version of the command
\newcommand{\MathTextFrametitleFGColor}[1]{%
  {\color{jFrametitleFGColor}
    \setbeamercolor{math text}{fg=jFrametitleFGColor} #1}
}

% Article and similar classes version of the command
% \newcommand{\MathTextWhiteColor}[1]{{\color{jFrametitleFGColor} #1}}





% Command for setting color of alert text for some text in math modestyle

% Beamer version of the command
\newcommand{\MathTextAlertColor}[1]{%
  {\color{jOrange} \setbeamercolor{math text}{fg=jOrange} #1}
}

% Article and similar classes version of the command
% \newcommand{\MathTextAlertColor}[1]{{\color{jOrange} #1}}





% Command that allow you to sets chosen color as the color of some text into
% math mode. Due to some nuances in the way that Beamer handle colors
% it not work in all cases. We hope that in the future we will improve it.

% Beamer version of the command
\newcommand{\SetMathTextsColor}[2]{%
  {\color{#1} \setbeamercolor{math text}{fg=#1} #2}
}


% Article and similar classes version of the command
% \newcommand{\SetMathTextColor}[2]{{\color{#1} #2}}










% ---------------------------------------
% Commands for setting background pictures for some slides
% ---------------------------------------
\newcommand{\TitleBackgroundPicture}
{./PresentationPictures/CommonPictures/Cute_dragon_BG_dark.png}
\newcommand{\SectionBackgroundPicture}
{./PresentationPictures/CommonPictures/Cute_dragon_small_BG_light.png}



\newcommand{\TitleSlideWithPicture}{
  \begingroup

  \usebackgroundtemplate{ % \hspace*{-11.5em}
    \includegraphics[height=\paperheight]{\TitleBackgroundPicture}}

  \maketitle

  \endgroup
}





\newcommand{\SectionSlideWithPicture}[1]{%
  \begingroup

  \usebackgroundtemplate{ % \hspace*{-11.5em}
    \includegraphics[height=\paperheight]{\SectionBackgroundPicture}}

  \setbeamercolor{titlelike}{fg=normal text.fg}

  \section{#1}

  \endgroup
}





\newcommand{\EndingSlide}[1]{%
  \begin{frame}[standout]

    \begingroup

    \color{jFrametitleFGColor}

    #1

    \endgroup

  \end{frame}
}










% ------------------------------------------------------------------------------------
% Importing packages, libraries and setting their configuration
% ------------------------------------------------------------------------------------





% ------------------------------------------------------
% BibLaTeX
% ------------------------------------------------------
% Package biblatex, with biber as its backend, allow us to handle
% bibliography entries that use Unicode symbols outside ASCII.
\usepackage[
language=polish,
backend=biber,
style=alphabetic,
url=false,
eprint=true,
]{biblatex}

\addbibresource{Herdegens-approach-and-two-quasi-delta-ETC-Bibliography.bib}





% ------------------------------------------------------
% Wonderful package PGF/TikZ
% ------------------------------------------------------

% Node and pics for drawing charts
\usepackage{./Local-packages/PGF-TikZ-Chart-nodes-and-pics}

% Styles for arrows
\usepackage{./Local-packages/PGF-TikZ-Arrows-styles}

% Pic for drawing functions
\usepackage{./Local-packages/PGF-TikZ-Functions-pics}






% ------------------------------------------------------
% Local packages
% ------------------------------------------------------
% Special configuration for this particular presentation
\usepackage{./Local-packages/local-settings}

% Package containing various command useful for working with a text
\usepackage{./Local-packages/general-commands}

% Package containing commands and other code useful for working with
% mathematical text
\usepackage{./Local-packages/math-commands}










% ------------------------------------------------------------------------------------------------------------------
\title{Algebraiczne podejście do~efektu Casimira w~układzie
  z~dwoma potencjałami typu quasi-delta}

\author{Kamil Ziemian \\
  \texttt{kziemianfvt@gmail.com}}


% \institute{Uniwersytet Jagielloński w~Krakowie}

\date[18 X 2024]{18 X 2024 r.}
% ------------------------------------------------------------------------------------------------------------------










% ####################################################################
% Beginning of the document
\begin{document}
% ####################################################################





% ######################################
% Number of chars: 47k+,
% Text is adjusted to the left and words are broken at the end of the line.
\RaggedRight
% ######################################





% ######################################
\maketitle
% ######################################





% ######################################
\begin{frame}
  \frametitle{Spis treści}


  \tableofcontents % Spis treści

\end{frame}
% ######################################










% ######################################
\section{Krótka historia efektu Casimira}
% ######################################



% ##################
\begin{frame}
  \frametitle{Początki}


  W~$1948$ Hendrik Brugt Gerhard Casimir ($1909\text{-}2000$)
  publikuje artykuł w~którym stwierdza, że~na skutek efektów kwantowych
  dwie elektrycznie obojętne metalowe płyty powinny ze~sobą oddziaływać
  \parencite{Casimir-On-the-Attraction-Between-ETC-Pub-1948}. Jego
  rozumowanie opiera~się na rozpatrzeniu \textbf{energii stanu zerowego},
  „nieskończonej stałej”, która jest rezultatem zastosowania procedury
  kanonicznej kwantyzacji do pola elektromagnetycznego. Choć wedle niego
  energia punktu zerowego jest sama w~sobie niefizyczna, to różnica
  energii punktu zerowego dla dwóch różnych typów warunków brzegowych
  jest fizyczna. Wspierając~się własnościami fizycznymi metali
  Casimir był w~stanie wyliczyć wyrażenie na energię przypadająca na
  jednostkę powierzchni, które powinno być uniwersalne dla tej konkretnej
  konfiguracji.
  \begin{equation}
    \label{eq:Krotka-historia-ETC-01}
    \frac{ \delta E }{ L^{ 2 } } =
    -\hbar c \frac{ \pi^{ 2 } }{ 720 } \frac{ 1 }{ a^{ 3 } },
  \end{equation}
  Jego podejście przewiduje więc przyciągającą siłę dla tego układu.

\end{frame}
% ##################





% ##################
\begin{frame}
  \frametitle{Badania eksperymentalne}


  W~$1955$ J.M. Lifszyc publikuje pracę, w~której pokazuje jak obliczyć
  efekt Casimira uwzględniając wkład od niezerowej temperatury
  i~konkretnych własności materiału (tłum. ang.
  \parencite{Lifshitz-The-theory-of-molecular-ETC-Pub-1956}). Przeważająca
  większość przeprowadzanych eksperymentów porównuje swoje wyniki z~którąś
  z~wersji teorii Lifszyca.

  Pierwsza próbę zmierzenia efektu Casimira podjął M.J. Sparnaay, swoje
  rezultaty opublikował w~$1958$ roku
  \parencite{Sparnaay-Measurments-of-attractive-forces-ETC-Pub-1958}.
  Testował on układ dwóch metalowych płytek, nie był jednak w~stanie
  rozstrzygnął kwestii istnienia tego efektu, gdyż dla dwóch zestawów
  płytek zaobserwował ich przyciąganie, dla jednego odpychanie. %  Jednocześnie
  % zidentyfikowana przez Sparnaay’a lista problemów, które stają na
  % przeszkodzie dokładnego zmierzenia tego efektu, miała bardzo istotny
  % wpływ na dalsze badania.
  Na dzień dzisiejszym najlepszych test eksperymentalny dla tej konfiguracji
  zawiera praca G.~Bressi’ego et al. z~$2002$ roku
  \parencite{Bressi-et-al-Measurement-of-the-Casimir-Force-ETC-2002}.
  Według niej wynik Casimira jest poprawny przy niepewności wynoszącej
  $15\%$ obserwowanego zjawiska. Oprócz tego efekt ten badano
  eksperymentalnie dla wielu innych geometrycznych układów i~typów
  materiałów.

\end{frame}
% ##################





% % ##################
% \begin{frame}
%   \frametitle{Badania eksperymentalne}


%   % Pierwsze eksperymentalne potwierdzenie istnienia efektu Casimira
%   % stało~się udziałem P. van~Bloklanda i~J.~Overbeeka, którzy
%   % opublikowali swoje wyniki w~$1970$
%   % \parencite{Blokland-Overbeek-van-der-Waals-Forces-between-ETC-1978}.
%   % Badali oni znacznie prostszy z~eksperymentalnego punktu widzenia
%   % układu sfera-płaszczyzna i~porównywali dane doświadczalne z~skomplikowanym
%   % modelem opartym o~teorię Lifszyca. Pomimo wielu trudności, przy
%   % niepewności wynoszącej $50\%$ mierzonego efektu, potwierdzili zgodność
%   % teorii z~eksperymentem.

%   Do dziś najważniejszy test oryginalnego efektu Casimira przedstawia
%   praca .
%   Stosując skomplikowaną procedurę pomiarową oraz wymagającą analizę
%   danych, potwierdzili oryginalne przewidywania Casimira przy błędzie
%   wynoszącym około $15\%$ całkowitego wyniku.

%   Do tej pory zmierzono efekt Casimira dla wielu typów materiałów
%   i~konfiguracji geometrycznych, przy czym ogromną rolę w~jego badaniu
%   odgrywają rozważania z~zakresu fizyki fazy skondensowanej.

% \end{frame}
% % ##################










% ######################################
\section{Algebraiczne podejście Herdegena do efektu Casimira}
% ######################################



% ##################
\begin{frame}
  \frametitle{Algebraiczne podejście Herdegena}


  Podejście algebraiczne do efektu Casimira na którym bazuje moja praca
  pracy zostało wprowadzone w~dwuczęściowym artykule
  \textit{Quantum backreaction (Casimir) effect} Andrzeja Herdegena
  \parencite{Herdegen-Quantum-backreaction-ETC-Part-I-Pub-2005},
  \parencite{Herdegen-Quantum-backreaction-ETC-Part-II-Pub-2006}.
  Pierwsza jego część zawiera między innymi zestaw aksjomatów odpowiedniej
  algebraicznej kwantowej teorii pola (\textsc{aqft}). W~drugiej
  przedstawiono analizę problemu pola elektromagnetyczne oraz
  dwóch metalowych płytek, poprzez modyfikowanie własności operatorów
  w~reprezentacji pędowej.

  Ten sam układ został przebadany w~pracy Andrzeja Herdegena i~Mariusza
  Stopy, gdzie pokazano jak można opisać korzystając z~reprezentacji
  położeń \parencite{Herdegen-Stopa-Global-vs-local-ETC-2010}. Praca ta była
  główną inspiracją dla moich badań.

\end{frame}
% ##################





% ##################
\begin{frame}
  \frametitle{Podstawowe założenia}


  W~podejściu Herdegen efekt Casimira rozumiemy w~następujący sposób.
  Przyjmijmy, że~badany układ można podzielić w~jasny sposób na część
  kwantową~$Q$ i~makroskopową $M$. Gdy w~skutek warunków zewnętrznych układ
  $M$ porusza~się w~sposób adiabatyczny, dochodzi do reakcji zwrotnej ze
  strony układu $Q$, który przekazuje energię do~$M$, którą ten ostatni
  podukład przekształca w~energię kinetyczną, co skutkuje pojawieniem~się
  odpowiedniej siły.

  Wedle naszej znajomości literatury dotyczącej eksperymentalnego badania
  tego efektu, założenie o~adiabatycznym charakterze ruchu zgadza~się
  dobrze z~tym, jak przeprowadzane są realne eksperymenty.

\end{frame}
% ##################









% ##################
\begin{frame}
  \frametitle{Schemat algebraiczny}


  Wedle przyjętych aksjomatów układ $Q$ jest określony przez algebrę
  $C^{ * }$, oznaczaną przez $\Acal$, oraz jej reprezentacje w~przestrzeni
  Hilberta, przy czym wszystkie te reprezentacje muszą być sobie równoważne.
  Algebra $\Acal$ jest zlokalizowana na hiperpłaszczyźnie $t = \const$, co
  wobec przyjęcia założenia o~adiabatycznej ewolucji układu nie prowadzi
  do problemów. Natomiast układ $M$ jest opisywany przez zestaw kilku
  makroskopowych zmiennych, które zbiorczo oznaczamy przez $a$.

  Energia Casimira jest zdefiniowana jako wartość oczekiwana
  hamiltonianu \alert{swobodnego} układu $Q$ w~stanie podstawowym
  sprzężonego układu $Q\text{-}M$:
  \begin{equation}
    \label{eq:Algebraiczne-podejscie-ETC-01}
    \Ecal_{ a } = ( \Omega_{ a }, H \Omega_{ a } ).
  \end{equation}
  % Choć takie określenie energii Casimira nie jest powszechnie przyjęte
  % w~literaturze, to jak zauważył Herdegen, w~tym problemie nie ma innego,
  % jednoznacznie zdefiniowanego operatora energii.
  Siła Casimira jest zdefiniowana w~naturalny sposób:
  \begin{equation}
    \label{eq:Algebraiczne-podejscie-ETC-02}
    \Fcal_{ a } = -\frac{ \partial \Ecal_{ a } }{ \partial a }.
  \end{equation}

\end{frame}
% ##################





% % ##################
% \begin{frame}
%   \frametitle{Schemat algebraiczny}


%   Wedle przyjętych aksjomatów układ $Q$ jest określony przez algebrę
%   $C^{ * }$, oznaczaną przez $\Acal$, oraz jej reprezentacje na przestrzeni
%   Hilberta, przy czym wszystkie te reprezentacje muszą być sobie równoważne.
%   Algebra $\Acal$ jest zlokalizowana na hiperpłaszczyźnie $t = \const$, co
%   wobec przyjęcia założenia o~adiabatycznej ewolucji układu nie prowadzi
%   do problemów. Natomiast układ $M$ jest opisywany przez zestaw kilku
%   makroskopowych zmiennych, które zbiorczo oznaczamy przez $M$.

%   Energia Casimira jest zdefiniowana jako wartość oczekiwana
%   hamiltonianu \alert{swobodnego} układu $Q$ w~stanie podstawowym
%   sprzężonego układu $Q\text{-}M$:
%   \begin{equation}
%     \label{eq:Algebraiczne-podejscie-ETC-03}
%     \Ecal_{ a } = ( \Omega_{ a }, H \Omega_{ a } ).
%   \end{equation}
%   % Choć takie określenie energii Casimira nie jest powszechnie przyjęte
%   % w~literaturze, to jak zauważył Herdegen, w~tym problemie nie ma innego,
%   % jednoznacznie zdefiniowanego operatora energii.
%   Siła Casimira jest zdefiniowana w~naturalny sposób:
%   \begin{equation}
%     \label{eq:Overview-of-Herdegens-approach-ETC-07}
%     \Fcal_{ a } = -\frac{ \partial \Ecal_{ a } }{ \partial a }.
%   \end{equation}

% \end{frame}
% % ##################





% % ##################
% \begin{frame}
%   \frametitle{Ewolucja pola skalarnego}


%   Zapiszemy teraz równania ewolucji swobodnego skalarnego pola kwantowego
%   oraz tego samego pole w~obecności układu~$M$:

%   \vspace{-2em}


%   \begin{subequations}

%     \begin{align}
%       \label{eq:Algebraiczne-podejscie-ETC-03}
%       \left( \partial_{ t }^{ 2 } + h^{ 2 } \right) \phi( t, \vecx ) = 0, \qquad
%       &\left( \partial_{ t }^{ 2 } + h_{ a }^{ 2 } \right) \phi( t, \vecx ) = 0, \\
%       h^{ 2 } = -\Delta, \qquad
%       &h_{ a }^{ 2 } = h^{ 2 } + V_{ a }.
%     \end{align}

%   \end{subequations}

%   \vspace{-2em}



%   Nas będzie interesowała tylko przypadek, gdy $V_{ a }$ jest operatorem
%   skończonego rzędu.

%   Herdegen, korzystając z~rozkładu pola na operator anihilacji i~kreacji,
%   pokazał, że~liczb kwantów \alert{swobodnego} pola $\Ncal_{ a }$
%   wytworzona przez wprowadzenie układu $M$ i~jego energia Casimira
%   wyrażają~się wzorami

%   \vspace{-1.8em}



%   \begin{subequations}

%     \begin{align}
%       \label{eq:Algebraiczne-podejscie-ETC-04-A}
%       \Ncal_{ a }^{ \vphantom{1} }
%       &=
%         \frac{ 1 }{ 4 } \Tr\big[ \HorSpaceOne h^{ -1/2 } (
%         h_{ \HorSpaceOne a }^{ \vphantom{1} } - h )
%         h_{ \HorSpaceOne a }^{ -1 }
%         ( h_{ a }^{ \vphantom{1} } - h ) h^{ -1/2 } \HorSpaceTwo \big], \\
%       \label{eq:Algebraiczne-podejscie-ETC-04-B}
%       \Ecal_{ \HorSpaceTwo a }
%       &=
%         ( \GroundStateAOne, H \HorSpaceOne \GroundStateAOne ) =
%         \frac{ 1 }{ 4 }
%         \Tr\big[ ( h_{ \HorSpaceOne a }^{ \vphantom{1} } - h )
%         h_{ \HorSpaceOne a }^{ -1 }
%         ( h_{ \HorSpaceOne a }^{ \vphantom{1} } - h ) \big].
%     \end{align}

%   \end{subequations}

% \end{frame}
% % ##################





% % ##################
% \begin{frame}
%   \frametitle{Elements of Herdegen's approach}


%   Względy matematyczny i~fizyczne wymagają od nas byśmy żądali
%   od problemu, iż~liczba kwantów swobodnego pola i~energia Casimira
%   są skończone:

%   \vspace{-2.2em}



%   \begin{subequations}

%     \begin{align}
%       \label{eq:Algebraiczne-podejscie-ETC-05-A}
%       \Ncal_{ a }^{ \vphantom{1} }
%       &=
%         \frac{ 1 }{ 4 } \Tr\big[ \HorSpaceOne h^{ -1/2 } (
%         h_{ \HorSpaceOne a }^{ \vphantom{1} } - h )
%         h_{ \HorSpaceOne a }^{ -1 }
%         ( h_{ a }^{ \vphantom{1} } - h ) h^{ -1/2 } \HorSpaceTwo \big]
%         < +\infty, \\[0.3em]
%       \label{eq:Algebraiczne-podejscie-ETC-05-B}
%       \Ecal_{ \HorSpaceTwo a }
%       &=
%         \frac{ 1 }{ 4 }
%         \Tr\big[ ( h_{ \HorSpaceOne a }^{ \vphantom{1} } - h )
%         h_{ \HorSpaceOne a }^{ -1 }
%         ( h_{ \HorSpaceOne a }^{ \vphantom{1} } - h ) \big] < +\infty.
%     \end{align}

%   \end{subequations}

%   \vspace{-1em}



%   Herdegen pokazał również, wychodząc od standardowego tensora energii-pędu,
%   że~lokalna gęstość energii tego problemu jest dystrybucją daną wzorem:
%   \begin{equation}
%     \label{eq:Algebraiczne-podejscie-ETC-06}
%     T_{ a }( \varphi, \psi ) =
%     \frac{ 1 }{ 4 } \big( \varphi, ( h_{ a }^{ \vphantom{1} } - h ) \psi \big) +
%     \frac{ 1 }{ 4 } \big( \nabla \varphi, ( h_{ a }^{ -1 } -
%     h^{ -1 }_{ \vphantom{a} } ) \nabla \psi \big).
%   \end{equation}

% \end{frame}
% % ##################










% ######################################
\section{Pole skalarne z~dwoma quasi-deltami}
% ######################################



% ##################
\begin{frame}
  \frametitle{Sformułowanie modelu}


  Równanie ewolucji pola skalarnego w~obecności układu $M$
  i~dla operatora ewolucji swobodnej $h^{ 2 } = -\Delta$ jest postaci:

  \vspace{-1.8em}


  \begin{equation}
    \label{eq:Algebraiczne-podejscie-ETC-07}
    \left( \partial_{ t }^{ 2 } + h_{ \veca }^{ 2 } \right) \phi( t, \vecx ) = 0, \qquad
    h_{ \veca }^{ 2 } = h^{ 2 } + V_{ \veca }.
  \end{equation}
  Chcemy by równanie to reprezentowało pole skalarne oddziałujące
  z~dwoma sferycznie symetrycznymi ciałami, których środki są połączone
  wektorem~$\veca$. Jego długość oznaczamy jako $a = \absOne{ \veca }$.

  Biorąc inspirację z~pracy Herdegena i~Stopy
  \parencite{Herdegen-Stopa-Global-vs-local-ETC-2010}, wybieramy
  quasi-potencjał $V_{ \veca }$, który w~reprezentacji położeń ma postać
  \begin{equation}
    \label{eq:Pole-skalarne-ETC-01}
    V_{ \veca }( \vecx, \vecy \HorSpaceThree ) =
    \sigma( g ) \! \left[ g\!\left( \vecx - \vecb \HorSpaceFive \right) \,
      \overline{ g\!\left( \vecy - \vecb \HorSpaceFive \right) }
      + g\!\left( \vecx + \vecb \HorSpaceFive \right) \,
      \overline{ g\!\left( \vecy + \vecb \HorSpaceFive \right) } \,
    \right], \;
    \vecb = \frac{ \veca }{ 2 }.
  \end{equation}
  O~funkcji $g( \vecx )$ przyjmujemy, że jest klasy $C^{ \infty }( \Rbb^{ 3 } )$,
  o~zwartym nośniku zawartym w~kuli $B( \vecZero, L / 2 )$. Z~technicznych
  względów potrzebujemy również założyć, że~$g( \vecx ) \geq 0$. Funkcjonał
  $\sigma( g )$ wyjaśnimy później.

\end{frame}
% ##################





% % ##################
% \begin{frame}
%   \frametitle{Symboliczne przedstawienie quasi-potencjału}


%   Poniżej można zobaczyć wykres przykładowych funkcji
%   $g( \vecx \pm \vecb )$ tworzących quasi-potencjał $V_{ \veca }$.

%   \vspace{5em}





%   \begin{figure}

%     \label{fig:Symbolic-sketch-of-quasi-potential}


%     \centering

%     \begin{tikzpicture}

%       % Smooth functions witch compact support
%       \pic at (-3,0) {smooth function with compact support fill 1};

%       \pic at (3,0) {smooth function with compact support fill 1};





%       % x axis
%       \draw[axis arrow] (-4.5,0) -- (5,0);

%       \pic at (5,0) {x mark for horizontal axis 1};


%       % y axis
%       \draw[axis arrow] (0,-0.2) -- (0,2.3);

%       \pic at (0,2.3) {V mark for vertical axis 1};

%       \node[right] at (0,-0.2) {$0$};





%       % Thicks on x axis
%       \pic at (-4,0) {tick x axis thin};

%       \node[below] at (-4,0) {$-4$};


%       \pic at (-3,0) {tick x axis thin};

%       \node[below] at (-3,0) {$-3$};


%       \pic at (-2,0) {tick x axis thin};

%       \node[below] at (-2,0) {$-2$};


%       \pic at (-1,0) {tick x axis thin};

%       \node[below] at (-1,0) {$-1$};


%       \pic at (1,0) {tick x axis thin};

%       \node[number below x axis] at (1,0) {$1$};


%       \pic at (2,0) {tick x axis thin};

%       \node[number below x axis] at (2,0) {$2$};


%       \pic at (3,0) {tick x axis thin};

%       \node[number below x axis] at (3,0) {$3$};


%       \pic at (4,0) {tick x axis thin 1};

%       \node[number below x axis] at (4,0) {$4$};


%       % \pic at (5,0) {tick x axis thin};

%       % \node[below] at (5,0) {$5$};





%       % Ticks on y axis
%       \pic at (0,0.5) {tick y axis thin 1};

%       \node[node scale small 2,left] at (0,0.5) {$0.5$};


%       \pic at (0,1) {tick y axis thin};

%       \node[left] at (0,1) {$1$};


%       \pic at (0,1.5) {tick y axis thin 1};

%       \node[node scale small 2,left] at (0,1.5) {$1.5$};

%     \end{tikzpicture}

%     \caption{Symboliczny wykres quasi-potencjału $V_{ \veca }$.}


%   \end{figure}

% \end{frame}
% % ##################





% ##################
\begin{frame}
  \frametitle{Własności modelu}


  Korzystając z~twierdzeń Kato-Rellicha, Kato-Rosenbluma, twierdzenia
  Weyla o~widmie istotnym, etc., oraz wykonując trochę obliczeń, pokazujemy,
  że~przy odpowiednich założeniach widma obu operatorów $h^{ 2 }$
  i~$h_{ \veca }^{ 2 }$ redukują~się do widma absolutnie ciągłego równego
  $[ 0, +\infty )$. Co więcej ich widma są unitarnie równoważne, co pozwala nam
  użyć metod z teorii rozpraszania do rozwiązania tego problemu.

  Potrzebujemy teraz wprowadzić trochę notacji dla funkcji. Choć jest ich
  niemało, najważniejsze jest to, że wszystkie wyrażają~się przez
  odpowiednie całki funkcji $g( \vecx )$.

  \vspace{-1.6em}



  \begin{subequations}

    \begin{align}
      \label{eq:Pole-skalarne-ETC-02-A}
      \gHat( p ) = \gHat( \vecp \HorSpaceFour )
      &=
      \frac{ 1 }{ \sqrt{ 2 \pi } }
      \int\limits_{ -\infty }^{ +\infty } dx \, \frac{ \sin( p x ) }{ p x }
        \HorSpaceTwo x^{ 2 } g( x ), \\
      \label{eq:Pole-skalarne-ETC-02-B}
      M_{ p }
      &= \gHat( p )^{ 2 }, \qquad
        I_{ k } =
        \int\limits_{ -\infty }^{ +\infty } dq \,
        \frac{ M_{ \HorSpaceOne q } - M_{ \HorSpaceOne k } }{
        k^{ \HorSpaceOne 2 } - q^{ \HorSpaceOne 2 } }.
    \end{align}

  \end{subequations}

\end{frame}
% ##################





% ##################
\begin{frame}
  \frametitle{Przypadek graniczny}

  \vspace{-2em}



  \begin{subequations}

    \begin{align}
      \label{eq:Pole-skalarne-ETC-03-A}
      \checkMx =
        \frac{ 1 }{ \sqrt{ 2 \pi } }
      \int\limits_{ -\infty }^{ +\infty } dp \, e^{ i p x }
      &M_{ p }, \quad
        b_{ \HorSpaceOne 1 } =
        \frac{ 1 }{ \sqrt{ 2 \pi } 2 \pi^{ 2 } }
        \int\limits_{ -\infty }^{ +\infty } dx \, \absOne{ x } \checkMx, \\
      \label{eq:Pole-skalarne-ETC-03-B}
      \chi( g )
      &=
        \frac{ 2 \pi }{ \alpha }
        \int\limits_{ -\infty }^{ +\infty } dp \, M_{ p } =
        \sqrt{ 2 \pi }^{ 3 } \frac{ \checkMzero }{ \alpha }, \\
      \label{eq:Pole-skalarne-ETC-03-C}
      \sigma( g )^{ -1 }
      &=
        -\alpha ( M_{ 0 } + \chi( g ) ).
    \end{align}

  \end{subequations}

  \vspace{-1.8em}



  W~problemie występuje wiele innych funkcji. Zdecydowaliśmy~się zapisać
  tylko te, które występują jawnie w~najważniejszych równaniach.

  Podana tu postać funkcjonału $\sigma( g )$ wynika z~analizy przeskalowanej
  rodziny modeli. Mianowicie, gdy ściągamy oba sferyczne ciała do punktów,
  nasz problem dąży w~sensie silnej granicy resolwentowej do problemu
  pola skalarnego z~dwoma deltami Diraka. Parametr $\alpha$ jest dobrze znany
  w~teorii takich układów i~opisuje długość rozpraszania
  \parencite{Albeverio-et-al-Solvable-Models-in-Quantum-Mechanics-Pub-1988}.

\end{frame}
% ##################





% ##################
\begin{frame}
  \frametitle{Liczba kwantów i~energia Casimira}


  Rozkład spektralny operatorów i~metody teoria rozpraszania pozwalają nam
  zapisać wzory na liczbę kwantów swobodnego pola obecnych w~układzie
  i~energię Casimira
  % (\eqref{eq:Algebraiczne-podejscie-ETC-05-A},
  % \eqref{eq:Algebraiczne-podejscie-ETC-05-B})
  za pomocą jednego wyrażenia:

  \vspace{-1.1em}



  \begin{equation}
    \label{eq:Pole-skalarne-ETC-04}
    \Pcal_{ \tau }( a ) =
    16 \Real \int\limits_{ \Rbb_{ + }^{ \, 2 } } dk \, dp \,
    \frac{ p^{ \HorSpaceOne 2 - \tau } }{ ( p + i k )^{ 2 } } M_{ \HorSpaceOne p }
    \frac{ t( i k ) + h( p a ) u( i k ) }{ t( i k )^{ 2 } - u( i k )^{ 2 } }.
  \end{equation}

  \vspace{-1.3em}



  Dla $\tau = 0$ dostajemy wyrażenie na energię, dla $\tau = 1$ na liczbę
  kwantów. Można łatwo pokazać, że~obie te całki są zbieżne, co gwarantuje
  spójność naszego podejścia.

  Poprzez przejście do granicy $a \nearrow +\infty$ we wzorze na energię można
  zidentyfikować człon odpowiedzialny za samooddziaływanie ciał
  makroskopowych i~podzielić wyrażenie na energię na część związaną
  z~samoodziaływaniem ciał i~oddziaływaniem między nimi.

\end{frame}
% ##################





% ##################
\begin{frame}
  \frametitle{Energia i~model przeskalowany}

  \vspace{-2em}


  \begin{subequations}

    \begin{align}
      \label{eq:Pole-skalarne-ETC-05-A}
      \ESelf
      &=
        16 \pi^{ 2 } \, \int\limits_{ \Rbb_{ + }^{ \, 2 } } dp \, dk \,
        \frac{ p^{ \HorSpaceOne 2 } k }{ ( p + k )^{ 2 } } \,
        \frac{ M_{ \HorSpaceOne p } \, M_{ \HorSpaceOne k } }{
        ( \alpha + 2 \pi k^{ 2 } I_{ k } )^{ 2 } +
        ( 2 \pi^{ 2 } k M_{ \HorSpaceOne k } )^{ 2 } }, \\
      \label{eq:Pole-skalarne-ETC-05-B}
      \EInt( a )
      &=
        16 \, \Real \int\limits_{ \Rbb_{ + }^{ \, 2 } } dp \, dk \,
        \frac{ p^{ \HorSpaceOne 2 } }{ ( p + i k )^{ 2 } } \,
        M_{ \HorSpaceOne p } \,
        \frac{ u( i k )^{ 2 } + h( p a ) t( i k ) u( i k ) }{
        t( i k ) [ t( i k )^{ 2 } - u( i k )^{ 2 } ] }.
    \end{align}

  \end{subequations}



  Ponieważ wzory te są dość skomplikowana, by wyizolować odpowiednie
  stopnie swobody, przebadaliśmy przeskalowaną wersję modelu i~obliczyliśmy
  jego rozwinięcie asymptotyczne. Tą przeskalowaną wersję problemu
  otrzymujemy zastępując funkcję $g( \vecx )$ przez
  \begin{equation}
    \label{eq:Pole-skalarne-ETC-06}
    g_{ \lambda }( \vecx ) =
    \lambda^{ -3 } g\left( \tfrac{ \vecx }{ \lambda } \right), \qquad
    \lambda \in ( 0, 1 ].
  \end{equation}
  Funkcja $g_{ \lambda }( \vecx )$, przy $\lambda$ ustalonym, posiada te same
  własności co funkcja $g( \vecx )$, więc wszystkie otrzymane wyniki
  pozostają w~mocy.

\end{frame}
% ##################





% ##################
\begin{frame}
  \frametitle{Granica skalowania}


  Z~postaci funkcji $g_{ \lambda }( \vecx )$ wynika, że~wzięcie granicy $\lambda \searrow 0$
  powoduje ściągnięcie rozważanych ciał do punktu, podczas gdy siła
  oddziaływania z~nimi rośnie do nieskończoności. Jak już mówiliśmy, nasz
  model jest wtedy zbieżny do problemu pola skalarnego z~dwoma deltami
  Diraka
  \parencite{Albeverio-et-al-Solvable-Models-in-Quantum-Mechanics-Pub-1988}.

  Dalsze wyniki zapiszemy korzystając z~naturalnej wersji zmiennej
  bezwymiarowej
  \begin{equation}
    \label{eq:Pole-skalarne-ETC-07}
    \gamma := \frac{ \alpha a }{ 2 \pi^{ 2 } }.
  \end{equation}
  Można pokazać, że~gdy spełniony jest warunek
  \begin{equation}
    \label{eq:Pole-skalarne-08}
    \gamma > 1,
  \end{equation}
  wówczas operator $h_{ \veca }^{ 2 }$ jest dodatnio określony i~cały
  formalizm pozostaje spójny. Gdy $\gamma \leq 1$, potencjalnie mogą pojawić~się
  stany związane do ujemnej energii.

\end{frame}
% ##################








% ##################
\begin{frame}
  \frametitle{Rozwinięcie asymptotyczne}


  Energia sammooddziaływania, jak należało się spodziewać, jest rozbieżna
  w~granicy $\lambda \searrow 0$. Energia oddziaływania również jest rozbieżna, jednak
  jej rozwinięcie asymptotyczne prowadzi do konkretnych przewidywań.
  \begin{equation}
    \label{eq:Pole-skalarne-ETC-09}
    \begin{split}
      &\EInt( a, \lambda ) = \\
      &=
        \frac{ 2 \alpha }{ \pi^{ 3 } } \! \Bigg[ \frac{ \chi }{ \lambda } \!
        \int\limits_{ 0 }^{ +\infty } \!
        \frac{ e^{ -2 l } \, dl }{ ( \gamma + l )
        \big[ ( \gamma + l )^{ 2 } - e^{ -2 l } \big] } \! + \!
        \frac{ b_{ \HorSpaceOne 1 } \chi }{ \gamma } \!
        \int\limits_{ 0 }^{ +\infty } \!
        \frac{ l^{ 2 } \big[ 3 ( \gamma + l )^{ 2 } e^{ -2 l } -
        e^{ -4 l } \big] }
        { ( \gamma + l )^{ 2 }
        \big[ ( \gamma + l )^{ 2 } - e^{ -2 l } \big]^{ 2 } } dl
        \, - \\[0.5em]
      &- \! \frac{ 2 }{ \gamma } \! \int\limits_{ 0 }^{ +\infty } \!
        \frac{ l e^{ -2 l }\, dl }{ ( \gamma + l )
        \big[ ( \gamma + l )^{ 2 } - e^{ -2 l } \big] } \! + \!
        \frac{ 1 }{ \gamma } \!
        \int\limits_{ 0 }^{ +\infty } \! \frac{ ( 1 - l ) e^{ -2 l } }{
        ( \gamma + l )^{ 2 } - e^{ -2 l } } \, dl + R( a, \lambda ) \Bigg],
    \end{split}
  \end{equation}
  Funkcjonały $\chi > 0$ i~$b_{ \HorSpaceOne 1 }$, zostały wprowadzone
  w~równaniach \eqref{eq:Pole-skalarne-ETC-03-A}
  i~\eqref{eq:Pole-skalarne-ETC-03-B}. Reszta rozwinięcia spełnia zaś
  nierówność $\absOne{ R( a, \lambda ) } \leq \const( a ) \lambda$.

\end{frame}
% ##################





% ##################
\begin{frame}
  \frametitle{Podstawowe wnioski}


  Korzystając z~kombinacji metod analitycznych i~numerycznych z~dużą
  dozą pewności ustaliliśmy, że~wzór
  \eqref{eq:Pole-skalarne-ETC-09} przewiduje występowanie
  \alert{odpychającej} siły między ciałami. Do tego wyniki numeryczne
  wskazują na to, że~dominujący jest człon \alert{zależny od modelu}.
  W~prezentowanym podejściu, nie możemy więc mówić o~uniwersalnej sile
  Casimira dla układu dwóch quasi-delt.

  Jest to wynik bardzo odbiegający od tego otrzymanego przez A.~Scardicchia
  dla tego samego problemu
  \parencite{Scardicchio-Casimir-dynamics-ETC-2005}. Używają wariantu
  podejścia opartego o~energię punktu zerowego, wprowadził on wzór
  przewidujący \alert{uniwersalną} i~\alert{przyciągającą} siłę Casimira.

\end{frame}
% ##################





% % ##################
% \begin{frame}
%   \frametitle{Asymptotic expansion of Casimir energy}

%   \vspace{-2em}



%   To analyse this formula we were forced to use combination of analytical
%   and numeric methods. With very high degree of confidence we established
%   that it predicts repulsive force that very quickly vanishes with the
%   distance. This is in striking contrast with previous works of A.
%   Sccardicchio \parencite{}, who used
%   version of zero-point energy approach to derive for this system force
%   that according to him is \alert{universal} and \alert{attractive}.

% \end{frame}
% % ##################





% % ##################
% \begin{frame}
%   \frametitle{Asymptotic expansion of Casimir energy}

%   \vspace{-2.5em}


%   \begin{equation}
%     \label{eq:Rescaled-version-of-the-model-06}
%     \begin{split}
%       &\EInt( a, \lambda ) = \\
%       &=
%         \frac{ 2 \alpha }{ \pi^{ 3 } } \! \Bigg[ \frac{ \chi }{ \lambda } \!
%         \int\limits_{ 0 }^{ +\infty } \!
%         \frac{ e^{ -2 l } \, dl }{ ( \gamma + l )
%         \big[ ( \gamma + l )^{ 2 } - e^{ -2 l } \big] } \! + \!
%         \frac{ b_{ \HorSpaceOne 1 } \chi }{ \gamma } \!
%         \int\limits_{ 0 }^{ +\infty } \!
%         \frac{ l^{ 2 } \big[ 3 ( \gamma + l )^{ 2 } e^{ -2 l } -
%         e^{ -4 l } \big] }
%         { ( \gamma + l )^{ 2 }
%         \big[ ( \gamma + l )^{ 2 } - e^{ -2 l } \big]^{ 2 } } dl
%         \, - \\[0.5em]
%       &- \! \frac{ 2 }{ \gamma } \! \int\limits_{ 0 }^{ +\infty } \!
%         \frac{ l e^{ -2 l }\, dl }{ ( \gamma + l )
%         \big[ ( \gamma + l )^{ 2 } - e^{ -2 l } \big] } \! + \!
%         \frac{ 1 }{ \gamma } \!
%         \int\limits_{ 0 }^{ +\infty } \! \frac{ ( 1 - l ) e^{ -2 l } }{
%         ( \gamma + l )^{ 2 } - e^{ -2 l } } \, dl + R( a, \lambda ) \Bigg],
%     \end{split}
%   \end{equation}
%   The four integrals in equation above, from left two right, all taken
%   with sign plus, we denoted as $I_{ 1 }$, $I_{ 2 }$, $I_{ 3 }$ and $I_{ 4 }$.
%   At the next slides we plot results of numerical computations of Casimir
%   energy for which we assumed:
%   \begin{equation}
%     \label{eq:Rescaled-version-of-the-model-07}
%     \frac{ \chi }{ \lambda } = 10.0, \qquad
%     \chi b_{ 1 } = 1.0.
%   \end{equation}

% \end{frame}
% % ##################





% ##################
\begin{frame}
  \frametitle{Cztery składowe rozwinięcia asymptotycznego}

  \vspace{-0.5em}


  \begin{figure}

    \label{fig:Terms-of-asymptotic-expansion}

    \centering


    \includegraphics[scale=0.525]
    {./Presentation-pictures/Terms\_of\_asymptotic\_expansion\_01.png}

    \caption{Wykres czterech całek obecnych we~wzorze
      \eqref{eq:Pole-skalarne-ETC-09}, wziętych ze znakiem
      plus.}


  \end{figure}

\end{frame}
% ##################





% ##################
\begin{frame}
  \frametitle{Energia Casimira}

  \vspace{-0.5em}


  \begin{figure}

    \label{fig:Asymptotic-expansion-of-Casimir-energy}

    \centering


    \includegraphics[scale=0.525]
    {./Presentation-pictures/Casimir\_energy\_asymptotic\_expansion\_01.png}

    \caption{Wykres sumy wiodących członów rozwinięcia asymptotycznego,
      przy $\chi / \lambda = 10.0$, $\chi b_{ \HorSpaceOne 1 } = 1.0$.}


  \end{figure}

\end{frame}
% ##################





% ##################
\begin{frame}
  \frametitle{Wynik A.~Scardicchia}

  \vspace{-0.5em}


  \begin{figure}

    \label{fig:Casimir-energy-Scardicchio}

    \centering


    \includegraphics[scale=0.525]
    {./Presentation-pictures/Casimir\_energy\_Scardicchio\_01.png}

    \caption{Energia Casimira obliczona przez A.~Sccardicchio.}


  \end{figure}

\end{frame}
% ##################





% ##################
\begin{frame}
  \frametitle{Lokalna gęstość energii}


  Lokalna gęstość energii również dzieli~się na gęstość energii
  samoodziaływania i~oddziaływania. W~granicy $\lambda \searrow 0$
  ta pierwsza jest regularną dystrybucją
  na~$\Rbb^{ 3 } \setminus \{ \vecZero \}$, ta druga zaś
  na~$\Rbb^{ 3 } \setminus \{ -\veca / 2, \veca / 2\}$.

  Poniższy wzór przedstawia gęstość energii samooddziaływania dla
  pojedynczej delty Diraka.

  \vspace{-1.5em}



  \begin{equation}
    \label{eq:Pole-skalarne-ETC-10}
    \ESingleDenLim( \vecx \HorSpaceOne ) =
    \frac{ 1 }{ 4 }
    \int\limits_{ 0 }^{ +\infty } dr \, \frac{ 1 }{ \alpha + 2 \pi^{ 2 } r }
    \left( 1 + 2 r \absOne{ \vecx } \right)
    \frac{ e^{ -2 r \absOne{ \vecx } } }
    { \absOne{ \vecx }^{ 4 } }.
  \end{equation}
  Gęstość energii oddziaływania znajduje~się na następnym slajdzie.

\end{frame}
% ##################





% ##################
\begin{frame}
  \frametitle{Gęstość energii oddziaływania}

  \vspace{-2em}


  \begin{equation}
    \label{eq:Pole-skalarne-ETC-11}
    \begin{split}
      &\EIntDenLim( \veca, \vecx \HorSpaceOne ) =
        \frac{ 1 }{ 8\pi^{ 2 } }
        \int\limits_{ 0 }^{ +\infty } dl \,
        \frac{ e^{ -2 l } }{ ( \gamma + l ) [ ( \gamma + l )^{ 2 } - e^{ -2 l } ] } \, \times
      \\[0.5em]
      &\hspace{2em}
        \times \Bigg[ \frac{ e^{ -2 l \absTwo{ \vecx + \vecb } / a } }{
        \absTwo{ \vecx + \vecb }^{ 4 } }
        \left( 1 + 2 l \frac{ \absTwo{ \vecx + \vecb } }{ a } \right) +
        \frac{ e^{ -2 l \absTwo{ \vecx - \vecb } / a } }{
        \absTwo{ \vecx - \vecb }^{ 4 } } \left( 1 +
        2 l \frac{ \absTwo{ \vecx - \vecb } }{ a } \right) \Bigg] \, -
      \\[0.5em]
      &\hspace{1em}
        - \frac{ 1 }{ 4 \pi^{ 2 } } \int\limits_{ 0 }^{ +\infty } dl \,
        \frac{ e^{ -l } }{ ( \gamma + l )^{ 2 } - e^{ -2 l } }
        \frac{ e^{ -l ( \absOne{ \vecx + \vecb } + |\,
        \vecx - \vecb\, | ) / a } }{ |\, \vecx + \vecb\, | \absTwo{ \vecx -
        \vecb } } \, \times \\[0.5em]
      &\hspace{2em}
        \times \Bigg[ \frac{ l^{ 2 } }{ a^{ 2 } } \Bigg( 1 - \frac{ (
        \vecx + \veca ) \cdot ( \vecx - \veca ) }{ |\, \vecx
        + \vecb\, | \absTwo{ \vecx - \vecb } } \Bigg) \, - \\[0.5em]
      &\hspace{3.5em}
        - \frac{ ( \vecx + \vecb \HorSpaceSix ) \cdot
        ( \vecx - \vecb \HorSpaceSix ) }{
        \absOne{ \vecx + \vecb } \HorSpaceFive
        \absOne{ \vecx - \vecb } }
        \frac{ 1 + l ( | \vecx
        + \vecb | + | \vecx - \vecb | ) / a }{ | \vecx +
        \vecb | \, \absOne{ \vecx - \vecb } } \Bigg].
    \end{split}
  \end{equation}

\end{frame}
% ##################





% ##################
\begin{frame}
  \frametitle{Wnioski}


  Należy zauważyć, że~\alert{globalna} energia dla badanego układu
  jest silnie zależna od modelu, to \alert{lokalna gęstość energii}
  na swoim zbiorze regularnym jest już od niego \alert{niezależna}.
  Do podobnych wniosków doszli w~swojej pracy Herdegen i~Stopa.

  Gęstość energii samoodziaływania pojedynczej~$\delta$, por.
  rów. \eqref{eq:Pole-skalarne-ETC-10}, jest taka sama jak ta uzyskana
  przez  Davide Fermi i~Livio Pizzocchero za pomocą zupełnie innego
  podejścia, opartego o~tzw. regularyzację funkcją zeta
  \parencite{Fermi-Pizzocchero-Local-Casimir-Effect-for-a-Scalar-ETC-2018}.

  Wyniki te każą uważać, że~relacja między globalną i~lokalną energią
  w~przypadku granicznym nie jest wcale prosta i~że gęstość energii jest
  „łatwiejsza” do przewidzenia.

\end{frame}
% ##################





% ##################
\begin{frame}
  \frametitle{Zakończenie}

  \vspace{7em}


  \begin{center}

    \Large

    Dziękuję za uwagę.

  \end{center}

\end{frame}
% ##################







% ####################################################################
\appendix
% ####################################################################





% ######################################
\section{Pytania od recenzentów}
% ######################################


% ##################
\begin{frame}
  \frametitle{Pytania od prof. Jana Derezińskiego}


  \textit{Czemu autor stwierdził, że zasada przesunięcia punktu odniesienia
    dla energii stosuje~się inaczej dla układów o~skończonej i~nieskończonej
    liczbie stopni swobody?}

  Rozpatrzymy dwa układy, jeden o~skończonej, drugi o~przeliczalnej liczbie
  stopni swobody, takie że energia całkowita jest sumą energii
  odpowiadającej poszczególnym stopniom swobody.

  \vspace{-1.8em}



  \begin{subequations}

    \begin{align}
      \label{eq:Algebraiczne-podejscie-ETC-01}
      E_{ \text{tot, fin} }
      &=
        \sum_{ i = 1 }^{ N } E_{ i } < +\infty, \\
        \label{eq:Algebraiczne-podejscie-ETC-01}
      E_{ \text{tot, con} }
      &=
        \sum_{ i = 1 }^{ \infty } E_{ i } < +\infty.
    \end{align}

  \end{subequations}

  \vspace{-1em}


  Dla układu o~skończonej liczbie stopni swobody przesunięcie energii
  każdego z~tych stopni o~skończoną wartość $E_{ \text{shi} }$, powoduje
  pojawienie~się skończonej stałej $N E_{ \text{shi} }$ w~wyrażeniu na
  energię całkowitą, co nie wpływa na stronę fizyczną
  i~matematyczną problemu. %  Dla układu o~przeliczalnej liczbie stopni
  % swobody ta sama
  % procedura powoduje pojawienie się „nieskończonej stałej”
  % $\infty \cdot E_{ \text{shi} }$.

\end{frame}
% ##################





% ##################
\begin{frame}
  \frametitle{Pytania od prof. Jana Derezińskiego}


  % \textit{Czemu autor stwierdził, że zasada przesunięcia energii
  %   odniesienia stosuje~się inaczej dla układów o~skończonej i~nieskończonej
  %   liczbie stopni swobody?}

  % Rozpatrzymy dwa układ, jeden o~skończonej, drugi o~przeliczalnej liczbie
  % stopni swobody, takie że energia całkowita jest sumą energii przypadającej
  % na każdy stopień swobody.

  % \vspace{-2em}



  % \begin{subequations}

  %   \begin{align}
  %     \label{eq:Algebraiczne-podejscie-ETC-01}
  %     E_{ \text{tot, fin} }
  %     &=
  %       \sum_{ i = 1 }^{ N } E_{ i } < +\infty, \\
  %       \label{eq:Algebraiczne-podejscie-ETC-01}
  %     E_{ \text{tot, con} }
  %     &=
  %       \sum_{ i = 1 }^{ \infty } E_{ i } < +\infty.
  %   \end{align}

  % \end{subequations}

  % Dla układu o~skończonej liczbie stopni swobody przesunięcie energii
  % każdego stopnia swobody od skończoną wartość $E_{ \text{shi} }$ powoduje
  % pojawienie~się skończonej stałej $N E_{ \text{shi} }$, co nie stanowi
  % żadnego problemu.
  Dla układu o~przeliczalnej liczbie stopni swobody ta sama
  procedura powoduje pojawienie się „nieskończonej stałej”
  $\infty \cdot E_{ \text{shi} }$:
  \begin{equation}
    \label{eq:Algebraiczne-podejscie-ETC-01}
    E_{ \text{tot, con} } \to E_{ \text{tot, con} } + \infty \cdot E_{ \text{shi} }.
  \end{equation}
  Z~punktu widzenia matematyki wyrażenie na energię całkowitą jest
  w~zasadzie pozbawione sensu. Z~punktu widzenia fizyki wymaga co najmniej
  jakiejś renormalizacji.

  Ponieważ uważam, że takie „nieskończone stałe” należy wykluczyć
  z~rozważań w~każdym przypadku, gdy to jest możliwe, według mnie taka
  procedura przesuwania punktu odniesienia energii być zabroniona dla układu
  o~nieskończonej liczbie stopni swobody.

\end{frame}
% ##################





% ##################
\begin{frame}
  \frametitle{Pytania od prof. Andrzeja Horzeli}


  \textit{Dlaczego w~pracy nie omówiono szerzej modeli efektu
    Casimira bazujących na~podejściu Lifszyca?}

  W~modelach opartych o~podejście Lifszyca bada~się wpływ szerokiego
  zakresu zjawisk z~fizyki fazy skondensowanej na efekt Casimira.
  Jest to tematyka zbyt szeroka, by móc ją choćby pobieżnie omówić w~pracy
  poświęconej prostemu modelowi dwóch sferycznie symetrycznych ciał.
  Dlatego ograniczyliśmy jej dyskusję do kilku najważniejszych
  w~naszej ocenie zagadnień.

  Ponadto już w~oryginalnej pracy Lifszyca, by uprościć obliczenia
  dokonujemy przejścia na zmienne czysto urojone. W~większości literatury
  przedmiotu operacja ta nie jest wykonywana na poziomie ścisłości,
  który fizyka matematyczna może uznać za satysfakcjonujący. Doszliśmy
  do~wniosku, że~dyskusja tego zagadnienia zaprowadziłaby nas zbyt daleko
  od efektu Casimira, co stanowiło dodatkowy powód, by nie poruszać jej
  głębiej.

\end{frame}
% ##################





% ##################
\begin{frame}
  \frametitle{Pytania od prof. Andrzeja Horzeli}


  \textit{Jakie jest uzasadnienie założenia, że~funkcja $g( \vecx )$ ma
    nośnik zwarty?}

  Naszym celem było zbadanie podstawowych własności efektu Casimira
  w~układzie z~dwoma sferycznie symetrycznymi ciałami. Przyjęcie,
  że~funkcja $g( \vecx )$ ma zwarty nośnik, odpowiada założeniu,
  iż~ciała mają dobrze określoną klasyczną granicę.

  Gdybyś my przyjęli jako model ciał, przykładowo funkcję klasy Schwartza,
  moglibyśmy argumentować, iż~jest to sposób uwzględnienia pewnych efektów
  powierzchniowych. Jednak tego typu efekty, choć ważne, są drugorzędne
  wobec takich, jak zależność przenikalności elektrycznej
  $\varepsilon( \omega )$ od częstości fal elektromagnetycznych, które naturalnie
  uwzględnia podejście Lifszyca. Ponieważ nie dysponujemy matematycznie
  satysfakcjonujący model efektu Casimira uwzględniającego funkcję
  $\varepsilon( \omega )$, próba uwzględnienia efektów powierzchniowych, nie jest obecnie
  uzasadniona.

\end{frame}
% ##################





% ##################
\begin{frame}
  \frametitle{Pytania od prof. Andrzeja Horzeli}


  \textit{Czy i~w~jaki sposób metody użyte w~rozprawie można przenieść
    na przypadek pola elektromagnetycznego?}

  W~pracy z~2006 roku Herdegen pokazał, że~dla problemu dwóch płyt,
  zagadnienie pola elektromagnetycznego redukuje~się do rozwiązania problemu
  pola skalarnego dla dwóch typów „rozmytych” warunków brzegowych
  \parencite{Herdegen-Quantum-backreaction-ETC-Part-II-Pub-2006}.
  Stwierdzenie czy podobną procedurę można zastosować w~problemie z~dwiema
  quasi-deltami wymaga dobrego zrozumienia oddziaływania pola
  elektromagnetycznego z~potencjałami typu delta i~quasi-potencjałami,
  który to zagadnienie wymaga dalszych studiów.

\end{frame}
% ##################



% % ######################################
% \section{Closing remarks}
% % ######################################














% % ##################
% \begin{frame}
%   \frametitle{Zakończenie}

%   \vspace{7em}


%   \begin{center}

%     \Large

%     Dziękuję za uwagę.

%   \end{center}

% \end{frame}
% % ##################










% ####################################################################
% ####################################################################
% Bibliography

\printbibliography





% ############################
% End of the document

\end{document}
