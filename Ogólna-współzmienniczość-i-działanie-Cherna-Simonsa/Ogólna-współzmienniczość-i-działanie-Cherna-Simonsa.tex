% ---------------------------------------------------------------------
% Basic configuration of Beamera and Jagiellonian
% ---------------------------------------------------------------------
\RequirePackage[l2tabu, orthodox]{nag}



\ifx\PresentationStyle\notset
\def\PresentationStyle{dark}
\fi



\documentclass[10pt,t]{beamer}
\mode<presentation>
\usetheme[style=\PresentationStyle,logoLang=Latin,logoColor=monochromaticJUwhite,JUlogotitle=yes]{jagiellonian}



% ---------------------------------------
% Configuration files of Jagiellonian loceted in catalog preambule
% ---------------------------------------
% Configuration for polish language
% Need description
\usepackage[polish]{babel}
% Need description
\usepackage[MeX]{polski}



% ------------------------------
% Better support of polish chars in technical parts of PDF
% ------------------------------
\hypersetup{pdfencoding=auto,psdextra}

% Package "textpos" give as enviroment "textblock" which is very usefull in
% arranging text on slides.

% This is standard configuration of "textpos"
\usepackage[overlay,absolute]{textpos}

% If you need to see bounds of "textblock's" comment line above and uncomment
% one below.

% Caution! When showboxes option is on significant ammunt of space is add
% to the top of textblock and as such, everyting put in them gone down.
% We need to check how to remove this bug.

% \usepackage[showboxes,overlay,absolute]{textpos}



% Setting scale length for package "textpos"
\setlength{\TPHorizModule}{10mm}
\setlength{\TPVertModule}{\TPHorizModule}


% ---------------------------------------
% TikZ
% ---------------------------------------
% Importing TikZ libraries
\usetikzlibrary{arrows.meta}
\usetikzlibrary{positioning}





% % Configuration package "bm" that need for making bold symbols
% \newcommand{\bmmax}{0}
% \newcommand{\hmmax}{0}
% \usepackage{bm}




% ---------------------------------------
% Packages for scientific texts
% ---------------------------------------
% \let\lll\undefined  % Sometimes you must use this line to allow
% "amsmath" package to works with packages with packages for polish
% languge imported
% /preambul/LanguageSettings/JagiellonianPolishLanguageSettings.tex.
% This comments (probably) removes polish letter Ł.
\usepackage{amsmath}  % Packages from American Mathematical Society (AMS)
\usepackage{amssymb}
\usepackage{amscd}
\usepackage{amsthm}
\usepackage{siunitx}  % Package for typsetting SI units.
\usepackage{upgreek}  % Better looking greek letters.
% Example of using upgreek: pi = \uppi


\usepackage{calrsfs}  % Zmienia czcionkę kaligraficzną w \mathcal
% na ładniejszą. Może w innych miejscach robi to samo, ale o tym nic
% nie wiem.










% ---------------------------------------
% Packages written for lectures "Geometria 3D dla twórców gier wideo"
% ---------------------------------------
% \usepackage{./ProgramowanieSymulacjiFizykiPaczki/ProgramowanieSymulacjiFizyki}
% \usepackage{./ProgramowanieSymulacjiFizykiPaczki/ProgramowanieSymulacjiFizykiIndeksy}
% \usepackage{./ProgramowanieSymulacjiFizykiPaczki/ProgramowanieSymulacjiFizykiTikZStyle}





% !!!!!!!!!!!!!!!!!!!!!!!!!!!!!!
% !!!!!!!!!!!!!!!!!!!!!!!!!!!!!!
% EVIL STUFF
\if\JUlogotitle1
\edef\LogoJUPath{LogoJU_\JUlogoLang/LogoJU_\JUlogoShape_\JUlogoColor.pdf}
\titlegraphic{\hfill\includegraphics[scale=0.22]
{./JagiellonianPictures/\LogoJUPath}}
\fi
% ---------------------------------------
% Commands for handling colors
% ---------------------------------------


% Command for setting normal text color for some text in math modestyle
% Text color depend on used style of Jagiellonian

% Beamer version of command
\newcommand{\TextWithNormalTextColor}[1]{%
  {\color{jNormalTextFGColor}
    \setbeamercolor{math text}{fg=jNormalTextFGColor} {#1}}
}

% Article and similar classes version of command
% \newcommand{\TextWithNormalTextColor}[1]{%
%   {\color{jNormalTextsFGColor} {#1}}
% }



% Beamer version of command
\newcommand{\NormalTextInMathMode}[1]{%
  {\color{jNormalTextFGColor}
    \setbeamercolor{math text}{fg=jNormalTextFGColor} \text{#1}}
}


% Article and similar classes version of command
% \newcommand{\NormalTextInMathMode}[1]{%
%   {\color{jNormalTextsFGColor} \text{#1}}
% }




% Command that sets color of some mathematical text to the same color
% that has normal text in header (?)

% Beamer version of the command
\newcommand{\MathTextFrametitleFGColor}[1]{%
  {\color{jFrametitleFGColor}
    \setbeamercolor{math text}{fg=jFrametitleFGColor} #1}
}

% Article and similar classes version of the command
% \newcommand{\MathTextWhiteColor}[1]{{\color{jFrametitleFGColor} #1}}





% Command for setting color of alert text for some text in math modestyle

% Beamer version of the command
\newcommand{\MathTextAlertColor}[1]{%
  {\color{jOrange} \setbeamercolor{math text}{fg=jOrange} #1}
}

% Article and similar classes version of the command
% \newcommand{\MathTextAlertColor}[1]{{\color{jOrange} #1}}





% Command that allow you to sets chosen color as the color of some text into
% math mode. Due to some nuances in the way that Beamer handle colors
% it not work in all cases. We hope that in the future we will improve it.

% Beamer version of the command
\newcommand{\SetMathTextsColor}[2]{%
  {\color{#1} \setbeamercolor{math text}{fg=#1} #2}
}


% Article and similar classes version of the command
% \newcommand{\SetMathTextColor}[2]{{\color{#1} #2}}










% ---------------------------------------
% Commands for setting background pictures for some slides
% ---------------------------------------
\newcommand{\TitleBackgroundPicture}
{./PresentationPictures/CommonPictures/Cute_dragon_BG_dark.png}
\newcommand{\SectionBackgroundPicture}
{./PresentationPictures/CommonPictures/Cute_dragon_small_BG_light.png}



\newcommand{\TitleSlideWithPicture}{
  \begingroup

  \usebackgroundtemplate{ % \hspace*{-11.5em}
    \includegraphics[height=\paperheight]{\TitleBackgroundPicture}}

  \maketitle

  \endgroup
}





\newcommand{\SectionSlideWithPicture}[1]{%
  \begingroup

  \usebackgroundtemplate{ % \hspace*{-11.5em}
    \includegraphics[height=\paperheight]{\SectionBackgroundPicture}}

  \setbeamercolor{titlelike}{fg=normal text.fg}

  \section{#1}

  \endgroup
}





\newcommand{\EndingSlide}[1]{%
  \begin{frame}[standout]

    \begingroup

    \color{jFrametitleFGColor}

    #1

    \endgroup

  \end{frame}
}










% ---------------------------------------
% Packages, libraries and their configuration
% ---------------------------------------
\usepackage{mathcommands}





% ---------------------------------------
% Configuration for this particular presentation
% ---------------------------------------










% ---------------------------------------------------------------------
\title{Ogólna współzmienniczość i~działanie Cherna-Simonsa}

\author{Kamil Ziemian,
  \texttt{kziemianfvt@gmail.com}}


\institute{II rok, fizyka teoretyczna, studia magisterskie}

\date[25 IV 2013]{25 kwietnia 2013 r.}
% --------------------------------------------------------------------










% ####################################################################
% Początek dokumentu
\begin{document}
% ####################################################################





% Wyrównanie do lewej z łamaniem wyrazów

\RaggedRight





% ######################################
\maketitle
% ######################################





% ##################
\begin{frame}
  \frametitle{Czym jest ogólna współzmienniczość?}


  Matematyczna strona teorii pola.
  Rozpatrzmy teorię pola od jej strony matematycznej. Możemy wtedy
  wyróżnić w niej ciąg warunkujących się poziomów struktur:
  \begin{itemize}

  \item Topologiczna.

  \item Różniczkowa.

  \item Pseudoriemannowska, zdana przez tensor metryczny $g_{ \mu \nu }$.

  \item Algebraiczna.

  \end{itemize}



  Zazwyczaj. Pracujemy na dwóch najwyższych poziomach, zaś teorie nie
  przejawiają oczywistych własności odbijających ich strukturę
  topologiczną i~różniczkową.

\end{frame}
% ##################





% ##################
\begin{frame}
  \frametitle{Czym jest ogólna współzmienniczość?}


  Ładunek topologiczny. Niemniej znane są przykłady wielkość które są
  zachowywane przez dowolne odwzorowania zachowujące topologię. Wielkość
  te są znane w~fizyce jako ładunki topologiczne.

  Wielkość współzmiennicze jest to wielkości którą można wyliczyć bez
  korzystania z metryki $g_{ \mu \nu }$ \cite{WittenQFTAndJonesPolynomial1989}.

\end{frame}
% ###################





% ##################
\begin{frame}
  \frametitle{Jak otrzymać teorię współzmienniczą?}


  Metoda tradycyjna wzorując się na ogólnej teorii względności napiszmy
  działanie jako:
  \begin{equation}
    \label{eq:Ogolna-wspolzmienniczosc-01}
    S_{ \mathrm{I} } =
    \int_{ M } d^{ D } x \, \sqrt{ g } ( R + \Lcal_{ M } ).
  \end{equation}
  Taka teoria zależy od metryki, jednak traktuje ją jako zmienną
  dynamiczną. Co należy zrobić aby teoria była w pełni od niej
  niezależna?

  Odpowiedź
  \begin{equation}
    \label{eq:Ogolna-wspolzmienniczosc-02}
    S =
    \int \Dcal g \int_{ M } d^{ D } x \, \sqrt{ g } ( R + \Lcal_{ M } ).
  \end{equation}

\end{frame}
% ##################





% ##################
\begin{frame}
  \frametitle{Podejście topologiczno-różniczkowe}


  Główny problem.
  Tensor metryczny definiuje nam, wraz z orientacją, niezmienniczy
  element całkowania, który jest potrzebny do dobrego zdefiniowania
  działania.

  Jednak w szczególnych przypadkach, może się okazać że wystarczy
  uboższa struktura matematyczna. Tak jest na przykład dla
  trójwymiarowej zorientowanej rozmaitości z polami typu Yanga-Millsa.

\end{frame}
% ##################





% ##################
\begin{frame}
  \frametitle{Teoria Cherna-Simonsa}


  Oznaczenia
  \begin{itemize}

  \item $A_{ i }$ -- pola cechowania.

  \item $[ A_{ i }, A_{ j } ]$ -- odpowiednik wyrażenia
    $\Gamma^{ i }_{ k l } v^{ k } w^{ l }$ z~ogólnej teorii
    względności.

  \end{itemize}


  Lagranżjan
  \begin{equation}
    \label{eq:Ogolna-wspolzmienniczosc-03}
    \begin{split}
      \Lcal
      &= \frac{ k }{ 4 \pi } \int_{ M } \Tr ( A \wedge dA
        + \frac{ 2 }{ 3 } A \wedge A \wedge A ) \\
      &= \frac{ k }{ 8 \pi } \int_{ M } \varepsilon^{ i j k } \Tr \big( A_{ i } ( \partial_{ j } A_{ k }
        - \partial_{ k } A_{ j } ) + \frac{ 2 }{ 3 } A_{ i } [ A_{ j }, A_{ k } ] \big).
      \end{split}
    \end{equation}

\end{frame}
% ##################










% ######################################
\appendix
% ######################################





% ######################################
\EndingSlide{Dziękuję! Pytania?}
% ######################################










% ##################
\begin{frame}


  \bibliographystyle{plalpha}

  \bibliography{PhilNaturArticles}{}

\end{frame}
% ##################










% ############################

% Koniec dokumentu
\end{document}
