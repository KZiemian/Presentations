% ---------------------------------------------------------------------
% Basic configuration of Beamera class and Jagiellonian theme
% ---------------------------------------------------------------------
\RequirePackage[l2tabu, orthodox]{nag}



\ifx\PresentationStyle\notset
  \def\PresentationStyle{dark}
\fi



% Options: t -- align text to the top of the frame
\documentclass[10pt,t]{beamer}
\mode<presentation>
\usetheme[style=\PresentationStyle]{jagiellonian}





% ---------------------------------------
% Procesing configuration files of Jagiellonian theme loceted in directory
% "preambule".
% ---------------------------------------
% Configuration for polish language
% Need description
\usepackage[polish]{babel}
% Need description
\usepackage[MeX]{polski}



% ------------------------------
% Better support of polish chars in technical parts of PDF
% ------------------------------
\hypersetup{pdfencoding=auto,psdextra}

% Package "textpos" give as enviroment "textblock" which is very usefull in
% arranging text on slides.

% This is standard configuration of "textpos"
\usepackage[overlay,absolute]{textpos}

% If you need to see bounds of "textblock's" comment line above and uncomment
% one below.

% Caution! When showboxes option is on significant ammunt of space is add
% to the top of textblock and as such, everyting put in them gone down.
% We need to check how to remove this bug.

% \usepackage[showboxes,overlay,absolute]{textpos}



% Setting scale length for package "textpos"
\setlength{\TPHorizModule}{10mm}
\setlength{\TPVertModule}{\TPHorizModule}


% ---------------------------------------
% TikZ
% ---------------------------------------
% Importing TikZ libraries
\usetikzlibrary{arrows.meta}
\usetikzlibrary{positioning}





% % Configuration package "bm" that need for making bold symbols
% \newcommand{\bmmax}{0}
% \newcommand{\hmmax}{0}
% \usepackage{bm}




% ---------------------------------------
% Packages for scientific texts
% ---------------------------------------
% \let\lll\undefined  % Sometimes you must use this line to allow
% "amsmath" package to works with packages with packages for polish
% languge imported
% /preambul/LanguageSettings/JagiellonianPolishLanguageSettings.tex.
% This comments (probably) removes polish letter Ł.
\usepackage{amsmath}  % Packages from American Mathematical Society (AMS)
\usepackage{amssymb}
\usepackage{amscd}
\usepackage{amsthm}
\usepackage{siunitx}  % Package for typsetting SI units.
\usepackage{upgreek}  % Better looking greek letters.
% Example of using upgreek: pi = \uppi


\usepackage{calrsfs}  % Zmienia czcionkę kaligraficzną w \mathcal
% na ładniejszą. Może w innych miejscach robi to samo, ale o tym nic
% nie wiem.










% ---------------------------------------
% Packages written for lectures "Geometria 3D dla twórców gier wideo"
% ---------------------------------------
% \usepackage{./ProgramowanieSymulacjiFizykiPaczki/ProgramowanieSymulacjiFizyki}
% \usepackage{./ProgramowanieSymulacjiFizykiPaczki/ProgramowanieSymulacjiFizykiIndeksy}
% \usepackage{./ProgramowanieSymulacjiFizykiPaczki/ProgramowanieSymulacjiFizykiTikZStyle}





% !!!!!!!!!!!!!!!!!!!!!!!!!!!!!!
% !!!!!!!!!!!!!!!!!!!!!!!!!!!!!!
% EVIL STUFF
\if\JUlogotitle1
\edef\LogoJUPath{LogoJU_\JUlogoLang/LogoJU_\JUlogoShape_\JUlogoColor.pdf}
\titlegraphic{\hfill\includegraphics[scale=0.22]
{./JagiellonianPictures/\LogoJUPath}}
\fi
% ---------------------------------------
% Commands for handling colors
% ---------------------------------------


% Command for setting normal text color for some text in math modestyle
% Text color depend on used style of Jagiellonian

% Beamer version of command
\newcommand{\TextWithNormalTextColor}[1]{%
  {\color{jNormalTextFGColor}
    \setbeamercolor{math text}{fg=jNormalTextFGColor} {#1}}
}

% Article and similar classes version of command
% \newcommand{\TextWithNormalTextColor}[1]{%
%   {\color{jNormalTextsFGColor} {#1}}
% }



% Beamer version of command
\newcommand{\NormalTextInMathMode}[1]{%
  {\color{jNormalTextFGColor}
    \setbeamercolor{math text}{fg=jNormalTextFGColor} \text{#1}}
}


% Article and similar classes version of command
% \newcommand{\NormalTextInMathMode}[1]{%
%   {\color{jNormalTextsFGColor} \text{#1}}
% }




% Command that sets color of some mathematical text to the same color
% that has normal text in header (?)

% Beamer version of the command
\newcommand{\MathTextFrametitleFGColor}[1]{%
  {\color{jFrametitleFGColor}
    \setbeamercolor{math text}{fg=jFrametitleFGColor} #1}
}

% Article and similar classes version of the command
% \newcommand{\MathTextWhiteColor}[1]{{\color{jFrametitleFGColor} #1}}





% Command for setting color of alert text for some text in math modestyle

% Beamer version of the command
\newcommand{\MathTextAlertColor}[1]{%
  {\color{jOrange} \setbeamercolor{math text}{fg=jOrange} #1}
}

% Article and similar classes version of the command
% \newcommand{\MathTextAlertColor}[1]{{\color{jOrange} #1}}





% Command that allow you to sets chosen color as the color of some text into
% math mode. Due to some nuances in the way that Beamer handle colors
% it not work in all cases. We hope that in the future we will improve it.

% Beamer version of the command
\newcommand{\SetMathTextColor}[2]{%
  {\color{#1} \setbeamercolor{math text}{fg=#1} #2}
}


% Article and similar classes version of the command
% \newcommand{\SetMathTextColor}[2]{{\color{#1} #2}}










% ---------------------------------------
% Commands for few special slides
% ---------------------------------------
\newcommand{\EndingSlide}[1]{%
  \begin{frame}[standout]

    \begingroup

    \color{jFrametitleFGColor}

    #1

    \endgroup

  \end{frame}
}










% ---------------------------------------
% Commands for setting background pictures for some slides
% ---------------------------------------
\newcommand{\TitleBackgroundPicture}
{./JagiellonianPictures/Backgrounds/LajkonikDark.png}
\newcommand{\SectionBackgroundPicture}
{./JagiellonianPictures/Backgrounds/LajkonikLight.png}



\newcommand{\TitleSlideWithPicture}{%
  \begingroup

  \usebackgroundtemplate{%
    \includegraphics[height=\paperheight]{\TitleBackgroundPicture}}

  \maketitle

  \endgroup
}





\newcommand{\SectionSlideWithPicture}[1]{%
  \begingroup

  \usebackgroundtemplate{%
    \includegraphics[height=\paperheight]{\SectionBackgroundPicture}}

  \setbeamercolor{titlelike}{fg=normal text.fg}

  \section{#1}

  \endgroup
}










% ---------------------------------------
% Commands for lectures "Geometria 3D dla twórców gier wideo"
% Polish version
% ---------------------------------------
% Komendy teraz wykomentowane były potrzebne, gdy loga były na niebieskim
% tle, nie na białym. A są na białym bo tego chcieli w biurze projektu.
% \newcommand{\FundingLogoWhitePicturePL}
% {./PresentationPictures/CommonPictures/logotypFundusze_biale_bez_tla2.pdf}
\newcommand{\FundingLogoColorPicturePL}
{./PresentationPictures/CommonPictures/European_Funds_color_PL.pdf}
% \newcommand{\EULogoWhitePicturePL}
% {./PresentationPictures/CommonPictures/logotypUE_biale_bez_tla2.pdf}
\newcommand{\EUSocialFundLogoColorPicturePL}
{./PresentationPictures/CommonPictures/EU_Social_Fund_color_PL.pdf}
% \newcommand{\ZintegrUJLogoWhitePicturePL}
% {./PresentationPictures/CommonPictures/zintegruj-logo-white.pdf}
\newcommand{\ZintegrUJLogoColorPicturePL}
{./PresentationPictures/CommonPictures/ZintegrUJ_color.pdf}
\newcommand{\JULogoColorPicturePL}
{./JagiellonianPictures/LogoJU_PL/LogoJU_A_color.pdf}





\newcommand{\GeometryThreeDSpecialBeginningSlidePL}{%
  \begin{frame}[standout]

    \begin{textblock}{11}(1,0.7)

      \begin{flushleft}

        \mdseries

        \footnotesize

        \color{jFrametitleFGColor}

        Materiał powstał w ramach projektu współfinansowanego ze środków
        Unii Europejskiej w ramach Europejskiego Funduszu Społecznego
        POWR.03.05.00-00-Z309/17-00.

      \end{flushleft}

    \end{textblock}





    \begin{textblock}{10}(0,2.2)

      \tikz \fill[color=jBackgroundStyleLight] (0,0) rectangle (12.8,-1.5);

    \end{textblock}


    \begin{textblock}{3.2}(1,2.45)

      \includegraphics[scale=0.3]{\FundingLogoColorPicturePL}

    \end{textblock}


    \begin{textblock}{2.5}(3.7,2.5)

      \includegraphics[scale=0.2]{\JULogoColorPicturePL}

    \end{textblock}


    \begin{textblock}{2.5}(6,2.4)

      \includegraphics[scale=0.1]{\ZintegrUJLogoColorPicturePL}

    \end{textblock}


    \begin{textblock}{4.2}(8.4,2.6)

      \includegraphics[scale=0.3]{\EUSocialFundLogoColorPicturePL}

    \end{textblock}

  \end{frame}
}



\newcommand{\GeometryThreeDTwoSpecialBeginningSlidesPL}{%
  \begin{frame}[standout]

    \begin{textblock}{11}(1,0.7)

      \begin{flushleft}

        \mdseries

        \footnotesize

        \color{jFrametitleFGColor}

        Materiał powstał w ramach projektu współfinansowanego ze środków
        Unii Europejskiej w ramach Europejskiego Funduszu Społecznego
        POWR.03.05.00-00-Z309/17-00.

      \end{flushleft}

    \end{textblock}





    \begin{textblock}{10}(0,2.2)

      \tikz \fill[color=jBackgroundStyleLight] (0,0) rectangle (12.8,-1.5);

    \end{textblock}


    \begin{textblock}{3.2}(1,2.45)

      \includegraphics[scale=0.3]{\FundingLogoColorPicturePL}

    \end{textblock}


    \begin{textblock}{2.5}(3.7,2.5)

      \includegraphics[scale=0.2]{\JULogoColorPicturePL}

    \end{textblock}


    \begin{textblock}{2.5}(6,2.4)

      \includegraphics[scale=0.1]{\ZintegrUJLogoColorPicturePL}

    \end{textblock}


    \begin{textblock}{4.2}(8.4,2.6)

      \includegraphics[scale=0.3]{\EUSocialFundLogoColorPicturePL}

    \end{textblock}

  \end{frame}





  \TitleSlideWithPicture
}



\newcommand{\GeometryThreeDSpecialEndingSlidePL}{%
  \begin{frame}[standout]

    \begin{textblock}{11}(1,0.7)

      \begin{flushleft}

        \mdseries

        \footnotesize

        \color{jFrametitleFGColor}

        Materiał powstał w ramach projektu współfinansowanego ze środków
        Unii Europejskiej w~ramach Europejskiego Funduszu Społecznego
        POWR.03.05.00-00-Z309/17-00.

      \end{flushleft}

    \end{textblock}





    \begin{textblock}{10}(0,2.2)

      \tikz \fill[color=jBackgroundStyleLight] (0,0) rectangle (12.8,-1.5);

    \end{textblock}


    \begin{textblock}{3.2}(1,2.45)

      \includegraphics[scale=0.3]{\FundingLogoColorPicturePL}

    \end{textblock}


    \begin{textblock}{2.5}(3.7,2.5)

      \includegraphics[scale=0.2]{\JULogoColorPicturePL}

    \end{textblock}


    \begin{textblock}{2.5}(6,2.4)

      \includegraphics[scale=0.1]{\ZintegrUJLogoColorPicturePL}

    \end{textblock}


    \begin{textblock}{4.2}(8.4,2.6)

      \includegraphics[scale=0.3]{\EUSocialFundLogoColorPicturePL}

    \end{textblock}





    \begin{textblock}{11}(1,4)

      \begin{flushleft}

        \mdseries

        \footnotesize

        \RaggedRight

        \color{jFrametitleFGColor}

        Treść niniejszego wykładu jest udostępniona na~licencji
        Creative Commons (\textsc{cc}), z~uzna\-niem autorstwa
        (\textsc{by}) oraz udostępnianiem na tych samych warunkach
        (\textsc{sa}). Rysunki i~wy\-kresy zawarte w~wykładzie są
        autorstwa dr.~hab.~Pawła Węgrzyna et~al. i~są dostępne
        na tej samej licencji, o~ile nie wskazano inaczej.
        W~prezentacji wykorzystano temat Beamera Jagiellonian,
        oparty na~temacie Metropolis Matthiasa Vogelgesanga,
        dostępnym na licencji \LaTeX{} Project Public License~1.3c
        pod adresem: \colorhref{https://github.com/matze/mtheme}
        {https://github.com/matze/mtheme}.

        Projekt typograficzny: Iwona Grabska-Gradzińska \\
        Skład: Kamil Ziemian;
        Korekta: Wojciech Palacz \\
        Modele: Dariusz Frymus, Kamil Nowakowski \\
        Rysunki i~wykresy: Kamil Ziemian, Paweł Węgrzyn, Wojciech Palacz

      \end{flushleft}

    \end{textblock}

  \end{frame}
}



\newcommand{\GeometryThreeDTwoSpecialEndingSlidesPL}[1]{%
  \begin{frame}[standout]


    \begin{textblock}{11}(1,0.7)

      \begin{flushleft}

        \mdseries

        \footnotesize

        \color{jFrametitleFGColor}

        Materiał powstał w ramach projektu współfinansowanego ze środków
        Unii Europejskiej w~ramach Europejskiego Funduszu Społecznego
        POWR.03.05.00-00-Z309/17-00.

      \end{flushleft}

    \end{textblock}





    \begin{textblock}{10}(0,2.2)

      \tikz \fill[color=jBackgroundStyleLight] (0,0) rectangle (12.8,-1.5);

    \end{textblock}


    \begin{textblock}{3.2}(1,2.45)

      \includegraphics[scale=0.3]{\FundingLogoColorPicturePL}

    \end{textblock}


    \begin{textblock}{2.5}(3.7,2.5)

      \includegraphics[scale=0.2]{\JULogoColorPicturePL}

    \end{textblock}


    \begin{textblock}{2.5}(6,2.4)

      \includegraphics[scale=0.1]{\ZintegrUJLogoColorPicturePL}

    \end{textblock}


    \begin{textblock}{4.2}(8.4,2.6)

      \includegraphics[scale=0.3]{\EUSocialFundLogoColorPicturePL}

    \end{textblock}





    \begin{textblock}{11}(1,4)

      \begin{flushleft}

        \mdseries

        \footnotesize

        \RaggedRight

        \color{jFrametitleFGColor}

        Treść niniejszego wykładu jest udostępniona na~licencji
        Creative Commons (\textsc{cc}), z~uzna\-niem autorstwa
        (\textsc{by}) oraz udostępnianiem na tych samych warunkach
        (\textsc{sa}). Rysunki i~wy\-kresy zawarte w~wykładzie są
        autorstwa dr.~hab.~Pawła Węgrzyna et~al. i~są dostępne
        na tej samej licencji, o~ile nie wskazano inaczej.
        W~prezentacji wykorzystano temat Beamera Jagiellonian,
        oparty na~temacie Metropolis Matthiasa Vogelgesanga,
        dostępnym na licencji \LaTeX{} Project Public License~1.3c
        pod adresem: \colorhref{https://github.com/matze/mtheme}
        {https://github.com/matze/mtheme}.

        Projekt typograficzny: Iwona Grabska-Gradzińska \\
        Skład: Kamil Ziemian;
        Korekta: Wojciech Palacz \\
        Modele: Dariusz Frymus, Kamil Nowakowski \\
        Rysunki i~wykresy: Kamil Ziemian, Paweł Węgrzyn, Wojciech Palacz

      \end{flushleft}

    \end{textblock}

  \end{frame}





  \begin{frame}[standout]

    \begingroup

    \color{jFrametitleFGColor}

    #1

    \endgroup

  \end{frame}
}



\newcommand{\GeometryThreeDSpecialEndingSlideVideoPL}{%
  \begin{frame}[standout]

    \begin{textblock}{11}(1,0.7)

      \begin{flushleft}

        \mdseries

        \footnotesize

        \color{jFrametitleFGColor}

        Materiał powstał w ramach projektu współfinansowanego ze środków
        Unii Europejskiej w~ramach Europejskiego Funduszu Społecznego
        POWR.03.05.00-00-Z309/17-00.

      \end{flushleft}

    \end{textblock}





    \begin{textblock}{10}(0,2.2)

      \tikz \fill[color=jBackgroundStyleLight] (0,0) rectangle (12.8,-1.5);

    \end{textblock}


    \begin{textblock}{3.2}(1,2.45)

      \includegraphics[scale=0.3]{\FundingLogoColorPicturePL}

    \end{textblock}


    \begin{textblock}{2.5}(3.7,2.5)

      \includegraphics[scale=0.2]{\JULogoColorPicturePL}

    \end{textblock}


    \begin{textblock}{2.5}(6,2.4)

      \includegraphics[scale=0.1]{\ZintegrUJLogoColorPicturePL}

    \end{textblock}


    \begin{textblock}{4.2}(8.4,2.6)

      \includegraphics[scale=0.3]{\EUSocialFundLogoColorPicturePL}

    \end{textblock}





    \begin{textblock}{11}(1,4)

      \begin{flushleft}

        \mdseries

        \footnotesize

        \RaggedRight

        \color{jFrametitleFGColor}

        Treść niniejszego wykładu jest udostępniona na~licencji
        Creative Commons (\textsc{cc}), z~uzna\-niem autorstwa
        (\textsc{by}) oraz udostępnianiem na tych samych warunkach
        (\textsc{sa}). Rysunki i~wy\-kresy zawarte w~wykładzie są
        autorstwa dr.~hab.~Pawła Węgrzyna et~al. i~są dostępne
        na tej samej licencji, o~ile nie wskazano inaczej.
        W~prezentacji wykorzystano temat Beamera Jagiellonian,
        oparty na~temacie Metropolis Matthiasa Vogelgesanga,
        dostępnym na licencji \LaTeX{} Project Public License~1.3c
        pod adresem: \colorhref{https://github.com/matze/mtheme}
        {https://github.com/matze/mtheme}.

        Projekt typograficzny: Iwona Grabska-Gradzińska;
        Skład: Kamil Ziemian \\
        Korekta: Wojciech Palacz;
        Modele: Dariusz Frymus, Kamil Nowakowski \\
        Rysunki i~wykresy: Kamil Ziemian, Paweł Węgrzyn, Wojciech Palacz \\
        Montaż: Agencja Filmowa Film \& Television Production~-- Zbigniew
        Masklak

      \end{flushleft}

    \end{textblock}

  \end{frame}
}





\newcommand{\GeometryThreeDTwoSpecialEndingSlidesVideoPL}[1]{%
  \begin{frame}[standout]

    \begin{textblock}{11}(1,0.7)

      \begin{flushleft}

        \mdseries

        \footnotesize

        \color{jFrametitleFGColor}

        Materiał powstał w ramach projektu współfinansowanego ze środków
        Unii Europejskiej w~ramach Europejskiego Funduszu Społecznego
        POWR.03.05.00-00-Z309/17-00.

      \end{flushleft}

    \end{textblock}





    \begin{textblock}{10}(0,2.2)

      \tikz \fill[color=jBackgroundStyleLight] (0,0) rectangle (12.8,-1.5);

    \end{textblock}


    \begin{textblock}{3.2}(1,2.45)

      \includegraphics[scale=0.3]{\FundingLogoColorPicturePL}

    \end{textblock}


    \begin{textblock}{2.5}(3.7,2.5)

      \includegraphics[scale=0.2]{\JULogoColorPicturePL}

    \end{textblock}


    \begin{textblock}{2.5}(6,2.4)

      \includegraphics[scale=0.1]{\ZintegrUJLogoColorPicturePL}

    \end{textblock}


    \begin{textblock}{4.2}(8.4,2.6)

      \includegraphics[scale=0.3]{\EUSocialFundLogoColorPicturePL}

    \end{textblock}





    \begin{textblock}{11}(1,4)

      \begin{flushleft}

        \mdseries

        \footnotesize

        \RaggedRight

        \color{jFrametitleFGColor}

        Treść niniejszego wykładu jest udostępniona na~licencji
        Creative Commons (\textsc{cc}), z~uzna\-niem autorstwa
        (\textsc{by}) oraz udostępnianiem na tych samych warunkach
        (\textsc{sa}). Rysunki i~wy\-kresy zawarte w~wykładzie są
        autorstwa dr.~hab.~Pawła Węgrzyna et~al. i~są dostępne
        na tej samej licencji, o~ile nie wskazano inaczej.
        W~prezentacji wykorzystano temat Beamera Jagiellonian,
        oparty na~temacie Metropolis Matthiasa Vogelgesanga,
        dostępnym na licencji \LaTeX{} Project Public License~1.3c
        pod adresem: \colorhref{https://github.com/matze/mtheme}
        {https://github.com/matze/mtheme}.

        Projekt typograficzny: Iwona Grabska-Gradzińska;
        Skład: Kamil Ziemian \\
        Korekta: Wojciech Palacz;
        Modele: Dariusz Frymus, Kamil Nowakowski \\
        Rysunki i~wykresy: Kamil Ziemian, Paweł Węgrzyn, Wojciech Palacz \\
        Montaż: Agencja Filmowa Film \& Television Production~-- Zbigniew
        Masklak

      \end{flushleft}

    \end{textblock}

  \end{frame}





  \begin{frame}[standout]


    \begingroup

    \color{jFrametitleFGColor}

    #1

    \endgroup

  \end{frame}
}










% ---------------------------------------
% Commands for lectures "Geometria 3D dla twórców gier wideo"
% English version
% ---------------------------------------
% \newcommand{\FundingLogoWhitePictureEN}
% {./PresentationPictures/CommonPictures/logotypFundusze_biale_bez_tla2.pdf}
\newcommand{\FundingLogoColorPictureEN}
{./PresentationPictures/CommonPictures/European_Funds_color_EN.pdf}
% \newcommand{\EULogoWhitePictureEN}
% {./PresentationPictures/CommonPictures/logotypUE_biale_bez_tla2.pdf}
\newcommand{\EUSocialFundLogoColorPictureEN}
{./PresentationPictures/CommonPictures/EU_Social_Fund_color_EN.pdf}
% \newcommand{\ZintegrUJLogoWhitePictureEN}
% {./PresentationPictures/CommonPictures/zintegruj-logo-white.pdf}
\newcommand{\ZintegrUJLogoColorPictureEN}
{./PresentationPictures/CommonPictures/ZintegrUJ_color.pdf}
\newcommand{\JULogoColorPictureEN}
{./JagiellonianPictures/LogoJU_EN/LogoJU_A_color.pdf}



\newcommand{\GeometryThreeDSpecialBeginningSlideEN}{%
  \begin{frame}[standout]

    \begin{textblock}{11}(1,0.7)

      \begin{flushleft}

        \mdseries

        \footnotesize

        \color{jFrametitleFGColor}

        This content was created as part of a project co-financed by the
        European Union within the framework of the European Social Fund
        POWR.03.05.00-00-Z309/17-00.

      \end{flushleft}

    \end{textblock}





    \begin{textblock}{10}(0,2.2)

      \tikz \fill[color=jBackgroundStyleLight] (0,0) rectangle (12.8,-1.5);

    \end{textblock}


    \begin{textblock}{3.2}(0.7,2.45)

      \includegraphics[scale=0.3]{\FundingLogoColorPictureEN}

    \end{textblock}


    \begin{textblock}{2.5}(4.15,2.5)

      \includegraphics[scale=0.2]{\JULogoColorPictureEN}

    \end{textblock}


    \begin{textblock}{2.5}(6.35,2.4)

      \includegraphics[scale=0.1]{\ZintegrUJLogoColorPictureEN}

    \end{textblock}


    \begin{textblock}{4.2}(8.4,2.6)

      \includegraphics[scale=0.3]{\EUSocialFundLogoColorPictureEN}

    \end{textblock}

  \end{frame}
}



\newcommand{\GeometryThreeDTwoSpecialBeginningSlidesEN}{%
  \begin{frame}[standout]

    \begin{textblock}{11}(1,0.7)

      \begin{flushleft}

        \mdseries

        \footnotesize

        \color{jFrametitleFGColor}

        This content was created as part of a project co-financed by the
        European Union within the framework of the European Social Fund
        POWR.03.05.00-00-Z309/17-00.

      \end{flushleft}

    \end{textblock}





    \begin{textblock}{10}(0,2.2)

      \tikz \fill[color=jBackgroundStyleLight] (0,0) rectangle (12.8,-1.5);

    \end{textblock}


    \begin{textblock}{3.2}(0.7,2.45)

      \includegraphics[scale=0.3]{\FundingLogoColorPictureEN}

    \end{textblock}


    \begin{textblock}{2.5}(4.15,2.5)

      \includegraphics[scale=0.2]{\JULogoColorPictureEN}

    \end{textblock}


    \begin{textblock}{2.5}(6.35,2.4)

      \includegraphics[scale=0.1]{\ZintegrUJLogoColorPictureEN}

    \end{textblock}


    \begin{textblock}{4.2}(8.4,2.6)

      \includegraphics[scale=0.3]{\EUSocialFundLogoColorPictureEN}

    \end{textblock}

  \end{frame}





  \TitleSlideWithPicture
}



\newcommand{\GeometryThreeDSpecialEndingSlideEN}{%
  \begin{frame}[standout]

    \begin{textblock}{11}(1,0.7)

      \begin{flushleft}

        \mdseries

        \footnotesize

        \color{jFrametitleFGColor}

        This content was created as part of a project co-financed by the
        European Union within the framework of the European Social Fund
        POWR.03.05.00-00-Z309/17-00.

      \end{flushleft}

    \end{textblock}





    \begin{textblock}{10}(0,2.2)

      \tikz \fill[color=jBackgroundStyleLight] (0,0) rectangle (12.8,-1.5);

    \end{textblock}


    \begin{textblock}{3.2}(0.7,2.45)

      \includegraphics[scale=0.3]{\FundingLogoColorPictureEN}

    \end{textblock}


    \begin{textblock}{2.5}(4.15,2.5)

      \includegraphics[scale=0.2]{\JULogoColorPictureEN}

    \end{textblock}


    \begin{textblock}{2.5}(6.35,2.4)

      \includegraphics[scale=0.1]{\ZintegrUJLogoColorPictureEN}

    \end{textblock}


    \begin{textblock}{4.2}(8.4,2.6)

      \includegraphics[scale=0.3]{\EUSocialFundLogoColorPictureEN}

    \end{textblock}





    \begin{textblock}{11}(1,4)

      \begin{flushleft}

        \mdseries

        \footnotesize

        \RaggedRight

        \color{jFrametitleFGColor}

        The content of this lecture is made available under a~Creative
        Commons licence (\textsc{cc}), giving the author the credits
        (\textsc{by}) and putting an obligation to share on the same terms
        (\textsc{sa}). Figures and diagrams included in the lecture are
        authored by Paweł Węgrzyn et~al., and are available under the same
        license unless indicated otherwise.\\ The presentation uses the
        Beamer Jagiellonian theme based on Matthias Vogelgesang’s
        Metropolis theme, available under license \LaTeX{} Project
        Public License~1.3c at: \colorhref{https://github.com/matze/mtheme}
        {https://github.com/matze/mtheme}.

        Typographic design: Iwona Grabska-Gradzińska \\
        \LaTeX{} Typesetting: Kamil Ziemian \\
        Proofreading: Wojciech Palacz,
        Monika Stawicka \\
        3D Models: Dariusz Frymus, Kamil Nowakowski \\
        Figures and charts: Kamil Ziemian, Paweł Węgrzyn, Wojciech Palacz

      \end{flushleft}

    \end{textblock}

  \end{frame}
}



\newcommand{\GeometryThreeDTwoSpecialEndingSlidesEN}[1]{%
  \begin{frame}[standout]


    \begin{textblock}{11}(1,0.7)

      \begin{flushleft}

        \mdseries

        \footnotesize

        \color{jFrametitleFGColor}

        This content was created as part of a project co-financed by the
        European Union within the framework of the European Social Fund
        POWR.03.05.00-00-Z309/17-00.

      \end{flushleft}

    \end{textblock}





    \begin{textblock}{10}(0,2.2)

      \tikz \fill[color=jBackgroundStyleLight] (0,0) rectangle (12.8,-1.5);

    \end{textblock}


    \begin{textblock}{3.2}(0.7,2.45)

      \includegraphics[scale=0.3]{\FundingLogoColorPictureEN}

    \end{textblock}


    \begin{textblock}{2.5}(4.15,2.5)

      \includegraphics[scale=0.2]{\JULogoColorPictureEN}

    \end{textblock}


    \begin{textblock}{2.5}(6.35,2.4)

      \includegraphics[scale=0.1]{\ZintegrUJLogoColorPictureEN}

    \end{textblock}


    \begin{textblock}{4.2}(8.4,2.6)

      \includegraphics[scale=0.3]{\EUSocialFundLogoColorPictureEN}

    \end{textblock}





    \begin{textblock}{11}(1,4)

      \begin{flushleft}

        \mdseries

        \footnotesize

        \RaggedRight

        \color{jFrametitleFGColor}

        The content of this lecture is made available under a~Creative
        Commons licence (\textsc{cc}), giving the author the credits
        (\textsc{by}) and putting an obligation to share on the same terms
        (\textsc{sa}). Figures and diagrams included in the lecture are
        authored by Paweł Węgrzyn et~al., and are available under the same
        license unless indicated otherwise.\\ The presentation uses the
        Beamer Jagiellonian theme based on Matthias Vogelgesang’s
        Metropolis theme, available under license \LaTeX{} Project
        Public License~1.3c at: \colorhref{https://github.com/matze/mtheme}
        {https://github.com/matze/mtheme}.

        Typographic design: Iwona Grabska-Gradzińska \\
        \LaTeX{} Typesetting: Kamil Ziemian \\
        Proofreading: Wojciech Palacz,
        Monika Stawicka \\
        3D Models: Dariusz Frymus, Kamil Nowakowski \\
        Figures and charts: Kamil Ziemian, Paweł Węgrzyn, Wojciech Palacz

      \end{flushleft}

    \end{textblock}

  \end{frame}





  \begin{frame}[standout]

    \begingroup

    \color{jFrametitleFGColor}

    #1

    \endgroup

  \end{frame}
}



\newcommand{\GeometryThreeDSpecialEndingSlideVideoVerOneEN}{%
  \begin{frame}[standout]

    \begin{textblock}{11}(1,0.7)

      \begin{flushleft}

        \mdseries

        \footnotesize

        \color{jFrametitleFGColor}

        This content was created as part of a project co-financed by the
        European Union within the framework of the European Social Fund
        POWR.03.05.00-00-Z309/17-00.

      \end{flushleft}

    \end{textblock}





    \begin{textblock}{10}(0,2.2)

      \tikz \fill[color=jBackgroundStyleLight] (0,0) rectangle (12.8,-1.5);

    \end{textblock}


    \begin{textblock}{3.2}(0.7,2.45)

      \includegraphics[scale=0.3]{\FundingLogoColorPictureEN}

    \end{textblock}


    \begin{textblock}{2.5}(4.15,2.5)

      \includegraphics[scale=0.2]{\JULogoColorPictureEN}

    \end{textblock}


    \begin{textblock}{2.5}(6.35,2.4)

      \includegraphics[scale=0.1]{\ZintegrUJLogoColorPictureEN}

    \end{textblock}


    \begin{textblock}{4.2}(8.4,2.6)

      \includegraphics[scale=0.3]{\EUSocialFundLogoColorPictureEN}

    \end{textblock}





    \begin{textblock}{11}(1,4)

      \begin{flushleft}

        \mdseries

        \footnotesize

        \RaggedRight

        \color{jFrametitleFGColor}

        The content of this lecture is made available under a Creative
        Commons licence (\textsc{cc}), giving the author the credits
        (\textsc{by}) and putting an obligation to share on the same terms
        (\textsc{sa}). Figures and diagrams included in the lecture are
        authored by Paweł Węgrzyn et~al., and are available under the same
        license unless indicated otherwise.\\ The presentation uses the
        Beamer Jagiellonian theme based on Matthias Vogelgesang’s
        Metropolis theme, available under license \LaTeX{} Project
        Public License~1.3c at: \colorhref{https://github.com/matze/mtheme}
        {https://github.com/matze/mtheme}.

        Typographic design: Iwona Grabska-Gradzińska;
        \LaTeX{} Typesetting: Kamil Ziemian \\
        Proofreading: Wojciech Palacz,
        Monika Stawicka \\
        3D Models: Dariusz Frymus, Kamil Nowakowski \\
        Figures and charts: Kamil Ziemian, Paweł Węgrzyn, Wojciech
        Palacz \\
        Film editing: Agencja Filmowa Film \& Television Production~--
        Zbigniew Masklak

      \end{flushleft}

    \end{textblock}

  \end{frame}
}



\newcommand{\GeometryThreeDSpecialEndingSlideVideoVerTwoEN}{%
  \begin{frame}[standout]

    \begin{textblock}{11}(1,0.7)

      \begin{flushleft}

        \mdseries

        \footnotesize

        \color{jFrametitleFGColor}

        This content was created as part of a project co-financed by the
        European Union within the framework of the European Social Fund
        POWR.03.05.00-00-Z309/17-00.

      \end{flushleft}

    \end{textblock}





    \begin{textblock}{10}(0,2.2)

      \tikz \fill[color=jBackgroundStyleLight] (0,0) rectangle (12.8,-1.5);

    \end{textblock}


    \begin{textblock}{3.2}(0.7,2.45)

      \includegraphics[scale=0.3]{\FundingLogoColorPictureEN}

    \end{textblock}


    \begin{textblock}{2.5}(4.15,2.5)

      \includegraphics[scale=0.2]{\JULogoColorPictureEN}

    \end{textblock}


    \begin{textblock}{2.5}(6.35,2.4)

      \includegraphics[scale=0.1]{\ZintegrUJLogoColorPictureEN}

    \end{textblock}


    \begin{textblock}{4.2}(8.4,2.6)

      \includegraphics[scale=0.3]{\EUSocialFundLogoColorPictureEN}

    \end{textblock}





    \begin{textblock}{11}(1,4)

      \begin{flushleft}

        \mdseries

        \footnotesize

        \RaggedRight

        \color{jFrametitleFGColor}

        The content of this lecture is made available under a Creative
        Commons licence (\textsc{cc}), giving the author the credits
        (\textsc{by}) and putting an obligation to share on the same terms
        (\textsc{sa}). Figures and diagrams included in the lecture are
        authored by Paweł Węgrzyn et~al., and are available under the same
        license unless indicated otherwise.\\ The presentation uses the
        Beamer Jagiellonian theme based on Matthias Vogelgesang’s
        Metropolis theme, available under license \LaTeX{} Project
        Public License~1.3c at: \colorhref{https://github.com/matze/mtheme}
        {https://github.com/matze/mtheme}.

        Typographic design: Iwona Grabska-Gradzińska;
        \LaTeX{} Typesetting: Kamil Ziemian \\
        Proofreading: Wojciech Palacz,
        Monika Stawicka \\
        3D Models: Dariusz Frymus, Kamil Nowakowski \\
        Figures and charts: Kamil Ziemian, Paweł Węgrzyn, Wojciech
        Palacz \\
        Film editing: IMAVI -- Joanna Kozakiewicz, Krzysztof Magda, Nikodem
        Frodyma

      \end{flushleft}

    \end{textblock}

  \end{frame}
}



\newcommand{\GeometryThreeDSpecialEndingSlideVideoVerThreeEN}{%
  \begin{frame}[standout]

    \begin{textblock}{11}(1,0.7)

      \begin{flushleft}

        \mdseries

        \footnotesize

        \color{jFrametitleFGColor}

        This content was created as part of a project co-financed by the
        European Union within the framework of the European Social Fund
        POWR.03.05.00-00-Z309/17-00.

      \end{flushleft}

    \end{textblock}





    \begin{textblock}{10}(0,2.2)

      \tikz \fill[color=jBackgroundStyleLight] (0,0) rectangle (12.8,-1.5);

    \end{textblock}


    \begin{textblock}{3.2}(0.7,2.45)

      \includegraphics[scale=0.3]{\FundingLogoColorPictureEN}

    \end{textblock}


    \begin{textblock}{2.5}(4.15,2.5)

      \includegraphics[scale=0.2]{\JULogoColorPictureEN}

    \end{textblock}


    \begin{textblock}{2.5}(6.35,2.4)

      \includegraphics[scale=0.1]{\ZintegrUJLogoColorPictureEN}

    \end{textblock}


    \begin{textblock}{4.2}(8.4,2.6)

      \includegraphics[scale=0.3]{\EUSocialFundLogoColorPictureEN}

    \end{textblock}





    \begin{textblock}{11}(1,4)

      \begin{flushleft}

        \mdseries

        \footnotesize

        \RaggedRight

        \color{jFrametitleFGColor}

        The content of this lecture is made available under a Creative
        Commons licence (\textsc{cc}), giving the author the credits
        (\textsc{by}) and putting an obligation to share on the same terms
        (\textsc{sa}). Figures and diagrams included in the lecture are
        authored by Paweł Węgrzyn et~al., and are available under the same
        license unless indicated otherwise.\\ The presentation uses the
        Beamer Jagiellonian theme based on Matthias Vogelgesang’s
        Metropolis theme, available under license \LaTeX{} Project
        Public License~1.3c at: \colorhref{https://github.com/matze/mtheme}
        {https://github.com/matze/mtheme}.

        Typographic design: Iwona Grabska-Gradzińska;
        \LaTeX{} Typesetting: Kamil Ziemian \\
        Proofreading: Wojciech Palacz,
        Monika Stawicka \\
        3D Models: Dariusz Frymus, Kamil Nowakowski \\
        Figures and charts: Kamil Ziemian, Paweł Węgrzyn, Wojciech
        Palacz \\
        Film editing: Agencja Filmowa Film \& Television Production~--
        Zbigniew Masklak \\
        Film editing: IMAVI -- Joanna Kozakiewicz, Krzysztof Magda, Nikodem
        Frodyma

      \end{flushleft}

    \end{textblock}

  \end{frame}
}



\newcommand{\GeometryThreeDTwoSpecialEndingSlidesVideoVerOneEN}[1]{%
  \begin{frame}[standout]

    \begin{textblock}{11}(1,0.7)

      \begin{flushleft}

        \mdseries

        \footnotesize

        \color{jFrametitleFGColor}

        This content was created as part of a project co-financed by the
        European Union within the framework of the European Social Fund
        POWR.03.05.00-00-Z309/17-00.

      \end{flushleft}

    \end{textblock}





    \begin{textblock}{10}(0,2.2)

      \tikz \fill[color=jBackgroundStyleLight] (0,0) rectangle (12.8,-1.5);

    \end{textblock}


    \begin{textblock}{3.2}(0.7,2.45)

      \includegraphics[scale=0.3]{\FundingLogoColorPictureEN}

    \end{textblock}


    \begin{textblock}{2.5}(4.15,2.5)

      \includegraphics[scale=0.2]{\JULogoColorPictureEN}

    \end{textblock}


    \begin{textblock}{2.5}(6.35,2.4)

      \includegraphics[scale=0.1]{\ZintegrUJLogoColorPictureEN}

    \end{textblock}


    \begin{textblock}{4.2}(8.4,2.6)

      \includegraphics[scale=0.3]{\EUSocialFundLogoColorPictureEN}

    \end{textblock}





    \begin{textblock}{11}(1,4)

      \begin{flushleft}

        \mdseries

        \footnotesize

        \RaggedRight

        \color{jFrametitleFGColor}

        The content of this lecture is made available under a Creative
        Commons licence (\textsc{cc}), giving the author the credits
        (\textsc{by}) and putting an obligation to share on the same terms
        (\textsc{sa}). Figures and diagrams included in the lecture are
        authored by Paweł Węgrzyn et~al., and are available under the same
        license unless indicated otherwise.\\ The presentation uses the
        Beamer Jagiellonian theme based on Matthias Vogelgesang’s
        Metropolis theme, available under license \LaTeX{} Project
        Public License~1.3c at: \colorhref{https://github.com/matze/mtheme}
        {https://github.com/matze/mtheme}.

        Typographic design: Iwona Grabska-Gradzińska;
        \LaTeX{} Typesetting: Kamil Ziemian \\
        Proofreading: Wojciech Palacz,
        Monika Stawicka \\
        3D Models: Dariusz Frymus, Kamil Nowakowski \\
        Figures and charts: Kamil Ziemian, Paweł Węgrzyn,
        Wojciech Palacz \\
        Film editing: Agencja Filmowa Film \& Television Production~--
        Zbigniew Masklak

      \end{flushleft}

    \end{textblock}

  \end{frame}





  \begin{frame}[standout]


    \begingroup

    \color{jFrametitleFGColor}

    #1

    \endgroup

  \end{frame}
}



\newcommand{\GeometryThreeDTwoSpecialEndingSlidesVideoVerTwoEN}[1]{%
  \begin{frame}[standout]

    \begin{textblock}{11}(1,0.7)

      \begin{flushleft}

        \mdseries

        \footnotesize

        \color{jFrametitleFGColor}

        This content was created as part of a project co-financed by the
        European Union within the framework of the European Social Fund
        POWR.03.05.00-00-Z309/17-00.

      \end{flushleft}

    \end{textblock}





    \begin{textblock}{10}(0,2.2)

      \tikz \fill[color=jBackgroundStyleLight] (0,0) rectangle (12.8,-1.5);

    \end{textblock}


    \begin{textblock}{3.2}(0.7,2.45)

      \includegraphics[scale=0.3]{\FundingLogoColorPictureEN}

    \end{textblock}


    \begin{textblock}{2.5}(4.15,2.5)

      \includegraphics[scale=0.2]{\JULogoColorPictureEN}

    \end{textblock}


    \begin{textblock}{2.5}(6.35,2.4)

      \includegraphics[scale=0.1]{\ZintegrUJLogoColorPictureEN}

    \end{textblock}


    \begin{textblock}{4.2}(8.4,2.6)

      \includegraphics[scale=0.3]{\EUSocialFundLogoColorPictureEN}

    \end{textblock}





    \begin{textblock}{11}(1,4)

      \begin{flushleft}

        \mdseries

        \footnotesize

        \RaggedRight

        \color{jFrametitleFGColor}

        The content of this lecture is made available under a Creative
        Commons licence (\textsc{cc}), giving the author the credits
        (\textsc{by}) and putting an obligation to share on the same terms
        (\textsc{sa}). Figures and diagrams included in the lecture are
        authored by Paweł Węgrzyn et~al., and are available under the same
        license unless indicated otherwise.\\ The presentation uses the
        Beamer Jagiellonian theme based on Matthias Vogelgesang’s
        Metropolis theme, available under license \LaTeX{} Project
        Public License~1.3c at: \colorhref{https://github.com/matze/mtheme}
        {https://github.com/matze/mtheme}.

        Typographic design: Iwona Grabska-Gradzińska;
        \LaTeX{} Typesetting: Kamil Ziemian \\
        Proofreading: Wojciech Palacz,
        Monika Stawicka \\
        3D Models: Dariusz Frymus, Kamil Nowakowski \\
        Figures and charts: Kamil Ziemian, Paweł Węgrzyn,
        Wojciech Palacz \\
        Film editing: IMAVI -- Joanna Kozakiewicz, Krzysztof Magda, Nikodem
        Frodyma

      \end{flushleft}

    \end{textblock}

  \end{frame}





  \begin{frame}[standout]


    \begingroup

    \color{jFrametitleFGColor}

    #1

    \endgroup

  \end{frame}
}



\newcommand{\GeometryThreeDTwoSpecialEndingSlidesVideoVerThreeEN}[1]{%
  \begin{frame}[standout]

    \begin{textblock}{11}(1,0.7)

      \begin{flushleft}

        \mdseries

        \footnotesize

        \color{jFrametitleFGColor}

        This content was created as part of a project co-financed by the
        European Union within the framework of the European Social Fund
        POWR.03.05.00-00-Z309/17-00.

      \end{flushleft}

    \end{textblock}





    \begin{textblock}{10}(0,2.2)

      \tikz \fill[color=jBackgroundStyleLight] (0,0) rectangle (12.8,-1.5);

    \end{textblock}


    \begin{textblock}{3.2}(0.7,2.45)

      \includegraphics[scale=0.3]{\FundingLogoColorPictureEN}

    \end{textblock}


    \begin{textblock}{2.5}(4.15,2.5)

      \includegraphics[scale=0.2]{\JULogoColorPictureEN}

    \end{textblock}


    \begin{textblock}{2.5}(6.35,2.4)

      \includegraphics[scale=0.1]{\ZintegrUJLogoColorPictureEN}

    \end{textblock}


    \begin{textblock}{4.2}(8.4,2.6)

      \includegraphics[scale=0.3]{\EUSocialFundLogoColorPictureEN}

    \end{textblock}





    \begin{textblock}{11}(1,4)

      \begin{flushleft}

        \mdseries

        \footnotesize

        \RaggedRight

        \color{jFrametitleFGColor}

        The content of this lecture is made available under a Creative
        Commons licence (\textsc{cc}), giving the author the credits
        (\textsc{by}) and putting an obligation to share on the same terms
        (\textsc{sa}). Figures and diagrams included in the lecture are
        authored by Paweł Węgrzyn et~al., and are available under the same
        license unless indicated otherwise. \\ The presentation uses the
        Beamer Jagiellonian theme based on Matthias Vogelgesang’s
        Metropolis theme, available under license \LaTeX{} Project
        Public License~1.3c at: \colorhref{https://github.com/matze/mtheme}
        {https://github.com/matze/mtheme}.

        Typographic design: Iwona Grabska-Gradzińska;
        \LaTeX{} Typesetting: Kamil Ziemian \\
        Proofreading: Leszek Hadasz, Wojciech Palacz,
        Monika Stawicka \\
        3D Models: Dariusz Frymus, Kamil Nowakowski \\
        Figures and charts: Kamil Ziemian, Paweł Węgrzyn,
        Wojciech Palacz \\
        Film editing: Agencja Filmowa Film \& Television Production~--
        Zbigniew Masklak \\
        Film editing: IMAVI -- Joanna Kozakiewicz, Krzysztof Magda, Nikodem
        Frodyma


      \end{flushleft}

    \end{textblock}

  \end{frame}





  \begin{frame}[standout]


    \begingroup

    \color{jFrametitleFGColor}

    #1

    \endgroup

  \end{frame}
}











% ------------------------------------------------------
% BibLaTeX
% ------------------------------------------------------
% Package biblatex, with biber as its backend, allow us to handle
% bibliography entries that use Unicode symbols outside ASCII.
\usepackage[
language=polish,
backend=biber,
style=alphabetic,
url=false,
eprint=true,
]{biblatex}

\addbibresource{Wybrane-algorytmy-kompilacji-Bibliography.bib}





% ------------------------------------------------------
% Packages, libraries and their settings
% ------------------------------------------------------
% Library improving positioning of nodes in graphs
\usetikzlibrary{positioning}





% ------------------------------------------------------
% Local packages
% ------------------------------------------------------
% Local configuration of this particular presentation
\usepackage{./Local-packages/local-settings}

% \usepackage{./Local-packages/PGF-TikZ-Diagram-styles}

% Patching problems with Jagiellonian
% \usepackage{./Local-packages/Jagiellonian-theme-colors}

% \usepackage{./Local-packages/Some-patches-for-Jagiellonian}

% Additional colors
% \usepackage{./Local-packages/Jagiellonian-theme-additional-colors}










% ---------------------------------------------------------------------
\title{Wybrane algorytmy kompilacji}
\subtitle{Drzew składniowe}

\author{Kamil Ziemian \\
  \email}



% \date{}
% ---------------------------------------------------------------------










% ####################################################################
% Beginning of the document.
\begin{document}
% ####################################################################





% ######################################
% Number of chars: 62k+, 11k+, 32k+, 55k+, 91k+,
% Text is adjusted to the left and words are broken at the end of the line.
\RaggedRight
% ######################################





% ######################################
\maketitle
% ######################################





% % ##################
% \begin{frame}
%   \frametitle{Table of contents}


%   \tableofcontents

% \end{frame}
% % ##################










% ######################################
\section{Gramatyki}

\label{sec:Podstawy-budowy-procesora-i-jezyka-asemblera}
% ######################################





% ##################
\begin{frame}
  \frametitle{Historyjka ilustrująca problem}


  Tematyka gramatyk jest stosunkowo abstrakcyjna, więc dla zbalansowania
  tego zacznijmy od trochę niepoważnej historii o~dwóch chłopcach, Janie
  i~Marianie. Siedzą oni na lekcji i~śmiertelnie~się nudzą (nie tylko oni).
  By umilić sobie czas, stwierdzili, że~będą przesyłać sobie zaszyfrowane
  wiadomości, tak by nauczyciel nie widział o~czym pisze. Ich pomysł był
  taki, żeby wziąć alfabet grecki i~przyporządkować im ciągi liter
  łacińskich, spacji i~kropek.

  By utrudnić nauczycielowi zrozumienie o~czym mówią, co jakiś czas
  zmieniają swój szyfr. By usprawnić zapisywanie ciągle zmieniających~się
  reguł, wprowadzili notację $\alpha \to \text{wbi}$. Oznacza ona, że~każde
  pojawienie~się symbolu $\alpha$ należy zastąpić ciągiem liter $\text{wbi}$.

  Jakie zdanie reprezentuje ciąg symboli
  $\alpha\gamma\sigma\beta\eta\theta\sigma\delta\sigma\lambda\kappa\zeta\omega$, jeśli szyfr jest podanej tu
  postaci? $\alpha \to \text{Az}$, $\beta \to \text{wbi}$, $\gamma = \text{or}$,
  $\delta = \text{do}$, $\zeta \to \text{oju}$, $\eta \to \text{e}$, $\theta \to \text{gł}$,
  $\kappa \to \text{k}$, $\lambda \to \text{po}$, $\sigma = "\;\;"$ (spacja),
  $\omega \to .$ (pojedyncza kropka).

\end{frame}
% ##################





% ##################
\begin{frame}
  \frametitle{Gramatyki}


  To co opracowali Janek i~Marian to prosty przykład gramatyki. Słowo
  „gramatyka” ma oczywiście kilka znaczeń, nas będzie interesować to jakie
  ma w~informatyce teoretycznej, gdzie oznacza sformalizowany system
  przepisywania tekstu.

  W~przykładzie z~Janem i~Marianem mieliśmy do czynienia z~ciągiem symboli
  alfabetu greckiego, który zgodnie z~podanymi zasadami przepisywaliśmy,
  aż dostaliśmy tekst zapisany za pomocą liter łacińskich, spacji i~kropki.
  To zasadniczo ilustruje o~co chodzi w~gramatykach: dostajemy ciąg
  „niezrozumiałych” symboli, który przekształcamy zgodnie z regułami tak
  długo, aż~dostaniemy ciąg symboli, który sens jesteśmy w~stanie pojąć.

  Ma to dla nas znaczenie, bo program w~języku C jest tekstem i~program
  w~dialekcie assemblera jest też tekstem. Kompilacja polega więc na
  przetworzeniu jednego tekstu w~drugi, co formalizujemy za pomocą
  gramatyki.

\end{frame}
% ##################





% ##################
\begin{frame}
  \frametitle{Definicja gramatyki}


  Ze względu na skromny charakter naszego kursu, nie możemy wchodzić zbyt
  głęboko w~definicję gramatyki. Najważniejsze jest dla nas wyrobienie
  sobie intuicji czym gramatyka jest oraz jak nią operować.

  Gramatykę definiują cztery rzeczy. \\
  1)~Skończony zbiór symboli terminalnych. \\
  2)~Skończony zbiór symboli nieterminalnych. \\
  3)~Reguły produkcji. \\
  4)~Symbol startowy.

  Teraz postaramy~się omówić każde z~tych pojęć.

  \textbf{Symbol terminalny} to symbol który oznacza samego siebie.
  W~kontekście programowania proszę myśleć o~taki rzeczach jak $0$,
  czy~$0.1$. Nie wchodząc w~szczegóły, powinno być jasne, że~zarówno
  $0$ jak i~$0.1$ oznaczają zasadniczo same siebie. W~opowieści
  z~Janem i~Marianem terminalami były litery łacińskie, spacja i~kropka.

\end{frame}
% ##################





% ##################
\begin{frame}
  \frametitle{Symbole nieterminalne, \ldots}


  \textbf{Symbole nieterminalne} to symbole które odnoszą nas do czegoś
  innego niż one same. W~kontekście programowania proszę myśleć o~nazwach
  zmiennych, takich jak \texttt{intVar1} czy \texttt{float64Var1}.
  Czasem, by uniknąć nieporozumień symbole te będziemy nazywać
  \textbf{nieterminalami właściwymi}.

  W~opowieści o~Janie i~Marianie
  symbolami nieterminalnymi były litery
  alfabetu greckiego $\alpha$, $\beta$, $\gamma$, \ldots{ } Proszę sobie przypomnieć, że~oni
  zastępowali symbole nieterminalne przez ciągi symboli terminalnych, wedle
  reguł postaci $\alpha \to \text{Az}$. Tym samym dochodzimy do~pojęcia reguły
  produkcji.

  \textbf{Reguły produkcji} to zestaw reguł, które tłumaczą jak
  przekształcić ciąg symboli zawierający symbole nieterminalne,
  w~inny ciągi symboli. Tym samym $\alpha \to \text{Az}$ jest regułą produkcji,
  której przykład użycia już widzieliśmy. Trochę dokładniej omówimy je
  za~chwilę.

\end{frame}
% ##################





% ##################
\begin{frame}
  \frametitle{Terminologia}


  \textbf{Symbol startowy} jest to symbol od którego zaczynamy tworzenie
  ciągu symboli. Z~punktu widzenia człowieka, jest to najmniej ważny
  element definiujący gramatykę, komputer może mieć inne zdanie. Dokładniej
  wyjaśnimy jego sens po omówieniu reguł produkcji.

  Zauważmy, że~język polski w~formie pisanej składa~się z~słów, które są
  ciągami litera, podobnie jest w~gramatykach, gdzie rozważamy ciągi
  symboli. Rozważane ciągi mogą~się składać z~samych symboli terminalnych,
  samych symboli nieterminalnych, lub jednocześnie z~symboli terminalnych
  i~nieterminalnych. Wracając do naszej opowieści $\alpha\beta$ jest ciągiem
  gramatyki składającym~się z~dwóch symboli nieterminalnych, $\text{Azor}$
  składa~się z~czterech symboli terminalnych, a~$\alpha\text{or}$ składa~się
  z~jednego symbolu nieterminalnego i~dwóch terminalnych.

\end{frame}
% ##################





% ##################
\begin{frame}
  \frametitle{Reguły produkcji}


  Dowolny ciąg symboli danej gramatyki zawierający co najmniej jeden symbol
  nieterminalny nazywać będziemy \textbf{nieterminalem}. Tym samym zarówno
  $\alpha\beta$, jak i~$\alpha\text{or}$ nazywa będziemy nieterminalami. Ciąg składający
  z~samych symboli terminalnych będziemy nazywali \textbf{terminalem}.
  Dodatkowo przyjmujemy, że~pojedynczy znak będziemy rozumieć jako ciąg
  o~długości jeden.

  Przypomnijmy podane wcześniej zasady przekształcania przyjęte przez
  Jana i~Mariana: $\alpha \to \text{Az}$, $\beta \to \text{wbi}$, $\gamma = \text{or}$,
  $\delta = \text{do}$, $\zeta \to \text{oju}$, $\eta \to \text{e}$, $\theta \to \text{gł}$,
  $\kappa \to \text{k}$, $\lambda \to \text{po}$, $\sigma = "\;\;"$ (spacja),
  $\omega \to .$ (kropka). Są to w~istocie reguły produkcji pewnej gramatyki.

\end{frame}
% ##################





% ##################
\begin{frame}
  \frametitle{Reguły produkcji}


  Dla człowieka jest proste w~zrozumieniu, że~skoro $\alpha \to \text{Az}$,
  $\gamma \to \text{or}$ to $\alpha\gamma \to \text{Azor}$, jednak na poziomie pracy
  z~kompilatorem jest to
  trochę zbyt mało precyzyjne. Dla człowieka jest jasne, że~jeśli mamy
  nieterminal $\alpha\alpha$, który rozwija~się $\text{AzAz}$, a~to słowo nie ma
  sensu, to coś musiał pójść nie tak. Jednak w~języku~C wraz \texttt{AzAz}
  jest dopuszczalną nazwą zmiennej (odradzam jednak jej stosowanie)
  i~nikogo to nie gorszy, więc kompilator nie ma powodu tego odrzucać.

  Przyjmijmy, że~Jan i~Marian wprowadzili dodatkowy nieterminal
  $\Gamma$ za~które możemy podstawić dowolne słowo. Jak możemy teraz
  przekształcić $\Gamma\sigma\Gamma$? Przyjmijmy reguły $\Gamma \to \text{Azor}$
  i~$\Gamma \to \text{wbiegła}$. Jeśli musielibyśmy za każde wystąpienie
  $\Gamma$ podstawić to samo słowo, to mamy tylko dwie możliwości
  $\Gamma\sigma\Gamma \to \text{Azor} \; \; \text{Azor}$
  i~$\Gamma\sigma\Gamma \to \text{wbiegł} \; \; \text{wbiegł}$, obie nonsensowne
  z~punktu widzenia języka polskiego.

  % \begin{figure}

  %   \begin{tikzpicture}

  %     \path (0,0) -- (0,-2);



  %     \node (przypisanie) at (0,0) {\texttt{=}};

  %     \node (liczba-1) at (-1.5,-1) {\texttt{intVar1}};

  %     \node (multiplication) at (1.5,-1) {\texttt{*}};


  %     \draw (przypisanie) -- (liczba-1);

  %     \draw (przypisanie) -- (multiplication);


  %     \node (number-2) at (0.5,-2) {};

  %     \node (int-variable-2) at (2.5,-2) {};


  %     \draw (multiplication) -- (number-2);

  %     \draw (multiplication) -- (int-variable-2);

  %   \end{tikzpicture}

  %   \caption{Budowanie drzewa składniowego wyrażenia
  %     \texttt{intVar1 = 3 * intVar2}}

  %   \label{fig:Scheme-of-CPU}

  % \end{figure}







\end{frame}
% ##################





% ##################
\begin{frame}
  \frametitle{Reguły produkcji}


  Jeśli jednak przyjmiemy, że~możemy w~każdym kroku wybrać \alert{jeden}
  symbol $\Gamma$ i~dla niego użyć reguły $\Gamma \to \text{Azor}$
  lub~$\Gamma \to \text{wbiegła}$, to mamy cztery możliwe wyniki: \\
  $\Gamma\sigma\Gamma \to \text{Azor} \; \; \text{Azor}$ \\
  $\Gamma\sigma\Gamma \to \text{wbiegł} \; \; \text{wbiegł}$ \\
  $\Gamma\sigma\Gamma \to \text{Azor} \; \; \text{wbiegł}$ \\
  $\Gamma\sigma\Gamma \to \text{wbiegł} \; \; \text{Azor}$ \\
  gdzie dwie ostatnie wyrażenia już mają jakiś sens.

  Ten przykład z~nieterminalem $\Gamma\sigma\Gamma$ może~się wydać sztuczny, jednak
  naświetla nam dwa ważne fakty. Po pierwsze, wprowadzenie więcej niż
  jednej reguły produkcji dla nieterminala $\Gamma$ może być użyteczne.
  Po drugie, zastępowanie kilku egzemplarzy nieterminala $\Gamma$,
  jeden po drugim, różnymi słowami również może prowadzić do~ciekawych
  wyników (najpierw $\Gamma \to \text{Azor}$, potem
  $\Gamma \to \text{wbiegł}$).

\end{frame}
% ##################





% ##################
\begin{frame}
  \frametitle{Reguły produkcji}


  Gdy to już wszystko wiemy, możemy sprecyzować co rozumiemy przez reguły
  produkcji danej gramatyki. Jest nim zbiór reguł postaci $A \to B$, gdzie
  $A$ \alert{musi} być nieterminalem, a~$B$ może być zarówno terminalem,
  lub nieterminalem.

  Widzimy więc, że~reguły $\alpha \to \text{Az}$, $\beta \to \text{wbi}$,
  $\gamma = \text{or}$, $\delta = \text{do}$, $\zeta \to \text{oju}$, $\eta \to \text{e}$,
  $\theta \to \text{gł}$, $\kappa \to \text{k}$, $\lambda \to \text{po}$, $\sigma = "\;\;"$
  (spacja), $\omega \to .$ (kropka) są ciągiem reguł produkcji, które zupełnie
  wystarczają dla Jana i~Mariana, ale nie są zadowalające z~punktu widzenia
  kompilatora. Do tego wrócimy za~chwilę.

  Reguł produkcji używamy w~następujący sposób. Jeśli otrzymamy dany
  nieterminal, powiedzmy $\alpha\gamma\sigma\beta\eta\theta$, dokonuje ciąg przekształceń zgodnie
  z~regułami produkcji, w~każdym kroku podmieniając \alert{jeden}
  nieterminal występujący po lewej stronie którejś z~reguły produkcji,
  na~ciąg symboli występujący po prawej stronie tej reguły.

\end{frame}
% ##################





% ##################
\begin{frame}
  \frametitle{Reguły produkcji}


  Korzystając z~reguł produkcji $\alpha \to \text{Az}$,
  $\beta \to \text{wbi}$, $\gamma = \text{or}$, $\delta = \text{do}$,
  $\zeta \to \text{oju}$, $\eta \to \text{e}$, $\theta \to \text{gł}$,
  $\kappa \to \text{k}$, $\lambda \to \text{po}$, $\sigma = "\;\;"$ (spacja),
  $\omega \to .$ (kropka), mamy następujący ciąg produkcji dla
  nieterminala $\alpha\gamma\sigma\beta\eta\theta$. \\
  $\alpha\gamma\sigma\beta\eta\theta \to \text{Az}\gamma\sigma\beta\eta\theta \to \text{Azor}\sigma\beta\eta\theta \to $
  $\text{Azor}\sigma\beta\eta\theta \to \text{Azor} \; \; \beta\eta\theta \to $ \\
  $\to \text{Azor} \; \; \text{wbi}\eta\theta \to $
  $\text{Azor} \; \; \text{wbie}\theta \to \text{Azor} \; \; \text{wbiegł}$

  Dowolnie taki ciąg przekształceń nazywamy \textbf{ciągiem produkcji}
  lub \textbf{produkcją} danej gramatyki. Jeśli dany nieterminal zaczyna
  dany ciąg produkcji, to mówimy, że~jest to ciąg produkcji tego
  nieterminalna. Zgodnie z~tym
  $\alpha\gamma \to \text{Az}\gamma \to \text{Azor}$ jest ciągiem produkcji
  nieterminala~$\alpha\gamma$.

  Powstaje pytanie. Czy nie możemy napisać po prostu
  $\alpha\gamma \to \text{Azor}$, skoro widać, że~to musi wyjść? Odpowiedź jest taka,
  że~gdy piszemy dany ciąg produkcji, po to by czytali go inni ludzie,
  którzy i~tak to zrozumieją, to oczywiście możemy dokonywać kilku takich
  podstawień w~jednym kroku.

\end{frame}
% ##################





% ##################
\begin{frame}
  \frametitle{Reguły produkcji i~symbol startowy}


  Należy jednak pamiętać, że~z~technicznych względów, jeśli chcemy być
  bardzo poprawni, to w~każdym kroku musimy skorzystać \alert{dokładnie}
  jeden raz z~\alert{dokładnie} jednej reguły produkcji. Nie wolno też
  przesadzić w~skracaniu ciągu produkcji, tak że nawet dla ludzi staje~się
  niezrozumiały. ;)

  Choć nie było to jawnie powiedziane, to zapewne już~się Państwo domyślili,
  że~jeżeli w~trakcie produkcji dojdziemy do ciągu zawierającego tylko
  symbole terminalne, to kończymy dany ciąg. Wszak symbole terminalne
  oznaczają same siebie, więc taki ciąg oznacza sam siebie.

  Teraz powinno być jasne, czemu po lewej stronie strzałki $\to$ w~regule
  produkcji, \alert{musi} stać co najmniej jeden symbol nieterminalny.
  Inaczej cały formalizm traci sens. Wracają do podanego przykładu,
  Jan i~Marian nie mogli ustalić reguły produkcji
  $\text{wbi} \to \text{wbiegł}$, bo wówczas ich system nie byłby gramatyką.
  Byłby szyfrem, ale nie gramatyką, ale szyfry to temat na zajęcia
  z~\colorhref{https://pl.wikipedia.org/wiki/Kryptologia}{kryptologii},
  a~nie z~podstaw kompilatorów.

\end{frame}
% ##################





% ##################
\begin{frame}
  \frametitle{Symbol startowy}


  Tutaj musimy wyjaśnić jeszcze jedno zagadnienie: skąd wiadomo,
  że~ciąg $\alpha\beta$ jest sensowny? Dla nas ludzi sprawa jest prosta. Sprawdzamy,
  że~oznacza on $\text{Azor}$ i~sprawdzamy, czy jest to sensowne słowo
  języka polskiego. Jednak nasze ludzkie pojęcie sensownego słowa jest
  ciężko przekładalne (a~jest w~ogóle?) na język komputera, więc musimy
  to jakoś obejść. By to zrobić wprowadzamy \textbf{symbol startowy}.

  Idea jest prosta. Wybieramy jakiś symbol nieterminalny od~którego musi
  zacząć~się każda pełna produkcja gramatyki i~dowolny ciąg symboli
  gramatyki uważamy za poprawny, jeśli możemy znaleźć ciąg produkcji, który
  zaczyna~się od symbolu startowego i~kończy na~rozważanym ciągu symboli.
  Jeśli danego ciągu symboli nie da~się wyprowadzić zaczynając od symbolu
  startowego to mówimy, że~dany ciąg jest \textbf{niepoprawnie
    skonstruowany} w~ramach danej gramatyki.

\end{frame}
% ##################





% ##################
\begin{frame}
  \frametitle{Symbol startowy}


  Wcześniej powiedzieliśmy, że~symbol startowy jest najmniej ważnym
  składnikiem gramatyki. Dokładnej, jest ona najmniej ważny z~punktu
  widzenia człowieka, co widać po tym, że~książkowych definicjach gramatyki
  często się go nawet nie podaje, uważając, że~same reguły produkcji
  wyjaśniają wystarczająco jasno jakie ciągi symboli można w~ramach jej
  utworzyć. Z~punktu widzenia teorii i~kompilatora jest on oczywiście
  bardzo ważny.

  Zwyczajowo, za~symbol startowy przyjmuje~się symbol $\epsilon$, który oznacza
  pusty ciąg symboli. My też będziemy tak przyjmować w~dalszym ciągu tych
  zajęć.

  Ciąg produkcji gramatyki zaczynający~się od symbolu startowego~$\epsilon$
  i~kończący na~ciągu symboli terminalnych, nazywamy \textbf{pełnym ciągiem
    produkcji}.

\end{frame}
% ##################





% ##################
\begin{frame}
  \frametitle{Uzupełnienie reguł produkcji}


  Jak teraz widzimy, z~technicznego punktu widzenia, gramatyka Jana
  i~Mariana jest niepełna bo nie określa jak wyprowadzać ciągi symboli
  zaczynając od symbolu startowego $\epsilon$. By~ją uzupełnić, obok symbolu
  $\epsilon$~wprowadzamy jeszcze jeden symbol nieterminalny $S$
  (od~słowa \textit{symbol}) i~reguły produkcji \\
  $\epsilon \to S$, $S \to SS$.

  Sens pierwszej z~nich jest prosty. Zaczynając od symbolu startowego
  możemy wygenerować pojedynczy symbol. Druga reguła mówi, że~za~pojedynczy
  symbol możemy podstawić dwa symbole. To pozwala nam otrzymać dowolnie
  długi, acz skończony, ciąg symboli poprzez produkcję: \\
  $\epsilon \to S \to SS \to SSS \to SSSS \to \ldots$

  Teraz potrzebujemy wprowadzić reguły postaci $S \to \alpha$, $S \to \beta$,
  $S \to \gamma$, \ldots{ } Jak łatwo zauważyć, szybko staje~się to dość żmudne.

\end{frame}
% ##################





% ##################
\begin{frame}
  \frametitle{Uzupełnienie reguł produkcji}


  By uprościć notację, wprowadzamy symbol $|$, który należy czytać
  „jedno z~dwojga”. Dwie reguły produkcji $S \to \alpha$, $S \to \beta$ możemy zapisać
  w~następujący sposób: $S \to \alpha \, | \, \beta$.

  Analogicznie reguły produkcji $S \to \alpha$, $S \to \beta$, $S \to \gamma$ możemy zapisać
  jako $S \to \alpha \, | \, \beta \, | \, \gamma$, etc.

  W~tym momencie możemy zapisać pełny zestaw reguł produkcji gramatyki
  której używali Jan i~Marian. \\
  $\epsilon \to S$, \\
  $S \to SS$, \\
  $S \to \alpha \, | \, \beta \, | \, \gamma \, | \, \delta \, | \, \zeta \, | \, \eta \, | \,
  \theta \, | \, \kappa \, | \, \lambda \, | \, \sigma \, | \, \omega$, \\
  $\alpha \to \text{Az}$, $\beta \to \text{wbi}$, $\gamma = \text{or}$,
  $\delta = \text{do}$,$\zeta \to \text{oju}$, $\eta \to \text{e}$,
  $\theta \to \text{gł}$, $\kappa \to \text{k}$, $\lambda \to \text{po}$,
  $\sigma = "\;\;"$ (spacja), $\omega \to .$ (kropka).

  Korzystając z~nich, pokażemy pełny ciąg produkcji ciągu symboli
  $\text{Azor wbiegł}$.

  % \begin{figure}

  %   \begin{tikzpicture}

  %     \path (0,0) -- (0,-2);



  %     \node (assignment) at (0,0) {\texttt{=}};

  %     \node (int-variable-4) at (-1.5,-1) {\texttt{intVar4}};

  %     \node (addition) at (1.5,-1) {\texttt{+}};


  %     \draw (assignment) -- (int-variable-4);

  %     \draw (assignment) -- (addition);


  %     \node (assigment-1) at (0.5,-2) {\texttt{=}};

  %     \node (int-variable-3) at (2.5,-2) {\texttt{intVar3}};


  %     \draw (addition) -- (int-variable-2);

  %     \draw (addition) -- (int-variable-3);


  %     \node (int-variable-1) at (-0.25,-2.75) {\texttt{intVar1}};

  %     \node (multiplication) at (1.25,-2.75) {\texttt{*}};


  %     \draw (assigment-1) -- (int-variable-1);

  %     \draw (assigment-1) -- (multiplication);


  %     % \node (number-2) at (0.25,-3.4) {\texttt{5}};

  %     % \node (int-variable-3) at (2.25,-3.4) {\texttt{intVar2}};


  %     % \draw (multiplication) -- (number-2);

  %     % \draw (multiplication) -- (int-variable-3);

  %   \end{tikzpicture}

  %   \caption{Budowanie drzewa składniowego wyrażenia
  %     \texttt{intVar1 = 5 * intVar2; intVar4 = intVar1 + intVar3}}

  %   \label{fig:Scheme-of-CPU}

  % \end{figure}





  % 4.~Przypisz do \texttt{intVar4} wynik punktu 3. \\
  % 3.~Dodaj wartość zmiennej \texttt{intVar1} do~wartości
  % \texttt{intVar3}. \\
  % 2.~Przypisz do \texttt{intVar1} wynik punktu~1. \\

\end{frame}
% ##################





% ##################
\begin{frame}
  \frametitle{Przykład ciągu produkcji}


  $\epsilon \to S$, $S \to SS$, \\
  $S \to \alpha \, | \, \beta \, | \, \gamma \, | \, \delta \, | \, \zeta \, | \, \eta \, | \,
  \theta \, | \, \kappa \, | \, \lambda \, | \, \sigma \, | \, \omega$, \\
  $\alpha \to \text{Az}$, $\beta \to \text{wbi}$, $\gamma = \text{or}$,
  $\delta = \text{do}$,$\zeta \to \text{oju}$, $\eta \to \text{e}$,
  $\theta \to \text{gł}$, $\kappa \to \text{k}$, $\lambda \to \text{po}$,
  $\sigma = "\;\;"$ (spacja), $\omega \to .$ (kropka).

  Ciąg produkcje $\text{Azor wbiegł}$ jest postaci:
  $\epsilon \to S \to SS \to SSS \to SSSS \to SSSSS \to SSSSSS \to \alpha SSSSS \to $
  $\to \alpha\gamma SSSS \to \alpha\gamma\sigma SSS \to \alpha\gamma\sigma\beta SS \to \alpha\gamma\sigma\beta\eta S \to \alpha\gamma\sigma\beta\eta\theta \to $
  $\to \text{Az}\gamma\sigma\beta\eta\theta \to \text{Azor}\sigma\beta\eta\theta \to $
  $\to \text{Azor }\beta\eta\theta \to \text{Azor wbi}\eta\theta \to \text{Azor wbie}\theta \to $
  $\text{Azor wbiegł}$

  Jak widzimy wypisywanie ciągów produkcji potrafi być bardzo żmudne
  i~należy~się cieszyć, że~robi to za nas komputer. :)

  Zauważmy, że~ciąg produkcji prowadzący do danego ciągu symboli
  terminalnych nie musi być jednoznaczny. Powyżej moglibyśmy nieterminal
  $SSSSS$ przekształcić w~następujący sposób
  $SSSSS \to \alpha SSSS \to \alpha SSSSS$ i~postępując jak poprzednio ponownie dojść
  do~$\text{Azor wbiegł}$.

\end{frame}
% ##################





% ##################
\begin{frame}
  \frametitle{Kilka pojęć}


  Dowolny ciąg terminali generowany przez daną gramatykę nazywamy
  \textbf{słowem}. Trochę mylące jest to, że~wedle tej terminologi zdanie
  języka polskiego „Azor wbiegł do pokoju.” jest \alert{słowem} rozważanej
  wcześniej gramatyki. Zbiór wszystkich słów generowanych przez daną
  gramatykę nazywamy \textbf{językiem}.

  Dwa podstawowe zadania przy pracy z~gramatykami są następujące.

  \begin{itemize}

  \item[1)] Jaki jest język generowany przez daną gramatykę.

  \item[2)] Dany jest ciąg symboli terminalnych danej gramatyki. Czy ten
    ciąg należy do języka przez nią generowanego?

  \end{itemize}

\end{frame}
% ##################



% % ######################################
% \section{Przykład drzewa składniowego, 1}

% \label{sec:Podstawy-budowy-procesora-i-jezyka-asemblera}
% % ######################################




















% % ##################
% \begin{frame}
%   \frametitle{Drzewo składniowe wyrażenia arytmetycznego}


%   \begin{figure}

%     \begin{tikzpicture}

%       \path (0,0) -- (0,-2);



%       \node (assignment) at (0,0) {\texttt{=}};

%       \node (liczba-1) at (-1.5,-1) {\texttt{intVar1}};

%       \node (addition) at (1.5,-1) {\texttt{+}};


%       \draw (assignment) -- (liczba-1);

%       \draw (assignment) -- (addition);


%       \node (int-variable-2) at (0.5,-2) {\texttt{intVar2}};

%       \node (multiplication) at (2.5,-2) {\texttt{*}};


%       \draw (addition) -- (int-variable-2);

%       \draw (addition) -- (multiplication);


%       \node (number-2) at (1.5,-3) {\texttt{4}};

%       \node (int-variable-3) at (3.5,-3) {\texttt{intVar3}};


%       \draw (multiplication) -- (number-2);

%       \draw (multiplication) -- (int-variable-3);

%     \end{tikzpicture}

%     \caption{Budowanie drzewa składniowego wyrażenia
%       \texttt{intVar1 = intVar2 + 4 * intVar3}}

%     \label{fig:Scheme-of-CPU}

%   \end{figure}





%   3.~Przypisz do zmiennej \texttt{intVar1} wynik punktu~2. \\
%   2.~Dodaj do \texttt{intVar2} wynik punktu~1. \\
%   1.~Pomnóż \texttt{4} i~\texttt{intVar3}.

% \end{frame}
% % ##################










% % ######################################
% \section{Przykład drzewa składniowego, 2}

% \label{sec:Podstawy-budowy-procesora-i-jezyka-asemblera}
% % ######################################


% % ##################
% \begin{frame}
%   \frametitle{Drzewo składniowe wyrażenia arytmetycznego}


%   Zbudujmy teraz drzewo składniowe bardziej skomplikowanego wyrażenia,
%   które składa~się z~dwóch linii instrukcji. \\
%   \texttt{1: intVar1 = 5 * intVar2} \\
%   \texttt{2: intVar4 = intVar1 + intVar3} \\
%   Należy zaznaczyć, że~\texttt{1:} i~\texttt{2:} nie są częścią składni,
%   ale jedynie pomocniczymi etykietami, które informują nas, że~ten
%   konkretny kod jest rozbity na dwie linie.

%   Jak poprzednio wypisujemy wszystkie działania w~odpowiedniej
%   kolejności. \\
%   1.~Pomnóż \texttt{5} i~\texttt{intVar2}. \\
%   2.~Przypisz do \texttt{intVar1} wynik punktu~1. \\
%   3.~Dodaj wartość zmiennej \texttt{intVar1} do~wartości
%   \texttt{intVar3}. \\
%   4.~Przypisz do \texttt{intVar4} wynik punktu 3. \\
%   % Na potrzeby naszej analizy przyjmujemy, że~każda operacja przypisania
%   % \texttt{=} przekazuje w~górę drzewa wartość jaką przypisana została jej
%   % lewej stronie.

% \end{frame}
% % ##################










% % ##################
% \begin{frame}
%   \frametitle{Drzewo składniowe wyrażenia arytmetycznego}


%   \begin{figure}

%     \begin{tikzpicture}

%       \path (0,0) -- (0,-2);



%       \node (assignment) at (0,0) {\texttt{=}};

%       \node (int-variable-4) at (-1.5,-1) {\texttt{intVar4}};

%       \node (addition) at (1.5,-1) {\texttt{+}};


%       \draw (assignment) -- (int-variable-4);

%       \draw (assignment) -- (addition);


%       \node (assigment-1) at (0.5,-2) {\texttt{=}};

%       \node (int-variable-3) at (2.5,-2) {\texttt{intVar3}};


%       \draw (addition) -- (int-variable-2);

%       \draw (addition) -- (multiplication);


%       % \node (number-2) at (1.5,-3) {\texttt{4}};

%       % \node (int-variable-3) at (3.5,-3) {\texttt{intVar3}};


%       % \draw (multiplication) -- (number-2);

%       % \draw (multiplication) -- (int-variable-3);

%     \end{tikzpicture}

%     \caption{Budowanie drzewa składniowego wyrażenia
%       \texttt{intVar1 = 5 * intVar2; intVar4 = intVar1 + intVar3}}

%     \label{fig:Scheme-of-CPU}

%   \end{figure}





%   4.~Przypisz do \texttt{intVar4} wynik punktu 3. \\
%   3.~Dodaj wartość zmiennej \texttt{intVar1} do~wartości
%   \texttt{intVar3}. \\
%   Na potrzeby naszej analizy przyjmujemy, że~każda operacja przypisania
%   \texttt{=} przekazuje w~górę drzewa wartość jaką przypisana została jej
%   lewej stronie. Stąd w~lewym wierzchołku \texttt{+} nie znajduje~się
%   \texttt{intVar1}, lecz symbol przypisania \texttt{=}.

% \end{frame}
% % ##################





% % ##################
% \begin{frame}
%   \frametitle{Drzewo składniowe wyrażenia arytmetycznego}


%   \begin{figure}

%     \begin{tikzpicture}

%       \path (0,0) -- (0,-2);



%       \node (assignment) at (0,0) {\texttt{=}};

%       \node (int-variable-4) at (-1.5,-1) {\texttt{intVar4}};

%       \node (addition) at (1.5,-1) {\texttt{+}};


%       \draw (assignment) -- (int-variable-4);

%       \draw (assignment) -- (addition);


%       \node (assigment-1) at (0.5,-2) {\texttt{=}};

%       \node (int-variable-3) at (2.5,-2) {\texttt{intVar3}};


%       \draw (addition) -- (int-variable-2);

%       \draw (addition) -- (multiplication);


%       % \node (number-2) at (1.5,-3) {\texttt{4}};

%       % \node (int-variable-3) at (3.5,-3) {\texttt{intVar3}};


%       % \draw (multiplication) -- (number-2);

%       % \draw (multiplication) -- (int-variable-3);

%     \end{tikzpicture}

%     \caption{Budowanie drzewa składniowego wyrażenia
%       \texttt{intVar1 = 5 * intVar2; intVar4 = intVar1 + intVar3}}

%     \label{fig:Scheme-of-CPU}

%   \end{figure}





%   4.~Przypisz do \texttt{intVar4} wynik punktu 3. \\
%   3.~Dodaj wartość zmiennej \texttt{intVar1} do~wartości
%   \texttt{intVar3}. \\
%   Na potrzeby naszej analizy przyjmujemy, że~każda operacja przypisania
%   \texttt{=} przekazuje w~górę drzewa wartość jaką przypisana została jej
%   lewej stronie. Stąd w~lewym wierzchołku \texttt{+} nie znajduje~się
%   \texttt{intVar1}, lecz symbol przypisania \texttt{=}.

% \end{frame}
% % ##################











% % ##################
% \begin{frame}
%   \frametitle{Drzewo składniowe wyrażenia arytmetycznego}


%   \begin{figure}

%     \begin{tikzpicture}

%       \path (0,0) -- (0,-2);



%       \node (assignment) at (0,0) {\texttt{=}};

%       \node (int-variable-4) at (-1.5,-1) {\texttt{intVar4}};

%       \node (addition) at (1.5,-1) {\texttt{+}};


%       \draw (assignment) -- (int-variable-4);

%       \draw (assignment) -- (addition);


%       \node (assigment-1) at (0.5,-2) {\texttt{=}};

%       \node (int-variable-3) at (2.5,-2) {\texttt{intVar3}};


%       \draw (addition) -- (int-variable-2);

%       \draw (addition) -- (int-variable-3);


%       \node (int-variable-1) at (-0.25,-2.75) {\texttt{intVar1}};

%       \node (multiplication) at (1.25,-2.75) {\texttt{*}};


%       \draw (assigment-1) -- (int-variable-1);

%       \draw (assigment-1) -- (multiplication);


%       \node (number-2) at (0.5,-3.4) {\texttt{5}};

%       \node (int-variable-3) at (2.25,-3.4) {\texttt{intVar2}};


%       \draw (multiplication) -- (number-2);

%       \draw (multiplication) -- (int-variable-3);

%     \end{tikzpicture}

%     \caption{Budowanie drzewa składniowego wyrażenia
%       \texttt{intVar1 = 5 * intVar2; intVar4 = intVar1 + intVar3}}

%     \label{fig:Scheme-of-CPU}

%   \end{figure}





%   4.~Przypisz do \texttt{intVar4} wynik punktu 3. \\
%   3.~Dodaj wartość zmiennej \texttt{intVar1} do~wartości
%   \texttt{intVar3}. \\
%   2.~Przypisz do \texttt{intVar1} wynik punktu~1. \\
%   1.~Pomnóż \texttt{5} i~\texttt{intVar2}.

% \end{frame}
% % ##################
















% % ######################################
% \section{Drzewa składniowe z~wywołaniem funkcji}

% \label{sec:Podstawy-budowy-procesora-i-jezyka-asemblera}
% % ######################################


% % ##################
% \begin{frame}
%   \frametitle{Drzewo składniowe wywołania funkcji}


%   \begin{figure}

%     \begin{tikzpicture}

%       % \path (0,0) -- (0,-2);



%       \node (parenties) at (0,0) {\texttt{( )}};

%       \node (sin) at (-1.5,-1) {\texttt{sin}};

%       \node (x) at (1.5,-1) {\texttt{x}};


%       \draw (parenties) -- (sin);

%       \draw (parenties) -- (x);


%       % \node (assigment-1) at (0.5,-2) {\texttt{=}};

%       % \node (int-variable-3) at (2.5,-2) {\texttt{intVar3}};


%       % \draw (addition) -- (int-variable-2);

%       % \draw (addition) -- (multiplication);


%       % \node (number-2) at (1.5,-3) {\texttt{4}};

%       % \node (int-variable-3) at (3.5,-3) {\texttt{intVar3}};


%       % \draw (multiplication) -- (number-2);

%       % \draw (multiplication) -- (int-variable-3);

%     \end{tikzpicture}

%     \caption{Drzewo składniowe wyrażenia
%       \texttt{sin(x)}}

%     \label{fig:Scheme-of-CPU}

%   \end{figure}





%   Wywołanie funkcji typu \texttt{sin(x)}, gdzie \texttt{x} jest zmienną,
%   a~\texttt{sin} to funkcja trygonometryczna, jest dobrze znane z~wielu
%   języków programowania. Może zaskakiwać, że~na poziomie drzew składniowych
%   wywołanie funkcji traktujemy jako wynik obliczań wykonany przez
%   operator \alert{dwuargumentowy}. Dokładniej nawiasy \texttt{( )}
%   traktujemy jako \alert{dwuargumentowy} operator, który jako pierwszą
%   wartość pobiera nazwę funkcji, a~jako drugi jej argument (może być
%   oczywiście więcej, niż jeden argument).
%   % 4.~Przypisz do \texttt{intVar4} wynik punktu 3. \\
%   % 3.~Dodaj wartość zmiennej \texttt{intVar1} do~wartości
%   % \texttt{intVar3}. \\
%   % Na potrzeby naszej analizy przyjmujemy, że~każda operacja przypisania
%   % \texttt{=} przekazuje w~górę drzewa wartość jaką przypisana została jej
%   % lewej stronie. Stąd w~lewym wierzchołku \texttt{+} nie znajduje~się
%   % \texttt{intVar1}, lecz symbol przypisania \texttt{=}.

% \end{frame}
% % ##################





% % ##################
% \begin{frame}
%   \frametitle{Drzewo składniowe z~wywołaniem funkcji}


%   \begin{figure}

%     \begin{tikzpicture}

%       % \path (0,0) -- (0,-2);



%       \node (parenties) at (0,0) {\texttt{( )}};

%       \node (sin) at (-1.5,-1) {\texttt{sin}};

%       \node (x) at (1.5,-1) {\texttt{x}};


%       \draw (parenties) -- (sin);

%       \draw (parenties) -- (x);


%       % \node (assigment-1) at (0.5,-2) {\texttt{=}};

%       % \node (int-variable-3) at (2.5,-2) {\texttt{intVar3}};


%       % \draw (addition) -- (int-variable-2);

%       % \draw (addition) -- (multiplication);


%       % \node (number-2) at (1.5,-3) {\texttt{4}};

%       % \node (int-variable-3) at (3.5,-3) {\texttt{intVar3}};


%       % \draw (multiplication) -- (number-2);

%       % \draw (multiplication) -- (int-variable-3);

%     \end{tikzpicture}

%     \caption{Drzewo składniowe wyrażenia
%       \texttt{sin(x)}}

%     \label{fig:Scheme-of-CPU}

%   \end{figure}

%   % \begin{figure}

%   %   \begin{tikzpicture}

%   %     \path (0,0) -- (0,-2);



%   %     \node (assignment) at (0,0) {\texttt{=}};

%   %     \node (float64Var1) at (-1.5,-1) {\texttt{float64Var1}};

%   %     \node (addition) at (1.5,-1) {\texttt{+}};


%   %     \draw (assignment) -- (float64Var1);

%   %     \draw (assignment) -- (addition);


%   %     \node (parenties) at (0.5,-2) {\texttt{( )}};

%   %     \node (number-1) at (2.5,-2) {\texttt{2.7182}};


%   %     % \draw (addition) -- (int-variable-2);

%   %     % \draw (addition) -- (multiplication);


%   %     % \node (number-2) at (1.5,-3) {\texttt{4}};

%   %     % \node (int-variable-3) at (3.5,-3) {\texttt{intVar3}};


%   %     % \draw (multiplication) -- (number-2);

%   %     % \draw (multiplication) -- (int-variable-3);

%   %   \end{tikzpicture}

%   %   \caption{Drzewo składniowe wyrażenia
%   %     \texttt{float64Var1 = sin(x) + 2.7182}}

%   %   \label{fig:Scheme-of-CPU}

%   % \end{figure}
%   % \begin{figure}

%   %   \begin{tikzpicture}

%   %     \path (0,0) -- (0,-2);



%   %     \node (assignment) at (0,0) {\texttt{=}};

%   %     \node (float64Var1) at (-1.5,-1) {\texttt{float64Var1}};

%   %     \node (addition) at (1.5,-1) {\texttt{+}};


%   %     \draw (assignment) -- (float64Var1);

%   %     \draw (assignment) -- (addition);


%   %     \node (parenties) at (0.5,-2) {\texttt{( )}};

%   %     \node (number-1) at (2.5,-2) {\texttt{2.7182}};


%   %     % \draw (addition) -- (int-variable-2);

%   %     % \draw (addition) -- (multiplication);


%   %     % \node (number-2) at (1.5,-3) {\texttt{4}};

%   %     % \node (int-variable-3) at (3.5,-3) {\texttt{intVar3}};


%   %     % \draw (multiplication) -- (number-2);

%   %     % \draw (multiplication) -- (int-variable-3);

%   %   \end{tikzpicture}

%   %   \caption{Drzewo składniowe wyrażenia
%   %     \texttt{float64Var1 = sin(x) + 2.7182}}

%   %   \label{fig:Scheme-of-CPU}

%   % \end{figure}





%   Wiedząc to, widzimy, że~powyższe drzewo składniowe poprawnie
%   reprezentuje wywołanie funkcji sinus postaci \texttt{sin(x)}.

%   Dysponując powyższymi informacjami, rozpisywanie drzew składniowych
%   wyrażeń zawierających funkcje powinno nie stanowić problemu.
%   % Wywołanie funkcji typu \texttt{sin(x)}, gdzie \texttt{x} jest zmienną,
%   % a~\texttt{sin} to funkcja trygonometryczna, jest dobrze znane z~wielu
%   % języków programowania. Może zaskakiwać, że~na poziomie drzew składniowych
%   % wywołanie funkcji traktujemy jako wynik obliczań wykonany przez
%   % operator \alert{dwuargumentowy}. Dokładniej nawiasy \texttt{( )}
%   % traktujemy jako \alert{dwuargumentowy} operator, który jako pierwszą
%   % wartość pobiera nazwę funkcji, a~jako drugi jej argument (może być
%   % oczywiście więcej, niż jeden argument).
%   % 4.~Przypisz do \texttt{intVar4} wynik punktu 3. \\
%   % 3.~Dodaj wartość zmiennej \texttt{intVar1} do~wartości
%   % \texttt{intVar3}. \\
%   % Na potrzeby naszej analizy przyjmujemy, że~każda operacja przypisania
%   % \texttt{=} przekazuje w~górę drzewa wartość jaką przypisana została jej
%   % lewej stronie. Stąd w~lewym wierzchołku \texttt{+} nie znajduje~się
%   % \texttt{intVar1}, lecz symbol przypisania \texttt{=}.

% \end{frame}
% % ##################





% % ##################
% \begin{frame}
%   \frametitle{Drzewo składniowe z~wywołaniem funkcji}


%   \begin{figure}

%     \begin{tikzpicture}

%       % \path (0,0) -- (0,-2);



%       \node (assignment) at (0,0) {\texttt{=}};

%       \node (float64Var1) at (-1.5,-1) {\texttt{float64Var1}};

%       \node (addition) at (1.5,-1) {\texttt{+}};


%       \draw (assignment) -- (float64Var1);

%       \draw (assignment) -- (addition);


%       \node (parenties) at (0.5,-2) {\texttt{( )}};

%       \node (number-1) at (2.5,-2) {\texttt{2.7182}};


%       \draw (addition) -- (parenties);

%       \draw (addition) -- (number-1);


%       \node (sin function) at (-0.25,-3) {\texttt{sin}};

%       \node (x variable) at (1.25,-3) {\texttt{x}};


%       \draw (parenties) -- (sin function);

%       \draw (parenties) -- (x variable);

%     \end{tikzpicture}

%     \caption{Drzewo składniowe wyrażenia
%       \texttt{float64Var1 = sin(x) + 2.7182}}

%     \label{fig:Scheme-of-CPU}

%   \end{figure}





%   % Wiedząc to, widzimy, że~powyższe drzewo składniowe poprawnie
%   % reprezentuje wywołanie funkcji sinus postaci \texttt{sin(x)}.

%   % Dysponując powyższymi informacjami, rozpisywanie drzew składniowych
%   % wyrażeń zawierających funkcje powinno nie stanowić problemu.
%   % Wywołanie funkcji typu \texttt{sin(x)}, gdzie \texttt{x} jest zmienną,
%   % a~\texttt{sin} to funkcja trygonometryczna, jest dobrze znane z~wielu
%   % języków programowania. Może zaskakiwać, że~na poziomie drzew składniowych
%   % wywołanie funkcji traktujemy jako wynik obliczań wykonany przez
%   % operator \alert{dwuargumentowy}. Dokładniej nawiasy \texttt{( )}
%   % traktujemy jako \alert{dwuargumentowy} operator, który jako pierwszą
%   % wartość pobiera nazwę funkcji, a~jako drugi jej argument (może być
%   % oczywiście więcej, niż jeden argument).
%   % 4.~Przypisz do \texttt{intVar4} wynik punktu 3. \\
%   % 3.~Dodaj wartość zmiennej \texttt{intVar1} do~wartości
%   % \texttt{intVar3}. \\
%   % Na potrzeby naszej analizy przyjmujemy, że~każda operacja przypisania
%   % \texttt{=} przekazuje w~górę drzewa wartość jaką przypisana została jej
%   % lewej stronie. Stąd w~lewym wierzchołku \texttt{+} nie znajduje~się
%   % \texttt{intVar1}, lecz symbol przypisania \texttt{=}.

% \end{frame}
% % ##################
















% % ######################################
% \section{Drzewa składniowe pętli for}

% \label{sec:Podstawy-budowy-procesora-i-jezyka-asemblera}
% % ######################################














% % ##################
% \begin{frame}
%   \frametitle{Drzewo składniowe pętli for}


%   \begin{figure}

%     \begin{tikzpicture}

%       \path (0,0) -- (0,-2);



%       \node[text width=5em] (run of for loop) at (0,0) {\texttt{wykonanie pętli for}};

%       % \node (int-variable-4) at (-1.5,-1) {\texttt{intVar4}};

%       % \node (addition) at (1.5,-1) {\texttt{+}};


%       % \draw (assignment) -- (int-variable-4);

%       % \draw (assignment) -- (addition);


%       \node (condition) at (-3,-1.5) {\texttt{warunek}};

%       \node[text width=7em] (body) at (0,-1.5) {\texttt{ciało pętli for}};

%       \node[text width=7em] (ending instruction) at (3,-1.5) {\texttt{instrukcja kończąca}};


%       \draw (run of for loop) -- (condition);

%       \draw (run of for loop) -- (body);

%       \draw (run of for loop) -- (ending instruction);


%       % \node (number-2) at (1.5,-3) {\texttt{4}};

%       % \node (int-variable-3) at (3.5,-3) {\texttt{intVar3}};


%       % \draw (multiplication) -- (number-2);

%       % \draw (multiplication) -- (int-variable-3);

%     \end{tikzpicture}

%     \caption{Drzewo składniowe \texttt{wykonanie pętli for}}

%     \label{fig:Scheme-of-CPU}

%   \end{figure}





%   Jak można~się domyśli, w~obecnej sytuacji kolejność czytania wierzchołków
%   w~drzewie rozkładu jest kluczowa. Ich układ na drzewie obrazuje,
%   że~najpierw należy sprawdzić warunek, jeśli jest prawdziwy to wykonać
%   ciało pętli, następnie instrukcję kończącą i~wrócić do warunku. Jeśli
%   warunek nie jest spełniony, to kończymy pracę pętli i~przechodzimy w~górę
%   drzewa składniowego.

%   Proszę pamiętać, że~normalnie pętle nie „fruwają w~powietrzu”. Drzewo
%   składniowe takiej pętli będzie więc zwykle częścią większego drzewa
%   składniowego całego programu i~do niego pętla \texttt{for} wyśle
%   odpowiednie informacje.
%   % Choć \textsc{nasm} posiada inne rejestry niż te całkowitoliczbowe, to nie
%   % będziemy~się tym zajmować. Zaprowadziłoby to nas zbyt daleko od właściwej
%   % treści tego przedmiotu.

%   % Możemy przyjąć, że~każdy z~$16$ rejestr odpowiada za konkretne zadanie.
%   % Umieszczając w~nim liczbę całkowitą o~odpowiedniej wartości przekazujemy
%   % procesorowi informację o~tym co ma zrobić.

%   % Przypomnijmy, że~jeden rejestr dialektu \textsc{nams} składa~się z~$64$
%   % bitów, pozwala więc zapisać liczby całkowite od~$0$
%   % do~$18 \, 446 \, 744 \, 073 \, 709 \, 551 \, 616 \approx 1.8 \cdot 10^{ 19 }$.
%   % W~skrócie, dużo liczb.

% \end{frame}
% % ##################





















% ####################################################################

% End of the document
\end{document}
