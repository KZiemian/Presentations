% ------------------------------------------------------------------------------------------------------------------
% Basic configuration of Beamera class and Jagiellonian theme
% ------------------------------------------------------------------------------------------------------------------
\RequirePackage[l2tabu, orthodox]{nag}



\ifx\PresentationStyle\notset
  \def\PresentationStyle{dark}
\fi



\documentclass[10pt,t]{beamer}
\mode<presentation>
\usetheme[style=\PresentationStyle,JUlogotitle=no]{jagiellonian}




% ------------------------------------------------------------------------------------
% Procesing configuration files of Jagiellonian theme located in directory
% "preambule"
% ------------------------------------------------------------------------------------
% Configuration for polish language
% Need description
\usepackage[polish]{babel}
% Need description
\usepackage[MeX]{polski}



% ------------------------------
% Better support of polish chars in technical parts of PDF
% ------------------------------
\hypersetup{pdfencoding=auto,psdextra}

% Package "textpos" give as enviroment "textblock" which is very usefull in
% arranging text on slides.

% This is standard configuration of "textpos"
\usepackage[overlay,absolute]{textpos}

% If you need to see bounds of "textblock's" comment line above and uncomment
% one below.

% Caution! When showboxes option is on significant ammunt of space is add
% to the top of textblock and as such, everyting put in them gone down.
% We need to check how to remove this bug.

% \usepackage[showboxes,overlay,absolute]{textpos}



% Setting scale length for package "textpos"
\setlength{\TPHorizModule}{10mm}
\setlength{\TPVertModule}{\TPHorizModule}


% ---------------------------------------
% TikZ
% ---------------------------------------
% Importing TikZ libraries
\usetikzlibrary{arrows.meta}
\usetikzlibrary{positioning}





% % Configuration package "bm" that need for making bold symbols
% \newcommand{\bmmax}{0}
% \newcommand{\hmmax}{0}
% \usepackage{bm}




% ---------------------------------------
% Packages for scientific texts
% ---------------------------------------
% \let\lll\undefined  % Sometimes you must use this line to allow
% "amsmath" package to works with packages with packages for polish
% languge imported
% /preambul/LanguageSettings/JagiellonianPolishLanguageSettings.tex.
% This comments (probably) removes polish letter Ł.
\usepackage{amsmath}  % Packages from American Mathematical Society (AMS)
\usepackage{amssymb}
\usepackage{amscd}
\usepackage{amsthm}
\usepackage{siunitx}  % Package for typsetting SI units.
\usepackage{upgreek}  % Better looking greek letters.
% Example of using upgreek: pi = \uppi


\usepackage{calrsfs}  % Zmienia czcionkę kaligraficzną w \mathcal
% na ładniejszą. Może w innych miejscach robi to samo, ale o tym nic
% nie wiem.










% ---------------------------------------
% Packages written for lectures "Geometria 3D dla twórców gier wideo"
% ---------------------------------------
% \usepackage{./ProgramowanieSymulacjiFizykiPaczki/ProgramowanieSymulacjiFizyki}
% \usepackage{./ProgramowanieSymulacjiFizykiPaczki/ProgramowanieSymulacjiFizykiIndeksy}
% \usepackage{./ProgramowanieSymulacjiFizykiPaczki/ProgramowanieSymulacjiFizykiTikZStyle}





% !!!!!!!!!!!!!!!!!!!!!!!!!!!!!!
% !!!!!!!!!!!!!!!!!!!!!!!!!!!!!!
% EVIL STUFF
\if\JUlogotitle1
\edef\LogoJUPath{LogoJU_\JUlogoLang/LogoJU_\JUlogoShape_\JUlogoColor.pdf}
\titlegraphic{\hfill\includegraphics[scale=0.22]
{./JagiellonianPictures/\LogoJUPath}}
\fi
% ---------------------------------------
% Commands for handling colors
% ---------------------------------------


% Command for setting normal text color for some text in math modestyle
% Text color depend on used style of Jagiellonian

% Beamer version of command
\newcommand{\TextWithNormalTextColor}[1]{%
  {\color{jNormalTextFGColor}
    \setbeamercolor{math text}{fg=jNormalTextFGColor} {#1}}
}

% Article and similar classes version of command
% \newcommand{\TextWithNormalTextColor}[1]{%
%   {\color{jNormalTextsFGColor} {#1}}
% }



% Beamer version of command
\newcommand{\NormalTextInMathMode}[1]{%
  {\color{jNormalTextFGColor}
    \setbeamercolor{math text}{fg=jNormalTextFGColor} \text{#1}}
}


% Article and similar classes version of command
% \newcommand{\NormalTextInMathMode}[1]{%
%   {\color{jNormalTextsFGColor} \text{#1}}
% }




% Command that sets color of some mathematical text to the same color
% that has normal text in header (?)

% Beamer version of the command
\newcommand{\MathTextFrametitleFGColor}[1]{%
  {\color{jFrametitleFGColor}
    \setbeamercolor{math text}{fg=jFrametitleFGColor} #1}
}

% Article and similar classes version of the command
% \newcommand{\MathTextWhiteColor}[1]{{\color{jFrametitleFGColor} #1}}





% Command for setting color of alert text for some text in math modestyle

% Beamer version of the command
\newcommand{\MathTextAlertColor}[1]{%
  {\color{jOrange} \setbeamercolor{math text}{fg=jOrange} #1}
}

% Article and similar classes version of the command
% \newcommand{\MathTextAlertColor}[1]{{\color{jOrange} #1}}





% Command that allow you to sets chosen color as the color of some text into
% math mode. Due to some nuances in the way that Beamer handle colors
% it not work in all cases. We hope that in the future we will improve it.

% Beamer version of the command
\newcommand{\SetMathTextColor}[2]{%
  {\color{#1} \setbeamercolor{math text}{fg=#1} #2}
}


% Article and similar classes version of the command
% \newcommand{\SetMathTextColor}[2]{{\color{#1} #2}}










% ---------------------------------------
% Commands for few special slides
% ---------------------------------------
\newcommand{\EndingSlide}[1]{%
  \begin{frame}[standout]

    \begingroup

    \color{jFrametitleFGColor}

    #1

    \endgroup

  \end{frame}
}










% ---------------------------------------
% Commands for setting background pictures for some slides
% ---------------------------------------
\newcommand{\TitleBackgroundPicture}
{./JagiellonianPictures/Backgrounds/LajkonikDark.png}
\newcommand{\SectionBackgroundPicture}
{./JagiellonianPictures/Backgrounds/LajkonikLight.png}



\newcommand{\TitleSlideWithPicture}{%
  \begingroup

  \usebackgroundtemplate{%
    \includegraphics[height=\paperheight]{\TitleBackgroundPicture}}

  \maketitle

  \endgroup
}





\newcommand{\SectionSlideWithPicture}[1]{%
  \begingroup

  \usebackgroundtemplate{%
    \includegraphics[height=\paperheight]{\SectionBackgroundPicture}}

  \setbeamercolor{titlelike}{fg=normal text.fg}

  \section{#1}

  \endgroup
}










% ---------------------------------------
% Commands for lectures "Geometria 3D dla twórców gier wideo"
% Polish version
% ---------------------------------------
% Komendy teraz wykomentowane były potrzebne, gdy loga były na niebieskim
% tle, nie na białym. A są na białym bo tego chcieli w biurze projektu.
% \newcommand{\FundingLogoWhitePicturePL}
% {./PresentationPictures/CommonPictures/logotypFundusze_biale_bez_tla2.pdf}
\newcommand{\FundingLogoColorPicturePL}
{./PresentationPictures/CommonPictures/European_Funds_color_PL.pdf}
% \newcommand{\EULogoWhitePicturePL}
% {./PresentationPictures/CommonPictures/logotypUE_biale_bez_tla2.pdf}
\newcommand{\EUSocialFundLogoColorPicturePL}
{./PresentationPictures/CommonPictures/EU_Social_Fund_color_PL.pdf}
% \newcommand{\ZintegrUJLogoWhitePicturePL}
% {./PresentationPictures/CommonPictures/zintegruj-logo-white.pdf}
\newcommand{\ZintegrUJLogoColorPicturePL}
{./PresentationPictures/CommonPictures/ZintegrUJ_color.pdf}
\newcommand{\JULogoColorPicturePL}
{./JagiellonianPictures/LogoJU_PL/LogoJU_A_color.pdf}





\newcommand{\GeometryThreeDSpecialBeginningSlidePL}{%
  \begin{frame}[standout]

    \begin{textblock}{11}(1,0.7)

      \begin{flushleft}

        \mdseries

        \footnotesize

        \color{jFrametitleFGColor}

        Materiał powstał w ramach projektu współfinansowanego ze środków
        Unii Europejskiej w ramach Europejskiego Funduszu Społecznego
        POWR.03.05.00-00-Z309/17-00.

      \end{flushleft}

    \end{textblock}





    \begin{textblock}{10}(0,2.2)

      \tikz \fill[color=jBackgroundStyleLight] (0,0) rectangle (12.8,-1.5);

    \end{textblock}


    \begin{textblock}{3.2}(1,2.45)

      \includegraphics[scale=0.3]{\FundingLogoColorPicturePL}

    \end{textblock}


    \begin{textblock}{2.5}(3.7,2.5)

      \includegraphics[scale=0.2]{\JULogoColorPicturePL}

    \end{textblock}


    \begin{textblock}{2.5}(6,2.4)

      \includegraphics[scale=0.1]{\ZintegrUJLogoColorPicturePL}

    \end{textblock}


    \begin{textblock}{4.2}(8.4,2.6)

      \includegraphics[scale=0.3]{\EUSocialFundLogoColorPicturePL}

    \end{textblock}

  \end{frame}
}



\newcommand{\GeometryThreeDTwoSpecialBeginningSlidesPL}{%
  \begin{frame}[standout]

    \begin{textblock}{11}(1,0.7)

      \begin{flushleft}

        \mdseries

        \footnotesize

        \color{jFrametitleFGColor}

        Materiał powstał w ramach projektu współfinansowanego ze środków
        Unii Europejskiej w ramach Europejskiego Funduszu Społecznego
        POWR.03.05.00-00-Z309/17-00.

      \end{flushleft}

    \end{textblock}





    \begin{textblock}{10}(0,2.2)

      \tikz \fill[color=jBackgroundStyleLight] (0,0) rectangle (12.8,-1.5);

    \end{textblock}


    \begin{textblock}{3.2}(1,2.45)

      \includegraphics[scale=0.3]{\FundingLogoColorPicturePL}

    \end{textblock}


    \begin{textblock}{2.5}(3.7,2.5)

      \includegraphics[scale=0.2]{\JULogoColorPicturePL}

    \end{textblock}


    \begin{textblock}{2.5}(6,2.4)

      \includegraphics[scale=0.1]{\ZintegrUJLogoColorPicturePL}

    \end{textblock}


    \begin{textblock}{4.2}(8.4,2.6)

      \includegraphics[scale=0.3]{\EUSocialFundLogoColorPicturePL}

    \end{textblock}

  \end{frame}





  \TitleSlideWithPicture
}



\newcommand{\GeometryThreeDSpecialEndingSlidePL}{%
  \begin{frame}[standout]

    \begin{textblock}{11}(1,0.7)

      \begin{flushleft}

        \mdseries

        \footnotesize

        \color{jFrametitleFGColor}

        Materiał powstał w ramach projektu współfinansowanego ze środków
        Unii Europejskiej w~ramach Europejskiego Funduszu Społecznego
        POWR.03.05.00-00-Z309/17-00.

      \end{flushleft}

    \end{textblock}





    \begin{textblock}{10}(0,2.2)

      \tikz \fill[color=jBackgroundStyleLight] (0,0) rectangle (12.8,-1.5);

    \end{textblock}


    \begin{textblock}{3.2}(1,2.45)

      \includegraphics[scale=0.3]{\FundingLogoColorPicturePL}

    \end{textblock}


    \begin{textblock}{2.5}(3.7,2.5)

      \includegraphics[scale=0.2]{\JULogoColorPicturePL}

    \end{textblock}


    \begin{textblock}{2.5}(6,2.4)

      \includegraphics[scale=0.1]{\ZintegrUJLogoColorPicturePL}

    \end{textblock}


    \begin{textblock}{4.2}(8.4,2.6)

      \includegraphics[scale=0.3]{\EUSocialFundLogoColorPicturePL}

    \end{textblock}





    \begin{textblock}{11}(1,4)

      \begin{flushleft}

        \mdseries

        \footnotesize

        \RaggedRight

        \color{jFrametitleFGColor}

        Treść niniejszego wykładu jest udostępniona na~licencji
        Creative Commons (\textsc{cc}), z~uzna\-niem autorstwa
        (\textsc{by}) oraz udostępnianiem na tych samych warunkach
        (\textsc{sa}). Rysunki i~wy\-kresy zawarte w~wykładzie są
        autorstwa dr.~hab.~Pawła Węgrzyna et~al. i~są dostępne
        na tej samej licencji, o~ile nie wskazano inaczej.
        W~prezentacji wykorzystano temat Beamera Jagiellonian,
        oparty na~temacie Metropolis Matthiasa Vogelgesanga,
        dostępnym na licencji \LaTeX{} Project Public License~1.3c
        pod adresem: \colorhref{https://github.com/matze/mtheme}
        {https://github.com/matze/mtheme}.

        Projekt typograficzny: Iwona Grabska-Gradzińska \\
        Skład: Kamil Ziemian;
        Korekta: Wojciech Palacz \\
        Modele: Dariusz Frymus, Kamil Nowakowski \\
        Rysunki i~wykresy: Kamil Ziemian, Paweł Węgrzyn, Wojciech Palacz

      \end{flushleft}

    \end{textblock}

  \end{frame}
}



\newcommand{\GeometryThreeDTwoSpecialEndingSlidesPL}[1]{%
  \begin{frame}[standout]


    \begin{textblock}{11}(1,0.7)

      \begin{flushleft}

        \mdseries

        \footnotesize

        \color{jFrametitleFGColor}

        Materiał powstał w ramach projektu współfinansowanego ze środków
        Unii Europejskiej w~ramach Europejskiego Funduszu Społecznego
        POWR.03.05.00-00-Z309/17-00.

      \end{flushleft}

    \end{textblock}





    \begin{textblock}{10}(0,2.2)

      \tikz \fill[color=jBackgroundStyleLight] (0,0) rectangle (12.8,-1.5);

    \end{textblock}


    \begin{textblock}{3.2}(1,2.45)

      \includegraphics[scale=0.3]{\FundingLogoColorPicturePL}

    \end{textblock}


    \begin{textblock}{2.5}(3.7,2.5)

      \includegraphics[scale=0.2]{\JULogoColorPicturePL}

    \end{textblock}


    \begin{textblock}{2.5}(6,2.4)

      \includegraphics[scale=0.1]{\ZintegrUJLogoColorPicturePL}

    \end{textblock}


    \begin{textblock}{4.2}(8.4,2.6)

      \includegraphics[scale=0.3]{\EUSocialFundLogoColorPicturePL}

    \end{textblock}





    \begin{textblock}{11}(1,4)

      \begin{flushleft}

        \mdseries

        \footnotesize

        \RaggedRight

        \color{jFrametitleFGColor}

        Treść niniejszego wykładu jest udostępniona na~licencji
        Creative Commons (\textsc{cc}), z~uzna\-niem autorstwa
        (\textsc{by}) oraz udostępnianiem na tych samych warunkach
        (\textsc{sa}). Rysunki i~wy\-kresy zawarte w~wykładzie są
        autorstwa dr.~hab.~Pawła Węgrzyna et~al. i~są dostępne
        na tej samej licencji, o~ile nie wskazano inaczej.
        W~prezentacji wykorzystano temat Beamera Jagiellonian,
        oparty na~temacie Metropolis Matthiasa Vogelgesanga,
        dostępnym na licencji \LaTeX{} Project Public License~1.3c
        pod adresem: \colorhref{https://github.com/matze/mtheme}
        {https://github.com/matze/mtheme}.

        Projekt typograficzny: Iwona Grabska-Gradzińska \\
        Skład: Kamil Ziemian;
        Korekta: Wojciech Palacz \\
        Modele: Dariusz Frymus, Kamil Nowakowski \\
        Rysunki i~wykresy: Kamil Ziemian, Paweł Węgrzyn, Wojciech Palacz

      \end{flushleft}

    \end{textblock}

  \end{frame}





  \begin{frame}[standout]

    \begingroup

    \color{jFrametitleFGColor}

    #1

    \endgroup

  \end{frame}
}



\newcommand{\GeometryThreeDSpecialEndingSlideVideoPL}{%
  \begin{frame}[standout]

    \begin{textblock}{11}(1,0.7)

      \begin{flushleft}

        \mdseries

        \footnotesize

        \color{jFrametitleFGColor}

        Materiał powstał w ramach projektu współfinansowanego ze środków
        Unii Europejskiej w~ramach Europejskiego Funduszu Społecznego
        POWR.03.05.00-00-Z309/17-00.

      \end{flushleft}

    \end{textblock}





    \begin{textblock}{10}(0,2.2)

      \tikz \fill[color=jBackgroundStyleLight] (0,0) rectangle (12.8,-1.5);

    \end{textblock}


    \begin{textblock}{3.2}(1,2.45)

      \includegraphics[scale=0.3]{\FundingLogoColorPicturePL}

    \end{textblock}


    \begin{textblock}{2.5}(3.7,2.5)

      \includegraphics[scale=0.2]{\JULogoColorPicturePL}

    \end{textblock}


    \begin{textblock}{2.5}(6,2.4)

      \includegraphics[scale=0.1]{\ZintegrUJLogoColorPicturePL}

    \end{textblock}


    \begin{textblock}{4.2}(8.4,2.6)

      \includegraphics[scale=0.3]{\EUSocialFundLogoColorPicturePL}

    \end{textblock}





    \begin{textblock}{11}(1,4)

      \begin{flushleft}

        \mdseries

        \footnotesize

        \RaggedRight

        \color{jFrametitleFGColor}

        Treść niniejszego wykładu jest udostępniona na~licencji
        Creative Commons (\textsc{cc}), z~uzna\-niem autorstwa
        (\textsc{by}) oraz udostępnianiem na tych samych warunkach
        (\textsc{sa}). Rysunki i~wy\-kresy zawarte w~wykładzie są
        autorstwa dr.~hab.~Pawła Węgrzyna et~al. i~są dostępne
        na tej samej licencji, o~ile nie wskazano inaczej.
        W~prezentacji wykorzystano temat Beamera Jagiellonian,
        oparty na~temacie Metropolis Matthiasa Vogelgesanga,
        dostępnym na licencji \LaTeX{} Project Public License~1.3c
        pod adresem: \colorhref{https://github.com/matze/mtheme}
        {https://github.com/matze/mtheme}.

        Projekt typograficzny: Iwona Grabska-Gradzińska;
        Skład: Kamil Ziemian \\
        Korekta: Wojciech Palacz;
        Modele: Dariusz Frymus, Kamil Nowakowski \\
        Rysunki i~wykresy: Kamil Ziemian, Paweł Węgrzyn, Wojciech Palacz \\
        Montaż: Agencja Filmowa Film \& Television Production~-- Zbigniew
        Masklak

      \end{flushleft}

    \end{textblock}

  \end{frame}
}





\newcommand{\GeometryThreeDTwoSpecialEndingSlidesVideoPL}[1]{%
  \begin{frame}[standout]

    \begin{textblock}{11}(1,0.7)

      \begin{flushleft}

        \mdseries

        \footnotesize

        \color{jFrametitleFGColor}

        Materiał powstał w ramach projektu współfinansowanego ze środków
        Unii Europejskiej w~ramach Europejskiego Funduszu Społecznego
        POWR.03.05.00-00-Z309/17-00.

      \end{flushleft}

    \end{textblock}





    \begin{textblock}{10}(0,2.2)

      \tikz \fill[color=jBackgroundStyleLight] (0,0) rectangle (12.8,-1.5);

    \end{textblock}


    \begin{textblock}{3.2}(1,2.45)

      \includegraphics[scale=0.3]{\FundingLogoColorPicturePL}

    \end{textblock}


    \begin{textblock}{2.5}(3.7,2.5)

      \includegraphics[scale=0.2]{\JULogoColorPicturePL}

    \end{textblock}


    \begin{textblock}{2.5}(6,2.4)

      \includegraphics[scale=0.1]{\ZintegrUJLogoColorPicturePL}

    \end{textblock}


    \begin{textblock}{4.2}(8.4,2.6)

      \includegraphics[scale=0.3]{\EUSocialFundLogoColorPicturePL}

    \end{textblock}





    \begin{textblock}{11}(1,4)

      \begin{flushleft}

        \mdseries

        \footnotesize

        \RaggedRight

        \color{jFrametitleFGColor}

        Treść niniejszego wykładu jest udostępniona na~licencji
        Creative Commons (\textsc{cc}), z~uzna\-niem autorstwa
        (\textsc{by}) oraz udostępnianiem na tych samych warunkach
        (\textsc{sa}). Rysunki i~wy\-kresy zawarte w~wykładzie są
        autorstwa dr.~hab.~Pawła Węgrzyna et~al. i~są dostępne
        na tej samej licencji, o~ile nie wskazano inaczej.
        W~prezentacji wykorzystano temat Beamera Jagiellonian,
        oparty na~temacie Metropolis Matthiasa Vogelgesanga,
        dostępnym na licencji \LaTeX{} Project Public License~1.3c
        pod adresem: \colorhref{https://github.com/matze/mtheme}
        {https://github.com/matze/mtheme}.

        Projekt typograficzny: Iwona Grabska-Gradzińska;
        Skład: Kamil Ziemian \\
        Korekta: Wojciech Palacz;
        Modele: Dariusz Frymus, Kamil Nowakowski \\
        Rysunki i~wykresy: Kamil Ziemian, Paweł Węgrzyn, Wojciech Palacz \\
        Montaż: Agencja Filmowa Film \& Television Production~-- Zbigniew
        Masklak

      \end{flushleft}

    \end{textblock}

  \end{frame}





  \begin{frame}[standout]


    \begingroup

    \color{jFrametitleFGColor}

    #1

    \endgroup

  \end{frame}
}










% ---------------------------------------
% Commands for lectures "Geometria 3D dla twórców gier wideo"
% English version
% ---------------------------------------
% \newcommand{\FundingLogoWhitePictureEN}
% {./PresentationPictures/CommonPictures/logotypFundusze_biale_bez_tla2.pdf}
\newcommand{\FundingLogoColorPictureEN}
{./PresentationPictures/CommonPictures/European_Funds_color_EN.pdf}
% \newcommand{\EULogoWhitePictureEN}
% {./PresentationPictures/CommonPictures/logotypUE_biale_bez_tla2.pdf}
\newcommand{\EUSocialFundLogoColorPictureEN}
{./PresentationPictures/CommonPictures/EU_Social_Fund_color_EN.pdf}
% \newcommand{\ZintegrUJLogoWhitePictureEN}
% {./PresentationPictures/CommonPictures/zintegruj-logo-white.pdf}
\newcommand{\ZintegrUJLogoColorPictureEN}
{./PresentationPictures/CommonPictures/ZintegrUJ_color.pdf}
\newcommand{\JULogoColorPictureEN}
{./JagiellonianPictures/LogoJU_EN/LogoJU_A_color.pdf}



\newcommand{\GeometryThreeDSpecialBeginningSlideEN}{%
  \begin{frame}[standout]

    \begin{textblock}{11}(1,0.7)

      \begin{flushleft}

        \mdseries

        \footnotesize

        \color{jFrametitleFGColor}

        This content was created as part of a project co-financed by the
        European Union within the framework of the European Social Fund
        POWR.03.05.00-00-Z309/17-00.

      \end{flushleft}

    \end{textblock}





    \begin{textblock}{10}(0,2.2)

      \tikz \fill[color=jBackgroundStyleLight] (0,0) rectangle (12.8,-1.5);

    \end{textblock}


    \begin{textblock}{3.2}(0.7,2.45)

      \includegraphics[scale=0.3]{\FundingLogoColorPictureEN}

    \end{textblock}


    \begin{textblock}{2.5}(4.15,2.5)

      \includegraphics[scale=0.2]{\JULogoColorPictureEN}

    \end{textblock}


    \begin{textblock}{2.5}(6.35,2.4)

      \includegraphics[scale=0.1]{\ZintegrUJLogoColorPictureEN}

    \end{textblock}


    \begin{textblock}{4.2}(8.4,2.6)

      \includegraphics[scale=0.3]{\EUSocialFundLogoColorPictureEN}

    \end{textblock}

  \end{frame}
}



\newcommand{\GeometryThreeDTwoSpecialBeginningSlidesEN}{%
  \begin{frame}[standout]

    \begin{textblock}{11}(1,0.7)

      \begin{flushleft}

        \mdseries

        \footnotesize

        \color{jFrametitleFGColor}

        This content was created as part of a project co-financed by the
        European Union within the framework of the European Social Fund
        POWR.03.05.00-00-Z309/17-00.

      \end{flushleft}

    \end{textblock}





    \begin{textblock}{10}(0,2.2)

      \tikz \fill[color=jBackgroundStyleLight] (0,0) rectangle (12.8,-1.5);

    \end{textblock}


    \begin{textblock}{3.2}(0.7,2.45)

      \includegraphics[scale=0.3]{\FundingLogoColorPictureEN}

    \end{textblock}


    \begin{textblock}{2.5}(4.15,2.5)

      \includegraphics[scale=0.2]{\JULogoColorPictureEN}

    \end{textblock}


    \begin{textblock}{2.5}(6.35,2.4)

      \includegraphics[scale=0.1]{\ZintegrUJLogoColorPictureEN}

    \end{textblock}


    \begin{textblock}{4.2}(8.4,2.6)

      \includegraphics[scale=0.3]{\EUSocialFundLogoColorPictureEN}

    \end{textblock}

  \end{frame}





  \TitleSlideWithPicture
}



\newcommand{\GeometryThreeDSpecialEndingSlideEN}{%
  \begin{frame}[standout]

    \begin{textblock}{11}(1,0.7)

      \begin{flushleft}

        \mdseries

        \footnotesize

        \color{jFrametitleFGColor}

        This content was created as part of a project co-financed by the
        European Union within the framework of the European Social Fund
        POWR.03.05.00-00-Z309/17-00.

      \end{flushleft}

    \end{textblock}





    \begin{textblock}{10}(0,2.2)

      \tikz \fill[color=jBackgroundStyleLight] (0,0) rectangle (12.8,-1.5);

    \end{textblock}


    \begin{textblock}{3.2}(0.7,2.45)

      \includegraphics[scale=0.3]{\FundingLogoColorPictureEN}

    \end{textblock}


    \begin{textblock}{2.5}(4.15,2.5)

      \includegraphics[scale=0.2]{\JULogoColorPictureEN}

    \end{textblock}


    \begin{textblock}{2.5}(6.35,2.4)

      \includegraphics[scale=0.1]{\ZintegrUJLogoColorPictureEN}

    \end{textblock}


    \begin{textblock}{4.2}(8.4,2.6)

      \includegraphics[scale=0.3]{\EUSocialFundLogoColorPictureEN}

    \end{textblock}





    \begin{textblock}{11}(1,4)

      \begin{flushleft}

        \mdseries

        \footnotesize

        \RaggedRight

        \color{jFrametitleFGColor}

        The content of this lecture is made available under a~Creative
        Commons licence (\textsc{cc}), giving the author the credits
        (\textsc{by}) and putting an obligation to share on the same terms
        (\textsc{sa}). Figures and diagrams included in the lecture are
        authored by Paweł Węgrzyn et~al., and are available under the same
        license unless indicated otherwise.\\ The presentation uses the
        Beamer Jagiellonian theme based on Matthias Vogelgesang’s
        Metropolis theme, available under license \LaTeX{} Project
        Public License~1.3c at: \colorhref{https://github.com/matze/mtheme}
        {https://github.com/matze/mtheme}.

        Typographic design: Iwona Grabska-Gradzińska \\
        \LaTeX{} Typesetting: Kamil Ziemian \\
        Proofreading: Wojciech Palacz,
        Monika Stawicka \\
        3D Models: Dariusz Frymus, Kamil Nowakowski \\
        Figures and charts: Kamil Ziemian, Paweł Węgrzyn, Wojciech Palacz

      \end{flushleft}

    \end{textblock}

  \end{frame}
}



\newcommand{\GeometryThreeDTwoSpecialEndingSlidesEN}[1]{%
  \begin{frame}[standout]


    \begin{textblock}{11}(1,0.7)

      \begin{flushleft}

        \mdseries

        \footnotesize

        \color{jFrametitleFGColor}

        This content was created as part of a project co-financed by the
        European Union within the framework of the European Social Fund
        POWR.03.05.00-00-Z309/17-00.

      \end{flushleft}

    \end{textblock}





    \begin{textblock}{10}(0,2.2)

      \tikz \fill[color=jBackgroundStyleLight] (0,0) rectangle (12.8,-1.5);

    \end{textblock}


    \begin{textblock}{3.2}(0.7,2.45)

      \includegraphics[scale=0.3]{\FundingLogoColorPictureEN}

    \end{textblock}


    \begin{textblock}{2.5}(4.15,2.5)

      \includegraphics[scale=0.2]{\JULogoColorPictureEN}

    \end{textblock}


    \begin{textblock}{2.5}(6.35,2.4)

      \includegraphics[scale=0.1]{\ZintegrUJLogoColorPictureEN}

    \end{textblock}


    \begin{textblock}{4.2}(8.4,2.6)

      \includegraphics[scale=0.3]{\EUSocialFundLogoColorPictureEN}

    \end{textblock}





    \begin{textblock}{11}(1,4)

      \begin{flushleft}

        \mdseries

        \footnotesize

        \RaggedRight

        \color{jFrametitleFGColor}

        The content of this lecture is made available under a~Creative
        Commons licence (\textsc{cc}), giving the author the credits
        (\textsc{by}) and putting an obligation to share on the same terms
        (\textsc{sa}). Figures and diagrams included in the lecture are
        authored by Paweł Węgrzyn et~al., and are available under the same
        license unless indicated otherwise.\\ The presentation uses the
        Beamer Jagiellonian theme based on Matthias Vogelgesang’s
        Metropolis theme, available under license \LaTeX{} Project
        Public License~1.3c at: \colorhref{https://github.com/matze/mtheme}
        {https://github.com/matze/mtheme}.

        Typographic design: Iwona Grabska-Gradzińska \\
        \LaTeX{} Typesetting: Kamil Ziemian \\
        Proofreading: Wojciech Palacz,
        Monika Stawicka \\
        3D Models: Dariusz Frymus, Kamil Nowakowski \\
        Figures and charts: Kamil Ziemian, Paweł Węgrzyn, Wojciech Palacz

      \end{flushleft}

    \end{textblock}

  \end{frame}





  \begin{frame}[standout]

    \begingroup

    \color{jFrametitleFGColor}

    #1

    \endgroup

  \end{frame}
}



\newcommand{\GeometryThreeDSpecialEndingSlideVideoVerOneEN}{%
  \begin{frame}[standout]

    \begin{textblock}{11}(1,0.7)

      \begin{flushleft}

        \mdseries

        \footnotesize

        \color{jFrametitleFGColor}

        This content was created as part of a project co-financed by the
        European Union within the framework of the European Social Fund
        POWR.03.05.00-00-Z309/17-00.

      \end{flushleft}

    \end{textblock}





    \begin{textblock}{10}(0,2.2)

      \tikz \fill[color=jBackgroundStyleLight] (0,0) rectangle (12.8,-1.5);

    \end{textblock}


    \begin{textblock}{3.2}(0.7,2.45)

      \includegraphics[scale=0.3]{\FundingLogoColorPictureEN}

    \end{textblock}


    \begin{textblock}{2.5}(4.15,2.5)

      \includegraphics[scale=0.2]{\JULogoColorPictureEN}

    \end{textblock}


    \begin{textblock}{2.5}(6.35,2.4)

      \includegraphics[scale=0.1]{\ZintegrUJLogoColorPictureEN}

    \end{textblock}


    \begin{textblock}{4.2}(8.4,2.6)

      \includegraphics[scale=0.3]{\EUSocialFundLogoColorPictureEN}

    \end{textblock}





    \begin{textblock}{11}(1,4)

      \begin{flushleft}

        \mdseries

        \footnotesize

        \RaggedRight

        \color{jFrametitleFGColor}

        The content of this lecture is made available under a Creative
        Commons licence (\textsc{cc}), giving the author the credits
        (\textsc{by}) and putting an obligation to share on the same terms
        (\textsc{sa}). Figures and diagrams included in the lecture are
        authored by Paweł Węgrzyn et~al., and are available under the same
        license unless indicated otherwise.\\ The presentation uses the
        Beamer Jagiellonian theme based on Matthias Vogelgesang’s
        Metropolis theme, available under license \LaTeX{} Project
        Public License~1.3c at: \colorhref{https://github.com/matze/mtheme}
        {https://github.com/matze/mtheme}.

        Typographic design: Iwona Grabska-Gradzińska;
        \LaTeX{} Typesetting: Kamil Ziemian \\
        Proofreading: Wojciech Palacz,
        Monika Stawicka \\
        3D Models: Dariusz Frymus, Kamil Nowakowski \\
        Figures and charts: Kamil Ziemian, Paweł Węgrzyn, Wojciech
        Palacz \\
        Film editing: Agencja Filmowa Film \& Television Production~--
        Zbigniew Masklak

      \end{flushleft}

    \end{textblock}

  \end{frame}
}



\newcommand{\GeometryThreeDSpecialEndingSlideVideoVerTwoEN}{%
  \begin{frame}[standout]

    \begin{textblock}{11}(1,0.7)

      \begin{flushleft}

        \mdseries

        \footnotesize

        \color{jFrametitleFGColor}

        This content was created as part of a project co-financed by the
        European Union within the framework of the European Social Fund
        POWR.03.05.00-00-Z309/17-00.

      \end{flushleft}

    \end{textblock}





    \begin{textblock}{10}(0,2.2)

      \tikz \fill[color=jBackgroundStyleLight] (0,0) rectangle (12.8,-1.5);

    \end{textblock}


    \begin{textblock}{3.2}(0.7,2.45)

      \includegraphics[scale=0.3]{\FundingLogoColorPictureEN}

    \end{textblock}


    \begin{textblock}{2.5}(4.15,2.5)

      \includegraphics[scale=0.2]{\JULogoColorPictureEN}

    \end{textblock}


    \begin{textblock}{2.5}(6.35,2.4)

      \includegraphics[scale=0.1]{\ZintegrUJLogoColorPictureEN}

    \end{textblock}


    \begin{textblock}{4.2}(8.4,2.6)

      \includegraphics[scale=0.3]{\EUSocialFundLogoColorPictureEN}

    \end{textblock}





    \begin{textblock}{11}(1,4)

      \begin{flushleft}

        \mdseries

        \footnotesize

        \RaggedRight

        \color{jFrametitleFGColor}

        The content of this lecture is made available under a Creative
        Commons licence (\textsc{cc}), giving the author the credits
        (\textsc{by}) and putting an obligation to share on the same terms
        (\textsc{sa}). Figures and diagrams included in the lecture are
        authored by Paweł Węgrzyn et~al., and are available under the same
        license unless indicated otherwise.\\ The presentation uses the
        Beamer Jagiellonian theme based on Matthias Vogelgesang’s
        Metropolis theme, available under license \LaTeX{} Project
        Public License~1.3c at: \colorhref{https://github.com/matze/mtheme}
        {https://github.com/matze/mtheme}.

        Typographic design: Iwona Grabska-Gradzińska;
        \LaTeX{} Typesetting: Kamil Ziemian \\
        Proofreading: Wojciech Palacz,
        Monika Stawicka \\
        3D Models: Dariusz Frymus, Kamil Nowakowski \\
        Figures and charts: Kamil Ziemian, Paweł Węgrzyn, Wojciech
        Palacz \\
        Film editing: IMAVI -- Joanna Kozakiewicz, Krzysztof Magda, Nikodem
        Frodyma

      \end{flushleft}

    \end{textblock}

  \end{frame}
}



\newcommand{\GeometryThreeDSpecialEndingSlideVideoVerThreeEN}{%
  \begin{frame}[standout]

    \begin{textblock}{11}(1,0.7)

      \begin{flushleft}

        \mdseries

        \footnotesize

        \color{jFrametitleFGColor}

        This content was created as part of a project co-financed by the
        European Union within the framework of the European Social Fund
        POWR.03.05.00-00-Z309/17-00.

      \end{flushleft}

    \end{textblock}





    \begin{textblock}{10}(0,2.2)

      \tikz \fill[color=jBackgroundStyleLight] (0,0) rectangle (12.8,-1.5);

    \end{textblock}


    \begin{textblock}{3.2}(0.7,2.45)

      \includegraphics[scale=0.3]{\FundingLogoColorPictureEN}

    \end{textblock}


    \begin{textblock}{2.5}(4.15,2.5)

      \includegraphics[scale=0.2]{\JULogoColorPictureEN}

    \end{textblock}


    \begin{textblock}{2.5}(6.35,2.4)

      \includegraphics[scale=0.1]{\ZintegrUJLogoColorPictureEN}

    \end{textblock}


    \begin{textblock}{4.2}(8.4,2.6)

      \includegraphics[scale=0.3]{\EUSocialFundLogoColorPictureEN}

    \end{textblock}





    \begin{textblock}{11}(1,4)

      \begin{flushleft}

        \mdseries

        \footnotesize

        \RaggedRight

        \color{jFrametitleFGColor}

        The content of this lecture is made available under a Creative
        Commons licence (\textsc{cc}), giving the author the credits
        (\textsc{by}) and putting an obligation to share on the same terms
        (\textsc{sa}). Figures and diagrams included in the lecture are
        authored by Paweł Węgrzyn et~al., and are available under the same
        license unless indicated otherwise.\\ The presentation uses the
        Beamer Jagiellonian theme based on Matthias Vogelgesang’s
        Metropolis theme, available under license \LaTeX{} Project
        Public License~1.3c at: \colorhref{https://github.com/matze/mtheme}
        {https://github.com/matze/mtheme}.

        Typographic design: Iwona Grabska-Gradzińska;
        \LaTeX{} Typesetting: Kamil Ziemian \\
        Proofreading: Wojciech Palacz,
        Monika Stawicka \\
        3D Models: Dariusz Frymus, Kamil Nowakowski \\
        Figures and charts: Kamil Ziemian, Paweł Węgrzyn, Wojciech
        Palacz \\
        Film editing: Agencja Filmowa Film \& Television Production~--
        Zbigniew Masklak \\
        Film editing: IMAVI -- Joanna Kozakiewicz, Krzysztof Magda, Nikodem
        Frodyma

      \end{flushleft}

    \end{textblock}

  \end{frame}
}



\newcommand{\GeometryThreeDTwoSpecialEndingSlidesVideoVerOneEN}[1]{%
  \begin{frame}[standout]

    \begin{textblock}{11}(1,0.7)

      \begin{flushleft}

        \mdseries

        \footnotesize

        \color{jFrametitleFGColor}

        This content was created as part of a project co-financed by the
        European Union within the framework of the European Social Fund
        POWR.03.05.00-00-Z309/17-00.

      \end{flushleft}

    \end{textblock}





    \begin{textblock}{10}(0,2.2)

      \tikz \fill[color=jBackgroundStyleLight] (0,0) rectangle (12.8,-1.5);

    \end{textblock}


    \begin{textblock}{3.2}(0.7,2.45)

      \includegraphics[scale=0.3]{\FundingLogoColorPictureEN}

    \end{textblock}


    \begin{textblock}{2.5}(4.15,2.5)

      \includegraphics[scale=0.2]{\JULogoColorPictureEN}

    \end{textblock}


    \begin{textblock}{2.5}(6.35,2.4)

      \includegraphics[scale=0.1]{\ZintegrUJLogoColorPictureEN}

    \end{textblock}


    \begin{textblock}{4.2}(8.4,2.6)

      \includegraphics[scale=0.3]{\EUSocialFundLogoColorPictureEN}

    \end{textblock}





    \begin{textblock}{11}(1,4)

      \begin{flushleft}

        \mdseries

        \footnotesize

        \RaggedRight

        \color{jFrametitleFGColor}

        The content of this lecture is made available under a Creative
        Commons licence (\textsc{cc}), giving the author the credits
        (\textsc{by}) and putting an obligation to share on the same terms
        (\textsc{sa}). Figures and diagrams included in the lecture are
        authored by Paweł Węgrzyn et~al., and are available under the same
        license unless indicated otherwise.\\ The presentation uses the
        Beamer Jagiellonian theme based on Matthias Vogelgesang’s
        Metropolis theme, available under license \LaTeX{} Project
        Public License~1.3c at: \colorhref{https://github.com/matze/mtheme}
        {https://github.com/matze/mtheme}.

        Typographic design: Iwona Grabska-Gradzińska;
        \LaTeX{} Typesetting: Kamil Ziemian \\
        Proofreading: Wojciech Palacz,
        Monika Stawicka \\
        3D Models: Dariusz Frymus, Kamil Nowakowski \\
        Figures and charts: Kamil Ziemian, Paweł Węgrzyn,
        Wojciech Palacz \\
        Film editing: Agencja Filmowa Film \& Television Production~--
        Zbigniew Masklak

      \end{flushleft}

    \end{textblock}

  \end{frame}





  \begin{frame}[standout]


    \begingroup

    \color{jFrametitleFGColor}

    #1

    \endgroup

  \end{frame}
}



\newcommand{\GeometryThreeDTwoSpecialEndingSlidesVideoVerTwoEN}[1]{%
  \begin{frame}[standout]

    \begin{textblock}{11}(1,0.7)

      \begin{flushleft}

        \mdseries

        \footnotesize

        \color{jFrametitleFGColor}

        This content was created as part of a project co-financed by the
        European Union within the framework of the European Social Fund
        POWR.03.05.00-00-Z309/17-00.

      \end{flushleft}

    \end{textblock}





    \begin{textblock}{10}(0,2.2)

      \tikz \fill[color=jBackgroundStyleLight] (0,0) rectangle (12.8,-1.5);

    \end{textblock}


    \begin{textblock}{3.2}(0.7,2.45)

      \includegraphics[scale=0.3]{\FundingLogoColorPictureEN}

    \end{textblock}


    \begin{textblock}{2.5}(4.15,2.5)

      \includegraphics[scale=0.2]{\JULogoColorPictureEN}

    \end{textblock}


    \begin{textblock}{2.5}(6.35,2.4)

      \includegraphics[scale=0.1]{\ZintegrUJLogoColorPictureEN}

    \end{textblock}


    \begin{textblock}{4.2}(8.4,2.6)

      \includegraphics[scale=0.3]{\EUSocialFundLogoColorPictureEN}

    \end{textblock}





    \begin{textblock}{11}(1,4)

      \begin{flushleft}

        \mdseries

        \footnotesize

        \RaggedRight

        \color{jFrametitleFGColor}

        The content of this lecture is made available under a Creative
        Commons licence (\textsc{cc}), giving the author the credits
        (\textsc{by}) and putting an obligation to share on the same terms
        (\textsc{sa}). Figures and diagrams included in the lecture are
        authored by Paweł Węgrzyn et~al., and are available under the same
        license unless indicated otherwise.\\ The presentation uses the
        Beamer Jagiellonian theme based on Matthias Vogelgesang’s
        Metropolis theme, available under license \LaTeX{} Project
        Public License~1.3c at: \colorhref{https://github.com/matze/mtheme}
        {https://github.com/matze/mtheme}.

        Typographic design: Iwona Grabska-Gradzińska;
        \LaTeX{} Typesetting: Kamil Ziemian \\
        Proofreading: Wojciech Palacz,
        Monika Stawicka \\
        3D Models: Dariusz Frymus, Kamil Nowakowski \\
        Figures and charts: Kamil Ziemian, Paweł Węgrzyn,
        Wojciech Palacz \\
        Film editing: IMAVI -- Joanna Kozakiewicz, Krzysztof Magda, Nikodem
        Frodyma

      \end{flushleft}

    \end{textblock}

  \end{frame}





  \begin{frame}[standout]


    \begingroup

    \color{jFrametitleFGColor}

    #1

    \endgroup

  \end{frame}
}



\newcommand{\GeometryThreeDTwoSpecialEndingSlidesVideoVerThreeEN}[1]{%
  \begin{frame}[standout]

    \begin{textblock}{11}(1,0.7)

      \begin{flushleft}

        \mdseries

        \footnotesize

        \color{jFrametitleFGColor}

        This content was created as part of a project co-financed by the
        European Union within the framework of the European Social Fund
        POWR.03.05.00-00-Z309/17-00.

      \end{flushleft}

    \end{textblock}





    \begin{textblock}{10}(0,2.2)

      \tikz \fill[color=jBackgroundStyleLight] (0,0) rectangle (12.8,-1.5);

    \end{textblock}


    \begin{textblock}{3.2}(0.7,2.45)

      \includegraphics[scale=0.3]{\FundingLogoColorPictureEN}

    \end{textblock}


    \begin{textblock}{2.5}(4.15,2.5)

      \includegraphics[scale=0.2]{\JULogoColorPictureEN}

    \end{textblock}


    \begin{textblock}{2.5}(6.35,2.4)

      \includegraphics[scale=0.1]{\ZintegrUJLogoColorPictureEN}

    \end{textblock}


    \begin{textblock}{4.2}(8.4,2.6)

      \includegraphics[scale=0.3]{\EUSocialFundLogoColorPictureEN}

    \end{textblock}





    \begin{textblock}{11}(1,4)

      \begin{flushleft}

        \mdseries

        \footnotesize

        \RaggedRight

        \color{jFrametitleFGColor}

        The content of this lecture is made available under a Creative
        Commons licence (\textsc{cc}), giving the author the credits
        (\textsc{by}) and putting an obligation to share on the same terms
        (\textsc{sa}). Figures and diagrams included in the lecture are
        authored by Paweł Węgrzyn et~al., and are available under the same
        license unless indicated otherwise. \\ The presentation uses the
        Beamer Jagiellonian theme based on Matthias Vogelgesang’s
        Metropolis theme, available under license \LaTeX{} Project
        Public License~1.3c at: \colorhref{https://github.com/matze/mtheme}
        {https://github.com/matze/mtheme}.

        Typographic design: Iwona Grabska-Gradzińska;
        \LaTeX{} Typesetting: Kamil Ziemian \\
        Proofreading: Leszek Hadasz, Wojciech Palacz,
        Monika Stawicka \\
        3D Models: Dariusz Frymus, Kamil Nowakowski \\
        Figures and charts: Kamil Ziemian, Paweł Węgrzyn,
        Wojciech Palacz \\
        Film editing: Agencja Filmowa Film \& Television Production~--
        Zbigniew Masklak \\
        Film editing: IMAVI -- Joanna Kozakiewicz, Krzysztof Magda, Nikodem
        Frodyma


      \end{flushleft}

    \end{textblock}

  \end{frame}





  \begin{frame}[standout]


    \begingroup

    \color{jFrametitleFGColor}

    #1

    \endgroup

  \end{frame}
}











% ------------------------------------------------------
% Importing packages, libraries and setting their configuration
% ------------------------------------------------------










% ---------------------------------------------------------------------
\title{Wybrane algorytmy kompilacji}
\subtitle{Tematy końcowe}

\author{Kamil Ziemian \\
  \email}


% \date{}
% ---------------------------------------------------------------------










% ####################################################################
% Beginning of the document
\begin{document}
% ####################################################################





% ######################################
% Text is adjusted to the left and words are broken at the end of the line.
% Number of chars: 62k+.
\RaggedRight
% ######################################





% ######################################
\maketitle
% ######################################





% % ##################
% \begin{frame}
%   \frametitle{Table of contents}


%   \tableofcontents

% \end{frame}
% % ##################










% % ######################################
% \section{}
% % ######################################







% ######################################
\section{Informacje dodatkowe}
% ######################################









% ##################
\begin{frame}
  \frametitle{?????}




\end{frame}
% ##################





% ##################
\begin{frame}
  \frametitle{Krótka i~niezbyt poprawna historia problemu}


  Na~przestrzeni tysięcy lat ludzie tworzyli urządzenia, które dziś
  rozpoznajemy jako prekursorów współczesnych komputerów, wypada wymienić
  kilku z~nich, którzy działali przed rokiem 1900-nym.

  \textbf{2003 r.} Chris Lattner i~Vikram Adve tworzą system \textsc{llvm}.
  Możemy przyjąć, że~w~tym momencie zaczyna~się okres rozwoju technik
  kompilacji w~którym jesteśmy dziś (2024 rok).

\end{frame}
% ##################






















% ######################################
% \section{Dodatkowe informacje}
% ######################################



% ##################
\begin{frame}
  \frametitle{Historia języków asemblera}


  W~$1978$ roku firma Intel (od ang.~\textit{INTegrated ELectronic})
  wypuszcza procesor $8086$, który rozpoczyna rodzinę procesorów
  $\text{x}86$. Nazwa „$\text{x}86$” wywodzie~się od tego, że~liczby
  będące nazwami procesorów miały~się kończyć na „$86$”, więc symbol
  „$\text{x}$” oznacza pozostałą część tej liczby. W~następny latach
  pojawiły~się procesory $80186$, $80286$ i~$80386$. Z~powodu praw
  autorskich i~spraw pokrewnych, procesor który powinien~się nazywać
  $80586$ został przemianowany na Pentium. I~tak dalej. Język asembler
  stworzony dla tych procesorów nosi nazwę \alert{x86}, używa~się również
  bardziej precyzyjnego określenia \alert{x86/Intel}.

  Jeśli~się Państwo zastanawiają „Skąd oni biorą te nazwy dla procesorów?”,
  to powiem, że~też chciałbym wiedzieć.

  W~$2003$ roku firma \textsc{amd} (od ang.~\textit{Advanced Micro Devices})
  wypuszcza 64-bitowy procesor Athlon~64 z~językiem asemblera o~nazwie
  \textsc{amd}64, wzorowanym na języku x86/Intel.

\end{frame}
% ##################





% ##################
\begin{frame}
  \frametitle{???}


  Zbiorczą nazwą stosowną dla 64-bitowych asemblerów \textsc{amd} i~Intela
  jest \alert{x86-64} lub \alert{x64}. Niektórzy używają również nazwy
  \alert{\textsc{amd}64}, a~także paru innych.

  W~1983 roku firma Acorn Computers Ltd. zaczyna pracę nad procesorem który
  zostanie nazwany \alert{\textsc{arm}} (od ang.~\textit{Acorn \textsc{risc}
    Machine}). Jednym z~wyników tego projektu jest opracowanie zbioru
  instrukcji architektury \textsc{risc} (ang.~\textit{Reduced Instruction
    Set Computing}). Pierwszym wyprodukowanym przed nich procesorem jest
  32-bitowy \textsc{arm1}. Dla tych procesorów utworzono język asemblera
  \alert{\textsc{arm}}.

  W~1990 roku z~Acorn Computers Ltd. wydzieliła~się firma znana dziś jako
  Arm~Holdings~pcl.

\end{frame}
% ##################





% ##################
\begin{frame}
  \frametitle{?????}


  Dwa najważniejsze dzisiaj zbiory instrukcji architektury to
  \alert{\textsc{cisc}} i~\alert{\textsc{risc}}.

  Akronim \textsc{cisc} pochodzi od angielskiego \textit{Complex
    Instruction Set Computing}, czyli „obliczenia za pomocą zbioru
  złożonych instrukcji”. Architektura ta charakteryzuje~się dużą ilością
  instrukcji wykonujących złożone (stąd nazwa) operacje, które wykonanie
  może zajmować wiele cykli procesora.

  Natomiast nazwa \textsc{risc} jest od \textit{Reduced Instruction Set
    Computing}, czyli „obliczenia za pomocą zbioru zredukowanych
  instrukcji”. W~architekturze tej dominują małe instrukcje, które są
  wykonywane w~jednym lub kilku cyklach procesora. Nazwa wzięła~się
  stąd, że~instrukcje w~niej stosowane, miały~się być zredukowaną wersją
  instrukcji architektury \textsc{cisc}.

\end{frame}
% ##################





% ##################
\begin{frame}
  \frametitle{????}


  W~praktyce trudno wskazać procesor który byłby czystą realizacją
  \textsc{cisc} albo \textsc{risc}. Jednak w~dobrym przybliżeniu procesor
  x86/Intel to \textsc{cisc}, zaś \textsc{arm} to \textsc{risc}. Od~2021
  roku dużo szumu jest wokół architektury \textsc{risc-v}, ale to temat na
  inne zajęcia.

\end{frame}
% ##################




% ##################
\begin{frame}
  \frametitle{?????}




\end{frame}
% ##################





% % ##################
% \begin{frame}
%   \frametitle{????}




% \end{frame}
% % ##################





% % ##################
% \begin{frame}
%   \frametitle{?????}




% \end{frame}
% % ##################





% % ##################
% \begin{frame}
%   \frametitle{????}




% \end{frame}
% % ##################





% % ##################
% \begin{frame}
%   \frametitle{??????}




% \end{frame}
% % ##################










% % ######################################
% \section{????}
% % ######################################



% % ##################
% \begin{frame}
%   \frametitle{???}



% \end{frame}
% % ##################





% % ##################
% \begin{frame}
%   \frametitle{?????}



% \end{frame}
% % ##################





% % ##################
% \begin{frame}
%   \frametitle{??????}



% \end{frame}
% % ##################










% % ######################################
% \SectionSlideWithPicture{????}
% % ######################################



% % ##################
% \begin{frame}
%   \frametitle{?????}




% \end{frame}
% % ##################





% % ##################
% \begin{frame}
%   \frametitle{??????}



% \end{frame}
% % ##################





% % ##################
% \begin{frame}
%   \frametitle{Rendering without perspective}



% \end{frame}
% % ##################





% % ##################
% \begin{frame}
%   \frametitle{Rendering with perspective}



% \end{frame}
% % ##################





% % ##################
% \begin{frame}
%   \frametitle{Another sample scene}



% \end{frame}
% % ##################





% % ##################
% \begin{frame}
%   \frametitle{Rendering without perspective}




% \end{frame}
% % ##################





% % ##################
% \begin{frame}
%   \frametitle{Rendering with perspective}




% \end{frame}
% % ##################





% % ##################
% \begin{frame}
%   \frametitle{Basic concepts of computer graphics}






% \end{frame}
% % ##################





% % ##################
% \begin{frame}
%   \frametitle{Data structures}




% \end{frame}
% % ##################





% % ##################
% \begin{frame}
%   \frametitle{From vertices to fragments}




% \end{frame}
% % ##################






% % ##################
% \begin{frame}
%   \frametitle{Basic concepts of computer graphics}




% \end{frame}
% % ##################





% % ##################
% \begin{frame}
%   \frametitle{Each model is based on  a set of vertices}




% \end{frame}
% % ##################





% % ##################
% \begin{frame}
%   \frametitle{Each model is based on  a set of vertices}




% \end{frame}
% % ##################





% % ##################
% \begin{frame}
%   \frametitle{Vertex attributes}




% \end{frame}
% % ##################





% % ##################
% \begin{frame}
%   \frametitle{Vertex attributes}



% \end{frame}
% % ##################






% % ##################
% \begin{frame}
%   \frametitle{Basic concepts of computer graphics}




% \end{frame}
% % ##################





% % ##################
% \begin{frame}
%   \frametitle{Primitives in the OpenGL library}



% \end{frame}
% % ##################





% % ##################
% \begin{frame}
%   \frametitle{Primitives in the OpenGL library}




% \end{frame}
% % ##################





% % ##################
% \begin{frame}
%   \frametitle{Primitives in the OpenGL library}




% \end{frame}
% % ##################





% % ##################
% \begin{frame}
%   \frametitle{Primitives in the OpenGL library}



% \end{frame}
% % ##################





% % ##################
% \begin{frame}
%   \frametitle{Primitives in the OpenGL library (deprecated since ver.~3.1)}



% \end{frame}
% % ##################





% % ##################
% \begin{frame}
%   \frametitle{Texturing and blending}



% \end{frame}
% % ##################





% % ##################
% \begin{frame}
%   \frametitle{Texturing and blending}



% \end{frame}
% % ##################





% % ##################
% \begin{frame}
%   \frametitle{Communication between CPU and GPU}


%   \begin{figure}

%     \centering

%     \begin{tikzpicture}[node distance=1.7em, small block loc/.style =
%       {rectangle, rounded corners, fill=jAxisBlue, text width=4em,
%         text centered, minimum height=4em,
%         text=jMathTextFGStyleDark}, block loc/.style = {rectangle,
%         fill=jAxisBlue, text centered, text width=6em, minimum
%         height=4em, text=jMathTextFGStyleDark},
%       block circle loc/.style = {circle, minimum size=3.3em, text
%         centered, text=jMathTextFGStyleDark}]


%       \node[block circle loc,fill=jOrange,scale=1.3] at (0,0) (CPU)
%       {CPU};

%       \node[small block loc,right=of CPU] (Aplikacja) {Application};

%       \draw[thick connection arrow] (CPU) -- (Aplikacja);



%       \node[small block loc,right=of Aplikacja] (OpenGL DirectX) {OpenGL
%         DirectX};

%       \draw[thick connection arrow] (Aplikacja) -- (OpenGL DirectX);



%       \node[small block loc,align=left,right=of OpenGL DirectX]
%       (Sterownik)
%       {{\small Video card \\
%           driver}};

%       \draw[thick connection arrow] (OpenGL DirectX) -- (Sterownik);



%       \node[block circle loc,fill=jOrange,scale=1.3,right=of
%       Sterownik] (GPU) {GPU};

%       \draw[thick connection arrow] (Sterownik) -- (GPU);


%       \node[block loc,below=6em of CPU] (Pamiec Operacyjna) {Main Memory};

%       \draw[{Triangle[scale width=0.6,scale
%         length=0.3]}-{Triangle[scale width=0.6,scale length=0.3]},
%       line width=7.5,color=jNormalTextFGStyleLight] (CPU) -- (Pamiec
%       Operacyjna);



%       \node[block loc,below=6em of GPU] (Pamiec Wideo) {Video Memory};

%       \draw[{Triangle[scale width=0.6,scale
%         length=0.3]}-{Triangle[scale width=0.6,scale length=0.3]},
%       line width=7.5,color=jNormalTextFGStyleLight] (GPU) -- (Pamiec
%       Wideo);


%       \draw[thick connection arrow] (Pamiec Operacyjna) -- (Pamiec
%       Wideo);

%       \node[align=left] at (4.3,-2.7) {vertex data \\
%         texture data \\
%         shader parameters};

%     \end{tikzpicture}

%   \end{figure}

% \end{frame}
% % ##################





% % ##################
% \begin{frame}
%   \frametitle{Data stored in the video memory}


%   \begin{itemize}
%     \RaggedRight

%     \setlength{\itemsep}{0.3em}

%   \item Pixel data of the image visible in the viewport --\\
%     front image buffer

%   \item Pixel data of the image invisible in the viewport --\\
%     back image buffer

%   \item Data that describes how deep a pixel lies in the image --\\ depth
%     buffer (also called z-buffer)

%   \item Data  used  to enable or disable drawing
%     on a per-pixel basis -- stencil buffer

%   \item Vertex data -- vertex buffers

%   \item Texture data -- texture maps

%   \end{itemize}

% \end{frame}
% % ##################





% % ##################
% \begin{frame}
%   \frametitle{Data stored in the video memory}


%   In the video memory we can store any data, just like in the
%   \textsc{ram}. Compared to the usual \textsc{ram}, the access to this data is
%   not through\\ pointers, but through buffers, textures and other objects.

%   Examples of data objects \vspace{-0.5em}
%   \begin{itemize}
%     \RaggedRight

%     \setlength{\itemsep}{0.3em}

%   \item Vertex Buffer
%     Objects

%   \item Pixel Buffer Objects

%   \item Shader Storage
%     Buffer Objects

%   \item Uniform Buffer
%     Objects

%   \item Sampler Objects

%   \end{itemize}

% \end{frame}
% % ##################





% % ##################
% \begin{frame}
%   \frametitle{Graphics pipeline (rendering pipeline)}


%   \begin{figure}

%     \begin{tikzpicture}

%       \node[diagrams block 1] (pobranie wierzcholkow)
%       {Vertex \\
%         Specification};

%       \node[diagrams block 1,below=of pobranie wierzcholkow] (szader
%       wierzcholkow) {Vertex Shader};

%       \draw[thick connection line] (pobranie wierzcholkow) -- (szader
%       wierzcholkow);



%       \node[diagrams block 1,below=of szader wierzcholkow] (szader
%       sterowania teselacja) {Tessellation Control Shader};

%       \draw[thick connection line] (szader wierzcholkow) -- (szader
%       sterowania teselacja);



%       \node[diagrams block 1,right=of szader sterowania teselacja]
%       (teselacja) {Tessellation};

%       \draw[thick connection line] (szader sterowania teselacja) --
%       (teselacja);



%       \node[diagrams block 1,above=of teselacja] (szader wyliczania
%       teselacji) {Tessellation Evaluation Shader};

%       \draw[thick connection line] (teselacja) -- (szader wyliczania
%       teselacji);



%       \node[diagrams block 1,above=of szader wyliczania teselacji]
%       (szader geometrii) {Geometry Shader};

%       \draw[thick connection line] (szader wyliczania teselacji) --
%       (szader geometrii);



%       \node[diagrams block 1,right=of szader geometrii] (rasteryzacja)
%       {Rasterization};

%       \draw[thick connection line] (szader geometrii) -- (rasteryzacja);



%       \node[diagrams block 1,below=of rasteryzacja] (szader fragmentow)
%       {Fragment Shader};

%       \draw[thick connection line] (rasteryzacja) -- (szader fragmentow);



%       \node[diagrams block 1,below=of szader fragmentow] (bufor ramki)
%       {Frame buffer \\ Operations};

%       \draw[thick connection line] (szader fragmentow) -- (bufor ramki);

%     \end{tikzpicture}

%   \end{figure}

% \end{frame}
% % ##################





% % ##################
% \begin{frame}
%   \frametitle{Basic concepts of computer graphics}


%   The graphics pipeline can be divided into two parts.

%   \begin{itemize}
%     \setlength{\itemsep}{0.75em}

%   \item \textbf{Front end} --~processing vertices and primitives, creating points, lines and triangles (assembling primitives), up to passing data\\ to rasterization.

%   \item \textbf{Back end} --~pixel processing, depth checking, fragment shading, color blending,  updating the final image.


%   \end{itemize}

% \end{frame}
% % ##################





% % ##################
% \begin{frame}
%   \frametitle{Basic concepts of computer graphics}


%   \textbf{Vertex shader}~--
%   a program that runs once for each downloaded\\ vertex or  control point of a patch (if you use tessellation).

%   \vspace{0.5em}


%   \textbf{Tessellation control shader}~--   a program that runs once for each  patch. A \textbf{patch} is a high-order primitive.\\ It is accessible only
%   if we use tessellation.


%   \vspace{0.5em}


%   \textbf{Tessellation} --~a fixed function transforming patches into smaller and simpler primitives (e.g. triangles).

%   \vspace{0.5em}


%   \textbf{Tessellation evaluation shader} --   a program that runs once for each vertex created by the tessellation mechanism.


% \end{frame}
% % ##################





% % ##################
% \begin{frame}
%   \frametitle{Basic concepts of computer graphics}


%   \textbf{Geometry shader} --~a program that runs once for each primitive. \\It has access to all vertices of the primitive. It can transform one type of primitives
%   into another, e.g. points can be transformed to triangles.


%   \vspace{0.5em}


%   \textbf{Compute shader} --~a program
%   that runs outside the graphics pipeline. It has no specific
%   role and can deal with any calculation;  it can have any
%   input/output. Used in graphics for postprocessing.


%   \vspace{0.5em}


%   \textbf{Clipping} --~after  creating primitives from the vertices, they are clipped in order to be contained in the visible area.

%   \vspace{0.5em}


%   \textbf{Viewport} --~a window area,
%   in which OpenGL renders its image.


% \end{frame}
% % ##################





% % ##################
% \begin{frame}
%   \frametitle{Vertex transformations (OpenGL nomenclature)}


%   \begin{textblock}{10.8}(1,1.2)

%     \begin{tikzpicture}[node distance=2.6em,scale=0.7, block loc/.style
%       = {rectangle, rounded corners, fill=jAxisBlue, text width=6em,
%         text centered, minimum height=4.1em,
%         text=jMathTextFGStyleDark}]


%       \node[block loc] (przestrzen obiektu)
%       {Object \\[-0.2em]
%         space};

%       \node[block loc,right=of przestrzen obiektu] (przestrzen swiata)
%       {World \\[-0.2em]
%         space};

%       \draw[thick connection arrow] (przestrzen obiektu) -- (przestrzen
%       swiata);


%       \node[block loc,right=of przestrzen swiata] (przestrzen kamery)
%       {Camera \\[-0.2em]
%         space};

%       \draw[thick connection arrow] (przestrzen swiata) -- (przestrzen
%       kamery);


%       \node[block loc,below=2.6em of przestrzen kamery] (jednorodna
%       przestrzen)
%       {Homogeneous \\[-0.2em]
%         clip space};

%       \draw[thick connection arrow] (przestrzen kamery) -- (jednorodna
%       przestrzen);


%       \node[block loc,below=2.6em of jednorodna przestrzen] (przestrzen
%       widoku) {Viewport space};

%       \draw[thick connection arrow] (jednorodna przestrzen) --
%       (przestrzen widoku);



%       \draw[thick connection arrow,rounded corners] (przestrzen obiektu)
%       -- ++(0,2) -| (przestrzen kamery);

%       \node at (4.8,2.65) {\alert{model-view transformation}};


%       \node at (11.77,-1.6) (rzutowanie) {\alert{projection}};


%       \node[rectangle,align=left] at (12,-5.05) (transformacja okna)
%       {\alert{viewport} \\[-0.2em]
%         \alert{transformation}};

%     \end{tikzpicture}

%   \end{textblock}

% \end{frame}
% % ##################





% % ##################
% \begin{frame}[label=Przestrzen-uklad-wspolrzednych-1]
%   \frametitle{Spaces --~coordinate systems}


%   \begin{textblock}{2.4}(10,1.4)

%     \hyperlink{Uwagi-pojecie-przestrzeni-Przestrzen-afiniczna-i-wektorowa-1}
%     {\beamergotobutton{Terminological notes}}

%   \end{textblock}


%   \vspace{1.8em}


%   \textbf{Object space}, or \textbf{model space}, is associated
%   with the local coordinate system native to the specific model.
%   In this space, we usually specify the positions of the vertices of the model.


%   \vspace{0.5em}


%   \textbf{World space} is 	associated with the global coordinate system\\ native to the scene. In this space, we usually specify the positions\\ and orientations of all models.


%   \vspace{0.5em}


%   \textbf{Camera space}, or \textbf{eye space}, is associated with the
%   coordinate\\ system  defined by the camera, the eye or  the screen.\\ The axes
%   $x$ and ~$y$ lie in the plane of the screen,\\ and the axis ~$z$ is perpendicular
%   to the screen.


% \end{frame}
% % ##################





% % ##################
% \begin{frame}
%   \frametitle{Spaces --~coordinate systems}


%   \textbf{Homogeneous clip space} is associated with the coordinate\\ system that has
%   highlighted visibility area.\\ It is a cube with vertices
%   in $( -1 \hspace{0.5em} {-1} \hspace{0.5em} {-1} )^{ T }$,
%   $( 1 \hspace{0.5em} {-1} \hspace{0.5em} {-1} )^{ T }$, \ldots,
%   $( 1 \hspace{0.5em} 1 \hspace{0.5em} 1 )^{ T }$.\\ Graphic primitives
%   which do not fit entirely inside the cube\\ are clipped.


%   \vspace{0.5em}


%   \textbf{Window space}, or \textbf{viewport space}, is
%   associated with the coordinate system defined by the view window.\\
%   It contains the information about  pixel coordinates and their depth.


% \end{frame}
% % ##################





% % ##################
% \begin{frame}
%   \frametitle{Rasterization and fragment operations}


%   \textbf{Rasterization} --~taking a graphic primitive  and converting it\\ into a raster image.\\ Graphic primitives are obtained after the transformation of
%   vertices of the model to the window space.\\ A raster image is a series of pixels.


%   \vspace{0.5em}


%   \textbf{Fragment} --~the information on
%   the localization of each pixel\\ of any primitive with its depth,
%   interpolated vertex colors \\
%   and interpolated texture coordinates.


% \end{frame}
% % ##################





% % ##################
% \begin{frame}
%   \frametitle{Rasterization and fragment operations}


%   \begin{textblock}{7.4}(2.7,1.6)

%     \begin{tikzpicture}[node distance=4em]
%       \draw[thick connection arrow] (-0.8,-0.5) -- (-1.35,-2);

%       \draw[thick connection arrow] (0.825,-0.5) -- (1.4,-2);


%       \node[diagrams block 1] (rasteryzacja) {Rasterization};


%       \node[diagrams block 1] (selekcja) [above=of rasteryzacja] {Selection\\
%         visible\\ primitives};


%       \draw[thick connection arrow] (selekcja) -- (rasteryzacja);


%       \node[diagrams block 1] at (-2.2,-2.8) (cieniowanie) {Fragment shading};


%       \node[diagrams block 1] at (2.25,-2.8) (operacje) {Fragment operations};
%     \end{tikzpicture}

%   \end{textblock}

% \end{frame}
% % ##################





% % ##################
% \begin{frame}
%   \frametitle{Fragment operations}


%   \textbf{Pixel ownership test} --~an application that determines
%   if a fragment lies in the region of the currently visible viewport.

%   \vspace{0.5em}


%   \textbf{Scissor test} --~an application that may specify a rectangle\\ in
%   the viewport, to which rendering should be restricted.

%   \vspace{0.5em}


%   \textbf{Alpha test} --~selective display of pixels based on
%   transparency \\conditions.


%   \vspace{0.5em}


%   \textbf{Stencil test} --~selective display of pixels based on the data specified \\in the stencil buffer.

%   \vspace{0.5em}


%   \textbf{Depth test} --~selective display of pixels based on the data specified\\ in the depth buffer.

%   \vspace{0.5em}


%   \textbf{Blending} --~determining the final color of a pixel by
%   mixing  all color components.



% \end{frame}
% % ##################





% % ##################
% \begin{frame}
%   \frametitle{Basic concepts of computer graphics}


%   \textbf{Fragment shader} --~a program that runs once for each fragment\\ of the rendered primitive.

%   \vspace{0.5em}


%   \textbf{Double buffering} --~a drawing technique of using
%   two memory buffers for image.


%   \vspace{0.5em}


%   \textbf{Aliasing} --~a loss of information related to the finite
%   resolution\\ of image reproduction.


%   \vspace{0.5em}


%   \textbf{Anti-aliasing} --~a rendering technique with smoothing
%   edges \\(we distinguish, among others: \textsc{fsaa}, \textsc{msaa},
%   \textsc{mfaa}, \textsc{csaa}, \textsc{cfaa}, \textsc{mlaa},
%   \textsc{smaa}, \textsc{hraa}, \textsc{fxaa}, \textsc{txaa},
%   \textsc{taa}, \textsc{ssaa}).


%   \vspace{0.5em}


%   \textbf{View frustum} --~a visible
%   area of space.


% \end{frame}
% % ##################





% % ##################
% \begin{frame}
%   \frametitle{Links}


%   [1] \colorhref{www.blender.org}{www.blender.org}

% \end{frame}
% % ##################










% % ######################################
% \appendix
% % ######################################





% % ##################
% \GeometryThreeDTwoSpecialEndingSlidesEN{Questions? Thank you for your attention.}
% % ##################



% % % ##################
% % \jagiellonianendslide{Dziękuję za~uwagę.}
% % % ##################










% % ######################################
% \SectionSlideWithPicture{Terminological notes}
% % ######################################



% % ##################
% \begin{frame}[label=Uwagi-pojecie-przestrzeni-Przestrzen-afiniczna-i-wektorowa-1]
%   \frametitle{The concept of space. Affine and vector space}


%   In geometry, we use two kinds of space.\\
%   \textbf{Vector space} is the corresponding set of vectors.\\
%   \textbf{Affine space} is a set of points that can be
%   moved by vectors from some vector space.

%   An example of affine space can be a two-dimensional plane.\\
%   It has no center, any two points of it can be
%   connected by a vector,\\ and at any point one can hook a system of
%   coordinates.

%   An example of a vector space can be the set of all possible
%   velocities of a point moving in the plane. Such
%   space has a distinguished\\ element, the zero vector (zero
%   velocity of the point, the resting point), which can be considered its ``center''.





%   \begin{textblock}{2.1}(1,8.7)

%     \hyperlink{Przestrzen-uklad-wspolrzednych-1}
%     {\beamerreturnbutton{Back to lecture}}

%   \end{textblock}


%   \begin{textblock}{2.1}(10,8.7)

%     \hyperlink{Uwagi-pojecie-przestrzeni-Przestrzen-afiniczna-i-wektorowa-2}
%     {\beamergotobutton{Continued \hspace{3.5em}}}

%   \end{textblock}

% \end{frame}
% % ##################





% % ##################
% \begin{frame}[label=Uwagi-pojecie-przestrzeni-Przestrzen-afiniczna-i-wektorowa-2]
%   \frametitle{The concept of space. Space and coordinate system}



% \end{frame}%
% ##################





% ##################
% \begin{frame}[label=Uwagi-pojecie-przestrzeni-Przestrzen-afiniczna-i-wektorowa-3]
%   \frametitle{The concept of space. Space and coordinate system}




% \end{frame}
% ##################





% ##################
% \begin{frame}[label=Uwagi-uklady-wspolrzednych-w-przestrzeni-Rodzaje-ukladow-1]
%   \frametitle{Coordinate systems in computer graphics}



% \end{frame}
% ##################





% ##################
% \begin{frame}[label=Uwagi-uklady-wspolrzednych-w-przestrzeni-Rodzaje-ukladow-2]
%   \frametitle{Coordinate systems in computer graphics}




% \end{frame}
% ##################





% ##################
% \begin{frame}[label=Uwagi-uklady-wspolrzednych-w-przestrzeni-Rodzaje-ukladow-3]
%   \frametitle{Coordinate systems in computer graphics}



% \end{frame}
% ##################










% ####################################################################

% End of the document
\end{document}
