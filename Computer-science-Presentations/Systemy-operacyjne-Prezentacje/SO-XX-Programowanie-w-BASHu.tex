% ------------------------------------------------------------------------------------------------------------------
% Basic configuration of Beamera class and Jagiellonian theme
% ------------------------------------------------------------------------------------------------------------------
\RequirePackage[l2tabu, orthodox]{nag}



\ifx\PresentationStyle\notset
  \def\PresentationStyle{light}
\fi



% Options: t - align frame text to the top.
\documentclass[10pt,t]{beamer}
\mode<presentation>
\usetheme[style=\PresentationStyle,JUlogotitle=no]{jagiellonian}




% ------------------------------------------------------------------------------------
% Procesing configuration files of Jagiellonian theme located
% in the directory "preambule"
% ------------------------------------------------------------------------------------
% Configuration for polish language
% Need description
\usepackage[polish]{babel}
% Need description
\usepackage[MeX]{polski}



% ------------------------------
% Better support of polish chars in technical parts of PDF
% ------------------------------
\hypersetup{pdfencoding=auto,psdextra}

% Package "textpos" give as enviroment "textblock" which is very usefull in
% arranging text on slides.

% This is standard configuration of "textpos"
\usepackage[overlay,absolute]{textpos}

% If you need to see bounds of "textblock's" comment line above and uncomment
% one below.

% Caution! When showboxes option is on significant ammunt of space is add
% to the top of textblock and as such, everyting put in them gone down.
% We need to check how to remove this bug.

% \usepackage[showboxes,overlay,absolute]{textpos}



% Setting scale length for package "textpos"
\setlength{\TPHorizModule}{10mm}
\setlength{\TPVertModule}{\TPHorizModule}


% ---------------------------------------
% Packages written for lectures "Geometria 3D dla twórców gier wideo"
% ---------------------------------------
% \usepackage{./Geometry3DPackages/Geometry3D}
% \usepackage{./Geometry3DPackages/Geometry3DIndices}
% \usepackage{./Geometry3DPackages/Geometry3DTikZStyle}
% \usepackage{./ProgramowanieSymulacjiFizykiPaczki/ProgramowanieSymulacjiFizykiTikZStyle}
% \usepackage{./Geometry3DPackages/mathcommands}


% ---------------------------------------
% TikZ
% ---------------------------------------
% Importing TikZ libraries
\usetikzlibrary{arrows.meta}
\usetikzlibrary{positioning}





% % Configuration package "bm" that need for making bold symbols
% \newcommand{\bmmax}{0}
% \newcommand{\hmmax}{0}
% \usepackage{bm}




% ---------------------------------------
% Packages for scientific texts
% ---------------------------------------
% \let\lll\undefined  % Sometimes you must use this line to allow
% "amsmath" package to works with packages with packages for polish
% languge imported
% /preambul/LanguageSettings/JagiellonianPolishLanguageSettings.tex.
% This comments (probably) removes polish letter Ł.
\usepackage{amsmath}  % Packages from American Mathematical Society (AMS)
\usepackage{amssymb}
\usepackage{amscd}
\usepackage{amsthm}
\usepackage{siunitx}  % Package for typsetting SI units.
\usepackage{upgreek}  % Better looking greek letters.
% Example of using upgreek: pi = \uppi


\usepackage{calrsfs}  % Zmienia czcionkę kaligraficzną w \mathcal
% na ładniejszą. Może w innych miejscach robi to samo, ale o tym nic
% nie wiem.










% ---------------------------------------
% Packages written for lectures "Geometria 3D dla twórców gier wideo"
% ---------------------------------------
% \usepackage{./ProgramowanieSymulacjiFizykiPaczki/ProgramowanieSymulacjiFizyki}
% \usepackage{./ProgramowanieSymulacjiFizykiPaczki/ProgramowanieSymulacjiFizykiIndeksy}
% \usepackage{./ProgramowanieSymulacjiFizykiPaczki/ProgramowanieSymulacjiFizykiTikZStyle}





% !!!!!!!!!!!!!!!!!!!!!!!!!!!!!!
% !!!!!!!!!!!!!!!!!!!!!!!!!!!!!!
% EVIL STUFF
\if\JUlogotitle1
\edef\LogoJUPath{LogoJU_\JUlogoLang/LogoJU_\JUlogoShape_\JUlogoColor.pdf}
\titlegraphic{\hfill\includegraphics[scale=0.22]
{./JagiellonianPictures/\LogoJUPath}}
\fi
% ---------------------------------------
% Commands for handling colors
% ---------------------------------------


% Command for setting normal text color for some text in math modestyle
% Text color depend on used style of Jagiellonian

% Beamer version of command
\newcommand{\TextWithNormalTextColor}[1]{%
  {\color{jNormalTextFGColor}
    \setbeamercolor{math text}{fg=jNormalTextFGColor} {#1}}
}

% Article and similar classes version of command
% \newcommand{\TextWithNormalTextColor}[1]{%
%   {\color{jNormalTextsFGColor} {#1}}
% }



% Beamer version of command
\newcommand{\NormalTextInMathMode}[1]{%
  {\color{jNormalTextFGColor}
    \setbeamercolor{math text}{fg=jNormalTextFGColor} \text{#1}}
}


% Article and similar classes version of command
% \newcommand{\NormalTextInMathMode}[1]{%
%   {\color{jNormalTextsFGColor} \text{#1}}
% }




% Command that sets color of some mathematical text to the same color
% that has normal text in header (?)

% Beamer version of the command
\newcommand{\MathTextFrametitleFGColor}[1]{%
  {\color{jFrametitleFGColor}
    \setbeamercolor{math text}{fg=jFrametitleFGColor} #1}
}

% Article and similar classes version of the command
% \newcommand{\MathTextWhiteColor}[1]{{\color{jFrametitleFGColor} #1}}





% Command for setting color of alert text for some text in math modestyle

% Beamer version of the command
\newcommand{\MathTextAlertColor}[1]{%
  {\color{jOrange} \setbeamercolor{math text}{fg=jOrange} #1}
}

% Article and similar classes version of the command
% \newcommand{\MathTextAlertColor}[1]{{\color{jOrange} #1}}





% Command that allow you to sets chosen color as the color of some text into
% math mode. Due to some nuances in the way that Beamer handle colors
% it not work in all cases. We hope that in the future we will improve it.

% Beamer version of the command
\newcommand{\SetMathTextsColor}[2]{%
  {\color{#1} \setbeamercolor{math text}{fg=#1} #2}
}


% Article and similar classes version of the command
% \newcommand{\SetMathTextColor}[2]{{\color{#1} #2}}










% ---------------------------------------
% Commands for setting background pictures for some slides
% ---------------------------------------
\newcommand{\TitleBackgroundPicture}
{./PresentationPictures/CommonPictures/Cute_dragon_BG_dark.png}
\newcommand{\SectionBackgroundPicture}
{./PresentationPictures/CommonPictures/Cute_dragon_small_BG_light.png}



\newcommand{\TitleSlideWithPicture}{
  \begingroup

  \usebackgroundtemplate{ % \hspace*{-11.5em}
    \includegraphics[height=\paperheight]{\TitleBackgroundPicture}}

  \maketitle

  \endgroup
}





\newcommand{\SectionSlideWithPicture}[1]{%
  \begingroup

  \usebackgroundtemplate{ % \hspace*{-11.5em}
    \includegraphics[height=\paperheight]{\SectionBackgroundPicture}}

  \setbeamercolor{titlelike}{fg=normal text.fg}

  \section{#1}

  \endgroup
}





\newcommand{\EndingSlide}[1]{%
  \begin{frame}[standout]

    \begingroup

    \color{jFrametitleFGColor}

    #1

    \endgroup

  \end{frame}
}










% ------------------------------------------------------
% BibLaTeX
% ------------------------------------------------------
% Package biblatex, with biber as its backend, allow us to handle
% bibliography entries that use Unicode symbols outside ASCII.
\usepackage[
language=polish,
backend=biber,
style=alphabetic,
url=false,
eprint=true,
]{biblatex}

\addbibresource{Systemy-operacyjne-Bibliography.bib}





% ------------------------------------------------------
% Importing packages, libraries and setting their configuration
% ------------------------------------------------------





% ------------------------------------------------------
% Local packages
% ------------------------------------------------------
% Local configuration of this particular presentation
\usepackage{./Local-packages/local-settings}










% ------------------------------------------------------------------------------------------------------------------
\title{Systemy operacyjne}
\subtitle{?. Programowanie w~BASHu}

\author{Kamil Ziemian \\
  \email}


% \date{}
% ------------------------------------------------------------------------------------------------------------------










% ####################################################################
% Beginning of the document
\begin{document}
% ####################################################################





% ######################################
% Number of chars: 20k+,
% Text is adjusted to the left and words are broken at the end of the line.
\RaggedRight
% ######################################





% ######################################
\maketitle
% ######################################





% ######################################
\EndingSlide{Ta prezentacja wciąż nie jest ukończona.}
% ######################################





% ##################
\begin{frame}
  \frametitle{Spis treści}


  \tableofcontents

\end{frame}
% ##################










% ######################################
\section{Programowanie w~powłoce BASHa}
% ######################################


% ##################
\begin{frame}
  \frametitle{Wyszukiwanie całych wyrazów}




\end{frame}
% ##################










% ######################################
\section{Wprowadzenie do programowania w~BASHu}
% ######################################


% ##################
\begin{frame}
  \frametitle{???}




\end{frame}
% ##################





% ##################
\begin{frame}
  \frametitle{Materiały do~nauki}


  Ryan Chadwick
  \parencite{Chadwick-BASH-Scripting-Tutorial-Introduction-ETC-Ver-2025}

\end{frame}
% ##################





% % ##################
% \begin{frame}
%   \frametitle{Mem dla ilustracji problemu}

%   \vspace{-0.5em}


%   \begin{figure}

%     \label{fig:How-to-regex}

%     \centering


%     \includegraphics[scale=0.18]
%     {./Presentations-pictures/Miscancellous-pictures/How-to-regex.jpg}

%   \end{figure}

% \end{frame}
% % ##################





% % ##################
% \begin{frame}
%   \frametitle{Co ja robię?}

%   \vspace{-0.5em}


%   \begin{figure}

%     \label{fig:How-to-regex}

%     \centering


%     \includegraphics[scale=0.52]
%     {./Presentations-pictures/Miscancellous-pictures/Lost-in-regex.png}

%   \end{figure}

% \end{frame}
% % ##################





% % ##################
% \begin{frame}
%   \frametitle{Praca z~\texttt{grep}em}


%   \textit{I~have a natural revulsion to any operating system that shows so
%     little plannning as to have to named all of its commands after digestive
%     noises (awk, grep, fsck, nroff).} \\
%     Autor nieznany, str.~$147$,
%     \parencite{Garfinkel-Weise-Strassmann-The-UNIX-HATERS-Handbook-Pub-1994}.

%   Nazwa polecenia \texttt{grep} pochodzi od~\textit{Global REgular
%     expression Print}, czyli na polski to byłoby coś w~stylu „globalny
%   wypisywacz wyrażeń regularnych”. Jest to niezwykle rozbudowane polecenie
%   i~moglibyśmy spędzić kilka zajęć na pracy z~nim, na~tym przedmiocie
%   ograniczymy~się do stosunkowo podstawowych przypadków. Więcej informacji
%   o~\texttt{grep}ie można znaleźć choćby tutaj
%   \parencite{Pankaj-Wailia-Mastering-Grep-command-in-Linux-Unix-ETC-Ver-2025}.

%   W~katalogu \texttt{Materiały-do-prezentacji\textbackslash
%     SO-XX-Grep\textbackslash} można znaleźć plik \textit{Lorem-ipsum.txt}
%   na którym będziemy pracować. Zawarty w~nim tekst ten został wygenerowany
%   za pomocą strony \textit{Lorem ipsum}
%   \colorhref{https://www.lipsum.com/}{https://www.lipsum.com/}.

% \end{frame}
% % ##################





% % ##################
% \begin{frame}
%   \frametitle{Praca z~\texttt{grep}em}

%   Teksty typu \textit{Lorem ipsum} są używany w~przemyśle wydawniczym od
%   XVI wieku, jak tekst do~testowania różnych układów strony, więc czemu by
%   nie użyć ich i~tutaj?

%   Załóżmy, że~jesteśmy w~katalogu, gdzie znajduje~się plik
%   \textit{Lorem-ipsum.txt}. Zaczniemy od najprostszego użycia polecenia
%   \texttt{grep} \\
%   \texttt{\$ grep "in"{} Lorem-ipsum.txt} \\
%   Od razu należy zaznaczyć, że~polecenie \texttt{grep} potrzebuje mnie dwa
%   argumenty: najpierw tekst który ma wyszukać, potem miejsce, gdzie go ma
%   wyszukać. (O~pewnych subtelnościach tego związanych z~przekierowaniami,
%   powiem sobie później.) Jeśli wprowadzimy za mało argumentów, to powłoka
%   może zacząć~się dziwnie zachowywać, wtedy należy spróbować użyć
%   pojedynczego \texttt{Ctrl-d} lub poprosić nas o~pomoc. Od~tego jesteśmy.

% \end{frame}
% % ##################





% ##################
\begin{frame}
  \frametitle{???}


  Warto za Ryanem Chadwickiem przytoczyć następujące stwierdzenie. \\
  \textit{Wszystko co możesz uruchomić w~wierszu poleceń (powłoki) możemy
    umieścić w~skrypcie i~będzie działało tak samo. Wszystko co możesz
    umieścić w~skrypcie możesz uruchomić w~wierszu poleceń i~będzie
    działało tak samo.}
  \parencite{Chadwick-BASH-Scripting-Tutorial-Part-I-What-is-ETC-Ver-2025}

\end{frame}
% ##################





% ##################
\begin{frame}
  \frametitle{Kilka słów wyjaśnienia i~ostrzeżenia}


  Gdy zaczynamy pisać skrypty \textsc{bash}a, ponownie natykamy~się na
  problem „Jak jest poprawna forma polecenia \textsc{bash}a?” oraz powiązane
  zagadnienie „Co mogę zmienić w~formie polecenia, by jego działanie nie
  uległo zmianie?” Jak już Państwo na pewno pamiętają, \textsc{bash} nie
  zna słowa litość i~jedna spacja w~złym miejscu może obrócić wszystko
  w~ruinę. Z~drugiej strony, pewne zmiany nie mają najmniejszego wpływu
  na działanie polecenia.

  Tak jak poprzednio, nie będziemy zagłębiać~się w~ciemny gąszcz detali,
  tłumaczących co można zrobić, a~czego nie, zamiast tego podamy jaki
  pisać kod \textsc{bash}a, który w~naszej opinii jest optymalny
  i~działa poprawnie. Uważamy, że~takie podejście jest optymalnym
  rozwiązaniem, biorąc pod uwagę warunki tego kursu.

  O~wszystkich znalezionych tu błędach, proszę nas informować pod adresem
  \email.

\end{frame}
% ##################
















% % ##################
% \begin{frame}
%   \frametitle{Wyszukiwanie całych wyrazów}


%   Tak jak wyjaśniliśmy wcześniej, flagi tego typu można odpowiednio łączyć,
%   tak~że polecenia: \\
%   \texttt{\$ grep -nw "in"{} Lorem-ipsum.txt} \\
%   \texttt{\$ grep -wn "in"{} Lorem-ipsum.txt} \\
%   znaczą to samo. W~poprzednich przykładach dla podkreślenia, że~testujemy
%   tylko flagę \texttt{-w} napisaliśmy ją osobno. Tą samą konwencję
%   będziemy stosować również dalej.

% \end{frame}
% % ##################





% % ##################
% \begin{frame}
%   \frametitle{Duże i~małe litery}


%   Domyślnie \texttt{grep} uwzględnia w~wyszukiwaniu wielkość liter, co
%   obrazują poniższe przykłady. \\
%   \texttt{\$ grep -n "Du"{} Lorem-ipsum.txt} \\
%   \texttt{{\color{green} 32}{\color{jAxisBlue} :}{\color{red} Du}is
%     fringilla dictum leo id
%     interdum. Ut sit amet sem a nunc porttitor} \\
%   \hspace{1em} $\vdots$ \\
%   \texttt{\$ grep -n "du"{} Lorem-ipsum.txt} \\
%   \texttt{{\color{green} 8}{\color{jAxisBlue} :}consectetur a sapien vitae,
%     euismod biben{\color{red} du}m eros.} \\
%   \hspace{1em} $\vdots$

%   By~\texttt{grep} ignorował różnicę między małymi i~dużymi literami,
%   należy mu przesłać flagę \texttt{-i}: \\
%   \texttt{\$ grep -n -i "Du"{} Lorem-ipsum.txt} \\
%   \texttt{{\color{green} 8}{\color{jAxisBlue} :}consectetur a sapien vitae,
%     euismod biben{\color{red} du}m eros.} \\
%   \hspace{1em} $\vdots$

% \end{frame}
% % ##################





% % ##################
% \begin{frame}
%   \frametitle{Zliczanie dopasowany linii}


%   By~polecenie \texttt{grep} wypisało w~jakiej liczbie linii został
%   znaleziony zadany ciąg symboli (do ilu linii został dopasowany wzorzec)
%   dajemy mu flagę~\texttt{-c}: \\
%   \texttt{\$ grep -n -c "Du"{} Lorem-ipsum.txt} \\
%   \texttt{$6$}

%   Jak już wiemy, domyślnie \texttt{grep} wypisuje linie w~których znalazł
%   dany ciąg symboli. By~zobaczyć linie \alert{nie} zawierające tego ciągu
%   należy użyć flagi~\texttt{-v}: \\
%   \texttt{\$ grep -n -v "Du"{} Lorem-ipsum.txt} \\
%   \texttt{{\color{green} 1}{\color{jAxisBlue} :}Lorem ipsum dolor sit amet,
%     consectetur adipiscing elit. Integer at est erat.} \\
%   \texttt{{\color{green} 2}{\color{jAxisBlue} :}Phasellus gravida ligula
%     eros, eget ornare purus consequat vitae. Quisque} \\
%   \hspace{1em} $\vdots$

% \end{frame}
% % ##################





% ##################
\begin{frame}
  \frametitle{Ćwiczenie~1}


  To ćwiczenie będzie wymagało trochę programowania w~języku~C,
  proszę to potraktować jako powtórkę z~tego języka. Jeśli w~treści jest
  napisane, że~należy utworzyć dany katalog, a~Państwo już utworzyli, to
  proszę pominąć ten fragment.

  W~pierwszy kroku tworzymy katalog w~katalogu domowym katalog
  \texttt{Praca-z-danymi/}, które incjalizujemy jako repozytorium Gita.
  Następnie tworzymy tam katalogi \texttt{Skrypty-BASHa-do-danych/},
  \texttt{Generowanie-danych/} i~\texttt{Dane-do-analizy/Miesiąc-Rok/},
  przy czym za~\texttt{Miesiąc} i~\texttt{Rok} wstawiamy obecny miesiąc
  i~rok.

  W~\texttt{Generowanie-danych/} piszemy programy w~języku~C o~nazwie
  \texttt{sin-with-noise.c}, który generuje nam wartości funkcji $\sin$
  „zaszumione” przez pseudolosowe wyniki generowane przez funkcję
  $\text{rand}$ z~biblioteki \texttt{stdlib}. Proszę przy tym ustawić by
  szum generował liczby $-0.5$, $-0.49$, $-0.48$, $-0.47$, \ldots,
  $0.49$,~$0.5$. Jeśli jest to trudne, to proszę poprosić prowadzącego
  o~pomoc.

\end{frame}
% ##################





% ##################
\begin{frame}
  \frametitle{Ćwiczenie~1}


  Program \texttt{sin-with-noise.c} ma generować zaszumione wartości funkcji
  $\sin$ dla argumentów od $x = -20.0$ do $20.0$ z~krokiem $0.1$.
  Wartości te ma zapisywać do pliku \texttt{Data-sin.dat}. Jeśli ktoś
  ma problem z~pracą z~plikami z~poziomu języka~C, co bywa trudne, to niech
  program ten wypisze te wyniki na ekranie.

  Następnie w~katalogu \texttt{Skrypty-BASHa-do-danych/} tworzymy
  skrypt \texttt{generate-data-with-noise-Sin.sh}, który kompiluje plik
  \texttt{sin-with-noise.c}, tworząc plik wykonywalny
  \texttt{generatingDataSin.out}. Następnie skrypt ma uruchomić program
  \texttt{generatingDataSin.out} i~wyniku jego działania zaszumione
  dane mają~się znaleźć w~pliku \texttt{Data-sin.dat}.
  Jeśli \texttt{generatingDataSin.out} wypisuje wyniki na ekranie, to
  można je zapisać w~danym pliku korzystając z~przekierowania.

  Przy pisaniu wszystkich tych poleceń dobrze jest korzystać z~zmiennej
  środowiskowej \texttt{HOME} i~utworzyć własne zmienne zawierające
  ścieżki do~konkretnych katalogów i~plików.

\end{frame}
% ##################





% ##################
\begin{frame}
  \frametitle{Ćwiczenie~1}


  W~następnym kroku skrypt \textsc{bash}a ma przenieść plik
  \texttt{Data-sin.dat} do~katalogu \texttt{Miesiąc-Rok/}.

  Kolejny element ćwiczenia polega na~sprawdzeniu czy zmienna środowiskowa
  \texttt{PATH} zawiera katalog \texttt{\$HOME/bin}. Proszę pamiętać,
  że~aby zobaczyć jak dokładnie wygląda ta ścieżka należy użyć polecenia \\
  \texttt{\$ echo \$HOME/bin} \\
  zaś zawartość zmiennej \texttt{PATH} sprawdzamy poleceniem \\
  \texttt{\$ echo \$PATH} \\
  Jeśli zmienna \texttt{PATH} nie zawiera pożądanej ścieżki, to na końcu
  pliku \texttt{\$HOME/.bashrc} dodajemy linię: \\
  \texttt{PATH=\$HOME/bin:\$PATH} \\
  i~restartujemy powłokę. Po restarcie sprawdzamy czy \texttt{PATH}
  zawiera pożądaną ścieżkę. Jeśli nie, to szukamy błędu w~pliku
  \texttt{.bashrc} lub prosimy prowadzącego o~pomoc.

\end{frame}
% ##################





% ##################
\begin{frame}
  \frametitle{Ćwiczenie~1}


  Proszę sprawdzić, czy na komputerze jest obecny katalog
  \texttt{\$HOME/bin}. Jeśli go nie ma, to proszę go utworzyć.

  Jeśli w~tym momencie zmienna \texttt{PATH} zawiera ścieżkę
  do~\texttt{\$HOME/bin}, to tworzymy link symboliczny do pliku
  \texttt{generate-data-with-noise-Sin.sh}, którym jest plik
  \texttt{\$HOME/bin/generateDataSin}. Proszę do tworzenia linku
  symbolicznego użyć ścieżek bezwzględnych, bo inaczej można utknąć
  na pewnych subtelnościach pracy \texttt{bash}a.

  Następnie proszę sprawdzić, czy wykonanie polecenia \\
  \texttt{\$ generateDataSin} \\
  poprawnie tworzy plik \texttt{Data-01.dat} w~katalogu
  \texttt{Miesiąc-Rok/}.

  Na koniec proszę wystawić na scenę pliki \texttt{sin-with-noise.c},
  \texttt{Miesiąc-Rok/Data-sin.dat}
  oraz~\texttt{generate-data-with-noise-Sin.sh} i~je skommitować.


\end{frame}
% ##################





% ##################
\begin{frame}
  \frametitle{Ćwiczenie~2}


  Ćwiczenie to polega na powtórzeniu poprzedniego, tym razem jednak
  tworzymy plik \texttt{exp-with-noise.c}, gdzie funkcja $\exp$ jest
  zaszumiana przez pseudolosowy sygnał z~zakresu $-1.0$, $-0.99$, \ldots,
  $0.99$, $1.0$. Odpowiedni program ma generować zaszumione wartości
  funkcji $\exp$ dla $x = -10.0$ do~$x = 10.0$ z~krokiem $0.1$.

  Następnie tworzymy skrypt \texttt{generate-data-with-noise-exp.sh}, który
  kompiluje \texttt{exp-with-noise.c} tworząc plik
  \texttt{generatingDataExp.out}, który z~kolei tworzy plik
  \texttt{Data-exp.dat}. Do skryptu \textsc{bash}a tworzymy symboliczny
  link \texttt{\$HOME/bin/generateDataSin}. Tak jak poprzednio wystawiamy
  na~scenę odpowiednie pliki i~kommitujemy.

\end{frame}
% ##################







% % ######################################
% \section{Polecenie \texttt{grep} i~wyrażenia regularne}
% % ######################################














% % ##################
% \begin{frame}
%   \frametitle{Wyrażenia regularne \texttt{grep}a}


%   Jak już mówiliśmy, notacja wyrażeń regularnych nie jest spójna między
%   programami i~musimy~się nauczyć z~tym żyć. W~\textsc{bash}u symbolem,
%   który oznacza dowolny pojedynczy symbol jest „\texttt{?}”, ale
%   w~\texttt{grep}ie jest nim kropka „\texttt{.}”. Czemu życie nie może być
%   proste?

%   Wobec tego wyrażenie regularne \texttt{"n.c"} wyszukuje ciąg symboli
%   który zaczyna~się od~„\texttt{n}”, potem następuje jeden pojedynczy
%   symbol, następnie zaś mamy symbol „\texttt{c}”. \\
%   \texttt{\$ grep -n "n.c"{} Lorem-ipsum.txt} \\
%   \texttt{{\color{green} 10}{\color{jAxisBlue} :}Do{\color{red}nec}
%     vehicula scelerisque efficitur. In laoreet} \\
%   \texttt{faucibus volutpat. Aliquam} \\
%   \texttt{{\color{green}13}{\color{jAxisBlue} :}convallis enim est, id
%     porta mauris lobortis eget. Do{\color{red}nec} tincidunt vel orci} \\
%   \hspace{1em} $\vdots$

% \end{frame}
% % ##################





% % ##################
% \begin{frame}
%   \frametitle{Wyrażenia regularne \texttt{grep}a}


%   Tak jak w~\textsc{bash}u, nawias kwadratowy „\texttt{[abcde]}” oznacza,
%   że~w~danym miejscu musi być \alert{dokładnie jedna} z~liter „a”, „b”,
%   „c”, „d” albo „e”. Również w~\texttt{grep}ie można zastąpić ten
%   nawias przez „\texttt{[a-e]}”.

%   \texttt{\$ grep -n "[a-e]us"{} Lorem-ipsum.txt} \\
%   \texttt{{\color{green}5}{\color{jAxisBlue} :}eleifend commodo. Etiam
%     ornare congue erat, dapi{\color{red} bus}} \\
%   \texttt{porttitor magna semper} \\
%   \texttt{{\color{green} 10}{\color{jAxisBlue} :}Donec vehicula scelerisque
%     efficitur. In laoreet} \\
%   \texttt{fauci{\color{red} bus} volutpat. Aliquam} \\
%   \hspace{1em} $\vdots$

%   Również wyrażenia regularne „\texttt{[Ab]}”, „\texttt{[012345]}”
%   i~„\texttt{[0-5]} działają tak samo jak w~\textsc{bash}u, nie będziemy
%   więc ich osobno omawiać.

% \end{frame}
% % ##################





% % ##################
% \begin{frame}
%   \frametitle{Funkcjonalności \texttt{grep}a można łączyć}


%   Oczywiście, wyrażenia regularne można łączyć z~flagami \texttt{grep}a.
%   Jeśli więc chcemy znaleźć wszystkie ciągi symboli pasujące do wyrażenia
%   regularnego \texttt{"n.c"}, które tworzą osobny wyraz to użyjemy
%   polecenia \\
%   \texttt{\$ grep -n -w "n.c"{} Lorem-ipsum.txt} \\
%   \texttt{{\color{green} 21}{\color{jAxisBlue} :}{\color{red} nec} ante
%     maximus tincidunt a vitae justo. Sed vitae convallis metus. Integer} \\
%   \texttt{{\color{green} 22}{\color{jAxisBlue} :}mollis orci
%     {\color{red} nec} urna dapibus, vel pellentesque ipsum porttitor. Nulla
%     eu} \\
%   \hspace{1em} $\vdots$

%   Jak łatwo się domyślić wyrażenie regularne \texttt{"n..c"} szuka dowolnego
%   ciągu symboli, który zaczyna~się od „\texttt{n}”, kończy na~„\texttt{c}”,
%   a~pośrodku są dwa, dokładnie dwa, dowolne symbole.

% \end{frame}
% % ##################





% % ##################
% \begin{frame}
%   \frametitle{Rozbudowane wyrażenia regularne}




% \end{frame}
% % ##################





% % ##################
% \begin{frame}
%   \frametitle{}


% \end{frame}
% % ##################




% % ######################################
% \section{}
% % ######################################








% % % ##################
% % \begin{frame}
% %   \frametitle{}




% % \end{frame}
% % % ##################





% % % ##################
% % \begin{frame}
% %   \frametitle{}



% % \end{frame}
% % % ##################





% % % ##################
% % \begin{frame}
% %   \frametitle{}




% % \end{frame}
% % % ##################





% % % ##################
% % \begin{frame}
% %   \frametitle{}



% % \end{frame}
% % % ##################










% % % ######################################
% % \section{}
% % % ######################################


% % % ##################
% % \begin{frame}
% %   \frametitle{}



% % \end{frame}
% % % ##################







% % % ######################################
% % \section{}
% % % ######################################


% % % ##################
% % \begin{frame}
% %   \frametitle{}




% % \end{frame}
% % % ##################





% % % ##################
% % \begin{frame}
% %   \frametitle{}



% % \end{frame}
% % % ##################





% % % ##################
% % \begin{frame}
% %   \frametitle{}



% % \end{frame}
% % % ##################





% % % ##################
% % \begin{frame}
% %   \frametitle{}



% % \end{frame}
% % % ##################





% % % ##################
% % \begin{frame}
% %   \frametitle{}


% % \end{frame}
% % % ##################













% % % ##################
% % \begin{frame}
% %   \frametitle{}



% % \end{frame}
% % % ##################





% % % ##################
% % \begin{frame}
% %   \frametitle{}




% % \end{frame}
% % % ##################





% % % ##################
% % \begin{frame}
% %   \frametitle{}



% % \end{frame}
% % % ##################










% % % ######################################
% % \section{}
% % % ######################################


% % % ##################
% % \begin{frame}
% %   \frametitle{}



% % \end{frame}
% % % ##################





% % % ##################
% % \begin{frame}
% %   \frametitle{}



% % \end{frame}
% % % ##################







% % % ##################
% % \begin{frame}
% %   \frametitle{}



% % \end{frame}
% % % ##################





% % % ##################
% % \begin{frame}
% %   \frametitle{}




% % \end{frame}
% % % ##################





% % % ##################
% % \begin{frame}
% %   \frametitle{}



% % \end{frame}
% % % ##################





% % % ##################
% % \begin{frame}
% %   \frametitle{}



% % \end{frame}
% % % ##################





% % ##################
% \begin{frame}
%   \frametitle{????}




% \end{frame}
% % ##################





% % ##################
% \begin{frame}
%   \frametitle{????}




% \end{frame}
% % ##################





% % ##################
% \begin{frame}
%   \frametitle{????}




% \end{frame}
% % ##################





% % ##################
% \begin{frame}
%   \frametitle{????}




% \end{frame}
% % ##################






% % ##################
% \begin{frame}
%   \frametitle{?????}




% \end{frame}
% % ##################





% % ##################
% \begin{frame}
%   \frametitle{?????}




% \end{frame}
% % ##################





% % ##################
% \begin{frame}
%   \frametitle{?????}




% \end{frame}
% % ##################





% % ##################
% \begin{frame}
%   \frametitle{?????}



% \end{frame}
% % ##################





% % ##################
% \begin{frame}
%   \frametitle{?????}



% \end{frame}
% % ##################





% % ##################
% \begin{frame}
%   \frametitle{?????}



% \end{frame}
% % ##################





% % ##################
% \begin{frame}
%   \frametitle{????}



% \end{frame}
% % ##################





% % ##################
% \begin{frame}
%   \frametitle{?????}



% \end{frame}
% % ##################





% % ##################
% \begin{frame}
%   \frametitle{????}



% \end{frame}
% % ##################





% % ##################
% \begin{frame}
%   \frametitle{?????}



% \end{frame}
% % ##################





% % ##################
% \begin{frame}
%   \frametitle{?????}



% \end{frame}
% % ##################





% % ##################
% \begin{frame}
%   \frametitle{????}



% \end{frame}
% % ##################





% % ##################
% \begin{frame}
%   \frametitle{????}



% \end{frame}
% % ##################





% % ##################
% \begin{frame}
%   \frametitle{?????}




% \end{frame}
% % ##################





% % ##################
% \begin{frame}
%   \frametitle{?????}




% \end{frame}
% % ##################





% % ##################
% \begin{frame}
%   \frametitle{?????}



% \end{frame}
% % ##################





% % ##################
% \begin{frame}
%   \frametitle{????}




% \end{frame}
% % ##################





% % ##################
% \begin{frame}
%   \frametitle{????}




% \end{frame}
% % ##################





% % ##################
% \begin{frame}
%   \frametitle{????}




% \end{frame}
% % ##################










% ######################################
\appendix
% ######################################





% ######################################
\EndingSlide{Dziękuję! Pytania?}
% ######################################










% ####################################################################
% ####################################################################
% Bibliography

\printbibliography





% ####################################################################
% End of the document

\end{document}
