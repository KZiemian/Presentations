% ------------------------------------------------------------------------------------------------------------------
% Basic configuration of Beamera class and Jagiellonian theme
% ------------------------------------------------------------------------------------------------------------------
\RequirePackage[l2tabu, orthodox]{nag}



\ifx\PresentationStyle\notset
  \def\PresentationStyle{dark}
\fi



% Options: t - align frame text to the top.
\documentclass[10pt,t]{beamer}
\mode<presentation>
\usetheme[style=\PresentationStyle]{jagiellonian}




% ------------------------------------------------------------------------------------
% Procesing configuration files of Jagiellonian theme located
% in the directory "preambule"
% ------------------------------------------------------------------------------------
% Configuration for polish language
% Need description
\usepackage[polish]{babel}
% Need description
\usepackage[MeX]{polski}



% ------------------------------
% Better support of polish chars in technical parts of PDF
% ------------------------------
\hypersetup{pdfencoding=auto,psdextra}

% Package "textpos" give as enviroment "textblock" which is very usefull in
% arranging text on slides.

% This is standard configuration of "textpos"
\usepackage[overlay,absolute]{textpos}

% If you need to see bounds of "textblock's" comment line above and uncomment
% one below.

% Caution! When showboxes option is on significant ammunt of space is add
% to the top of textblock and as such, everyting put in them gone down.
% We need to check how to remove this bug.

% \usepackage[showboxes,overlay,absolute]{textpos}



% Setting scale length for package "textpos"
\setlength{\TPHorizModule}{10mm}
\setlength{\TPVertModule}{\TPHorizModule}


% ---------------------------------------
% Packages written for lectures "Geometria 3D dla twórców gier wideo"
% ---------------------------------------
% \usepackage{./Geometry3DPackages/Geometry3D}
% \usepackage{./Geometry3DPackages/Geometry3DIndices}
% \usepackage{./Geometry3DPackages/Geometry3DTikZStyle}
% \usepackage{./ProgramowanieSymulacjiFizykiPaczki/ProgramowanieSymulacjiFizykiTikZStyle}
% \usepackage{./Geometry3DPackages/mathcommands}


% ---------------------------------------
% TikZ
% ---------------------------------------
% Importing TikZ libraries
\usetikzlibrary{arrows.meta}
\usetikzlibrary{positioning}





% % Configuration package "bm" that need for making bold symbols
% \newcommand{\bmmax}{0}
% \newcommand{\hmmax}{0}
% \usepackage{bm}




% ---------------------------------------
% Packages for scientific texts
% ---------------------------------------
% \let\lll\undefined  % Sometimes you must use this line to allow
% "amsmath" package to works with packages with packages for polish
% languge imported
% /preambul/LanguageSettings/JagiellonianPolishLanguageSettings.tex.
% This comments (probably) removes polish letter Ł.
\usepackage{amsmath}  % Packages from American Mathematical Society (AMS)
\usepackage{amssymb}
\usepackage{amscd}
\usepackage{amsthm}
\usepackage{siunitx}  % Package for typsetting SI units.
\usepackage{upgreek}  % Better looking greek letters.
% Example of using upgreek: pi = \uppi


\usepackage{calrsfs}  % Zmienia czcionkę kaligraficzną w \mathcal
% na ładniejszą. Może w innych miejscach robi to samo, ale o tym nic
% nie wiem.










% ---------------------------------------
% Packages written for lectures "Geometria 3D dla twórców gier wideo"
% ---------------------------------------
% \usepackage{./ProgramowanieSymulacjiFizykiPaczki/ProgramowanieSymulacjiFizyki}
% \usepackage{./ProgramowanieSymulacjiFizykiPaczki/ProgramowanieSymulacjiFizykiIndeksy}
% \usepackage{./ProgramowanieSymulacjiFizykiPaczki/ProgramowanieSymulacjiFizykiTikZStyle}





% !!!!!!!!!!!!!!!!!!!!!!!!!!!!!!
% !!!!!!!!!!!!!!!!!!!!!!!!!!!!!!
% EVIL STUFF
\if\JUlogotitle1
\edef\LogoJUPath{LogoJU_\JUlogoLang/LogoJU_\JUlogoShape_\JUlogoColor.pdf}
\titlegraphic{\hfill\includegraphics[scale=0.22]
{./JagiellonianPictures/\LogoJUPath}}
\fi
% ---------------------------------------
% Commands for handling colors
% ---------------------------------------


% Command for setting normal text color for some text in math modestyle
% Text color depend on used style of Jagiellonian

% Beamer version of command
\newcommand{\TextWithNormalTextColor}[1]{%
  {\color{jNormalTextFGColor}
    \setbeamercolor{math text}{fg=jNormalTextFGColor} {#1}}
}

% Article and similar classes version of command
% \newcommand{\TextWithNormalTextColor}[1]{%
%   {\color{jNormalTextsFGColor} {#1}}
% }



% Beamer version of command
\newcommand{\NormalTextInMathMode}[1]{%
  {\color{jNormalTextFGColor}
    \setbeamercolor{math text}{fg=jNormalTextFGColor} \text{#1}}
}


% Article and similar classes version of command
% \newcommand{\NormalTextInMathMode}[1]{%
%   {\color{jNormalTextsFGColor} \text{#1}}
% }




% Command that sets color of some mathematical text to the same color
% that has normal text in header (?)

% Beamer version of the command
\newcommand{\MathTextFrametitleFGColor}[1]{%
  {\color{jFrametitleFGColor}
    \setbeamercolor{math text}{fg=jFrametitleFGColor} #1}
}

% Article and similar classes version of the command
% \newcommand{\MathTextWhiteColor}[1]{{\color{jFrametitleFGColor} #1}}





% Command for setting color of alert text for some text in math modestyle

% Beamer version of the command
\newcommand{\MathTextAlertColor}[1]{%
  {\color{jOrange} \setbeamercolor{math text}{fg=jOrange} #1}
}

% Article and similar classes version of the command
% \newcommand{\MathTextAlertColor}[1]{{\color{jOrange} #1}}





% Command that allow you to sets chosen color as the color of some text into
% math mode. Due to some nuances in the way that Beamer handle colors
% it not work in all cases. We hope that in the future we will improve it.

% Beamer version of the command
\newcommand{\SetMathTextsColor}[2]{%
  {\color{#1} \setbeamercolor{math text}{fg=#1} #2}
}


% Article and similar classes version of the command
% \newcommand{\SetMathTextColor}[2]{{\color{#1} #2}}










% ---------------------------------------
% Commands for setting background pictures for some slides
% ---------------------------------------
\newcommand{\TitleBackgroundPicture}
{./PresentationPictures/CommonPictures/Cute_dragon_BG_dark.png}
\newcommand{\SectionBackgroundPicture}
{./PresentationPictures/CommonPictures/Cute_dragon_small_BG_light.png}



\newcommand{\TitleSlideWithPicture}{
  \begingroup

  \usebackgroundtemplate{ % \hspace*{-11.5em}
    \includegraphics[height=\paperheight]{\TitleBackgroundPicture}}

  \maketitle

  \endgroup
}





\newcommand{\SectionSlideWithPicture}[1]{%
  \begingroup

  \usebackgroundtemplate{ % \hspace*{-11.5em}
    \includegraphics[height=\paperheight]{\SectionBackgroundPicture}}

  \setbeamercolor{titlelike}{fg=normal text.fg}

  \section{#1}

  \endgroup
}





\newcommand{\EndingSlide}[1]{%
  \begin{frame}[standout]

    \begingroup

    \color{jFrametitleFGColor}

    #1

    \endgroup

  \end{frame}
}










% ------------------------------------------------------
% BibLaTeX
% ------------------------------------------------------
% Package biblatex, with biber as its backend, allow us to handle
% bibliography entries that use Unicode symbols outside ASCII.
\usepackage[
language=polish,
backend=biber,
style=alphabetic,
url=false,
eprint=true,
]{biblatex}

\addbibresource{Systemy-operacyjne-Bibliography.bib}





% ------------------------------------------------------
% Importing packages, libraries and setting their configuration
% ------------------------------------------------------





% ------------------------------------------------------
% Local packages
% ------------------------------------------------------
% Local configuration of this particular presentation
\usepackage{./Local-packages/local-settings}










% ------------------------------------------------------------------------------------------------------------------
\title{Systemy operacyjne}
\subtitle{Polecenia \textsc{bash}a, ciąg dalszy}

\author{Kamil Ziemian \\
  \email}


% \date{}
% ------------------------------------------------------------------------------------------------------------------










% ####################################################################
% Beginning of the document
\begin{document}
% ####################################################################





% ######################################
% Number of chars: 43k+,
% Text is adjusted to the left and words are broken at the end of the line.
\RaggedRight
% ######################################





% ######################################
\maketitle
% ######################################





% ##################
\begin{frame}
  \frametitle{Spis treści}


  \tableofcontents

\end{frame}
% ##################










% ######################################
\EndingSlide{Ta prezentacja jest nieskończona.}
% ######################################










% ######################################
\section{Polecenia \textsc{bash}a, ciąg dalszy}
% ######################################


% ##################
\begin{frame}
  \frametitle{Przesyłanie wielu argumentów do polecenia}


  Wiele poleceń \textsc{bash}a może przyjąć jako swoje argumenty więcej
  niż jedną nazwę obiektu, wyjaśnimy to na przykładzie. Materiały
  dla tego przykładu można znaleźć
  w~\texttt{Materiały-do-prezentacji/SO-03-A-Polecenia-BASHa-ETC/}.

  Przyjmijmy, że~w~katalogu w~którym~się znajdujemy są pliki
  \texttt{Data-01.dat} i~\texttt{Data-02.dat} oraz katalog
  \texttt{Dane-do-analizy/}. Aby przenieść oba pliki z~danymi do
  wspomnianego katalogu wystarczy jedno polecenie: \\
  \texttt{\$ mv Data-01.dat Data-02.dat Dane-do-analizy/} \\
  Tak samo możemy przynieść do katalogu trzy lub więcej plików. Należy
  dodać, że~katalog do którego przenosimy te pliki, \alert{zawsze}
  musi być ostatnim argumentem przesłanym do danego polecenia. Lepiej nie
  sprawdzać co może~się stać w~przeciwnym wypadku.

\end{frame}
% ##################





% ##################
\begin{frame}
  \frametitle{Przesyłanie wielu argumentów do polecenia}


  Analogicznie, możemy za~pomocą jednego polecenia \texttt{touch} utworzyć
  dwa lub więcej plików: \\
  \texttt{\$ touch Plik-01.txt Plik-02.txt}

  W~tym miejscu musimy dodać, że~istnieją znacznie potężniejsze, ale też
  bardziej skomplikowane, metody przesyłania nazw wielu obiektów do jednego
  polecenia. Ich omówienie odkładamy do momentu, gdy zapoznamy~się już
  z~podstawowymi sposobami korzystania z~powłoki.

  Zwykle dość łatwo poprawnie zgadnąć, czy dane polecenie może przyjąć
  wiele argumentów, czy tylko jedne. Mianowicie, jeśli jest dość oczywiste,
  że~dane polecenie można zastosować dla wielu argumentów, to zapewne
  działa właśnie tak jak myślimy. Na razie takie podejście powinno nam
  wystarczyć.

\end{frame}
% ##################





% ##################
\begin{frame}
  \frametitle{Polecenia i~ścieżki}


  Załóżmy teraz, że~chcemy zobaczyć zawartość katalogu
  \texttt{Dane-do-analizy/}. Żeby to zrobić możemy wejść do tego katalogu
  i~użyć polecenia \texttt{ls} tak jak poniżej. \\
  \texttt{\$ cd Dane-do-analizy/} \\
  \texttt{\$ ls} \\
  \texttt{Data-01.dat} \quad \texttt{Data-02.dat}

  Taki sam rezultat otrzymamy, jeśli poleceniu \texttt{ls} prześlemy jako
  argument nazwę katalogu, którego zawartość chcemy zobaczyć. \\
  \texttt{\$ ls Dane-do-analizy/} \\
  \texttt{Data-01.dat} \quad \texttt{Data-02.dat}

\end{frame}
% ##################





% ##################
\begin{frame}
  \frametitle{Polecenia i~ścieżki}


  Do~poleceń takich jak \texttt{ls} możemy przesyłać argumenty bardziej
  złożone, niż tylko pojedyncze nazwy katalogów. Jeśli katalog
  \texttt{Dane-do-analizy/} zawiera katalog \texttt{Dane-styczeń/} to
  zawartość tego ostatniego katalogu możemy zobaczyć wpisując
  polecenie \\
  \texttt{\$ ls Dane-do-analizy/Dane-styczeń/} \\
  \texttt{Data-January-01.dat} \quad \texttt{Data-January-02.dat} \\
  \texttt{Data-January-03.dat} \quad \texttt{Data-January-04.dat}

  Należy przyjąć, że~w~każdym poleceniu, którego argumentem może być nazwa
  pliku lub katalogu, może wystąpić też dowolnie skomplikowana ścieżka do
  niego. Ścieżka bezwzględna, zaczynająca~się od symbolu roota
  „\texttt{/}”, jest zawsze poprawna. Jeśli chodzi o~ścieżki względne, to
  prawie zawsze też są poprawne, acz niekiedy można~się natknąć na pewne
  wyjątki od reguły.

\end{frame}
% ##################





% ##################
\begin{frame}
  \frametitle{Polecenia i~ścieżki}


  Przykładowo, by utworzyć plik \texttt{Data-January-04.dat} w~katalogu
  \texttt{Dane-do-analizy/Dane-styczeń/} możemy użyć polecenia \\
  \texttt{\$ touch Dane-do-analizy/Dane-styczeń/Data-Januray-04.dat}


\end{frame}
% ##################





% % ##################
% \begin{frame}
%   \frametitle{Architektura systemu operacyjnego}




% \end{frame}
% % ##################





% % ##################
% \begin{frame}
%   \frametitle{Rola systemu operacyjnego}




% \end{frame}
% % ##################





% % ##################
% \begin{frame}
%   \frametitle{Tryby działania systemu operacyjnego}




% \end{frame}
% % ##################





% % ######################################
% \section{Włączanie powłoki, przypomnienie}
% % ######################################


% % ##################
% \begin{frame}
%   \frametitle{Powłoka BASH}

%   \vspace{-0.5em}


%   \begin{figure}

%     \centering


%     \includegraphics[scale=0.23]
%     {./Presentations-pictures/Miscancellous-pictures/BASH-shell.png}


%     \caption{Przykładowy wygląd włączonej powłoki \textsc{bash}, naszego
%       głównego narzędzia pracy na tym przedmiocie.}


%     \label{fig:BASH-shell}

%   \end{figure}

% \end{frame}
% % ##################





% % ##################
% \begin{frame}
%   \frametitle{Postęp technologiczny}

%   \vspace{-0.5em}


%   \begin{figure}

%     \label{fig:Evolution-of-OS}

%     \centering


%     \includegraphics[scale=0.3]
%     {./Presentations-pictures/Miscancellous-pictures/Evolution-of-operating-systems.jpg}

%   \end{figure}

% \end{frame}
% % ##################
















% % ##################
% \begin{frame}
%   \frametitle{Problemy z~bezpieczeństwem}


%   Alternatywą jest uruchamianie tych programów na serwerze
%   gry, który i~tak dysponuje wszystkimi tego typu informacjami. Co nie jest
%   w~interesie firmy dostarczającej te serwery, dla nich jest lepiej jak
%   serwery obsługują grę, a~wykrywaniem cheaterów zajmują~się komputery
%   graczy.

% \end{frame}
% % ##################










% % ######################################
% \section{Pliki i~procesy}
% % ######################################



% % ##################
% \begin{frame}
%   \frametitle{Pliki i~procesy}


%   Dwie podstawowe abstrakcje, których będziemy używać na tym kursie,
%   to~\textbf{pliki} i~\textbf{procesy}. Nie wnikając w~szczegóły wystarczy
%   nam stwierdzenie, że~plik czymś jest, a~proces coś robi. Warto tutaj
%   przytoczyć następujące powiedzenie, które funkcjonuje w~wielu różnych
%   wersjach, nie zawsze odnoszących się do sytemu Linux.

%   \textit{Wszystko w~systemie Linux jest plikiem. Jeśli coś nie jest
%     plikiem to jest procesem.}

%   Inaczej mówiąc, jeśli na komputerze znajdują~się jakieś dane, to są
%   wewnątrz jakiegoś pliku. Jeśli komputer coś robi, to jest za to
%   odpowiedzialny odpowiedni proces. Zrozumienie tego na tym poziomie jest
%   wystarczające na~potrzeby tych zajęć. W~przyszłości może zajrzymy pod
%   warstwę abstrakcji plików i~procesów, by~się przekonać, że~ich wewnętrzna
%   budowa wcale nie jest banalna.

% \end{frame}
% % ##################





% % ##################
% \begin{frame}
%   \frametitle{Pliki i~procesy}


%   Ponieważ pliki są mniej abstrakcyjne (hehe), zaczniemy nasze zajęcia od
%   zaznajomieniem~się z~pracą z~plikami pod powłoką \textsc{bash}.
%   Przyjmujemy przy tym, że~znają Państwo pewne pojęcia z~systemu Windows.

% \end{frame}
% % ##################










% % ######################################
% \section{Podstawy pracy z~poziomu powłoki BASH}
% % ######################################



% % ##################
% \begin{frame}
%   \frametitle{BASH nie zna słowa litość}


%   \textit{Gdy samochody rozwijały~się tak jak komputery, to dziś
%     Rolls-Royce kosztowałby jakieś 100\$, przejeżdżałaby milion mil na
%     jednym galonie benzyny i~raz do roku eksplodował zabijając wszystkich
%     w~środku.} \\
%   Robert Cringely, \textit{InfoWorld}, tłum. swobodne, str.~221,
%   \parencite{Garfinkel-Weise-Strassmann-The-UNIX-HATERS-Handbook-Pub-1994}.

%   Bóg jest litościwy, \textsc{bash} nie zna tego słowa. Przekonają~się
%   Państwo o~tym wielokrotnie. Każdy z~nas popełnia literówki i~\textsc{bash}
%   bez najmniejszej krztyny litości wytknie nam każdą z~nich, albo
%   z~jej powodu zinterpretuje wszystko na~opak. Trzeba nauczyć~się
%   żyć z~tym, że~każda najmniejsza literówka może być źródłem problemu.

%   Poza tym, język~C, czy nawet Python dają nam dużą swobodę w~rozstawianiu
%   białych znaków w~tekście. Swoboda jaką mamy w~\textsc{bash}u, gdy chodzi
%   o~białe znaki, to wolność więźnia zamkniętego w~celi. Niby można~się
%   poruszać, ale co to za~wolność. Uczulam więc Państwa, na to by pilnować,
%   gdzie są spacje, taby i~znaki nowej linii. Jeśli coś w~\textsc{bash}u
%   nie działa, to możliwe, iż~jest to wynik spacji w~złym miejscu.

% \end{frame}
% % ##################





% % ##################
% \begin{frame}
%   \frametitle{Zaczynamy}


%   Zacznijmy od przyjęcia, że~pracujemy na komputerze jako użytkownik,
%   którego nazwa w~systemie operacyjnym została ustawiona jako
%   \texttt{adam}. Jeśli włączymy konsolę zobaczymy pewnie coś takiego: \\
%   \texttt{adam:$\sim$\$} \\
%   Od tego momentu będziemy zwykle skracali ten znak zachęty \\
%   (ang.~\textit{prompt}) do symbolu \texttt{\$}. Przepisywanie pełnej jego
%   postaci wnosi bardzo mało informacji, za~to zabiera sporo miejsca.

%   Na początku zajmiemy~się tym, jak poruszać~się po systemie plików
%   i~katalogów. Na pewno wiedzą Państwo, że~na komputerze znajdują~się
%   katalogi (ang.~\textit{directories}), zwane też folderami, które zawierają
%   pliki i~inne katalogi. Z~punktu widzenia systemu GNU/Linux
%   katalog jest plikiem, jedynie specjalnego rodzaju. Dla tego systemu
%   wszystko jest plikiem lub procesem.

% \end{frame}
% % ##################





% % ##################
% \begin{frame}
%   \frametitle{Polecenia}


%   Podstawy działania katalogów w~systemie GNU/Linux są w~zasadzie takie
%   same jak w~systemie Windows, musimy~się jedynie przyzwyczaić do innych
%   konwencji używanych w~tym systemie. Dużo większym wyzwaniem zapewne
%   będzie zamienienie środowiska graficznego (\textsc{gui},
%   ang.~\textit{Graphic User Interface}) na pracę spod poziomu powłoki.

%   Podstawą pracy w~powłoce \textsc{bash} są polecenia (ang.
%   \textit{commands}). Bez wnikania w~szczegóły, każde polecenie posiada
%   nazwę i~wykonuje pewne działanie. By wykonać dane polecenie w~powłoce
%   należy najpierw wpisać jej nazwę, następnie zaś jej odpowiednie
%   opcje (flagi) i~argumenty. Część poleceń można wykonać bez podawania opcji
%   czy~argumentów.

%   Trzeba~się niestety przyzwyczaić do tego, że~nazwy poleceń, jak i~ich
%   opcji tych poleceń w~systemie GNU/Linux potrafią być dość dziwne i~trudne
%   do~zapamiętania.

% \end{frame}
% % ##################





% % ##################
% \begin{frame}
%   \frametitle{Polecenia}


%   Podstawowych poleceń systemu GNU/Linux jest około~$100$, do tego każde ma
%   kilkanaście albo więcej opcji. Z~tego powodu nauka ich jest bardzo
%   czasochłonna, a~potrafi być trudna. Teoretycznie w~każdej chwili
%   w~\textsc{bash}u możemy skorzystać z~manuala (podręcznika), gdzie są opisy
%   wszystkich poleceń. W~praktyce, żeby zrozumieć jego treść, wymagana jest
%   już dość spora wiedza.

%   Zdecydowaliśmy~się przyjąć następujące podejście. Naszym głównym celem
%   jest przekazanie Państwu umiejętności pracy spod powłoki \textsc{bash}
%   i~naszym głównym celem, jest by możliwie szybko osiągnęli Państwo
%   odpowiednią biegłość w~tym. Dlatego też, dość szybko wprowadzimy możliwie
%   dużo poleceń, bez zagłębiania~się w~zbyt wiele szczegółów. Niekiedy
%   będziemy wracać do wcześniej już omówionych poleceń, by uzupełnić
%   wiadomości o~nich. Wyczerpujące omówienie wszystkich podstawowych poleceń
%   i~tak zabrałoby zbyt dużo czasu, to zaś podejście uważamy, za rozsądny
%   kompromis.

% \end{frame}
% % ##################





% % ##################
% \begin{frame}
%   \frametitle{Polecenie \texttt{ls}}


%   Zacznijmy od~polecenia
%   \colorhref{https://linux.fandom.com/pl/wiki/Ls}{\texttt{ls}}. Jego nazwa
%   pochodzi prawdopodobnie od~ang.~\textit{List fileS}, ale są różne opinie
%   na ten temat. Polecenie to wyświetla zawartość katalogu w~którym~się
%   znajdujemy, z~tego powodu wynik działania tego polecenia będzie~się cały
%   czas zmieniać, jednym z~możliwych może być ten przedstawiony poniżej. \\
%   \texttt{\$ ls} \\
%   \texttt{{\color{jAxisBlue} Dokumenty}} \quad
%   \texttt{introduction-to-GNU-linux.pdf} \quad
%   \texttt{{\color{jAxisBlue} Pobrane}}

%   Polecenie \texttt{ls} jest przykładem takiego polecenia, które można
%   uruchomić bez dodatkowych opcji czy argumentów, ono samo ustali w~jakim
%   katalogu jesteśmy i~wypisze jego zawartość.

%   Obecnie jest standardem, że~powłoka używa kolorów i~różnych typów fontów,
%   by~informować nas o~typie i~własnościach pliku. W~przykładzie powyżej
%   nazwy katalogów są wyróżnione na niebiesko, trzeba jednak pamiętać,
%   że~wszystko to~zależy od~konkretnych ustawień systemu operacyjnego
%   i~łatwo może zostać zmienione.

% \end{frame}
% % ##################





% % ##################
% \begin{frame}
%   \frametitle{Polecenie \texttt{pwd}}


%   Kolejnym poleceniem jest
%   \colorhref{https://www.arturpyszczuk.pl/commands-pwd.html}{\texttt{pwd}}
%   od ang.~\textit{Print Working Directory}, rezultat jego działania można
%   zobaczyć poniżej. \\
%   \texttt{\$ pwd} \\
%   \texttt{/home/adam}

%   Ta krótka linia zawiera bardzo dużo informacji i~potrzebujemy przynajmniej
%   kilka z~nich wyjaśnić. Sposób w~jaki pliki i~katalogi są ułożone na~dysku
%   jest zdefiniowane przez \textbf{system plików} danego systemu
%   operacyjnego i~ten używany w~systemie GNU/Linux różni~się mocno tego
%   w~systemie Windows.

%   Zacznijmy od kilku uwag odnośnie notacji. Jeśli pracowalibyśmy w~tym
%   ostatnim systemie i~na dysku~C znajdowałby~się
%   katalog \texttt{Systemy-operacyjne-Ćwiczenia} a~w~nim katalog
%   \texttt{Zajęcia-01}, to ścieżka do niego miałaby postać: \\
%   \texttt{C:\textbackslash\textbackslash
%     Systemy-operacyjne-Ćwiczenia\textbackslash Zajęcia-01} \\

% \end{frame}
% % ##################





% % ##################
% \begin{frame}
%   \frametitle{System plików}


%   Odpowiednikiem ścieżki z~systemu Windows: \\
%   \texttt{C:\textbackslash\textbackslash
%     Systemy-operacyjne-Ćwiczenia\textbackslash Zajęcia-01} \\
%   w~systemie GNU/Linux byłaby ścieżka: \\
%   \texttt{/home/adam/Systemy-operacyjne-Ćwiczenia/Zajęcia-01} \\
%   Widzimy więc, że~gdy Windows do oddzielania nazw katalogów i~plików
%   używa symbolu „backslasha”: „\texttt{\textbackslash}”, to GNU/Linux
%   używa symbolu „slasha”: „\texttt{/}”. Jest to stosunkowo mała różnica,
%   ale trzeba~się do niej przyzwyczaić.

%   Poważniejsza różnica jest następująca. Komputery z~systemem GNU/Linux
%   mogą posiadać wiele dysków, nie jest to żadnym problemem, ale~system
%   plików GNU/Linuxa nie zna pojęcia dysku jako takiego. W~systemie GNU/Linux
%   istnienie naczelny katalog zwany rootem, czyli po polsku „korzeniem”,
%   w~którym zawierając~się wszystkie inne pliki oraz katalogi, nieważne na
%   którym fizycznym dysku są one umieszczone. Katalog root oznaczamy
%   symbolem~„\texttt{/}”.

% \end{frame}
% % ##################





% % ##################
% \begin{frame}
%   \frametitle{System plików}


%   Odpowiednikiem ścieżki z~systemu Windows: \\
%   \texttt{C:\textbackslash\textbackslash
%     Systemy-operacyjne-Ćwiczenia\textbackslash Zajęcia-01} \\
%   w~systemie GNU/Linux byłaby ścieżka: \\
%   \texttt{/home/adam/Systemy-operacyjne-Ćwiczenia/Zajęcia-01} \\
%   Widzimy więc, że~gdy Windows do oddzielania nazw katalogów i~plików
%   używa symbolu „backslasha”: „\texttt{\textbackslash}”, to GNU/Linux
%   używa symbolu „slasha”: „\texttt{/}”. Jest to stosunkowo mała różnica,
%   ale trzeba~się do niej przyzwyczaić.

%   Poważniejsza różnica jest następująca. Komputery z~systemem GNU/Linux
%   mogą posiadać wiele dysków, nie jest to żadnym problemem, ale~system
%   plików GNU/Linuxa nie zna pojęcia dysku jako takiego. W~systemie GNU/Linux
%   istnienie naczelny katalog zwany rootem, czyli po polsku „korzeniem”,
%   w~którym zawierając~się wszystkie inne pliki oraz katalogi, nieważne na
%   którym fizycznym dysku są one umieszczone. Katalog root oznaczamy
%   symbolem~„\texttt{/}”.

% \end{frame}
% % ##################





% % ##################
% \begin{frame}
%   \frametitle{System plików}


%   Wróćmy do polecenia \\
%   \texttt{\$ pwd} \\
%   \texttt{/home/adam} \\
%   Linię \texttt{/home/adam} czytamy „W~katalogu root („\texttt{/}”)
%   znajduje~się katalog \texttt{home}, w~nim zaś znajduje~się katalog
%   \texttt{adam}.” Polecenie \texttt{pwd} podaje więc w~którym katalogu~się
%   obecnie znajdujemy, za~pomocą ścieżki, która zaczyna~się od katalogu root:
%   „\texttt{/}”.

%   Katalog w~którym obecnie jesteśmy nosi nazwę \textbf{katalogu bieżącego}
%   (ang.~\textit{working directory}), a~jak powiedzieliśmy \texttt{pwd} jest
%   akronimem od~\textit{Print Working Directory}, czyli „wyświetl katalog
%   roboczy”.

%   Ścieżkę podającą lokalizację katalogu lub pliku nazywamy
%   \textbf{ścieżką bezwzględną}, jeśli zaczyna~się ona od katalogu root.
%   W~przeciwnym razie mówimy o~\textbf{ścieżce względnej} (podawanej
%   względem innego katalogu niż root).

% \end{frame}
% % ##################





% % ##################
% \begin{frame}
%   \frametitle{System plików}


%   Bez wielkich uproszczeń można przyjąć, że~\textsc{bash}u \alert{każdy}
%   ciąg symboli zaczynający od~symbolu „\texttt{/}” zostanie zinterpretowany
%   jako ścieżka do jakiegoś pliku lub katalogu. Z~tego powodu trzeba
%   \alert{bardzo} uważać na to~jak używamy symbolu „\texttt{/}”.

%   W~szczególności przy symbolu \texttt{/} trzeba bardzo uważać na spacje.
%   Ścieżka postaci \\
%   \texttt{/home/adam} \\
%   jest poprawna. Ale już ścieżka \\
%   \texttt{/ home/adam} \\
%   jest \alert{błędna}! To w~jaki sposób jest błędna, to omówimy innym
%   razem. Doświadczenie uczy, że~na początku sprawia to ludziom
%   bardzo dużo problemów.

%   To, że~główny katalog nosi nazwę „korzenia” (ang.~\textit{root}) nie jest
%   przypadkiem, ale do tej kwestii wrócimy innym razem.

% \end{frame}
% % ##################





% % ##################
% \begin{frame}
%   \frametitle{Katalog domowy}


%   W~systemie GNU/Linux każdy użytkownik ma swój katalog domowy, w~którym
%   może przechowywać potrzebne mu pliki. Standardowo, choć mogą być wyjątki,
%   użytkownik \texttt{adam} otrzyma katalog \texttt{/home/adam}, użytkownik
%   \texttt{barbara} katalog \texttt{/home/barbara},~etc. Katalog domowy
%   bieżącego użytkownika posiada też alternatywną nazwę „$\sim$”. Stąd znak
%   zachęty \\
%   \texttt{adam:$\sim$\$} \\
%   przekazuje nam informację, że~działamy jako użytkownik \texttt{adam}
%   i~znajdujemy~się w~jego katalogu domowym.

%   Załóżmy, że~katalogu domowym znajduje~się katalog o~nazwie
%   \texttt{Systemy-operacyjne-Ćwiczenia}. By sprawdzić, czy tak jest
%   naprawdę możemy użyć polecenia \texttt{ls}: \\
%   \texttt{\$ ls} \\
%   \texttt{{\color{jAxisBlue} Dokumenty}} \quad
%   \texttt{introduction-to-GNU-linux.pdf} \quad
%   \texttt{{\color{jAxisBlue} Pobrane}} \\
%   \texttt{{\color{jAxisBlue} Systemy-operacyjne-Ćwiczenia}} \\

% \end{frame}
% % ##################





% % ##################
% \begin{frame}
%   \frametitle{Polecenie \texttt{cd}}


%   Jeśli w~bieżącym katalogu znajduje~się katalog
%   \texttt{Systemy-operacyjne-Ćwiczenia} to możemy do niego wejść
%   używając polecenia
%   \colorhref{https://linux.fandom.com/pl/wiki/Cd}{\texttt{cd}}
%   (od ang.~\textit{Change Directory}): \\
%   \texttt{adam:$\sim$\$ cd Systemy-operacyjne-Ćwiczenia} \\
%   \texttt{adam:$\sim$/Systemy-operacyjne-Ćwiczenia\$} \\
%   Wyjątkowo zapisaliśmy pełniejszą wersję znaku zachęty, by podkreślić,
%   iż~informuje on~nas w~którym katalogu jesteśmy. Mianowicie ścieżka
%   \texttt{$\sim$/Systemy-operacyjne-Ćwiczenia} informuje nas, że~w~katalogu
%   domowych użytkownika \texttt{adam} znajduje~się katalog systemy
%   operacyjne \texttt{Systemy-operacyjne-Ćwiczenia} i~w~nim właśnie jesteśmy.

%   Użyte wyżej polecenie \texttt{cd} jest przykładem polecenia, które
%   pobiera argument, mianowicie nazwę katalogu do którego możemy wejść.
%   W~istocie polecenie to pozwala nam na bardzo swobodne poruszanie~się
%   po systemie plików, jeśli opanujemy dobrze zapis ścieżek w~GNU/Linuxie,
%   ale wszystko po kolei.

% \end{frame}
% % ##################





% % ##################
% \begin{frame}
%   \frametitle{Polecenie \texttt{cd}}


%   Załóżmy, że~teraz chcielibyśmy przejść do katalogu nad nami. By to zrobić
%   należy przekazać poleceniu \texttt{cd} argument \texttt{..}: \\
%   \texttt{adam:$\sim$/Systemy-operacyjne-Ćwiczenia\$ cd ..} \\
%   \texttt{adam:$\sim$\$} \\
%   Dlaczego jako argument przesyłamy akurat \texttt{..} wyjaśnimy w~swoim
%   czasie, na razie przyjmijmy, iż~tak po prostu jest. Dla lepszej
%   ilustracji tego co~zrobiliśmy, ponownie wypisaliśmy znak zachęty,

%   Załóżmy, że~chcemy teraz wejść do katalogu
%   \texttt{Systemy-operacyjne-Ćwiczenia} i~utworzyć tam plik
%   \texttt{Notatki-z-zajęć.txt}. By utworzyć ten plik możemy skorzystać
%   z~polecenia
%   \colorhref{https://linux.fandom.com/pl/wiki/Touch}{\texttt{touch}}. Pełna
%   sekwencja poleceń będzie wyglądać tak. \\
%   \texttt{\$ cd Systemy-operacyjne-Ćwiczenia} \\
%   \texttt{\$ touch Notatki-z-zajęć.txt} \\
%   Możemy teraz użyć polecenia \texttt{ls} by~się przekonać, czy~plik
%   \texttt{Notatki-z-zajęć.txt} został poprawnie utworzony.

% \end{frame}
% % ##################





% % ##################
% \begin{frame}
%   \frametitle{Edycja plików z~pomocą programu gedit}


%   Z~komputerami nigdy nic nie wiadomo, więc warto sprawdzać, czy plik
%   został poprawnie utworzony. Teraz otworzymy go do edycji za~pomocą
%   programu \colorhref{https://gedit-text-editor.org/}{gedit}. By to zrobić
%   wpisujemy w~powłoce \\
%   \texttt{\$ gedit Notatki-z-zajęć.txt \&} \\
%   Proszę pamiętać o~dodaniu na końcu polecenia znaku
%   \alert{ampersand}:~„\&”. Inaczej Gedit zablokuje nam możliwość wpisywania
%   kolejnych poleceń do powłoki i~by ją odzyskać będziemy musieli
%   wyłączyć Gedita albo użyć kombinacji poleceń \texttt{jobs} i~\texttt{bg}.

%   Czemu uruchomienie gedita bez symbolu ampersanda ma takie skutki, omówimy
%   gdy będziemy dyskutowali zagadnienie procesów i~tego jak działają
%   w~systemie GNU/Linux. Nie jest to najprostszy temat na świecie.

%   Jeśli wszystko poszło dobrze, to na ekranie zobaczymy okienko Gedita,
%   które wygląda mniej więcej tak jak to
%   na~zdjęciu~\eqref{fig:Gedit-window}.

% \end{frame}
% % ##################





% % ##################
% \begin{frame}
%   \frametitle{Okienko gedita}


%   \begin{figure}

%     \centering

%     \includegraphics[scale=0.18]
%     {./Presentations-pictures/Miscancellous-pictures/gedit-window.png}

%     \caption{Otwarte okno Gedita.}


%     \label{fig:Gedit-window}

%   \end{figure}

% \end{frame}
% % ##################





% % ##################
% \begin{frame}
%   \frametitle{Edytory tekstów}


%   Teraz możemy do pliku \texttt{Notatki-z-zajęć.txt} wpisać tekst, choćby
%   \textit{Notatki z~przedmiotu „Systemy operacyjne”}. Proszę pamiętać, by
%   zapisywać wprowadzone zmiany, służy do tego choćby skrót~\texttt{Ctrl-s}.

%   Kilka słów dlaczego wybraliśmy do~edycji plików program gedit. Na~tym
%   kursie korzystamy głównie z~niego, gdyż jest on prosty w~obsłudze
%   i~w~kilka sekund po jego otwarciu użytkownik może już pisać odpowiedni
%   tekst. Zostawiamy jednak Państwu pełną swobodę w~wyborze edytora tekstu.
%   Nie jest ważne jakiego programu~się używa, ważne jest, żeby udało~się
%   wykonać zadanie jakie przed nami stoi.

%   Jeśli chodzi o~edytory dostępne pod systemem GNU/Linux, to w~naszej
%   ocenie warto zwrócić uwagę choćby na~dwa owiane już legendą programy
%   do edycji tekstu \colorhref{https://www.vim.org/}{vim}
%   i~\colorhref{https://www.gnu.org/software/emacs/}{GNU Emacs}.

% \end{frame}
% % ##################










% % ######################################
% \section{Dalsze informacje o~pracy z~BASHem}
% % ######################################


% % ##################
% \begin{frame}
%   \frametitle{Pewne subtelności}


%   Przyjmijmy, że~jesteśmy w~katalogu \texttt{Systemy-operacyjne-Ćwiczenia}
%   i~mamy utworzony plik \texttt{Notatki-z-zajęć.txt}. Każdy system
%   operacyjny musi podjąć decyzję, czy traktuje on duże i~małe litery
%   jako różne symbole, czy jako dwie wersje tego samego symbolu. Problem
%   sprowadza~się do tego, czy \texttt{Notatki-z-zajęć.txt}
%   i~\texttt{notatki-z-zajęć.txt} są dwoma różnymi nazwami tego samego pliku,
%   czy też są to dwie różne nazwy? Kwestia ta jest sprawą konwencji,
%   ale~należy ją rozstrzygnąć by uniknąć nieporozumień.

%   Symbol GNU/Linux przyjął, że~duże i~małe litery to \alert{różne} symbole,
%   więc \texttt{Notatki-z-zajęć.txt} i~\texttt{notatki-z-zajęć.txt} to dwie
%   różne nazwy. Tak jak w~systemie Windows, w~systemie GNU/Linux w~danym
%   katalogu może być tylko jeden plik o~danej nazwie. Jest to rozsądny wybór,
%   gdyż inaczej użytkownik łatwo by~się pogubił.

% \end{frame}
% % ##################





% % ##################
% \begin{frame}
%   \frametitle{Polecenie \texttt{mkdir}}


%   Na mocy tego co powiedzieliśmy wyżej, w~tym samym katalogu mogą
%   współistnieć pliki o~nazwie \texttt{Notatki-z-zajęć.txt}
%   i~\texttt{notatki-z-zajęć.txt}. Byłyby to \alert{różne} pliki, zawartość
%   jednego z~nich nie miałaby nic wspólnego z~zawartością drugiego. Jednak
%   by uniknąć bałaganu, należy unikać podobnych nazw plików.

%   Powiedzmy, że~chcemy teraz utworzyć w~naszym katalogu
%   (\texttt{Systemy-operacyjne-Ćwiczenia}) katalog \texttt{Zajęcia-01}.
%   Służy do tego polecenie
%   \colorhref{https://www.arturpyszczuk.pl/commands-mkdir.html}
%   {\texttt{mkdir}}. Polecenie to wymaga do działania argumentu, którym
%   jest nazwa katalogu, który ma zostać utworzony (logiczne): \\
%   \texttt{\$ mkdir Zajęcia-01} \\
%   \texttt{\$ ls} \\
%   \directoryName{Zajęcia-01} \quad \texttt{Notatki-z-zajęć.txt} \\
%   Powiedzmy, że~stwierdziliśmy, że~chcemy zmienić nazwę pliku
%   \texttt{Notatki-z-zajęć.txt} na
%   \texttt{Notatki-z-systemów-operacyjnych-01.txt} i~przenieść do katalogu
%   \texttt{Zajęcia-01}.

% \end{frame}
% % ##################





% % ##################
% \begin{frame}
%   \frametitle{Polecenie \texttt{mv}}


%   W~systemie GNU/Linux zarówno do zmiany nazwy pliku, jak i~przenoszenia
%   ich między katalogami służy polecenie
%   \colorhref{https://www.geeksforgeeks.org/mv-command-linux-examples/}
%   {\texttt{mv}} (ang.~\textit{MoVe}). Możemy więc zmienić nazwę pliku
%   za~pomocą polecenia: \\
%   \texttt{\$ mv Notatki-z-zajęć.txt Notatki-z-systemów-operacyjnych-01.txt}

%   Następnie możemy przenieść plik
%   \texttt{Notatki-z-systemów-operacyjnych-01.txt} go do katalogu
%   \texttt{Zajęcia-01} za pomocą polecenia: \\
%   \texttt{\$ mv Notatki-z-systemów-operacyjnych-01.txt Zajęcia-01/} \\
%   Symbol „\texttt{/}” na końcu nazwy katalogu nie jest niezbędny,
%   ale~dobrze go tam umieszczać, bo~stanowi dodatkową informację
%   dla~\textsc{bash}a i~użytkownika, że~konkretna nazwa jest nazwą
%   katalogu.

%   Ogólnie unikanie wszelkich niejednoznaczności jest dobrą praktyką.
%   Dlatego dobrze jest w~ramach możliwości, jawnie zaznaczać charakter
%   wszystkich bytów, występujących w~poleceniach.

% \end{frame}
% % ##################





% % ##################
% \begin{frame}
%   \frametitle{Polecenie \texttt{mv}}


%   Możemy też przenieść plik do katalogu i~zmienić jego nazwę za pomocą
%   pojedynczego polecenia: \\
%   \texttt{\$ mv Notatki-z-zajęć.txt
%     Zajęcia-01/Notatki-z-systemów-operacyjnych-01.txt} \\
%   Ponieważ polecenie \texttt{mv} może przenosić pliki tylko do istniejących
%   już katalogów, samo potrafi wykryć co jest nazwą katalogu, a~co nową
%   nazwą pliku. Jak z~większością poleceń GNU/Linuxa należy jednak unikać
%   ich użycia, jeśli nie jesteśmy pewni co dane polecenie zrobi.

%   Wszyscy zapewne dobrze wiemy, że w~systemie Windows, jeśli chcemy
%   przekopiować plik o~nazwie \texttt{Dane.dat} do katalogu, który już
%   zawiera plik o~tej nazwie, to zobaczymy komunikat typu
%   \textit{Plik „Dane.dat” zostanie nadpisany. Czy chcesz kontynuować?}
%   Niestety, to \alert{nie} jest domyślne działanie polecenia \texttt{mv}
%   i~wielu innych poleceń w~systemie GNU/Linux.

% \end{frame}
% % ##################





% % ##################
% \begin{frame}
%   \frametitle{Zagrożenia jakie stwarza \texttt{mv}}


%   Załóżmy więc, że~jesteśmy w~katalogu w~którym znajduje~się plik
%   \texttt{Ważne-dane.dat} i~plik \texttt{Tekst-popularnych-piosenek.txt}.
%   Jeśli przez przypadek wprowadzimy polecenie \\
%   \texttt{\$ mv Teksty-popularnych-piosenek.txt Ważne-dane.dat} \\
%   to plik \texttt{Ważne-dane.dat} zostanie nadpisany przez niezbyt ważny
%   plik \texttt{Teksty-popularnych-piosenek.txt}. Odzyskanie pliku
%   \texttt{Ważne-dane.dat} nie jest niemożliwe, ale jest bardzo, ale to
%   \alert{bardzo} trudne i~wymaga sporych umiejętności pracy z~systemem
%   GNU/Linux.

%   \textit{Jesteś prawdziwym miłośnikiem ryzyka, jeśli przechowujesz ważne
%     pliki na~komputerze z~systemem \textsc{unix}.} (Stwierdzenie to
%   odnosi~się też do systemu GNU/Linux.) \\
%   Robert E.~Seastrom, tłum. swobodne, str.~$261$
%   \parencite{Garfinkel-Weise-Strassmann-The-UNIX-HATERS-Handbook-Pub-1994}.

%   Pomimo tych problemów, polecenia takie jak \texttt{mv} są zbyt ważne,
%   by~ich nie używać. Jak~się jednak zabezpieczyć przed takimi problemami?

% \end{frame}
% % ##################





% % ##################
% \begin{frame}
%   \frametitle{Jak rozsądnie korzystać z~\texttt{mv}?}


%   Poprzez dodanie odpowiednich opcji (flag) do~polecenia \texttt{mv}
%   może sprawić, że~zawsze, gdy polecenie to nadpisałoby istniejący
%   plik zostaniemy zapytani, czy naprawdę chcielibyśmy to zrobić.
%   W~tym przypadku opcja tą jest \texttt{-i}, rezultat jej działania
%   przedstawiony jest poniżej. \\
%   \texttt{\$ mv -i Teksty-popularnych-piosenek.txt Ważne-dane.dat} \\
%   \texttt{mv: zamazać 'Ważne-dane.dat'?} \\
%   Forma pytania jaką otrzymamy może być inna na Państwa komputerze.
%   Odpowiedzi twierdzącej na to pytanie, możemy udzielić wpisując
%   „\texttt{t}”, „\texttt{tak}”, „\texttt{y}” lub „\texttt{yes}”, wówczas
%   plik \texttt{Ważne-dane.dat} zostanie nadpisany. Jeśli chcemy udzielić
%   odpowiedzi przeczącej możemy wpisać „\texttt{n}”, „\texttt{nie}”
%   lub~„\texttt{no}”.

%   Opcje, jak sama nazwa wskazuje, zmieniają działanie polecenia i~na tym
%   kursie nie musimy wiedzieć o~nich wiele więcej.

% \end{frame}
% % ##################





% % ##################
% \begin{frame}
%   \frametitle{Opcje (flagi)}


%   Ponieważ istnieje duże ryzyko, że~osoby rozpoczynają swoją przygodę
%   z~powłoką nadpiszą przez przypadek ważny plik stosując polecenie
%   \texttt{mv}, we wszystkich przykładach będziemy do niego dodawali
%   opcję~\textit{-i}. Wyjątkiem od tej zasady będą sytuacja, gdy chcemy
%   by~polecenie \texttt{mv} nadpisało już istniejący plik.

%   Doświadczeni użytkownicy systemu GNU/Linux już parokrotnie stracili ważne
%   dane przez~\texttt{mv}, więc używają go znacznie ostrożniej i~przez
%   to ryzyko utraty kolejnych danych tą metodą jest znacznie mniejsze.

%   Opcje postaci~\texttt{-i} mogą wyglądać osobliwe i~być ciężkie do
%   zapamiętania, niestety, musimy nauczyć~się z~tym żyć. Ich poznawanie nie
%   należy do przyjemnych, wymaga dużo zapamiętywania, częstego googlowania
%   lub~nabycie umiejętności czytania trudnego w~zrozumieniu manuala
%   (\texttt{man}).

% \end{frame}
% % ##################





% % ##################
% \begin{frame}
%   \frametitle{Czemu życie nie może być proste?}


%   Wszystkie te problemy nie zmieniają faktu, że~polecenia dostępne
%   w~\textsc{bash}u wraz z~ich opcjami dają nam bardzo potężne narzędzie
%   do~ręki i~ich poznanie jest niezbędne, jeśli chcemy naprawdę dobrze
%   poznać system operacyjny GNU/Linux.

%   O~opcjach różnych poleceń i~ich pokręconej logice powiemy sobie więcej
%   w~podczas następnych spotkań.

% \end{frame}
% % ##################










% % ######################################
% \section{Autouzupełnianie w~BASHu}
% % ######################################


% % ##################
% \begin{frame}
%   \frametitle{Autouzupełnianie w~BASHu}


%   Autouzupełnianie to jedna z~funkcjonalności \textsc{bash}a, bez których
%   nie można w~nim normalnie pracować. Jest ono standardowo przypisane
%   klawiszowi \texttt{[TAB]}, jego działanie zilustrujemy na~przykładzie
%   polecenia~\texttt{mv}, ale w~pozostałych przypadkach działa ono
%   analogicznie.

%   Załóżmy, że~naszym katalogu znajdują~się pliki \texttt{Dane-01.dat}
%   i~\texttt{Dane-02.dat}, przy drugi z~tych plików chcemy przenieść
%   do~katalogu \texttt{Dane-do-analizy}. Wprowadźmy w~powłoce \texttt{mv D}
%   i~wciśnijmy klawisz \texttt{[TAB]}: \\
%   \texttt{\$ mv D [TAB]} \\
%   Na~podstawie wprowadzonej litery „\texttt{D}” i~znanej zawartości
%   katalogu powłoka postara~się uzupełnić jak największą część nazwy pliku,
%   dając w~rezultacie: \\
%   \texttt{\$ mv Dane-}

% \end{frame}
% % ##################





% % ##################
% \begin{frame}
%   \frametitle{Autouzupełnianie w~BASHu}


%   Obecnie nasze polecenie ma postać \\
%   \texttt{\$ mv Dane-} \\
%   W~tym momencie musimy wprowadzić dodatkowe znaki, tak by powłoka otrzymała
%   informację, czy chodzi o~plik \texttt{Dane-01.dat},
%   \texttt{Dane-02.dat}, czy katalog \texttt{Dane-do-analizy/}. Po wpisaniu
%   znaku „\texttt{1}” możemy znów skorzystać z~autouzupełniania. \\
%   \texttt{\$ mv Dane-02 [TAB]} \\
%   \texttt{\$ mv Dane-02.dat}

%   Jeśli nie pamiętamy jak konkretnie nazywają~się pliki, które są w~danym
%   katalogu, to możemy kliknąć przycisk \texttt{[TAB]} dwa lub więcej razy,
%   wtedy powłoka pokaże nam nazwy wszystkich plików, które pasuję
%   do~dotychczas wprowadzonej nazwy. \\
%   \texttt{\$ mv Dane- [TAB] \hspace{-1em} [TAB]} \\
%   \texttt{Dane-01.dat} \quad \texttt{Dane-02.dat} \quad \texttt{Dane-do-analizy/}

% \end{frame}
% % ##################





% % ##################
% \begin{frame}
%   \frametitle{Autouzupełnianie w~BASHu}


%   Możemy zastosować autouzupełnianie również do nazwy katalogu do~którego
%   chcemy przenieść omawiany plik: \\
%   \texttt{\$ mv Dane-02.dat Dane-d [TAB]} \\
%   \texttt{\$ mv Dane-02.dat Dane-do-analizy/}

%   Praca w~powłoce bez autouzupełniana jest zupełnie nie do zniesienia,
%   dlatego polecamy Państwu potrenować użycie tej funkcjonalności w~praktyce.
%   Jak w~przypadku wielu innych rzecz, jest to najlepszy sposób by~się jej
%   nauczyć.

% \end{frame}
% % ##################










% % ######################################
% \appendix
% % ######################################





% % ######################################
% \EndingSlide{Dziękuję! Pytania?}
% % ######################################










% % ######################################
% \section{Dlaczego BASH~się tak nazywa}
% % ######################################



% % ##################
% \begin{frame}
%   \frametitle{Dlaczego BASH~się tak nazywa?}


%   Historia zaczyna~się
%   od~\colorhref{https://en.wikipedia.org/wiki/Stephen\_R.\_Bourne}
%   {Steve Bourne’a}, który w~$1979$~roku napisał powłokę, która w~naturalny
%   sposób została nazwana
%   \colorhref{https://en.wikipedia.org/wiki/Bourne\_shell}{powłoką Bourne’a}
%   (ang.~\textit{Bourne shell}). Bazując na niej
%   \colorhref{https://en.wikipedia.org/wiki/Brian\_Fox\_(programmer)}
%   {Brian Jhan Fox} napisał w~$1989$~roku \textit{Bourne Again SHell},
%   czyli właśnie
%   \colorhref{https://en.wikipedia.org/wiki/Bash\_(Unix\_shell)}
%   {\textsc{bash}a}.

%   Sama nazwa \textit{Bourne again shell} to również żart językowy. Nazwę tę
%   można bowiem przetłumaczyć jako „ponownie powłoka Borne’a”, jak
%   i~„odrodzona na nowo powłoka”. W~tym drugim przypadku mamy do czynienia
%   z~grą słowną z~terminem
%   \colorhref{https://billygraham.org/answers/what-is-your-definition-of-a-born-again-christian}
%   {\textit{being born again}}, który zyskał
%   duże znaczenie teologiczne wśród protestantów z~nurtu
%   \colorhref{https://pl.wikipedia.org/wiki/Ewangelikalizm}{ewangelikanizmu}.

% \end{frame}
% % ##################






% % ##################
% \begin{frame}
%   \frametitle{????}




% \end{frame}
% % ##################






% % ##################
% \endingslide???{}
% % ##################



% % % ##################
% % \jagiellonianendslide{?????}
% % % ##################





% ####################################################################
% ####################################################################
% Bibliography

\printbibliography





% ####################################################################
% End of the document

\end{document}
