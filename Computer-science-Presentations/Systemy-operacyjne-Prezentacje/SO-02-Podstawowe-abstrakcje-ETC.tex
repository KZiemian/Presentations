% ------------------------------------------------------------------------------------------------------------------
% Basic configuration of Beamera class and Jagiellonian theme
% ------------------------------------------------------------------------------------------------------------------
\RequirePackage[l2tabu, orthodox]{nag}



\ifx\PresentationStyle\notset
  \def\PresentationStyle{dark}
\fi



% Options: t - align frame text to the top.
\documentclass[10pt,t]{beamer}
\mode<presentation>
\usetheme[style=\PresentationStyle,JUlogotitle=no]{jagiellonian}




% ------------------------------------------------------------------------------------
% Procesing configuration files of Jagiellonian theme located
% in the directory "preambule"
% ------------------------------------------------------------------------------------
% Configuration for polish language
% Need description
\usepackage[polish]{babel}
% Need description
\usepackage[MeX]{polski}



% ------------------------------
% Better support of polish chars in technical parts of PDF
% ------------------------------
\hypersetup{pdfencoding=auto,psdextra}

% Package "textpos" give as enviroment "textblock" which is very usefull in
% arranging text on slides.

% This is standard configuration of "textpos"
\usepackage[overlay,absolute]{textpos}

% If you need to see bounds of "textblock's" comment line above and uncomment
% one below.

% Caution! When showboxes option is on significant ammunt of space is add
% to the top of textblock and as such, everyting put in them gone down.
% We need to check how to remove this bug.

% \usepackage[showboxes,overlay,absolute]{textpos}



% Setting scale length for package "textpos"
\setlength{\TPHorizModule}{10mm}
\setlength{\TPVertModule}{\TPHorizModule}


% ---------------------------------------
% Packages written for lectures "Geometria 3D dla twórców gier wideo"
% ---------------------------------------
% \usepackage{./Geometry3DPackages/Geometry3D}
% \usepackage{./Geometry3DPackages/Geometry3DIndices}
% \usepackage{./Geometry3DPackages/Geometry3DTikZStyle}
% \usepackage{./ProgramowanieSymulacjiFizykiPaczki/ProgramowanieSymulacjiFizykiTikZStyle}
% \usepackage{./Geometry3DPackages/mathcommands}


% ---------------------------------------
% TikZ
% ---------------------------------------
% Importing TikZ libraries
\usetikzlibrary{arrows.meta}
\usetikzlibrary{positioning}





% % Configuration package "bm" that need for making bold symbols
% \newcommand{\bmmax}{0}
% \newcommand{\hmmax}{0}
% \usepackage{bm}




% ---------------------------------------
% Packages for scientific texts
% ---------------------------------------
% \let\lll\undefined  % Sometimes you must use this line to allow
% "amsmath" package to works with packages with packages for polish
% languge imported
% /preambul/LanguageSettings/JagiellonianPolishLanguageSettings.tex.
% This comments (probably) removes polish letter Ł.
\usepackage{amsmath}  % Packages from American Mathematical Society (AMS)
\usepackage{amssymb}
\usepackage{amscd}
\usepackage{amsthm}
\usepackage{siunitx}  % Package for typsetting SI units.
\usepackage{upgreek}  % Better looking greek letters.
% Example of using upgreek: pi = \uppi


\usepackage{calrsfs}  % Zmienia czcionkę kaligraficzną w \mathcal
% na ładniejszą. Może w innych miejscach robi to samo, ale o tym nic
% nie wiem.










% ---------------------------------------
% Packages written for lectures "Geometria 3D dla twórców gier wideo"
% ---------------------------------------
% \usepackage{./ProgramowanieSymulacjiFizykiPaczki/ProgramowanieSymulacjiFizyki}
% \usepackage{./ProgramowanieSymulacjiFizykiPaczki/ProgramowanieSymulacjiFizykiIndeksy}
% \usepackage{./ProgramowanieSymulacjiFizykiPaczki/ProgramowanieSymulacjiFizykiTikZStyle}





% !!!!!!!!!!!!!!!!!!!!!!!!!!!!!!
% !!!!!!!!!!!!!!!!!!!!!!!!!!!!!!
% EVIL STUFF
\if\JUlogotitle1
\edef\LogoJUPath{LogoJU_\JUlogoLang/LogoJU_\JUlogoShape_\JUlogoColor.pdf}
\titlegraphic{\hfill\includegraphics[scale=0.22]
{./JagiellonianPictures/\LogoJUPath}}
\fi
% ---------------------------------------
% Commands for handling colors
% ---------------------------------------


% Command for setting normal text color for some text in math modestyle
% Text color depend on used style of Jagiellonian

% Beamer version of command
\newcommand{\TextWithNormalTextColor}[1]{%
  {\color{jNormalTextFGColor}
    \setbeamercolor{math text}{fg=jNormalTextFGColor} {#1}}
}

% Article and similar classes version of command
% \newcommand{\TextWithNormalTextColor}[1]{%
%   {\color{jNormalTextsFGColor} {#1}}
% }



% Beamer version of command
\newcommand{\NormalTextInMathMode}[1]{%
  {\color{jNormalTextFGColor}
    \setbeamercolor{math text}{fg=jNormalTextFGColor} \text{#1}}
}


% Article and similar classes version of command
% \newcommand{\NormalTextInMathMode}[1]{%
%   {\color{jNormalTextsFGColor} \text{#1}}
% }




% Command that sets color of some mathematical text to the same color
% that has normal text in header (?)

% Beamer version of the command
\newcommand{\MathTextFrametitleFGColor}[1]{%
  {\color{jFrametitleFGColor}
    \setbeamercolor{math text}{fg=jFrametitleFGColor} #1}
}

% Article and similar classes version of the command
% \newcommand{\MathTextWhiteColor}[1]{{\color{jFrametitleFGColor} #1}}





% Command for setting color of alert text for some text in math modestyle

% Beamer version of the command
\newcommand{\MathTextAlertColor}[1]{%
  {\color{jOrange} \setbeamercolor{math text}{fg=jOrange} #1}
}

% Article and similar classes version of the command
% \newcommand{\MathTextAlertColor}[1]{{\color{jOrange} #1}}





% Command that allow you to sets chosen color as the color of some text into
% math mode. Due to some nuances in the way that Beamer handle colors
% it not work in all cases. We hope that in the future we will improve it.

% Beamer version of the command
\newcommand{\SetMathTextsColor}[2]{%
  {\color{#1} \setbeamercolor{math text}{fg=#1} #2}
}


% Article and similar classes version of the command
% \newcommand{\SetMathTextColor}[2]{{\color{#1} #2}}










% ---------------------------------------
% Commands for setting background pictures for some slides
% ---------------------------------------
\newcommand{\TitleBackgroundPicture}
{./PresentationPictures/CommonPictures/Cute_dragon_BG_dark.png}
\newcommand{\SectionBackgroundPicture}
{./PresentationPictures/CommonPictures/Cute_dragon_small_BG_light.png}



\newcommand{\TitleSlideWithPicture}{
  \begingroup

  \usebackgroundtemplate{ % \hspace*{-11.5em}
    \includegraphics[height=\paperheight]{\TitleBackgroundPicture}}

  \maketitle

  \endgroup
}





\newcommand{\SectionSlideWithPicture}[1]{%
  \begingroup

  \usebackgroundtemplate{ % \hspace*{-11.5em}
    \includegraphics[height=\paperheight]{\SectionBackgroundPicture}}

  \setbeamercolor{titlelike}{fg=normal text.fg}

  \section{#1}

  \endgroup
}





\newcommand{\EndingSlide}[1]{%
  \begin{frame}[standout]

    \begingroup

    \color{jFrametitleFGColor}

    #1

    \endgroup

  \end{frame}
}










% ------------------------------------------------------
% BibLaTeX
% ------------------------------------------------------
% Package biblatex, with biber as its backend, allow us to handle
% bibliography entries that use Unicode symbols outside ASCII.
\usepackage[
language=polish,
backend=biber,
style=alphabetic,
url=false,
eprint=true,
]{biblatex}

\addbibresource{Systemy-operacyjne-Bibliography.bib}





% ------------------------------------------------------
% Importing packages, libraries and setting their configuration
% ------------------------------------------------------





% ------------------------------------------------------
% Local packages
% ------------------------------------------------------
% Local configuration of this particular presentation
\usepackage{./Local-packages/local-settings}










% ------------------------------------------------------------------------------------------------------------------
\title{Systemy operacyjne}
\subtitle{Wprowadzenie do przedmiotu}

\author{Kamil Ziemian \\
  \email}


% \date{}
% ------------------------------------------------------------------------------------------------------------------










% ####################################################################
% Beginning of the document
\begin{document}
% ####################################################################





% ######################################
% Number of chars: 62k+,
% Text is adjusted to the left and words are broken at the end of the line.
\RaggedRight
% ######################################





% ######################################
\maketitle
% ######################################





% ##################
\begin{frame}
  \frametitle{Spis treści}


  \tableofcontents

\end{frame}
% ##################





% ######################################
\section{Włączanie powłoki, przypomnienie}
% ######################################


% ##################
\begin{frame}
  \frametitle{Powłoka BASH}

  \vspace{-0.5em}


  \begin{figure}

    \label{fig:BASH-shell}

    \centering


    \includegraphics[scale=0.23]
    {./Presentations-pictures/Miscancellous-pictures/BASH-shell.png}


    \caption{Przykładowy wygląd włączonej powłoki \textsc{bash}, naszego
      głównego narzędzia pracy na tym przedmiocie.}

  \end{figure}

\end{frame}
% ##################





% ##################
\begin{frame}
  \frametitle{Postęp technologiczny}

  \vspace{-0.5em}


  \begin{figure}

    \label{fig:Evolution-of-OS}

    \centering


    \includegraphics[scale=0.3]
    {./Presentations-pictures/Miscancellous-pictures/Evolution-of-operating-systems.jpg}

  \end{figure}

\end{frame}
% ##################





% ##################
\begin{frame}
  \frametitle{Włączanie powłoki}


  Naszym główny narzędziem na tych zajęciach będzie powłoka \textsc{bash},
  więc upewnijmy~się, że~umiemy ją włączyć, co na samy początku może być
  naprawdę niewdzięcznym zadaniem. W~pierwszym kroku proszę wcisnąć
  \texttt{Ctrl-Alt-t} i~zobaczyć, czy powłoka~się otworzyć.

  Jeśli to nie zadziała, to proszę pochodzić kursorem po ekranie,
  aż~znajdziemy okienko „Szukaj”, „Wyszukaj”, etc. Proszę tam wpisać
  „Terminal”, „Konsola” lub „Consol” i~zobaczyć, czy wyświetli~się
  odpowiednia ikona.

\end{frame}
% ##################










% ######################################
\section{Pliki i~procesy}
% ######################################



% ##################
\begin{frame}
  \frametitle{Pliki i~procesy}


  Dwie podstawowe abstrakcje, których będziemy używać na tym kursie,
  to~\textbf{pliki} i~\textbf{procesy}. Nie wnikając w~szczegóły wystarczy
  nam stwierdzenie, że~plik czymś jest, a~proces coś robi. Warto tutaj
  przytoczyć następujące powiedzenie, które funkcjonuje w~wielu różnych
  wersjach, nie zawsze odnoszących się do sytemu Linux.

  \textit{Wszystko w~systemie Linux jest plikiem. Jeśli coś nie jest
    plikiem to jest procesem.}

  Inaczej mówiąc, jeśli na komputerze znajdują~się jakieś dane, to są
  wewnątrz jakiegoś pliku. Jeśli komputer coś robi, to jest za to
  odpowiedzialny odpowiedni proces. Ponieważ pliki są mniej abstrakcyjne
  (hehe), zaczniemy nasze zajęcia od zaznajomieniem~się z~pracą z~plikami.

\end{frame}
% ##################





% % ##################
% \begin{frame}
%   \frametitle{Rozwój GNU/Linuxa}

%   \vspace{-0.5em}


%   \begin{figure}

%     \label{fig:Evolution-of-OS}

%     \centering


%     \includegraphics[scale=0.3]
%     {./Presentations-pictures/Miscancellous-pictures/Evolution-of-operating-systems.jpg}

%   \end{figure}

% \end{frame}
% % ##################





% % ##################
% \begin{frame}
%   \frametitle{Bardzo ważne}


%   Na zajęciach nie tylko można, ale \alert{należy} zadawać pytania
%   na~dowolne związany z~nimi temat. W~szczególności \alert{należy} zadawać
%   pytania, jeśli coś jest niezrozumiałe albo niejasne. Ten kurs ma być
%   wstępem do systemu operacyjnego GNU/Linux, \alert{nie} zakładamy, że~są
%   Państwo doświadczonymi administratorami tego systemu. Byłoby to bardzo
%   niewłaściwe założenie.

%   Proszę pamiętać, że~gdy chodzi o~tematy związane z~zajęciami \alert{nie}
%   ma pytań zbyt elementarnych czy głupich. Są~tylko niezadowalające
%   odpowiedzi udzielane na~Państwa pytania. Jesteśmy tutaj by Państwu pomóc
%   w~zaznajomieni~się z~pracą z~systemem operacyjnym na głębszym poziomie,
%   pytania z~Państwa strony bardzo ułatwiają nam to zadanie. Zadawanie pytań
%   nie oznacza, że~ktoś czegoś nie umie, tylko że~chce~się czegoś nauczyć.

% \end{frame}
% % ##################





% % ##################
% \begin{frame}
%   \frametitle{Bardzo ważne}


%   Pytanie można i~\alert{należy} zadawać w~trakcie zajęć, po zajęciach, jak
%   też pisząc na adres \email. Przy pisaniu na e-maila prosimy o~nadawanie
%   tytułów postaci „Pytanie o X”, „Pytanie z przedmiotu >>Systemy
%   operacyjne<<”,~etc. W~trakcie semestru zwykle dostaję dużą ilość
%   wiadomości z~zestawami zadań i~pojedynczy e-mail z~nic nie mówiącym
%   tytułem, łatwo zginie w~nich

%   Pytania typu „Jaki jest obecny kierunek rozwoju gry \textit{Path~of
%     Exile~2}?” musimy jednak zostawić na~czas po zajęciach.

%   Ten kurs jest tylko \alert{wstępem} do~systemów operacyjnych. Ze~względu
%   na rozmiar i~poziom skomplikowania tej dziedziny, dużą liczbę rzeczy
%   będziemy musieli \alert{upraszczać}. Proszę więc pamiętać, że~choć
%   staramy~się przekazywać rzetelną wiedzę, to to co mówimy będzie w~wielu
%   wypadkach bardzo mocny uproszczeniem rzeczywistości.

% \end{frame}
% % ##################










% % ######################################
% \section{Podstawy GNU/Linuxa: włączanie powłoki}
% % ######################################


% % ##################
% \begin{frame}
%   \frametitle{Powłoka BASH}

%   \vspace{-0.5em}


%   \begin{figure}

%     \label{fig:BASH-shell}

%     \centering


%     \includegraphics[scale=0.23]
%     {./Presentations-pictures/Miscancellous-pictures/BASH-shell.png}


%     \caption{Przykładowy wygląd włączonej powłoki \textsc{bash}, naszego
%       głównego narzędzia pracy na tym przedmiocie.}

%   \end{figure}

% \end{frame}
% % ##################





% % ##################
% \begin{frame}
%   \frametitle{Postęp technologiczny}

%   \vspace{-0.5em}


%   \begin{figure}

%     \label{fig:Evolution-of-OS}

%     \centering


%     \includegraphics[scale=0.3]
%     {./Presentations-pictures/Miscancellous-pictures/Evolution-of-operating-systems.jpg}

%   \end{figure}

% \end{frame}
% % ##################





% % ##################
% \begin{frame}
%   \frametitle{Włączanie powłoki}


%   Naszym główny narzędziem na tych zajęciach będzie powłoka \textsc{bash}.
%   Jej włączenie na samym początku potrafi być bardzo niewdzięcznym zadaniem.
%   W~pierwszym kroku proszę wcisnąć \texttt{Ctrl-Alt-t} i~zobaczyć, czy
%   powłoka~się otworzyć.

%   Jeśli to nie zadziała, to proszę pochodzić kursorem po ekranie,
%   aż~znajdziemy okienko „Szukaj”, „Wyszukaj”, etc. Proszę tam wpisać
%   „Terminal”, „Konsola” lub „Consol” i~zobaczyć, czy wyświetli~się
%   odpowiednia ikona. O~tym czy~się różni pojęcie „powłoki” od „terminala”
%   czy „konsoli” powiem potem. Nie jest to jakaś szczególnie ważna rzecz.

% \end{frame}
% % ##################










% % ######################################
% \section{Zajęcia, konsultacje, sposób zaliczenia}
% % ######################################


% % ##################
% \begin{frame}
%   \frametitle{Charakter zajęć}


%   Wykład zwykle ma bardziej charakter teoretyczny, a~te zajęcia
%   \alert{praktyczny}. Pewna ilość teorii musi~się tu pojawić,
%   ale~będziemy~się starali ją minimalizować.

%   Możemy ustalić jeden termin na konsultacje, ale moje doświadczenie mówi,
%   że~to nie jest dobry pomysł. W~zasadzie nikt wtedy nie przychodzi, a~ja
%   wyznaję zasadę, że~konsultacje są dla Państwa, nie dla mnie. Jeśli Państwo
%   chcą ustalenia takiego terminu, to najlepiej byłoby, gdy Państwo ustalili
%   wspólnie jakiś termin, a potem przesłali mi informacji o~swoim wyborze.

%   W~przeciwny razie zawsze można do mnie napisać z~prośbą o~konsultacje
%   indywidualne, choćby pisząc na~adres \email. Ponawiam prośbę o~nadawanie
%   e-mailom tytułów typu „Prośba o~konsultacje”, „Konsultacje z~przedmiotu
%   Systemy operacyjne”,~etc. Po prostu e-mail z~nazwą „Systemy operacyjne”
%   łatwo umknie mojej uwadze wśród wielu mu podobnych, a~tego nie chcę.

% \end{frame}
% % ##################





% % ##################
% \begin{frame}
%   \frametitle{Nie lubię zbyt poważnych zajęć, ale\ldots}

%   \vspace{-0.5em}


%   \begin{figure}

%     \label{fig:Jak-to-bywa-na-zajeciach}

%     \centering


%     \includegraphics[scale=0.42]
%     {./Presentations-pictures/Miscancellous-pictures/Jak-to-bywa-na-zajeciach.jpeg}

%   \end{figure}

% \end{frame}
% % ##################





% % ##################
% \begin{frame}
%   \frametitle{Nagrywanie zajęć}


%   Będę~się starał nagrywać na \textsc{ms}~Teamsach każde nasze spotkanie.
%   Proszę mi o~tym ciągle przypominać, bo jestem roztrzepany i~któregoś
%   razu o~tym zapomnę.

%   Proszę mi też zwracać uwagę, że~na ekranie czegoś nie widać,
%   że~czcionka za mała, że~kolory kłują w~oczy, że~nagrany dźwięk
%   jest niskiej jakości,~etc. Zajęcia są dla Państwa, naszym obowiązkiem jest
%   dostarczyć Państwu najlepszej jakości materiały do nauki jakie jesteśmy
%   w~stanie stworzyć.

%   Niestety, jakość dźwięku to coś, na co mamy mały wpływ. Mogę~się starać
%   mówić możliwie blisko mikrofonu, ale nie wiem co więcej mogę zrobić.
%   Poza tym, na~pewno nie wyjdzie zbyt dobrze, bo te zajęcia często wymagają
%   bym~się poruszał, poza tym w~którymś momencie na pewno o~tym zapomnę.
%   Swoje uwagi na temat jakości nagrań proszę kierować do ludzi
%   odpowiedzialnych za~sprawy studenckie na \textsc{wsz}i\textsc{b}ie.

% \end{frame}
% % ##################





% % ##################
% \begin{frame}
%   \frametitle{Sposób uzyskania zaliczania}


%   Można też uzyskać zaliczenie zaoczne, proszę~się w~tej sprawie zwrócić
%   do~mnie, osobiście lub piszą pod adres \email. Jak poprzednio, proszę
%   o~nadanie e-mailowi nazwy tłumaczącej jego treść.

%   Zaliczenie zdobywa~się na podstawie zadań domowych i~jednego albo dwóch
%   testów, za~które otrzymają Państwo punkty. Za każdy z~testów będzie do
%   zdobycie 10 punktów. Punktacje zestawów zadań zależą od tego jak
%   oszacujemy ich poziom trudności i~nie jesteśmy w~stanie podać liczby
%   punktów do zdobycia za nie w~tym momencie.

%   Zastanawiam~się czy nie dodać do tego projektu zaliczeniowego, ale są
%   pewne problemy techniczne, które stoją mi na drodze. Na dzień dzisiejszy
%   proszę przyjąć, że~projekt zaliczeniowego w~tym semestrze nie będzie.

% \end{frame}
% % ##################





% % ##################
% \begin{frame}
%   \frametitle{Prace domowe}


%   Zadania domowe proszę \alert{spróbować} potraktować nie jako ciężki
%   obowiązek do odhaczenia, tylko jako możliwość nauczenia~się czegoś
%   w~praktyce. Tak wiem, łatwo~się mówi, żeby nie traktować tego jako
%   ciężkiego obowiązku, a~to wcale nie jest łatwe.

%   Proszę jednak wiedzieć, że~prace domowe są zadawane w~pierwszym rzędzie
%   po~to, żeby Państwo~się czegoś nauczyli przy ich robieniu. Dopiero
%   w~drugim rzędzie, by móc Państwa ocenić. Jeśli Państwo oszukują przy
%   oddawaniu tych zdań, to z~mojej perspektywy, oszukują Państwo przede
%   wszystkim samych siebie.

%   W~przypadku wszystkich prac domowych, to należy próbować rozwiązać je
%   możliwie samodzielnie. W~razie napotkania problemów nie tylko można,
%   ale i~\alert{należy} prosić o~pomoc kolegów, korzystać z~materiałów
%   w~internecie i~używać programów takich jak Chat\textsc{gpt}.

% \end{frame}
% % ##################





% % ##################
% \begin{frame}
%   \frametitle{Sposób uzyskania zaliczania}


%   Przypominam, że~głównym celem prac domowych jest danie Państwu dodatkowej
%   możliwość, by~się czegoś nauczyć, a~w~nauce można i~\alert{należy}
%   korzystać z~pomocy. Po prostu niech, ich zrobienie nie ma zostać
%   zredukowane do użycia metody Copy’ego-Pasta.

%   Pod koniec semestru punkt zostaną podliczone, a~oceny wystawione według
%   następującej skali.

%   \begin{itemize}

%   \item $41\%\text{--}50\%$~-- ocena dostateczna ($3.0$).

%   \item $51\%\text{--}60\%$ -- ocena plus dostateczna ($3.5$, $3+$).

%   \item $61\%\text{--}70\%$ -- ocena dobra ($4.0$).

%   \item $71\%\text{--}84\%$ -- ocena puls dobry ($4.5$, $4+$).

%   \item $85\%\text{--}100\%$ -- ocena bardzo dobry ($5.0$).

%   \end{itemize}

% \end{frame}
% % ##################










% % ######################################
% \section{Materiały do nauki}
% % ######################################


% % ##################
% \begin{frame}
%   \frametitle{Materiały do nauki}


%   Na Sake będzie dostępna w~formacie \textsc{pdf} \alert{lista zagadnień do
%     opanowania z~tego przedmiotu}, która będzie główny punktem odniesieniem
%   przy tworzeniu pytań testowych. Jak również różne materiały do nauki,
%   jakie to dokładnie będą materiały, nie umiem powiedzieć w~tym momencie.

%   Będą tam również dostępne te prezentacje w~formacie \textsc{pdf}ów.
%   W~formie źródłowej (plików \LaTeX a) są dostępna na serwisie GitHub.
%   Każdy kto ma na komputerze program Git i~dostęp do internetu może je
%   pobrać wpisując \\
%   \texttt{\$ git clone https://github.com/KZiemian/Presentation} \\
%   Znajdują~się one w~katalogu „Systemy-operacyjne-Prezentacje”.

%   Można też obejrzeć to repozytorium jak normalny człowiek. Czyli
%   w~przeglądarce: \\
%   \colorhref{https://github.com/KZiemian/Presentation}
%   {https://github.com/KZiemian/Presentation}.

% \end{frame}
% % ##################







% % ######################################
% \section{Systemy operacyjne wokół nas}
% % ######################################


% % ##################
% \begin{frame}
%   \frametitle{Jakie systemy operacyjne są dziś używane?}


%   Obecnie na komputerach osobistych dominują trzy rodziny systemów
%   operacyjnych (\textsc{os}, ang. \textit{operating system}). Są to:
%   \begin{itemize}

%   \item GNU/Linux (w~skrócie: Linux);

%   \item mac\textsc{os};

%   \item Windows.

%   \end{itemize}

%   Gdy chodzi o~smartfony, to istnieją co~najmniej dwie duże rodziny systemów
%   operacyjnych dla nich:
%   \begin{itemize}

%   \item Android;

%   \item i\textsc{os}.

%   \end{itemize}
%   Niemniej ja nie jestem ekspertem w~kwestii smartfonów, więc nie będę
%   wnikał w~tą tematykę.


% \end{frame}
% % ##################





% % ##################
% \begin{frame}
%   \frametitle{Jakie systemy operacyjne są dziś używane?}


%   Jeśli chodzi o~inne systemy operacyjne to możemy wymienić i~wymieniać:
%   Fire~\textsc{os}, Free\textsc{bsd}, Free\textsc{dos}, Haiku,
%   Harmony\textsc{os}, Heli\textsc{os}, Inferno, \textsc{minix},
%   OpenHarmony, OpenSolaris, Phantom~\textsc{os}, Plan~9
%   from Bell Labs, React\textsc{os}, Redox (bardzo interesujący projekt),
%   Thesueus~\textsc{os} (inny interesujący projekt), Visopsys, etc.

%   Poza tym, systemy operacyjne można podzielić na klasy, w~zależności od
%   tego na jakim sprzęcie mają one działać. Listę podstawowych klas można
%   znaleźć poniżej.

%   \begin{itemize}

%   \item Systemy operacyjne komputerów mainframowych. Ten rodzaj komputerów
%     spotyka~się zwykle w~centrach obliczeniowych różnorakiego
%     przeznaczenia. Tutaj często używane są specjalne dystrybucje Linuxa,
%     jak~również systemy operacyjne tworzone specjalnie dla nich, jak~z/OS.

%   \end{itemize}

% \end{frame}
% % ##################





% % ##################
% \begin{frame}
%   \frametitle{Różne typy systemów operacyjnych}


%   \begin{itemize}

%   \item Systemy operacyjne serwerów. Tutaj również Linux jest popularny.

%   \item Systemy operacyjne komputerów osobistych. O~nich jest ten kurs.

%   \item Systemy operacyjne smartfonów, jak Android czy i\textsc{os}.

%   \item Wbudowane systemy operacyjne. Są to systemy operacyjne obsługujące
%     pralki, kuchenki mikrofalowe, samochody, etc.

%   \item Systemy operacyjne kart elektronicznych. Temat na inne zajęcia.

%   \item Systemy operacyjne węzłów sensorowych. Przykładem węzła sensorowego
%     są czujniki przeciwpożarowe, układy do~mierzenia temperatury, ilości
%     opadów, etc.

%   \end{itemize}

%   Dodatkowo systemy operacyjne można podzielić na monolityczne,
%   wielowarstwowe, etc.

% \end{frame}
% % ##################





% % ##################
% \begin{frame}
%   \frametitle{Komputery i~systemy operacyjne}


%   Tak jak nasz świat jest pełen komputerów, tak jest pełen systemów
%   operacyjnych. Bez wielkiej przesady można powiedzieć, że~bez systemów
%   operacyjnych setki milionów komputerów stałby~się dla większości ludzi
%   bezużyteczną stertą złomu.

%   \alert{Nie} muszą się Państwo uczyć powyższej listy. Jedyne co chcemy by
%   Państwo z~niej wynieśli, to świadomość, że~tematyka systemów operacyjnych
%   jest dość skomplikowana i~należy odczuwać zdrowy respekt wobec
%   ludzi którzy te systemy tworzą. Pamiętając, że~ci sami ludzie
%   całkiem dużo rzeczy mocno spaprali.

%   Informatyka w~ogólności, a~systemy operacyjne w~szczególności, pełna jest
%   maszyn
%   \colorhref{https://en.wikipedia.org/wiki/Rube\_Goldberg\_machine}{Rube
%     Goldbergera}. Państwo już na pewno nie raz
%   widzieli taką maszynę, a~jedną z~nich można zobaczyć
%   \colorhref{https://www.youtube.com/watch?v=vn-g1Mn2\_3g}{tutaj}.
%   Bardzo dużo w~informatyce można uprościć, ale coś mi mówi, że~zajmie nam
%   to jeszcze dużo czasu.

  % \begin{itemize}

  % \item Systemy operacyjne serwerów. Tutaj również Linux jest popularny.

  % \item Systemy operacyjne komputerów osobistych. O~nich jest ten kurs.

  % \item Systemy operacyjne smartfonów, jak Android czy i\textsc{os}.

  % \item Wbudowane systemy operacyjne. Są to systemy operacyjne obsługujące
  %   pralki, kuchenki mikrofalowych, samochodach, etc.

  % \item Systemy operacyjne kart elektronicznych. Temat na zupełnie inne
  %   zajęcia.

  % \item Systemy operacyjne węzłów sensorowych. Przykładem węzła sensorowego
  %   są czujniki przeciwpożarowe, układy do~mierzenia temperatury, ilości
  %   opadów, etc.

  % \end{itemize}

  % Dodatkowo systemy operacyjne można podzielić na monolityczne,
  % wielowarstwowe, etc.

% \end{frame}
% % ##################





% % ##################
% \begin{frame}
%   \frametitle{Dlaczego życie jest takie skomplikowane?}




%  Jeżeli
%   więc w~czasie tego kursu komuś z~Państwa nasunie~się myśl „Czy naprawdę
%   nie da~się tego zrobić prościej?”, to często odpowiedź brzmi, że~owszem,
%   można to zrobić prościej. W~informatyce wiele rzeczy można uczynić
%   znacznie prostszymi, ale, w~mojej skromnej ocenie, zajmie nam to jeszcze
%   parę ładnych dekad.

% \end{frame}
% % ##################










% % ######################################
% \section{Abstrakcje i~interfejsy}
% % ######################################


% % ##################
% \begin{frame}
%   \frametitle{Czym jest interfejs?}


%   Do kluczowych pojęć informatyki, systemów operacyjnych w~szczególności,
%   należą „abstrakcja” i~„interfejs”. By wyjaśnić sens w~jakim będziemy ich
%   używać, posłużymy~się przykładem smartfona. Jest to układ elektroniczny,
%   posiadają rozliczne kable przez które płynie prąd, akumulator,
%   procesor,~etc. Jednak na co dzień nie muszę o~tym myśleć, obsługuję go
%   klikając w~ikony.

%   Ponieważ ikony pozwalają mi zapomnieć o~elektronicy wewnątrz smartfona,
%   są one rodzaje abstrakcji. Ponieważ pozwalają mi~się komunikować z~nim,
%   są~one rodzajem interfejsu. Teraz trochę ostrzej zdefiniujemy te pojęcia.

%   \textbf{Interfejsem} (pl.~\textit{międzymordzie} ;)) nazywamy metodę
%   komunikowania~się dwóch obiektów. W~tym sensie można powiedzieć,
%   że~telefony komórkowe~są interfejsami między ludźmi, lecz my zawęzimy
%   nasze rozważania do~interfejsów właściwych komputerom.
  % Pojęcie abstrakcji ma w~informatyce swój specjalny posmak.
  % Przez \textbf{abstrakcję} rozumiemy zastąpienie jednego opisu
  % rzeczywistości i~sposobu oddziaływania na nią, inny opisem, który jest
  % bardziej odległy od rzeczywistości, ale ma być bardziej użyteczny
  % od~poprzedniego. Definicja ta ma bardziej przekazywać intuicję, niż
  % ściśle definiować pojęcie, które jest w~praktyce dość nieostre, więc
  % nie musimy~się przejmować występującymi w~niej niejasnościami.

  % Jeśli nowy opis rzeczywistości jest bardziej użyteczny dla nas nisz
  % poprzedni, to mówimy, że jest on \textbf{dobrą abstrakcją}. Jeśli nowy
  % opis nie jest bardziej użyteczny, to mówimy, że~jest \textbf{złą
  %   abstrakcją}.

  % \textbf{Przykład.} Każdy procesor posiada zestaw wbudowanych komórek
  % pamięci zwanych rejestrami. Ile rejestrów posiada procesor Państwa
  % smartfona? Jeśli nie umieją Państwo odpowiedzieć na to pytanie, to
  % znaczy, że~oprogramowanie smartfona jest dobrą abstrakcją dla
  % znajdującego~się niżej sprzętu elektronicznego.

% \end{frame}
% % ##################





% % ##################
% \begin{frame}
%   \frametitle{Czym jest abstrakcja w~informatyce?}


%   Ikony w~smartfonie są więc interfejsem, bo pozwalają na komunikację
%   między człowiekiem, a~tym urządzeniem.

%   \textbf{Abstrakcją} będziemy nazywać zmianę sposobu opisu rzeczywistości,
%   na taki, który jest z~jakiegoś powodu bardziej użyteczny dla danej grupy
%   ludzi. Ikona na smartfonie jest w~tym sensie abstrakcją, bo dzięki ich
%   istnieniu nie musimy myśleć o~tym jak płynie prąd przez smartfona.
%   Klikamy w~ikonę i~prąd popłynie w~odpowiedni sposób, zdefiniowany przez
%   tą abstrakcję.

%   Nie będziemy bardziej precyzować tych pojęć, gdyż na tym kursie ważne
%   jest tylko intuicyjne ich zrozumienie. Nie będziemy też
%   specjalnie dyskutować co odróżnia dobrą abstrakcję od~złej oraz dobry
%   interfejs od złego. Warto jednak zauważyć, że~problem dostarczenia
%   dobrych abstrakcji i~interfejsów przewija~się przez całą informatykę.

% \end{frame}
% % ##################





% % ##################
% \begin{frame}
%   \frametitle{Abstrakcje i~interfejsy są wszędzie}


%   Warto zauważyć, że~coś może być jednocześnie abstrakcją i~interfejsem,
%   jak ikony w~smartfonie. W~temat relacji między tymi pojęciami nie
%   będziemy~się jednak zagłębiać.

%   Abstrakcje i~interfejsy w~informatyce są wszędzie i~występują warstwowo:
%   gdy zdejmę jedną warstwę abstrakcji (interfejsu) pod nią zwykle zobaczę
%   kolejną warstwę abstrakcji.

%   Rozpatrzmy następujący przypadek, który nie musi być specjalnie zgodny
%   z~rzeczywistością. Załóżmy, że~mam program do przeglądania zdjęć, który
%   oczywiście sam w~sobie jest abstrakcją. Gdy zdejmę tą warstwę abstrakcji,
%   to ukaże mi~się kod źródłowy napisany w~języku Python. Jednak kod
%   źródłowy w~języku Python sam jest \alert{abstrakcją}, bo sam ten język
%   (jego interpreter) jest napisany w~języku~C. Język~C jest zaś abstrakcją
%   wobec dialektu assemblera naszego komputer. Język assemblera też jest
%   pewną abstrakcją, ale proszę, nie idźmy dalej.

% \end{frame}
% % ##################










% % ######################################
% \section{Do czego służy system operacyjny?}
% % ######################################


% % ##################
% \begin{frame}
%   \frametitle{Do czego służy system operacyjny?}


%   Komputer składa~się z~wielu części elektronicznych, takich jak procesor,
%   dysk główny, pamięć \textsc{ram}, monitor,~etc. Każdy z~tych elementów
%   posiada własny interfejs, dzięki któremu możemy~się z~nim komunikować,
%   wysyłając odpowiednie polecenia i~otrzymując odpowiednie informacje
%   zwrotne. Interfejsy te są, koniec końców, oparte o~odpowiedni przepływ
%   prądu elektrycznego. Interfejsy te są prymitywne, niewygodne w~użyciu
%   i~brzydkie. Możemy o~nich powiedzieć więcej, ale wątpię by Państwo byli
%   tym zainteresowani.

%   System operacyjny jest programem, który zarządza tymi nieprzyjemnymi
%   rzeczami za nas. Inaczej mówiąc, jest on abstrakcją nałożoną na~sprzęt
%   komputerowy. Gdy korzystamy z~dowolnego programu użytkowego, ten program
%   kontaktuje~się przez odpowiedni interfejs z~systemem operacyjnym,
%   a~system operacyjny komunikuje~się ze~sprzętem za pomocą interfejsu tego
%   sprzętu. System operacyjny pełni tutaj rolę pośrednika.

  % Do kluczowych pojęć informatyki, systemów operacyjnych w~szczególności,
  % należą „abstrakcji” i~„interfejs”. Wyjaśnimy w~jakim sensie ich używamy,
  % na przykładzie typowego smartfona. Jest to układ elektroniczny, posiadają
  % rozliczne kable przez które płynie prąd, akumulator, procesor, etc.
  % Jednak na co dzień nie muszę o~tym myśleć, obsługuję go klikając w~ikony.

  % Ponieważ ikony pozwalają mi zapomnieć o~elektronicy wewnątrz smartfona,
  % są one rodzaje abstrakcji. Ponieważ pozwalają mi~się komunikować z~nim,
  % są~one rodzajem interfejsu. Teraz trochę ostrzej zdefiniujemy te pojęcia.

  % \textbf{Interfejsem} (pl.~\textit{międzymordzie};)) nazywamy metodę
  % komunikowania~się dwóch obiektów. W~tym sensie można powiedzieć,
  % że~telefony komórkowe~są interfejsami, lecz my zawęzimy nasze rozważania
  % do~interfejsów komputerowych.
  % Pojęcie abstrakcji ma w~informatyce swój specjalny posmak.
  % Przez \textbf{abstrakcję} rozumiemy zastąpienie jednego opisu
  % rzeczywistości i~sposobu oddziaływania na nią, inny opisem, który jest
  % bardziej odległy od rzeczywistości, ale ma być bardziej użyteczny
  % od~poprzedniego. Definicja ta ma bardziej przekazywać intuicję, niż
  % ściśle definiować pojęcie, które jest w~praktyce dość nieostre, więc
  % nie musimy~się przejmować występującymi w~niej niejasnościami.

  % Jeśli nowy opis rzeczywistości jest bardziej użyteczny dla nas nisz
  % poprzedni, to mówimy, że jest on \textbf{dobrą abstrakcją}. Jeśli nowy
  % opis nie jest bardziej użyteczny, to mówimy, że~jest \textbf{złą
  %   abstrakcją}.

  % \textbf{Przykład.} Każdy procesor posiada zestaw wbudowanych komórek
  % pamięci zwanych rejestrami. Ile rejestrów posiada procesor Państwa
  % smartfona? Jeśli nie umieją Państwo odpowiedzieć na to pytanie, to
  % znaczy, że~oprogramowanie smartfona jest dobrą abstrakcją dla
  % znajdującego~się niżej sprzętu elektronicznego.

% \end{frame}
% % ##################





% % ##################
% \begin{frame}
%   \frametitle{Uproszczony schemat działania programu}


%   \begin{figure}

%     \label{fig:Scheme-of-CPU}


%     \begin{tikzpicture}

%       \fill[fill=brown] (-3,0) rectangle (3,1);


%       % \draw[color=black,dashed] (2.05,2.1) rectangle (2.65,2.7);

%       % \node[color=jMathTextFGColorOfStyleLight] at (2.35,2.4)
%       % {Sprzęt komputerowy};



%       % \draw[color=black,dashed] (2.05,1.2) rectangle (2.65,1.8);

%       % \node[color=jMathTextFGColorOfStyleLight] at (2.35,1.5)
%       % {\texttt{R1}};



%       % \draw[color=black,dashed] (2.05,0.3) rectangle (2.65,0.9);

%       % \node[color=jMathTextFGColorOfStyleLight] at (2.35,0.6)
%       % {\texttt{R2}};



%       % \draw[color=black,dashed] (2.05,-0.6) rectangle (2.65,0);

%       % \node[color=jMathTextFGColorOfStyleLight] at (2.35,-0.3)
%       % {\texttt{R3}};



%       % \draw[color=black,dashed] (2.05,-1.5) rectangle (2.65,-0.9);

%       % \node[color=jMathTextFGColorOfStyleLight] at (2.35,-1.2)
%       % {\texttt{R4}};



%       % \draw[color=black,dashed] (2.05,-2.4) rectangle (2.65,-1.8);

%       % \node[color=jMathTextFGColorOfStyleLight] at (2.35,-2.1)
%       % {\texttt{R5}};


%       % \draw[color=black,dashed] (2.05,-3.3) rectangle (2.65,-2.7);

%       % \node[color=jMathTextFGColorOfStyleLight] at (2.35,-3) {R5};

%     \end{tikzpicture}

%     \caption{Uproszczony schemat łączenia~się programu komputerowego
%       ze~sprzętem}


%   \end{figure}

% \end{frame}
% % ##################







% % ##################
% \begin{frame}
%   \frametitle{Architektura systemu operacyjnego}


%   Ponieważ system operacyjny jest abstrakcją, jego interfejs który
%   nam udostępnia, może być znacznie bardziej wyrafinowany, prostszy
%   w~użyciu i~piękniejszy od~interfejsu sprzętowego.

%   W~tym momencie potrzebujemy wprowadzić nowe pojęcie.
%   \textbf{Architekturą systemu operacyjnego} nazywamy ogólny plan budowy
%   danego systemu operacyjnego. Architektura takiego systemu jest więc
%   zbiorem idei, których używamy do tworzenia oraz opisywania systemów
%   operacyjnych. Jak poprzednio, nie będziemy potrzebowali ściślejszej
%   definicji tego pojęcia.

%   Podstawowa dziś architektura stosowana dzisiaj, mówi nam, że~system
%   operacyjny może działać w~dwóch różnych trybach: trybie jądra i~trybie
%   użytkownika. Żeby zrozumieć czemu ten podział został wprowadzony,
%   wróćmy do zagadnienia: po co istnieje system operacyjny?

  % System ten istnieje, by zdjąć z~nas obowiązek bezpośredniego zarządzania
  % sprzętem, robienia takich rzeczy jak ręczne ustawienie wartości rejestrów
  % procesora.
  % Informatyka to osobna dziedzina nauki i~jeśli zabrnie~się odpowiednio
  % głęboko, to robi~się naprawdę złożona i~niebanalna. Używając przed chwilą
  % wprowadzonej terminologi, powiem, że~informatyka robi~się niebanalna, gdy
  % zdejmiemy odpowiednio dużo poziomów abstrakcji. Na wysokim poziomie
  % abstrakcji to czy jest on trudna czy nie, to mocno zależy od~odczuć
  % konkretnej osoby.

  % Zadajmy sobie następujące pytanie: czy włączenie komputera jest
  % skomplikowane? Odpowiemy na to pytanie na dwóch poziomach abstrakcji.
  % Pierwszy to poziom normalnego użytkownika, drugi to opis pochodzący
  % z~książki Andrewa S.~Tanenbauma \textit{Systemy operacyjne. Wydanie~III}
  % \parencite{Tannenbaum-Systemy-Operacyjne-Wydanie-III-Pub-2013}, dotyczący
  % komputera z~systemem Pentium.

% \end{frame}
% % ##################





% % ##################
% \begin{frame}
%   \frametitle{Architektura systemu operacyjnego}


%   System ten istnieje, by zdjąć z~nas obowiązek bezpośredniego zarządzania
%   sprzętem, robienia takich rzeczy jak ręczne ustawienie wartości rejestrów
%   procesora, bo normalny użytkownik nie chce mieć nigdy do czynienia
%   z~takimi rzeczami. Oprócz tego, że~życie użytkownika staje~się prostsze,
%   to jest też bezpieczniejsze, bo pracując na niskim poziomie abstrakcji,
%   łatwiej o~pomyłki i~poważne błędy. Przykładowo, potrzebuje mieć możliwość
%   kasowania danych na dysku. Co jeśli przypadkiem zamiast skasować jeden
%   plik rozkażę skasować całą zawartość dysku?

%   Co więcej, jeśli kilka osób ma dostęp do tego samego komputera, to
%   należy zapewnić, żeby dane jednej osoby, które powinny pozostać tajne, nie
%   były od ręki dostępne innym użytkownikom tego komputera.

% \end{frame}
% % ##################





% % ##################
% \begin{frame}
%   \frametitle{Jądro systemu operacyjnego}


%   Aby zapewnić możliwie dużą wygodę, bezpieczeństwo i~tajność danych,
%   w~obecnej architekturze systemów operacyjnych możliwe są dwa tryby
%   działania: \textbf{tryb jądra} (ang. \textit{kernel mode})
%   i~\textbf{tryb użytkownika} (ang. \textit{user mode}). Należy pamiętać,
%   że~to są dwa różne tryby działania tego samego programu.

%   Działając w~trybie jądra, możemy korzystać z~wszystkich możliwości jakie
%   udostępnia nam sprzęt oraz widzimy całą pamięć. Tryb ten musi istnieć,
%   bo~wszystkie operacje sprzętu są potrzebne do działania komputera, więc
%   ktoś musi móc je uruchomić. Tak samo, ktoś musi mieć dostęp do całej
%   pamięci, aby móc przydzielać ją innym.

%   Następnie jest tryb użytkownika. Gdy jesteśmy w~tym trybie, system
%   operacyjny wyłącza nam dostęp do pewny możliwości naszego komputera.
%   W~szczególności, jeśli jakaś część pamięci jest przydzielona innemu
%   użytkownikowi, jest ona dla nas niedostępna w~tym trybie.

% \end{frame}
% % ##################





% % ##################
% \begin{frame}
%   \frametitle{Jądro systemu operacyjnego}


%   Jak tego typu ograniczenia działają w~praktyce? System operacyjny jest
%   programem jak każdy inny, więc jego działanie zależy od~ustawień. Jeśli
%   działając jako użytkownik chcemy zobaczyć zawartość jakiegoś fragmentu
%   pamięci, to wysyłamy zapytanie do systemu operacyjnego o~jej wyświetlenie
%   nam. System, który zawsze widzi całą dostępną pamięć, sprawdza, czy ma
%   prawo udostępnić nam ten fragment pamięci. Jeśli nie to przesyła nam
%   wiadomość, że~nie mamy uprawnień do zobaczenia zawartości tego fragmentu
%   pamięci.

%   Jak można~się z~tego domyślić, jeśli zdobędę dostęp do trybu jądra, mogę
%   z~tym komputerem zrobić wszystko. Dlatego hakerzy dokładają tylu starań,
%   by uzyskać do niego dostęp \alert{bez} posiadania odpowiednich uprawnień.
%   Dlatego też takie antycheaterskie programy takie jak Vanguard, który miał
%   być dodany choćby do
%   \colorhref{https://www.youtube.com/watch?v=nk6aKV2rY7E}{\textit{League~of
%       Legends}}~są tak problematyczne
%   (\parencite{Low-Level-Why-Riots-anti-cheat-is-a-HUGE-problem-Ver-2024}).

  % Jeśli posiadamy dostęp do danego miejsca w~pamięci, to system operacyjny
  % korzystając z~odpowiednich własności sprzętu wczyta ten fragment pamięci
  % i~nam go da. Jeśli uzna, że~nie mam dostępu, to po prostu prześle nam
  % informację, iż~prosimy o~rzecz do której nie mamy uprawnień.
  % Jądro systemu operacyjnego to ta jego część, która bezpośrednio
  % komunikuje~się ze sprzętem. Z~tego powodu jądro musi posiadać możliwość
  % dostania~się do dowolnej części pamięci i~wykonania dowolnej czynności
  % jaką posiada sprzęt.

  % Następnie jest część peryferyjna. Jeśli jesteśmy w~części peryferyjnej,
  % to jądro udostępnia nam tylko \alert{część} swoich możliwości.
  % W~szczególności, jeśli jesteśmy w~części peryferyjnej to część pamięci,
  % jak ta przydzielona innemu użytkownikowi jest dla nas niedostępna.

  % Jak takie ograniczenie działa w~praktyce? System operacyjnym jest
  % programem i~jak każdy program, działa wedle określonych reguł. Gdy jako
  % jeden z~użytkowników komputera chcę~się dostać do określonego fragmentu
  % pamięci, to muszę wysłać odpowiednią informację do systemu operacyjnego.
  % System operacyjny, który widzi całą pamięć, sprawdza czy mam prawo prosić
  % o~dostęp do tego miejsca.

% \end{frame}
% % ##################





% % ##################
% \begin{frame}
%   \frametitle{Jądro systemu operacyjnego}


%   Vanguard działa bowiem w~trybie jądra, ma więc możliwość by zrobić
%   z~moich komputerem absolutnie wszystko co chce. Czy ufamy firmie
%   tworzącej ten program, że~nie nadużyją tych możliwości?

%   Pytanie, czemu w~ogóle pozwalamy istnieć taki programom jak Vanguard?
%   Dlatego, że~jeśli program ten pracuje w~trybie jądra to do wyszukiwania
%   cheaterów używam zasobów \alert{naszych} komputera. Jeśli pracowałby
%   w~trybie użytkownika, nie byłoby to możliwe, gdyż z~definicji trybu
%   użytkownika, program ten nie miałby dostępu do informacji potrzebnych by
%   wykryć cheatera. Alternatywą jest uruchamianie tych programów na serwerze
%   gry, który i~tak dysponuje wszystkimi tego typu informacjami.

  % Jak tego typu ograniczenia działają w~praktyce? System operacyjny jest
  % programem jak każdy inny, więc jego działanie zależy od~ustawień. Jeśli
  % działając jako użytkownik chcemy zobaczyć zawartość jakiegoś fragmentu
  % pamięci, to wysyłamy zapytanie do systemu operacyjnego o~jej wyświetlenie
  % nam. System, który zawsze widzi całą dostępną pamięć, sprawdza, czy ma
  % prawo udostępnić nam ten fragment pamięci. Jeśli nie to przesyła nam
  % wiadomość, że~nie mamy uprawnień by ją widzieć.

  % Jak można~się z~tego domyślić, jeśli zdobędę dostęp do trybu jądra, mogę
  % z~tym komputerem zrobić wszystko. Dlatego hakerzy dokładają tylu starań,
  % by uzyskać do niego dostęp \alert{bez} posiadania odpowiednich uprawnień.
  % Dlatego też takie antycheaterskie programy takie jak Vanguard, który miał
  % być dodany choćby do
  % \colorhref{https://www.youtube.com/watch?v=nk6aKV2rY7E}{\textit{League~of
  %     Legends}}~są tak problematyczne
  % (\parencite{Low-Level-Why-Riots-anti-cheat-is-a-HUGE-problem-Ver-2024}).

  % Jeśli posiadamy dostęp do danego miejsca w~pamięci, to system operacyjny
  % korzystając z~odpowiednich własności sprzętu wczyta ten fragment pamięci
  % i~nam go da. Jeśli uzna, że~nie mam dostępu, to po prostu prześle nam
  % informację, iż~prosimy o~rzecz do której nie mamy uprawnień.
  % Jądro systemu operacyjnego to ta jego część, która bezpośrednio
  % komunikuje~się ze sprzętem. Z~tego powodu jądro musi posiadać możliwość
  % dostania~się do dowolnej części pamięci i~wykonania dowolnej czynności
  % jaką posiada sprzęt.

  % Następnie jest część peryferyjna. Jeśli jesteśmy w~części peryferyjnej,
  % to jądro udostępnia nam tylko \alert{część} swoich możliwości.
  % W~szczególności, jeśli jesteśmy w~części peryferyjnej to część pamięci,
  % jak ta przydzielona innemu użytkownikowi jest dla nas niedostępna.

  % Jak takie ograniczenie działa w~praktyce? System operacyjnym jest
  % programem i~jak każdy program, działa wedle określonych reguł. Gdy jako
  % jeden z~użytkowników komputera chcę~się dostać do określonego fragmentu
  % pamięci, to muszę wysłać odpowiednią informację do systemu operacyjnego.
  % System operacyjny, który widzi całą pamięć, sprawdza czy mam prawo prosić
  % o~dostęp do tego miejsca.

% \end{frame}
% % ##################










% % ######################################
% \section{Czy ten przedmiot będzie trudny?}
% % ######################################



% % ##################
% \begin{frame}
%   \frametitle{Czy ten przedmiot będzie trudny?}


%   Informatyka to osobna dziedzina nauki i~jeśli zabrnie~się odpowiednio
%   głęboko, to robi~się naprawdę złożona i~niebanalna. Używając przed chwilą
%   wprowadzonej terminologi, powiem, że~informatyka robi~się niebanalna, gdy
%   zdejmiemy odpowiednio dużo warstw abstrakcji. Na wysokim poziomie
%   abstrakcji to czy jest on trudna czy nie, to mocno zależy od~odczuć
%   konkretnej osoby.

%   Zadajmy sobie następujące pytanie: czy włączenie komputera jest
%   skomplikowane? Odpowiemy na to pytanie na dwóch poziomach abstrakcji.
%   Pierwszy to poziom normalnego użytkownika, drugi to opis pochodzący
%   z~książki Andrewa S.~Tanenbauma \textit{Systemy operacyjne. Wydanie~III}
%   \parencite{Tannenbaum-Systemy-Operacyjne-Wydanie-III-Pub-2013}, dotyczący
%   komputera z~systemem Pentium.

% \end{frame}
% % ##################





% % ##################
% \begin{frame}
%   \frametitle{Włączanie komputera, poziom normalnego użytkownika}


%   \begin{enumerate}

%   \item Wciskamy przycisk \texttt{Power}.



%   \item Czekamy minutę albo dłużej.



%   \item Wybieramy użytkownika i~wchodzimy na swoje konto.

%   \end{enumerate}

%   Co w~tym trudnego?

% \end{frame}
% % ##################





% % ##################
% \begin{frame}
%   \frametitle{Kilka pojęcia}


%   Oczywiście, opis włączania komputera z~książki Tanenbauma jest tak
%   skomplikowany, że~trzeba wprowadzić trochę pojęć wstępnych.

%   \textbf{\textsc{rom}}, ang.~\textit{Read Only Memory}, pl.~\textit{pamięć
%     wyłącznie do~odczytu}. Pamięć komputera której zawartość została
%   zapisana przez firmę, która ten fragment pamięci wyprodukowała
%   i~użytkownik nie może zmodyfikować jej zawartości. Przynajmniej nie żadnym
%   normalny sposobem.

%   \textbf{\textsc{ram}}, ang.~\textit{Random Access Memory},
%   pl.~\textit{pamięć o~dostępie w~trybie losowym}. Pamięć komputera o~tej
%   własności, że~jeśli wylosuję dowolny jej element, to czas odczytania
%   informacje z~tego elementu nie będzie zależał od tego, który element
%   został wylosowany. Inaczej mówiąc dostęp do dowolnego miejsca tej pamięci
%   zajmuje tyle samo czasu.

%   Tak naprawdę czas odczytu zależy w~pewnym stopniu od tego, w~jaki
%   konkretny sposób pamięć \textsc{ram} jest odczytywana, ale jeszcze długo
%   nie będziemy się musieli tym przejmować.

% \end{frame}
% % ##################





% % ##################
% \begin{frame}
%   \frametitle{Kilka pojęcia}


%   \textbf{Pamięć ulotna}, ang.~\textit{volatile memory}. Pamięć której
%   zawartość jest tracona, gdy przestaje przez nią płynąć prąd. Typowym
%   przykładem takiej pamięci jest \textsc{ram}.

%   \textbf{Pamięć nieulotna}, ang.~\textit{non-volatile memory}. Pamięć,
%   której treść jest zachowana, gdy przez układ przestaje płynąć prąd,
%   typowym przykładem jest dysk \textsc{ssd}.

%   Żeby skomplikować życie, pamięcią nieulotną nazywa~się także tą pamięć,
%   które jest ulotna w~ścisłym sensie, ale ponieważ jest zaopatrzona
%   we~własną baterię, jej zawartość jest zachowana również po wyłączeniu
%   komputera z~prądu. Bo~niby czemu życie ma być proste?

% \end{frame}
% % ##################





% % ##################
% \begin{frame}
%   \frametitle{Kilka pojęcia}


%   \textbf{Pamięć \textsc{cmos}}, często po prostu \textbf{\textsc{cmos}}.
%   Skrót pochodzi od nazwy technologi ang.~\textit{Complementary
%     Metal-Oxide-Semiconductor}, pl.~\textit{komplementarny półprzewodnik
%     metalowo-tlenkowy}, w~której ta pamięć jest wykonana. Musi być zasilana
%   prądem, by~zachowywała swój stan, ale ponieważ wyposażona jest w~baterię
%   klasyfikowana jest jako nieulotna.

%   \textbf{\textsc{bios}} ang.~\textit{Basic Input Output System}, pl.
%   \textit{podstawowy system wejścia, wyjścia}. Program znajdujący~się
%   na płycie głównej komputera, odpowiedzialny między innymi za odczytywanie
%   klawiatury, zapisywanie ekranu oraz operacje wejścia-wyjścia dysków.

% \end{frame}
% % ##################





% % ##################
% \begin{frame}
%   \frametitle{Uruchamianie komputera z~systemem Pentium}


%   \begin{itemize}

%   \item[1)] Wciskamy przycisk \texttt{Power}.



%   \item[2)] Z~płyty głównej ładowany jest program \textsc{bios}. Sprawdza on
%     ilość zainstalowanej pamięci \textsc{ram}, czy komputer dysponuje
%     klawiaturą i~innymi podstawowymi urządzeniami oraz sprawdza czy
%     odpowiadają one w~sposób prawidłowy. W~pierwszej kolejności skanowane
%     są magistrale \textsc{ISA} (ang. \textit{Industry Standard
%       Architecture}) i~\textsc{pci} (ang.~\textit{Peripheral Component
%       Interconnect}) w~celu wykrycia podłączonych do nich urządzeń.



%   \item[3)] Jeśli do komputera podłączone są inne urządzenia, niż te które
%     były dostępne przy jego ostatni uruchomieniu, nowe urządzenia są
%     konfigurowane.



%   \item[4)] Program \textsc{bios} odczytuje listę tzw. urządzeń rozruchowych
%     z~pamięci \textsc{cmos}. Urządzenia rozruchowe to te, które zawierają
%     system operacyjny. W~przeszłości były nimi dyskietki, płyty
%     \textsc{cd}-\textsc{rom}, \textsc{dvd}, dziś choćby pendriwy
%     i~dyski~\textsc{ssd}.

%   \end{itemize}

% \end{frame}
% % ##################





% % ##################
% \begin{frame}
%   \frametitle{Uruchamianie komputera z~systemem Pentium}


%   \begin{itemize}

%   \item[5)] \textsc{bios} testuje po kolei urządzenia rozruchowe
%     z~wspomnianej wcześniej listy, aż~znajdzie pierwszy, który zawiera
%     działający system operacyjny.



%   \item[6)] \textsc{bios} wczytuje pierwszy sektor ze~znalezionego
%     w~poprzednim punkcie urządzenia rozruchowego do pamięci i~go uruchamia.




%   \item[7)] Program z~pierwszego sektora sprawdza zapisaną na jego końcu
%     listę partycji, by~ustalić która z~nich jest partycją aktywną.
%     Następnie wczytuje z~tej partycji pomocniczy program rozruchowy.



%   \item[8)] Pomocniczy program rozruchowy wczytuje system operacyjny
%     z~aktywnej partycji i~go uruchamia.



%   \item[9)] System operacyjny odczytuje informacje konfiguracyjne z~systemu
%     \textsc{bios}. Dla każdego dostępnego urządzenia sprawdza, czy posiada
%     do niego sterowniki. Jeśli nie, to prosi o~ich zainstalowanie
%     z~odpowiedniego źródła.

%   \end{itemize}

% \end{frame}
% % ##################





% % ##################
% \begin{frame}
%   \frametitle{Rasterization and fragment operations}


%   \begin{itemize}

%   \item[10)] Jeśli system operacyjny dysponuje wszystkimi sterownikami,
%     to ładuje je do jądra systemu.



%   \item[11)] System operacyjny tworzy tabele systemowe oraz procesy
%     działające w~tle.



%   \item[12)] Uruchamiane jest okno logowania.

%   \end{itemize}

% \end{frame}
% % ##################






% % ##################
% \begin{frame}
%   \frametitle{Bootowanie}


%   W~literaturze funkcjonuje termin \textbf{bootwoanie}, zwane też
%   \textbf{uruchamianiem} lub \textbf{rozruchem}. Odnosi~się ono albo do
%   całej procedury uruchamiania komputer opisanej powyżej, albo tylko
%   stawiania systemu operacyjnego, czyli od kiedy \textsc{bios} wczytał
%   pierwszy jego sektor do pamięci (punkt siedem i~dalej). Acz to pojęcie
%   nie jest specjalnie ostro zdefiniowane.

% \end{frame}
% % ##################





% % ##################
% \begin{frame}
%   \frametitle{Czy uruchomienie komputera jest proste czy trudne?}


%   Zależy jak do tego podchodzimy. I~tak jest z~większością rzeczy
%   w~informatyce.

% \end{frame}
% % ##################










% % ######################################
% \section{Krótka historia powstania systemu GNU/Linuxa}
% % ######################################


% % ######################################
% \section{Krótka historia powstania systemu GNU/Linuxa}
% % ######################################


% % ##################
% \begin{frame}
%   \frametitle{Ten, od którego wszystko~się zaczęło}


%   \begin{figure}

%     \centering

%     \includegraphics[scale=0.09]
%     {./PresentationsPictures/OS-heroes-Pictures/Ken-Thompson.jpg}

%     \caption{Ken Thompson (ur.~1943), główny twórca systemu UNIX.}

%     \label{fig:Ken-Thompson}

%   \end{figure}

% \end{frame}
% % ##################





% % ##################
% \begin{frame}
%   \frametitle{Wkracza Dennis Ritchie i~język~C}


%   \begin{figure}

%     \centering


%     \includegraphics[scale=0.225]
%     {./PresentationsPictures/OS-heroes-Pictures/Dennis-Ritchie.jpeg}

%     \caption{Dennis Ritchie (1941--2011), współtwórca UNIXa. Na~potrzeby
%       napisania nowej wersji tego systemu stworzył język~C, za~datę
%       powstania tego języka możemy uważać rok~1972~r.}

%     \label{fig:Dennis-Ritchie}

%   \end{figure}

% \end{frame}
% % ##################





% % ##################
% \begin{frame}
%   \frametitle{Tak to wtedy wyglądało}


%   \begin{figure}

%     \centering


%     \includegraphics[scale=0.3]
%     {./PresentationsPictures/OS-heroes-Pictures/Dennis-Ritchie-Ken-Thompson-PDP-11.jpg}

%     \caption{Dennis Ritchie (stojący) i~Ken Thompson pracują na komputerze
%       PDP-11.}

%     \label{fig:Ritchie-Thompson-PDP-11}

%   \end{figure}

% \end{frame}
% % ##################





% % ##################
% \begin{frame}
%   \frametitle{Richard Stallman, free software i~GNU}


%   \begin{figure}

%     \centering


%     \includegraphics[scale=0.35]
%     {./PresentationsPictures/OS-heroes-Pictures/Richard-Stallman.jpeg}

%     \caption{Richard Matthew Stallman (ur.~1953). W~1983 rozpoczął działanie
%       GNU Project.}

%     \label{fig:Richard-M-Stallman}

%   \end{figure}

% \end{frame}
% % ##################





% % ##################
% \begin{frame}
%   \frametitle{Tanenbaum i~jego MINIX}


%   \begin{figure}

%     \centering


%     \includegraphics[scale=0.65]
%     {./PresentationsPictures/OS-heroes-Pictures/Andrew-S-Tanenbaum.jpeg}

%     \caption{Andrew S.~Tanenbaum (ur.~1944). W~roku 1988 stworzył system
%       MINIX (ang.~\textit{mini-Unix}).}

%     \label{fig:Andrew-S-Tanenbaum}

%   \end{figure}

% \end{frame}
% % ##################





% % ##################
% \begin{frame}
%   \frametitle{Ludzie dzięki którym powstał system GNU/Linux}


%   \begin{figure}

%     \centering


%     \includegraphics[scale=0.286]
%     {./PresentationsPictures/OS-heroes-Pictures/Linus-Torvalds.jpg}

%     \caption{Linus Torvalds (ur.~1969). W~roku 1991 roku stworzył jądro
%       systemu Linux (\textit{Linus + UNIX}).}

%     \label{fig:Linus-Torvalds}

%   \end{figure}

% \end{frame}
% % ##################










% % ######################################
% \section{Historia powłoki BASH}
% % ######################################



% % ##################
% \begin{frame}
%   \frametitle{Ludzie którzy dali nam powłokę BASH}


%   \begin{figure}

%     \centering


%     \includegraphics[scale=0.55]
%     {./PresentationsPictures/OS-heroes-Pictures/Steve-Bourne.jpeg}

%     \caption{Stephen Bourne (ur.~1944). W~1979~roku stworzył powłokę
%       Bourne’a.}

%     \label{fig:Stephen-Bourne}

%   \end{figure}

% \end{frame}
% % ##################





% % ##################
% \begin{frame}
%   \frametitle{Ludzie którzy dali nam powłokę BASH}


%   \begin{figure}

%     \centering


%     \includegraphics[scale=0.05]
%     {./PresentationsPictures/OS-heroes-Pictures/Brian-J-Fox.png}

%     \caption{Brian J.~Fox (ur.~1964). W~1989~r. stworzył powłokę
%       \textsc{bash} (ang. \textit{Bourne-Again SHell}).}

%     \label{fig:Brain-J-Fox}

%   \end{figure}

% \end{frame}
% % ##################










% % ######################################
% \section{Maszyny, bo one też mają znaczenie}
% % ######################################



% % ##################
% \begin{frame}
%   \frametitle{Dalekopis (ang. \textit{teletypewriter})}


%   \begin{figure}

%     \centering


%     \includegraphics[scale=0.6]
%     {./PresentationsPictures/Machines-Pictures/Hughes-telegraph.jpeg}

%     \caption{Dalekopis zaprojektowany przez Davida Edwarda Hughesa
%       (1830-1900), wyprodukowany około 1855 roku.}

%     \label{fig:Dalekopis-Hughesa}

%   \end{figure}

% \end{frame}
% % ##################





% % ##################
% \begin{frame}
%   \frametitle{Komputer PDP-7, lata 60-te XX wieku}


%   \begin{figure}

%     \centering


%     \includegraphics[scale=0.2]
%     {./PresentationsPictures/Machines-Pictures/PDP-7-with-teletype.jpeg}

%     \caption{Komputer PDP-7 z~dalekopisem.}

%     \label{fig:PDP-7-z-dalekopisem}

%   \end{figure}

% \end{frame}
% % ##################





% % ##################
% \begin{frame}
%   \frametitle{PDP-11, lata 70-te XX-wieku}


%   \begin{figure}

%     \centering


%     \includegraphics[scale=0.225]
%     {./PresentationsPictures/Machines-Pictures/PDP-11.jpeg}

%     \caption{Komputer PDP-11.}

%     \label{fig:PDP-11}

%   \end{figure}

% \end{frame}
% % ##################





% % ##################
% \begin{frame}
%   \frametitle{Dalekopis Teletype Model 33 ASR}


%   \begin{figure}

%     \centering


%     \includegraphics[scale=0.028]
%     {./PresentationsPictures/Machines-Pictures/Teletype-Model-33-ASR-01.jpeg}
%     \includegraphics[scale=0.2]
%     {./PresentationsPictures/Machines-Pictures/Teletype-Model-33-ASR-02.jpeg}

%     \caption{Dalekopis Teletype Model~33 firmy Teletype Corporation,
%       dostępny w~wersji komercyjnej od 1963~r. Jedna z~pierwszych urządzeń,
%       które wspierało system kodowania ASCII. }

%     \label{fig:Teletype-Model-33-ASR}

%   \end{figure}

% \end{frame}
% % ##################










% % ######################################
% \section{W~jakich językach pisze~się systemy opearcyjne?}
% % ######################################



% % ##################
% \begin{frame}
%   \frametitle{W~jakich językach pisze~się systemy opearcyjne?}


%   W~pewnym uproszczeniu, wszystkie pięć wielkich rodzin systemów
%   operacyjnych, GNU/Linux, macOS, Windows, Android oraz iOS są napisane
%   w~języku~C.

% \end{frame}
% % ##################








% % ##################
% \begin{frame}
%   \frametitle{Kilka banalnych uwag na zakończenie}


%   W~literaturze znajduje~się zarówno zapis „UNIX”, ale też i~„Unix”.
%   Podobnie spotyka~się zarówno „MINIX” jak i~„Minix”, a~także warianty
%   innych nazw. We~wszystkich tego typu przypadkach będę~się starał podążać
%   za konwencjami używanymi w~książce Andrewa S. Tanenbauma \textit{Systemy
%     operacyjne. Wydanie~III}
%   \parencite{Tannenbaum-Systemy-Operacyjne-Wydanie-III-Pub-2013}, chyba
%   że~względy estetyczne wymuszą użycie innej.

% \end{frame}
% % ##################



% % ##################
% \begin{frame}
%   \frametitle{Ten, od którego wszystko~się zaczęło}


%   \begin{figure}

%     \centering

%     \includegraphics[scale=0.09]
%     {./PresentationsPictures/OS-heroes-Pictures/Ken-Thompson.jpg}

%     \caption{Ken Thompson (ur.~1943), główny twórca systemu UNIX.}

%     \label{fig:Ken-Thompson}

%   \end{figure}

% \end{frame}
% % ##################





% % ##################
% \begin{frame}
%   \frametitle{Wkracza Dennis Ritchie i~język~C}


%   \begin{figure}

%     \centering


%     \includegraphics[scale=0.225]
%     {./PresentationsPictures/OS-heroes-Pictures/Dennis-Ritchie.jpeg}

%     \caption{Dennis Ritchie (1941--2011), współtwórca UNIXa. Na~potrzeby
%       napisania nowej wersji tego systemu stworzył język~C, za~datę
%       powstania tego języka możemy uważać rok~1972~r.}

%     \label{fig:Dennis-Ritchie}

%   \end{figure}

% \end{frame}
% % ##################





% % ##################
% \begin{frame}
%   \frametitle{Tak to wtedy wyglądało}


%   \begin{figure}

%     \centering


%     \includegraphics[scale=0.3]
%     {./PresentationsPictures/OS-heroes-Pictures/Dennis-Ritchie-Ken-Thompson-PDP-11.jpg}

%     \caption{Dennis Ritchie (stojący) i~Ken Thompson pracują na komputerze
%       PDP-11.}

%     \label{fig:Ritchie-Thompson-PDP-11}

%   \end{figure}

% \end{frame}
% % ##################





% % ##################
% \begin{frame}
%   \frametitle{Richard Stallman, free software i~GNU}


%   \begin{figure}

%     \centering


%     \includegraphics[scale=0.35]
%     {./PresentationsPictures/OS-heroes-Pictures/Richard-Stallman.jpeg}

%     \caption{Richard Matthew Stallman (ur.~1953). W~1983 rozpoczął działanie
%       GNU Project.}

%     \label{fig:Richard-M-Stallman}

%   \end{figure}

% \end{frame}
% % ##################





% % ##################
% \begin{frame}
%   \frametitle{Tanenbaum i~jego MINIX}


%   \begin{figure}

%     \centering


%     \includegraphics[scale=0.65]
%     {./PresentationsPictures/OS-heroes-Pictures/Andrew-S-Tanenbaum.jpeg}

%     \caption{Andrew S.~Tanenbaum (ur.~1944). W~roku 1988 stworzył system
%       MINIX (ang.~\textit{mini-Unix}).}

%     \label{fig:Andrew-S-Tanenbaum}

%   \end{figure}

% \end{frame}
% % ##################





% % ##################
% \begin{frame}
%   \frametitle{Ludzie dzięki którym powstał system GNU/Linux}


%   \begin{figure}

%     \centering


%     \includegraphics[scale=0.286]
%     {./PresentationsPictures/OS-heroes-Pictures/Linus-Torvalds.jpg}

%     \caption{Linus Torvalds (ur.~1969). W~roku 1991 roku stworzył jądro
%       systemu Linux (\textit{Linus + UNIX}).}

%     \label{fig:Linus-Torvalds}

%   \end{figure}

% \end{frame}
% % ##################










% % ######################################
% \section{Historia powłoki BASH}
% % ######################################



% % ##################
% \begin{frame}
%   \frametitle{Ludzie którzy dali nam powłokę BASH}


%   \begin{figure}

%     \centering


%     \includegraphics[scale=0.55]
%     {./PresentationsPictures/OS-heroes-Pictures/Steve-Bourne.jpeg}

%     \caption{Stephen Bourne (ur.~1944). W~1979~roku stworzył powłokę
%       Bourne’a.}

%     \label{fig:Stephen-Bourne}

%   \end{figure}

% \end{frame}
% % ##################





% % ##################
% \begin{frame}
%   \frametitle{Ludzie którzy dali nam powłokę BASH}


%   \begin{figure}

%     \centering


%     \includegraphics[scale=0.05]
%     {./PresentationsPictures/OS-heroes-Pictures/Brian-J-Fox.png}

%     \caption{Brian J.~Fox (ur.~1964). W~1989~r. stworzył powłokę
%       \textsc{bash} (ang. \textit{Bourne-Again SHell}).}

%     \label{fig:Brain-J-Fox}

%   \end{figure}

% \end{frame}
% % ##################










% % ######################################
% \section{Maszyny, bo one też mają znaczenie}
% % ######################################



% % ##################
% \begin{frame}
%   \frametitle{Dalekopis (ang. \textit{teletypewriter})}


%   \begin{figure}

%     \centering


%     \includegraphics[scale=0.6]
%     {./PresentationsPictures/Machines-Pictures/Hughes-telegraph.jpeg}

%     \caption{Dalekopis zaprojektowany przez Davida Edwarda Hughesa
%       (1830-1900), wyprodukowany około 1855 roku.}

%     \label{fig:Dalekopis-Hughesa}

%   \end{figure}

% \end{frame}
% % ##################





% % ##################
% \begin{frame}
%   \frametitle{Komputer PDP-7, lata 60-te XX wieku}


%   \begin{figure}

%     \centering


%     \includegraphics[scale=0.2]
%     {./PresentationsPictures/Machines-Pictures/PDP-7-with-teletype.jpeg}

%     \caption{Komputer PDP-7 z~dalekopisem.}

%     \label{fig:PDP-7-z-dalekopisem}

%   \end{figure}

% \end{frame}
% % ##################





% % ##################
% \begin{frame}
%   \frametitle{PDP-11, lata 70-te XX-wieku}


%   \begin{figure}

%     \centering


%     \includegraphics[scale=0.225]
%     {./PresentationsPictures/Machines-Pictures/PDP-11.jpeg}

%     \caption{Komputer PDP-11.}

%     \label{fig:PDP-11}

%   \end{figure}

% \end{frame}
% % ##################





% % ##################
% \begin{frame}
%   \frametitle{Dalekopis Teletype Model 33 ASR}


%   \begin{figure}

%     \centering


%     \includegraphics[scale=0.028]
%     {./PresentationsPictures/Machines-Pictures/Teletype-Model-33-ASR-01.jpeg}
%     \includegraphics[scale=0.2]
%     {./PresentationsPictures/Machines-Pictures/Teletype-Model-33-ASR-02.jpeg}

%     \caption{Dalekopis Teletype Model~33 firmy Teletype Corporation,
%       dostępny w~wersji komercyjnej od 1963~r. Jedna z~pierwszych urządzeń,
%       które wspierało system kodowania ASCII. }

%     \label{fig:Teletype-Model-33-ASR}

%   \end{figure}

% \end{frame}
% % ##################










% % ######################################
% \section{W~jakich językach pisze~się systemy opearcyjne?}
% % ######################################



% % ##################
% \begin{frame}
%   \frametitle{W~jakich językach pisze~się systemy opearcyjne?}


%   W~pewnym uproszczeniu, wszystkie pięć wielkich rodzin systemów
%   operacyjnych, GNU/Linux, macOS, Windows, Android oraz iOS są napisane
%   w~języku~C.

% \end{frame}
% % ##################








% % ##################
% \begin{frame}
%   \frametitle{Kilka banalnych uwag na zakończenie}


%   W~literaturze znajduje~się zarówno zapis „UNIX”, ale też i~„Unix”.
%   Podobnie spotyka~się zarówno „MINIX” jak i~„Minix”, a~także warianty
%   innych nazw. We~wszystkich tego typu przypadkach będę~się starał podążać
%   za konwencjami używanymi w~książce Andrewa S. Tanenbauma \textit{Systemy
%     operacyjne. Wydanie~III}
%   \parencite{Tannenbaum-Systemy-Operacyjne-Wydanie-III-Pub-2013}, chyba
%   że~względy estetyczne wymuszą użycie innej.

% \end{frame}
% % ##################











































% % ##################
% \begin{frame}
%   \frametitle{????}




% \end{frame}
% % ##################





% % ##################
% \begin{frame}
%   \frametitle{????}




% \end{frame}
% % ##################





% % ##################
% \begin{frame}
%   \frametitle{????}




% \end{frame}
% % ##################





% % ##################
% \begin{frame}
%   \frametitle{????}




% \end{frame}
% % ##################






% % ##################
% \begin{frame}
%   \frametitle{?????}




% \end{frame}
% % ##################





% % ##################
% \begin{frame}
%   \frametitle{?????}




% \end{frame}
% % ##################





% % ##################
% \begin{frame}
%   \frametitle{?????}




% \end{frame}
% % ##################





% % ##################
% \begin{frame}
%   \frametitle{?????}



% \end{frame}
% % ##################





% % ##################
% \begin{frame}
%   \frametitle{?????}



% \end{frame}
% % ##################





% % ##################
% \begin{frame}
%   \frametitle{?????}



% \end{frame}
% % ##################





% % ##################
% \begin{frame}
%   \frametitle{????}



% \end{frame}
% % ##################





% % ##################
% \begin{frame}
%   \frametitle{?????}



% \end{frame}
% % ##################





% % ##################
% \begin{frame}
%   \frametitle{????}



% \end{frame}
% % ##################





% % ##################
% \begin{frame}
%   \frametitle{?????}



% \end{frame}
% % ##################





% % ##################
% \begin{frame}
%   \frametitle{?????}



% \end{frame}
% % ##################





% % ##################
% \begin{frame}
%   \frametitle{????}



% \end{frame}
% % ##################





% % ##################
% \begin{frame}
%   \frametitle{????}



% \end{frame}
% % ##################





% % ##################
% \begin{frame}
%   \frametitle{?????}




% \end{frame}
% % ##################





% % ##################
% \begin{frame}
%   \frametitle{?????}




% \end{frame}
% % ##################





% % ##################
% \begin{frame}
%   \frametitle{?????}



% \end{frame}
% % ##################





% % ##################
% \begin{frame}
%   \frametitle{????}




% \end{frame}
% % ##################





% % ##################
% \begin{frame}
%   \frametitle{????}




% \end{frame}
% % ##################





% % ##################
% \begin{frame}
%   \frametitle{????}




% \end{frame}
% % ##################

































% ######################################
\appendix
% ######################################



% ##################
\begin{frame}
  \frametitle{Co to jest bootstraping?}


  Warto zacząć od wyjaśnienia skąd to angielskie słowo pochodzi.
  \textit{Bootstrap} oznacza ucho od buta, zaś samo „bootstraping”
  podnoszenie siebie samego z~ziemi poprzez ciągnięcie za ucha swoich butów.
  Żeby coś takiego było możliwe, świat nasz musiałby dopuszczać dość mocne
  złamanie trzeciej zasady dynamiki Newtona, a~nie wydaje~się by tak było.

  W~informatyce termin \textit{bootstraping} odnosi~się do wielu sytuacji,
  gdy program „uruchamia sam siebie”. Co dokładnie to znaczy, to zależy
  od~konkretnego przypadku, niemniej informatyka zna wiele ciekawych
  przypadków użycia procedury bootstrapingu.

\end{frame}
% ##################






% % ##################
% \begin{frame}
%   \frametitle{????}




% \end{frame}
% % ##################


% % ##################
% \endingslide???{}
% % ##################



% % % ##################
% % \jagiellonianendslide{?????}
% % % ##################





% ####################################################################
% ####################################################################
% Bibliography

\printbibliography





% ####################################################################
% End of the document

\end{document}
