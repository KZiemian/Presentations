% ------------------------------------------------------------------------------------------------------------------
% Basic configuration of Beamera class and Jagiellonian theme
% ------------------------------------------------------------------------------------------------------------------
\RequirePackage[l2tabu, orthodox]{nag}



\ifx\PresentationStyle\notset
  \def\PresentationStyle{dark}
\fi



% Options: t - align frame text to the top.
\documentclass[10pt,t]{beamer}
\mode<presentation>
\usetheme[style=\PresentationStyle,JUlogotitle=no]{jagiellonian}




% ------------------------------------------------------------------------------------
% Procesing configuration files of Jagiellonian theme located
% in the directory "preambule"
% ------------------------------------------------------------------------------------
% Configuration for polish language
% Need description
\usepackage[polish]{babel}
% Need description
\usepackage[MeX]{polski}



% ------------------------------
% Better support of polish chars in technical parts of PDF
% ------------------------------
\hypersetup{pdfencoding=auto,psdextra}

% Package "textpos" give as enviroment "textblock" which is very usefull in
% arranging text on slides.

% This is standard configuration of "textpos"
\usepackage[overlay,absolute]{textpos}

% If you need to see bounds of "textblock's" comment line above and uncomment
% one below.

% Caution! When showboxes option is on significant ammunt of space is add
% to the top of textblock and as such, everyting put in them gone down.
% We need to check how to remove this bug.

% \usepackage[showboxes,overlay,absolute]{textpos}



% Setting scale length for package "textpos"
\setlength{\TPHorizModule}{10mm}
\setlength{\TPVertModule}{\TPHorizModule}


% ---------------------------------------
% Packages written for lectures "Geometria 3D dla twórców gier wideo"
% ---------------------------------------
% \usepackage{./Geometry3DPackages/Geometry3D}
% \usepackage{./Geometry3DPackages/Geometry3DIndices}
% \usepackage{./Geometry3DPackages/Geometry3DTikZStyle}
% \usepackage{./ProgramowanieSymulacjiFizykiPaczki/ProgramowanieSymulacjiFizykiTikZStyle}
% \usepackage{./Geometry3DPackages/mathcommands}


% ---------------------------------------
% TikZ
% ---------------------------------------
% Importing TikZ libraries
\usetikzlibrary{arrows.meta}
\usetikzlibrary{positioning}





% % Configuration package "bm" that need for making bold symbols
% \newcommand{\bmmax}{0}
% \newcommand{\hmmax}{0}
% \usepackage{bm}




% ---------------------------------------
% Packages for scientific texts
% ---------------------------------------
% \let\lll\undefined  % Sometimes you must use this line to allow
% "amsmath" package to works with packages with packages for polish
% languge imported
% /preambul/LanguageSettings/JagiellonianPolishLanguageSettings.tex.
% This comments (probably) removes polish letter Ł.
\usepackage{amsmath}  % Packages from American Mathematical Society (AMS)
\usepackage{amssymb}
\usepackage{amscd}
\usepackage{amsthm}
\usepackage{siunitx}  % Package for typsetting SI units.
\usepackage{upgreek}  % Better looking greek letters.
% Example of using upgreek: pi = \uppi


\usepackage{calrsfs}  % Zmienia czcionkę kaligraficzną w \mathcal
% na ładniejszą. Może w innych miejscach robi to samo, ale o tym nic
% nie wiem.










% ---------------------------------------
% Packages written for lectures "Geometria 3D dla twórców gier wideo"
% ---------------------------------------
% \usepackage{./ProgramowanieSymulacjiFizykiPaczki/ProgramowanieSymulacjiFizyki}
% \usepackage{./ProgramowanieSymulacjiFizykiPaczki/ProgramowanieSymulacjiFizykiIndeksy}
% \usepackage{./ProgramowanieSymulacjiFizykiPaczki/ProgramowanieSymulacjiFizykiTikZStyle}





% !!!!!!!!!!!!!!!!!!!!!!!!!!!!!!
% !!!!!!!!!!!!!!!!!!!!!!!!!!!!!!
% EVIL STUFF
\if\JUlogotitle1
\edef\LogoJUPath{LogoJU_\JUlogoLang/LogoJU_\JUlogoShape_\JUlogoColor.pdf}
\titlegraphic{\hfill\includegraphics[scale=0.22]
{./JagiellonianPictures/\LogoJUPath}}
\fi
% ---------------------------------------
% Commands for handling colors
% ---------------------------------------


% Command for setting normal text color for some text in math modestyle
% Text color depend on used style of Jagiellonian

% Beamer version of command
\newcommand{\TextWithNormalTextColor}[1]{%
  {\color{jNormalTextFGColor}
    \setbeamercolor{math text}{fg=jNormalTextFGColor} {#1}}
}

% Article and similar classes version of command
% \newcommand{\TextWithNormalTextColor}[1]{%
%   {\color{jNormalTextsFGColor} {#1}}
% }



% Beamer version of command
\newcommand{\NormalTextInMathMode}[1]{%
  {\color{jNormalTextFGColor}
    \setbeamercolor{math text}{fg=jNormalTextFGColor} \text{#1}}
}


% Article and similar classes version of command
% \newcommand{\NormalTextInMathMode}[1]{%
%   {\color{jNormalTextsFGColor} \text{#1}}
% }




% Command that sets color of some mathematical text to the same color
% that has normal text in header (?)

% Beamer version of the command
\newcommand{\MathTextFrametitleFGColor}[1]{%
  {\color{jFrametitleFGColor}
    \setbeamercolor{math text}{fg=jFrametitleFGColor} #1}
}

% Article and similar classes version of the command
% \newcommand{\MathTextWhiteColor}[1]{{\color{jFrametitleFGColor} #1}}





% Command for setting color of alert text for some text in math modestyle

% Beamer version of the command
\newcommand{\MathTextAlertColor}[1]{%
  {\color{jOrange} \setbeamercolor{math text}{fg=jOrange} #1}
}

% Article and similar classes version of the command
% \newcommand{\MathTextAlertColor}[1]{{\color{jOrange} #1}}





% Command that allow you to sets chosen color as the color of some text into
% math mode. Due to some nuances in the way that Beamer handle colors
% it not work in all cases. We hope that in the future we will improve it.

% Beamer version of the command
\newcommand{\SetMathTextsColor}[2]{%
  {\color{#1} \setbeamercolor{math text}{fg=#1} #2}
}


% Article and similar classes version of the command
% \newcommand{\SetMathTextColor}[2]{{\color{#1} #2}}










% ---------------------------------------
% Commands for setting background pictures for some slides
% ---------------------------------------
\newcommand{\TitleBackgroundPicture}
{./PresentationPictures/CommonPictures/Cute_dragon_BG_dark.png}
\newcommand{\SectionBackgroundPicture}
{./PresentationPictures/CommonPictures/Cute_dragon_small_BG_light.png}



\newcommand{\TitleSlideWithPicture}{
  \begingroup

  \usebackgroundtemplate{ % \hspace*{-11.5em}
    \includegraphics[height=\paperheight]{\TitleBackgroundPicture}}

  \maketitle

  \endgroup
}





\newcommand{\SectionSlideWithPicture}[1]{%
  \begingroup

  \usebackgroundtemplate{ % \hspace*{-11.5em}
    \includegraphics[height=\paperheight]{\SectionBackgroundPicture}}

  \setbeamercolor{titlelike}{fg=normal text.fg}

  \section{#1}

  \endgroup
}





\newcommand{\EndingSlide}[1]{%
  \begin{frame}[standout]

    \begingroup

    \color{jFrametitleFGColor}

    #1

    \endgroup

  \end{frame}
}










% ------------------------------------------------------
% BibLaTeX
% ------------------------------------------------------
% Package biblatex, with biber as its backend, allow us to handle
% bibliography entries that use Unicode symbols outside ASCII.
\usepackage[
language=polish,
backend=biber,
style=alphabetic,
url=false,
eprint=true,
]{biblatex}

\addbibresource{Systemy-operacyjne-Bibliography.bib}





% ------------------------------------------------------
% Importing packages, libraries and setting their configuration
% ------------------------------------------------------





% ------------------------------------------------------
% Local packages
% ------------------------------------------------------
% Local configuration of this particular presentation
\usepackage{./Local-packages/local-settings}










% ------------------------------------------------------------------------------------------------------------------
\title{Systemy operacyjne}
\subtitle{Grep???}

\author{Kamil Ziemian \\
  \email}


% \date{}
% ------------------------------------------------------------------------------------------------------------------










% ####################################################################
% Beginning of the document
\begin{document}
% ####################################################################





% ######################################
% Number of chars:
% Text is adjusted to the left and words are broken at the end of the line.
\RaggedRight
% ######################################





% ######################################
\maketitle
% ######################################





% ##################
\begin{frame}
  \frametitle{Spis treści}


  \tableofcontents

\end{frame}
% ##################





% ######################################
\section{Grep i~wyrażenia regularne}
% ######################################


% ##################
\begin{frame}
  \frametitle{Dlaczego poświęcamy uwagę grepowi?}


  Polecenie \texttt{grep} nie należy do najbardziej intuicyjnych w~użyciu,
  ale~potrafi być przydatne. Oprócz tego jest dobrą ilustrują z~jednej
  strony potęgi jaką oferuje nam system GNU/Linux, z~drugiej to ile
  trudu jej zdobycie zwykle kosztuje. Oprócz tego pozwoli nam zdobyć
  pewne umiejętności przy korzystaniu
  z~\colorhref{https://en.wikipedia.org/wiki/Regular\_expression}{wyrażeniami
    regularnymi} („regex” lub „regexp”, ang.~\textit{regular expression}).

  Wyrażenia regularne, można postrzegać jako ulepszoną wersję wyszukiwania
  tekstu w~dokumencie, czyli słynnego \texttt{Ctrl-f}. W~najprostszym
  wypadku, działają dokładnie tak jak \texttt{Ctrl-f}: przesyłamy im
  jakiś napis i~znajdują one jego wystąpienia w~tekście. Różnica polega na
  tym, że~za pomocą specjalnej składni, możemy im szukać bardziej złożonych
  ciągów symboli, niż po prostu podany tekst, jednak do tego przejdziemy
  w~swoim czasie.

\end{frame}
% ##################





% ##################
\begin{frame}
  \frametitle{Problemy z~wyrażeniami regularnymi}


  Wyrażenia regularne mają dwie podstawowe wady. Po pierwsze są trudne
  i~w~pisaniu, a~jeszcze bardziej w~czytaniu. Po~drugie, istnieją dziesiątki
  sposobów notowania wyrażeń regularny, które służą do tego samego, ale mają
  inną składnię. W~połączeniu z~ich nieczytelnością, sprawia to,
  iż~korzystanie z~nich wymaga zwykle bardzo dużo pracy z~naszej strony,
  a~błędy popełnić jest bardzo łatwo. Warto tutaj przytoczyć jedną
  z~mądrości informatyki.

  \textit{Pewien programista miał problem. Stwierdził „Użyję wyrażeń
    regularnych.” Teraz ma dwa problemy.}

  Choć wyrażenia regularne potrafią być bardzo użyteczne, to jednak nie
  należy ich stosować, jeśli dany problem da~się rozwiązać w~sposób
  prostszy. Nie dotyczy to oczywiście najprostszych wyrażeń regularnych,
  które można stosować równie często jak \texttt{Ctrl-f}.

\end{frame}
% ##################





% ##################
\begin{frame}
  \frametitle{Mem dla ilustracji problemu}

  \vspace{-0.5em}


  \begin{figure}

    \label{fig:How-to-regex}

    \centering


    \includegraphics[scale=0.18]
    {./Presentations-pictures/Miscancellous-pictures/How-to-regex.jpg}

  \end{figure}

\end{frame}
% ##################





% ##################
\begin{frame}
  \frametitle{Co ja robię?}

  \vspace{-0.5em}


  \begin{figure}

    \label{fig:How-to-regex}

    \centering


    \includegraphics[scale=0.52]
    {./Presentations-pictures/Miscancellous-pictures/Lost-in-regex.png}

  \end{figure}

\end{frame}
% ##################





% ##################
\begin{frame}
  \frametitle{Praca z~\texttt{grep}em}


  \textit{I~have a natural revulsion to any operating system that shows so
    little plannning as to have to named all of its commands after digestive
    noises (awk, grep, fsck, nroff).} \\
    Autor nieznany, str.~$147$,
    \parencite{Garfinkel-Weise-Strassmann-The-UNIX-HATERS-Handbook-Pub-1994}.

  Nazwa polecenia \texttt{grep} pochodzi od~\textit{Global REgular
    expression Print}, czyli na polski to byłoby coś w~stylu „globalny
  wypisywacz wyrażeń regularnych”. Jest to niezwykle rozbudowane polecenie
  i~moglibyśmy spędzić kilka zajęć na pracy z~nim, na~tym przedmiocie
  ograniczymy~się do stosunkowo podstawowych przypadków. Więcej informacji
  o~\texttt{grep}ie można znaleźć choćby tutaj
  \parencite{Pankaj-Wailia-Mastering-Grep-command-in-Linux-Unix-ETC-Ver-2025}.

  W~katalogu \texttt{Materiały-do-prezentacji\textbackslash
    SO-XX-Grep\textbackslash} można znaleźć plik \textit{Lorem-ipsum.txt}
  na którym będziemy pracować. Zawarty w~nim tekst ten został wygenerowany
  za pomocą strony \textit{Lorem ipsum}
  \colorhref{https://www.lipsum.com/}{https://www.lipsum.com/}.

\end{frame}
% ##################





% ##################
\begin{frame}
  \frametitle{Praca z~\texttt{grep}em}

  Teksty typu \textit{Lorem ipsum} są używany w~przemyśle wydawniczym od
  XVI wieku, jak tekst do~testowania różnych układów strony, więc czemu by
  nie użyć ich i~tutaj?

  Załóżmy, że~jesteśmy w~katalogu, gdzie znajduje~się plik
  \textit{Lorem-ipsum.txt}. Zaczniemy od najprostszego użycia polecenia
  \texttt{grep}, będzie nim \\
  \texttt{\$ grep "in" Lorem-ipsum.txt} \\
  Od razu należy zaznaczyć, że~polecenie \texttt{grep} potrzebuje mnie dwa
  argumenty: najpierw tekst który ma wyszukać, potem miejsce, gdzie go ma
  wyszukać. (O~pewnych subtelnościach tego związanych z~przekierowaniami,
  powiem sobie później.) Jeśli wprowadzimy za mało argumentów, to powłoka
  może zacząć~się dziwnie zachowywać, wtedy należy spróbować użyć
  pojedynczego \texttt{Ctrl-d} lub poprosić nas o~pomoc. Od~tego jesteśmy.

\end{frame}
% ##################





% ##################
\begin{frame}
  \frametitle{Praca z~\texttt{grep}em}


  Przyjrzyjmy~się rezultatowi tego polecenia. \\
  \texttt{\$ grep "in"{} Lorem-ipsum.txt} \\
  \texttt{Lorem ipsum dolor sit amet, consectetur
    adipisc{\color{red} in}g elit. Integer at est erat.} \\
  \texttt{vitae. In ultrices turpis et arcu venenatis, at feugiat elit
    lac{\color{red} in}ia. Quisque} \\
  \texttt{t{\color{red} in}cidunt vitae massa euismod maximus. Vestibulum
    sed eros vel arcu iaculis} \\
  \hspace{1em} $\vdots$ \\
  Resztę linii wyniku pominęliśmy dla przejrzystość. Jak widzimy,
  \texttt{grep} wyszukuje wszystkie linie w~których znajduje~się ciąg
  symboli „in”, niezależnie czy jest to pojedyncze słowo, czy część
  większego i~wyróżnia znaleziony fragment odpowiednim kolorem. Jeśli
  znaleziony tekst nie wyświetla~się komuś na kolorowo, proszę spróbować
  polecenia \\
  \texttt{\$ grep --color "in"{} Lorem-ipsum.txt}

\end{frame}
% ##################





% ##################
\begin{frame}
  \frametitle{Praca z~\texttt{grep}em}


  Wynik staje~się łatwiejszy do czytania, jeśli do~\texttt{grep}a prześlemy
  flagę \texttt{-n}, która sprawia, że~polecenie wypisuje numery
  odpowiednich wierszy z~pliku: \\
  \texttt{\$ grep -n "in"{} Lorem-ipsum.txt} \\
  \texttt{{\color{green} 1:}Lorem ipsum dolor sit amet, consectetur
    adipisc{\color{red} in}g elit. Integer at est erat.} \\
  \texttt{{\color{green}6:}vitae. In ultrices turpis et arcu venenatis, at
    feugiat elit lac{\color{red} in}ia. Quisque} \\
  \texttt{{\color{green}11:}t{\color{red} in}cidunt vitae massa euismod
    maximus. Vestibulum sed eros vel arcu iaculis} \\
  \hspace{1em} $\vdots$

  W~dalszych przykładach będziemy standardowo dodawać flagę~\texttt{-n}, by
  uczynić zwracane rezultaty prostszymi w~czytaniu.

\end{frame}
% ##################





% ##################
\begin{frame}
  \frametitle{Wyszukiwanie całych wyrazów}


  Jak widzieliśmy, normalnie \texttt{grep} wyszukuje ciąg symboli,
  niezależnie, czy tworzą one w~tekście osoby wyraz, czy nie. Żeby
  wyszukać tylko te wiersze, w~których dany ciąg symboli pojawia~się
  tylko jako osobne słowo, używamy flagi~\texttt{-w}: \\
  \texttt{\$ grep -n -w "in"{} Lorem-ipsum.txt} \\
  \texttt{{\color{green} 14:}quis interdum. Integer at orci at neque
    pulvinar pretium. Donec {\color{red} in} elementum} \\
  \texttt{{\color{green} 24:}massa et rhoncus aliquam. Suspendisse rhoncus
    nunc {\color{red} in} finibus mattis.} \\
  \texttt{{\color{green}37:}vel aliquet. Proin {\color{red} in} hendrerit
    elit, a finibus ipsum. Integer sollicitudin} \\
  \hspace{1em} $\vdots$ \\

  % Tak jak wyjaśniliśmy wcześniej, flagi tego typu można łączyć, tak~że
  % polecenia: \\
  % \texttt{\$ grep -nw "in"{} Lorem-ipsum.txt} \\
  % \texttt{\$ grep -wn "in"{} Lorem-ipsum.txt} \\
  % znaczą to samo. Jednak dla podkreślenia, że~testujemy tylko flagę
  % \texttt{-w} umieściliśmy ją osobno. Tą samą konwencję będziemy stosować
  % również dalej.

\end{frame}
% ##################





% ##################
\begin{frame}
  \frametitle{Wyszukiwanie całych wyrazów}


  Tak jak wyjaśniliśmy wcześniej, flagi tego typu można odpowiednio łączyć,
  tak~że polecenia: \\
  \texttt{\$ grep -nw "in"{} Lorem-ipsum.txt} \\
  \texttt{\$ grep -wn "in"{} Lorem-ipsum.txt} \\
  znaczą to samo. Jednak dla podkreślenia, że~testujemy tylko flagę
  \texttt{-w} umieściliśmy ją osobno. Tą samą konwencję będziemy stosować
  również dalej.

\end{frame}
% ##################





% ##################
\begin{frame}
  \frametitle{Duże i~małe litery}


  Domyślnie \texttt{grep} uwzględnia w~wyszukiwaniu wielkość liter, co
  obrazują poniższe przykłady. \\
  \texttt{\$ grep -n "Du"{} Lorem-ipsum.txt} \\
  \texttt{{\color{green} 32:}{\color{red} Du}is fringilla dictum leo id
    interdum. Ut sit amet sem a nunc porttitor} \\
  \hspace{1em} $\vdots$ \\
  \texttt{\$ grep -n "du"{} Lorem-ipsum.txt} \\
  \texttt{{\color{green} 8:}consectetur a sapien vitae, euismod
    biben{\color{red} du}m eros.} \\
  \hspace{1em} $\vdots$

  By~\texttt{grep} ignorował różnicę między małymi i~dużymi literami,
  należy mu przesłać flagę \texttt{-i}: \\
  \texttt{\$ grep -n -i "Du"{} Lorem-ipsum.txt} \\
  \texttt{{\color{green} 8:}consectetur a sapien vitae, euismod
    biben{\color{red} du}m eros.} \\
  \hspace{1em} $\vdots$

\end{frame}
% ##################





% ##################
\begin{frame}
  \frametitle{Zliczanie dopasowany linii}


  By~\texttt{grep} wypisał w~jakiej liczbie linii został znaleziony zadany
  ciąg symboli (do ilu linii został dopasowany wzorzec) dajemy mu
  flagę~\texttt{-c}: \\
  \texttt{\$ grep -n -c "Du"{} Lorem-ipsum.txt} \\
  \texttt{$6$}

\end{frame}
% ##################





% % ##################
% \begin{frame}
%   \frametitle{}




% \end{frame}
% % ##################










% % ######################################
% \section{Pliki i~procesy}
% % ######################################



% % ##################
% \begin{frame}
%   \frametitle{}



% \end{frame}
% % ##################





% % % ##################
% % \begin{frame}
% %   \frametitle{Rozwój GNU/Linuxa}

% %   \vspace{-0.5em}


% %   \begin{figure}

% %     \label{fig:Evolution-of-OS}

% %     \centering


% %     \includegraphics[scale=0.3]
% %     {./Presentations-pictures/Miscancellous-pictures/Evolution-of-operating-systems.jpg}

% %   \end{figure}

% % \end{frame}
% % % ##################





% % % ##################
% % \begin{frame}
% %   \frametitle{}




% % \end{frame}
% % % ##################





% % % ##################
% % \begin{frame}
% %   \frametitle{}



% % \end{frame}
% % % ##################










% % % ######################################
% % \section{Podstawy GNU/Linuxa: włączanie powłoki}
% % % ######################################


% % % ##################
% % \begin{frame}
% %   \frametitle{}


% % \end{frame}
% % % ##################





% % % ##################
% % \begin{frame}
% %   \frametitle{}



% % \end{frame}
% % % ##################





% % % ##################
% % \begin{frame}
% %   \frametitle{}



% % \end{frame}
% % % ##################










% % ######################################
% \section{}
% % ######################################


% % % ##################
% % \begin{frame}
% %   \frametitle{}


% % \end{frame}
% % % ##################





% % % ##################
% % \begin{frame}
% %   \frametitle{}


% % \end{frame}
% % % ##################





% % % ##################
% % \begin{frame}
% %   \frametitle{}




% % \end{frame}
% % % ##################





% % % ##################
% % \begin{frame}
% %   \frametitle{}



% % \end{frame}
% % % ##################





% % % ##################
% % \begin{frame}
% %   \frametitle{}




% % \end{frame}
% % % ##################





% % % ##################
% % \begin{frame}
% %   \frametitle{}



% % \end{frame}
% % % ##################










% % % ######################################
% % \section{}
% % % ######################################


% % % ##################
% % \begin{frame}
% %   \frametitle{}



% % \end{frame}
% % % ##################







% % % ######################################
% % \section{}
% % % ######################################


% % % ##################
% % \begin{frame}
% %   \frametitle{}




% % \end{frame}
% % % ##################





% % % ##################
% % \begin{frame}
% %   \frametitle{}



% % \end{frame}
% % % ##################





% % % ##################
% % \begin{frame}
% %   \frametitle{}



% % \end{frame}
% % % ##################





% % % ##################
% % \begin{frame}
% %   \frametitle{}



% % \end{frame}
% % % ##################





% % % ##################
% % \begin{frame}
% %   \frametitle{}


% % \end{frame}
% % % ##################










% % % ######################################
% % \section{}
% % % ######################################


% % % ##################
% % \begin{frame}
% %   \frametitle{}



% % \end{frame}
% % % ##################





% % % ##################
% % \begin{frame}
% %   \frametitle{}




% % \end{frame}
% % % ##################





% % % ##################
% % \begin{frame}
% %   \frametitle{}



% % \end{frame}
% % % ##################










% % % ######################################
% % \section{}
% % % ######################################


% % % ##################
% % \begin{frame}
% %   \frametitle{}



% % \end{frame}
% % % ##################





% % % ##################
% % \begin{frame}
% %   \frametitle{}



% % \end{frame}
% % % ##################







% % % ##################
% % \begin{frame}
% %   \frametitle{}



% % \end{frame}
% % % ##################





% % % ##################
% % \begin{frame}
% %   \frametitle{}




% % \end{frame}
% % % ##################





% % % ##################
% % \begin{frame}
% %   \frametitle{}



% % \end{frame}
% % % ##################





% % % ##################
% % \begin{frame}
% %   \frametitle{}



% % \end{frame}
% % % ##################





% % ##################
% \begin{frame}
%   \frametitle{????}




% \end{frame}
% % ##################





% % ##################
% \begin{frame}
%   \frametitle{????}




% \end{frame}
% % ##################





% % ##################
% \begin{frame}
%   \frametitle{????}




% \end{frame}
% % ##################





% % ##################
% \begin{frame}
%   \frametitle{????}




% \end{frame}
% % ##################






% % ##################
% \begin{frame}
%   \frametitle{?????}




% \end{frame}
% % ##################





% % ##################
% \begin{frame}
%   \frametitle{?????}




% \end{frame}
% % ##################





% % ##################
% \begin{frame}
%   \frametitle{?????}




% \end{frame}
% % ##################





% % ##################
% \begin{frame}
%   \frametitle{?????}



% \end{frame}
% % ##################





% % ##################
% \begin{frame}
%   \frametitle{?????}



% \end{frame}
% % ##################





% % ##################
% \begin{frame}
%   \frametitle{?????}



% \end{frame}
% % ##################





% % ##################
% \begin{frame}
%   \frametitle{????}



% \end{frame}
% % ##################





% % ##################
% \begin{frame}
%   \frametitle{?????}



% \end{frame}
% % ##################





% % ##################
% \begin{frame}
%   \frametitle{????}



% \end{frame}
% % ##################





% % ##################
% \begin{frame}
%   \frametitle{?????}



% \end{frame}
% % ##################





% % ##################
% \begin{frame}
%   \frametitle{?????}



% \end{frame}
% % ##################





% % ##################
% \begin{frame}
%   \frametitle{????}



% \end{frame}
% % ##################





% % ##################
% \begin{frame}
%   \frametitle{????}



% \end{frame}
% % ##################





% % ##################
% \begin{frame}
%   \frametitle{?????}




% \end{frame}
% % ##################





% % ##################
% \begin{frame}
%   \frametitle{?????}




% \end{frame}
% % ##################





% % ##################
% \begin{frame}
%   \frametitle{?????}



% \end{frame}
% % ##################





% % ##################
% \begin{frame}
%   \frametitle{????}




% \end{frame}
% % ##################





% % ##################
% \begin{frame}
%   \frametitle{????}




% \end{frame}
% % ##################





% % ##################
% \begin{frame}
%   \frametitle{????}




% \end{frame}
% % ##################










% ######################################
\appendix
% ######################################





% ######################################
\EndingSlide{Dziękuję! Pytania?}
% ######################################










% ####################################################################
% ####################################################################
% Bibliography

\printbibliography





% ####################################################################
% End of the document

\end{document}
