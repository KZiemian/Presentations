% ------------------------------------------------------------------------------------------------------------------
% Basic configuration of Beamera class and Jagiellonian theme
% ------------------------------------------------------------------------------------------------------------------
\RequirePackage[l2tabu, orthodox]{nag}



\ifx\PresentationStyle\notset
  \def\PresentationStyle{light}
\fi



% Options: t - align frame text to the top.
\documentclass[10pt,t]{beamer}
\mode<presentation>
\usetheme[style=\PresentationStyle,JUlogotitle=no]{jagiellonian}




% ------------------------------------------------------------------------------------
% Procesing configuration files of Jagiellonian theme located
% in the directory "preambule"
% ------------------------------------------------------------------------------------
% Configuration for polish language
% Need description
\usepackage[polish]{babel}
% Need description
\usepackage[MeX]{polski}



% ------------------------------
% Better support of polish chars in technical parts of PDF
% ------------------------------
\hypersetup{pdfencoding=auto,psdextra}

% Package "textpos" give as enviroment "textblock" which is very usefull in
% arranging text on slides.

% This is standard configuration of "textpos"
\usepackage[overlay,absolute]{textpos}

% If you need to see bounds of "textblock's" comment line above and uncomment
% one below.

% Caution! When showboxes option is on significant ammunt of space is add
% to the top of textblock and as such, everyting put in them gone down.
% We need to check how to remove this bug.

% \usepackage[showboxes,overlay,absolute]{textpos}



% Setting scale length for package "textpos"
\setlength{\TPHorizModule}{10mm}
\setlength{\TPVertModule}{\TPHorizModule}


% ---------------------------------------
% Packages written for lectures "Geometria 3D dla twórców gier wideo"
% ---------------------------------------
% \usepackage{./Geometry3DPackages/Geometry3D}
% \usepackage{./Geometry3DPackages/Geometry3DIndices}
% \usepackage{./Geometry3DPackages/Geometry3DTikZStyle}
% \usepackage{./ProgramowanieSymulacjiFizykiPaczki/ProgramowanieSymulacjiFizykiTikZStyle}
% \usepackage{./Geometry3DPackages/mathcommands}


% ---------------------------------------
% TikZ
% ---------------------------------------
% Importing TikZ libraries
\usetikzlibrary{arrows.meta}
\usetikzlibrary{positioning}





% % Configuration package "bm" that need for making bold symbols
% \newcommand{\bmmax}{0}
% \newcommand{\hmmax}{0}
% \usepackage{bm}




% ---------------------------------------
% Packages for scientific texts
% ---------------------------------------
% \let\lll\undefined  % Sometimes you must use this line to allow
% "amsmath" package to works with packages with packages for polish
% languge imported
% /preambul/LanguageSettings/JagiellonianPolishLanguageSettings.tex.
% This comments (probably) removes polish letter Ł.
\usepackage{amsmath}  % Packages from American Mathematical Society (AMS)
\usepackage{amssymb}
\usepackage{amscd}
\usepackage{amsthm}
\usepackage{siunitx}  % Package for typsetting SI units.
\usepackage{upgreek}  % Better looking greek letters.
% Example of using upgreek: pi = \uppi


\usepackage{calrsfs}  % Zmienia czcionkę kaligraficzną w \mathcal
% na ładniejszą. Może w innych miejscach robi to samo, ale o tym nic
% nie wiem.










% ---------------------------------------
% Packages written for lectures "Geometria 3D dla twórców gier wideo"
% ---------------------------------------
% \usepackage{./ProgramowanieSymulacjiFizykiPaczki/ProgramowanieSymulacjiFizyki}
% \usepackage{./ProgramowanieSymulacjiFizykiPaczki/ProgramowanieSymulacjiFizykiIndeksy}
% \usepackage{./ProgramowanieSymulacjiFizykiPaczki/ProgramowanieSymulacjiFizykiTikZStyle}





% !!!!!!!!!!!!!!!!!!!!!!!!!!!!!!
% !!!!!!!!!!!!!!!!!!!!!!!!!!!!!!
% EVIL STUFF
\if\JUlogotitle1
\edef\LogoJUPath{LogoJU_\JUlogoLang/LogoJU_\JUlogoShape_\JUlogoColor.pdf}
\titlegraphic{\hfill\includegraphics[scale=0.22]
{./JagiellonianPictures/\LogoJUPath}}
\fi
% ---------------------------------------
% Commands for handling colors
% ---------------------------------------


% Command for setting normal text color for some text in math modestyle
% Text color depend on used style of Jagiellonian

% Beamer version of command
\newcommand{\TextWithNormalTextColor}[1]{%
  {\color{jNormalTextFGColor}
    \setbeamercolor{math text}{fg=jNormalTextFGColor} {#1}}
}

% Article and similar classes version of command
% \newcommand{\TextWithNormalTextColor}[1]{%
%   {\color{jNormalTextsFGColor} {#1}}
% }



% Beamer version of command
\newcommand{\NormalTextInMathMode}[1]{%
  {\color{jNormalTextFGColor}
    \setbeamercolor{math text}{fg=jNormalTextFGColor} \text{#1}}
}


% Article and similar classes version of command
% \newcommand{\NormalTextInMathMode}[1]{%
%   {\color{jNormalTextsFGColor} \text{#1}}
% }




% Command that sets color of some mathematical text to the same color
% that has normal text in header (?)

% Beamer version of the command
\newcommand{\MathTextFrametitleFGColor}[1]{%
  {\color{jFrametitleFGColor}
    \setbeamercolor{math text}{fg=jFrametitleFGColor} #1}
}

% Article and similar classes version of the command
% \newcommand{\MathTextWhiteColor}[1]{{\color{jFrametitleFGColor} #1}}





% Command for setting color of alert text for some text in math modestyle

% Beamer version of the command
\newcommand{\MathTextAlertColor}[1]{%
  {\color{jOrange} \setbeamercolor{math text}{fg=jOrange} #1}
}

% Article and similar classes version of the command
% \newcommand{\MathTextAlertColor}[1]{{\color{jOrange} #1}}





% Command that allow you to sets chosen color as the color of some text into
% math mode. Due to some nuances in the way that Beamer handle colors
% it not work in all cases. We hope that in the future we will improve it.

% Beamer version of the command
\newcommand{\SetMathTextColor}[2]{%
  {\color{#1} \setbeamercolor{math text}{fg=#1} #2}
}


% Article and similar classes version of the command
% \newcommand{\SetMathTextColor}[2]{{\color{#1} #2}}










% ---------------------------------------
% Commands for few special slides
% ---------------------------------------
\newcommand{\EndingSlide}[1]{%
  \begin{frame}[standout]

    \begingroup

    \color{jFrametitleFGColor}

    #1

    \endgroup

  \end{frame}
}










% ---------------------------------------
% Commands for setting background pictures for some slides
% ---------------------------------------
\newcommand{\TitleBackgroundPicture}
{./JagiellonianPictures/Backgrounds/LajkonikDark.png}
\newcommand{\SectionBackgroundPicture}
{./JagiellonianPictures/Backgrounds/LajkonikLight.png}



\newcommand{\TitleSlideWithPicture}{%
  \begingroup

  \usebackgroundtemplate{%
    \includegraphics[height=\paperheight]{\TitleBackgroundPicture}}

  \maketitle

  \endgroup
}





\newcommand{\SectionSlideWithPicture}[1]{%
  \begingroup

  \usebackgroundtemplate{%
    \includegraphics[height=\paperheight]{\SectionBackgroundPicture}}

  \setbeamercolor{titlelike}{fg=normal text.fg}

  \section{#1}

  \endgroup
}










% ---------------------------------------
% Commands for lectures "Geometria 3D dla twórców gier wideo"
% Polish version
% ---------------------------------------
% Komendy teraz wykomentowane były potrzebne, gdy loga były na niebieskim
% tle, nie na białym. A są na białym bo tego chcieli w biurze projektu.
% \newcommand{\FundingLogoWhitePicturePL}
% {./PresentationPictures/CommonPictures/logotypFundusze_biale_bez_tla2.pdf}
\newcommand{\FundingLogoColorPicturePL}
{./PresentationPictures/CommonPictures/European_Funds_color_PL.pdf}
% \newcommand{\EULogoWhitePicturePL}
% {./PresentationPictures/CommonPictures/logotypUE_biale_bez_tla2.pdf}
\newcommand{\EUSocialFundLogoColorPicturePL}
{./PresentationPictures/CommonPictures/EU_Social_Fund_color_PL.pdf}
% \newcommand{\ZintegrUJLogoWhitePicturePL}
% {./PresentationPictures/CommonPictures/zintegruj-logo-white.pdf}
\newcommand{\ZintegrUJLogoColorPicturePL}
{./PresentationPictures/CommonPictures/ZintegrUJ_color.pdf}
\newcommand{\JULogoColorPicturePL}
{./JagiellonianPictures/LogoJU_PL/LogoJU_A_color.pdf}





\newcommand{\GeometryThreeDSpecialBeginningSlidePL}{%
  \begin{frame}[standout]

    \begin{textblock}{11}(1,0.7)

      \begin{flushleft}

        \mdseries

        \footnotesize

        \color{jFrametitleFGColor}

        Materiał powstał w ramach projektu współfinansowanego ze środków
        Unii Europejskiej w ramach Europejskiego Funduszu Społecznego
        POWR.03.05.00-00-Z309/17-00.

      \end{flushleft}

    \end{textblock}





    \begin{textblock}{10}(0,2.2)

      \tikz \fill[color=jBackgroundStyleLight] (0,0) rectangle (12.8,-1.5);

    \end{textblock}


    \begin{textblock}{3.2}(1,2.45)

      \includegraphics[scale=0.3]{\FundingLogoColorPicturePL}

    \end{textblock}


    \begin{textblock}{2.5}(3.7,2.5)

      \includegraphics[scale=0.2]{\JULogoColorPicturePL}

    \end{textblock}


    \begin{textblock}{2.5}(6,2.4)

      \includegraphics[scale=0.1]{\ZintegrUJLogoColorPicturePL}

    \end{textblock}


    \begin{textblock}{4.2}(8.4,2.6)

      \includegraphics[scale=0.3]{\EUSocialFundLogoColorPicturePL}

    \end{textblock}

  \end{frame}
}



\newcommand{\GeometryThreeDTwoSpecialBeginningSlidesPL}{%
  \begin{frame}[standout]

    \begin{textblock}{11}(1,0.7)

      \begin{flushleft}

        \mdseries

        \footnotesize

        \color{jFrametitleFGColor}

        Materiał powstał w ramach projektu współfinansowanego ze środków
        Unii Europejskiej w ramach Europejskiego Funduszu Społecznego
        POWR.03.05.00-00-Z309/17-00.

      \end{flushleft}

    \end{textblock}





    \begin{textblock}{10}(0,2.2)

      \tikz \fill[color=jBackgroundStyleLight] (0,0) rectangle (12.8,-1.5);

    \end{textblock}


    \begin{textblock}{3.2}(1,2.45)

      \includegraphics[scale=0.3]{\FundingLogoColorPicturePL}

    \end{textblock}


    \begin{textblock}{2.5}(3.7,2.5)

      \includegraphics[scale=0.2]{\JULogoColorPicturePL}

    \end{textblock}


    \begin{textblock}{2.5}(6,2.4)

      \includegraphics[scale=0.1]{\ZintegrUJLogoColorPicturePL}

    \end{textblock}


    \begin{textblock}{4.2}(8.4,2.6)

      \includegraphics[scale=0.3]{\EUSocialFundLogoColorPicturePL}

    \end{textblock}

  \end{frame}





  \TitleSlideWithPicture
}



\newcommand{\GeometryThreeDSpecialEndingSlidePL}{%
  \begin{frame}[standout]

    \begin{textblock}{11}(1,0.7)

      \begin{flushleft}

        \mdseries

        \footnotesize

        \color{jFrametitleFGColor}

        Materiał powstał w ramach projektu współfinansowanego ze środków
        Unii Europejskiej w~ramach Europejskiego Funduszu Społecznego
        POWR.03.05.00-00-Z309/17-00.

      \end{flushleft}

    \end{textblock}





    \begin{textblock}{10}(0,2.2)

      \tikz \fill[color=jBackgroundStyleLight] (0,0) rectangle (12.8,-1.5);

    \end{textblock}


    \begin{textblock}{3.2}(1,2.45)

      \includegraphics[scale=0.3]{\FundingLogoColorPicturePL}

    \end{textblock}


    \begin{textblock}{2.5}(3.7,2.5)

      \includegraphics[scale=0.2]{\JULogoColorPicturePL}

    \end{textblock}


    \begin{textblock}{2.5}(6,2.4)

      \includegraphics[scale=0.1]{\ZintegrUJLogoColorPicturePL}

    \end{textblock}


    \begin{textblock}{4.2}(8.4,2.6)

      \includegraphics[scale=0.3]{\EUSocialFundLogoColorPicturePL}

    \end{textblock}





    \begin{textblock}{11}(1,4)

      \begin{flushleft}

        \mdseries

        \footnotesize

        \RaggedRight

        \color{jFrametitleFGColor}

        Treść niniejszego wykładu jest udostępniona na~licencji
        Creative Commons (\textsc{cc}), z~uzna\-niem autorstwa
        (\textsc{by}) oraz udostępnianiem na tych samych warunkach
        (\textsc{sa}). Rysunki i~wy\-kresy zawarte w~wykładzie są
        autorstwa dr.~hab.~Pawła Węgrzyna et~al. i~są dostępne
        na tej samej licencji, o~ile nie wskazano inaczej.
        W~prezentacji wykorzystano temat Beamera Jagiellonian,
        oparty na~temacie Metropolis Matthiasa Vogelgesanga,
        dostępnym na licencji \LaTeX{} Project Public License~1.3c
        pod adresem: \colorhref{https://github.com/matze/mtheme}
        {https://github.com/matze/mtheme}.

        Projekt typograficzny: Iwona Grabska-Gradzińska \\
        Skład: Kamil Ziemian;
        Korekta: Wojciech Palacz \\
        Modele: Dariusz Frymus, Kamil Nowakowski \\
        Rysunki i~wykresy: Kamil Ziemian, Paweł Węgrzyn, Wojciech Palacz

      \end{flushleft}

    \end{textblock}

  \end{frame}
}



\newcommand{\GeometryThreeDTwoSpecialEndingSlidesPL}[1]{%
  \begin{frame}[standout]


    \begin{textblock}{11}(1,0.7)

      \begin{flushleft}

        \mdseries

        \footnotesize

        \color{jFrametitleFGColor}

        Materiał powstał w ramach projektu współfinansowanego ze środków
        Unii Europejskiej w~ramach Europejskiego Funduszu Społecznego
        POWR.03.05.00-00-Z309/17-00.

      \end{flushleft}

    \end{textblock}





    \begin{textblock}{10}(0,2.2)

      \tikz \fill[color=jBackgroundStyleLight] (0,0) rectangle (12.8,-1.5);

    \end{textblock}


    \begin{textblock}{3.2}(1,2.45)

      \includegraphics[scale=0.3]{\FundingLogoColorPicturePL}

    \end{textblock}


    \begin{textblock}{2.5}(3.7,2.5)

      \includegraphics[scale=0.2]{\JULogoColorPicturePL}

    \end{textblock}


    \begin{textblock}{2.5}(6,2.4)

      \includegraphics[scale=0.1]{\ZintegrUJLogoColorPicturePL}

    \end{textblock}


    \begin{textblock}{4.2}(8.4,2.6)

      \includegraphics[scale=0.3]{\EUSocialFundLogoColorPicturePL}

    \end{textblock}





    \begin{textblock}{11}(1,4)

      \begin{flushleft}

        \mdseries

        \footnotesize

        \RaggedRight

        \color{jFrametitleFGColor}

        Treść niniejszego wykładu jest udostępniona na~licencji
        Creative Commons (\textsc{cc}), z~uzna\-niem autorstwa
        (\textsc{by}) oraz udostępnianiem na tych samych warunkach
        (\textsc{sa}). Rysunki i~wy\-kresy zawarte w~wykładzie są
        autorstwa dr.~hab.~Pawła Węgrzyna et~al. i~są dostępne
        na tej samej licencji, o~ile nie wskazano inaczej.
        W~prezentacji wykorzystano temat Beamera Jagiellonian,
        oparty na~temacie Metropolis Matthiasa Vogelgesanga,
        dostępnym na licencji \LaTeX{} Project Public License~1.3c
        pod adresem: \colorhref{https://github.com/matze/mtheme}
        {https://github.com/matze/mtheme}.

        Projekt typograficzny: Iwona Grabska-Gradzińska \\
        Skład: Kamil Ziemian;
        Korekta: Wojciech Palacz \\
        Modele: Dariusz Frymus, Kamil Nowakowski \\
        Rysunki i~wykresy: Kamil Ziemian, Paweł Węgrzyn, Wojciech Palacz

      \end{flushleft}

    \end{textblock}

  \end{frame}





  \begin{frame}[standout]

    \begingroup

    \color{jFrametitleFGColor}

    #1

    \endgroup

  \end{frame}
}



\newcommand{\GeometryThreeDSpecialEndingSlideVideoPL}{%
  \begin{frame}[standout]

    \begin{textblock}{11}(1,0.7)

      \begin{flushleft}

        \mdseries

        \footnotesize

        \color{jFrametitleFGColor}

        Materiał powstał w ramach projektu współfinansowanego ze środków
        Unii Europejskiej w~ramach Europejskiego Funduszu Społecznego
        POWR.03.05.00-00-Z309/17-00.

      \end{flushleft}

    \end{textblock}





    \begin{textblock}{10}(0,2.2)

      \tikz \fill[color=jBackgroundStyleLight] (0,0) rectangle (12.8,-1.5);

    \end{textblock}


    \begin{textblock}{3.2}(1,2.45)

      \includegraphics[scale=0.3]{\FundingLogoColorPicturePL}

    \end{textblock}


    \begin{textblock}{2.5}(3.7,2.5)

      \includegraphics[scale=0.2]{\JULogoColorPicturePL}

    \end{textblock}


    \begin{textblock}{2.5}(6,2.4)

      \includegraphics[scale=0.1]{\ZintegrUJLogoColorPicturePL}

    \end{textblock}


    \begin{textblock}{4.2}(8.4,2.6)

      \includegraphics[scale=0.3]{\EUSocialFundLogoColorPicturePL}

    \end{textblock}





    \begin{textblock}{11}(1,4)

      \begin{flushleft}

        \mdseries

        \footnotesize

        \RaggedRight

        \color{jFrametitleFGColor}

        Treść niniejszego wykładu jest udostępniona na~licencji
        Creative Commons (\textsc{cc}), z~uzna\-niem autorstwa
        (\textsc{by}) oraz udostępnianiem na tych samych warunkach
        (\textsc{sa}). Rysunki i~wy\-kresy zawarte w~wykładzie są
        autorstwa dr.~hab.~Pawła Węgrzyna et~al. i~są dostępne
        na tej samej licencji, o~ile nie wskazano inaczej.
        W~prezentacji wykorzystano temat Beamera Jagiellonian,
        oparty na~temacie Metropolis Matthiasa Vogelgesanga,
        dostępnym na licencji \LaTeX{} Project Public License~1.3c
        pod adresem: \colorhref{https://github.com/matze/mtheme}
        {https://github.com/matze/mtheme}.

        Projekt typograficzny: Iwona Grabska-Gradzińska;
        Skład: Kamil Ziemian \\
        Korekta: Wojciech Palacz;
        Modele: Dariusz Frymus, Kamil Nowakowski \\
        Rysunki i~wykresy: Kamil Ziemian, Paweł Węgrzyn, Wojciech Palacz \\
        Montaż: Agencja Filmowa Film \& Television Production~-- Zbigniew
        Masklak

      \end{flushleft}

    \end{textblock}

  \end{frame}
}





\newcommand{\GeometryThreeDTwoSpecialEndingSlidesVideoPL}[1]{%
  \begin{frame}[standout]

    \begin{textblock}{11}(1,0.7)

      \begin{flushleft}

        \mdseries

        \footnotesize

        \color{jFrametitleFGColor}

        Materiał powstał w ramach projektu współfinansowanego ze środków
        Unii Europejskiej w~ramach Europejskiego Funduszu Społecznego
        POWR.03.05.00-00-Z309/17-00.

      \end{flushleft}

    \end{textblock}





    \begin{textblock}{10}(0,2.2)

      \tikz \fill[color=jBackgroundStyleLight] (0,0) rectangle (12.8,-1.5);

    \end{textblock}


    \begin{textblock}{3.2}(1,2.45)

      \includegraphics[scale=0.3]{\FundingLogoColorPicturePL}

    \end{textblock}


    \begin{textblock}{2.5}(3.7,2.5)

      \includegraphics[scale=0.2]{\JULogoColorPicturePL}

    \end{textblock}


    \begin{textblock}{2.5}(6,2.4)

      \includegraphics[scale=0.1]{\ZintegrUJLogoColorPicturePL}

    \end{textblock}


    \begin{textblock}{4.2}(8.4,2.6)

      \includegraphics[scale=0.3]{\EUSocialFundLogoColorPicturePL}

    \end{textblock}





    \begin{textblock}{11}(1,4)

      \begin{flushleft}

        \mdseries

        \footnotesize

        \RaggedRight

        \color{jFrametitleFGColor}

        Treść niniejszego wykładu jest udostępniona na~licencji
        Creative Commons (\textsc{cc}), z~uzna\-niem autorstwa
        (\textsc{by}) oraz udostępnianiem na tych samych warunkach
        (\textsc{sa}). Rysunki i~wy\-kresy zawarte w~wykładzie są
        autorstwa dr.~hab.~Pawła Węgrzyna et~al. i~są dostępne
        na tej samej licencji, o~ile nie wskazano inaczej.
        W~prezentacji wykorzystano temat Beamera Jagiellonian,
        oparty na~temacie Metropolis Matthiasa Vogelgesanga,
        dostępnym na licencji \LaTeX{} Project Public License~1.3c
        pod adresem: \colorhref{https://github.com/matze/mtheme}
        {https://github.com/matze/mtheme}.

        Projekt typograficzny: Iwona Grabska-Gradzińska;
        Skład: Kamil Ziemian \\
        Korekta: Wojciech Palacz;
        Modele: Dariusz Frymus, Kamil Nowakowski \\
        Rysunki i~wykresy: Kamil Ziemian, Paweł Węgrzyn, Wojciech Palacz \\
        Montaż: Agencja Filmowa Film \& Television Production~-- Zbigniew
        Masklak

      \end{flushleft}

    \end{textblock}

  \end{frame}





  \begin{frame}[standout]


    \begingroup

    \color{jFrametitleFGColor}

    #1

    \endgroup

  \end{frame}
}










% ---------------------------------------
% Commands for lectures "Geometria 3D dla twórców gier wideo"
% English version
% ---------------------------------------
% \newcommand{\FundingLogoWhitePictureEN}
% {./PresentationPictures/CommonPictures/logotypFundusze_biale_bez_tla2.pdf}
\newcommand{\FundingLogoColorPictureEN}
{./PresentationPictures/CommonPictures/European_Funds_color_EN.pdf}
% \newcommand{\EULogoWhitePictureEN}
% {./PresentationPictures/CommonPictures/logotypUE_biale_bez_tla2.pdf}
\newcommand{\EUSocialFundLogoColorPictureEN}
{./PresentationPictures/CommonPictures/EU_Social_Fund_color_EN.pdf}
% \newcommand{\ZintegrUJLogoWhitePictureEN}
% {./PresentationPictures/CommonPictures/zintegruj-logo-white.pdf}
\newcommand{\ZintegrUJLogoColorPictureEN}
{./PresentationPictures/CommonPictures/ZintegrUJ_color.pdf}
\newcommand{\JULogoColorPictureEN}
{./JagiellonianPictures/LogoJU_EN/LogoJU_A_color.pdf}



\newcommand{\GeometryThreeDSpecialBeginningSlideEN}{%
  \begin{frame}[standout]

    \begin{textblock}{11}(1,0.7)

      \begin{flushleft}

        \mdseries

        \footnotesize

        \color{jFrametitleFGColor}

        This content was created as part of a project co-financed by the
        European Union within the framework of the European Social Fund
        POWR.03.05.00-00-Z309/17-00.

      \end{flushleft}

    \end{textblock}





    \begin{textblock}{10}(0,2.2)

      \tikz \fill[color=jBackgroundStyleLight] (0,0) rectangle (12.8,-1.5);

    \end{textblock}


    \begin{textblock}{3.2}(0.7,2.45)

      \includegraphics[scale=0.3]{\FundingLogoColorPictureEN}

    \end{textblock}


    \begin{textblock}{2.5}(4.15,2.5)

      \includegraphics[scale=0.2]{\JULogoColorPictureEN}

    \end{textblock}


    \begin{textblock}{2.5}(6.35,2.4)

      \includegraphics[scale=0.1]{\ZintegrUJLogoColorPictureEN}

    \end{textblock}


    \begin{textblock}{4.2}(8.4,2.6)

      \includegraphics[scale=0.3]{\EUSocialFundLogoColorPictureEN}

    \end{textblock}

  \end{frame}
}



\newcommand{\GeometryThreeDTwoSpecialBeginningSlidesEN}{%
  \begin{frame}[standout]

    \begin{textblock}{11}(1,0.7)

      \begin{flushleft}

        \mdseries

        \footnotesize

        \color{jFrametitleFGColor}

        This content was created as part of a project co-financed by the
        European Union within the framework of the European Social Fund
        POWR.03.05.00-00-Z309/17-00.

      \end{flushleft}

    \end{textblock}





    \begin{textblock}{10}(0,2.2)

      \tikz \fill[color=jBackgroundStyleLight] (0,0) rectangle (12.8,-1.5);

    \end{textblock}


    \begin{textblock}{3.2}(0.7,2.45)

      \includegraphics[scale=0.3]{\FundingLogoColorPictureEN}

    \end{textblock}


    \begin{textblock}{2.5}(4.15,2.5)

      \includegraphics[scale=0.2]{\JULogoColorPictureEN}

    \end{textblock}


    \begin{textblock}{2.5}(6.35,2.4)

      \includegraphics[scale=0.1]{\ZintegrUJLogoColorPictureEN}

    \end{textblock}


    \begin{textblock}{4.2}(8.4,2.6)

      \includegraphics[scale=0.3]{\EUSocialFundLogoColorPictureEN}

    \end{textblock}

  \end{frame}





  \TitleSlideWithPicture
}



\newcommand{\GeometryThreeDSpecialEndingSlideEN}{%
  \begin{frame}[standout]

    \begin{textblock}{11}(1,0.7)

      \begin{flushleft}

        \mdseries

        \footnotesize

        \color{jFrametitleFGColor}

        This content was created as part of a project co-financed by the
        European Union within the framework of the European Social Fund
        POWR.03.05.00-00-Z309/17-00.

      \end{flushleft}

    \end{textblock}





    \begin{textblock}{10}(0,2.2)

      \tikz \fill[color=jBackgroundStyleLight] (0,0) rectangle (12.8,-1.5);

    \end{textblock}


    \begin{textblock}{3.2}(0.7,2.45)

      \includegraphics[scale=0.3]{\FundingLogoColorPictureEN}

    \end{textblock}


    \begin{textblock}{2.5}(4.15,2.5)

      \includegraphics[scale=0.2]{\JULogoColorPictureEN}

    \end{textblock}


    \begin{textblock}{2.5}(6.35,2.4)

      \includegraphics[scale=0.1]{\ZintegrUJLogoColorPictureEN}

    \end{textblock}


    \begin{textblock}{4.2}(8.4,2.6)

      \includegraphics[scale=0.3]{\EUSocialFundLogoColorPictureEN}

    \end{textblock}





    \begin{textblock}{11}(1,4)

      \begin{flushleft}

        \mdseries

        \footnotesize

        \RaggedRight

        \color{jFrametitleFGColor}

        The content of this lecture is made available under a~Creative
        Commons licence (\textsc{cc}), giving the author the credits
        (\textsc{by}) and putting an obligation to share on the same terms
        (\textsc{sa}). Figures and diagrams included in the lecture are
        authored by Paweł Węgrzyn et~al., and are available under the same
        license unless indicated otherwise.\\ The presentation uses the
        Beamer Jagiellonian theme based on Matthias Vogelgesang’s
        Metropolis theme, available under license \LaTeX{} Project
        Public License~1.3c at: \colorhref{https://github.com/matze/mtheme}
        {https://github.com/matze/mtheme}.

        Typographic design: Iwona Grabska-Gradzińska \\
        \LaTeX{} Typesetting: Kamil Ziemian \\
        Proofreading: Wojciech Palacz,
        Monika Stawicka \\
        3D Models: Dariusz Frymus, Kamil Nowakowski \\
        Figures and charts: Kamil Ziemian, Paweł Węgrzyn, Wojciech Palacz

      \end{flushleft}

    \end{textblock}

  \end{frame}
}



\newcommand{\GeometryThreeDTwoSpecialEndingSlidesEN}[1]{%
  \begin{frame}[standout]


    \begin{textblock}{11}(1,0.7)

      \begin{flushleft}

        \mdseries

        \footnotesize

        \color{jFrametitleFGColor}

        This content was created as part of a project co-financed by the
        European Union within the framework of the European Social Fund
        POWR.03.05.00-00-Z309/17-00.

      \end{flushleft}

    \end{textblock}





    \begin{textblock}{10}(0,2.2)

      \tikz \fill[color=jBackgroundStyleLight] (0,0) rectangle (12.8,-1.5);

    \end{textblock}


    \begin{textblock}{3.2}(0.7,2.45)

      \includegraphics[scale=0.3]{\FundingLogoColorPictureEN}

    \end{textblock}


    \begin{textblock}{2.5}(4.15,2.5)

      \includegraphics[scale=0.2]{\JULogoColorPictureEN}

    \end{textblock}


    \begin{textblock}{2.5}(6.35,2.4)

      \includegraphics[scale=0.1]{\ZintegrUJLogoColorPictureEN}

    \end{textblock}


    \begin{textblock}{4.2}(8.4,2.6)

      \includegraphics[scale=0.3]{\EUSocialFundLogoColorPictureEN}

    \end{textblock}





    \begin{textblock}{11}(1,4)

      \begin{flushleft}

        \mdseries

        \footnotesize

        \RaggedRight

        \color{jFrametitleFGColor}

        The content of this lecture is made available under a~Creative
        Commons licence (\textsc{cc}), giving the author the credits
        (\textsc{by}) and putting an obligation to share on the same terms
        (\textsc{sa}). Figures and diagrams included in the lecture are
        authored by Paweł Węgrzyn et~al., and are available under the same
        license unless indicated otherwise.\\ The presentation uses the
        Beamer Jagiellonian theme based on Matthias Vogelgesang’s
        Metropolis theme, available under license \LaTeX{} Project
        Public License~1.3c at: \colorhref{https://github.com/matze/mtheme}
        {https://github.com/matze/mtheme}.

        Typographic design: Iwona Grabska-Gradzińska \\
        \LaTeX{} Typesetting: Kamil Ziemian \\
        Proofreading: Wojciech Palacz,
        Monika Stawicka \\
        3D Models: Dariusz Frymus, Kamil Nowakowski \\
        Figures and charts: Kamil Ziemian, Paweł Węgrzyn, Wojciech Palacz

      \end{flushleft}

    \end{textblock}

  \end{frame}





  \begin{frame}[standout]

    \begingroup

    \color{jFrametitleFGColor}

    #1

    \endgroup

  \end{frame}
}



\newcommand{\GeometryThreeDSpecialEndingSlideVideoVerOneEN}{%
  \begin{frame}[standout]

    \begin{textblock}{11}(1,0.7)

      \begin{flushleft}

        \mdseries

        \footnotesize

        \color{jFrametitleFGColor}

        This content was created as part of a project co-financed by the
        European Union within the framework of the European Social Fund
        POWR.03.05.00-00-Z309/17-00.

      \end{flushleft}

    \end{textblock}





    \begin{textblock}{10}(0,2.2)

      \tikz \fill[color=jBackgroundStyleLight] (0,0) rectangle (12.8,-1.5);

    \end{textblock}


    \begin{textblock}{3.2}(0.7,2.45)

      \includegraphics[scale=0.3]{\FundingLogoColorPictureEN}

    \end{textblock}


    \begin{textblock}{2.5}(4.15,2.5)

      \includegraphics[scale=0.2]{\JULogoColorPictureEN}

    \end{textblock}


    \begin{textblock}{2.5}(6.35,2.4)

      \includegraphics[scale=0.1]{\ZintegrUJLogoColorPictureEN}

    \end{textblock}


    \begin{textblock}{4.2}(8.4,2.6)

      \includegraphics[scale=0.3]{\EUSocialFundLogoColorPictureEN}

    \end{textblock}





    \begin{textblock}{11}(1,4)

      \begin{flushleft}

        \mdseries

        \footnotesize

        \RaggedRight

        \color{jFrametitleFGColor}

        The content of this lecture is made available under a Creative
        Commons licence (\textsc{cc}), giving the author the credits
        (\textsc{by}) and putting an obligation to share on the same terms
        (\textsc{sa}). Figures and diagrams included in the lecture are
        authored by Paweł Węgrzyn et~al., and are available under the same
        license unless indicated otherwise.\\ The presentation uses the
        Beamer Jagiellonian theme based on Matthias Vogelgesang’s
        Metropolis theme, available under license \LaTeX{} Project
        Public License~1.3c at: \colorhref{https://github.com/matze/mtheme}
        {https://github.com/matze/mtheme}.

        Typographic design: Iwona Grabska-Gradzińska;
        \LaTeX{} Typesetting: Kamil Ziemian \\
        Proofreading: Wojciech Palacz,
        Monika Stawicka \\
        3D Models: Dariusz Frymus, Kamil Nowakowski \\
        Figures and charts: Kamil Ziemian, Paweł Węgrzyn, Wojciech
        Palacz \\
        Film editing: Agencja Filmowa Film \& Television Production~--
        Zbigniew Masklak

      \end{flushleft}

    \end{textblock}

  \end{frame}
}



\newcommand{\GeometryThreeDSpecialEndingSlideVideoVerTwoEN}{%
  \begin{frame}[standout]

    \begin{textblock}{11}(1,0.7)

      \begin{flushleft}

        \mdseries

        \footnotesize

        \color{jFrametitleFGColor}

        This content was created as part of a project co-financed by the
        European Union within the framework of the European Social Fund
        POWR.03.05.00-00-Z309/17-00.

      \end{flushleft}

    \end{textblock}





    \begin{textblock}{10}(0,2.2)

      \tikz \fill[color=jBackgroundStyleLight] (0,0) rectangle (12.8,-1.5);

    \end{textblock}


    \begin{textblock}{3.2}(0.7,2.45)

      \includegraphics[scale=0.3]{\FundingLogoColorPictureEN}

    \end{textblock}


    \begin{textblock}{2.5}(4.15,2.5)

      \includegraphics[scale=0.2]{\JULogoColorPictureEN}

    \end{textblock}


    \begin{textblock}{2.5}(6.35,2.4)

      \includegraphics[scale=0.1]{\ZintegrUJLogoColorPictureEN}

    \end{textblock}


    \begin{textblock}{4.2}(8.4,2.6)

      \includegraphics[scale=0.3]{\EUSocialFundLogoColorPictureEN}

    \end{textblock}





    \begin{textblock}{11}(1,4)

      \begin{flushleft}

        \mdseries

        \footnotesize

        \RaggedRight

        \color{jFrametitleFGColor}

        The content of this lecture is made available under a Creative
        Commons licence (\textsc{cc}), giving the author the credits
        (\textsc{by}) and putting an obligation to share on the same terms
        (\textsc{sa}). Figures and diagrams included in the lecture are
        authored by Paweł Węgrzyn et~al., and are available under the same
        license unless indicated otherwise.\\ The presentation uses the
        Beamer Jagiellonian theme based on Matthias Vogelgesang’s
        Metropolis theme, available under license \LaTeX{} Project
        Public License~1.3c at: \colorhref{https://github.com/matze/mtheme}
        {https://github.com/matze/mtheme}.

        Typographic design: Iwona Grabska-Gradzińska;
        \LaTeX{} Typesetting: Kamil Ziemian \\
        Proofreading: Wojciech Palacz,
        Monika Stawicka \\
        3D Models: Dariusz Frymus, Kamil Nowakowski \\
        Figures and charts: Kamil Ziemian, Paweł Węgrzyn, Wojciech
        Palacz \\
        Film editing: IMAVI -- Joanna Kozakiewicz, Krzysztof Magda, Nikodem
        Frodyma

      \end{flushleft}

    \end{textblock}

  \end{frame}
}



\newcommand{\GeometryThreeDSpecialEndingSlideVideoVerThreeEN}{%
  \begin{frame}[standout]

    \begin{textblock}{11}(1,0.7)

      \begin{flushleft}

        \mdseries

        \footnotesize

        \color{jFrametitleFGColor}

        This content was created as part of a project co-financed by the
        European Union within the framework of the European Social Fund
        POWR.03.05.00-00-Z309/17-00.

      \end{flushleft}

    \end{textblock}





    \begin{textblock}{10}(0,2.2)

      \tikz \fill[color=jBackgroundStyleLight] (0,0) rectangle (12.8,-1.5);

    \end{textblock}


    \begin{textblock}{3.2}(0.7,2.45)

      \includegraphics[scale=0.3]{\FundingLogoColorPictureEN}

    \end{textblock}


    \begin{textblock}{2.5}(4.15,2.5)

      \includegraphics[scale=0.2]{\JULogoColorPictureEN}

    \end{textblock}


    \begin{textblock}{2.5}(6.35,2.4)

      \includegraphics[scale=0.1]{\ZintegrUJLogoColorPictureEN}

    \end{textblock}


    \begin{textblock}{4.2}(8.4,2.6)

      \includegraphics[scale=0.3]{\EUSocialFundLogoColorPictureEN}

    \end{textblock}





    \begin{textblock}{11}(1,4)

      \begin{flushleft}

        \mdseries

        \footnotesize

        \RaggedRight

        \color{jFrametitleFGColor}

        The content of this lecture is made available under a Creative
        Commons licence (\textsc{cc}), giving the author the credits
        (\textsc{by}) and putting an obligation to share on the same terms
        (\textsc{sa}). Figures and diagrams included in the lecture are
        authored by Paweł Węgrzyn et~al., and are available under the same
        license unless indicated otherwise.\\ The presentation uses the
        Beamer Jagiellonian theme based on Matthias Vogelgesang’s
        Metropolis theme, available under license \LaTeX{} Project
        Public License~1.3c at: \colorhref{https://github.com/matze/mtheme}
        {https://github.com/matze/mtheme}.

        Typographic design: Iwona Grabska-Gradzińska;
        \LaTeX{} Typesetting: Kamil Ziemian \\
        Proofreading: Wojciech Palacz,
        Monika Stawicka \\
        3D Models: Dariusz Frymus, Kamil Nowakowski \\
        Figures and charts: Kamil Ziemian, Paweł Węgrzyn, Wojciech
        Palacz \\
        Film editing: Agencja Filmowa Film \& Television Production~--
        Zbigniew Masklak \\
        Film editing: IMAVI -- Joanna Kozakiewicz, Krzysztof Magda, Nikodem
        Frodyma

      \end{flushleft}

    \end{textblock}

  \end{frame}
}



\newcommand{\GeometryThreeDTwoSpecialEndingSlidesVideoVerOneEN}[1]{%
  \begin{frame}[standout]

    \begin{textblock}{11}(1,0.7)

      \begin{flushleft}

        \mdseries

        \footnotesize

        \color{jFrametitleFGColor}

        This content was created as part of a project co-financed by the
        European Union within the framework of the European Social Fund
        POWR.03.05.00-00-Z309/17-00.

      \end{flushleft}

    \end{textblock}





    \begin{textblock}{10}(0,2.2)

      \tikz \fill[color=jBackgroundStyleLight] (0,0) rectangle (12.8,-1.5);

    \end{textblock}


    \begin{textblock}{3.2}(0.7,2.45)

      \includegraphics[scale=0.3]{\FundingLogoColorPictureEN}

    \end{textblock}


    \begin{textblock}{2.5}(4.15,2.5)

      \includegraphics[scale=0.2]{\JULogoColorPictureEN}

    \end{textblock}


    \begin{textblock}{2.5}(6.35,2.4)

      \includegraphics[scale=0.1]{\ZintegrUJLogoColorPictureEN}

    \end{textblock}


    \begin{textblock}{4.2}(8.4,2.6)

      \includegraphics[scale=0.3]{\EUSocialFundLogoColorPictureEN}

    \end{textblock}





    \begin{textblock}{11}(1,4)

      \begin{flushleft}

        \mdseries

        \footnotesize

        \RaggedRight

        \color{jFrametitleFGColor}

        The content of this lecture is made available under a Creative
        Commons licence (\textsc{cc}), giving the author the credits
        (\textsc{by}) and putting an obligation to share on the same terms
        (\textsc{sa}). Figures and diagrams included in the lecture are
        authored by Paweł Węgrzyn et~al., and are available under the same
        license unless indicated otherwise.\\ The presentation uses the
        Beamer Jagiellonian theme based on Matthias Vogelgesang’s
        Metropolis theme, available under license \LaTeX{} Project
        Public License~1.3c at: \colorhref{https://github.com/matze/mtheme}
        {https://github.com/matze/mtheme}.

        Typographic design: Iwona Grabska-Gradzińska;
        \LaTeX{} Typesetting: Kamil Ziemian \\
        Proofreading: Wojciech Palacz,
        Monika Stawicka \\
        3D Models: Dariusz Frymus, Kamil Nowakowski \\
        Figures and charts: Kamil Ziemian, Paweł Węgrzyn,
        Wojciech Palacz \\
        Film editing: Agencja Filmowa Film \& Television Production~--
        Zbigniew Masklak

      \end{flushleft}

    \end{textblock}

  \end{frame}





  \begin{frame}[standout]


    \begingroup

    \color{jFrametitleFGColor}

    #1

    \endgroup

  \end{frame}
}



\newcommand{\GeometryThreeDTwoSpecialEndingSlidesVideoVerTwoEN}[1]{%
  \begin{frame}[standout]

    \begin{textblock}{11}(1,0.7)

      \begin{flushleft}

        \mdseries

        \footnotesize

        \color{jFrametitleFGColor}

        This content was created as part of a project co-financed by the
        European Union within the framework of the European Social Fund
        POWR.03.05.00-00-Z309/17-00.

      \end{flushleft}

    \end{textblock}





    \begin{textblock}{10}(0,2.2)

      \tikz \fill[color=jBackgroundStyleLight] (0,0) rectangle (12.8,-1.5);

    \end{textblock}


    \begin{textblock}{3.2}(0.7,2.45)

      \includegraphics[scale=0.3]{\FundingLogoColorPictureEN}

    \end{textblock}


    \begin{textblock}{2.5}(4.15,2.5)

      \includegraphics[scale=0.2]{\JULogoColorPictureEN}

    \end{textblock}


    \begin{textblock}{2.5}(6.35,2.4)

      \includegraphics[scale=0.1]{\ZintegrUJLogoColorPictureEN}

    \end{textblock}


    \begin{textblock}{4.2}(8.4,2.6)

      \includegraphics[scale=0.3]{\EUSocialFundLogoColorPictureEN}

    \end{textblock}





    \begin{textblock}{11}(1,4)

      \begin{flushleft}

        \mdseries

        \footnotesize

        \RaggedRight

        \color{jFrametitleFGColor}

        The content of this lecture is made available under a Creative
        Commons licence (\textsc{cc}), giving the author the credits
        (\textsc{by}) and putting an obligation to share on the same terms
        (\textsc{sa}). Figures and diagrams included in the lecture are
        authored by Paweł Węgrzyn et~al., and are available under the same
        license unless indicated otherwise.\\ The presentation uses the
        Beamer Jagiellonian theme based on Matthias Vogelgesang’s
        Metropolis theme, available under license \LaTeX{} Project
        Public License~1.3c at: \colorhref{https://github.com/matze/mtheme}
        {https://github.com/matze/mtheme}.

        Typographic design: Iwona Grabska-Gradzińska;
        \LaTeX{} Typesetting: Kamil Ziemian \\
        Proofreading: Wojciech Palacz,
        Monika Stawicka \\
        3D Models: Dariusz Frymus, Kamil Nowakowski \\
        Figures and charts: Kamil Ziemian, Paweł Węgrzyn,
        Wojciech Palacz \\
        Film editing: IMAVI -- Joanna Kozakiewicz, Krzysztof Magda, Nikodem
        Frodyma

      \end{flushleft}

    \end{textblock}

  \end{frame}





  \begin{frame}[standout]


    \begingroup

    \color{jFrametitleFGColor}

    #1

    \endgroup

  \end{frame}
}



\newcommand{\GeometryThreeDTwoSpecialEndingSlidesVideoVerThreeEN}[1]{%
  \begin{frame}[standout]

    \begin{textblock}{11}(1,0.7)

      \begin{flushleft}

        \mdseries

        \footnotesize

        \color{jFrametitleFGColor}

        This content was created as part of a project co-financed by the
        European Union within the framework of the European Social Fund
        POWR.03.05.00-00-Z309/17-00.

      \end{flushleft}

    \end{textblock}





    \begin{textblock}{10}(0,2.2)

      \tikz \fill[color=jBackgroundStyleLight] (0,0) rectangle (12.8,-1.5);

    \end{textblock}


    \begin{textblock}{3.2}(0.7,2.45)

      \includegraphics[scale=0.3]{\FundingLogoColorPictureEN}

    \end{textblock}


    \begin{textblock}{2.5}(4.15,2.5)

      \includegraphics[scale=0.2]{\JULogoColorPictureEN}

    \end{textblock}


    \begin{textblock}{2.5}(6.35,2.4)

      \includegraphics[scale=0.1]{\ZintegrUJLogoColorPictureEN}

    \end{textblock}


    \begin{textblock}{4.2}(8.4,2.6)

      \includegraphics[scale=0.3]{\EUSocialFundLogoColorPictureEN}

    \end{textblock}





    \begin{textblock}{11}(1,4)

      \begin{flushleft}

        \mdseries

        \footnotesize

        \RaggedRight

        \color{jFrametitleFGColor}

        The content of this lecture is made available under a Creative
        Commons licence (\textsc{cc}), giving the author the credits
        (\textsc{by}) and putting an obligation to share on the same terms
        (\textsc{sa}). Figures and diagrams included in the lecture are
        authored by Paweł Węgrzyn et~al., and are available under the same
        license unless indicated otherwise. \\ The presentation uses the
        Beamer Jagiellonian theme based on Matthias Vogelgesang’s
        Metropolis theme, available under license \LaTeX{} Project
        Public License~1.3c at: \colorhref{https://github.com/matze/mtheme}
        {https://github.com/matze/mtheme}.

        Typographic design: Iwona Grabska-Gradzińska;
        \LaTeX{} Typesetting: Kamil Ziemian \\
        Proofreading: Leszek Hadasz, Wojciech Palacz,
        Monika Stawicka \\
        3D Models: Dariusz Frymus, Kamil Nowakowski \\
        Figures and charts: Kamil Ziemian, Paweł Węgrzyn,
        Wojciech Palacz \\
        Film editing: Agencja Filmowa Film \& Television Production~--
        Zbigniew Masklak \\
        Film editing: IMAVI -- Joanna Kozakiewicz, Krzysztof Magda, Nikodem
        Frodyma


      \end{flushleft}

    \end{textblock}

  \end{frame}





  \begin{frame}[standout]


    \begingroup

    \color{jFrametitleFGColor}

    #1

    \endgroup

  \end{frame}
}











% ------------------------------------------------------
% BibLaTeX
% ------------------------------------------------------
% Package biblatex, with biber as its backend, allow us to handle
% bibliography entries that use Unicode symbols outside ASCII.
\usepackage[
language=polish,
backend=biber,
style=alphabetic,
url=false,
eprint=true,
]{biblatex}

\addbibresource{Systemy-operacyjne-Bibliography.bib}





% ------------------------------------------------------
% Importing packages, libraries and setting their configuration
% ------------------------------------------------------





% ------------------------------------------------------
% Local packages
% ------------------------------------------------------
% Local configuration of this particular presentation
\usepackage{./Local-packages/local-settings}










% ------------------------------------------------------------------------------------------------------------------
\title{Systemy operacyjne}
\subtitle{Grep???}

\author{Kamil Ziemian \\
  \email}


% \date{}
% ------------------------------------------------------------------------------------------------------------------










% ####################################################################
% Beginning of the document
\begin{document}
% ####################################################################





% ######################################
% Number of chars: 20k+,
% Text is adjusted to the left and words are broken at the end of the line.
\RaggedRight
% ######################################





% ######################################
\maketitle
% ######################################





% ##################
\begin{frame}
  \frametitle{Spis treści}


  \tableofcontents

\end{frame}
% ##################





% ######################################
\section{Podstawy korzystania z~\texttt{grep}a}
% ######################################


% ##################
\begin{frame}
  \frametitle{Dlaczego poświęcamy uwagę \texttt{grep}owi?}


  Polecenie \texttt{grep} nie należy do najbardziej intuicyjnych w~użyciu,
  ale~potrafi być przydatne. Oprócz tego jest dobrą ilustrują z~jednej
  strony potęgi jaką oferuje nam system GNU/Linux, z~drugiej to ile
  trudu jej zdobycie zwykle kosztuje. Oprócz tego pozwoli nam zdobyć
  pewne umiejętności przy korzystaniu
  z~\colorhref{https://en.wikipedia.org/wiki/Regular\_expression}{wyrażeniami
    regularnymi} („regex” lub „regexp”, ang.~\textit{regular expression}).

  Wyrażenia regularne, można postrzegać jako ulepszoną wersję wyszukiwania
  tekstu w~dokumencie, czyli słynnego \texttt{Ctrl-f}. W~najprostszym
  wypadku, działają dokładnie tak jak \texttt{Ctrl-f}: przesyłamy im
  jakiś napis i~znajdują one jego wystąpienia w~tekście. Różnica polega na
  tym, że~za pomocą specjalnej składni, możemy im szukać bardziej złożonych
  ciągów symboli, niż po prostu konkretny tekst, do tego jednak przejdziemy
  po ogólnym omówieniu \texttt{grep}a.

\end{frame}
% ##################





% ##################
\begin{frame}
  \frametitle{Problemy z~wyrażeniami regularnymi}


  Wyrażenia regularne mają dwie podstawowe wady. Po pierwsze są trudne
  i~w~pisaniu, a~jeszcze bardziej w~czytaniu. Po~drugie, istnieją dziesiątki
  sposobów notowania wyrażeń regularny, które służą do tego samego, ale mają
  inną składnię. W~połączeniu z~ich nieczytelnością, sprawia to,
  iż~korzystanie z~nich wymaga zwykle bardzo dużo pracy z~naszej strony,
  a~błędy popełnić jest bardzo łatwo. Warto tutaj przytoczyć jedną
  z~mądrości informatyki.

  \textit{Pewien programista miał problem. Stwierdził „Użyję wyrażeń
    regularnych.” Teraz ma dwa problemy.}

  Choć wyrażenia regularne potrafią być bardzo użyteczne, to jednak nie
  należy ich stosować, jeśli dany problem da~się rozwiązać w~sposób
  prostszy. Nie dotyczy to oczywiście najprostszych wyrażeń regularnych,
  które można stosować równie często jak \texttt{Ctrl-f}.

\end{frame}
% ##################





% ##################
\begin{frame}
  \frametitle{Mem dla ilustracji problemu}

  \vspace{-0.5em}


  \begin{figure}

    \label{fig:How-to-regex}

    \centering


    \includegraphics[scale=0.18]
    {./Presentations-pictures/Miscancellous-pictures/How-to-regex.jpg}

  \end{figure}

\end{frame}
% ##################





% ##################
\begin{frame}
  \frametitle{Co ja robię?}

  \vspace{-0.5em}


  \begin{figure}

    \label{fig:How-to-regex}

    \centering


    \includegraphics[scale=0.52]
    {./Presentations-pictures/Miscancellous-pictures/Lost-in-regex.png}

  \end{figure}

\end{frame}
% ##################





% ##################
\begin{frame}
  \frametitle{Praca z~\texttt{grep}em}


  \textit{I~have a natural revulsion to any operating system that shows so
    little plannning as to have to named all of its commands after digestive
    noises (awk, grep, fsck, nroff).} \\
    Autor nieznany, str.~$147$,
    \parencite{Garfinkel-Weise-Strassmann-The-UNIX-HATERS-Handbook-Pub-1994}.

  Nazwa polecenia \texttt{grep} pochodzi od~\textit{Global REgular
    expression Print}, czyli na polski to byłoby coś w~stylu „globalny
  wypisywacz wyrażeń regularnych”. Jest to niezwykle rozbudowane polecenie
  i~moglibyśmy spędzić kilka zajęć na pracy z~nim, na~tym przedmiocie
  ograniczymy~się do stosunkowo podstawowych przypadków. Więcej informacji
  o~\texttt{grep}ie można znaleźć choćby tutaj
  \parencite{Pankaj-Wailia-Mastering-Grep-command-in-Linux-Unix-ETC-Ver-2025}.

  W~katalogu \texttt{Materiały-do-prezentacji\textbackslash
    SO-XX-Grep\textbackslash} można znaleźć plik \textit{Lorem-ipsum.txt}
  na którym będziemy pracować. Zawarty w~nim tekst ten został wygenerowany
  za pomocą strony \textit{Lorem ipsum}
  \colorhref{https://www.lipsum.com/}{https://www.lipsum.com/}.

\end{frame}
% ##################





% ##################
\begin{frame}
  \frametitle{Praca z~\texttt{grep}em}

  Teksty typu \textit{Lorem ipsum} są używany w~przemyśle wydawniczym od
  XVI wieku, jak tekst do~testowania różnych układów strony, więc czemu by
  nie użyć ich i~tutaj?

  Załóżmy, że~jesteśmy w~katalogu, gdzie znajduje~się plik
  \textit{Lorem-ipsum.txt}. Zaczniemy od najprostszego użycia polecenia
  \texttt{grep} \\
  \texttt{\$ grep "in"{} Lorem-ipsum.txt} \\
  Od razu należy zaznaczyć, że~polecenie \texttt{grep} potrzebuje mnie dwa
  argumenty: najpierw tekst który ma wyszukać, potem miejsce, gdzie go ma
  wyszukać. (O~pewnych subtelnościach tego związanych z~przekierowaniami,
  powiem sobie później.) Jeśli wprowadzimy za mało argumentów, to powłoka
  może zacząć~się dziwnie zachowywać, wtedy należy spróbować użyć
  pojedynczego \texttt{Ctrl-d} lub poprosić nas o~pomoc. Od~tego jesteśmy.

\end{frame}
% ##################





% ##################
\begin{frame}
  \frametitle{Praca z~\texttt{grep}em}


  Przyjrzyjmy~się rezultatowi tego polecenia. \\
  \texttt{\$ grep "in"{} Lorem-ipsum.txt} \\
  \texttt{Lorem ipsum dolor sit amet, consectetur
    adipisc{\color{red} in}g elit. Integer at est erat.} \\
  \texttt{vitae. In ultrices turpis et arcu venenatis, at feugiat elit
    lac{\color{red} in}ia. Quisque} \\
  \texttt{t{\color{red} in}cidunt vitae massa euismod maximus. Vestibulum
    sed eros vel arcu iaculis} \\
  \hspace{1em} $\vdots$ \\
  Resztę linii wyniku pominęliśmy dla przejrzystość. Jak widzimy,
  \texttt{grep} wyszukuje wszystkie linie w~których znajduje~się ciąg
  symboli „in”, niezależnie czy jest to pojedyncze słowo, czy część
  większego i~wyróżnia znaleziony fragment odpowiednim kolorem. Jeśli
  znaleziony tekst nie wyświetla~się komuś na kolorowo, proszę spróbować
  polecenia \\
  \texttt{\$ grep --color "in"{} Lorem-ipsum.txt}

\end{frame}
% ##################





% ##################
\begin{frame}
  \frametitle{Praca z~\texttt{grep}em}


  Wynik staje~się łatwiejszy do czytania, jeśli do~\texttt{grep}a prześlemy
  flagę \texttt{-n}, która sprawia, że~polecenie wypisuje numery
  odpowiednich wierszy z~pliku: \\
  \texttt{\$ grep -n "in"{} Lorem-ipsum.txt} \\
  \texttt{{\color{green} 1}{\color{jAxisBlue} :}Lorem ipsum dolor sit amet,
    consectetur adipisc{\color{red} in}g elit. Integer at est erat.} \\
  \texttt{{\color{green} 6}{\color{jAxisBlue} :}vitae. In ultrices turpis et
    arcu venenatis, at feugiat elit lac{\color{red} in}ia. Quisque} \\
  \texttt{{\color{green} 11}{\color{jAxisBlue} :}t{\color{red} in}cidunt
    vitae massa euismod maximus. Vestibulum sed eros vel arcu iaculis} \\
  \hspace{1em} $\vdots$

  W~dalszych przykładach będziemy standardowo dodawać flagę~\texttt{-n}, by
  uczynić zwracane rezultaty prostszymi w~czytaniu.

\end{frame}
% ##################





% ##################
\begin{frame}
  \frametitle{Wyszukiwanie całych wyrazów}


  Jak widzieliśmy, normalnie \texttt{grep} wyszukuje ciąg symboli,
  niezależnie, czy tworzą one w~tekście osoby wyraz, czy nie. Żeby
  wyszukać tylko te wiersze, w~których dany ciąg symboli pojawia~się
  tylko jako osobne słowo, używamy flagi~\texttt{-w}: \\
  \texttt{\$ grep -n -w "in"{} Lorem-ipsum.txt} \\
  \texttt{{\color{green} 14}{\color{jAxisBlue} :}quis interdum. Integer at
    orci at neque pulvinar pretium. Donec {\color{red} in} elementum} \\
  \texttt{{\color{green} 24}{\color{jAxisBlue} :}massa et rhoncus aliquam.
    Suspendisse rhoncus nunc {\color{red} in} finibus mattis.} \\
  \texttt{{\color{green}37}{\color{jAxisBlue} :}vel aliquet. Proin
    {\color{red} in} hendrerit elit, a finibus ipsum. Integer
    sollicitudin} \\
  \hspace{1em} $\vdots$ \\

\end{frame}
% ##################





% ##################
\begin{frame}
  \frametitle{Wyszukiwanie całych wyrazów}


  Tak jak wyjaśniliśmy wcześniej, flagi tego typu można odpowiednio łączyć,
  tak~że polecenia: \\
  \texttt{\$ grep -nw "in"{} Lorem-ipsum.txt} \\
  \texttt{\$ grep -wn "in"{} Lorem-ipsum.txt} \\
  znaczą to samo. W~poprzednich przykładach dla podkreślenia, że~testujemy
  tylko flagę \texttt{-w} napisaliśmy ją osobno. Tą samą konwencję
  będziemy stosować również dalej.

\end{frame}
% ##################





% ##################
\begin{frame}
  \frametitle{Duże i~małe litery}


  Domyślnie \texttt{grep} uwzględnia w~wyszukiwaniu wielkość liter, co
  obrazują poniższe przykłady. \\
  \texttt{\$ grep -n "Du"{} Lorem-ipsum.txt} \\
  \texttt{{\color{green} 32}{\color{jAxisBlue} :}{\color{red} Du}is
    fringilla dictum leo id
    interdum. Ut sit amet sem a nunc porttitor} \\
  \hspace{1em} $\vdots$ \\
  \texttt{\$ grep -n "du"{} Lorem-ipsum.txt} \\
  \texttt{{\color{green} 8}{\color{jAxisBlue} :}consectetur a sapien vitae,
    euismod biben{\color{red} du}m eros.} \\
  \hspace{1em} $\vdots$

  By~\texttt{grep} ignorował różnicę między małymi i~dużymi literami,
  należy mu przesłać flagę \texttt{-i}: \\
  \texttt{\$ grep -n -i "Du"{} Lorem-ipsum.txt} \\
  \texttt{{\color{green} 8}{\color{jAxisBlue} :}consectetur a sapien vitae,
    euismod biben{\color{red} du}m eros.} \\
  \hspace{1em} $\vdots$

\end{frame}
% ##################





% ##################
\begin{frame}
  \frametitle{Zliczanie dopasowany linii}


  By~polecenie \texttt{grep} wypisało w~jakiej liczbie linii został
  znaleziony zadany ciąg symboli (do ilu linii został dopasowany wzorzec)
  dajemy mu flagę~\texttt{-c}: \\
  \texttt{\$ grep -n -c "Du"{} Lorem-ipsum.txt} \\
  \texttt{$6$}

  Jak już wiemy, domyślnie \texttt{grep} wypisuje linie w~których znalazł
  dany ciąg symboli. By~zobaczyć linie \alert{nie} zawierające tego ciągu
  należy użyć flagi~\texttt{-v}: \\
  \texttt{\$ grep -n -v "Du"{} Lorem-ipsum.txt} \\
  \texttt{{\color{green} 1}{\color{jAxisBlue} :}Lorem ipsum dolor sit amet,
    consectetur adipiscing elit. Integer at est erat.} \\
  \texttt{{\color{green} 2}{\color{jAxisBlue} :}Phasellus gravida ligula
    eros, eget ornare purus consequat vitae. Quisque} \\
  \hspace{1em} $\vdots$

\end{frame}
% ##################





% ##################
\begin{frame}
  \frametitle{Jak zobaczyć szerszy kontekst?}


  Powiedzmy, że~oprócz linii w~której znaleziono ciąg symboli, chcemy
  zobaczyć też linię która następuje zaraz za~nią. W~tym celu do polecenia
  dodajemy flagę \texttt{-A 1}. Analogicznie, by zobaczyć dwie linie
  następujące po każdej linii w~której znaleziono ten ciąg dodajemy
  flagę \texttt{-A 2}, gdy chcemy zobaczyć trzy linie flagę
  \texttt{-A 3}.

  \texttt{\$ grep -n -A 1 "Du"{} Lorem-ipsum.txt} \\
  \texttt{{\color{green} 32}{\color{jAxisBlue} :}{\color{red} Du}is
    fringilla dictum leo id interdum. Ut sit amet sem a nunc porttitor} \\
  \texttt{{\color{green} 33}{\color{jAxisBlue} -}vestibulum.} \\
  \texttt{{\color{jAxisBlue} -{}-}} \\
  \texttt{{\color{green} 46}{\color{jAxisBlue} :}nisi. Maecenas euismod
    magna et est iaculis sollicitudin. {\color{red} Du}is turpis elit,} \\
  \texttt{{\color{green} 47}{\color{jAxisBlue} -}vulputate quis felis vel,
    laoreet suscipit diam. Sed vitae diam eu felis} \\
  \hspace{1em} $\vdots$

\end{frame}
% ##################





% ##################
\begin{frame}
  \frametitle{Jak zobaczyć szerszy kontekst?}


  Aby zobaczyć wiersz, który znajduje~się przed tym w~którym znaleziono
  szukany ciąg symboli, używamy flagi \texttt{-B 1}. Analogicznie działają
  flagi \texttt{-B 2}, \texttt{-B 3}, etc.

  \texttt{\$ grep -n -B 1 "Du"{} Lorem-ipsum.txt} \\
  \texttt{{\color{green} 31}{\color{jAxisBlue}:}Maecenas bibendum egestas
    condimentum. Nunc lobortis ac tellus at malesuada.} \\
  \texttt{{\color{green} 32}{\color{jAxisBlue} :}{\color{red} Du}is
    fringilla dictum leo id interdum. Ut sit amet sem a nunc porttitor} \\
  \texttt{{\color{jAxisBlue} -{}-}} \\
  \texttt{{\color{green} 45}{\color{jAxisBlue}-}suscipit. Nullam at neque
    nulla. Proin quis rhoncus dolor. Nam quis tellus} \\
  \texttt{{\color{green}46}{\color{jAxisBlue} :}nisi. Maecenas euismod
    magna et est iaculis} \\
  \texttt{sollicitudin. {\color{red} Du}is turpis elit,}

\end{frame}
% ##################





% ##################
\begin{frame}
  \frametitle{Rozwój GNU/Linuxa}


  Tekst który wyszykujemy \texttt{grep}em nie musi~się składać tylko
  i~wyłącznie z~liter, co obrazuje przykład poniżej.

  \texttt{\$ grep -n "na sem"{} Lorem-ipsum.txt} \\
  \texttt{{\color{green} 5}{\color{jAxisBlue} :}eleifend commodo. Etiam
    ornare congue erat, dapibus} \\
  \texttt{porttitor mag{\color{red}na sem}per} \\
  \texttt{{\color{green} 120}{\color{jAxisBlue} :}sollicitudin sapien.
    Integer et mag{\color{red}na sem}per, commodo nunc vel, eleifend}

\end{frame}
% ##################










% ######################################
\section{Polecenie \texttt{grep} i~wyrażenia regularne}
% ######################################


% ##################
\begin{frame}
  \frametitle{Wyrażenia regularne \texttt{grep}a}


  Do tej pory pracowaliśmy z~ciągami symboli, które reprezentowały „same
  siebie”, teraz przejdziemy do bardziej zaawansowanych sposobów określania
  ciągów symboli, zwanych
  \colorhref{https://en.wikipedia.org/wiki/Regular_expression}
  {wyrażeniami regularnymi}.
  Rozpoczniemy od wyrażenia
  \texttt{"\hspace{0.2em}$\hat{}$\hspace{0.2em}te"}. Symbol
  „\hspace{0.3em}$\hat{}$\hspace{0.3em}” oznacza, że~\texttt{grep} będzie
  szukał ciągu „\texttt{te}”, który rozpoczyna linię tekstu. \\
  \texttt{\$ grep -n "\hspace{0.2em}$\hat{}$\hspace{0.2em}te"{}
    Lorem-ipsum.txt} \\
  \texttt{{\color{green} 62}{\color{jAxisBlue} :}{\color{red} te}mpor
    commodo, sapien leo bibendum augue, nec bibendum nibh nunc vitae} \\
  \texttt{{\color{green} 99}{\color{jAxisBlue} :}{\color{red} te}llus nec
    metus tincidunt tincidunt blandit vitae enim. Phasellus cursus}

\end{frame}
% ##################





% ##################
\begin{frame}
  \frametitle{Wyrażenia regularne \texttt{grep}a}


  Jeśli chcemy by \texttt{grep} szukał tylko ciągu symboli tylko na końcu
  linii, to~kończymy ten ciąg znakiem „\texttt{\$}”. \\
  \texttt{\$ grep -n "ue\$"{} Lorem-ipsum.txt} \\
  \texttt{{\color{green} 2}{\color{jAxisBlue} :}Phasellus gravida ligula
    eros, eget ornare purus} \\
  \texttt{consequat vitae. Quisq{\color{red} ue}} \\
  \texttt{{\color{green} 6}{\color{jAxisBlue} :}vitae. In ultrices turpis
    et arcu venenatis, at feugiat elit lacinia. Quisq{\color{red} ue}} \\
  \hspace{1em} $\vdots$

\end{frame}
% ##################





% ##################
\begin{frame}
  \frametitle{Wyrażenia regularne \texttt{grep}a}


  Jak już mówiliśmy, notacja wyrażeń regularnych nie jest spójna między
  programami i~musimy~się nauczyć z~tym żyć. W~\textsc{bash}u symbolem,
  który oznacza dowolny pojedynczy symbol jest „\texttt{?}”, ale
  w~\texttt{grep}ie jest nim kropka „\texttt{.}”. Czemu życie nie może być
  proste?

  Wobec tego wyrażenie regularne \texttt{"n.c"} wyszukuje ciąg symboli
  który zaczyna~się od~„\texttt{n}”, potem następuje jeden pojedynczy
  symbol, następnie zaś mamy symbol „\texttt{c}”. \\
  \texttt{\$ grep -n "n.c"{} Lorem-ipsum.txt} \\
  \texttt{{\color{green} 10}{\color{jAxisBlue} :}Do{\color{red}nec}
    vehicula scelerisque efficitur. In laoreet} \\
  \texttt{faucibus volutpat. Aliquam} \\
  \texttt{{\color{green}13}{\color{jAxisBlue} :}convallis enim est, id
    porta mauris lobortis eget. Do{\color{red}nec} tincidunt vel orci} \\
  \hspace{1em} $\vdots$

\end{frame}
% ##################





% ##################
\begin{frame}
  \frametitle{Wyrażenia regularne \texttt{grep}a}


  Tak jak w~\textsc{bash}u, nawias kwadratowy „\texttt{[abcde]}” oznacza,
  że~w~danym miejscu musi być \alert{dokładnie jedna} z~liter „a”, „b”,
  „c”, „d” albo „e”. Również w~\texttt{grep}ie można zastąpić ten
  nawias przez „\texttt{[a-e]}”.

  \texttt{\$ grep -n "[a-e]us"{} Lorem-ipsum.txt} \\
  \texttt{{\color{green}5}{\color{jAxisBlue} :}eleifend commodo. Etiam
    ornare congue erat, dapi{\color{red} bus}} \\
  \texttt{porttitor magna semper} \\
  \texttt{{\color{green} 10}{\color{jAxisBlue} :}Donec vehicula scelerisque
    efficitur. In laoreet} \\
  \texttt{fauci{\color{red} bus} volutpat. Aliquam} \\
  \hspace{1em} $\vdots$

  Również wyrażenia regularne „\texttt{[Ab]}”, „\texttt{[012345]}”
  i~„\texttt{[0-5]} działają tak samo jak w~\textsc{bash}u, nie będziemy
  więc ich osobno omawiać.

\end{frame}
% ##################





% ##################
\begin{frame}
  \frametitle{Funkcjonalności \texttt{grep}a można łączyć}


  Oczywiście, wyrażenia regularne można łączyć z~flagami \texttt{grep}a.
  Jeśli więc chcemy znaleźć wszystkie ciągi symboli pasujące do wyrażenia
  regularnego \texttt{"n.c"}, które tworzą osobny wyraz to użyjemy
  polecenia \\
  \texttt{\$ grep -n -w "n.c"{} Lorem-ipsum.txt} \\
  \texttt{{\color{green} 21}{\color{jAxisBlue} :}{\color{red} nec} ante
    maximus tincidunt a vitae justo. Sed vitae convallis metus. Integer} \\
  \texttt{{\color{green} 22}{\color{jAxisBlue} :}mollis orci
    {\color{red} nec} urna dapibus, vel pellentesque ipsum porttitor. Nulla
    eu} \\
  \hspace{1em} $\vdots$

  Jak łatwo się domyślić wyrażenie regularne \texttt{"n..c"} szuka dowolnego
  ciągu symboli, który zaczyna~się od „\texttt{n}”, kończy na~„\texttt{c}”,
  a~pośrodku są dwa, dokładnie dwa, dowolne symbole.

\end{frame}
% ##################





% ##################
\begin{frame}
  \frametitle{Rozbudowane wyrażenia regularne}


  Różne rodzaje wyrażeń regularnych można ze sobą łączyć, co zdaniem
  krytyków prowadzi do zupełnie niezrozumiałych ciągów symboli.

  Przykładowo, rozpatrzymy wciąż stosunkowo proste wyrażenie regularne
  \texttt{"\hspace{0.2em}$\hat{}$\hspace{0.1em}.n"}. Oznacza ono ciąg
  symboli, który znajduje~się na~początku linii, pierwszy symbol tego ciągu
  jest dowolny, a~drugim jest~„\texttt{n}”. Mówiąc inaczej, to wyrażenie
  wyszukuje wszystkie linie, których drugą literą jest~„\texttt{n}”. \\
  \texttt{\$ grep -n "\hspace{0.2em}$\hat{}$\hspace{0.1em}.n"{}
    Lorem-ipsum.txt} \\
  \texttt{{\color{green} 30}{\color{jAxisBlue} :}{\color{red}In} varius
    rutrum efficitur. Suspendisse pretium at augue eget pellentesque.} \\
  \texttt{{\color{green} 98}{\color{jAxisBlue} :}{\color{red} In}teger leo
    nunc, semper id eros a, commodo iaculis dui. Etiam sit amet} \\
  \hspace{1em} $\vdots$

\end{frame}
% ##################





% % ##################
% \begin{frame}
%   \frametitle{}


% \end{frame}
% % ##################




% % ######################################
% \section{}
% % ######################################








% % % ##################
% % \begin{frame}
% %   \frametitle{}




% % \end{frame}
% % % ##################





% % % ##################
% % \begin{frame}
% %   \frametitle{}



% % \end{frame}
% % % ##################





% % % ##################
% % \begin{frame}
% %   \frametitle{}




% % \end{frame}
% % % ##################





% % % ##################
% % \begin{frame}
% %   \frametitle{}



% % \end{frame}
% % % ##################










% % % ######################################
% % \section{}
% % % ######################################


% % % ##################
% % \begin{frame}
% %   \frametitle{}



% % \end{frame}
% % % ##################







% % % ######################################
% % \section{}
% % % ######################################


% % % ##################
% % \begin{frame}
% %   \frametitle{}




% % \end{frame}
% % % ##################





% % % ##################
% % \begin{frame}
% %   \frametitle{}



% % \end{frame}
% % % ##################





% % % ##################
% % \begin{frame}
% %   \frametitle{}



% % \end{frame}
% % % ##################





% % % ##################
% % \begin{frame}
% %   \frametitle{}



% % \end{frame}
% % % ##################





% % % ##################
% % \begin{frame}
% %   \frametitle{}


% % \end{frame}
% % % ##################













% % % ##################
% % \begin{frame}
% %   \frametitle{}



% % \end{frame}
% % % ##################





% % % ##################
% % \begin{frame}
% %   \frametitle{}




% % \end{frame}
% % % ##################





% % % ##################
% % \begin{frame}
% %   \frametitle{}



% % \end{frame}
% % % ##################










% % % ######################################
% % \section{}
% % % ######################################


% % % ##################
% % \begin{frame}
% %   \frametitle{}



% % \end{frame}
% % % ##################





% % % ##################
% % \begin{frame}
% %   \frametitle{}



% % \end{frame}
% % % ##################







% % % ##################
% % \begin{frame}
% %   \frametitle{}



% % \end{frame}
% % % ##################





% % % ##################
% % \begin{frame}
% %   \frametitle{}




% % \end{frame}
% % % ##################





% % % ##################
% % \begin{frame}
% %   \frametitle{}



% % \end{frame}
% % % ##################





% % % ##################
% % \begin{frame}
% %   \frametitle{}



% % \end{frame}
% % % ##################





% % ##################
% \begin{frame}
%   \frametitle{????}




% \end{frame}
% % ##################





% % ##################
% \begin{frame}
%   \frametitle{????}




% \end{frame}
% % ##################





% % ##################
% \begin{frame}
%   \frametitle{????}




% \end{frame}
% % ##################





% % ##################
% \begin{frame}
%   \frametitle{????}




% \end{frame}
% % ##################






% % ##################
% \begin{frame}
%   \frametitle{?????}




% \end{frame}
% % ##################





% % ##################
% \begin{frame}
%   \frametitle{?????}




% \end{frame}
% % ##################





% % ##################
% \begin{frame}
%   \frametitle{?????}




% \end{frame}
% % ##################





% % ##################
% \begin{frame}
%   \frametitle{?????}



% \end{frame}
% % ##################





% % ##################
% \begin{frame}
%   \frametitle{?????}



% \end{frame}
% % ##################





% % ##################
% \begin{frame}
%   \frametitle{?????}



% \end{frame}
% % ##################





% % ##################
% \begin{frame}
%   \frametitle{????}



% \end{frame}
% % ##################





% % ##################
% \begin{frame}
%   \frametitle{?????}



% \end{frame}
% % ##################





% % ##################
% \begin{frame}
%   \frametitle{????}



% \end{frame}
% % ##################





% % ##################
% \begin{frame}
%   \frametitle{?????}



% \end{frame}
% % ##################





% % ##################
% \begin{frame}
%   \frametitle{?????}



% \end{frame}
% % ##################





% % ##################
% \begin{frame}
%   \frametitle{????}



% \end{frame}
% % ##################





% % ##################
% \begin{frame}
%   \frametitle{????}



% \end{frame}
% % ##################





% % ##################
% \begin{frame}
%   \frametitle{?????}




% \end{frame}
% % ##################





% % ##################
% \begin{frame}
%   \frametitle{?????}




% \end{frame}
% % ##################





% % ##################
% \begin{frame}
%   \frametitle{?????}



% \end{frame}
% % ##################





% % ##################
% \begin{frame}
%   \frametitle{????}




% \end{frame}
% % ##################





% % ##################
% \begin{frame}
%   \frametitle{????}




% \end{frame}
% % ##################





% % ##################
% \begin{frame}
%   \frametitle{????}




% \end{frame}
% % ##################










% ######################################
\appendix
% ######################################





% ######################################
\EndingSlide{Dziękuję! Pytania?}
% ######################################










% ####################################################################
% ####################################################################
% Bibliography

\printbibliography





% ####################################################################
% End of the document

\end{document}
