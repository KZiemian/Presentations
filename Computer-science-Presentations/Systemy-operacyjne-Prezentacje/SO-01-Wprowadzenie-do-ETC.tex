% ------------------------------------------------------------------------------------------------------------------
% Basic configuration of Beamera class and Jagiellonian theme
% ------------------------------------------------------------------------------------------------------------------
\RequirePackage[l2tabu, orthodox]{nag}



\ifx\PresentationStyle\notset
  \def\PresentationStyle{dark}
\fi



% Options: t - align frame text to the top.
\documentclass[10pt,t]{beamer}
\mode<presentation>
\usetheme[style=\PresentationStyle,JUlogotitle=no]{jagiellonian}




% ------------------------------------------------------------------------------------
% Procesing configuration files of Jagiellonian theme located
% in the directory "preambule"
% ------------------------------------------------------------------------------------
% Configuration for polish language
% Need description
\usepackage[polish]{babel}
% Need description
\usepackage[MeX]{polski}



% ------------------------------
% Better support of polish chars in technical parts of PDF
% ------------------------------
\hypersetup{pdfencoding=auto,psdextra}

% Package "textpos" give as enviroment "textblock" which is very usefull in
% arranging text on slides.

% This is standard configuration of "textpos"
\usepackage[overlay,absolute]{textpos}

% If you need to see bounds of "textblock's" comment line above and uncomment
% one below.

% Caution! When showboxes option is on significant ammunt of space is add
% to the top of textblock and as such, everyting put in them gone down.
% We need to check how to remove this bug.

% \usepackage[showboxes,overlay,absolute]{textpos}



% Setting scale length for package "textpos"
\setlength{\TPHorizModule}{10mm}
\setlength{\TPVertModule}{\TPHorizModule}


% ---------------------------------------
% Packages written for lectures "Geometria 3D dla twórców gier wideo"
% ---------------------------------------
% \usepackage{./Geometry3DPackages/Geometry3D}
% \usepackage{./Geometry3DPackages/Geometry3DIndices}
% \usepackage{./Geometry3DPackages/Geometry3DTikZStyle}
% \usepackage{./ProgramowanieSymulacjiFizykiPaczki/ProgramowanieSymulacjiFizykiTikZStyle}
% \usepackage{./Geometry3DPackages/mathcommands}


% ---------------------------------------
% TikZ
% ---------------------------------------
% Importing TikZ libraries
\usetikzlibrary{arrows.meta}
\usetikzlibrary{positioning}





% % Configuration package "bm" that need for making bold symbols
% \newcommand{\bmmax}{0}
% \newcommand{\hmmax}{0}
% \usepackage{bm}




% ---------------------------------------
% Packages for scientific texts
% ---------------------------------------
% \let\lll\undefined  % Sometimes you must use this line to allow
% "amsmath" package to works with packages with packages for polish
% languge imported
% /preambul/LanguageSettings/JagiellonianPolishLanguageSettings.tex.
% This comments (probably) removes polish letter Ł.
\usepackage{amsmath}  % Packages from American Mathematical Society (AMS)
\usepackage{amssymb}
\usepackage{amscd}
\usepackage{amsthm}
\usepackage{siunitx}  % Package for typsetting SI units.
\usepackage{upgreek}  % Better looking greek letters.
% Example of using upgreek: pi = \uppi


\usepackage{calrsfs}  % Zmienia czcionkę kaligraficzną w \mathcal
% na ładniejszą. Może w innych miejscach robi to samo, ale o tym nic
% nie wiem.










% ---------------------------------------
% Packages written for lectures "Geometria 3D dla twórców gier wideo"
% ---------------------------------------
% \usepackage{./ProgramowanieSymulacjiFizykiPaczki/ProgramowanieSymulacjiFizyki}
% \usepackage{./ProgramowanieSymulacjiFizykiPaczki/ProgramowanieSymulacjiFizykiIndeksy}
% \usepackage{./ProgramowanieSymulacjiFizykiPaczki/ProgramowanieSymulacjiFizykiTikZStyle}





% !!!!!!!!!!!!!!!!!!!!!!!!!!!!!!
% !!!!!!!!!!!!!!!!!!!!!!!!!!!!!!
% EVIL STUFF
\if\JUlogotitle1
\edef\LogoJUPath{LogoJU_\JUlogoLang/LogoJU_\JUlogoShape_\JUlogoColor.pdf}
\titlegraphic{\hfill\includegraphics[scale=0.22]
{./JagiellonianPictures/\LogoJUPath}}
\fi
% ---------------------------------------
% Commands for handling colors
% ---------------------------------------


% Command for setting normal text color for some text in math modestyle
% Text color depend on used style of Jagiellonian

% Beamer version of command
\newcommand{\TextWithNormalTextColor}[1]{%
  {\color{jNormalTextFGColor}
    \setbeamercolor{math text}{fg=jNormalTextFGColor} {#1}}
}

% Article and similar classes version of command
% \newcommand{\TextWithNormalTextColor}[1]{%
%   {\color{jNormalTextsFGColor} {#1}}
% }



% Beamer version of command
\newcommand{\NormalTextInMathMode}[1]{%
  {\color{jNormalTextFGColor}
    \setbeamercolor{math text}{fg=jNormalTextFGColor} \text{#1}}
}


% Article and similar classes version of command
% \newcommand{\NormalTextInMathMode}[1]{%
%   {\color{jNormalTextsFGColor} \text{#1}}
% }




% Command that sets color of some mathematical text to the same color
% that has normal text in header (?)

% Beamer version of the command
\newcommand{\MathTextFrametitleFGColor}[1]{%
  {\color{jFrametitleFGColor}
    \setbeamercolor{math text}{fg=jFrametitleFGColor} #1}
}

% Article and similar classes version of the command
% \newcommand{\MathTextWhiteColor}[1]{{\color{jFrametitleFGColor} #1}}





% Command for setting color of alert text for some text in math modestyle

% Beamer version of the command
\newcommand{\MathTextAlertColor}[1]{%
  {\color{jOrange} \setbeamercolor{math text}{fg=jOrange} #1}
}

% Article and similar classes version of the command
% \newcommand{\MathTextAlertColor}[1]{{\color{jOrange} #1}}





% Command that allow you to sets chosen color as the color of some text into
% math mode. Due to some nuances in the way that Beamer handle colors
% it not work in all cases. We hope that in the future we will improve it.

% Beamer version of the command
\newcommand{\SetMathTextColor}[2]{%
  {\color{#1} \setbeamercolor{math text}{fg=#1} #2}
}


% Article and similar classes version of the command
% \newcommand{\SetMathTextColor}[2]{{\color{#1} #2}}










% ---------------------------------------
% Commands for few special slides
% ---------------------------------------
\newcommand{\EndingSlide}[1]{%
  \begin{frame}[standout]

    \begingroup

    \color{jFrametitleFGColor}

    #1

    \endgroup

  \end{frame}
}










% ---------------------------------------
% Commands for setting background pictures for some slides
% ---------------------------------------
\newcommand{\TitleBackgroundPicture}
{./JagiellonianPictures/Backgrounds/LajkonikDark.png}
\newcommand{\SectionBackgroundPicture}
{./JagiellonianPictures/Backgrounds/LajkonikLight.png}



\newcommand{\TitleSlideWithPicture}{%
  \begingroup

  \usebackgroundtemplate{%
    \includegraphics[height=\paperheight]{\TitleBackgroundPicture}}

  \maketitle

  \endgroup
}





\newcommand{\SectionSlideWithPicture}[1]{%
  \begingroup

  \usebackgroundtemplate{%
    \includegraphics[height=\paperheight]{\SectionBackgroundPicture}}

  \setbeamercolor{titlelike}{fg=normal text.fg}

  \section{#1}

  \endgroup
}










% ---------------------------------------
% Commands for lectures "Geometria 3D dla twórców gier wideo"
% Polish version
% ---------------------------------------
% Komendy teraz wykomentowane były potrzebne, gdy loga były na niebieskim
% tle, nie na białym. A są na białym bo tego chcieli w biurze projektu.
% \newcommand{\FundingLogoWhitePicturePL}
% {./PresentationPictures/CommonPictures/logotypFundusze_biale_bez_tla2.pdf}
\newcommand{\FundingLogoColorPicturePL}
{./PresentationPictures/CommonPictures/European_Funds_color_PL.pdf}
% \newcommand{\EULogoWhitePicturePL}
% {./PresentationPictures/CommonPictures/logotypUE_biale_bez_tla2.pdf}
\newcommand{\EUSocialFundLogoColorPicturePL}
{./PresentationPictures/CommonPictures/EU_Social_Fund_color_PL.pdf}
% \newcommand{\ZintegrUJLogoWhitePicturePL}
% {./PresentationPictures/CommonPictures/zintegruj-logo-white.pdf}
\newcommand{\ZintegrUJLogoColorPicturePL}
{./PresentationPictures/CommonPictures/ZintegrUJ_color.pdf}
\newcommand{\JULogoColorPicturePL}
{./JagiellonianPictures/LogoJU_PL/LogoJU_A_color.pdf}





\newcommand{\GeometryThreeDSpecialBeginningSlidePL}{%
  \begin{frame}[standout]

    \begin{textblock}{11}(1,0.7)

      \begin{flushleft}

        \mdseries

        \footnotesize

        \color{jFrametitleFGColor}

        Materiał powstał w ramach projektu współfinansowanego ze środków
        Unii Europejskiej w ramach Europejskiego Funduszu Społecznego
        POWR.03.05.00-00-Z309/17-00.

      \end{flushleft}

    \end{textblock}





    \begin{textblock}{10}(0,2.2)

      \tikz \fill[color=jBackgroundStyleLight] (0,0) rectangle (12.8,-1.5);

    \end{textblock}


    \begin{textblock}{3.2}(1,2.45)

      \includegraphics[scale=0.3]{\FundingLogoColorPicturePL}

    \end{textblock}


    \begin{textblock}{2.5}(3.7,2.5)

      \includegraphics[scale=0.2]{\JULogoColorPicturePL}

    \end{textblock}


    \begin{textblock}{2.5}(6,2.4)

      \includegraphics[scale=0.1]{\ZintegrUJLogoColorPicturePL}

    \end{textblock}


    \begin{textblock}{4.2}(8.4,2.6)

      \includegraphics[scale=0.3]{\EUSocialFundLogoColorPicturePL}

    \end{textblock}

  \end{frame}
}



\newcommand{\GeometryThreeDTwoSpecialBeginningSlidesPL}{%
  \begin{frame}[standout]

    \begin{textblock}{11}(1,0.7)

      \begin{flushleft}

        \mdseries

        \footnotesize

        \color{jFrametitleFGColor}

        Materiał powstał w ramach projektu współfinansowanego ze środków
        Unii Europejskiej w ramach Europejskiego Funduszu Społecznego
        POWR.03.05.00-00-Z309/17-00.

      \end{flushleft}

    \end{textblock}





    \begin{textblock}{10}(0,2.2)

      \tikz \fill[color=jBackgroundStyleLight] (0,0) rectangle (12.8,-1.5);

    \end{textblock}


    \begin{textblock}{3.2}(1,2.45)

      \includegraphics[scale=0.3]{\FundingLogoColorPicturePL}

    \end{textblock}


    \begin{textblock}{2.5}(3.7,2.5)

      \includegraphics[scale=0.2]{\JULogoColorPicturePL}

    \end{textblock}


    \begin{textblock}{2.5}(6,2.4)

      \includegraphics[scale=0.1]{\ZintegrUJLogoColorPicturePL}

    \end{textblock}


    \begin{textblock}{4.2}(8.4,2.6)

      \includegraphics[scale=0.3]{\EUSocialFundLogoColorPicturePL}

    \end{textblock}

  \end{frame}





  \TitleSlideWithPicture
}



\newcommand{\GeometryThreeDSpecialEndingSlidePL}{%
  \begin{frame}[standout]

    \begin{textblock}{11}(1,0.7)

      \begin{flushleft}

        \mdseries

        \footnotesize

        \color{jFrametitleFGColor}

        Materiał powstał w ramach projektu współfinansowanego ze środków
        Unii Europejskiej w~ramach Europejskiego Funduszu Społecznego
        POWR.03.05.00-00-Z309/17-00.

      \end{flushleft}

    \end{textblock}





    \begin{textblock}{10}(0,2.2)

      \tikz \fill[color=jBackgroundStyleLight] (0,0) rectangle (12.8,-1.5);

    \end{textblock}


    \begin{textblock}{3.2}(1,2.45)

      \includegraphics[scale=0.3]{\FundingLogoColorPicturePL}

    \end{textblock}


    \begin{textblock}{2.5}(3.7,2.5)

      \includegraphics[scale=0.2]{\JULogoColorPicturePL}

    \end{textblock}


    \begin{textblock}{2.5}(6,2.4)

      \includegraphics[scale=0.1]{\ZintegrUJLogoColorPicturePL}

    \end{textblock}


    \begin{textblock}{4.2}(8.4,2.6)

      \includegraphics[scale=0.3]{\EUSocialFundLogoColorPicturePL}

    \end{textblock}





    \begin{textblock}{11}(1,4)

      \begin{flushleft}

        \mdseries

        \footnotesize

        \RaggedRight

        \color{jFrametitleFGColor}

        Treść niniejszego wykładu jest udostępniona na~licencji
        Creative Commons (\textsc{cc}), z~uzna\-niem autorstwa
        (\textsc{by}) oraz udostępnianiem na tych samych warunkach
        (\textsc{sa}). Rysunki i~wy\-kresy zawarte w~wykładzie są
        autorstwa dr.~hab.~Pawła Węgrzyna et~al. i~są dostępne
        na tej samej licencji, o~ile nie wskazano inaczej.
        W~prezentacji wykorzystano temat Beamera Jagiellonian,
        oparty na~temacie Metropolis Matthiasa Vogelgesanga,
        dostępnym na licencji \LaTeX{} Project Public License~1.3c
        pod adresem: \colorhref{https://github.com/matze/mtheme}
        {https://github.com/matze/mtheme}.

        Projekt typograficzny: Iwona Grabska-Gradzińska \\
        Skład: Kamil Ziemian;
        Korekta: Wojciech Palacz \\
        Modele: Dariusz Frymus, Kamil Nowakowski \\
        Rysunki i~wykresy: Kamil Ziemian, Paweł Węgrzyn, Wojciech Palacz

      \end{flushleft}

    \end{textblock}

  \end{frame}
}



\newcommand{\GeometryThreeDTwoSpecialEndingSlidesPL}[1]{%
  \begin{frame}[standout]


    \begin{textblock}{11}(1,0.7)

      \begin{flushleft}

        \mdseries

        \footnotesize

        \color{jFrametitleFGColor}

        Materiał powstał w ramach projektu współfinansowanego ze środków
        Unii Europejskiej w~ramach Europejskiego Funduszu Społecznego
        POWR.03.05.00-00-Z309/17-00.

      \end{flushleft}

    \end{textblock}





    \begin{textblock}{10}(0,2.2)

      \tikz \fill[color=jBackgroundStyleLight] (0,0) rectangle (12.8,-1.5);

    \end{textblock}


    \begin{textblock}{3.2}(1,2.45)

      \includegraphics[scale=0.3]{\FundingLogoColorPicturePL}

    \end{textblock}


    \begin{textblock}{2.5}(3.7,2.5)

      \includegraphics[scale=0.2]{\JULogoColorPicturePL}

    \end{textblock}


    \begin{textblock}{2.5}(6,2.4)

      \includegraphics[scale=0.1]{\ZintegrUJLogoColorPicturePL}

    \end{textblock}


    \begin{textblock}{4.2}(8.4,2.6)

      \includegraphics[scale=0.3]{\EUSocialFundLogoColorPicturePL}

    \end{textblock}





    \begin{textblock}{11}(1,4)

      \begin{flushleft}

        \mdseries

        \footnotesize

        \RaggedRight

        \color{jFrametitleFGColor}

        Treść niniejszego wykładu jest udostępniona na~licencji
        Creative Commons (\textsc{cc}), z~uzna\-niem autorstwa
        (\textsc{by}) oraz udostępnianiem na tych samych warunkach
        (\textsc{sa}). Rysunki i~wy\-kresy zawarte w~wykładzie są
        autorstwa dr.~hab.~Pawła Węgrzyna et~al. i~są dostępne
        na tej samej licencji, o~ile nie wskazano inaczej.
        W~prezentacji wykorzystano temat Beamera Jagiellonian,
        oparty na~temacie Metropolis Matthiasa Vogelgesanga,
        dostępnym na licencji \LaTeX{} Project Public License~1.3c
        pod adresem: \colorhref{https://github.com/matze/mtheme}
        {https://github.com/matze/mtheme}.

        Projekt typograficzny: Iwona Grabska-Gradzińska \\
        Skład: Kamil Ziemian;
        Korekta: Wojciech Palacz \\
        Modele: Dariusz Frymus, Kamil Nowakowski \\
        Rysunki i~wykresy: Kamil Ziemian, Paweł Węgrzyn, Wojciech Palacz

      \end{flushleft}

    \end{textblock}

  \end{frame}





  \begin{frame}[standout]

    \begingroup

    \color{jFrametitleFGColor}

    #1

    \endgroup

  \end{frame}
}



\newcommand{\GeometryThreeDSpecialEndingSlideVideoPL}{%
  \begin{frame}[standout]

    \begin{textblock}{11}(1,0.7)

      \begin{flushleft}

        \mdseries

        \footnotesize

        \color{jFrametitleFGColor}

        Materiał powstał w ramach projektu współfinansowanego ze środków
        Unii Europejskiej w~ramach Europejskiego Funduszu Społecznego
        POWR.03.05.00-00-Z309/17-00.

      \end{flushleft}

    \end{textblock}





    \begin{textblock}{10}(0,2.2)

      \tikz \fill[color=jBackgroundStyleLight] (0,0) rectangle (12.8,-1.5);

    \end{textblock}


    \begin{textblock}{3.2}(1,2.45)

      \includegraphics[scale=0.3]{\FundingLogoColorPicturePL}

    \end{textblock}


    \begin{textblock}{2.5}(3.7,2.5)

      \includegraphics[scale=0.2]{\JULogoColorPicturePL}

    \end{textblock}


    \begin{textblock}{2.5}(6,2.4)

      \includegraphics[scale=0.1]{\ZintegrUJLogoColorPicturePL}

    \end{textblock}


    \begin{textblock}{4.2}(8.4,2.6)

      \includegraphics[scale=0.3]{\EUSocialFundLogoColorPicturePL}

    \end{textblock}





    \begin{textblock}{11}(1,4)

      \begin{flushleft}

        \mdseries

        \footnotesize

        \RaggedRight

        \color{jFrametitleFGColor}

        Treść niniejszego wykładu jest udostępniona na~licencji
        Creative Commons (\textsc{cc}), z~uzna\-niem autorstwa
        (\textsc{by}) oraz udostępnianiem na tych samych warunkach
        (\textsc{sa}). Rysunki i~wy\-kresy zawarte w~wykładzie są
        autorstwa dr.~hab.~Pawła Węgrzyna et~al. i~są dostępne
        na tej samej licencji, o~ile nie wskazano inaczej.
        W~prezentacji wykorzystano temat Beamera Jagiellonian,
        oparty na~temacie Metropolis Matthiasa Vogelgesanga,
        dostępnym na licencji \LaTeX{} Project Public License~1.3c
        pod adresem: \colorhref{https://github.com/matze/mtheme}
        {https://github.com/matze/mtheme}.

        Projekt typograficzny: Iwona Grabska-Gradzińska;
        Skład: Kamil Ziemian \\
        Korekta: Wojciech Palacz;
        Modele: Dariusz Frymus, Kamil Nowakowski \\
        Rysunki i~wykresy: Kamil Ziemian, Paweł Węgrzyn, Wojciech Palacz \\
        Montaż: Agencja Filmowa Film \& Television Production~-- Zbigniew
        Masklak

      \end{flushleft}

    \end{textblock}

  \end{frame}
}





\newcommand{\GeometryThreeDTwoSpecialEndingSlidesVideoPL}[1]{%
  \begin{frame}[standout]

    \begin{textblock}{11}(1,0.7)

      \begin{flushleft}

        \mdseries

        \footnotesize

        \color{jFrametitleFGColor}

        Materiał powstał w ramach projektu współfinansowanego ze środków
        Unii Europejskiej w~ramach Europejskiego Funduszu Społecznego
        POWR.03.05.00-00-Z309/17-00.

      \end{flushleft}

    \end{textblock}





    \begin{textblock}{10}(0,2.2)

      \tikz \fill[color=jBackgroundStyleLight] (0,0) rectangle (12.8,-1.5);

    \end{textblock}


    \begin{textblock}{3.2}(1,2.45)

      \includegraphics[scale=0.3]{\FundingLogoColorPicturePL}

    \end{textblock}


    \begin{textblock}{2.5}(3.7,2.5)

      \includegraphics[scale=0.2]{\JULogoColorPicturePL}

    \end{textblock}


    \begin{textblock}{2.5}(6,2.4)

      \includegraphics[scale=0.1]{\ZintegrUJLogoColorPicturePL}

    \end{textblock}


    \begin{textblock}{4.2}(8.4,2.6)

      \includegraphics[scale=0.3]{\EUSocialFundLogoColorPicturePL}

    \end{textblock}





    \begin{textblock}{11}(1,4)

      \begin{flushleft}

        \mdseries

        \footnotesize

        \RaggedRight

        \color{jFrametitleFGColor}

        Treść niniejszego wykładu jest udostępniona na~licencji
        Creative Commons (\textsc{cc}), z~uzna\-niem autorstwa
        (\textsc{by}) oraz udostępnianiem na tych samych warunkach
        (\textsc{sa}). Rysunki i~wy\-kresy zawarte w~wykładzie są
        autorstwa dr.~hab.~Pawła Węgrzyna et~al. i~są dostępne
        na tej samej licencji, o~ile nie wskazano inaczej.
        W~prezentacji wykorzystano temat Beamera Jagiellonian,
        oparty na~temacie Metropolis Matthiasa Vogelgesanga,
        dostępnym na licencji \LaTeX{} Project Public License~1.3c
        pod adresem: \colorhref{https://github.com/matze/mtheme}
        {https://github.com/matze/mtheme}.

        Projekt typograficzny: Iwona Grabska-Gradzińska;
        Skład: Kamil Ziemian \\
        Korekta: Wojciech Palacz;
        Modele: Dariusz Frymus, Kamil Nowakowski \\
        Rysunki i~wykresy: Kamil Ziemian, Paweł Węgrzyn, Wojciech Palacz \\
        Montaż: Agencja Filmowa Film \& Television Production~-- Zbigniew
        Masklak

      \end{flushleft}

    \end{textblock}

  \end{frame}





  \begin{frame}[standout]


    \begingroup

    \color{jFrametitleFGColor}

    #1

    \endgroup

  \end{frame}
}










% ---------------------------------------
% Commands for lectures "Geometria 3D dla twórców gier wideo"
% English version
% ---------------------------------------
% \newcommand{\FundingLogoWhitePictureEN}
% {./PresentationPictures/CommonPictures/logotypFundusze_biale_bez_tla2.pdf}
\newcommand{\FundingLogoColorPictureEN}
{./PresentationPictures/CommonPictures/European_Funds_color_EN.pdf}
% \newcommand{\EULogoWhitePictureEN}
% {./PresentationPictures/CommonPictures/logotypUE_biale_bez_tla2.pdf}
\newcommand{\EUSocialFundLogoColorPictureEN}
{./PresentationPictures/CommonPictures/EU_Social_Fund_color_EN.pdf}
% \newcommand{\ZintegrUJLogoWhitePictureEN}
% {./PresentationPictures/CommonPictures/zintegruj-logo-white.pdf}
\newcommand{\ZintegrUJLogoColorPictureEN}
{./PresentationPictures/CommonPictures/ZintegrUJ_color.pdf}
\newcommand{\JULogoColorPictureEN}
{./JagiellonianPictures/LogoJU_EN/LogoJU_A_color.pdf}



\newcommand{\GeometryThreeDSpecialBeginningSlideEN}{%
  \begin{frame}[standout]

    \begin{textblock}{11}(1,0.7)

      \begin{flushleft}

        \mdseries

        \footnotesize

        \color{jFrametitleFGColor}

        This content was created as part of a project co-financed by the
        European Union within the framework of the European Social Fund
        POWR.03.05.00-00-Z309/17-00.

      \end{flushleft}

    \end{textblock}





    \begin{textblock}{10}(0,2.2)

      \tikz \fill[color=jBackgroundStyleLight] (0,0) rectangle (12.8,-1.5);

    \end{textblock}


    \begin{textblock}{3.2}(0.7,2.45)

      \includegraphics[scale=0.3]{\FundingLogoColorPictureEN}

    \end{textblock}


    \begin{textblock}{2.5}(4.15,2.5)

      \includegraphics[scale=0.2]{\JULogoColorPictureEN}

    \end{textblock}


    \begin{textblock}{2.5}(6.35,2.4)

      \includegraphics[scale=0.1]{\ZintegrUJLogoColorPictureEN}

    \end{textblock}


    \begin{textblock}{4.2}(8.4,2.6)

      \includegraphics[scale=0.3]{\EUSocialFundLogoColorPictureEN}

    \end{textblock}

  \end{frame}
}



\newcommand{\GeometryThreeDTwoSpecialBeginningSlidesEN}{%
  \begin{frame}[standout]

    \begin{textblock}{11}(1,0.7)

      \begin{flushleft}

        \mdseries

        \footnotesize

        \color{jFrametitleFGColor}

        This content was created as part of a project co-financed by the
        European Union within the framework of the European Social Fund
        POWR.03.05.00-00-Z309/17-00.

      \end{flushleft}

    \end{textblock}





    \begin{textblock}{10}(0,2.2)

      \tikz \fill[color=jBackgroundStyleLight] (0,0) rectangle (12.8,-1.5);

    \end{textblock}


    \begin{textblock}{3.2}(0.7,2.45)

      \includegraphics[scale=0.3]{\FundingLogoColorPictureEN}

    \end{textblock}


    \begin{textblock}{2.5}(4.15,2.5)

      \includegraphics[scale=0.2]{\JULogoColorPictureEN}

    \end{textblock}


    \begin{textblock}{2.5}(6.35,2.4)

      \includegraphics[scale=0.1]{\ZintegrUJLogoColorPictureEN}

    \end{textblock}


    \begin{textblock}{4.2}(8.4,2.6)

      \includegraphics[scale=0.3]{\EUSocialFundLogoColorPictureEN}

    \end{textblock}

  \end{frame}





  \TitleSlideWithPicture
}



\newcommand{\GeometryThreeDSpecialEndingSlideEN}{%
  \begin{frame}[standout]

    \begin{textblock}{11}(1,0.7)

      \begin{flushleft}

        \mdseries

        \footnotesize

        \color{jFrametitleFGColor}

        This content was created as part of a project co-financed by the
        European Union within the framework of the European Social Fund
        POWR.03.05.00-00-Z309/17-00.

      \end{flushleft}

    \end{textblock}





    \begin{textblock}{10}(0,2.2)

      \tikz \fill[color=jBackgroundStyleLight] (0,0) rectangle (12.8,-1.5);

    \end{textblock}


    \begin{textblock}{3.2}(0.7,2.45)

      \includegraphics[scale=0.3]{\FundingLogoColorPictureEN}

    \end{textblock}


    \begin{textblock}{2.5}(4.15,2.5)

      \includegraphics[scale=0.2]{\JULogoColorPictureEN}

    \end{textblock}


    \begin{textblock}{2.5}(6.35,2.4)

      \includegraphics[scale=0.1]{\ZintegrUJLogoColorPictureEN}

    \end{textblock}


    \begin{textblock}{4.2}(8.4,2.6)

      \includegraphics[scale=0.3]{\EUSocialFundLogoColorPictureEN}

    \end{textblock}





    \begin{textblock}{11}(1,4)

      \begin{flushleft}

        \mdseries

        \footnotesize

        \RaggedRight

        \color{jFrametitleFGColor}

        The content of this lecture is made available under a~Creative
        Commons licence (\textsc{cc}), giving the author the credits
        (\textsc{by}) and putting an obligation to share on the same terms
        (\textsc{sa}). Figures and diagrams included in the lecture are
        authored by Paweł Węgrzyn et~al., and are available under the same
        license unless indicated otherwise.\\ The presentation uses the
        Beamer Jagiellonian theme based on Matthias Vogelgesang’s
        Metropolis theme, available under license \LaTeX{} Project
        Public License~1.3c at: \colorhref{https://github.com/matze/mtheme}
        {https://github.com/matze/mtheme}.

        Typographic design: Iwona Grabska-Gradzińska \\
        \LaTeX{} Typesetting: Kamil Ziemian \\
        Proofreading: Wojciech Palacz,
        Monika Stawicka \\
        3D Models: Dariusz Frymus, Kamil Nowakowski \\
        Figures and charts: Kamil Ziemian, Paweł Węgrzyn, Wojciech Palacz

      \end{flushleft}

    \end{textblock}

  \end{frame}
}



\newcommand{\GeometryThreeDTwoSpecialEndingSlidesEN}[1]{%
  \begin{frame}[standout]


    \begin{textblock}{11}(1,0.7)

      \begin{flushleft}

        \mdseries

        \footnotesize

        \color{jFrametitleFGColor}

        This content was created as part of a project co-financed by the
        European Union within the framework of the European Social Fund
        POWR.03.05.00-00-Z309/17-00.

      \end{flushleft}

    \end{textblock}





    \begin{textblock}{10}(0,2.2)

      \tikz \fill[color=jBackgroundStyleLight] (0,0) rectangle (12.8,-1.5);

    \end{textblock}


    \begin{textblock}{3.2}(0.7,2.45)

      \includegraphics[scale=0.3]{\FundingLogoColorPictureEN}

    \end{textblock}


    \begin{textblock}{2.5}(4.15,2.5)

      \includegraphics[scale=0.2]{\JULogoColorPictureEN}

    \end{textblock}


    \begin{textblock}{2.5}(6.35,2.4)

      \includegraphics[scale=0.1]{\ZintegrUJLogoColorPictureEN}

    \end{textblock}


    \begin{textblock}{4.2}(8.4,2.6)

      \includegraphics[scale=0.3]{\EUSocialFundLogoColorPictureEN}

    \end{textblock}





    \begin{textblock}{11}(1,4)

      \begin{flushleft}

        \mdseries

        \footnotesize

        \RaggedRight

        \color{jFrametitleFGColor}

        The content of this lecture is made available under a~Creative
        Commons licence (\textsc{cc}), giving the author the credits
        (\textsc{by}) and putting an obligation to share on the same terms
        (\textsc{sa}). Figures and diagrams included in the lecture are
        authored by Paweł Węgrzyn et~al., and are available under the same
        license unless indicated otherwise.\\ The presentation uses the
        Beamer Jagiellonian theme based on Matthias Vogelgesang’s
        Metropolis theme, available under license \LaTeX{} Project
        Public License~1.3c at: \colorhref{https://github.com/matze/mtheme}
        {https://github.com/matze/mtheme}.

        Typographic design: Iwona Grabska-Gradzińska \\
        \LaTeX{} Typesetting: Kamil Ziemian \\
        Proofreading: Wojciech Palacz,
        Monika Stawicka \\
        3D Models: Dariusz Frymus, Kamil Nowakowski \\
        Figures and charts: Kamil Ziemian, Paweł Węgrzyn, Wojciech Palacz

      \end{flushleft}

    \end{textblock}

  \end{frame}





  \begin{frame}[standout]

    \begingroup

    \color{jFrametitleFGColor}

    #1

    \endgroup

  \end{frame}
}



\newcommand{\GeometryThreeDSpecialEndingSlideVideoVerOneEN}{%
  \begin{frame}[standout]

    \begin{textblock}{11}(1,0.7)

      \begin{flushleft}

        \mdseries

        \footnotesize

        \color{jFrametitleFGColor}

        This content was created as part of a project co-financed by the
        European Union within the framework of the European Social Fund
        POWR.03.05.00-00-Z309/17-00.

      \end{flushleft}

    \end{textblock}





    \begin{textblock}{10}(0,2.2)

      \tikz \fill[color=jBackgroundStyleLight] (0,0) rectangle (12.8,-1.5);

    \end{textblock}


    \begin{textblock}{3.2}(0.7,2.45)

      \includegraphics[scale=0.3]{\FundingLogoColorPictureEN}

    \end{textblock}


    \begin{textblock}{2.5}(4.15,2.5)

      \includegraphics[scale=0.2]{\JULogoColorPictureEN}

    \end{textblock}


    \begin{textblock}{2.5}(6.35,2.4)

      \includegraphics[scale=0.1]{\ZintegrUJLogoColorPictureEN}

    \end{textblock}


    \begin{textblock}{4.2}(8.4,2.6)

      \includegraphics[scale=0.3]{\EUSocialFundLogoColorPictureEN}

    \end{textblock}





    \begin{textblock}{11}(1,4)

      \begin{flushleft}

        \mdseries

        \footnotesize

        \RaggedRight

        \color{jFrametitleFGColor}

        The content of this lecture is made available under a Creative
        Commons licence (\textsc{cc}), giving the author the credits
        (\textsc{by}) and putting an obligation to share on the same terms
        (\textsc{sa}). Figures and diagrams included in the lecture are
        authored by Paweł Węgrzyn et~al., and are available under the same
        license unless indicated otherwise.\\ The presentation uses the
        Beamer Jagiellonian theme based on Matthias Vogelgesang’s
        Metropolis theme, available under license \LaTeX{} Project
        Public License~1.3c at: \colorhref{https://github.com/matze/mtheme}
        {https://github.com/matze/mtheme}.

        Typographic design: Iwona Grabska-Gradzińska;
        \LaTeX{} Typesetting: Kamil Ziemian \\
        Proofreading: Wojciech Palacz,
        Monika Stawicka \\
        3D Models: Dariusz Frymus, Kamil Nowakowski \\
        Figures and charts: Kamil Ziemian, Paweł Węgrzyn, Wojciech
        Palacz \\
        Film editing: Agencja Filmowa Film \& Television Production~--
        Zbigniew Masklak

      \end{flushleft}

    \end{textblock}

  \end{frame}
}



\newcommand{\GeometryThreeDSpecialEndingSlideVideoVerTwoEN}{%
  \begin{frame}[standout]

    \begin{textblock}{11}(1,0.7)

      \begin{flushleft}

        \mdseries

        \footnotesize

        \color{jFrametitleFGColor}

        This content was created as part of a project co-financed by the
        European Union within the framework of the European Social Fund
        POWR.03.05.00-00-Z309/17-00.

      \end{flushleft}

    \end{textblock}





    \begin{textblock}{10}(0,2.2)

      \tikz \fill[color=jBackgroundStyleLight] (0,0) rectangle (12.8,-1.5);

    \end{textblock}


    \begin{textblock}{3.2}(0.7,2.45)

      \includegraphics[scale=0.3]{\FundingLogoColorPictureEN}

    \end{textblock}


    \begin{textblock}{2.5}(4.15,2.5)

      \includegraphics[scale=0.2]{\JULogoColorPictureEN}

    \end{textblock}


    \begin{textblock}{2.5}(6.35,2.4)

      \includegraphics[scale=0.1]{\ZintegrUJLogoColorPictureEN}

    \end{textblock}


    \begin{textblock}{4.2}(8.4,2.6)

      \includegraphics[scale=0.3]{\EUSocialFundLogoColorPictureEN}

    \end{textblock}





    \begin{textblock}{11}(1,4)

      \begin{flushleft}

        \mdseries

        \footnotesize

        \RaggedRight

        \color{jFrametitleFGColor}

        The content of this lecture is made available under a Creative
        Commons licence (\textsc{cc}), giving the author the credits
        (\textsc{by}) and putting an obligation to share on the same terms
        (\textsc{sa}). Figures and diagrams included in the lecture are
        authored by Paweł Węgrzyn et~al., and are available under the same
        license unless indicated otherwise.\\ The presentation uses the
        Beamer Jagiellonian theme based on Matthias Vogelgesang’s
        Metropolis theme, available under license \LaTeX{} Project
        Public License~1.3c at: \colorhref{https://github.com/matze/mtheme}
        {https://github.com/matze/mtheme}.

        Typographic design: Iwona Grabska-Gradzińska;
        \LaTeX{} Typesetting: Kamil Ziemian \\
        Proofreading: Wojciech Palacz,
        Monika Stawicka \\
        3D Models: Dariusz Frymus, Kamil Nowakowski \\
        Figures and charts: Kamil Ziemian, Paweł Węgrzyn, Wojciech
        Palacz \\
        Film editing: IMAVI -- Joanna Kozakiewicz, Krzysztof Magda, Nikodem
        Frodyma

      \end{flushleft}

    \end{textblock}

  \end{frame}
}



\newcommand{\GeometryThreeDSpecialEndingSlideVideoVerThreeEN}{%
  \begin{frame}[standout]

    \begin{textblock}{11}(1,0.7)

      \begin{flushleft}

        \mdseries

        \footnotesize

        \color{jFrametitleFGColor}

        This content was created as part of a project co-financed by the
        European Union within the framework of the European Social Fund
        POWR.03.05.00-00-Z309/17-00.

      \end{flushleft}

    \end{textblock}





    \begin{textblock}{10}(0,2.2)

      \tikz \fill[color=jBackgroundStyleLight] (0,0) rectangle (12.8,-1.5);

    \end{textblock}


    \begin{textblock}{3.2}(0.7,2.45)

      \includegraphics[scale=0.3]{\FundingLogoColorPictureEN}

    \end{textblock}


    \begin{textblock}{2.5}(4.15,2.5)

      \includegraphics[scale=0.2]{\JULogoColorPictureEN}

    \end{textblock}


    \begin{textblock}{2.5}(6.35,2.4)

      \includegraphics[scale=0.1]{\ZintegrUJLogoColorPictureEN}

    \end{textblock}


    \begin{textblock}{4.2}(8.4,2.6)

      \includegraphics[scale=0.3]{\EUSocialFundLogoColorPictureEN}

    \end{textblock}





    \begin{textblock}{11}(1,4)

      \begin{flushleft}

        \mdseries

        \footnotesize

        \RaggedRight

        \color{jFrametitleFGColor}

        The content of this lecture is made available under a Creative
        Commons licence (\textsc{cc}), giving the author the credits
        (\textsc{by}) and putting an obligation to share on the same terms
        (\textsc{sa}). Figures and diagrams included in the lecture are
        authored by Paweł Węgrzyn et~al., and are available under the same
        license unless indicated otherwise.\\ The presentation uses the
        Beamer Jagiellonian theme based on Matthias Vogelgesang’s
        Metropolis theme, available under license \LaTeX{} Project
        Public License~1.3c at: \colorhref{https://github.com/matze/mtheme}
        {https://github.com/matze/mtheme}.

        Typographic design: Iwona Grabska-Gradzińska;
        \LaTeX{} Typesetting: Kamil Ziemian \\
        Proofreading: Wojciech Palacz,
        Monika Stawicka \\
        3D Models: Dariusz Frymus, Kamil Nowakowski \\
        Figures and charts: Kamil Ziemian, Paweł Węgrzyn, Wojciech
        Palacz \\
        Film editing: Agencja Filmowa Film \& Television Production~--
        Zbigniew Masklak \\
        Film editing: IMAVI -- Joanna Kozakiewicz, Krzysztof Magda, Nikodem
        Frodyma

      \end{flushleft}

    \end{textblock}

  \end{frame}
}



\newcommand{\GeometryThreeDTwoSpecialEndingSlidesVideoVerOneEN}[1]{%
  \begin{frame}[standout]

    \begin{textblock}{11}(1,0.7)

      \begin{flushleft}

        \mdseries

        \footnotesize

        \color{jFrametitleFGColor}

        This content was created as part of a project co-financed by the
        European Union within the framework of the European Social Fund
        POWR.03.05.00-00-Z309/17-00.

      \end{flushleft}

    \end{textblock}





    \begin{textblock}{10}(0,2.2)

      \tikz \fill[color=jBackgroundStyleLight] (0,0) rectangle (12.8,-1.5);

    \end{textblock}


    \begin{textblock}{3.2}(0.7,2.45)

      \includegraphics[scale=0.3]{\FundingLogoColorPictureEN}

    \end{textblock}


    \begin{textblock}{2.5}(4.15,2.5)

      \includegraphics[scale=0.2]{\JULogoColorPictureEN}

    \end{textblock}


    \begin{textblock}{2.5}(6.35,2.4)

      \includegraphics[scale=0.1]{\ZintegrUJLogoColorPictureEN}

    \end{textblock}


    \begin{textblock}{4.2}(8.4,2.6)

      \includegraphics[scale=0.3]{\EUSocialFundLogoColorPictureEN}

    \end{textblock}





    \begin{textblock}{11}(1,4)

      \begin{flushleft}

        \mdseries

        \footnotesize

        \RaggedRight

        \color{jFrametitleFGColor}

        The content of this lecture is made available under a Creative
        Commons licence (\textsc{cc}), giving the author the credits
        (\textsc{by}) and putting an obligation to share on the same terms
        (\textsc{sa}). Figures and diagrams included in the lecture are
        authored by Paweł Węgrzyn et~al., and are available under the same
        license unless indicated otherwise.\\ The presentation uses the
        Beamer Jagiellonian theme based on Matthias Vogelgesang’s
        Metropolis theme, available under license \LaTeX{} Project
        Public License~1.3c at: \colorhref{https://github.com/matze/mtheme}
        {https://github.com/matze/mtheme}.

        Typographic design: Iwona Grabska-Gradzińska;
        \LaTeX{} Typesetting: Kamil Ziemian \\
        Proofreading: Wojciech Palacz,
        Monika Stawicka \\
        3D Models: Dariusz Frymus, Kamil Nowakowski \\
        Figures and charts: Kamil Ziemian, Paweł Węgrzyn,
        Wojciech Palacz \\
        Film editing: Agencja Filmowa Film \& Television Production~--
        Zbigniew Masklak

      \end{flushleft}

    \end{textblock}

  \end{frame}





  \begin{frame}[standout]


    \begingroup

    \color{jFrametitleFGColor}

    #1

    \endgroup

  \end{frame}
}



\newcommand{\GeometryThreeDTwoSpecialEndingSlidesVideoVerTwoEN}[1]{%
  \begin{frame}[standout]

    \begin{textblock}{11}(1,0.7)

      \begin{flushleft}

        \mdseries

        \footnotesize

        \color{jFrametitleFGColor}

        This content was created as part of a project co-financed by the
        European Union within the framework of the European Social Fund
        POWR.03.05.00-00-Z309/17-00.

      \end{flushleft}

    \end{textblock}





    \begin{textblock}{10}(0,2.2)

      \tikz \fill[color=jBackgroundStyleLight] (0,0) rectangle (12.8,-1.5);

    \end{textblock}


    \begin{textblock}{3.2}(0.7,2.45)

      \includegraphics[scale=0.3]{\FundingLogoColorPictureEN}

    \end{textblock}


    \begin{textblock}{2.5}(4.15,2.5)

      \includegraphics[scale=0.2]{\JULogoColorPictureEN}

    \end{textblock}


    \begin{textblock}{2.5}(6.35,2.4)

      \includegraphics[scale=0.1]{\ZintegrUJLogoColorPictureEN}

    \end{textblock}


    \begin{textblock}{4.2}(8.4,2.6)

      \includegraphics[scale=0.3]{\EUSocialFundLogoColorPictureEN}

    \end{textblock}





    \begin{textblock}{11}(1,4)

      \begin{flushleft}

        \mdseries

        \footnotesize

        \RaggedRight

        \color{jFrametitleFGColor}

        The content of this lecture is made available under a Creative
        Commons licence (\textsc{cc}), giving the author the credits
        (\textsc{by}) and putting an obligation to share on the same terms
        (\textsc{sa}). Figures and diagrams included in the lecture are
        authored by Paweł Węgrzyn et~al., and are available under the same
        license unless indicated otherwise.\\ The presentation uses the
        Beamer Jagiellonian theme based on Matthias Vogelgesang’s
        Metropolis theme, available under license \LaTeX{} Project
        Public License~1.3c at: \colorhref{https://github.com/matze/mtheme}
        {https://github.com/matze/mtheme}.

        Typographic design: Iwona Grabska-Gradzińska;
        \LaTeX{} Typesetting: Kamil Ziemian \\
        Proofreading: Wojciech Palacz,
        Monika Stawicka \\
        3D Models: Dariusz Frymus, Kamil Nowakowski \\
        Figures and charts: Kamil Ziemian, Paweł Węgrzyn,
        Wojciech Palacz \\
        Film editing: IMAVI -- Joanna Kozakiewicz, Krzysztof Magda, Nikodem
        Frodyma

      \end{flushleft}

    \end{textblock}

  \end{frame}





  \begin{frame}[standout]


    \begingroup

    \color{jFrametitleFGColor}

    #1

    \endgroup

  \end{frame}
}



\newcommand{\GeometryThreeDTwoSpecialEndingSlidesVideoVerThreeEN}[1]{%
  \begin{frame}[standout]

    \begin{textblock}{11}(1,0.7)

      \begin{flushleft}

        \mdseries

        \footnotesize

        \color{jFrametitleFGColor}

        This content was created as part of a project co-financed by the
        European Union within the framework of the European Social Fund
        POWR.03.05.00-00-Z309/17-00.

      \end{flushleft}

    \end{textblock}





    \begin{textblock}{10}(0,2.2)

      \tikz \fill[color=jBackgroundStyleLight] (0,0) rectangle (12.8,-1.5);

    \end{textblock}


    \begin{textblock}{3.2}(0.7,2.45)

      \includegraphics[scale=0.3]{\FundingLogoColorPictureEN}

    \end{textblock}


    \begin{textblock}{2.5}(4.15,2.5)

      \includegraphics[scale=0.2]{\JULogoColorPictureEN}

    \end{textblock}


    \begin{textblock}{2.5}(6.35,2.4)

      \includegraphics[scale=0.1]{\ZintegrUJLogoColorPictureEN}

    \end{textblock}


    \begin{textblock}{4.2}(8.4,2.6)

      \includegraphics[scale=0.3]{\EUSocialFundLogoColorPictureEN}

    \end{textblock}





    \begin{textblock}{11}(1,4)

      \begin{flushleft}

        \mdseries

        \footnotesize

        \RaggedRight

        \color{jFrametitleFGColor}

        The content of this lecture is made available under a Creative
        Commons licence (\textsc{cc}), giving the author the credits
        (\textsc{by}) and putting an obligation to share on the same terms
        (\textsc{sa}). Figures and diagrams included in the lecture are
        authored by Paweł Węgrzyn et~al., and are available under the same
        license unless indicated otherwise. \\ The presentation uses the
        Beamer Jagiellonian theme based on Matthias Vogelgesang’s
        Metropolis theme, available under license \LaTeX{} Project
        Public License~1.3c at: \colorhref{https://github.com/matze/mtheme}
        {https://github.com/matze/mtheme}.

        Typographic design: Iwona Grabska-Gradzińska;
        \LaTeX{} Typesetting: Kamil Ziemian \\
        Proofreading: Leszek Hadasz, Wojciech Palacz,
        Monika Stawicka \\
        3D Models: Dariusz Frymus, Kamil Nowakowski \\
        Figures and charts: Kamil Ziemian, Paweł Węgrzyn,
        Wojciech Palacz \\
        Film editing: Agencja Filmowa Film \& Television Production~--
        Zbigniew Masklak \\
        Film editing: IMAVI -- Joanna Kozakiewicz, Krzysztof Magda, Nikodem
        Frodyma


      \end{flushleft}

    \end{textblock}

  \end{frame}





  \begin{frame}[standout]


    \begingroup

    \color{jFrametitleFGColor}

    #1

    \endgroup

  \end{frame}
}











% ------------------------------------------------------
% BibLaTeX
% ------------------------------------------------------
% Package biblatex, with biber as its backend, allow us to handle
% bibliography entries that use Unicode symbols outside ASCII.
\usepackage[
language=polish,
backend=biber,
style=alphabetic,
url=false,
eprint=true,
]{biblatex}

\addbibresource{Systemy-operacyjne-Bibliography.bib}





% ------------------------------------------------------
% Importing packages, libraries and setting their configuration
% ------------------------------------------------------





% ------------------------------------------------------
% Local packages
% ------------------------------------------------------
% Local configuration of this particular presentation
\usepackage{./Local-packages/local-settings}

\usepackage{./Local-packages/PGF-TikZ-Arrows-styles}

\usepackage{./Local-packages/PGF-TikZ-Diagram-styles}










% ------------------------------------------------------------------------------------------------------------------
\title{Systemy operacyjne}
\subtitle{Wprowadzenie do przedmiotu}

\author{Kamil Ziemian \\
  \email}


% \date{}
% ------------------------------------------------------------------------------------------------------------------










% ####################################################################
% Beginning of the document
\begin{document}
% ####################################################################





% ######################################
% Number of chars: 38k+,
% Text is adjusted to the left and words are broken at the end of the line.
\RaggedRight
% ######################################





% ######################################
\maketitle
% ######################################





% ##################
\begin{frame}
  \frametitle{Spis treści}


  \tableofcontents

\end{frame}
% ##################





% ######################################
\section{Informacje wstępne}
% ######################################



% ##################
\begin{frame}
  \frametitle{Informacje wstępne}


  Obawiam~się, że~na tych konkretnych zajęciach będzie sporo przynudzania,
  ale nie widzę sposobu, by~tego uniknąć.

  Według mnie to zajęcia są dla studentów, nie studenci dla zajęć. Tak samo
  ja jestem tu dla Państwa, a~nie Państwo dla mnie. Jestem tu po to, by
  pomóc Państwu pomóc zaznajomić~się z~pracą z~bardzo potężnym, ale też
  często topornym systemem operacyjnym GNU/Linux. W~związku z~tym,
  ja będę Państwa rozliczał tylko i~wyłącznie z~umiejętności i~wiedzy,
  z~niczego innego. Wychodzę bowiem z~założenia, że~Państwo sami najlepiej
  wiedzą, czemu warto poświęcić swój czas. (Choć jak wiadomo, nie jeden raz
  potem stwierdzamy, że~nasz wybór mógł być jednak lepszy.)

  \textbf{Pytanie.} Kto z~Państwa \alert{nie} używa na co dzień systemu
  operacyjnego GNU/Linux?

\end{frame}
% ##################





% ##################
\begin{frame}
  \frametitle{Rozwój systemu GNU/Linux}

  \vspace{-0.5em}


  \begin{figure}

    \label{fig:Evolution-of-OS}

    \centering


    \includegraphics[scale=0.3]
    {./Presentations-pictures/Miscancellous-pictures/Evolution-of-operating-systems.jpg}

  \end{figure}

\end{frame}
% ##################





% ##################
\begin{frame}
  \frametitle{Korzystanie z~systemu GNU/Linux}


  Jeśli ktoś z~Państwa chce korzystać u~siebie z~systemu operacyjnego
  GNU/Linux i~ma problem z~jego zainstalowaniem proszę do mnie zwrócić
  bezpośrednio, lub pisząc pod adres \email. W~przypadku e-maila proszę mu
  nadać tytuł, który informuje o~treści wiadomości, jak
  \textit{Prośba o~pomoc w~zainstalowaniu systemu GNU/Linux}. Zwykle
  dostaję dużo wiadomości z~zadaniami domowymi, jedna o~niewiele mówiącym
  tytule \textit{Systemy operacyjne} łatwo mi umknie, a~wszak wymaga ona
  szybkiej odpowiedzi. To naprawdę ważne, dla tego powtórzę tę prośbę
  jeszcze parę razy.

  Na zajęciach nie tylko można, ale \alert{należy} zadawać pytania
  na~dowolne związany z~nimi temat. W~szczególności \alert{należy} zadawać
  pytania, jeśli coś jest niezrozumiałe albo niejasne. Ten kurs ma być
  wstępem do systemu operacyjnego GNU/Linux, \alert{nie} zakładamy, że~są
  Państwo doświadczonymi administratorami tego systemu. Byłoby to bardzo
  niewłaściwe założenie.

\end{frame}
% ##################





% ##################
\begin{frame}
  \frametitle{Zadawanie pytań}


  Proszę pamiętać, że~gdy chodzi o~tematy związane z~zajęciami \alert{nie}
  ma pytań zbyt elementarnych czy głupich. Są~tylko niezadowalające
  odpowiedzi udzielane na~Państwa pytania. Jesteśmy tutaj by Państwu pomóc
  w~zaznajomieni~się z~pracą z~systemem operacyjnym na głębszym poziomie,
  pytania z~Państwa strony bardzo ułatwiają nam to zadanie. Zadawanie pytań
  nie oznacza, że~ktoś czegoś nie umie, tylko że~chce~się czegoś nauczyć.

  Pytanie można i~\alert{należy} zadawać w~trakcie zajęć, po zajęciach, jak
  też pisząc na adres \email. Przy pisaniu na e-maila prosimy o~nadawanie
  tytułów postaci \textit{Pytanie o~X}, \textit{Pytanie z~przedmiotu
    „Systemy operacyjne”},~etc. Nie chcę by tego typu wiadomości zginęły
  wśród wielu innych, które mogą poczekać, a~brak dobrego tytuły niestety
  często do tego prowadzi. Proszę mieć trochę wyrozumiałości dla mnie.

\end{frame}
% ##################





% ##################
\begin{frame}
  \frametitle{Będziemy wiele upraszczać}


  Pytania typu „Jaki oceniam obecny kierunek rozwoju gry \textit{Path~of
    Exile~2}?” musimy jednak zostawić na~czas po zajęciach.

  Ten kurs jest tylko \alert{wstępem} do~systemów operacyjnych. Ze~względu
  na rozmiar i~poziom skomplikowania tej dziedziny, dużą liczbę rzeczy
  będziemy musieli \alert{upraszczać}. Proszę więc pamiętać, że~choć
  staramy~się przekazywać rzetelną wiedzę, to to co mówimy będzie w~wielu
  wypadkach bardzo mocny uproszczeniem rzeczywistości.

\end{frame}
% ##################










% ######################################
\section{Podstawy GNU/Linuxa: włączanie powłoki}
% ######################################


% ##################
\begin{frame}
  \frametitle{Powłoka BASH}

  \vspace{-0.5em}


  \begin{figure}

    \label{fig:BASH-shell}

    \centering


    \includegraphics[scale=0.23]
    {./Presentations-pictures/Miscancellous-pictures/BASH-shell.png}


    \caption{Przykładowy wygląd włączonej powłoki \textsc{bash}, naszego
      głównego narzędzia pracy na tym przedmiocie.}

  \end{figure}

\end{frame}
% ##################





% ##################
\begin{frame}
  \frametitle{Postęp technologiczny}

  \vspace{-0.5em}


  \begin{figure}

    \label{fig:Evolution-of-OS}

    \centering


    \includegraphics[scale=0.3]
    {./Presentations-pictures/Miscancellous-pictures/Evolution-of-operating-systems.jpg}

  \end{figure}

\end{frame}
% ##################





% ##################
\begin{frame}
  \frametitle{Włączanie powłoki}


  Naszym główny narzędziem na tych zajęciach będzie powłoka \textsc{bash}.
  Jej włączenie na samym początku potrafi być bardzo niewdzięcznym zadaniem,
  proszę spróbować to teraz zrobić. W~pierwszym kroku proszę wcisnąć
  \texttt{Ctrl-Alt-t} i~zobaczyć, czy powłoka~się otworzyć.

  Jeśli to nie zadziała, to proszę pochodzić kursorem po ekranie,
  aż~znajdziemy okienko „Szukaj”, „Wyszukaj”, etc. Proszę tam wpisać
  „Terminal”, „Konsola” lub „Consol” i~zobaczyć, czy wyświetli~się
  odpowiednia ikona. O~tym czy~się różni pojęcie „powłoki” od „terminala”
  czy „konsoli” powiem potem. Nie jest to jakaś szczególnie ważna rzecz.

\end{frame}
% ##################










% ######################################
\section{Zajęcia, konsultacje, sposób zaliczenia}
% ######################################


% ##################
\begin{frame}
  \frametitle{Charakter zajęć}


  Wykład zwykle ma bardziej charakter teoretyczny, a~te zajęcia
  \alert{praktyczny}. Pewna ilość teorii musi~się tu pojawić,
  ale~będziemy~się starali ją minimalizować.

  Możemy ustalić jeden termin na konsultacje, ale moje doświadczenie mówi,
  że~to nie jest dobry pomysł. W~zasadzie nikt wtedy nie przychodzi, a~ja
  wyznaję zasadę, że~konsultacje są dla Państwa, nie dla mnie. Jeśli Państwo
  chcą ustalenia takiego terminu, to najlepiej byłoby, gdy Państwo ustalili
  wspólnie jakiś termin, a potem przesłali mi informacji o~swoim wyborze.

  W~przeciwny razie zawsze można do mnie napisać z~prośbą o~konsultacje
  indywidualne, choćby pisząc na~adres \email. Ponawiam prośbę o~nadawanie
  e-mailom tytułów typu \textit{Prośba o~konsultacje}, \textit{Konsultacje
    z~przedmiotu „Systemy operacyjne”},~etc.

\end{frame}
% ##################





% ##################
\begin{frame}
  \frametitle{Nie lubię zbyt poważnych zajęć, ale\ldots}

  \vspace{-0.5em}


  \begin{figure}

    \label{fig:Jak-to-bywa-na-zajeciach}

    \centering


    \includegraphics[scale=0.42]
    {./Presentations-pictures/Miscancellous-pictures/Jak-to-bywa-na-zajeciach.jpeg}

  \end{figure}

\end{frame}
% ##################





% ##################
\begin{frame}
  \frametitle{Nagrywanie zajęć}


  Będę~się starał nagrywać na \textsc{ms}~Teamsach każde nasze spotkanie.
  Proszę mi o~tym ciągle przypominać, bo jestem roztrzepany i~któregoś
  razu o~tym zapomnę.

  Proszę mi też zwracać uwagę, że~na ekranie czegoś nie widać,
  że~czcionka za mała, że~kolory kłują w~oczy, że~nagrany dźwięk
  jest niskiej jakości,~etc. Zajęcia są dla Państwa, naszym obowiązkiem jest
  dostarczyć Państwu najlepszej jakości materiały do nauki jakie jesteśmy
  w~stanie w~danej chwili stworzyć.

  Niestety, jakość dźwięku to coś, na co mamy mały wpływ. Mogę~się starać
  mówić możliwie blisko mikrofonu, ale nie wiem co więcej mogę zrobić.
  Poza tym, na~pewno nie wyjdzie zbyt dobrze, bo te zajęcia często wymagają
  bym~się poruszał, poza tym w~którymś momencie na pewno o~tym zapomnę.
  Swoje uwagi na temat jakości nagrań proszę kierować do ludzi
  odpowiedzialnych za~sprawy studenckie na \textsc{wsz}i\textsc{b}ie.

\end{frame}
% ##################





% ##################
\begin{frame}
  \frametitle{Sposób uzyskania zaliczania}


  Można uzyskać zaliczenie zaoczne, proszę~się w~tej sprawie zwrócić
  do~mnie, osobiście lub piszą pod adres \email. Jak poprzednio, proszę
  o~nadanie e-mailowi nazwy tłumaczącej jego treść.

  Zaliczenie w~normalnym trybie zdobywa~się na podstawie zadań domowych
  i~jednego albo dwóch testów, za~które otrzymają Państwo punkty. Za każdy
  z~testów będzie do zdobycie 10 punktów. Punktacje zestawów zadań zależą
  od tego jak oszacujemy ich poziom trudności i~nie jesteśmy w~stanie podać
  liczby punktów do zdobycia za nie w~tym momencie.

  Zastanawiam~się czy nie dodać do tego projektu zaliczeniowego, ale są
  pewne problemy techniczne, które stoją nam na drodze. Na dzień dzisiejszy
  proszę przyjąć, że~projekt zaliczeniowego w~tym semestrze nie będzie.

\end{frame}
% ##################





% ##################
\begin{frame}
  \frametitle{Prace domowe}


  Zadania domowe proszę \alert{spróbować} potraktować nie jako ciężki
  obowiązek do odhaczenia, tylko jako możliwość nauczenia~się czegoś
  w~praktyce. Tak wiem, łatwo~się mówi, żeby nie traktować tego jako
  ciężkiego obowiązku, a~to wcale nie jest łatwe.

  Proszę jednak wiedzieć, że~prace domowe są zadawane w~pierwszym rzędzie
  po~to, żeby Państwo mogli~się czegoś nauczyli przy ich robieniu. Dopiero
  w~drugim rzędzie, by móc wystawić Państwu oceny. Jeśli Państwo oszukują
  przy oddawaniu tych zdań, to z~naszej perspektywy, oszukują Państwo przede
  wszystkim samych siebie.

  W~przypadku wszystkich prac domowych, to należy próbować rozwiązać je
  możliwie samodzielnie. W~razie napotkania problemów nie tylko można,
  ale i~\alert{należy} prosić o~pomoc kolegów, korzystać z~materiałów
  w~internecie i~używać programów takich jak Chat\textsc{gpt}. Można też
  pisać do nas po adres \email.

\end{frame}
% ##################





% ##################
\begin{frame}
  \frametitle{Sposób uzyskania zaliczania}


  Przypominam, że~głównym celem prac domowych jest danie Państwu dodatkowej
  możliwość, by~się czegoś nauczyć, a~w~nauce można i~\alert{należy}
  korzystać z~pomocy. Po prostu ich zrobienie nie ma zostać zredukowane
  do~użycia metody Copy’ego-Pasta.

  Pod koniec semestru punkt zostaną podliczone, a~oceny wystawione według
  następującej skali.

  \begin{itemize}

  \item $41\%\text{--}50\%$~-- ocena dostateczna ($3.0$).

  \item $51\%\text{--}60\%$ -- ocena plus dostateczna ($3.5$, $3+$).

  \item $61\%\text{--}70\%$ -- ocena dobra ($4.0$).

  \item $71\%\text{--}84\%$ -- ocena puls dobry ($4.5$, $4+$).

  \item $85\%\text{--}100\%$ -- ocena bardzo dobry ($5.0$).

  \end{itemize}

\end{frame}
% ##################










% ######################################
\section{Materiały do nauki}
% ######################################


% ##################
\begin{frame}
  \frametitle{Materiały do nauki}


  Na Sake będzie dostępna w~formacie \textsc{pdf} \alert{lista zagadnień do
    opanowania z~tego przedmiotu}, która będzie główny punktem odniesieniem
  przy tworzeniu pytań testowych. Jak również różne materiały do nauki,
  jakie to dokładnie będą materiały, nie umiem powiedzieć w~tym momencie.

  Będą tam również dostępne te prezentacje w~formacie \textsc{pdf}ów.
  W~formie źródłowej (plików \LaTeX a) są dostępna na serwisie GitHub.
  Każdy kto ma na komputerze program Git i~dostęp do internetu może je
  pobrać wpisując \\
  \texttt{\$ git clone https://github.com/KZiemian/Presentation} \\
  Znajdują~się one w~katalogu „Systemy-operacyjne-Prezentacje”. Należy
  dodać, że~gałąź \texttt{main} jest zwykle w~tyle
  za~\texttt{Gałąź-robocza}.

  Można też obejrzeć to repozytorium jak normalny człowiek. Czyli
  w~przeglądarce: \\
  \colorhref{https://github.com/KZiemian/Presentation}
  {https://github.com/KZiemian/Presentation}.

\end{frame}
% ##################





% ##################
\begin{frame}
  \frametitle{Materiały do nauki}


  Podstawowo książką do systemów operacyjne są
  \textit{Systemy operacyjne. Wyd.~V} Andrewa S.~Tanenbauma i~Bos Herberta.
  Pierwszy jej rozdział to prawdziwa skarbnica wiedzy o~informatyce, którą
  każdy informatyk powinien znać. Jeśli Państwo chcą to udostępnię na Sake
  zdjęcia pierwszego rozdziału mojego egzemplarza, który należy do
  trzeciego wydania tej pozycji.

  Jeśli jakieś zdjęcia są nieczytelne, proszę do mnie napisać. Do tej
  pory~się nikt nie skarżył, ale też chyba mało kto te zdjęcia czytał.

  Jest wiele pozycji do nauki systemu operacyjnego GNU/Linux, jeśli ktoś
  chce, podam kilka z~nich.

\end{frame}
% ##################










% ######################################
\section{Systemy operacyjne wokół nas}
% ######################################


% ##################
\begin{frame}
  \frametitle{Jakie systemy operacyjne są dziś używane?}


  Obecnie na komputerach osobistych dominują trzy rodziny systemów
  operacyjnych (\textsc{os}, ang. \textit{operating system}). Są to:
  \begin{itemize}

  \item GNU/Linux (w~skrócie: Linux);

  \item mac\textsc{os};

  \item Windows.

  \end{itemize}

  Gdy chodzi o~smartfony, to istnieją co~najmniej dwie duże rodziny systemów
  operacyjnych dla nich:
  \begin{itemize}

  \item Android;

  \item i\textsc{os}.

  \end{itemize}
  Niemniej ja nie jestem ekspertem w~kwestii smartfonów, więc nie będę
  wnikał w~tą tematykę.


\end{frame}
% ##################





% ##################
\begin{frame}
  \frametitle{Jakie systemy operacyjne są dziś używane?}


  Jeśli chodzi o~inne systemy operacyjne to możemy wymienić i~wymieniać:
  Fire~\textsc{os}, Free\textsc{bsd}, Free\textsc{dos}, Haiku,
  Harmony\textsc{os}, Heli\textsc{os}, Inferno, \textsc{minix},
  OpenHarmony, OpenSolaris, Phantom~\textsc{os}, Plan~9
  from Bell Labs, React\textsc{os}, Redox (bardzo interesujący projekt),
  Thesueus~\textsc{os} (inny interesujący projekt), Visopsys, etc.

  Poza tym, systemy operacyjne można podzielić na klasy, w~zależności od
  tego na jakim sprzęcie mają one działać. Listę podstawowych klas można
  znaleźć poniżej.

  \begin{itemize}

  \item Systemy operacyjne komputerów mainframowych. Ten rodzaj komputerów
    spotyka~się zwykle w~centrach obliczeniowych różnorakiego
    przeznaczenia. Tutaj często używane są specjalne dystrybucje Linuxa,
    jak~również systemy operacyjne tworzone specjalnie dla nich, jak~z/OS.

  \end{itemize}

\end{frame}
% ##################





% ##################
\begin{frame}
  \frametitle{Różne typy systemów operacyjnych}


  \begin{itemize}

  \item Systemy operacyjne serwerów. Tutaj również Linux jest popularny.

  \item Systemy operacyjne komputerów osobistych. O~nich jest ten kurs.

  \item Systemy operacyjne smartfonów, jak Android czy i\textsc{os}.

  \item Wbudowane systemy operacyjne. Są to systemy operacyjne obsługujące
    pralki, kuchenki mikrofalowe, samochody, etc.

  \item Systemy operacyjne kart elektronicznych. Temat na inne zajęcia.

  \item Systemy operacyjne węzłów sensorowych. Przykładem węzła sensorowego
    są czujniki przeciwpożarowe, układy do~mierzenia temperatury, ilości
    opadów, etc.

  \end{itemize}

  Dodatkowo systemy operacyjne można podzielić na monolityczne,
  wielowarstwowe, etc.

\end{frame}
% ##################





% ##################
\begin{frame}
  \frametitle{Komputery i~systemy operacyjne}


  Tak jak nasz świat jest pełen komputerów, tak jest pełen systemów
  operacyjnych. Bez wielkiej przesady można powiedzieć, że~bez systemów
  operacyjnych setki milionów komputerów stałby~się dla większości ludzi
  bezużyteczną stertą złomu.

  \alert{Nie} muszą się Państwo uczyć powyższej listy. Jedyne co chcemy by
  Państwo z~niej wynieśli, to świadomość, że~tematyka systemów operacyjnych
  jest dość skomplikowana i~należy odczuwać zdrowy respekt wobec
  ludzi którzy te systemy tworzą. Pamiętając, że~ci sami ludzie
  całkiem dużo rzeczy mocno spaprali.

  Informatyka w~ogólności, a~systemy operacyjne w~szczególności, pełna jest
  maszyn
  \colorhref{https://en.wikipedia.org/wiki/Rube\_Goldberg\_machine}{Rube
    Goldbergera}. Państwo już na pewno nie raz
  widzieli taką maszynę, a~jedną z~nich można obejrzeć
  \colorhref{https://www.youtube.com/watch?v=vn-g1Mn2\_3g}{tutaj}.
  Bardzo dużo w~informatyce można uprościć, ale coś mi mówi, że~zajmie nam
  to jeszcze dużo czasu.

\end{frame}
% ##################










% ######################################
\section{Abstrakcje i~interfejsy}
% ######################################


% ##################
\begin{frame}
  \frametitle{Czym jest interfejs?}


  Do kluczowych pojęć informatyki, systemów operacyjnych w~szczególności,
  należą „abstrakcja” i~„interfejs”. By wyjaśnić sens w~jakim będziemy ich
  używać, posłużymy~się przykładem smartfona. Jest to układ elektroniczny,
  posiadają rozliczne kable przez które płynie prąd, akumulator,
  procesor,~etc. Jednak na co dzień nie muszę o~tym myśleć, obsługuję go
  klikając w~ikony.

  Ponieważ ikony pozwalają mi zapomnieć o~elektronicy wewnątrz smartfona,
  są one rodzaje abstrakcji. Ponieważ pozwalają mi~się komunikować z~nim,
  są~one rodzajem interfejsu. Teraz trochę ostrzej zdefiniujemy te pojęcia.

  \textbf{Interfejsem} (pl.~\textit{międzymordzie} ;)) nazywamy metodę
  komunikowania~się dwóch obiektów. W~tym sensie można powiedzieć,
  że~telefony komórkowe~są interfejsami między ludźmi, lecz my zawęzimy
  nasze rozważania do~interfejsów właściwych komputerom.

\end{frame}
% ##################





% ##################
\begin{frame}
  \frametitle{Czym jest abstrakcja w~informatyce?}


  Ikony w~smartfonie są więc interfejsem, bo pozwalają na komunikację
  między człowiekiem, a~tym urządzeniem.

  \textbf{Abstrakcją} będziemy nazywać zmianę sposobu opisu rzeczywistości,
  na taki, który jest z~jakiegoś powodu bardziej użyteczny dla danej grupy
  ludzi. Ikona na smartfonie jest w~tym sensie abstrakcją, bo dzięki ich
  istnieniu nie musimy myśleć o~tym jak płynie prąd przez smartfona.
  Klikamy w~ikonę i~prąd popłynie w~odpowiedni sposób, zdefiniowany przez
  tą abstrakcję.

  Nie będziemy bardziej precyzować tych pojęć, gdyż na tym kursie ważne
  jest tylko intuicyjne ich zrozumienie. Nie będziemy też
  specjalnie dyskutować co odróżnia dobrą abstrakcję od~złej oraz dobry
  interfejs od złego. Należy jednak zauważyć, że~problem dostarczenia
  dobrych abstrakcji i~interfejsów przewija~się przez całą informatykę.

\end{frame}
% ##################





% ##################
\begin{frame}
  \frametitle{Abstrakcje i~interfejsy są wszędzie}


  Warto zauważyć, że~coś może być jednocześnie abstrakcją i~interfejsem,
  jak ikony w~smartfonie. W~temat relacji między tymi pojęciami nie
  będziemy~się jednak zagłębiać.

  Abstrakcje i~interfejsy w~informatyce są wszędzie i~występują warstwowo:
  gdy zdejmę jedną warstwę abstrakcji (interfejsu) pod nią zwykle zobaczę
  kolejną warstwę abstrakcji.

  Rozpatrzmy następujący przypadek, który nie musi być specjalnie zgodny
  z~rzeczywistością. Załóżmy, że~mam program do przeglądania zdjęć, który
  oczywiście sam w~sobie jest abstrakcją. Gdy zdejmę tą warstwę abstrakcji,
  to ukaże mi~się kod źródłowy napisany w~języku Python. Jednak kod
  źródłowy w~języku Python sam jest \alert{abstrakcją}, bo sam ten język
  (jego interpreter) jest napisany w~języku~C. Język~C jest zaś abstrakcją
  wobec dialektu assemblera naszego komputer. Język assemblera też jest
  pewną abstrakcją, ale proszę, nie idźmy dalej.

\end{frame}
% ##################










% ######################################
\section{Do czego służy system operacyjny?}
% ######################################


% ##################
\begin{frame}
  \frametitle{Do czego służy system operacyjny?}


  Komputer składa~się z~wielu części elektronicznych, takich jak procesor,
  dysk główny, pamięć \textsc{ram}, monitor,~etc. Każdy z~tych elementów
  posiada własny interfejs, dzięki któremu możemy~się z~nim komunikować,
  wysyłając odpowiednie polecenia i~otrzymując odpowiednie informacje
  zwrotne. Interfejsy te są, koniec końców, oparte o~odpowiedni przepływ
  prądu elektrycznego. Interfejsy te są prymitywne, niewygodne w~użyciu
  i~brzydkie. Możemy o~nich powiedzieć więcej, ale wątpię by Państwo byli
  tym zainteresowani.

  System operacyjny jest programem, który zarządza tymi nieprzyjemnymi
  rzeczami za nas. Inaczej mówiąc, jest on abstrakcją nałożoną na~sprzęt
  komputerowy. Gdy korzystamy z~dowolnego programu użytkowego, ten program
  kontaktuje~się przez odpowiedni interfejs z~systemem operacyjnym,
  a~system operacyjny komunikuje~się ze~sprzętem za pomocą interfejsu tego
  sprzętu. System operacyjny pełni tutaj rolę pośrednika.

\end{frame}
% ##################





% ##################
\begin{frame}
  \frametitle{Uproszczony schemat działania programu}


  \begin{figure}

    \label{fig:Scheme-of-CPU}


    \begin{tikzpicture}

      \fill[color=brown] (-2.5,0) rectangle (2.5,1.2);

      \node[color=white,scale=1.1] at (0,0.6) {Hardware};


      \fill[color=teal] (-2.5,1.2) rectangle (2.5,1.8);



      \draw[pointing arrow] (3.5,1.5) -- (2.5,1.5);

      \node at (3.7,1.5) {B};

      \node at (4,1.4) {rz};

      \node[rotate=20] at (4.3,1.5) {y};

      \node[rotate=-15] at (4.5,1.4) {d};

      \node at (4.75,1.6) {k};

      \node at (4.9,1.5) {i};


      \node at (5.7,1.475) {interfejs};

      \node at (4.5,1) {hardware’u};



      \fill[color=cyan] (-2.5,1.8) rectangle (2.5,3);

      \node[color=white,scale=1.1] at (0,2.4) {System operacyjny};


      \draw[thick diagram arrow] (-2,2.1) -- (-2,0.7);

      \draw[thick diagram arrow] (2,0.9) -- (2,2.3);



      \fill[color=teal] (-2.5,3) rectangle (2.5,3.6);



      \draw[pointing arrow] (3.5,3.3) -- (2.5,3.3);

      \node[text width=8em] at (5,3.025)
      {Piękny interfejs systemu op.};


      \fill[color=jAxisGreen] (-2.5,3.6) rectangle (2.5,4.8);

      \node[color=white,scale=1.1] at (0,4.2) {Program użytkowy};


      \draw[thick diagram arrow] (-2,3.9) -- (-2,2.5);

      \draw[thick diagram arrow] (2,2.7) -- (2,4.1);

    \end{tikzpicture}

    \caption{Uproszczony schemat łączenia~się programu komputerowego
      ze~sprzętem, za pośrednictwem systemu operacyjnego. Rysunek wzorowany
      na ilustracji 1.2 ze str. 34,
      \parencite{Tannenbaum-Systemy-Operacyjne-Wydanie-III-Pub-2013}.}


  \end{figure}

\end{frame}
% ##################







% ##################
\begin{frame}
  \frametitle{Architektura systemu operacyjnego}


  Ponieważ system operacyjny jest abstrakcją, jego interfejs który
  nam udostępnia, może być znacznie bardziej wyrafinowany, prostszy
  w~użyciu i~piękniejszy od~interfejsu sprzętowego.

  W~tym momencie potrzebujemy wprowadzić nowe pojęcie.
  \textbf{Architekturą systemu operacyjnego} nazywamy ogólny plan budowy
  danego systemu operacyjnego. Architektura takiego systemu jest więc
  zbiorem idei, których używamy do tworzenia oraz opisywania systemów
  operacyjnych. Jak poprzednio, nie będziemy potrzebowali ściślejszej
  definicji tego pojęcia.

  Podstawowa dziś architektura stosowana dzisiaj, mówi nam, że~system
  operacyjny może działać w~dwóch różnych trybach: trybie jądra i~trybie
  użytkownika. Żeby zrozumieć czemu ten podział został wprowadzony,
  wróćmy do zagadnienia: po co istnieje system operacyjny?

\end{frame}
% ##################





% ##################
\begin{frame}
  \frametitle{Rola systemu operacyjnego}


  System ten istnieje, by zdjąć z~nas obowiązek bezpośredniego zarządzania
  sprzętem, robienia takich rzeczy jak ręczne ustawienie wartości rejestrów
  procesora, bo normalny użytkownik nie chce mieć nigdy do czynienia
  z~takimi rzeczami. Oprócz tego, że~życie użytkownika staje~się prostsze,
  to jest też bezpieczniejsze, bo pracując na niskim poziomie abstrakcji,
  łatwiej o~pomyłki i~poważne błędy. Przykładowo, potrzebuje mieć możliwość
  kasowania danych na dysku. Co jeśli przypadkiem zamiast skasować jeden
  plik wydamy polecenie, by~skasować całą zawartość dysku?

  Co więcej, jeśli kilka osób ma dostęp do tego samego komputera, to
  należy zapewnić, żeby dane jednej osoby, które powinny pozostać tajne, nie
  były od ręki dostępne innym użytkownikom tego komputera.

\end{frame}
% ##################





% ##################
\begin{frame}
  \frametitle{Tryby działania systemu operacyjnego}


  Aby zapewnić możliwie dużą wygodę, bezpieczeństwo i~tajność danych,
  w~obecnej architekturze systemów operacyjnych możliwe są dwa tryby
  działania: \textbf{tryb jądra} (ang. \textit{kernel mode})
  i~\textbf{tryb użytkownika} (ang. \textit{user mode}). Należy pamiętać,
  że~to są dwa różne tryby działania tego samego programu.

  Działając w~trybie jądra, możemy korzystać z~wszystkich możliwości jakie
  udostępnia nam sprzęt oraz widzimy całą pamięć. Tryb ten musi istnieć,
  bo~wszystkie operacje sprzętu są potrzebne do działania komputera, więc
  ktoś musi móc je uruchomić. Tak samo, ktoś musi mieć dostęp do całej
  pamięci, aby móc przydzielać ją innym.

  Następnie jest tryb użytkownika. Gdy jesteśmy w~tym trybie, system
  operacyjny wyłącza nam dostęp do pewny możliwości naszego komputera.
  W~szczególności, jeśli jakaś część pamięci jest przydzielona innemu
  użytkownikowi, jest ona dla nas niedostępna w~tym trybie.

\end{frame}
% ##################





% ##################
\begin{frame}
  \frametitle{Jak to działa?}


  Jak tego typu ograniczenia działają w~praktyce? System operacyjny jest
  programem jak każdy inny, więc jego działanie zależy od~ustawień. Jeśli
  działając jako użytkownik chcemy zobaczyć zawartość jakiegoś fragmentu
  pamięci, to wysyłamy zapytanie do systemu operacyjnego o~jej wyświetlenie
  nam. System, który zawsze widzi całą dostępną pamięć, sprawdza, czy ma
  prawo udostępnić nam ten fragment pamięci. Jeśli nie to przesyła nam
  wiadomość, że~nie mamy uprawnień do zobaczenia zawartości tego fragmentu
  pamięci.

  Jak można~się z~tego domyślić, jeśli zdobędę dostęp do trybu jądra, mogę
  z~tym komputerem zrobić wszystko. Dlatego hakerzy dokładają tylu starań,
  by uzyskać do niego dostęp \alert{bez} posiadania odpowiednich uprawnień.
  Dlatego też takie antycheaterskie programy takie jak Vanguard, który miał
  być dodany choćby do
  \colorhref{https://www.youtube.com/watch?v=nk6aKV2rY7E}{\textit{League~of
      Legends}}~są tak problematyczne
  \parencite{Low-Level-Why-Riots-anti-cheat-is-a-HUGE-problem-Ver-2024}.

\end{frame}
% ##################





% ##################
\begin{frame}
  \frametitle{Problemy z~bezpieczeństwem}


  Vanguard działa bowiem w~trybie jądra, ma więc możliwość by zrobić
  z~moich komputerem absolutnie wszystko co chce. Czy ufamy firmie
  tworzącej ten program, że~nie nadużyją tych możliwości?

  Pytanie, czemu w~ogóle pozwalamy istnieć taki programom jak Vanguard?
  Dlatego, że~jeśli program ten pracuje w~trybie jądra to do wyszukiwania
  cheaterów używam zasobów \alert{naszych} komputera. Jeśli pracowałby
  w~trybie użytkownika, nie byłoby to możliwe, gdyż z~definicji trybu
  użytkownika, program ten nie miałby dostępu do informacji potrzebnych by
  wykryć cheatera. Alternatywą jest uruchamianie tych programów na serwerze
  gry, który i~tak dysponuje wszystkimi tego typu informacjami.


\end{frame}
% ##################





% ##################
\begin{frame}
  \frametitle{System operacyjny a~zarządzanie zasobami}


  Jak powiedzieliśmy wcześniej, system operacyjny jest abstrakcją nałożoną
  na interfejsy procesora, dysków, monitora,~etc. Jednak aby wypełniał
  on swoją rolę jako abstrakcja, to musi być on odpowiednio
  \alert{zarządzać} wszystkimi tymi zasobami. Musi umieć przydzielać czas
  procesora, przestrzeń na dysku,~etc. Jest to jakby druga strona działania
  systemu operacyjnego. Należy zaznaczyć, że~zarządzaniem tymi zasobami nie
  jest wcale rzeczą prostą.

  By sprostać temu zadaniu system operacyjny sam dzieli~się na różne,
  jakżeby inaczej, warstwy abstrakcji. Na tych zajęciach zajrzymy pod kilka
  z~nich, aby zobaczyć jak system tam działa i~jak korzystając z~tej wiedzy
  możemy go kontrolować.

\end{frame}
% ##################










% ######################################
\section{Czy ten przedmiot będzie trudny?}
% ######################################



% ##################
\begin{frame}
  \frametitle{Czy ten przedmiot będzie trudny?}


  Informatyka to osobna dziedzina nauki i~jeśli zabrnie~się odpowiednio
  głęboko, to robi~się naprawdę złożona i~niebanalna. Używając przed chwilą
  wprowadzonej terminologi, powiem, że~informatyka robi~się niebanalna, gdy
  zdejmiemy odpowiednio dużo warstw abstrakcji. Na wysokim poziomie
  abstrakcji to czy jest on trudna czy nie, to mocno zależy od~odczuć
  konkretnej osoby.

  Zadajmy sobie następujące pytanie: czy włączenie komputera jest
  skomplikowane? Odpowiemy na to pytanie na dwóch poziomach abstrakcji.
  Pierwszy to poziom normalnego użytkownika, drugi to opis pochodzący
  z~książki Andrewa S.~Tanenbauma \textit{Systemy operacyjne. Wydanie~III}
  \parencite{Tannenbaum-Systemy-Operacyjne-Wydanie-III-Pub-2013}, dotyczący
  komputera z~systemem Pentium.

\end{frame}
% ##################





% ##################
\begin{frame}
  \frametitle{Włączanie komputera, poziom normalnego użytkownika}


  \begin{enumerate}

  \item Wciskamy przycisk \texttt{Power}.

  \item Czekamy minutę albo dłużej.

  \item Wybieramy użytkownika i~wchodzimy na swoje konto.

  \end{enumerate}



  Co w~tym trudnego?

\end{frame}
% ##################





% ##################
\begin{frame}
  \frametitle{Kilka pojęcia}


  Oczywiście, opis włączania komputera z~książki Tanenbauma jest tak
  skomplikowany, że~trzeba wprowadzić trochę pojęć wstępnych.

  \textbf{\textsc{rom}}, ang.~\textit{Read Only Memory}, pl.~\textit{pamięć
    wyłącznie do~odczytu}. Pamięć komputera której zawartość została
  zapisana przez firmę, która ten fragment pamięci wyprodukowała
  i~użytkownik nie może zmodyfikować jej zawartości. Przynajmniej nie żadnym
  normalny sposobem.

  \textbf{\textsc{ram}}, ang.~\textit{Random Access Memory},
  pl.~\textit{pamięć o~dostępie w~trybie losowym}. Pamięć komputera o~tej
  własności, że~jeśli wylosuję dowolny jej element, to czas odczytania
  informacje z~tego elementu nie będzie zależał od tego, który element
  został wylosowany. Inaczej mówiąc dostęp do dowolnego miejsca tej pamięci
  zajmuje tyle samo czasu.

  Tak naprawdę czas odczytu zależy w~pewnym stopniu od tego, w~jaki
  konkretny sposób pamięć \textsc{ram} jest odczytywana, ale jeszcze długo
  nie będziemy się musieli tym przejmować.

\end{frame}
% ##################





% ##################
\begin{frame}
  \frametitle{Kilka pojęcia}


  \textbf{Pamięć ulotna}, ang.~\textit{volatile memory}. Pamięć której
  zawartość jest tracona, gdy przestaje przez nią płynąć prąd. Typowym
  przykładem takiej pamięci jest \textsc{ram}.

  \textbf{Pamięć nieulotna}, ang.~\textit{non-volatile memory}. Pamięć,
  której treść jest zachowana, gdy przez układ przestaje płynąć prąd,
  typowym przykładem jest dysk \textsc{ssd}.

  Żeby skomplikować życie, pamięcią nieulotną nazywa~się także tą pamięć,
  które jest ulotna w~ścisłym sensie, ale ponieważ jest zaopatrzona
  we~własną baterię, jej zawartość jest zachowana również po wyłączeniu
  komputera z~prądu. Bo~niby czemu życie ma być proste?

\end{frame}
% ##################





% ##################
\begin{frame}
  \frametitle{Kilka pojęcia}


  \textbf{Pamięć \textsc{cmos}}, często po prostu \textbf{\textsc{cmos}}.
  Skrót pochodzi od nazwy technologi ang.~\textit{Complementary
    Metal-Oxide-Semiconductor}, pl.~\textit{komplementarny półprzewodnik
    metalowo-tlenkowy}, w~której ta pamięć jest wykonana. Musi być zasilana
  prądem, by~zachowywała swój stan, ale ponieważ wyposażona jest w~baterię
  klasyfikowana jest jako nieulotna.

  \textbf{\textsc{bios}} ang.~\textit{Basic Input Output System}, pl.
  \textit{podstawowy system wejścia, wyjścia}. Program znajdujący~się
  na płycie głównej komputera, odpowiedzialny między innymi za odczytywanie
  klawiatury, zapisywanie ekranu oraz operacje wejścia-wyjścia dysków.

\end{frame}
% ##################





% ##################
\begin{frame}
  \frametitle{Uruchamianie komputera z~systemem Pentium}


  \begin{itemize}

  \item[1)] Wciskamy przycisk \texttt{Power}.

  \item[2)] Z~płyty głównej ładowany jest program \textsc{bios}. Sprawdza on
    ilość zainstalowanej pamięci \textsc{ram}, czy komputer dysponuje
    klawiaturą i~innymi podstawowymi urządzeniami oraz sprawdza czy
    odpowiadają one w~sposób prawidłowy. W~pierwszej kolejności skanowane
    są magistrale \textsc{ISA} (ang. \textit{Industry Standard
      Architecture}) i~\textsc{pci} (ang.~\textit{Peripheral Component
      Interconnect}) w~celu wykrycia podłączonych do nich urządzeń.

  \item[3)] Jeśli do komputera podłączone są inne urządzenia, niż te które
    były dostępne przy jego ostatni uruchomieniu, nowe urządzenia są
    konfigurowane.

  \item[4)] Program \textsc{bios} odczytuje listę tzw. urządzeń rozruchowych
    z~pamięci \textsc{cmos}. Urządzenia rozruchowe to te, które zawierają
    system operacyjny. W~przeszłości były nimi dyskietki, płyty
    \textsc{cd}-\textsc{rom}, \textsc{dvd}, dziś choćby pendriwy
    i~dyski~\textsc{ssd}.

  \end{itemize}

\end{frame}
% ##################





% ##################
\begin{frame}
  \frametitle{Uruchamianie komputera z~systemem Pentium}


  \begin{itemize}

  \item[5)] \textsc{bios} testuje po kolei urządzenia rozruchowe
    z~wspomnianej wcześniej listy, aż~znajdzie pierwszy, który zawiera
    działający system operacyjny.

  \item[6)] \textsc{bios} wczytuje pierwszy sektor ze~znalezionego
    w~poprzednim punkcie urządzenia rozruchowego do pamięci i~go uruchamia.

  \item[7)] Program z~pierwszego sektora sprawdza zapisaną na jego końcu
    listę partycji, by~ustalić która z~nich jest partycją aktywną.
    Następnie wczytuje z~tej partycji pomocniczy program rozruchowy.

  \item[8)] Pomocniczy program rozruchowy wczytuje system operacyjny
    z~aktywnej partycji i~go uruchamia.

  \item[9)] System operacyjny odczytuje informacje konfiguracyjne z~systemu
    \textsc{bios}. Dla każdego dostępnego urządzenia sprawdza, czy posiada
    do niego sterowniki. Jeśli nie, to prosi o~ich zainstalowanie
    z~odpowiedniego źródła.

  \end{itemize}

\end{frame}
% ##################





% ##################
\begin{frame}
  \frametitle{Rasterization and fragment operations}


  \begin{itemize}

  \item[10)] Jeśli system operacyjny dysponuje wszystkimi sterownikami,
    to ładuje je do jądra systemu.

  \item[11)] System operacyjny tworzy tabele systemowe oraz procesy
    działające w~tle.

  \item[12)] Uruchamiane jest okno logowania.

  \end{itemize}

\end{frame}
% ##################






% ##################
\begin{frame}
  \frametitle{Bootowanie}


  W~literaturze funkcjonuje termin \textbf{bootwoanie}, zwane też
  \textbf{uruchamianiem} lub \textbf{rozruchem}. Odnosi~się ono albo do
  całej procedury uruchamiania komputer opisanej powyżej, albo tylko
  stawiania systemu operacyjnego, czyli od kiedy \textsc{bios} wczytał
  pierwszy jego sektor do pamięci (punkt siedem i~dalej). Acz to pojęcie
  nie jest specjalnie ostro zdefiniowane.

\end{frame}
% ##################





% ##################
\begin{frame}
  \frametitle{Czy uruchomienie komputera jest proste czy trudne?}


  Zależy jak do tego podchodzimy. I~tak jest z~większością rzeczy
  w~informatyce.

\end{frame}
% ##################










% ######################################
\appendix
% ######################################





% ######################################
\EndingSlide{Dziękuję! Pytania?}
% ######################################





% ####################################################################
% ####################################################################
% Bibliography

\printbibliography










% ####################################################################
% End of the document

\end{document}
