% ------------------------------------------------------------------------------------------------------------------
% Basic configuration and packages
% ------------------------------------------------------------------------------------------------------------------
% Package for discovering wrong and outdated usage of LaTeX.
% More information to be found in l2tabu English version.
\RequirePackage[l2tabu, orthodox]{nag}
% Class of LaTeX document: [size of paper, size of font]{document class}
\documentclass[a4paper,11pt]{article}



% ------------------------------------------------------
% Packages not tied to particular normal language
% ------------------------------------------------------
% This package should improved spaces in the text.
\usepackage{microtype}
% Add few important symbols, like text Celcius degree
\usepackage{textcomp}



% ------------------------------------------------------
% Polonization of LaTeX document
% ------------------------------------------------------
% Basic polonization of the text
\usepackage[MeX]{polski}
% Switching on UTF-8 encoding
\usepackage[utf8]{inputenc}
% Adding font Latin Modern
\usepackage{lmodern}
% Package is need for fonts Latin Modern
\usepackage[T1]{fontenc}



% ------------------------------------------------------
% Setting margins
% ------------------------------------------------------
\usepackage[a4paper, total={14cm, 25cm}]{geometry}



% ------------------------------------------------------
% Setting vertical spaces in the text
% ------------------------------------------------------
% Setting space between lines
\renewcommand{\baselinestretch}{1.1}

% Setting space between lines in tables
\renewcommand{\arraystretch}{1.4}





% ------------------------------------------------------
% BibLaTeX
% ------------------------------------------------------
% Package biblatex, with biber as its backend, allow us to handle
% bibliography entries that use Unicode symbols outside ASCII.
\usepackage[
language=polish,
backend=biber,
style=alphabetic,
url=false,
eprint=true,
]{biblatex}

\addbibresource{SO-Listy-materiałów-do-nauki-Bibliography.bib}





% ------------------------------------------------------
% Local packages
% You need to put them in the same directory as .tex file
% ------------------------------------------------------
% Package containing local settings and commands for LaTeX
\usepackage{./Local-packages/local-settings}

% Package containing various command useful for working with a text
\usepackage{./Local-packages/general-commands}

% Package containing commands and other code useful for working with
% mathematical text
% \usepackage{math-commands}





% ------------------------------------------------------
% Package "hyperref"
% They advised to put it on the end of preambule
% ------------------------------------------------------
% It allows you to use hyperlinks in the text
\usepackage{hyperref}










% ------------------------------------------------------------------------------------------------------------------
% Title and author of the text
\title{Podstawy informatyki z~językiem~C \\
  {\Large Lista materiałów do nauki}}

\author{Kamil Ziemian \\
  \email}

% \date{}
% ------------------------------------------------------------------------------------------------------------------










% ####################################################################
% Beginning of the document
\begin{document}
% ####################################################################





% ######################################
% Title of the text
\maketitle
% ######################################





% ######################################
\section{Informacje ogólne}

\label{sec:Informacje-ogólne}
% ######################################


Niniejsza lista została napisana z~myślą o ???% tych, którzy dopiero zaczynają
% swoją przygodę z~informatyką i~programowaniem. Dlatego staraliśmy~się
% uczynić ją stosunkowo krótką i~dobierać głównie proste materiały do~nauki.
% Niewątpliwie można ją jeszcze pod wieloma względami poprawić.

Jeśli mieliby Państwo jakiekolwiek pytanie odnośnie języka~systemów
operacyjnych, lub nie mogli zrozumieć jakiegoś fragmentu materiałów,
proszę do mnie napisać pod adres~\email. Również wszelkie uwagi odnośnie
tego dokumentu są mile widziane, proszę je kierować pod adres \email.
W~szczególności proszę przesyłać informacje, że~któreś hiperlinki nie
działają poprawnie. Taka jest natura rzeczy w~internecie, że~są usuwane lub
przenoszone w~inne miejsca.










% ######################################
\section{???}

\label{sec:Informacje-ogólne}
% ######################################














% ####################################################################
% ####################################################################
% Bibliography

\printbibliography





% ############################

% End of the document
\end{document}
