% ------------------------------------------------------------------------------------------------------------------
% Basic configuration and packages
% ------------------------------------------------------------------------------------------------------------------
% Package for discovering wrong and outdated usage of LaTeX.
% More information to be found in l2tabu English version.
\RequirePackage[l2tabu, orthodox]{nag}
% Class of LaTeX document: {size of paper, size of font}[document class]
\documentclass[a4paper,11pt]{article}



% ------------------------------------------------------
% Packages not tied to particular normal language
% ------------------------------------------------------
% This package should improved spaces in the text.
\usepackage{microtype}
% Add few important symbols, like text Celcius degree
\usepackage{textcomp}



% ------------------------------------------------------
% Polonization of LaTeX document
% ------------------------------------------------------
% Basic polonization of the text
% \usepackage[MeX]{polski}
% Switching on UTF-8 encoding
\usepackage[utf8]{inputenc}
% Adding font Latin Modern
\usepackage{lmodern}
% Package is need for fonts Latin Modern
\usepackage[T1]{fontenc}



% ------------------------------------------------------
% Setting margins
% ------------------------------------------------------
\usepackage[a4paper, total={14cm, 25cm}]{geometry}



% ------------------------------------------------------
% Setting vertical spaces in the text
% ------------------------------------------------------
% Setting space between lines
\renewcommand{\baselinestretch}{1.1}

% Setting space between lines in tables
% \renewcommand{\arraystretch}{1.4}



% ------------------------------------------------------
% Packages for scientific papers
% ------------------------------------------------------
% Switching off \lll symbol, that I guess is representing letter "Ł"
% It collide with `amsmath' package's command with the same name
% \let\lll\undefined
% Basic package from American Mathematical Society (AMS)
% \usepackage[intlimits]{amsmath}
% Equations are numbered separately in every section.
% \numberwithin{equation}{section}

% Other very useful packages from AMS
% \usepackage{amsfonts}
% \usepackage{amssymb}
% \usepackage{amscd}
% \usepackage{amsthm}

% Package for writting physical units
% \usepackage{siunitx}

% Package with better looking calligraphy fonts
% \usepackage{calrsfs}

% Package with better looking greek letters
% Example of use: pi -> \uppi
% \usepackage{upgreek}
% Improving look of lambda letter
% \let\oldlambda\Lambda
% \renewcommand{\lambda}{\uplambda}




% ------------------------------------------------------
% BibLaTeX
% ------------------------------------------------------
% Package biblatex, with biber as its backend, allow us to handle
% bibliography entries that use Unicode symbols outside ASCII.
\usepackage[
language=polish,
backend=biber,
style=alphabetic,
url=false,
eprint=true,
]{biblatex}

\addbibresource{Java-polish-dictionary-Bibliography.bib}





% ------------------------------------------------------
% Defining new environments (?)
% ------------------------------------------------------
% Defining enviroment "Wniosek"
\newtheorem{corollary}{Wniosek}
\newtheorem{definition}{Definicja}
\newtheorem{theorem}{Twierdzenie}





% ------------------------------------------------------
% Local packages
% You need to put them in the same directory as .tex file
% ------------------------------------------------------
% Package containing various command useful for working with a text
\usepackage{./Local-packages/general-commands}





% ------------------------------------------------------
% Package "hyperref"
% They advised to put it on the end of preambule
% ------------------------------------------------------
% It allows you to use hyperlinks in the text
\usepackage{hyperref}










% ------------------------------------------------------------------------------------------------------------------
% Title and author of the text
\title{Java-polish dictionary}

% \author{}


% \date{}
% ------------------------------------------------------------------------------------------------------------------










% ####################################################################
% Beginning of the document
\begin{document}
% ####################################################################





% ######################################
\maketitle
% ######################################





This dictionary is based on Bruce Eckel \textit{Thinking in~Java. Edycja
  polska, wydanie~III},
\parencite{Eckel-Thinking-in-Java-Ed-polska-Wyd-III-Pub-2003}.

\vspace{1.5em}



\noindent
\textbf{Access modifiers}~-- modyfikatory dostępu. \\
\textbf{Access specifiers}~-- specyfikatory dostępu. \\
\textbf{Aggregation}~-- agregacja. \\
\textbf{Base class}~-- klasa bazowa. \\
\textbf{By-like relationship}~-- relacja bycia podobnym. \\
\textbf{Class creators}~-- twórcy klas. \\
\textbf{Client programmers}~-- programiści-klienci. \\
\textbf{Composition}~-- kompozycja. \\
\textbf{Compilation unit}~-- jednostka kompilacji. \\
\textbf{Concurrency}~-- współbieżność. \\
\textbf{Downcasting}~-- rzutowanie w~dół. \\
\textbf{Early binding}~-- wczesne wiązanie. \\
\textbf{Exception}~-- wyjątek. \\
\textbf{Exception handler}~-- procedura obsługi wyjątku. \\
\textbf{Exception handling}~-- obsługa wyjątków. \\
\textbf{Generics}~-- generyki. \\
\textbf{Inheritance}~-- dziedziczenie. \\
\textbf{Is-a relationship}~-- relacja bycia czymś. \\
\textbf{Late binding}~-- późne wiązania. \\
\textbf{Ligthweight peristance}~-- ograniczona trwałość. \\
\textbf{Middleware}~-- warstwa pośrednia. \\
\textbf{Multithreading}~-- wielowątkowość. \\
\textbf{Overriding}~-- przesłonięcie. \\
\textbf{Package}~-- pakiet, paczka. \\
\textbf{Package access}~-- dostęp pakietowy. \\
\textbf{Parametrized type}~-- typ parametryzowalny. \\
\textbf{Persistent}~-- trwałość. \\
\textbf{Substitution principle}~--zasada zastępowalności. \\
\textbf{Throwing an exception}~-- rzucanie wyjątków. \\
\textbf{Upcasting}~--rzutowanie w~górę. \\
Klasa potomna, \\













% ####################################################################
% ####################################################################
% Bibliography

\printbibliography





% ############################
% End of the document

\end{document}
