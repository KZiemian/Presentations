% ------------------------------------------------------------------------------------------------------------------
% Basic configuration of Beamera class and Jagiellonian theme
% ------------------------------------------------------------------------------------------------------------------
\RequirePackage[l2tabu, orthodox]{nag}



\ifx\PresentationStyle\notset
  \def\PresentationStyle{dark}
\fi



% Options: t - align frame text to the top.
\documentclass[10pt,t]{beamer}
\mode<presentation>
\usetheme[style=\PresentationStyle]{jagiellonian}




% ------------------------------------------------------------------------------------
% Procesing configuration files of Jagiellonian theme located
% in the directory "preambule"
% ------------------------------------------------------------------------------------
% Configuration for polish language
% Need description
\usepackage[english]{babel}





% % ------------------------------
% % Better support of polish chars in technical parts of PDF
% % ------------------------------
% \hypersetup{pdfencoding=auto,psdextra}

% Package "textpos" give as enviroment "textblock" which is very usefull in
% arranging text on slides.

% This is standard configuration of "textpos"
\usepackage[overlay,absolute]{textpos}

% If you need to see bounds of "textblock's" comment line above and uncomment
% one below.

% Caution! When showboxes option is on significant ammunt of space is add
% to the top of textblock and as such, everyting put in them gone down.
% We need to check how to remove this bug.

% \usepackage[showboxes,overlay,absolute]{textpos}



% Setting scale length for package "textpos"
\setlength{\TPHorizModule}{10mm}
\setlength{\TPVertModule}{\TPHorizModule}


% ---------------------------------------
% Packages written for lectures "Geometria 3D dla twórców gier wideo"
% ---------------------------------------
% \usepackage{./Geometry3DPackages/Geometry3D}
% \usepackage{./Geometry3DPackages/Geometry3DIndices}
% \usepackage{./Geometry3DPackages/Geometry3DTikZStyle}
% \usepackage{./ProgramowanieSymulacjiFizykiPaczki/ProgramowanieSymulacjiFizykiTikZStyle}
% \usepackage{./Geometry3DPackages/mathcommands}


% ---------------------------------------
% TikZ
% ---------------------------------------
% Importing TikZ libraries
\usetikzlibrary{arrows.meta}
\usetikzlibrary{positioning}





% % Configuration package "bm" that need for making bold symbols
% \newcommand{\bmmax}{0}
% \newcommand{\hmmax}{0}
% \usepackage{bm}




% ---------------------------------------
% Packages for scientific texts
% ---------------------------------------
% \let\lll\undefined  % Sometimes you must use this line to allow
% "amsmath" package to works with packages with packages for polish
% languge imported
% /preambul/LanguageSettings/JagiellonianPolishLanguageSettings.tex.
% This comments (probably) removes polish letter Ł.
\usepackage{amsmath}  % Packages from American Mathematical Society (AMS)
\usepackage{amssymb}
\usepackage{amscd}
\usepackage{amsthm}
\usepackage{siunitx}  % Package for typsetting SI units.
\usepackage{upgreek}  % Better looking greek letters.
% Example of using upgreek: pi = \uppi


\usepackage{calrsfs}  % Zmienia czcionkę kaligraficzną w \mathcal
% na ładniejszą. Może w innych miejscach robi to samo, ale o tym nic
% nie wiem.










% ---------------------------------------
% Packages written for lectures "Geometria 3D dla twórców gier wideo"
% ---------------------------------------
% \usepackage{./ProgramowanieSymulacjiFizykiPaczki/ProgramowanieSymulacjiFizyki}
% \usepackage{./ProgramowanieSymulacjiFizykiPaczki/ProgramowanieSymulacjiFizykiIndeksy}
% \usepackage{./ProgramowanieSymulacjiFizykiPaczki/ProgramowanieSymulacjiFizykiTikZStyle}





% !!!!!!!!!!!!!!!!!!!!!!!!!!!!!!
% !!!!!!!!!!!!!!!!!!!!!!!!!!!!!!
% EVIL STUFF
\if\JUlogotitle1
\edef\LogoJUPath{LogoJU_\JUlogoLang/LogoJU_\JUlogoShape_\JUlogoColor.pdf}
\titlegraphic{\hfill\includegraphics[scale=0.22]
{./JagiellonianPictures/\LogoJUPath}}
\fi
% ---------------------------------------
% Commands for handling colors
% ---------------------------------------


% Command for setting normal text color for some text in math modestyle
% Text color depend on used style of Jagiellonian

% Beamer version of command
\newcommand{\TextWithNormalTextColor}[1]{%
  {\color{jNormalTextFGColor}
    \setbeamercolor{math text}{fg=jNormalTextFGColor} {#1}}
}

% Article and similar classes version of command
% \newcommand{\TextWithNormalTextColor}[1]{%
%   {\color{jNormalTextsFGColor} {#1}}
% }



% Beamer version of command
\newcommand{\NormalTextInMathMode}[1]{%
  {\color{jNormalTextFGColor}
    \setbeamercolor{math text}{fg=jNormalTextFGColor} \text{#1}}
}


% Article and similar classes version of command
% \newcommand{\NormalTextInMathMode}[1]{%
%   {\color{jNormalTextsFGColor} \text{#1}}
% }




% Command that sets color of some mathematical text to the same color
% that has normal text in header (?)

% Beamer version of the command
\newcommand{\MathTextFrametitleFGColor}[1]{%
  {\color{jFrametitleFGColor}
    \setbeamercolor{math text}{fg=jFrametitleFGColor} #1}
}

% Article and similar classes version of the command
% \newcommand{\MathTextWhiteColor}[1]{{\color{jFrametitleFGColor} #1}}





% Command for setting color of alert text for some text in math modestyle

% Beamer version of the command
\newcommand{\MathTextAlertColor}[1]{%
  {\color{jOrange} \setbeamercolor{math text}{fg=jOrange} #1}
}

% Article and similar classes version of the command
% \newcommand{\MathTextAlertColor}[1]{{\color{jOrange} #1}}





% Command that allow you to sets chosen color as the color of some text into
% math mode. Due to some nuances in the way that Beamer handle colors
% it not work in all cases. We hope that in the future we will improve it.

% Beamer version of the command
\newcommand{\SetMathTextsColor}[2]{%
  {\color{#1} \setbeamercolor{math text}{fg=#1} #2}
}


% Article and similar classes version of the command
% \newcommand{\SetMathTextColor}[2]{{\color{#1} #2}}










% ---------------------------------------
% Commands for setting background pictures for some slides
% ---------------------------------------
\newcommand{\TitleBackgroundPicture}
{./PresentationPictures/CommonPictures/Cute_dragon_BG_dark.png}
\newcommand{\SectionBackgroundPicture}
{./PresentationPictures/CommonPictures/Cute_dragon_small_BG_light.png}



\newcommand{\TitleSlideWithPicture}{
  \begingroup

  \usebackgroundtemplate{ % \hspace*{-11.5em}
    \includegraphics[height=\paperheight]{\TitleBackgroundPicture}}

  \maketitle

  \endgroup
}





\newcommand{\SectionSlideWithPicture}[1]{%
  \begingroup

  \usebackgroundtemplate{ % \hspace*{-11.5em}
    \includegraphics[height=\paperheight]{\SectionBackgroundPicture}}

  \setbeamercolor{titlelike}{fg=normal text.fg}

  \section{#1}

  \endgroup
}





\newcommand{\EndingSlide}[1]{%
  \begin{frame}[standout]

    \begingroup

    \color{jFrametitleFGColor}

    #1

    \endgroup

  \end{frame}
}










% ------------------------------------------------------
% BibLaTeX
% ------------------------------------------------------
% Package biblatex, with biber as its backend, allow us to handle
% bibliography entries that use Unicode symbols outside ASCII.
\usepackage[
language=polish,
backend=biber,
style=alphabetic,
url=false,
eprint=true,
]{biblatex}

\addbibresource{Programming-in-Java-Bibliography.bib}





% ------------------------------------------------------
% Importing packages, libraries and setting their configuration
% ------------------------------------------------------





% ------------------------------------------------------
% Local packages
% ------------------------------------------------------
% Local configuration of this particular presentation
\usepackage{./Local-packages/local-settings}

% \usepackage{./Local-packages/PGF-TikZ-Arrows-styles}

% \usepackage{./Local-packages/PGF-TikZ-Diagram-styles}










% ------------------------------------------------------------------------------------------------------------------
\title{Programming in Java}
\subtitle{1.~Introduction to the subject}

\author{Kamil Ziemian \\
  \email}


% \date{}
% ------------------------------------------------------------------------------------------------------------------










% ####################################################################
% Beginning of the document
\begin{document}
% ####################################################################





% ######################################
% Number of chars: 38k+,
% Text is adjusted to the left and words are broken at the end of the line.
\RaggedRight
% ######################################





% ######################################
\maketitle
% ######################################





% ##################
\begin{frame}
  \frametitle{Table of contents}


  \tableofcontents

\end{frame}
% ##################





% ######################################
\section{Introductory information}
% ######################################



% ##################
\begin{frame}
  \frametitle{English language warning}


  If you spot any mistake in our use of English, please write to us at
  \email. We try to check our writing, but achieving high quality English
  is a~very hard task.

  Also, if the things we said are hard to understand, you should point it
  to us immediately. We have no illusion that our English is very good.
  Maybe, not even particularly good.

\end{frame}
% ##################





% ##################
\begin{frame}
  \frametitle{Aim of this subject}


  The~aim~of this subject is to learn the basics of programming in Java,
  with the main stress being put on its object-oriented side.
  This subject is for \alert{you} not for us.

  If you have any question about the subject, you should ask it, since this
  subject is dedicated to \alert{you}. Also, we know that it is far from
  being perfect, we work on improving it, so all questions and comments
  about it are welcomed.

  Question of the type ``What is the~best boss in the Hollow Knight?'' or
  ``What is your opinion about Hollow Knight: Silksong?'' must be left
  for the time after the~lesson.

  We want this curse to be maximal about writing and running programs,
  not about learning theory. Some theory is unavoidable and also external
  constraints can limit our ability to make this subject about running
  programs in Java, but we will do what we can.

\end{frame}
% ##################





% ##################
\begin{frame}
  \frametitle{Learning materials and requirements}


  We are still developing learning materials for this course, you will
  have access to them by Sake. We hope that this materials will help
  you learn the basics of Java. They are by no means perfect, so please,
  if you found any mistake in them, like bad English text, write to us at
  \email.

  We prepared this subject assuming that you know such concepts of
  programming as variables, arithmetic's operators, if statements and loops,
  but we still provide a quick overview of how they work in Java. If you
  need more explanation of them, please just ask us. This course is for
  you learn basics~of Java, regardless of your current level of programming.
  Everyone was a beginner at some time.

  We don't assume any knowledge of object-oriented programming, which
  is at the~heart of Java.  We will introduce it from scratch.

\end{frame}
% ##################





% ##################
\begin{frame}
  \frametitle{Passing the subject}


  This is so important, that we should repeat it. We don't assume any
  knowledge of object-oriented programming, which is at the~heart of Java.
  We will introduce it from scratch. Unfortunately Java is not the best
  language to learn for the first time object-oriented programming, so
  previous experience with it from languages like C++ and Python will help
  you a~lot.

  We will have two homework lists in the semester, plus on the last meeting
  we will have $45$~minutes live exam. Half of the mark is form homework
  and the other half form the~live exam.

  The~live exam will be as follows. Everyone is \alert{required} to
  physically be at it. You will get one of two exercises to solve and then
  send a solution to email \email or give it back on the paper sheet (it all
  depends on what room we will have at that time). In the times of
  chat\textsc{gpt} and other such live check of your knowledge is a
  necessity.

\end{frame}
% ##################





% ##################
\begin{frame}
  \frametitle{Live exam}


  The exercises will be probably along the line ``Write the~class that
  implement class Dishwasher with following functionality. Here is the code
  that need to run with your implementation of such class.''

  If someone wouldn't be on the live exam for \alert{a~good reason}, we will
  organize another term, on which such a person needs to physically come
  and solve one of two exercises under our eyes. Again, the internet,
  smartphones and~\textsc{ai} make it necessary.

\end{frame}
% ##################










% ######################################
\appendix
% ######################################





% ######################################
\EndingSlide{Thank you. Any questions?}
% ######################################





% ##################
\begin{frame}
  \frametitle{Materials used for prepering this course}


  In prepering this course we used following materials.

  Bruce Eckel, \textit{Thinking in~Java. Edycja polska, wydanie~III},
  \parencite{Eckel-Thinking-in-Java-Ed-polska-Wyd-III-Pub-2003}. \\
  freeCodeCamp.org, \textit{Java Programming for~Beginners~-- Full Course},
  \parencite{freeCodeCamp-org-Java-Programming-for-Beginners-ETC-Ver-2022}.

\end{frame}
% ##################





% ####################################################################
% ####################################################################
% Bibliography

\printbibliography










% ####################################################################
% End of the document

\end{document}
