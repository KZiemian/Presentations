% ------------------------------------------------------------------------------------------------------------------
% Basic configuration of Beamera class and Jagiellonian theme
% ------------------------------------------------------------------------------------------------------------------
\RequirePackage[l2tabu, orthodox]{nag}



\ifx\PresentationStyle\notset
  \def\PresentationStyle{dark}
\fi



% Options: t - align frame text to the top.
\documentclass[10pt,t]{beamer}
\mode<presentation>
\usetheme[style=\PresentationStyle]{jagiellonian}




% ------------------------------------------------------------------------------------
% Procesing configuration files of Jagiellonian theme located
% in the directory "preambule"
% ------------------------------------------------------------------------------------
% Configuration for polish language
% Need description
\usepackage[english]{babel}





% % ------------------------------
% % Better support of polish chars in technical parts of PDF
% % ------------------------------
% \hypersetup{pdfencoding=auto,psdextra}

% Package "textpos" give as enviroment "textblock" which is very usefull in
% arranging text on slides.

% This is standard configuration of "textpos"
\usepackage[overlay,absolute]{textpos}

% If you need to see bounds of "textblock's" comment line above and uncomment
% one below.

% Caution! When showboxes option is on significant ammunt of space is add
% to the top of textblock and as such, everyting put in them gone down.
% We need to check how to remove this bug.

% \usepackage[showboxes,overlay,absolute]{textpos}



% Setting scale length for package "textpos"
\setlength{\TPHorizModule}{10mm}
\setlength{\TPVertModule}{\TPHorizModule}


% ---------------------------------------
% Packages written for lectures "Geometria 3D dla twórców gier wideo"
% ---------------------------------------
% \usepackage{./Geometry3DPackages/Geometry3D}
% \usepackage{./Geometry3DPackages/Geometry3DIndices}
% \usepackage{./Geometry3DPackages/Geometry3DTikZStyle}
% \usepackage{./ProgramowanieSymulacjiFizykiPaczki/ProgramowanieSymulacjiFizykiTikZStyle}
% \usepackage{./Geometry3DPackages/mathcommands}


% ---------------------------------------
% TikZ
% ---------------------------------------
% Importing TikZ libraries
\usetikzlibrary{arrows.meta}
\usetikzlibrary{positioning}





% % Configuration package "bm" that need for making bold symbols
% \newcommand{\bmmax}{0}
% \newcommand{\hmmax}{0}
% \usepackage{bm}




% ---------------------------------------
% Packages for scientific texts
% ---------------------------------------
% \let\lll\undefined  % Sometimes you must use this line to allow
% "amsmath" package to works with packages with packages for polish
% languge imported
% /preambul/LanguageSettings/JagiellonianPolishLanguageSettings.tex.
% This comments (probably) removes polish letter Ł.
\usepackage{amsmath}  % Packages from American Mathematical Society (AMS)
\usepackage{amssymb}
\usepackage{amscd}
\usepackage{amsthm}
\usepackage{siunitx}  % Package for typsetting SI units.
\usepackage{upgreek}  % Better looking greek letters.
% Example of using upgreek: pi = \uppi


\usepackage{calrsfs}  % Zmienia czcionkę kaligraficzną w \mathcal
% na ładniejszą. Może w innych miejscach robi to samo, ale o tym nic
% nie wiem.










% ---------------------------------------
% Packages written for lectures "Geometria 3D dla twórców gier wideo"
% ---------------------------------------
% \usepackage{./ProgramowanieSymulacjiFizykiPaczki/ProgramowanieSymulacjiFizyki}
% \usepackage{./ProgramowanieSymulacjiFizykiPaczki/ProgramowanieSymulacjiFizykiIndeksy}
% \usepackage{./ProgramowanieSymulacjiFizykiPaczki/ProgramowanieSymulacjiFizykiTikZStyle}





% !!!!!!!!!!!!!!!!!!!!!!!!!!!!!!
% !!!!!!!!!!!!!!!!!!!!!!!!!!!!!!
% EVIL STUFF
\if\JUlogotitle1
\edef\LogoJUPath{LogoJU_\JUlogoLang/LogoJU_\JUlogoShape_\JUlogoColor.pdf}
\titlegraphic{\hfill\includegraphics[scale=0.22]
{./JagiellonianPictures/\LogoJUPath}}
\fi
% ---------------------------------------
% Commands for handling colors
% ---------------------------------------


% Command for setting normal text color for some text in math modestyle
% Text color depend on used style of Jagiellonian

% Beamer version of command
\newcommand{\TextWithNormalTextColor}[1]{%
  {\color{jNormalTextFGColor}
    \setbeamercolor{math text}{fg=jNormalTextFGColor} {#1}}
}

% Article and similar classes version of command
% \newcommand{\TextWithNormalTextColor}[1]{%
%   {\color{jNormalTextsFGColor} {#1}}
% }



% Beamer version of command
\newcommand{\NormalTextInMathMode}[1]{%
  {\color{jNormalTextFGColor}
    \setbeamercolor{math text}{fg=jNormalTextFGColor} \text{#1}}
}


% Article and similar classes version of command
% \newcommand{\NormalTextInMathMode}[1]{%
%   {\color{jNormalTextsFGColor} \text{#1}}
% }




% Command that sets color of some mathematical text to the same color
% that has normal text in header (?)

% Beamer version of the command
\newcommand{\MathTextFrametitleFGColor}[1]{%
  {\color{jFrametitleFGColor}
    \setbeamercolor{math text}{fg=jFrametitleFGColor} #1}
}

% Article and similar classes version of the command
% \newcommand{\MathTextWhiteColor}[1]{{\color{jFrametitleFGColor} #1}}





% Command for setting color of alert text for some text in math modestyle

% Beamer version of the command
\newcommand{\MathTextAlertColor}[1]{%
  {\color{jOrange} \setbeamercolor{math text}{fg=jOrange} #1}
}

% Article and similar classes version of the command
% \newcommand{\MathTextAlertColor}[1]{{\color{jOrange} #1}}





% Command that allow you to sets chosen color as the color of some text into
% math mode. Due to some nuances in the way that Beamer handle colors
% it not work in all cases. We hope that in the future we will improve it.

% Beamer version of the command
\newcommand{\SetMathTextsColor}[2]{%
  {\color{#1} \setbeamercolor{math text}{fg=#1} #2}
}


% Article and similar classes version of the command
% \newcommand{\SetMathTextColor}[2]{{\color{#1} #2}}










% ---------------------------------------
% Commands for setting background pictures for some slides
% ---------------------------------------
\newcommand{\TitleBackgroundPicture}
{./PresentationPictures/CommonPictures/Cute_dragon_BG_dark.png}
\newcommand{\SectionBackgroundPicture}
{./PresentationPictures/CommonPictures/Cute_dragon_small_BG_light.png}



\newcommand{\TitleSlideWithPicture}{
  \begingroup

  \usebackgroundtemplate{ % \hspace*{-11.5em}
    \includegraphics[height=\paperheight]{\TitleBackgroundPicture}}

  \maketitle

  \endgroup
}





\newcommand{\SectionSlideWithPicture}[1]{%
  \begingroup

  \usebackgroundtemplate{ % \hspace*{-11.5em}
    \includegraphics[height=\paperheight]{\SectionBackgroundPicture}}

  \setbeamercolor{titlelike}{fg=normal text.fg}

  \section{#1}

  \endgroup
}





\newcommand{\EndingSlide}[1]{%
  \begin{frame}[standout]

    \begingroup

    \color{jFrametitleFGColor}

    #1

    \endgroup

  \end{frame}
}










% ------------------------------------------------------
% BibLaTeX
% ------------------------------------------------------
% Package biblatex, with biber as its backend, allow us to handle
% bibliography entries that use Unicode symbols outside ASCII.
\usepackage[
language=polish,
backend=biber,
style=alphabetic,
url=false,
eprint=true,
]{biblatex}

\addbibresource{Programming-in-Java-Bibliography.bib}





% ------------------------------------------------------
% Importing packages, libraries and setting their configuration
% ------------------------------------------------------





% ------------------------------------------------------
% Local packages
% ------------------------------------------------------
% Local configuration of this particular presentation
\usepackage{./Local-packages/local-settings}

% \usepackage{./Local-packages/PGF-TikZ-Arrows-styles}

% \usepackage{./Local-packages/PGF-TikZ-Diagram-styles}










% ------------------------------------------------------------------------------------------------------------------
\title{Programming in Java}
\subtitle{1.~Introduction to the~course}

\author{Kamil Ziemian \\
  \email}


% \date{}
% ------------------------------------------------------------------------------------------------------------------










% ####################################################################
% Beginning of the document
\begin{document}
% ####################################################################





% ######################################
% Number of chars: 9k+,
% Text is adjusted to the left and words are broken at the end of the line.
\RaggedRight
% ######################################





% ######################################
\maketitle
% ######################################





% % ##################
% \begin{frame}
%   \frametitle{Table of contents}


%   \tableofcontents

% \end{frame}
% % ##################





% ######################################
\section{Introductory information}
% ######################################



% ##################
\begin{frame}
  \frametitle{Warning about bad English}


  If you spot any language mistakes in our materials, please write to us at
  \email. Achieving high quality English is a~very hard task for us.

  Also, if the things we said are hard to understand, you should point it
  to us immediately. We have no illusion that our English is very good.
  Maybe not even particularly good.

\end{frame}
% ##################





% ##################
\begin{frame}
  \frametitle{Aim of the course}


  The~aim~of the~course is to learn the basics of programming in Java,
  with the main focus being put on its object-oriented side.
  This subject is for \alert{you}, not for us. If you have any question about
  it, you should ask it immediately. \textit{Programming in Java} is
  dedicated to \alert{you}, we will stress it again and again.

  Also, we know that our course is far from being perfect and we are
  working on improving it. For these reasons we welcome all questions and
  comments from you, since you are the~most important people in this course.
  \alert{You} are important, not us.

  Questions of the type ``What is the~best boss in the Hollow Knight?'' or
  ``What is your opinion about Hollow Knight: Silksong?'' must be left
  for the time after the~lessons.

\end{frame}
% ##################





% ##################
\begin{frame}
  \frametitle{Writing and running programs}


  We want this course to be maximal about writing and running programs,
  not about learning theory. Some amount of theory is unavoidable. Also
  external constraints can limit our ability to make this subject as much
  a~hands-on experience as we want, but we will do what we can, with what
  we have.

  We are still developing learning materials for this course, you will
  have access to them by Sake. We hope that these materials will help you
  learn the basics of Java. They are by no means perfect, so please, if
  you find any mistakes in them, like bad English language, write to us at
  ziemian@wszib.edu.pl.

  If you know git, you can download this materials by typing in shell \\
  \texttt{\$ git clone https://github.com/KZiemian/Presentations} \\
  Or you may check and download them just using a~web browser, they are at
  \colorhref{https://github.com/KZiemian/Presentations}
  {https://github.com/KZiemian/Presentations}.

\end{frame}
% ##################





% ##################
\begin{frame}
  \frametitle{Writing and running programs}


  Materials for \textit{Programming in Java} are inside the~directory \\
  \texttt{Computer-science-Presentations/Programming-in-Java/}. \\
  Java programs with commentaries are to be find inside \\
  \texttt{Course-materials/Learning-Java/}.

  We prepared this subject assuming that you know such concepts of
  programming as variables, arithmetic's operators, if statements and loops,
  but we still provide a quick overview of how they work in Java. If you
  need more explanation of them, please just ask us. This course is for
  you to learn basics~of Java, regardless of your current level of
  programming. Everyone was a beginner at some time.

  We don't assume any knowledge of object-oriented programming, which
  is at the~heart of Java.  We will introduce it from scratch.

\end{frame}
% ##################





% ##################
\begin{frame}
  \frametitle{We need to simplify a~lot of things}


  This is so important, that we should repeat it. We don't assume any
  knowledge of object-oriented programming, which is at the~heart of Java.
  We will introduce it from scratch. Unfortunately Java is not the best
  language to learn for the first time object-oriented programming, so
  previous experience with it from languages like C++ and Python will help
  you a~lot.

  We will assume that you have some basic knowledge of coding under
  GNU/Linux operating system, but if you struggle, give us a~hint about it.
  We are here to help \alert{you}.

  Due to the introductory nature of our course and time constraints, we will
  need to simplify a~lot of things concerning Java. Aim of this course is
  not to present you everything that is to know about Java with all the
  details, which is simply unfeasible. Its aim is to give you solid
  foundations~of Java that helps you learn it more and create valuable
  software using it.

\end{frame}
% ##################





% ##################
\begin{frame}
  \frametitle{Passing the~course}


  Providing you with such a~good foundation of Java in the time we have is
  already quite a~challenging task. You will judge for yourself, how well we
  passed it.

  We will have two homework lists in the semester, plus on the last meeting
  we will have a~$45$~minutes live exam. Half of the mark is from homework
  and the other half from the~live exam.

  The~live exam will be as follows. Everyone is \alert{required} to
  physically be at it. You will get one of two exercises to solve and then
  send a~solution to email \email{} or give it back on the paper sheet (it
  all depends on what room we will have at that time). In the~times of
  Chat\textsc{gpt} and things like that, checking your knowledge in real
  time is a~necessity.

\end{frame}
% ##################





% ##################
\begin{frame}
  \frametitle{Live exam}


  The exercises will be probably along the line ``Write the~class that
  implements class Dishwasher with following functionality. Here is the code
  that needs to run with your implementation of such a~class.''

  If someone wouldn't be on the live exam for \alert{a~good reason}, we will
  organize another term, on which such a person needs to physically present
  and solve exercises under our watch. Again, the internet, smartphones
  and~\textsc{ai} make it necessary.

  There is also the~possibility of passing the~course by part-time exam. Due
  to character~of it we doubt, that many people would choose such way~of
  passing it, but since the~students are the~most important people here,
  this is of course possible. If you want part-time exam, make a~contact
  with us, for example at \email.

\end{frame}
% ##################










% ######################################
\appendix
% ######################################





% ######################################
\EndingSlide{Thank you. Any questions?}
% ######################################










% ##################
\begin{frame}
  \frametitle{Materials used for prepering this course}


  In preparing this course we used following materials.

  Bruce Eckel, \textit{Thinking in~Java. Edycja polska, wydanie~III},
  \parencite{Eckel-Thinking-in-Java-Ed-polska-Wyd-III-Pub-2003}. \\
  freeCodeCamp.org, \textit{Java Programming for~Beginners~-- Full Course},
  \parencite{freeCodeCamp-org-Java-Programming-for-Beginners-ETC-Ver-2022}.

\end{frame}
% ##################





% ####################################################################
% ####################################################################
% Bibliography

\printbibliography










% ####################################################################
% End of the document

\end{document}
