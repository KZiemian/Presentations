% ------------------------------------------------------------------------------------------------------------------
% Basic configuration of Beamera class and Jagiellonian theme
% ------------------------------------------------------------------------------------------------------------------
\RequirePackage[l2tabu, orthodox]{nag}



\ifx\PresentationStyle\notset
  \def\PresentationStyle{light}
\fi



% Options: t - align frame text to the top.
\documentclass[10pt,t]{beamer}
\mode<presentation>
\usetheme[style=\PresentationStyle]{jagiellonian}




% ------------------------------------------------------------------------------------
% Procesing configuration files of Jagiellonian theme located
% in the directory "preambule".
% ------------------------------------------------------------------------------------
% Configuration for polish language
% Need description
\usepackage[english]{babel}





% % ------------------------------
% % Better support of polish chars in technical parts of PDF
% % ------------------------------
% \hypersetup{pdfencoding=auto,psdextra}

% Package "textpos" give as enviroment "textblock" which is very usefull in
% arranging text on slides.

% This is standard configuration of "textpos"
\usepackage[overlay,absolute]{textpos}

% If you need to see bounds of "textblock's" comment line above and uncomment
% one below.

% Caution! When showboxes option is on significant ammunt of space is add
% to the top of textblock and as such, everyting put in them gone down.
% We need to check how to remove this bug.

% \usepackage[showboxes,overlay,absolute]{textpos}



% Setting scale length for package "textpos"
\setlength{\TPHorizModule}{10mm}
\setlength{\TPVertModule}{\TPHorizModule}


% ---------------------------------------
% Packages written for lectures "Geometria 3D dla twórców gier wideo"
% ---------------------------------------
% \usepackage{./Geometry3DPackages/Geometry3D}
% \usepackage{./Geometry3DPackages/Geometry3DIndices}
% \usepackage{./Geometry3DPackages/Geometry3DTikZStyle}
% \usepackage{./ProgramowanieSymulacjiFizykiPaczki/ProgramowanieSymulacjiFizykiTikZStyle}
% \usepackage{./Geometry3DPackages/mathcommands}


% ---------------------------------------
% TikZ
% ---------------------------------------
% Importing TikZ libraries
\usetikzlibrary{arrows.meta}
\usetikzlibrary{positioning}





% % Configuration package "bm" that need for making bold symbols
% \newcommand{\bmmax}{0}
% \newcommand{\hmmax}{0}
% \usepackage{bm}




% ---------------------------------------
% Packages for scientific texts
% ---------------------------------------
% \let\lll\undefined  % Sometimes you must use this line to allow
% "amsmath" package to works with packages with packages for polish
% languge imported
% /preambul/LanguageSettings/JagiellonianPolishLanguageSettings.tex.
% This comments (probably) removes polish letter Ł.
\usepackage{amsmath}  % Packages from American Mathematical Society (AMS)
\usepackage{amssymb}
\usepackage{amscd}
\usepackage{amsthm}
\usepackage{siunitx}  % Package for typsetting SI units.
\usepackage{upgreek}  % Better looking greek letters.
% Example of using upgreek: pi = \uppi


\usepackage{calrsfs}  % Zmienia czcionkę kaligraficzną w \mathcal
% na ładniejszą. Może w innych miejscach robi to samo, ale o tym nic
% nie wiem.










% ---------------------------------------
% Packages written for lectures "Geometria 3D dla twórców gier wideo"
% ---------------------------------------
% \usepackage{./ProgramowanieSymulacjiFizykiPaczki/ProgramowanieSymulacjiFizyki}
% \usepackage{./ProgramowanieSymulacjiFizykiPaczki/ProgramowanieSymulacjiFizykiIndeksy}
% \usepackage{./ProgramowanieSymulacjiFizykiPaczki/ProgramowanieSymulacjiFizykiTikZStyle}





% !!!!!!!!!!!!!!!!!!!!!!!!!!!!!!
% !!!!!!!!!!!!!!!!!!!!!!!!!!!!!!
% EVIL STUFF
\if\JUlogotitle1
\edef\LogoJUPath{LogoJU_\JUlogoLang/LogoJU_\JUlogoShape_\JUlogoColor.pdf}
\titlegraphic{\hfill\includegraphics[scale=0.22]
{./JagiellonianPictures/\LogoJUPath}}
\fi
% ---------------------------------------
% Commands for handling colors
% ---------------------------------------


% Command for setting normal text color for some text in math modestyle
% Text color depend on used style of Jagiellonian

% Beamer version of command
\newcommand{\TextWithNormalTextColor}[1]{%
  {\color{jNormalTextFGColor}
    \setbeamercolor{math text}{fg=jNormalTextFGColor} {#1}}
}

% Article and similar classes version of command
% \newcommand{\TextWithNormalTextColor}[1]{%
%   {\color{jNormalTextsFGColor} {#1}}
% }



% Beamer version of command
\newcommand{\NormalTextInMathMode}[1]{%
  {\color{jNormalTextFGColor}
    \setbeamercolor{math text}{fg=jNormalTextFGColor} \text{#1}}
}


% Article and similar classes version of command
% \newcommand{\NormalTextInMathMode}[1]{%
%   {\color{jNormalTextsFGColor} \text{#1}}
% }




% Command that sets color of some mathematical text to the same color
% that has normal text in header (?)

% Beamer version of the command
\newcommand{\MathTextFrametitleFGColor}[1]{%
  {\color{jFrametitleFGColor}
    \setbeamercolor{math text}{fg=jFrametitleFGColor} #1}
}

% Article and similar classes version of the command
% \newcommand{\MathTextWhiteColor}[1]{{\color{jFrametitleFGColor} #1}}





% Command for setting color of alert text for some text in math modestyle

% Beamer version of the command
\newcommand{\MathTextAlertColor}[1]{%
  {\color{jOrange} \setbeamercolor{math text}{fg=jOrange} #1}
}

% Article and similar classes version of the command
% \newcommand{\MathTextAlertColor}[1]{{\color{jOrange} #1}}





% Command that allow you to sets chosen color as the color of some text into
% math mode. Due to some nuances in the way that Beamer handle colors
% it not work in all cases. We hope that in the future we will improve it.

% Beamer version of the command
\newcommand{\SetMathTextsColor}[2]{%
  {\color{#1} \setbeamercolor{math text}{fg=#1} #2}
}


% Article and similar classes version of the command
% \newcommand{\SetMathTextColor}[2]{{\color{#1} #2}}










% ---------------------------------------
% Commands for setting background pictures for some slides
% ---------------------------------------
\newcommand{\TitleBackgroundPicture}
{./PresentationPictures/CommonPictures/Cute_dragon_BG_dark.png}
\newcommand{\SectionBackgroundPicture}
{./PresentationPictures/CommonPictures/Cute_dragon_small_BG_light.png}



\newcommand{\TitleSlideWithPicture}{
  \begingroup

  \usebackgroundtemplate{ % \hspace*{-11.5em}
    \includegraphics[height=\paperheight]{\TitleBackgroundPicture}}

  \maketitle

  \endgroup
}





\newcommand{\SectionSlideWithPicture}[1]{%
  \begingroup

  \usebackgroundtemplate{ % \hspace*{-11.5em}
    \includegraphics[height=\paperheight]{\SectionBackgroundPicture}}

  \setbeamercolor{titlelike}{fg=normal text.fg}

  \section{#1}

  \endgroup
}





\newcommand{\EndingSlide}[1]{%
  \begin{frame}[standout]

    \begingroup

    \color{jFrametitleFGColor}

    #1

    \endgroup

  \end{frame}
}










% ------------------------------------------------------
% BibLaTeX
% ------------------------------------------------------
% Package biblatex, with biber as its backend, allow us to handle
% bibliography entries that use Unicode symbols outside ASCII.
\usepackage[
language=polish,
backend=biber,
style=alphabetic,
url=false,
eprint=true,
]{biblatex}

\addbibresource{Programming-in-Java-Bibliography.bib}





% ------------------------------------------------------
% Importing packages, libraries and setting their configuration
% ------------------------------------------------------





% ------------------------------------------------------
% Local packages
% ------------------------------------------------------
% Local configuration of this particular presentation
\usepackage{./Local-packages/local-settings}










% ------------------------------------------------------------------------------------------------------------------
\title{Programming in Java}
\subtitle{2.~Basics information about Java}

\author{Kamil Ziemian \\
  \email}


% \date{}
% ------------------------------------------------------------------------------------------------------------------










% ####################################################################
% Beginning of the document
\begin{document}
% ####################################################################





% ######################################
% Number of chars: 43k+, 22k+,
% Text is adjusted to the left and words are broken at the end of the line.
\RaggedRight
% ######################################





% ######################################
\maketitle
% ######################################





% ##################
\begin{frame}
  \frametitle{Table~of contents}


  \tableofcontents

\end{frame}
% ##################










% ######################################
\section{Starting with Java}
% ######################################


% ##################
\begin{frame}
  \frametitle{Starting with Java}


  We assume that we have a computer with a GNU/Linux operating system and
  all programs that we need installed.

  We open terminal, preferably \textsc{bash} and create directory
  \texttt{Programming-in-Java\textbackslash}: \\
  \texttt{\$ mkdir Programming-in-Java\textbackslash} \\
  We enter this directory and create file \texttt{PinJ-01.java} in it. \\
  \texttt{\$ cd Programming-in-Java\textbackslash} \\
  \texttt{\$ touch PinJ-01.java}

  The~name of the file \texttt{PinJ-01.java} is problematic for Java
  programs, but for learning is fine. We will talk about proper names for
  Java files, at a~right time.

  Java is notoriously verbose, you always need a~lot of code to do anything.
  We will see this right in the moment.

\end{frame}
% ##################





% ##################
\begin{frame}
  \frametitle{Starting with Java}


  We open the~file \texttt{PinJ-01.java} with some text editor and write
  inside it, what we should expect: a~``Hello, World!'' program.

  \texttt{class Main \{ } \\
  \hphantom{aaaa} \texttt{public static void main(String[] args) \{ } \\
  \hphantom{aaaaaaaa} \texttt{System.out.println("Hello, World!");} \\
  \hphantom{aaaa} \texttt{\} } \\
  \texttt{\} }

  We now compile it witch command \\
  \texttt{\$ javac PinJ-01.java} \\
  and then run it using \\
  \texttt{\$ java Main} \\
  If everything went smoothly, we will see on the screen \\
  \texttt{Hello, World!}

\end{frame}
% ##################










% ######################################
\section{Java background}
% ######################################


% ##################
\begin{frame}
  \frametitle{Historical background}


  We are afraid that now there will be some boring stuff.

  $1972$, the first public version of the~C programming language was
  released, the main designer was
  \colorhref{https://en.wikipedia.org/wiki/Dennis\_Ritchie}{Dennis Ritchie}
  ($1941\text{--}2011$).

  $1985$, the C++ language designed by
  \colorhref{https://en.wikipedia.org/wiki/Bjarne_Stroustrup}
  {Bjarne Stroustrup} (b.~$1950$) enters the stage.

  $1995$, the Java programming language was created
  by~\colorhref{https://en.wikipedia.org/wiki/James\_Gosling}
  {James Gosling} (b.~$1955$).

  Why is this timeline important? Because C++ was intended to be
  improved~C (this is why it is named ``C plus plus''), while Java was very
  heavily influenced by~C++. To be clear Java \alert{wasn't} designed to
  replace C++, but both good and bad parts of C++ influenced it
  a~lot.

\end{frame}
% ##################





% ##################
\begin{frame}
  \frametitle{Java popularity}


  Java was designed with writing web applications in mind. It was
  programming language for the~Internet, pure and simple. In the decade
  $2000\text{--}2010$ many people knew Java, because their web browser told
  them that they don't have it ;).

  Current version of Java is Java $24$ released in March $2025$. Peculiar
  feature of Java is that as of today, \alert{four other} versions of Java
  are still supported, which is a~big number for a~language.

  Today, \currentYear, outside the newest release Java $24$, are supported
  Java~$8$ (released~$2014$), Java $11$ (released~$2018$), Java $17$
  (released~$2021$) and Java $23$ (released~$2023$).

  Why so many versions? It stems mostly from the popularity of Java. If you
  have a~lot of code written in some version of language, you want it to
  keep working.

\end{frame}
% ##################





% ##################
\begin{frame}
  \frametitle{Sun, Oracle and Java}


  James Gosling created Java when he was working for
  \colorhref{https://en.wikipedia.org/wiki/Sun\_Microsystems}
  {Sun Microsystems} and this company held legal control over Java
  programming language. In $2010$ Sun was acquired by
  \colorhref{https://en.wikipedia.org/wiki/Oracle\_Corporation}
  {Oracle Corporation} and now they have rights to Java in their
  hands. It must be noted that in many circles Oracle has a~very bad
  reputation, for using its position and rights to the~disadvantages of
  the~computer science community. Rights to Java are of course part~of
  the~problem.

  We also should note that the situation where one company holds rights
  to programming language is not the norm. Many languages are in the~public
  domain, others are properties of companies. You should always check
  what the~status of your language of interest is.

\end{frame}
% ##################





% ##################
\begin{frame}
  \frametitle{Languages popularity at September 2025}


  The~\colorhref{https://www.tiobe.com/tiobe-index/}{\textsc{tiobe}
    Programming Community Index} is a~tool that tries to measure the
  popularity of programming languages. It is worth looking at the~top $10$
  languages according to \textsc{tiobe} and their percent ratings as for
  September $2025$.

  1) Python, $25.98 \text{\%}$. \\
  2) C++, $8.80 \text{\%}$. \\
  3) C, $8.65 \text{\%}$. \\
  4) Java, $8.35 \text{\%}$. \\
  5) C\#, $6.38 \text{\%}$. \\
  6) JavaScript, $3.22 \text{\%}$. \\
  7) Visual Basic, $2.84 \text{\%}$. \\
  8) Go, $2.32 \text{\%}$. \\
  9) Delphi\textbackslash Object Pascal, $2.26 \text{\%}$. \\
  10) Perl, $2.03 \text{\%}$. \\

\end{frame}
% ##################





% ##################
\begin{frame}
  \frametitle{Few trivia}


  As \textsc{tiobe} index~creators acknowledge measuring language
  popularity is a very hard task and their index isn't perfect, but it is
  still a good tool to find which languages are the~most popular. From at
  least $2002$ Java was always in the top five languages.

  Java language is infamously verbose. To write any program you need to type
  a~lot~of letters.

  Java language was named after Java coffee, which is a~type of coffee
  from
  \colorhref{https://en.wikipedia.org/wiki/Coffee\_production\_in\_Indonesia}
  {Indonesia}.

  Language
  \colorhref{https://en.wikipedia.org/wiki/C\_Sharp\_(programming_language)}
  {C\#}, created in $2000$ by
  \colorhref{https://en.wikipedia.org/wiki/Anders\_Hejlsberg}{Anders
    Hejlsberg} (b.~$1960$), is very heavily influenced by Java. Old joke
  says that C\# is not a~language. It is a~Microsoft environment for Java.

\end{frame}
% ##################





% ##################
\begin{frame}
  \frametitle{Few trivia}


  There is no relation between Java and JavaScript languages. There is
  a~joke that relation between \alert{Java} and \alert{Java}Script is like
  between \alert{car} and \alert{car}pet.
  \colorhref{https://en.wikipedia.org/wiki/Brendan\_Eich}{Brendan Eich}
  (b.~$1961$) the main person behind creating JavaScript, in the interview
  from $2016$, explained origin~of the~name in the following way
  \parencite{Fin-JS-Brendan-Eich-CEO-of-Brave-Ver-2016}.

  JavaScript was developed under the company Netscape in~$1995$. At this
  time in the~\textsc{usa} everything related to the Internet was easy to
  sell and Java language was among one of the~hottest things that year.
  Remember that Java was ``the programming language of the~Internet''.
  According to Eich, Netscape decided to include the word ``Java'' in
  the~name ``JavaScript'', to get a~free advertisement. So, the~name
  ``JavaScript'' is basically a~gimmick, a~marketing ploy.

\end{frame}
% ##################










% ######################################
\section{Some technical terms}
% ######################################


% ##################
\begin{frame}
  \frametitle{Implementation}


  The easiest way to tell what things like implementations means, is to use
  some examples. Consider a~bike. We can tell that a~bike is ``vehicle with
  two wheels that is powered by human muscle using pedals'', which is
  definition good enough for our purpose. At the same time there are many
  different bikes, sold by various companies, in different sizes, made with
  different materials, colors,~etc.

  We will say that the~\textbf{concept} of bike, which is its definition
  given above, has different \textbf{implementations}: every particular bike
  is an implementation of an abstract concept. We hope that it makes the
  meaning~of the~word ``implementation'' clear enough.

\end{frame}
% ##################





% ##################
\begin{frame}
  \frametitle{Operators}


  For the~purpose of this course \textbf{operator} is a~thing that takes
  two values and returns some results computed from them. Basic examples is
  addition operator~\texttt{+}: \\
  \texttt{1 + 2}

  We can spend two hours talking about operators, pointing to holes in our
  definition, but for our curse it is good enough.

\end{frame}
% ##################










% ######################################
\section{Basics information about Java}
% ######################################


% ##################
\begin{frame}
  \frametitle{Too many versions}


  As we mentioned earlier, in the current year \currentYear{} there are
  \alert{five} versions of Java: $8$, $11$, $17$, $23$ and $24$.
  Java had changed a~lot in the time span that passed from $1995$ and old
  code will probably not work in current versions of Java.

  We will spend $90\text{\%}$ or more of our course learning Java core
  functionalities, so our code should work in all available versions of
  Java. But you should be aware that having \alert{five} different
  versions~of the~language used at once is a~source of problems. If you use
  features from Java $17$ in version~$8$ everything will crash and other
  ``funny'' things will happen.

  The~situation can be even worse, since with probability $99.99\text{\%}$
  there are companies that today use unsupported versions of Java. Some
  good programmer wrote it around $2005$ and are still working? So why
  should we update it?

\end{frame}
% ##################





% ##################
\begin{frame}
  \frametitle{Modern Java is fast}


  In this course we aim at learning the core of Java. We hope that this
  course will give you solid foundations for learning deeper layers of this
  programming language, things like``How to handle different versions of
  Java?''. We~also hope that you understand that learning the~core of Java
  is a~big task itself and going beyond it is already too challenging for
  this particular course.

  In the past Java was infamous for its slowness, but things have changed
  a~lot since then and new versions of Java (too many versions ;)) are
  among the fastest languages on the market.

  To be clear, you can write slow code in \alert{every} language. Slowness
  is easy, high speed is hard. But some languages allow you \alert{only}
  to write slow code, while others give you a~chance to make you code
  fast. Modern Java is one of the~fastest languages in that sense.
  At this course we will focus on writing Java code that works. The~problem
  of writing fast code will be tackled, if we find enough time.

\end{frame}
% ##################





% ##################
\begin{frame}
  \frametitle{Java Code Conventions}


  Like most languages, Java gives us a~lot of freedom in the way
  we can write the~same code. To put some order in it,
  \colorhref{https://www.oracle.com/docs/tech/java/codeconventions.pdf}
  {\textit{Java Code Conventions}} document was created and published
  on~September~$12$,~$1997$. On this course we will try to follow this
  document, but fully implementing it is for us a~thing of the~future.

\end{frame}
% ##################










% ######################################
\section{OOP, the C++ way}
% ######################################


% ##################
\begin{frame}
  \frametitle{OOP, the C++ way}


  Java is
  \colorhref{https://en.wikipedia.org/wiki/Object-oriented\_programming}
  {\textsc{oop}} language from top to bottom. The~acronym \textsc{oop}
  stands of course for ``object-oriented programming'', which was one~of
  the coolest concepts in programming for a few decades. Maybe it still is
  today. This concept is far more vague than beginners may imagine, so we
  should look at it for a~few moments.

  C++ was probably the~first language that was object-oriented and became
  widely successful in various practical fields, so for many people
  ``object-oriented'' is synonymous to ``it works like in C++''. Java is
  heavily influenced by C++, so it is object-oriented in the~same
  way as~it.

  There is nothing wrong if for someone \textsc{oop} means ``it's like
  C++'', if she/he understood that it is not the only possible way~of
  doing \textsc{oop}. You don't need to know the other kinds of
  \textsc{oop}, but you should be aware that they exist.

\end{frame}
% ##################





% ##################
\begin{frame}
  \frametitle{OOP, the C++ way}


  We shouldn't waste our time arguing ``Is language X object-oriented
  or~not?''. On this course we adopted broad definition~of
  object-orientation and think that arguing about details should be
  delegated to our free time as a~form~of leisure. What we need to know
  instead is the~fact that C++ and Java way of implementing
  object-orientation are of the~same kind, but not $1 \! : \! 1$ identical.
  This topic can be a~quite subtle one, so we look at it more closely.

  Java adopted C++ way of object-oriented programming, but added its own
  flavor to it. The~main idea in both languages is as follows. The~object
  is a~thing that gathers together data and methods that can be used to
  change a~state of data. Let's use some examples, we will expand on it
  in the~future.

  Consider that we represent a~lamp in our program. We have some date
  about it: the power of the light bubble, current temperature of this
  bubble, information is it on or off, etc.

\end{frame}
% ##################





% ##################
\begin{frame}
  \frametitle{Our lamp, the~OOP way}


  In Java code we can represent our lamp by the variable \texttt{myLamp}.
  Inside it we have variable \texttt{m\_powerOfLightBubble} that contains
  information about the~power of the currently used light bubble. We also
  have methods that work on this lamp. First we have the method
  \texttt{showLightBubblePower()}. Glossing over details, if we put in our
  code line \\
  \texttt{myLamp.showLightBubblePower();} \\
  we will see on the screen something like \\
  \texttt{Light bubble has 60 W of power.}

  Consider now that the representation of our lamp has a~variable named
  \texttt{m\_stateOfLightBubble}, which tells us is a~lamp turned on or off.
  We also have a~method \texttt{showLampState()}, that writes on the screen
  text informing us if the lamp is turned on or off.

\end{frame}
% ##################





% ##################
\begin{frame}
  \frametitle{Our lamp, the~OOP way}


  In other words, if we put in our code the line \\
  \texttt{myLamp.showLampState();} \\
  we will see on the screen \\
  \texttt{The lamp is turned on.} \\
  or \\
  \texttt{The lamp is turned off.}

  It is very natural to have a~methods which turn lamps on or off, we call
  them \texttt{switchLampOn()} and~\texttt{switchLampOff()}. We can use
  first of them by putting in the code line \\
  \texttt{myLamp.switchLampOn();}

  This example can be already hard to understand for the people that didn't
  encounter C++ style \textsc{oop} before, but they shouldn't be worried.
  We will unpack it later in the~course. What is important now is the fact
  that these ideas are used both in C++ and Java, and also in languages like
  Python.

\end{frame}
% ##################





% ##################
\begin{frame}
  \frametitle{Our lamp, the~OOP way}


  We use a~coding convention, in which we prefix the~name of all object
  variables by ``\texttt{m\_}'', which stands for ``member''. This is
  done for~pedagogical reasons, which should bec0me clear when we advance
  in our knowledge of Java.

  You can find code that represents our lamps in the~file
  \texttt{PinJ-01-B-How-to-implement-lamp-in-Java.java}. Don't worry
  if you didn't understand it at this moment.

  At the~same time C++ and Java are not identical and the code that
  allows us to represent lamps in our programs is different in both
  languages. This is basically what we mean when we say that ``one idea
  has different implementations''.

  We now need to leave our lamp and learn the basics of Java. When we will
  know enough about it, we will wrote such code for representing our lamp
  ourselves.

\end{frame}
% ##################










% ######################################
\section{Objects and classes in Java}
% ######################################


% ##################
\begin{frame}
  \frametitle{Objects in Java}


  \textit{Object has a~state, behavior and identity.} (pl.~\textit{Obiekt
    ma stan, zachowanie i~tożsamość.}) Booch, str.~$41$
  \parencite{Eckel-Thinking-in-Java-Ed-polska-Wyd-III-Pub-2003}.

  It is quite hard to give a precise definition of the~object. For
  the~purpose~of this course we will use the following one.
  The~\textbf{object} is a~thing created by the Java language that can
  gather together data and operations. This definition left $1 \, 000$
  questions open, but the proper way to think about it is to treat it as
  the starting point of our journey into the Java world of object-oriented
  programming.

  We previously talked about objects representing lamps, that have
  information about the state of this lamp, is it turned on or off, and
  methods that can turn it on and off. The next concept that we need to
  know is class. There is no exaggeration in saying that in Java classes
  are everywhere.

\end{frame}
% ##################





% ##################
\begin{frame}
  \frametitle{What is class?}


  In our programs we already see the word ``class'' many times, because
  in~Java nothing works without classes. You can think of class as the
  blueprint from the object. For example, a car have its blueprint, project
  that is a set of technical specifications that can be put on the paper
  or~store in computer memory. In Java this would correspond to
  the~definition of the class. At the same time, we can have many cars
  built according to this blueprint, that you can drive. In our Java
  jargon we would call them ``instances of the class''.

  In other words, class is an abstract idea that describes what a given
  object is and how it behaves. Instance of the class is concrete object
  (we are all fully aware that we use the word ``object'') builder
  according to this idea. Let us see some examples.

\end{frame}
% ##################










% ######################################
\section{References}
% ######################################



% ##################
\begin{frame}
  \frametitle{What is a~reference?}


  Concept of references is very close to the concept of pointers (C, C++),
  but they aren't 100\% identical. So, if you understand pointers,
  references will be easy for you.

  To illustrate what is referenced, we will use an~example. Imagine
  that you have a~friend called Joseph Smith and his phone number
  is $123 \, 456$. So, in your phone you have a~contact named ``Joseph
  Smith''. But, because Smith is a~handyman that can repair anything, you
  have another contact in your phone named ``Mr. Handyman'', which
  has the same phone number $123 \, 456$.

  Is ``Joseph Smith'' different from ``Mr. Handyman''? Of course not.
  You just created two contacts to the same person, given them two different
  names, because you think that is useful. Regardless which contact you will
  use, ``Joseph Smith'' or ``Mr. Handyman'', you will make a~call to the
  same person. This is what in computer science we call \textbf{references}.

\end{frame}
% ##################





% ##################
\begin{frame}
  \frametitle{How do references work?}


  In other words the reference is a~way of accessing concrete objects
  and you have multiple references for one and the same object.

  In our examples two contacts ``Joseph Smith'' and ``Mr. Handyman'' shared
  the phone number. It makes sense, since this number is what really allows
  your phone to find the phone of Joseph Smith. This is very similar to what
  is going on in the computers. Everything in the memory of computer
  has its address, the number that tell computer in which place of memory
  object is, just like the phone number identifies particular phone.

  You can give different names to one and the~same object, as long as you
  pair with every name the same address in the memory. Just like you
  can have multiple contacts to the same person in your phone, as long as
  every such contact has the same phone number.

\end{frame}
% ##################





% ##################
\begin{frame}
  \frametitle{Null reference}


  In Java, like in many other languages, there exists the~concept of
  \textbf{null reference}. Imagine that you have contact in your
  phone ``Emma Smith'', but you still didn't get her number, so you
  left it blank. It is example of null reference: you have a~name of
  a~person, but still don't have her phone number. So if you try to call
  Emma Smith, your phone will signal an~error.

  You can imagine that inside a~phone, your contacts look like that. \\
  \texttt{Joseph Smith, number: $123 \, 456$.} \\
  \texttt{Mr. Handyman, number: $123 \, 456$.} \\
  \texttt{Emma Smith, number: null.} \\
  The same is possible in Java. You can create a~name for something,
  but you don't need to~assign the~address of this object (phone number) to
  it. In such Java will assign the~address of such an~object as null. We
  hope that analogy with contacts in your phone makes it easy to understand.

\end{frame}
% ##################





% ##################
\begin{frame}
  \frametitle{We will show example now}


  We will now show you an example~of how references work in practice.
  The~main point is this. Basic types are passing by value, objects by
  references. This is why ``coping'' objects in Java can be confusing.

  Going a~bit deeper, everything that computers use must occupy some
  memory. For example, you need a~byte to store number~$255$.
  Objects of the~class \texttt{TwoNumbers} need at least as much space
  as is needed to store two numbers. Every part of memory has an address
  (phone number), so everything inside the~computer is described
  by~the~address of the memory inside which it is stored.

  Now consider that two people use your program at the same time and
  both need the value stored inside the~variable
  \texttt{veryImportantThing}. If this variable stores integers, and
  integers occupy, let say, four bytes of memory each, we can just copy its
  value to a~new part of memory and now both users have their own copy of
  this important number. This is called \textbf{passing by value}.

\end{frame}
% ##################





% ##################
\begin{frame}
  \frametitle{Passing by~reference}


  But, if \texttt{veryImportantThing} contains a~very big object, making
  a~copy of it becomes problematic. Believe us, managing big objects
  inside memory can be true pain. In such cases it is reasonable
  to use \textbf{passing by reference}. Each user gets information
  where in memory variable \texttt{veryImportantThing} is located and both
  can use the same object.

  Because life cannot be easy, it leads to problems when both users
  want to modify such shared objects, but how to deal with that is quite
  an~advanced topic. We will talk about it, if we have enough time.

  What is important now, is to know that small basic types like
  \texttt{int} and \texttt{double} are passed by copy and everything
  else is passed by reference.

  Our discussion of references required a~lot of simplifications. We also
  omitted many things that are related to it. If something is unclear
  or you want to know more about it, tell us.

\end{frame}
% ##################










% ######################################
\appendix
% ######################################





% ######################################
\EndingSlide{Thank you. Any questions?}
% ######################################





% ####################################################################
% ####################################################################
% Bibliography

\printbibliography





% ####################################################################
% End of the document

\end{document}
