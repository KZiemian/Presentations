% ------------------------------------------------------------------------------------------------------------------
% Basic configuration of Beamera class and Jagiellonian theme
% ------------------------------------------------------------------------------------------------------------------
\RequirePackage[l2tabu, orthodox]{nag}



\ifx\PresentationStyle\notset
  \def\PresentationStyle{dark}
\fi



% Options: t - align frame text to the top.
\documentclass[10pt,t]{beamer}
\mode<presentation>
\usetheme[style=\PresentationStyle]{jagiellonian}




% ------------------------------------------------------------------------------------
% Procesing configuration files of Jagiellonian theme located
% in the directory "preambule"
% ------------------------------------------------------------------------------------
% Configuration for polish language
% Need description
\usepackage[polish]{babel}
% Need description
\usepackage[MeX]{polski}



% ------------------------------
% Better support of polish chars in technical parts of PDF
% ------------------------------
\hypersetup{pdfencoding=auto,psdextra}

% Package "textpos" give as enviroment "textblock" which is very usefull in
% arranging text on slides.

% This is standard configuration of "textpos"
\usepackage[overlay,absolute]{textpos}

% If you need to see bounds of "textblock's" comment line above and uncomment
% one below.

% Caution! When showboxes option is on significant ammunt of space is add
% to the top of textblock and as such, everyting put in them gone down.
% We need to check how to remove this bug.

% \usepackage[showboxes,overlay,absolute]{textpos}



% Setting scale length for package "textpos"
\setlength{\TPHorizModule}{10mm}
\setlength{\TPVertModule}{\TPHorizModule}


% ---------------------------------------
% Packages written for lectures "Geometria 3D dla twórców gier wideo"
% ---------------------------------------
% \usepackage{./Geometry3DPackages/Geometry3D}
% \usepackage{./Geometry3DPackages/Geometry3DIndices}
% \usepackage{./Geometry3DPackages/Geometry3DTikZStyle}
% \usepackage{./ProgramowanieSymulacjiFizykiPaczki/ProgramowanieSymulacjiFizykiTikZStyle}
% \usepackage{./Geometry3DPackages/mathcommands}


% ---------------------------------------
% TikZ
% ---------------------------------------
% Importing TikZ libraries
\usetikzlibrary{arrows.meta}
\usetikzlibrary{positioning}





% % Configuration package "bm" that need for making bold symbols
% \newcommand{\bmmax}{0}
% \newcommand{\hmmax}{0}
% \usepackage{bm}




% ---------------------------------------
% Packages for scientific texts
% ---------------------------------------
% \let\lll\undefined  % Sometimes you must use this line to allow
% "amsmath" package to works with packages with packages for polish
% languge imported
% /preambul/LanguageSettings/JagiellonianPolishLanguageSettings.tex.
% This comments (probably) removes polish letter Ł.
\usepackage{amsmath}  % Packages from American Mathematical Society (AMS)
\usepackage{amssymb}
\usepackage{amscd}
\usepackage{amsthm}
\usepackage{siunitx}  % Package for typsetting SI units.
\usepackage{upgreek}  % Better looking greek letters.
% Example of using upgreek: pi = \uppi


\usepackage{calrsfs}  % Zmienia czcionkę kaligraficzną w \mathcal
% na ładniejszą. Może w innych miejscach robi to samo, ale o tym nic
% nie wiem.










% ---------------------------------------
% Packages written for lectures "Geometria 3D dla twórców gier wideo"
% ---------------------------------------
% \usepackage{./ProgramowanieSymulacjiFizykiPaczki/ProgramowanieSymulacjiFizyki}
% \usepackage{./ProgramowanieSymulacjiFizykiPaczki/ProgramowanieSymulacjiFizykiIndeksy}
% \usepackage{./ProgramowanieSymulacjiFizykiPaczki/ProgramowanieSymulacjiFizykiTikZStyle}





% !!!!!!!!!!!!!!!!!!!!!!!!!!!!!!
% !!!!!!!!!!!!!!!!!!!!!!!!!!!!!!
% EVIL STUFF
\if\JUlogotitle1
\edef\LogoJUPath{LogoJU_\JUlogoLang/LogoJU_\JUlogoShape_\JUlogoColor.pdf}
\titlegraphic{\hfill\includegraphics[scale=0.22]
{./JagiellonianPictures/\LogoJUPath}}
\fi
% ---------------------------------------
% Commands for handling colors
% ---------------------------------------


% Command for setting normal text color for some text in math modestyle
% Text color depend on used style of Jagiellonian

% Beamer version of command
\newcommand{\TextWithNormalTextColor}[1]{%
  {\color{jNormalTextFGColor}
    \setbeamercolor{math text}{fg=jNormalTextFGColor} {#1}}
}

% Article and similar classes version of command
% \newcommand{\TextWithNormalTextColor}[1]{%
%   {\color{jNormalTextsFGColor} {#1}}
% }



% Beamer version of command
\newcommand{\NormalTextInMathMode}[1]{%
  {\color{jNormalTextFGColor}
    \setbeamercolor{math text}{fg=jNormalTextFGColor} \text{#1}}
}


% Article and similar classes version of command
% \newcommand{\NormalTextInMathMode}[1]{%
%   {\color{jNormalTextsFGColor} \text{#1}}
% }




% Command that sets color of some mathematical text to the same color
% that has normal text in header (?)

% Beamer version of the command
\newcommand{\MathTextFrametitleFGColor}[1]{%
  {\color{jFrametitleFGColor}
    \setbeamercolor{math text}{fg=jFrametitleFGColor} #1}
}

% Article and similar classes version of the command
% \newcommand{\MathTextWhiteColor}[1]{{\color{jFrametitleFGColor} #1}}





% Command for setting color of alert text for some text in math modestyle

% Beamer version of the command
\newcommand{\MathTextAlertColor}[1]{%
  {\color{jOrange} \setbeamercolor{math text}{fg=jOrange} #1}
}

% Article and similar classes version of the command
% \newcommand{\MathTextAlertColor}[1]{{\color{jOrange} #1}}





% Command that allow you to sets chosen color as the color of some text into
% math mode. Due to some nuances in the way that Beamer handle colors
% it not work in all cases. We hope that in the future we will improve it.

% Beamer version of the command
\newcommand{\SetMathTextColor}[2]{%
  {\color{#1} \setbeamercolor{math text}{fg=#1} #2}
}


% Article and similar classes version of the command
% \newcommand{\SetMathTextColor}[2]{{\color{#1} #2}}










% ---------------------------------------
% Commands for few special slides
% ---------------------------------------
\newcommand{\EndingSlide}[1]{%
  \begin{frame}[standout]

    \begingroup

    \color{jFrametitleFGColor}

    #1

    \endgroup

  \end{frame}
}










% ---------------------------------------
% Commands for setting background pictures for some slides
% ---------------------------------------
\newcommand{\TitleBackgroundPicture}
{./JagiellonianPictures/Backgrounds/LajkonikDark.png}
\newcommand{\SectionBackgroundPicture}
{./JagiellonianPictures/Backgrounds/LajkonikLight.png}



\newcommand{\TitleSlideWithPicture}{%
  \begingroup

  \usebackgroundtemplate{%
    \includegraphics[height=\paperheight]{\TitleBackgroundPicture}}

  \maketitle

  \endgroup
}





\newcommand{\SectionSlideWithPicture}[1]{%
  \begingroup

  \usebackgroundtemplate{%
    \includegraphics[height=\paperheight]{\SectionBackgroundPicture}}

  \setbeamercolor{titlelike}{fg=normal text.fg}

  \section{#1}

  \endgroup
}










% ---------------------------------------
% Commands for lectures "Geometria 3D dla twórców gier wideo"
% Polish version
% ---------------------------------------
% Komendy teraz wykomentowane były potrzebne, gdy loga były na niebieskim
% tle, nie na białym. A są na białym bo tego chcieli w biurze projektu.
% \newcommand{\FundingLogoWhitePicturePL}
% {./PresentationPictures/CommonPictures/logotypFundusze_biale_bez_tla2.pdf}
\newcommand{\FundingLogoColorPicturePL}
{./PresentationPictures/CommonPictures/European_Funds_color_PL.pdf}
% \newcommand{\EULogoWhitePicturePL}
% {./PresentationPictures/CommonPictures/logotypUE_biale_bez_tla2.pdf}
\newcommand{\EUSocialFundLogoColorPicturePL}
{./PresentationPictures/CommonPictures/EU_Social_Fund_color_PL.pdf}
% \newcommand{\ZintegrUJLogoWhitePicturePL}
% {./PresentationPictures/CommonPictures/zintegruj-logo-white.pdf}
\newcommand{\ZintegrUJLogoColorPicturePL}
{./PresentationPictures/CommonPictures/ZintegrUJ_color.pdf}
\newcommand{\JULogoColorPicturePL}
{./JagiellonianPictures/LogoJU_PL/LogoJU_A_color.pdf}





\newcommand{\GeometryThreeDSpecialBeginningSlidePL}{%
  \begin{frame}[standout]

    \begin{textblock}{11}(1,0.7)

      \begin{flushleft}

        \mdseries

        \footnotesize

        \color{jFrametitleFGColor}

        Materiał powstał w ramach projektu współfinansowanego ze środków
        Unii Europejskiej w ramach Europejskiego Funduszu Społecznego
        POWR.03.05.00-00-Z309/17-00.

      \end{flushleft}

    \end{textblock}





    \begin{textblock}{10}(0,2.2)

      \tikz \fill[color=jBackgroundStyleLight] (0,0) rectangle (12.8,-1.5);

    \end{textblock}


    \begin{textblock}{3.2}(1,2.45)

      \includegraphics[scale=0.3]{\FundingLogoColorPicturePL}

    \end{textblock}


    \begin{textblock}{2.5}(3.7,2.5)

      \includegraphics[scale=0.2]{\JULogoColorPicturePL}

    \end{textblock}


    \begin{textblock}{2.5}(6,2.4)

      \includegraphics[scale=0.1]{\ZintegrUJLogoColorPicturePL}

    \end{textblock}


    \begin{textblock}{4.2}(8.4,2.6)

      \includegraphics[scale=0.3]{\EUSocialFundLogoColorPicturePL}

    \end{textblock}

  \end{frame}
}



\newcommand{\GeometryThreeDTwoSpecialBeginningSlidesPL}{%
  \begin{frame}[standout]

    \begin{textblock}{11}(1,0.7)

      \begin{flushleft}

        \mdseries

        \footnotesize

        \color{jFrametitleFGColor}

        Materiał powstał w ramach projektu współfinansowanego ze środków
        Unii Europejskiej w ramach Europejskiego Funduszu Społecznego
        POWR.03.05.00-00-Z309/17-00.

      \end{flushleft}

    \end{textblock}





    \begin{textblock}{10}(0,2.2)

      \tikz \fill[color=jBackgroundStyleLight] (0,0) rectangle (12.8,-1.5);

    \end{textblock}


    \begin{textblock}{3.2}(1,2.45)

      \includegraphics[scale=0.3]{\FundingLogoColorPicturePL}

    \end{textblock}


    \begin{textblock}{2.5}(3.7,2.5)

      \includegraphics[scale=0.2]{\JULogoColorPicturePL}

    \end{textblock}


    \begin{textblock}{2.5}(6,2.4)

      \includegraphics[scale=0.1]{\ZintegrUJLogoColorPicturePL}

    \end{textblock}


    \begin{textblock}{4.2}(8.4,2.6)

      \includegraphics[scale=0.3]{\EUSocialFundLogoColorPicturePL}

    \end{textblock}

  \end{frame}





  \TitleSlideWithPicture
}



\newcommand{\GeometryThreeDSpecialEndingSlidePL}{%
  \begin{frame}[standout]

    \begin{textblock}{11}(1,0.7)

      \begin{flushleft}

        \mdseries

        \footnotesize

        \color{jFrametitleFGColor}

        Materiał powstał w ramach projektu współfinansowanego ze środków
        Unii Europejskiej w~ramach Europejskiego Funduszu Społecznego
        POWR.03.05.00-00-Z309/17-00.

      \end{flushleft}

    \end{textblock}





    \begin{textblock}{10}(0,2.2)

      \tikz \fill[color=jBackgroundStyleLight] (0,0) rectangle (12.8,-1.5);

    \end{textblock}


    \begin{textblock}{3.2}(1,2.45)

      \includegraphics[scale=0.3]{\FundingLogoColorPicturePL}

    \end{textblock}


    \begin{textblock}{2.5}(3.7,2.5)

      \includegraphics[scale=0.2]{\JULogoColorPicturePL}

    \end{textblock}


    \begin{textblock}{2.5}(6,2.4)

      \includegraphics[scale=0.1]{\ZintegrUJLogoColorPicturePL}

    \end{textblock}


    \begin{textblock}{4.2}(8.4,2.6)

      \includegraphics[scale=0.3]{\EUSocialFundLogoColorPicturePL}

    \end{textblock}





    \begin{textblock}{11}(1,4)

      \begin{flushleft}

        \mdseries

        \footnotesize

        \RaggedRight

        \color{jFrametitleFGColor}

        Treść niniejszego wykładu jest udostępniona na~licencji
        Creative Commons (\textsc{cc}), z~uzna\-niem autorstwa
        (\textsc{by}) oraz udostępnianiem na tych samych warunkach
        (\textsc{sa}). Rysunki i~wy\-kresy zawarte w~wykładzie są
        autorstwa dr.~hab.~Pawła Węgrzyna et~al. i~są dostępne
        na tej samej licencji, o~ile nie wskazano inaczej.
        W~prezentacji wykorzystano temat Beamera Jagiellonian,
        oparty na~temacie Metropolis Matthiasa Vogelgesanga,
        dostępnym na licencji \LaTeX{} Project Public License~1.3c
        pod adresem: \colorhref{https://github.com/matze/mtheme}
        {https://github.com/matze/mtheme}.

        Projekt typograficzny: Iwona Grabska-Gradzińska \\
        Skład: Kamil Ziemian;
        Korekta: Wojciech Palacz \\
        Modele: Dariusz Frymus, Kamil Nowakowski \\
        Rysunki i~wykresy: Kamil Ziemian, Paweł Węgrzyn, Wojciech Palacz

      \end{flushleft}

    \end{textblock}

  \end{frame}
}



\newcommand{\GeometryThreeDTwoSpecialEndingSlidesPL}[1]{%
  \begin{frame}[standout]


    \begin{textblock}{11}(1,0.7)

      \begin{flushleft}

        \mdseries

        \footnotesize

        \color{jFrametitleFGColor}

        Materiał powstał w ramach projektu współfinansowanego ze środków
        Unii Europejskiej w~ramach Europejskiego Funduszu Społecznego
        POWR.03.05.00-00-Z309/17-00.

      \end{flushleft}

    \end{textblock}





    \begin{textblock}{10}(0,2.2)

      \tikz \fill[color=jBackgroundStyleLight] (0,0) rectangle (12.8,-1.5);

    \end{textblock}


    \begin{textblock}{3.2}(1,2.45)

      \includegraphics[scale=0.3]{\FundingLogoColorPicturePL}

    \end{textblock}


    \begin{textblock}{2.5}(3.7,2.5)

      \includegraphics[scale=0.2]{\JULogoColorPicturePL}

    \end{textblock}


    \begin{textblock}{2.5}(6,2.4)

      \includegraphics[scale=0.1]{\ZintegrUJLogoColorPicturePL}

    \end{textblock}


    \begin{textblock}{4.2}(8.4,2.6)

      \includegraphics[scale=0.3]{\EUSocialFundLogoColorPicturePL}

    \end{textblock}





    \begin{textblock}{11}(1,4)

      \begin{flushleft}

        \mdseries

        \footnotesize

        \RaggedRight

        \color{jFrametitleFGColor}

        Treść niniejszego wykładu jest udostępniona na~licencji
        Creative Commons (\textsc{cc}), z~uzna\-niem autorstwa
        (\textsc{by}) oraz udostępnianiem na tych samych warunkach
        (\textsc{sa}). Rysunki i~wy\-kresy zawarte w~wykładzie są
        autorstwa dr.~hab.~Pawła Węgrzyna et~al. i~są dostępne
        na tej samej licencji, o~ile nie wskazano inaczej.
        W~prezentacji wykorzystano temat Beamera Jagiellonian,
        oparty na~temacie Metropolis Matthiasa Vogelgesanga,
        dostępnym na licencji \LaTeX{} Project Public License~1.3c
        pod adresem: \colorhref{https://github.com/matze/mtheme}
        {https://github.com/matze/mtheme}.

        Projekt typograficzny: Iwona Grabska-Gradzińska \\
        Skład: Kamil Ziemian;
        Korekta: Wojciech Palacz \\
        Modele: Dariusz Frymus, Kamil Nowakowski \\
        Rysunki i~wykresy: Kamil Ziemian, Paweł Węgrzyn, Wojciech Palacz

      \end{flushleft}

    \end{textblock}

  \end{frame}





  \begin{frame}[standout]

    \begingroup

    \color{jFrametitleFGColor}

    #1

    \endgroup

  \end{frame}
}



\newcommand{\GeometryThreeDSpecialEndingSlideVideoPL}{%
  \begin{frame}[standout]

    \begin{textblock}{11}(1,0.7)

      \begin{flushleft}

        \mdseries

        \footnotesize

        \color{jFrametitleFGColor}

        Materiał powstał w ramach projektu współfinansowanego ze środków
        Unii Europejskiej w~ramach Europejskiego Funduszu Społecznego
        POWR.03.05.00-00-Z309/17-00.

      \end{flushleft}

    \end{textblock}





    \begin{textblock}{10}(0,2.2)

      \tikz \fill[color=jBackgroundStyleLight] (0,0) rectangle (12.8,-1.5);

    \end{textblock}


    \begin{textblock}{3.2}(1,2.45)

      \includegraphics[scale=0.3]{\FundingLogoColorPicturePL}

    \end{textblock}


    \begin{textblock}{2.5}(3.7,2.5)

      \includegraphics[scale=0.2]{\JULogoColorPicturePL}

    \end{textblock}


    \begin{textblock}{2.5}(6,2.4)

      \includegraphics[scale=0.1]{\ZintegrUJLogoColorPicturePL}

    \end{textblock}


    \begin{textblock}{4.2}(8.4,2.6)

      \includegraphics[scale=0.3]{\EUSocialFundLogoColorPicturePL}

    \end{textblock}





    \begin{textblock}{11}(1,4)

      \begin{flushleft}

        \mdseries

        \footnotesize

        \RaggedRight

        \color{jFrametitleFGColor}

        Treść niniejszego wykładu jest udostępniona na~licencji
        Creative Commons (\textsc{cc}), z~uzna\-niem autorstwa
        (\textsc{by}) oraz udostępnianiem na tych samych warunkach
        (\textsc{sa}). Rysunki i~wy\-kresy zawarte w~wykładzie są
        autorstwa dr.~hab.~Pawła Węgrzyna et~al. i~są dostępne
        na tej samej licencji, o~ile nie wskazano inaczej.
        W~prezentacji wykorzystano temat Beamera Jagiellonian,
        oparty na~temacie Metropolis Matthiasa Vogelgesanga,
        dostępnym na licencji \LaTeX{} Project Public License~1.3c
        pod adresem: \colorhref{https://github.com/matze/mtheme}
        {https://github.com/matze/mtheme}.

        Projekt typograficzny: Iwona Grabska-Gradzińska;
        Skład: Kamil Ziemian \\
        Korekta: Wojciech Palacz;
        Modele: Dariusz Frymus, Kamil Nowakowski \\
        Rysunki i~wykresy: Kamil Ziemian, Paweł Węgrzyn, Wojciech Palacz \\
        Montaż: Agencja Filmowa Film \& Television Production~-- Zbigniew
        Masklak

      \end{flushleft}

    \end{textblock}

  \end{frame}
}





\newcommand{\GeometryThreeDTwoSpecialEndingSlidesVideoPL}[1]{%
  \begin{frame}[standout]

    \begin{textblock}{11}(1,0.7)

      \begin{flushleft}

        \mdseries

        \footnotesize

        \color{jFrametitleFGColor}

        Materiał powstał w ramach projektu współfinansowanego ze środków
        Unii Europejskiej w~ramach Europejskiego Funduszu Społecznego
        POWR.03.05.00-00-Z309/17-00.

      \end{flushleft}

    \end{textblock}





    \begin{textblock}{10}(0,2.2)

      \tikz \fill[color=jBackgroundStyleLight] (0,0) rectangle (12.8,-1.5);

    \end{textblock}


    \begin{textblock}{3.2}(1,2.45)

      \includegraphics[scale=0.3]{\FundingLogoColorPicturePL}

    \end{textblock}


    \begin{textblock}{2.5}(3.7,2.5)

      \includegraphics[scale=0.2]{\JULogoColorPicturePL}

    \end{textblock}


    \begin{textblock}{2.5}(6,2.4)

      \includegraphics[scale=0.1]{\ZintegrUJLogoColorPicturePL}

    \end{textblock}


    \begin{textblock}{4.2}(8.4,2.6)

      \includegraphics[scale=0.3]{\EUSocialFundLogoColorPicturePL}

    \end{textblock}





    \begin{textblock}{11}(1,4)

      \begin{flushleft}

        \mdseries

        \footnotesize

        \RaggedRight

        \color{jFrametitleFGColor}

        Treść niniejszego wykładu jest udostępniona na~licencji
        Creative Commons (\textsc{cc}), z~uzna\-niem autorstwa
        (\textsc{by}) oraz udostępnianiem na tych samych warunkach
        (\textsc{sa}). Rysunki i~wy\-kresy zawarte w~wykładzie są
        autorstwa dr.~hab.~Pawła Węgrzyna et~al. i~są dostępne
        na tej samej licencji, o~ile nie wskazano inaczej.
        W~prezentacji wykorzystano temat Beamera Jagiellonian,
        oparty na~temacie Metropolis Matthiasa Vogelgesanga,
        dostępnym na licencji \LaTeX{} Project Public License~1.3c
        pod adresem: \colorhref{https://github.com/matze/mtheme}
        {https://github.com/matze/mtheme}.

        Projekt typograficzny: Iwona Grabska-Gradzińska;
        Skład: Kamil Ziemian \\
        Korekta: Wojciech Palacz;
        Modele: Dariusz Frymus, Kamil Nowakowski \\
        Rysunki i~wykresy: Kamil Ziemian, Paweł Węgrzyn, Wojciech Palacz \\
        Montaż: Agencja Filmowa Film \& Television Production~-- Zbigniew
        Masklak

      \end{flushleft}

    \end{textblock}

  \end{frame}





  \begin{frame}[standout]


    \begingroup

    \color{jFrametitleFGColor}

    #1

    \endgroup

  \end{frame}
}










% ---------------------------------------
% Commands for lectures "Geometria 3D dla twórców gier wideo"
% English version
% ---------------------------------------
% \newcommand{\FundingLogoWhitePictureEN}
% {./PresentationPictures/CommonPictures/logotypFundusze_biale_bez_tla2.pdf}
\newcommand{\FundingLogoColorPictureEN}
{./PresentationPictures/CommonPictures/European_Funds_color_EN.pdf}
% \newcommand{\EULogoWhitePictureEN}
% {./PresentationPictures/CommonPictures/logotypUE_biale_bez_tla2.pdf}
\newcommand{\EUSocialFundLogoColorPictureEN}
{./PresentationPictures/CommonPictures/EU_Social_Fund_color_EN.pdf}
% \newcommand{\ZintegrUJLogoWhitePictureEN}
% {./PresentationPictures/CommonPictures/zintegruj-logo-white.pdf}
\newcommand{\ZintegrUJLogoColorPictureEN}
{./PresentationPictures/CommonPictures/ZintegrUJ_color.pdf}
\newcommand{\JULogoColorPictureEN}
{./JagiellonianPictures/LogoJU_EN/LogoJU_A_color.pdf}



\newcommand{\GeometryThreeDSpecialBeginningSlideEN}{%
  \begin{frame}[standout]

    \begin{textblock}{11}(1,0.7)

      \begin{flushleft}

        \mdseries

        \footnotesize

        \color{jFrametitleFGColor}

        This content was created as part of a project co-financed by the
        European Union within the framework of the European Social Fund
        POWR.03.05.00-00-Z309/17-00.

      \end{flushleft}

    \end{textblock}





    \begin{textblock}{10}(0,2.2)

      \tikz \fill[color=jBackgroundStyleLight] (0,0) rectangle (12.8,-1.5);

    \end{textblock}


    \begin{textblock}{3.2}(0.7,2.45)

      \includegraphics[scale=0.3]{\FundingLogoColorPictureEN}

    \end{textblock}


    \begin{textblock}{2.5}(4.15,2.5)

      \includegraphics[scale=0.2]{\JULogoColorPictureEN}

    \end{textblock}


    \begin{textblock}{2.5}(6.35,2.4)

      \includegraphics[scale=0.1]{\ZintegrUJLogoColorPictureEN}

    \end{textblock}


    \begin{textblock}{4.2}(8.4,2.6)

      \includegraphics[scale=0.3]{\EUSocialFundLogoColorPictureEN}

    \end{textblock}

  \end{frame}
}



\newcommand{\GeometryThreeDTwoSpecialBeginningSlidesEN}{%
  \begin{frame}[standout]

    \begin{textblock}{11}(1,0.7)

      \begin{flushleft}

        \mdseries

        \footnotesize

        \color{jFrametitleFGColor}

        This content was created as part of a project co-financed by the
        European Union within the framework of the European Social Fund
        POWR.03.05.00-00-Z309/17-00.

      \end{flushleft}

    \end{textblock}





    \begin{textblock}{10}(0,2.2)

      \tikz \fill[color=jBackgroundStyleLight] (0,0) rectangle (12.8,-1.5);

    \end{textblock}


    \begin{textblock}{3.2}(0.7,2.45)

      \includegraphics[scale=0.3]{\FundingLogoColorPictureEN}

    \end{textblock}


    \begin{textblock}{2.5}(4.15,2.5)

      \includegraphics[scale=0.2]{\JULogoColorPictureEN}

    \end{textblock}


    \begin{textblock}{2.5}(6.35,2.4)

      \includegraphics[scale=0.1]{\ZintegrUJLogoColorPictureEN}

    \end{textblock}


    \begin{textblock}{4.2}(8.4,2.6)

      \includegraphics[scale=0.3]{\EUSocialFundLogoColorPictureEN}

    \end{textblock}

  \end{frame}





  \TitleSlideWithPicture
}



\newcommand{\GeometryThreeDSpecialEndingSlideEN}{%
  \begin{frame}[standout]

    \begin{textblock}{11}(1,0.7)

      \begin{flushleft}

        \mdseries

        \footnotesize

        \color{jFrametitleFGColor}

        This content was created as part of a project co-financed by the
        European Union within the framework of the European Social Fund
        POWR.03.05.00-00-Z309/17-00.

      \end{flushleft}

    \end{textblock}





    \begin{textblock}{10}(0,2.2)

      \tikz \fill[color=jBackgroundStyleLight] (0,0) rectangle (12.8,-1.5);

    \end{textblock}


    \begin{textblock}{3.2}(0.7,2.45)

      \includegraphics[scale=0.3]{\FundingLogoColorPictureEN}

    \end{textblock}


    \begin{textblock}{2.5}(4.15,2.5)

      \includegraphics[scale=0.2]{\JULogoColorPictureEN}

    \end{textblock}


    \begin{textblock}{2.5}(6.35,2.4)

      \includegraphics[scale=0.1]{\ZintegrUJLogoColorPictureEN}

    \end{textblock}


    \begin{textblock}{4.2}(8.4,2.6)

      \includegraphics[scale=0.3]{\EUSocialFundLogoColorPictureEN}

    \end{textblock}





    \begin{textblock}{11}(1,4)

      \begin{flushleft}

        \mdseries

        \footnotesize

        \RaggedRight

        \color{jFrametitleFGColor}

        The content of this lecture is made available under a~Creative
        Commons licence (\textsc{cc}), giving the author the credits
        (\textsc{by}) and putting an obligation to share on the same terms
        (\textsc{sa}). Figures and diagrams included in the lecture are
        authored by Paweł Węgrzyn et~al., and are available under the same
        license unless indicated otherwise.\\ The presentation uses the
        Beamer Jagiellonian theme based on Matthias Vogelgesang’s
        Metropolis theme, available under license \LaTeX{} Project
        Public License~1.3c at: \colorhref{https://github.com/matze/mtheme}
        {https://github.com/matze/mtheme}.

        Typographic design: Iwona Grabska-Gradzińska \\
        \LaTeX{} Typesetting: Kamil Ziemian \\
        Proofreading: Wojciech Palacz,
        Monika Stawicka \\
        3D Models: Dariusz Frymus, Kamil Nowakowski \\
        Figures and charts: Kamil Ziemian, Paweł Węgrzyn, Wojciech Palacz

      \end{flushleft}

    \end{textblock}

  \end{frame}
}



\newcommand{\GeometryThreeDTwoSpecialEndingSlidesEN}[1]{%
  \begin{frame}[standout]


    \begin{textblock}{11}(1,0.7)

      \begin{flushleft}

        \mdseries

        \footnotesize

        \color{jFrametitleFGColor}

        This content was created as part of a project co-financed by the
        European Union within the framework of the European Social Fund
        POWR.03.05.00-00-Z309/17-00.

      \end{flushleft}

    \end{textblock}





    \begin{textblock}{10}(0,2.2)

      \tikz \fill[color=jBackgroundStyleLight] (0,0) rectangle (12.8,-1.5);

    \end{textblock}


    \begin{textblock}{3.2}(0.7,2.45)

      \includegraphics[scale=0.3]{\FundingLogoColorPictureEN}

    \end{textblock}


    \begin{textblock}{2.5}(4.15,2.5)

      \includegraphics[scale=0.2]{\JULogoColorPictureEN}

    \end{textblock}


    \begin{textblock}{2.5}(6.35,2.4)

      \includegraphics[scale=0.1]{\ZintegrUJLogoColorPictureEN}

    \end{textblock}


    \begin{textblock}{4.2}(8.4,2.6)

      \includegraphics[scale=0.3]{\EUSocialFundLogoColorPictureEN}

    \end{textblock}





    \begin{textblock}{11}(1,4)

      \begin{flushleft}

        \mdseries

        \footnotesize

        \RaggedRight

        \color{jFrametitleFGColor}

        The content of this lecture is made available under a~Creative
        Commons licence (\textsc{cc}), giving the author the credits
        (\textsc{by}) and putting an obligation to share on the same terms
        (\textsc{sa}). Figures and diagrams included in the lecture are
        authored by Paweł Węgrzyn et~al., and are available under the same
        license unless indicated otherwise.\\ The presentation uses the
        Beamer Jagiellonian theme based on Matthias Vogelgesang’s
        Metropolis theme, available under license \LaTeX{} Project
        Public License~1.3c at: \colorhref{https://github.com/matze/mtheme}
        {https://github.com/matze/mtheme}.

        Typographic design: Iwona Grabska-Gradzińska \\
        \LaTeX{} Typesetting: Kamil Ziemian \\
        Proofreading: Wojciech Palacz,
        Monika Stawicka \\
        3D Models: Dariusz Frymus, Kamil Nowakowski \\
        Figures and charts: Kamil Ziemian, Paweł Węgrzyn, Wojciech Palacz

      \end{flushleft}

    \end{textblock}

  \end{frame}





  \begin{frame}[standout]

    \begingroup

    \color{jFrametitleFGColor}

    #1

    \endgroup

  \end{frame}
}



\newcommand{\GeometryThreeDSpecialEndingSlideVideoVerOneEN}{%
  \begin{frame}[standout]

    \begin{textblock}{11}(1,0.7)

      \begin{flushleft}

        \mdseries

        \footnotesize

        \color{jFrametitleFGColor}

        This content was created as part of a project co-financed by the
        European Union within the framework of the European Social Fund
        POWR.03.05.00-00-Z309/17-00.

      \end{flushleft}

    \end{textblock}





    \begin{textblock}{10}(0,2.2)

      \tikz \fill[color=jBackgroundStyleLight] (0,0) rectangle (12.8,-1.5);

    \end{textblock}


    \begin{textblock}{3.2}(0.7,2.45)

      \includegraphics[scale=0.3]{\FundingLogoColorPictureEN}

    \end{textblock}


    \begin{textblock}{2.5}(4.15,2.5)

      \includegraphics[scale=0.2]{\JULogoColorPictureEN}

    \end{textblock}


    \begin{textblock}{2.5}(6.35,2.4)

      \includegraphics[scale=0.1]{\ZintegrUJLogoColorPictureEN}

    \end{textblock}


    \begin{textblock}{4.2}(8.4,2.6)

      \includegraphics[scale=0.3]{\EUSocialFundLogoColorPictureEN}

    \end{textblock}





    \begin{textblock}{11}(1,4)

      \begin{flushleft}

        \mdseries

        \footnotesize

        \RaggedRight

        \color{jFrametitleFGColor}

        The content of this lecture is made available under a Creative
        Commons licence (\textsc{cc}), giving the author the credits
        (\textsc{by}) and putting an obligation to share on the same terms
        (\textsc{sa}). Figures and diagrams included in the lecture are
        authored by Paweł Węgrzyn et~al., and are available under the same
        license unless indicated otherwise.\\ The presentation uses the
        Beamer Jagiellonian theme based on Matthias Vogelgesang’s
        Metropolis theme, available under license \LaTeX{} Project
        Public License~1.3c at: \colorhref{https://github.com/matze/mtheme}
        {https://github.com/matze/mtheme}.

        Typographic design: Iwona Grabska-Gradzińska;
        \LaTeX{} Typesetting: Kamil Ziemian \\
        Proofreading: Wojciech Palacz,
        Monika Stawicka \\
        3D Models: Dariusz Frymus, Kamil Nowakowski \\
        Figures and charts: Kamil Ziemian, Paweł Węgrzyn, Wojciech
        Palacz \\
        Film editing: Agencja Filmowa Film \& Television Production~--
        Zbigniew Masklak

      \end{flushleft}

    \end{textblock}

  \end{frame}
}



\newcommand{\GeometryThreeDSpecialEndingSlideVideoVerTwoEN}{%
  \begin{frame}[standout]

    \begin{textblock}{11}(1,0.7)

      \begin{flushleft}

        \mdseries

        \footnotesize

        \color{jFrametitleFGColor}

        This content was created as part of a project co-financed by the
        European Union within the framework of the European Social Fund
        POWR.03.05.00-00-Z309/17-00.

      \end{flushleft}

    \end{textblock}





    \begin{textblock}{10}(0,2.2)

      \tikz \fill[color=jBackgroundStyleLight] (0,0) rectangle (12.8,-1.5);

    \end{textblock}


    \begin{textblock}{3.2}(0.7,2.45)

      \includegraphics[scale=0.3]{\FundingLogoColorPictureEN}

    \end{textblock}


    \begin{textblock}{2.5}(4.15,2.5)

      \includegraphics[scale=0.2]{\JULogoColorPictureEN}

    \end{textblock}


    \begin{textblock}{2.5}(6.35,2.4)

      \includegraphics[scale=0.1]{\ZintegrUJLogoColorPictureEN}

    \end{textblock}


    \begin{textblock}{4.2}(8.4,2.6)

      \includegraphics[scale=0.3]{\EUSocialFundLogoColorPictureEN}

    \end{textblock}





    \begin{textblock}{11}(1,4)

      \begin{flushleft}

        \mdseries

        \footnotesize

        \RaggedRight

        \color{jFrametitleFGColor}

        The content of this lecture is made available under a Creative
        Commons licence (\textsc{cc}), giving the author the credits
        (\textsc{by}) and putting an obligation to share on the same terms
        (\textsc{sa}). Figures and diagrams included in the lecture are
        authored by Paweł Węgrzyn et~al., and are available under the same
        license unless indicated otherwise.\\ The presentation uses the
        Beamer Jagiellonian theme based on Matthias Vogelgesang’s
        Metropolis theme, available under license \LaTeX{} Project
        Public License~1.3c at: \colorhref{https://github.com/matze/mtheme}
        {https://github.com/matze/mtheme}.

        Typographic design: Iwona Grabska-Gradzińska;
        \LaTeX{} Typesetting: Kamil Ziemian \\
        Proofreading: Wojciech Palacz,
        Monika Stawicka \\
        3D Models: Dariusz Frymus, Kamil Nowakowski \\
        Figures and charts: Kamil Ziemian, Paweł Węgrzyn, Wojciech
        Palacz \\
        Film editing: IMAVI -- Joanna Kozakiewicz, Krzysztof Magda, Nikodem
        Frodyma

      \end{flushleft}

    \end{textblock}

  \end{frame}
}



\newcommand{\GeometryThreeDSpecialEndingSlideVideoVerThreeEN}{%
  \begin{frame}[standout]

    \begin{textblock}{11}(1,0.7)

      \begin{flushleft}

        \mdseries

        \footnotesize

        \color{jFrametitleFGColor}

        This content was created as part of a project co-financed by the
        European Union within the framework of the European Social Fund
        POWR.03.05.00-00-Z309/17-00.

      \end{flushleft}

    \end{textblock}





    \begin{textblock}{10}(0,2.2)

      \tikz \fill[color=jBackgroundStyleLight] (0,0) rectangle (12.8,-1.5);

    \end{textblock}


    \begin{textblock}{3.2}(0.7,2.45)

      \includegraphics[scale=0.3]{\FundingLogoColorPictureEN}

    \end{textblock}


    \begin{textblock}{2.5}(4.15,2.5)

      \includegraphics[scale=0.2]{\JULogoColorPictureEN}

    \end{textblock}


    \begin{textblock}{2.5}(6.35,2.4)

      \includegraphics[scale=0.1]{\ZintegrUJLogoColorPictureEN}

    \end{textblock}


    \begin{textblock}{4.2}(8.4,2.6)

      \includegraphics[scale=0.3]{\EUSocialFundLogoColorPictureEN}

    \end{textblock}





    \begin{textblock}{11}(1,4)

      \begin{flushleft}

        \mdseries

        \footnotesize

        \RaggedRight

        \color{jFrametitleFGColor}

        The content of this lecture is made available under a Creative
        Commons licence (\textsc{cc}), giving the author the credits
        (\textsc{by}) and putting an obligation to share on the same terms
        (\textsc{sa}). Figures and diagrams included in the lecture are
        authored by Paweł Węgrzyn et~al., and are available under the same
        license unless indicated otherwise.\\ The presentation uses the
        Beamer Jagiellonian theme based on Matthias Vogelgesang’s
        Metropolis theme, available under license \LaTeX{} Project
        Public License~1.3c at: \colorhref{https://github.com/matze/mtheme}
        {https://github.com/matze/mtheme}.

        Typographic design: Iwona Grabska-Gradzińska;
        \LaTeX{} Typesetting: Kamil Ziemian \\
        Proofreading: Wojciech Palacz,
        Monika Stawicka \\
        3D Models: Dariusz Frymus, Kamil Nowakowski \\
        Figures and charts: Kamil Ziemian, Paweł Węgrzyn, Wojciech
        Palacz \\
        Film editing: Agencja Filmowa Film \& Television Production~--
        Zbigniew Masklak \\
        Film editing: IMAVI -- Joanna Kozakiewicz, Krzysztof Magda, Nikodem
        Frodyma

      \end{flushleft}

    \end{textblock}

  \end{frame}
}



\newcommand{\GeometryThreeDTwoSpecialEndingSlidesVideoVerOneEN}[1]{%
  \begin{frame}[standout]

    \begin{textblock}{11}(1,0.7)

      \begin{flushleft}

        \mdseries

        \footnotesize

        \color{jFrametitleFGColor}

        This content was created as part of a project co-financed by the
        European Union within the framework of the European Social Fund
        POWR.03.05.00-00-Z309/17-00.

      \end{flushleft}

    \end{textblock}





    \begin{textblock}{10}(0,2.2)

      \tikz \fill[color=jBackgroundStyleLight] (0,0) rectangle (12.8,-1.5);

    \end{textblock}


    \begin{textblock}{3.2}(0.7,2.45)

      \includegraphics[scale=0.3]{\FundingLogoColorPictureEN}

    \end{textblock}


    \begin{textblock}{2.5}(4.15,2.5)

      \includegraphics[scale=0.2]{\JULogoColorPictureEN}

    \end{textblock}


    \begin{textblock}{2.5}(6.35,2.4)

      \includegraphics[scale=0.1]{\ZintegrUJLogoColorPictureEN}

    \end{textblock}


    \begin{textblock}{4.2}(8.4,2.6)

      \includegraphics[scale=0.3]{\EUSocialFundLogoColorPictureEN}

    \end{textblock}





    \begin{textblock}{11}(1,4)

      \begin{flushleft}

        \mdseries

        \footnotesize

        \RaggedRight

        \color{jFrametitleFGColor}

        The content of this lecture is made available under a Creative
        Commons licence (\textsc{cc}), giving the author the credits
        (\textsc{by}) and putting an obligation to share on the same terms
        (\textsc{sa}). Figures and diagrams included in the lecture are
        authored by Paweł Węgrzyn et~al., and are available under the same
        license unless indicated otherwise.\\ The presentation uses the
        Beamer Jagiellonian theme based on Matthias Vogelgesang’s
        Metropolis theme, available under license \LaTeX{} Project
        Public License~1.3c at: \colorhref{https://github.com/matze/mtheme}
        {https://github.com/matze/mtheme}.

        Typographic design: Iwona Grabska-Gradzińska;
        \LaTeX{} Typesetting: Kamil Ziemian \\
        Proofreading: Wojciech Palacz,
        Monika Stawicka \\
        3D Models: Dariusz Frymus, Kamil Nowakowski \\
        Figures and charts: Kamil Ziemian, Paweł Węgrzyn,
        Wojciech Palacz \\
        Film editing: Agencja Filmowa Film \& Television Production~--
        Zbigniew Masklak

      \end{flushleft}

    \end{textblock}

  \end{frame}





  \begin{frame}[standout]


    \begingroup

    \color{jFrametitleFGColor}

    #1

    \endgroup

  \end{frame}
}



\newcommand{\GeometryThreeDTwoSpecialEndingSlidesVideoVerTwoEN}[1]{%
  \begin{frame}[standout]

    \begin{textblock}{11}(1,0.7)

      \begin{flushleft}

        \mdseries

        \footnotesize

        \color{jFrametitleFGColor}

        This content was created as part of a project co-financed by the
        European Union within the framework of the European Social Fund
        POWR.03.05.00-00-Z309/17-00.

      \end{flushleft}

    \end{textblock}





    \begin{textblock}{10}(0,2.2)

      \tikz \fill[color=jBackgroundStyleLight] (0,0) rectangle (12.8,-1.5);

    \end{textblock}


    \begin{textblock}{3.2}(0.7,2.45)

      \includegraphics[scale=0.3]{\FundingLogoColorPictureEN}

    \end{textblock}


    \begin{textblock}{2.5}(4.15,2.5)

      \includegraphics[scale=0.2]{\JULogoColorPictureEN}

    \end{textblock}


    \begin{textblock}{2.5}(6.35,2.4)

      \includegraphics[scale=0.1]{\ZintegrUJLogoColorPictureEN}

    \end{textblock}


    \begin{textblock}{4.2}(8.4,2.6)

      \includegraphics[scale=0.3]{\EUSocialFundLogoColorPictureEN}

    \end{textblock}





    \begin{textblock}{11}(1,4)

      \begin{flushleft}

        \mdseries

        \footnotesize

        \RaggedRight

        \color{jFrametitleFGColor}

        The content of this lecture is made available under a Creative
        Commons licence (\textsc{cc}), giving the author the credits
        (\textsc{by}) and putting an obligation to share on the same terms
        (\textsc{sa}). Figures and diagrams included in the lecture are
        authored by Paweł Węgrzyn et~al., and are available under the same
        license unless indicated otherwise.\\ The presentation uses the
        Beamer Jagiellonian theme based on Matthias Vogelgesang’s
        Metropolis theme, available under license \LaTeX{} Project
        Public License~1.3c at: \colorhref{https://github.com/matze/mtheme}
        {https://github.com/matze/mtheme}.

        Typographic design: Iwona Grabska-Gradzińska;
        \LaTeX{} Typesetting: Kamil Ziemian \\
        Proofreading: Wojciech Palacz,
        Monika Stawicka \\
        3D Models: Dariusz Frymus, Kamil Nowakowski \\
        Figures and charts: Kamil Ziemian, Paweł Węgrzyn,
        Wojciech Palacz \\
        Film editing: IMAVI -- Joanna Kozakiewicz, Krzysztof Magda, Nikodem
        Frodyma

      \end{flushleft}

    \end{textblock}

  \end{frame}





  \begin{frame}[standout]


    \begingroup

    \color{jFrametitleFGColor}

    #1

    \endgroup

  \end{frame}
}



\newcommand{\GeometryThreeDTwoSpecialEndingSlidesVideoVerThreeEN}[1]{%
  \begin{frame}[standout]

    \begin{textblock}{11}(1,0.7)

      \begin{flushleft}

        \mdseries

        \footnotesize

        \color{jFrametitleFGColor}

        This content was created as part of a project co-financed by the
        European Union within the framework of the European Social Fund
        POWR.03.05.00-00-Z309/17-00.

      \end{flushleft}

    \end{textblock}





    \begin{textblock}{10}(0,2.2)

      \tikz \fill[color=jBackgroundStyleLight] (0,0) rectangle (12.8,-1.5);

    \end{textblock}


    \begin{textblock}{3.2}(0.7,2.45)

      \includegraphics[scale=0.3]{\FundingLogoColorPictureEN}

    \end{textblock}


    \begin{textblock}{2.5}(4.15,2.5)

      \includegraphics[scale=0.2]{\JULogoColorPictureEN}

    \end{textblock}


    \begin{textblock}{2.5}(6.35,2.4)

      \includegraphics[scale=0.1]{\ZintegrUJLogoColorPictureEN}

    \end{textblock}


    \begin{textblock}{4.2}(8.4,2.6)

      \includegraphics[scale=0.3]{\EUSocialFundLogoColorPictureEN}

    \end{textblock}





    \begin{textblock}{11}(1,4)

      \begin{flushleft}

        \mdseries

        \footnotesize

        \RaggedRight

        \color{jFrametitleFGColor}

        The content of this lecture is made available under a Creative
        Commons licence (\textsc{cc}), giving the author the credits
        (\textsc{by}) and putting an obligation to share on the same terms
        (\textsc{sa}). Figures and diagrams included in the lecture are
        authored by Paweł Węgrzyn et~al., and are available under the same
        license unless indicated otherwise. \\ The presentation uses the
        Beamer Jagiellonian theme based on Matthias Vogelgesang’s
        Metropolis theme, available under license \LaTeX{} Project
        Public License~1.3c at: \colorhref{https://github.com/matze/mtheme}
        {https://github.com/matze/mtheme}.

        Typographic design: Iwona Grabska-Gradzińska;
        \LaTeX{} Typesetting: Kamil Ziemian \\
        Proofreading: Leszek Hadasz, Wojciech Palacz,
        Monika Stawicka \\
        3D Models: Dariusz Frymus, Kamil Nowakowski \\
        Figures and charts: Kamil Ziemian, Paweł Węgrzyn,
        Wojciech Palacz \\
        Film editing: Agencja Filmowa Film \& Television Production~--
        Zbigniew Masklak \\
        Film editing: IMAVI -- Joanna Kozakiewicz, Krzysztof Magda, Nikodem
        Frodyma


      \end{flushleft}

    \end{textblock}

  \end{frame}





  \begin{frame}[standout]


    \begingroup

    \color{jFrametitleFGColor}

    #1

    \endgroup

  \end{frame}
}











% ------------------------------------------------------
% BibLaTeX
% ------------------------------------------------------
% Package biblatex, with biber as its backend, allow us to handle
% bibliography entries that use Unicode symbols outside ASCII.
\usepackage[
language=polish,
backend=biber,
style=alphabetic,
url=false,
eprint=true,
]{biblatex}

\addbibresource{Systemy-operacyjne-Bibliography.bib}





% ------------------------------------------------------
% Importing packages, libraries and setting their configuration
% ------------------------------------------------------





% ------------------------------------------------------
% Local packages
% ------------------------------------------------------
% Local configuration of this particular presentation
\usepackage{./Local-packages/local-settings}










% ------------------------------------------------------------------------------------------------------------------
\title{Systemy operacyjne}
\subtitle{3.~Polecenia \textsc{bash}a, ciąg dalszy}

\author{Kamil Ziemian \\
  \email}


% \date{}
% ------------------------------------------------------------------------------------------------------------------










% ####################################################################
% Beginning of the document
\begin{document}
% ####################################################################





% ######################################
% Number of chars: 43k+,
% Text is adjusted to the left and words are broken at the end of the line.
\RaggedRight
% ######################################





% ######################################
\maketitle
% ######################################





% ##################
\begin{frame}
  \frametitle{Spis treści}


  \tableofcontents

\end{frame}
% ##################










% ######################################
\EndingSlide{Ta prezentacja wciąż nie jest ukończona.}
% ######################################










% ######################################
\section{Polecenia \textsc{bash}a, ciąg dalszy}
% ######################################


% ##################
\begin{frame}
  \frametitle{Przesyłanie wielu argumentów do polecenia}


  Wiele poleceń \textsc{bash}a może przyjąć jako swoje argumenty więcej
  niż jedną nazwę obiektu, wyjaśnimy to na przykładzie. Materiały
  dla tego przykładu można znaleźć
  w~\texttt{Materiały-do-prezentacji/SO-03-A-Polecenia-BASHa-ETC/}.

  W~katalogu \texttt{SO-03-A-Polecenia-BASHa-ETC/} w~którym~się znajdujemy
  są m.in. pliki \texttt{Data-01.dat} i~\texttt{Data-02.dat} oraz katalog
  \texttt{Data-for-analysis/}. Aby przenieść oba pliki z~danymi do
  wspomnianego katalogu wystarczy jedno polecenie: \\
  \texttt{\$ mv Data-01.dat Data-02.dat Data-for-analysis/} \\
  Tak samo możemy przynieść do katalogu trzy lub więcej plików: \\
  \texttt{\$ mv Data-01.dat Data-02.dat Data-03.dat Data-for-analysis/}

  Należy dodać, że~katalog do którego przenosimy dwa lub więcej plików,
  \alert{zawsze} musi być ostatnim argumentem przesłanym do~\texttt{mv}. \\
  \texttt{\$ mv Data-01.dat Data-for-analysis/ Data-02.dat} \\
  \texttt{mv: cel 'Data-02.dat' nie jest katalogiem}

\end{frame}
% ##################





% ##################
\begin{frame}
  \frametitle{Przesyłanie wielu argumentów do polecenia}


  \texttt{\$ mv Data-01.dat Data-for-analysis/ Data-02.dat} \\
  Problem z~powyższym poleceniem polega na tym, że~twórcy polecenia
  \texttt{mv}, nie potrafili wymyślić, co w~zasadzie miałoby takie jego
  wywołanie znaczyć. Nie możemy przenieść pliku \texttt{Data-01.dat}
  i~katalogu \texttt{Data-for-analysis/} do pliku \texttt{Data-02.dat},
  bo przecież plik nie służy do przechowywania plików lub katalogów.

  Może więc użytkownik chciał zmienić nazwę jednego z~tych plików?
  Jeśli tak, to którego i~na co? Czyżby chciał zmienić nazwę pliku
  \texttt{Data-01.dat} na nazwę katalogu \texttt{Data-for-analysis/}?
  Rozważania tego typu pokazują, że~decyzja twórców tego polecenia,
  by~uznać, że~przesłanie do~\texttt{mv} trzech lub więcej argumentów,
  przy czym ostatni z~nich \alert{nie} jest katalogiem, uznać za błąd
  użytkownika, była bardzo rozsądna.

  % Analogicznie, możemy za~pomocą jednego polecenia \texttt{touch} utworzyć
  % dwa lub więcej plików: \\
  % \texttt{\$ touch File-01.txt File-02.txt}

  % W~tym miejscu musimy dodać, że~istnieją znacznie potężniejsze, ale też
  % bardziej skomplikowane, metody przesyłania nazw wielu obiektów do jednego
  % polecenia. Ich omówienie odkładamy do momentu, gdy zapoznamy~się już
  % z~podstawowymi sposobami korzystania z~powłoki.

  % Zwykle dość łatwo poprawnie zgadnąć, czy dane polecenie może przyjąć
  % wiele argumentów, czy tylko jedne. Mianowicie, jeśli jest dość oczywiste,
  % że~dane polecenie można zastosować dla wielu argumentów, to zapewne
  % działa właśnie tak jak myślimy. Na razie takie podejście powinno nam
  % wystarczyć.

\end{frame}
% ##################





% ##################
\begin{frame}
  \frametitle{Przesyłanie wielu argumentów do polecenia}


  % \texttt{\$ mv Data-01.dat Data-for-analysis/ Data-02.dat} \\
  % Problem z~powyższym poleceniem polega na tym, że~twórcy polecenia
  % \texttt{mv}, nie potrafili wymyślić, co w~zasadzie miałoby takie jego
  % wywołanie znaczyć. Nie możemy przenieść pliku \texttt{Data-01.dat}
  % i~katalogu \texttt{Data-for-analysis/} do pliku \texttt{Data-02.dat},
  % bo przecież plik nie służy do przechowywania plików lub katalogów.

  % Może więc użytkownik chciał zmienić nazwę jednego z~tych plików?
  % Jeśli tak, to którego i~na co? Czyżby chciał zmienić nazwę pliku
  % \texttt{Data-01.dat} na nazwę katalogu \texttt{Data-for-analysis/}?
  % Rozważania tego typu pokazują, że~decyzja twórców tego polecenia,
  % by~uznać, że~przesłanie do~\texttt{mv} trzech lub więcej argumentów,
  % przy czym ostatni z~nich \alert{nie} jest katalogiem, uznać za błąd
  % użytkownika, była bardzo rozsądna.

  Analogicznie, możemy za~pomocą jednego polecenia \texttt{touch} utworzyć
  dwa lub więcej plików: \\
  \texttt{\$ touch File-01.txt File-02.txt}

  W~tym miejscu musimy dodać, że~istnieją znacznie potężniejsze, ale też
  bardziej skomplikowane, metody przesyłania nazw wielu obiektów do jednego
  polecenia. Ich omówienie odkładamy do momentu, gdy zapoznamy~się już
  z~podstawowymi sposobami korzystania z~powłoki.

  Zwykle dość łatwo poprawnie zgadnąć, czy dane polecenie może przyjąć
  wiele argumentów, czy tylko jedne. Mianowicie, jeśli jest dość oczywiste,
  że~dane polecenie można zastosować dla wielu argumentów, to zapewne
  działa właśnie tak jak myślimy. Na razie takie podejście powinno nam
  wystarczyć.

\end{frame}
% ##################





% ##################
\begin{frame}
  \frametitle{Polecenia i~ścieżki do plików oraz katalogów}


  Załóżmy teraz, że~w~katalogu \texttt{Data-for-analysis/} chcemy utworzyć
  katalog \texttt{Data-January/}. Żeby to zrobić możemy wejść do tego
  katalogu i~użyć tam polecenia \texttt{mkdir} tak jak przedstawiono to
  poniżej. \\
  \texttt{\$ cd Data-for-analysis/} \\
  \texttt{\$ mkdir Data-January/}

  Taki sam rezultat otrzymamy, jeśli poleceniu \texttt{mkdir} prześlemy jako
  argument ścieżkę katalogu, który chcemy utworzyć. \\
  \texttt{\$ mkdir Data-for-analysis/Data-January/} \\
  Zauważmy, że~do polecenia tego przesłaliśmy ścieżkę do~katalogu,
  który \alert{jeszcze} nie istniał. Jest to zupełnie normalne, wszak
  polecenie to służy do tworzenia nowych katalogów. To~i~inne polecenia,
  które służą do tworzenia plików oraz katalogów, w~$99.999\%$ przypadków
  poprawnie „domyślą~się” co mają utworzyć.

  % Oczywiście ścieżki przesyłane do~\texttt{mkdir} mogą być dużo dłuższe.
  % Długość ścieżki nie ma naprawdę znaczenia, byleby tylko wskazywała ona
  % na~poprawną lokalizację miejsca, gdzie ma zostać utworzony katalog.

\end{frame}
% ##################





% ##################
\begin{frame}
  \frametitle{Polecenia i~ścieżki do plików oraz katalogów}


  Oczywiście ścieżki przesyłane do~\texttt{mkdir} mogą być dużo dłuższe.
  Długość ścieżki nie ma naprawdę znaczenia, byleby tylko wskazywała ona
  na~poprawną lokalizację miejsca, gdzie ma zostać utworzony katalog.

  Należy jednak zaznaczyć, że~polecenie \texttt{mkdir} zostało napisane tak,
  że~może utworzyć tylko ostatni katalog pojawiający~się w~danej ścieżce.
  Przykładowo, jeśli wpiszemy \\
  \texttt{\$ mkdir Data-for-analysis/Data-February/Special-data/} \\
  to zobaczymy komunikat typu \\
  \texttt{mkdir: nie można utworzyć katalogu
    „Data-for-analysis/Data-February/Special-data/: Nie ma takiego
    pliku ani katalogu}

  Jest tak dlatego, że~wymagałoby to utworzenia najpierw katalogu
  \texttt{Data-February/}, a~potem utworzenia wewnątrz niego katalogu
  \texttt{Special-data/}. Ponieważ jednak \texttt{mkdir} stworzono w~taki
  sposób, że~ma prawo utworzyć tylko ten katalog, który jest na końcu
  podanej ścieżki, czyli \texttt{Special-data/}, więc polecenie to zgłasza
  błąd.
  % Załóżmy teraz, że~w~katalogu \texttt{Data-for-analysis/} chcemy utworzyć
  % katalog \texttt{Data-January/}. Żeby to zrobić możemy wejść do tego
  % katalogu i~użyć tam polecenia \texttt{mkdir} tak jak przedstawiono to
  % poniżej. \\
  % \texttt{\$ cd Data-for-analysis/} \\
  % \texttt{\$ mkdir Data-January/}

  % Taki sam rezultat otrzymamy, jeśli poleceniu \texttt{mkdir} prześlemy jako
  % argument ścieżkę katalogu, który chcemy utworzyć. \\
  % \texttt{\$ mkdir Data-for-analysis/Data-January/} \\
  % Zauważmy, że~do polecenia tego przesłaliśmy ścieżkę do~katalogu,
  % który \alert{jeszcze} nie istniał. Jest to zupełnie normalne, wszak
  % polecenie to służy do tworzenia nowych katalogów. To~i~inne polecenia,
  % które służą do tworzenia plików oraz katalogów, w~$99.999\%$ przypadków
  % poprawnie „domyślą~się” co mają utworzyć.

\end{frame}
% ##################





% ##################
\begin{frame}
  \frametitle{Polecenia i~ścieżki}


  Dla porządku zaznaczymy, że~pojedyncze polecenie \texttt{mkdir} może bez
  problemu utworzyć dwa lub więcej katalogów, jeśli tylko podamy ich nazwy
  (ścieżki) jako dwa osobne argumenty. Przykładowo \\
  \texttt{\$ mkdir Data-February/ Data-March/} \\
  utworzy w~katalogu w~którym~się znajdujemy dwa katalogi o~wymienionych
  wyżej nazwach.

  Co jednak jeśli katalog \texttt{Data-February/} jeszcze nie istnieje,
  a~my wprowadzimy następujące polecenie? \\
  \texttt{\$ mkdir Data-February/ Data-February/Special-data/} \\
  Mówiąc zupełnie otwarcie, tego typu polecenia każdy musi przetestować
  na~własną odpowiedzialność. My stoimy na stanowisku, że~najlepiej jest
  takich rzeczy unikać, bo zysk z~ich stosowania jest niewielki, za~to
  problemy do jakich mogą prowadzić, potrafią być bardzo duże. Jedna
  z~mądrości informatyki mądrze mówi: \\
  \texttt{Better to be clear, than clever.}

\end{frame}
% ##################





% ##################
\begin{frame}
  \frametitle{Polecenia i~ścieżki}


  W~ogólności, należy przyjąć, że~w~każdym poleceniu, którego argumentem
  może być nazwa pliku lub katalogu, może wystąpić też dowolnie
  skomplikowana ścieżka do~niego. Zawsze możemy jako argumentu polecenia
  przesłać ścieżkę bezwzględna, czyli taką, która zaczyna~się od symbolu
  katalogu root „\texttt{/}”. Jeśli chodzi o~ścieżki względne, to prawie
  zawsze są one poprawnymi argumentami, acz niekiedy można~się natknąć na
  pewne wyjątki od~tej reguły.

  Rozpatrzmy teraz odrobinę inny przykład. Załóżmy teraz, że~w~katalogu
  gdzie jesteśmy, znajduje~się katalog \texttt{Data-for-analysis/}
  i~chcemy zobaczyć jego zawartość. Jak poprzednio, żeby to zrobić wystarczy
  wpisać polecenie \\
  \texttt{\$ ls Data-for-analysis/} \\
  \texttt{Data-01.dat} \quad \texttt{Data-02.dat} \\
  Również tutaj, długość ścieżki nie ma znaczenie, ważne jest jedynie,
  czy~jest ona poprawna.

\end{frame}
% ##################





% ##################
\begin{frame}
  \frametitle{Polecenia i~ścieżki}


  Przykładowo, zakładając, że~poniższa ścieżka do~katalogu
  \texttt{Data-for-analysis/} jest poprawna, możemy ten sam rezultat uzyskać
  pisząc \\
  \texttt{ls /home/adam/Data-directory/Data-for-analysis/}

  Niektórym z~Państwa może~się pojawić w~głowie takie pytanie. „Dlaczego
  teraz przesłaliśmy argument do polecenia \texttt{ls}, choć wcześniej
  wywoływaliśmy je bez żadnego argumentu?” Nie wchodząc w~szczegóły,
  w~poprzednich przypadkach polecenie \texttt{ls} samo znajdowało sobie
  argument, a~był nim bieżący katalog. To powinno być dość proste
  do~zrozumienia z~kontekstu, dlatego nie będziemy zbyt często
  zagłębiać~się w~tego typu kwestie.

\end{frame}
% ##################





% ##################
\begin{frame}
  \frametitle{Usuwanie katalogów: polecenie \texttt{rmdir}}


  Dla zupełności wykładu, wspomnimy trochę o~tym jak możemy usunąć
  niepotrzebny katalog. Do usuwania katalogów należy w~pierwszy rzędzie
  użyć bardzo bezpiecznego polecenia \texttt{rmdir}: \\
  \texttt{\$ rmdir Niepotrzebny-katalog/} \\
  Polecenie to odmówi usunięcia danego katalogu, jeśli nie jest on~pusty,
  chroni więc nas przed usunięciem katalogu zawierającym ważne dane.

  Oczywiście, czasem tym czego chcemy jest usunięcie katalogu wraz
  z~plikami które \alert{zawiera}. To oczywiście też jest do za pomocą
  polecenia powłoki \textsc{bash}, ale ponieważ to polecenie jest
  niebezpieczne w~użyciu omówimy je dopiero za~jakiś czas.

\end{frame}
% ##################










% ######################################
\section{Jeszcze o~poruszaniu~się po katalogach}
% ######################################


% ##################
\begin{frame}
  \frametitle{Rola systemu operacyjnego}


  Zacznijmy od~przyjęcia, że~jak poprzednio znajdujemy~się w~katalogu
  \texttt{SO-03-A-Polecenia-BASHa-ETC/}. Tak jak system Windows,
  GNU/Linux posiada koncepcję plików ukrytych, czyli takich, które
  normalnie nie~są widoczne dla użytkownika. Ze~względów historycznych,
  przyjęto zasadę, że~jeśli nazwa pliku zaczyna~się od symbolu~„\texttt{.}”
  to plik jest niewidocznych od użytkownika.

  By zobaczyć wszystkie pliki, które zawiera dany katalog należy
  do~polecenia \texttt{ls} przesłać opcję~\texttt{-a}
  (od ang.~\textit{all}). \\
  \texttt{\$ ls -a} \\
  \texttt{.} \quad \texttt{..} \quad \texttt{Data-01.dat} \quad
  \texttt{Data-02.dat} \quad \directoryName{\texttt{Data-for-analysis}} \\
  Widzimy, że~w~katalogu tym mamy dwa pliki ukryte, o~dość osobliwych
  nazwach, jedna kropka i~dwie kropki. Pliki te znajdują~się w~praktycznie
  każdym katalogu w~systemie GNU/Linux, a~ich nazwy to pewnie zaszłość
  historyczna z~lat $70$-tych XX wieku, kiedy pamięć była bardzo mała
  i~każdy symbol~się liczył.

\end{frame}
% ##################





% ##################
\begin{frame}
  \frametitle{Pliki „\texttt{.}” i~„\texttt{..}”}


  Zgodnie z~zasadą, że~w~systemie GNU/Linux wszystko jest plikiem, a~co nie
  jest plikiem jest procesem, pliki „\texttt{.}” i~„\texttt{..}” pozwalają
  systemowi zarządzać katalogami. Katalog to wszak tylko inny rodzaj pliku.

  Zacznijmy od~pliku „\texttt{.}”. Plik ten oznacza po prostu katalog
  w~którym~się znajdujemy. Co więcej możemy za jego pomocą przejść
  do~katalogu w~którym jesteśmy (czyli nic nie zrobić), za pomocą
  polecenia: \\
  \texttt{\$ cd .} \\
  Jakkolwiek z~praktycznego punktu widzenia to polecenie jest bezużyteczne,
  warto wiedzieć, że~istnieje i~działa poprawnie.

\end{frame}
% ##################








% % ##################
% \begin{frame}
%   \frametitle{Powłoka BASH}



% \end{frame}
% % ##################





% % ##################
% \begin{frame}
%   \frametitle{Postęp technologiczny}



% \end{frame}
% % ##################





% % ##################
% \begin{frame}
%   \frametitle{Problemy z~bezpieczeństwem}




% \end{frame}
% % ##################










% % ######################################
% \section{Pliki i~procesy}
% % ######################################



% % ##################
% \begin{frame}
%   \frametitle{Pliki i~procesy}




% \end{frame}
% % ##################





% % ##################
% \begin{frame}
%   \frametitle{Pliki i~procesy}




% \end{frame}
% % ##################










% % ######################################
% \section{Podstawy pracy z~poziomu powłoki BASH}
% % ######################################



% % ##################
% \begin{frame}
%   \frametitle{BASH nie zna słowa litość}




% \end{frame}
% % ##################





% % ##################
% \begin{frame}
%   \frametitle{Zaczynamy}


% \end{frame}
% % ##################





% % ##################
% \begin{frame}
%   \frametitle{Polecenia}



% \end{frame}
% % ##################





% % ##################
% \begin{frame}
%   \frametitle{Polecenia}




% \end{frame}
% % ##################





% % ##################
% \begin{frame}
%   \frametitle{Polecenie \texttt{ls}}




% \end{frame}
% % ##################





% % ##################
% \begin{frame}
%   \frametitle{Polecenie \texttt{pwd}}




% \end{frame}
% % ##################





% % ##################
% \begin{frame}
%   \frametitle{System plików}



% \end{frame}
% % ##################





% % ##################
% \begin{frame}
%   \frametitle{System plików}




% \end{frame}
% % ##################





% % ##################
% \begin{frame}
%   \frametitle{System plików}



% \end{frame}
% % ##################





% % ##################
% \begin{frame}
%   \frametitle{System plików}



% \end{frame}
% % ##################





% % ##################
% \begin{frame}
%   \frametitle{Katalog domowy}




% \end{frame}
% % ##################





% % ##################
% \begin{frame}
%   \frametitle{Polecenie \texttt{cd}}




% \end{frame}
% % ##################





% % ##################
% \begin{frame}
%   \frametitle{Polecenie \texttt{cd}}



% \end{frame}
% % ##################





% % ##################
% \begin{frame}
%   \frametitle{Edycja plików z~pomocą programu gedit}




% \end{frame}
% % ##################





% % ##################
% \begin{frame}
%   \frametitle{Okienko gedita}


% \end{frame}
% % ##################





% % ##################
% \begin{frame}
%   \frametitle{Edytory tekstów}



% \end{frame}
% % ##################










% % ######################################
% \section{Dalsze informacje o~pracy z~BASHem}
% % ######################################


% % ##################
% \begin{frame}
%   \frametitle{Pewne subtelności}




% \end{frame}
% % ##################





% % ##################
% \begin{frame}
%   \frametitle{Polecenie \texttt{mkdir}}



% \end{frame}
% % ##################





% % ##################
% \begin{frame}
%   \frametitle{Polecenie \texttt{mv}}



% \end{frame}
% % ##################





% % ##################
% \begin{frame}
%   \frametitle{Polecenie \texttt{mv}}




% \end{frame}
% % ##################





% % ##################
% \begin{frame}
%   \frametitle{Zagrożenia jakie stwarza \texttt{mv}}



% \end{frame}
% % ##################





% % ##################
% \begin{frame}
%   \frametitle{Jak rozsądnie korzystać z~\texttt{mv}?}




% \end{frame}
% % ##################





% % ##################
% \begin{frame}
%   \frametitle{Opcje (flagi)}



% \end{frame}
% % ##################





% % ##################
% \begin{frame}
%   \frametitle{Czemu życie nie może być proste?}




% \end{frame}
% % ##################










% % ######################################
% \section{Autouzupełnianie w~BASHu}
% % ######################################


% % ##################
% \begin{frame}
%   \frametitle{Autouzupełnianie w~BASHu}




% \end{frame}
% % ##################





% % ##################
% \begin{frame}
%   \frametitle{Autouzupełnianie w~BASHu}



% \end{frame}
% % ##################





% % ##################
% \begin{frame}
%   \frametitle{Autouzupełnianie w~BASHu}




% \end{frame}
% % ##################










% % ######################################
% \appendix
% % ######################################





% % ######################################
% \EndingSlide{Dziękuję! Pytania?}
% % ######################################










% % ######################################
% \section{Dlaczego BASH~się tak nazywa}
% % ######################################



% % ##################
% \begin{frame}
%   \frametitle{Dlaczego BASH~się tak nazywa?}




% \end{frame}
% % ##################






% % ##################
% \begin{frame}
%   \frametitle{????}




% \end{frame}
% % ##################






% % ##################
% \endingslide???{}
% % ##################



% % % ##################
% % \jagiellonianendslide{?????}
% % % ##################





% ####################################################################
% ####################################################################
% Bibliography

\printbibliography





% ####################################################################
% End of the document

\end{document}
