% ------------------------------------------------------------------------------------------------------------------
% Basic configuration of Beamera class and Jagiellonian theme
% ------------------------------------------------------------------------------------------------------------------
\RequirePackage[l2tabu, orthodox]{nag}



\ifx\PresentationStyle\notset
  \def\PresentationStyle{dark}
\fi



% Options: t -- align frame text to the top.
\documentclass[10pt,t]{beamer}
\mode<presentation>
\usetheme[style=\PresentationStyle]{jagiellonian}




% ------------------------------------------------------------------------------------
% Procesing configuration files of Jagiellonian theme located
% in the directory "preambule"
% ------------------------------------------------------------------------------------
% Configuration for polish language
% Need description
\usepackage[polish]{babel}
% Need description
\usepackage[MeX]{polski}



% ------------------------------
% Better support of polish chars in technical parts of PDF
% ------------------------------
\hypersetup{pdfencoding=auto,psdextra}

% Package "textpos" give as enviroment "textblock" which is very usefull in
% arranging text on slides.

% This is standard configuration of "textpos"
\usepackage[overlay,absolute]{textpos}

% If you need to see bounds of "textblock's" comment line above and uncomment
% one below.

% Caution! When showboxes option is on significant ammunt of space is add
% to the top of textblock and as such, everyting put in them gone down.
% We need to check how to remove this bug.

% \usepackage[showboxes,overlay,absolute]{textpos}



% Setting scale length for package "textpos"
\setlength{\TPHorizModule}{10mm}
\setlength{\TPVertModule}{\TPHorizModule}


% ---------------------------------------
% Packages written for lectures "Geometria 3D dla twórców gier wideo"
% ---------------------------------------
% \usepackage{./Geometry3DPackages/Geometry3D}
% \usepackage{./Geometry3DPackages/Geometry3DIndices}
% \usepackage{./Geometry3DPackages/Geometry3DTikZStyle}
% \usepackage{./ProgramowanieSymulacjiFizykiPaczki/ProgramowanieSymulacjiFizykiTikZStyle}
% \usepackage{./Geometry3DPackages/mathcommands}


% ---------------------------------------
% TikZ
% ---------------------------------------
% Importing TikZ libraries
\usetikzlibrary{arrows.meta}
\usetikzlibrary{positioning}





% % Configuration package "bm" that need for making bold symbols
% \newcommand{\bmmax}{0}
% \newcommand{\hmmax}{0}
% \usepackage{bm}




% ---------------------------------------
% Packages for scientific texts
% ---------------------------------------
% \let\lll\undefined  % Sometimes you must use this line to allow
% "amsmath" package to works with packages with packages for polish
% languge imported
% /preambul/LanguageSettings/JagiellonianPolishLanguageSettings.tex.
% This comments (probably) removes polish letter Ł.
\usepackage{amsmath}  % Packages from American Mathematical Society (AMS)
\usepackage{amssymb}
\usepackage{amscd}
\usepackage{amsthm}
\usepackage{siunitx}  % Package for typsetting SI units.
\usepackage{upgreek}  % Better looking greek letters.
% Example of using upgreek: pi = \uppi


\usepackage{calrsfs}  % Zmienia czcionkę kaligraficzną w \mathcal
% na ładniejszą. Może w innych miejscach robi to samo, ale o tym nic
% nie wiem.










% ---------------------------------------
% Packages written for lectures "Geometria 3D dla twórców gier wideo"
% ---------------------------------------
% \usepackage{./ProgramowanieSymulacjiFizykiPaczki/ProgramowanieSymulacjiFizyki}
% \usepackage{./ProgramowanieSymulacjiFizykiPaczki/ProgramowanieSymulacjiFizykiIndeksy}
% \usepackage{./ProgramowanieSymulacjiFizykiPaczki/ProgramowanieSymulacjiFizykiTikZStyle}





% !!!!!!!!!!!!!!!!!!!!!!!!!!!!!!
% !!!!!!!!!!!!!!!!!!!!!!!!!!!!!!
% EVIL STUFF
\if\JUlogotitle1
\edef\LogoJUPath{LogoJU_\JUlogoLang/LogoJU_\JUlogoShape_\JUlogoColor.pdf}
\titlegraphic{\hfill\includegraphics[scale=0.22]
{./JagiellonianPictures/\LogoJUPath}}
\fi
% ---------------------------------------
% Commands for handling colors
% ---------------------------------------


% Command for setting normal text color for some text in math modestyle
% Text color depend on used style of Jagiellonian

% Beamer version of command
\newcommand{\TextWithNormalTextColor}[1]{%
  {\color{jNormalTextFGColor}
    \setbeamercolor{math text}{fg=jNormalTextFGColor} {#1}}
}

% Article and similar classes version of command
% \newcommand{\TextWithNormalTextColor}[1]{%
%   {\color{jNormalTextsFGColor} {#1}}
% }



% Beamer version of command
\newcommand{\NormalTextInMathMode}[1]{%
  {\color{jNormalTextFGColor}
    \setbeamercolor{math text}{fg=jNormalTextFGColor} \text{#1}}
}


% Article and similar classes version of command
% \newcommand{\NormalTextInMathMode}[1]{%
%   {\color{jNormalTextsFGColor} \text{#1}}
% }




% Command that sets color of some mathematical text to the same color
% that has normal text in header (?)

% Beamer version of the command
\newcommand{\MathTextFrametitleFGColor}[1]{%
  {\color{jFrametitleFGColor}
    \setbeamercolor{math text}{fg=jFrametitleFGColor} #1}
}

% Article and similar classes version of the command
% \newcommand{\MathTextWhiteColor}[1]{{\color{jFrametitleFGColor} #1}}





% Command for setting color of alert text for some text in math modestyle

% Beamer version of the command
\newcommand{\MathTextAlertColor}[1]{%
  {\color{jOrange} \setbeamercolor{math text}{fg=jOrange} #1}
}

% Article and similar classes version of the command
% \newcommand{\MathTextAlertColor}[1]{{\color{jOrange} #1}}





% Command that allow you to sets chosen color as the color of some text into
% math mode. Due to some nuances in the way that Beamer handle colors
% it not work in all cases. We hope that in the future we will improve it.

% Beamer version of the command
\newcommand{\SetMathTextsColor}[2]{%
  {\color{#1} \setbeamercolor{math text}{fg=#1} #2}
}


% Article and similar classes version of the command
% \newcommand{\SetMathTextColor}[2]{{\color{#1} #2}}










% ---------------------------------------
% Commands for setting background pictures for some slides
% ---------------------------------------
\newcommand{\TitleBackgroundPicture}
{./PresentationPictures/CommonPictures/Cute_dragon_BG_dark.png}
\newcommand{\SectionBackgroundPicture}
{./PresentationPictures/CommonPictures/Cute_dragon_small_BG_light.png}



\newcommand{\TitleSlideWithPicture}{
  \begingroup

  \usebackgroundtemplate{ % \hspace*{-11.5em}
    \includegraphics[height=\paperheight]{\TitleBackgroundPicture}}

  \maketitle

  \endgroup
}





\newcommand{\SectionSlideWithPicture}[1]{%
  \begingroup

  \usebackgroundtemplate{ % \hspace*{-11.5em}
    \includegraphics[height=\paperheight]{\SectionBackgroundPicture}}

  \setbeamercolor{titlelike}{fg=normal text.fg}

  \section{#1}

  \endgroup
}





\newcommand{\EndingSlide}[1]{%
  \begin{frame}[standout]

    \begingroup

    \color{jFrametitleFGColor}

    #1

    \endgroup

  \end{frame}
}










% ------------------------------------------------------
% BibLaTeX
% ------------------------------------------------------
% Package biblatex, with biber as its backend, allow us to handle
% bibliography entries that use Unicode symbols outside ASCII.
\usepackage[
language=polish,
backend=biber,
style=alphabetic,
url=false,
eprint=true,
]{biblatex}

\addbibresource{Systemy-operacyjne-Bibliography.bib}





% ------------------------------------------------------
% Importing packages, libraries and setting their configuration
% ------------------------------------------------------





% ------------------------------------------------------
% Local packages
% ------------------------------------------------------
% Local configuration of this particular presentation
\usepackage{./Local-packages/local-settings}










% ------------------------------------------------------------------------------------------------------------------
\title{Programming in Java}
\subtitle{Bitwise operators}

\author{Kamil Ziemian \\
  \email}


% \date{}
% ------------------------------------------------------------------------------------------------------------------










% ####################################################################
% Beginning of the document
\begin{document}
% ####################################################################





% ######################################
% Number of chars: 43k+,
% Text is adjusted to the left and words are broken at the end of the line.
\RaggedRight
% ######################################





% ######################################
\maketitle
% ######################################





% ##################
\begin{frame}
  \frametitle{Spis treści}


  \tableofcontents

\end{frame}
% ##################






% ######################################
\EndingSlide{This presentation isn't finished.}
% ######################################



% ######################################
\section{Bitewise operators}
% ######################################


% ##################
\begin{frame}
  \frametitle{Bitewise operators can be hard}


  Since bitwise operators are often confusing for beginners, so it can be
  valuable to spent some time on them. We start from the question, why did
  Java have them in the first place? The answer is twofold. First of all,
  C~has them, because C~is a~language designed to work very close to
  hardware, so it needed a~way to configure single bits, like bits in
  register of some machine. Do everybody know what is a~register?

  Second, Java was created in $1995$ and one of their main goals at this
  time was to wrote code for various tools used with TV adapters. Which also
  requires working on very low level and changing single bits
  (p.~\parencite{Eckel-Thinking-in-Java-Ed-polska-Wyd-III-Pub-2003}).

  \alert{Attention.} In this section of presentations $0$ means
  \alert{single} bit in the zeroth state, the~same is true for $1$.
  To describe numbers in decimal base, we will add to them subscript
  $\text{d}$: $0_{ \text{d} }, 1_{ \text{d} }, 2_{ \text{d} }, 3_{ \text{d} }, \ldots$

\end{frame}
% ##################





% ##################
\begin{frame}
  \frametitle{Bitewise conjunction}


  We first need to understand how bitwise operators work for the pair of
  bits. We start with bitwise conjunction operator $\&$, which obey
  following rules: \\
  $1 \, \& \, 1 == 1$, $1 \, \& \, 0 == 1$, $0 \, \& \, 1 == 0$,
  $0 \, \& \, 0 == 0$.

  We should note that Java we \alert{cannot} use $\&\&$ or any other logical
  operator with numbers. This is a~good decisions, a~significant improvement
  on C and C++. Bitwise operators can be used on \alert{both} numbers
  and boolean variables. In such case $\text{true}$ count as single bit
  $1$ and $\text{false}$ as single~$0$. \\
  $\text{true} \, \& \, \text{true} == \text{true}$,
  $\text{true} \, \& \, \text{false} == \text{false}$,
  $\text{false} \, \& \, \text{true} == \text{false}$,
  $\text{false} \, \& \, \text{false} == \text{false}$.

  We need to stress that in Java you can use bitwise operators only with
  boolean and integer types. We will now cover the use of bitwise operators
  for integer numbers.

\end{frame}
% ##################





% ##################
\begin{frame}
  \frametitle{Bitewise conjunction for integers}


  To make our considerations easier to understand, we will use only integers
  numbers than can be written down using for bits, which basically means
  from $0$ to $15$. We adopt the notation inspired by language Go, in which
  $7_{ \text{d} } == 0\text{b}0111 == 1_{ \text{d} } + 1_{ \text{d} } \cdot
  ( 2_{ \text{d} } )^{ 1_{ \text{d} } } + 1_{ \text{d} } \cdot
  ( 2_{ \text{d} } )^{ 2_{ \text{d} } } + 0_{ \text{d} } \cdot
  ( 2_{ \text{d} } )^{ 3_{ \text{d} } }$. At the end of this section you
  will find full table of numbers from $0_{ \text{d} }$ to~$15_{ \text{d} }$
  in binary notation.

  Basically bitwise operators take two numbers, perform operation on every
  pair of the coresponding bits and return number computed in such way.
  Simple examples should explain it clearly.

  We have $7_{ \text{d} } == 0\text{b}0111$
  and~$14_{ \text{d} } == 0\text{b}1110$. \\
  $7_{ \text{d} } \, \& \, 14_{ \text{d} } == 0\text{b}0111 \, \& \,
  0\text{b}1110 ==
  0\text{b}( 0 \, \& \, 1 )( 1 \, \& \, 1 )( 1 \, \& \, 1 )
  ( 1 \, \& \, 0 ) ==$ \\
  $== 0\text{b}0110 == 6_{ \text{d} }$,
  so~$7_{ \text{d} } \, \& \, 14_{ \text{d} } == 6_{ \text{d} }$.

\end{frame}
% ##################





% ##################
\begin{frame}
  \frametitle{Bitewise conjunction for integers}


  $5_{ \text{d} } == 0\text{b}0101$, $9_{ \text{d} } == 0\text{b}1001$. \\
  $5_{ \text{d} } \, \& \, 9_{ \text{d} } == 0\text{b}0101 \, \& \,
  0\text{b}1001 == 0\text{b}( 0 \, \& \, 1 )( 1 \, \& \, 0 )
  ( 0 \, \& \, 0 )( 1 \, \& \, 1 ) ==$ \\
  $== 0\text{b}0001 == 1_{ \text{d} }$,
  so~$5_{ \text{d} } \, \& \, 9_{ \text{d} } == 1_{ \text{d} }$.

  $3_{ \text{d} } == 0\text{b}0011$, $12_{ \text{d} } == 0\text{b}1100$. \\
  $3_{ \text{d} } \, \& \, 12_{ \text{d} } == 0\text{b}0011 \, \& \,
  0\text{b}1100 == 0\text{b}( 0 \, \& \, 1 )( 0 \, \& \, 1 )
  ( 1 \, \& \, 0 )( 1 \, \& \, 0 ) ==$ \\
  $== 0\text{b}0000 == 0_{ \text{d} }$,
  so~$3_{ \text{d} } \, \& \, 12_{ \text{d} } == 0_{ \text{d} }$.

  $1_{ \text{d} } == 0\text{b}0001$, $3_{ \text{d} } == 0\text{b}0011$. \\
  $1_{ \text{d} } \, \& \, 3_{ \text{d} } == 0\text{b}0001 \, \& \,
  0\text{b}0011 == 0\text{b}( 0 \, \& \, 0 )( 0 \, \& \, 0 )
  ( 0 \, \& \, 1 )( 1 \, \& \, 1 ) ==$ \\
  $== 0\text{b}0001 == 1_{ \text{d} }$,
  so~$1_{ \text{d} } \, \& \, 3_{ \text{d} } == 1_{ \text{d} }$.

  $1_{ \text{d} } == 0\text{b}0001$, $2_{ \text{d} } == 0\text{b}0010$. \\
  $1_{ \text{d} } \, \& \, 2_{ \text{d} } == 0\text{b}0001 \, \& \,
  0\text{b}0010 == 0\text{b}( 0 \, \& \, 0 )( 0 \, \& \, 0 )
  ( 0 \, \& \, 1 )( 1 \, \& \, 0 ) ==$ \\
  $== 0\text{b}0000 == 0_{ \text{d} }$,
  so~$1_{ \text{d} } \, \& \, 2_{ \text{d} } == 0_{ \text{d} }$.

\end{frame}
% ##################





% ##################
\begin{frame}
  \frametitle{Bitwise alternative}

  $1 \, | \, 1 == 1$, $1 \, | \, 0 == 1$, $0 \, | \, 1 == 0$,
  $0 \, | \, 0 == 0$. \\
  $\text{true} \, | \, \text{true} == \text{true}$,
  $\text{true} \, | \, \text{false} == \text{true}$,
  $\text{false} \, | \, \text{true} == \text{true}$, \\
  $\text{false} \, | \, \text{false} == \text{false}$.

  $7_{ \text{d} } \, | \, 14_{ \text{d} } == 0\text{b}0111 \, | \,
  0\text{b}1110 ==
  0\text{b}( 0 \, | \, 1 )( 1 \, | \, 1 )( 1 \, | \, 1 )
  ( 1 \, | \, 0 ) ==$ \\
  $== 0\text{b}1111 == 15_{ \text{d} }$,
  so~$7_{ \text{d} } \, \& \, 14_{ \text{d} } == 15_{ \text{d} }$.

  $5_{ \text{d} } == 0\text{b}0101$, $9_{ \text{d} } == 0\text{b}1001$. \\
  $5_{ \text{d} } \, | \, 9_{ \text{d} } == 0\text{b}0101 \, | \,
  0\text{b}1001 == 0\text{b}( 0 \, | \, 1 )( 1 \, | \, 0 )
  ( 0 \, | \, 0 )( 1 \, | \, 1 ) ==$ \\
  $== 0\text{b}1101 == 13_{ \text{d} }$,
  so~$5_{ \text{d} } \, | \, 9_{ \text{d} } == 13_{ \text{d} }$.

  $3_{ \text{d} } == 0\text{b}0011$, $12_{ \text{d} } == 0\text{b}1100$. \\
  $3_{ \text{d} } \, | \, 12_{ \text{d} } == 0\text{b}0011 \, | \,
  0\text{b}1100 == 0\text{b}( 0 \, | \, 1 )( 0 \, | \, 1 )
  ( 1 \, | \, 0 )( 1 \, | \, 0 ) ==$ \\
  $== 0\text{b}1111 == 0_{ \text{d} }$,
  so~$3_{ \text{d} } \, | \, 12_{ \text{d} } == 15_{ \text{d} }$.

  $1_{ \text{d} } == 0\text{b}0001$, $3_{ \text{d} } == 0\text{b}0011$. \\
  $1_{ \text{d} } \, | \, 3_{ \text{d} } == 0\text{b}0001 \, | \,
  0\text{b}0011 == 0\text{b}( 0 \, | \, 0 )( 0 \, | \, 0 )
  ( 0 \, | \, 1 )( 1 \, | \, 1 ) ==$ \\
  $== 0\text{b}0011 == 3_{ \text{d} }$,
  so~$1_{ \text{d} } \, | \, 3_{ \text{d} } == 3_{ \text{d} }$.

\end{frame}
% ##################





% ##################
\begin{frame}
  \frametitle{Bitwise alternative}



  $1_{ \text{d} } == 0\text{b}0001$, $2_{ \text{d} } == 0\text{b}0010$. \\
  $1_{ \text{d} } \, | \, 2_{ \text{d} } == 0\text{b}0001 \, | \,
  0\text{b}0010 == 0\text{b}( 0 \, | \, 0 )( 0 \, | \, 0 )
  ( 0 \, | \, 1 )( 1 \, | \, 0 ) ==$ \\
  $== 0\text{b}0011 == 3_{ \text{d} }$,
  so~$1_{ \text{d} } \, | \, 2_{ \text{d} } == 3_{ \text{d} }$.

\end{frame}
% ##################





% ##################
\begin{frame}
  \frametitle{Bitwise XOR}


  $1 \; \hat{} \; 1 == 0$, $1 \; \hat{} \; 0 == 1$,
  $0 \; \hat{} \; 1 == 1$, $0 \; \hat{} \; 0 == 0$. \\
  $\text{true} \; \hat{} \; \text{true} == \text{false}$,
  $\text{true} \; \hat{} \; \text{false} == \text{true}$,
  $\text{false} \; \hat{} \; \text{true} == \text{true}$, \\
  $\text{false} \; \hat{} \; \text{false} == \text{false}$.

  $7_{ \text{d} } \; \hat{} \; 14_{ \text{d} } == 0\text{b}0111 \; \hat{} \;
  0\text{b}1110 ==
  0\text{b}( 0 \; \hat{} \; 1 )( 1 \; \hat{} \; 1 )( 1 \; \hat{} \; 1 )
  ( 1 \; \hat{} \; 0 ) ==$ \\
  $== 0\text{b}1001 == 13_{ \text{d} }$,
  so~$7_{ \text{d} } \; \hat{} \; 14_{ \text{d} } == 13_{ \text{d} }$.

  $5_{ \text{d} } == 0\text{b}0101$, $9_{ \text{d} } == 0\text{b}1001$. \\
  $5_{ \text{d} } \; \hat{} \; 9_{ \text{d} } == 0\text{b}0101 \; \hat{} \;
  0\text{b}1001 == 0\text{b}( 0 \; \hat{} \; 1 )( 1 \; \hat{} \; 0 )
  ( 0 \; \hat{} \; 0 )( 1 \; \hat{} \; 1 ) ==$ \\
  $== 0\text{b}1100 == 12_{ \text{d} }$,
  so~$5_{ \text{d} } \; \hat{} \; 9_{ \text{d} } == 12_{ \text{d} }$.

  $3_{ \text{d} } == 0\text{b}0011$, $12_{ \text{d} } == 0\text{b}1100$. \\
  $3_{ \text{d} } \; \hat{} \; 12_{ \text{d} } == 0\text{b}0011 \; \hat{} \;
  0\text{b}1100 == 0\text{b}( 0 \; \hat{} \; 1 )( 0 \; \hat{} \; 1 )
  ( 1 \; \hat{} \; 0 )( 1 \; \hat{} \; 0 ) ==$ \\
  $== 0\text{b}1111 == 15_{ \text{d} }$,
  so~$3_{ \text{d} } \; \hat{} \; 12_{ \text{d} } == 15_{ \text{d} }$.

  $1_{ \text{d} } == 0\text{b}0001$, $3_{ \text{d} } == 0\text{b}0011$. \\
  $1_{ \text{d} } \; \hat{} \; 3_{ \text{d} } == 0\text{b}0001 \; \hat{} \;
  0\text{b}0011 == 0\text{b}( 0 \; \hat{} \; 0 )( 0 \; \hat{} \; 0 )
  ( 0 \; \hat{} \; 1 )( 1 \; \hat{} \; 1 ) ==$ \\
  $== 0\text{b}0010 == 2_{ \text{d} }$,
  so~$1_{ \text{d} } \; \hat{} \; 3_{ \text{d} } == 2_{ \text{d} }$.


\end{frame}
% ##################





% ##################
\begin{frame}
  \frametitle{Bitwise XOR}


  $1_{ \text{d} } == 0\text{b}0001$, $2_{ \text{d} } == 0\text{b}0010$. \\
  $1_{ \text{d} } \; \hat{} \; 2_{ \text{d} } == 0\text{b}0001 \; \hat{} \;
  0\text{b}0010 == 0\text{b}( 0 \; \hat{} \; 0 )( 0 \; \hat{} \; 0 )
  ( 0 \; \hat{} \; 1 )( 1 \; \hat{} \; 0 ) ==$ \\
  $== 0\text{b}0011 == 3_{ \text{d} }$,
  so~$1_{ \text{d} } \; \hat{} \; 2_{ \text{d} } == 3_{ \text{d} }$.

\end{frame}
% ##################





% % ##################
% \begin{frame}
%   \frametitle{????}



% \end{frame}
% % ##################





% % ##################
% \begin{frame}
%   \frametitle{????}




% \end{frame}
% % ##################





% ##################
\begin{frame}
  \frametitle{Decimal and binary form of numbers}



  \begin{center}

    \begin{tabular}{|c|c|}
      \hline
      Decimal & Binary \\
      \hline
      $0_{ \text{d} }$ & $0\text{b}0000$ \\
      \hline
    \end{tabular}

  \end{center}

\end{frame}
% ##################










% % ######################################
% \section{}
% % ######################################



% % ##################
% \begin{frame}
%   \frametitle{????}




% \end{frame}
% % ##################





% % ##################
% \begin{frame}
%   \frametitle{}




% \end{frame}
% % ##################










% % ######################################
% \section{}
% % ######################################


% % ##################
% \begin{frame}
%   \frametitle{????}




% \end{frame}
% % ##################





% % ##################
% \begin{frame}
%   \frametitle{????}




% \end{frame}
% % ##################










% % ######################################
% \section{}
% % ######################################



% % ##################
% \begin{frame}
%   \frametitle{????}




% \end{frame}
% % ##################





% % ##################
% \begin{frame}
%   \frametitle{???}




% \end{frame}
% % ##################





% % ##################
% \begin{frame}
%   \frametitle{????}




% \end{frame}
% % ##################





% % ##################
% \begin{frame}
%   \frametitle{????}




% \end{frame}
% % ##################





% % ##################
% \begin{frame}
%   \frametitle{???}



% \end{frame}
% % ##################





% % ##################
% \begin{frame}
%   \frametitle{???}




% \end{frame}
% % ##################





% % ##################
% \begin{frame}
%   \frametitle{???}




% \end{frame}
% % ##################





% % ##################
% \begin{frame}
%   \frametitle{???}




% \end{frame}
% % ##################





% % ##################
% \begin{frame}
%   \frametitle{???}




% \end{frame}
% % ##################





% % ##################
% \begin{frame}
%   \frametitle{????}



% \end{frame}
% % ##################





% % ##################
% \begin{frame}
%   \frametitle{????}




% \end{frame}
% % ##################





% % ##################
% \begin{frame}
%   \frametitle{????}




% \end{frame}
% % ##################





% % ##################
% \begin{frame}
%   \frametitle{????}




% \end{frame}
% % ##################





% % ##################
% \begin{frame}
%   \frametitle{????}



% \end{frame}
% % ##################





% % ##################
% \begin{frame}
%   \frametitle{???}




% \end{frame}
% % ##################










% ######################################
\section{Słownik Java-polski}
% ######################################


% ##################
\begin{frame}
  \frametitle{Słownik Java-polski}



\end{frame}
% ##################





% % ##################
% \begin{frame}
%   \frametitle{????}




% \end{frame}
% % ##################





% % ##################
% \begin{frame}
%   \frametitle{????}




% \end{frame}
% % ##################





% % ##################
% \begin{frame}
%   \frametitle{????}




% \end{frame}
% % ##################





% % ##################
% \begin{frame}
%   \frametitle{????}




% \end{frame}
% % ##################





% % ##################
% \begin{frame}
%   \frametitle{???}




% \end{frame}
% % ##################





% % ##################
% \begin{frame}
%   \frametitle{????}




% \end{frame}
% % ##################





% % ##################
% \begin{frame}
%   \frametitle{????}




% \end{frame}
% % ##################










% % ######################################
% \section{???}
% % ######################################


% % ##################
% \begin{frame}
%   \frametitle{????}




% \end{frame}
% % ##################





% % ##################
% \begin{frame}
%   \frametitle{???}




% \end{frame}
% % ##################





% % ##################
% \begin{frame}
%   \frametitle{???}




% \end{frame}
% % ##################










% % ######################################
% \section{????}
% % ######################################


% % ##################
% \begin{frame}
%   \frametitle{????}




% \end{frame}
% % ##################





% % ##################
% \begin{frame}
%   \frametitle{????}




% \end{frame}
% % ##################





% % ##################
% \begin{frame}
%   \frametitle{????}




% \end{frame}
% % ##################










% % ######################################
% \section{}
% % ######################################


% % ##################
% \begin{frame}
%   \frametitle{????}




% \end{frame}
% % ##################










% ######################################
\appendix
% ######################################





% ######################################
\EndingSlide{Thank you. Any questions?}
% ######################################










% ####################################################################
% ####################################################################
% Bibliography

\printbibliography





% ####################################################################
% End of the document

\end{document}
