% ------------------------------------------------------------------------------------------------------------------
% Basic configuration of Beamera class and Jagiellonian theme
% ------------------------------------------------------------------------------------------------------------------
\RequirePackage[l2tabu, orthodox]{nag}



\ifx\PresentationStyle\notset
  \def\PresentationStyle{dark}
\fi



% Options: t -- align text to the top of the frame
\documentclass[10pt,t]{beamer}
\mode<presentation>
\usetheme[style=\PresentationStyle,frametitlecolorstyle=general,titleframebackgroundcolorstyle=dark]{jagiellonian}





% ------------------------------------------------------------------------------------
% Procesing configuration files of Jagiellonian theme located in
% the directory "preambule"
% ------------------------------------------------------------------------------------
% Configuration for polish language
% Need description
\usepackage[polish]{babel}
% Need description
\usepackage[MeX]{polski}



% ------------------------------
% Better support of polish chars in technical parts of PDF
% ------------------------------
\hypersetup{pdfencoding=auto,psdextra}

% Package "textpos" give as enviroment "textblock" which is very usefull in
% arranging text on slides.

% This is standard configuration of "textpos"
\usepackage[overlay,absolute]{textpos}

% If you need to see bounds of "textblock's" comment line above and uncomment
% one below.

% Caution! When showboxes option is on significant ammunt of space is add
% to the top of textblock and as such, everyting put in them gone down.
% We need to check how to remove this bug.

% \usepackage[showboxes,overlay,absolute]{textpos}



% Setting scale length for package "textpos"
\setlength{\TPHorizModule}{10mm}
\setlength{\TPVertModule}{\TPHorizModule}


% ---------------------------------------
% Packages written for lectures "Geometria 3D dla twórców gier wideo"
% ---------------------------------------
% \usepackage{./Geometry3DPackages/Geometry3D}
% \usepackage{./Geometry3DPackages/Geometry3DIndices}
% \usepackage{./Geometry3DPackages/Geometry3DTikZStyle}
% \usepackage{./ProgramowanieSymulacjiFizykiPaczki/ProgramowanieSymulacjiFizykiTikZStyle}
% \usepackage{./Geometry3DPackages/mathcommands}


% ---------------------------------------
% TikZ
% ---------------------------------------
% Importing TikZ libraries
\usetikzlibrary{arrows.meta}
\usetikzlibrary{positioning}





% % Configuration package "bm" that need for making bold symbols
% \newcommand{\bmmax}{0}
% \newcommand{\hmmax}{0}
% \usepackage{bm}




% ---------------------------------------
% Packages for scientific texts
% ---------------------------------------
% \let\lll\undefined  % Sometimes you must use this line to allow
% "amsmath" package to works with packages with packages for polish
% languge imported
% /preambul/LanguageSettings/JagiellonianPolishLanguageSettings.tex.
% This comments (probably) removes polish letter Ł.
\usepackage{amsmath}  % Packages from American Mathematical Society (AMS)
\usepackage{amssymb}
\usepackage{amscd}
\usepackage{amsthm}
\usepackage{siunitx}  % Package for typsetting SI units.
\usepackage{upgreek}  % Better looking greek letters.
% Example of using upgreek: pi = \uppi


\usepackage{calrsfs}  % Zmienia czcionkę kaligraficzną w \mathcal
% na ładniejszą. Może w innych miejscach robi to samo, ale o tym nic
% nie wiem.










% ---------------------------------------
% Packages written for lectures "Geometria 3D dla twórców gier wideo"
% ---------------------------------------
% \usepackage{./ProgramowanieSymulacjiFizykiPaczki/ProgramowanieSymulacjiFizyki}
% \usepackage{./ProgramowanieSymulacjiFizykiPaczki/ProgramowanieSymulacjiFizykiIndeksy}
% \usepackage{./ProgramowanieSymulacjiFizykiPaczki/ProgramowanieSymulacjiFizykiTikZStyle}





% !!!!!!!!!!!!!!!!!!!!!!!!!!!!!!
% !!!!!!!!!!!!!!!!!!!!!!!!!!!!!!
% EVIL STUFF
\if\JUlogotitle1
\edef\LogoJUPath{LogoJU_\JUlogoLang/LogoJU_\JUlogoShape_\JUlogoColor.pdf}
\titlegraphic{\hfill\includegraphics[scale=0.22]
{./JagiellonianPictures/\LogoJUPath}}
\fi
% ---------------------------------------
% Commands for handling colors
% ---------------------------------------


% Command for setting normal text color for some text in math modestyle
% Text color depend on used style of Jagiellonian

% Beamer version of command
\newcommand{\TextWithNormalTextColor}[1]{%
  {\color{jNormalTextFGColor}
    \setbeamercolor{math text}{fg=jNormalTextFGColor} {#1}}
}

% Article and similar classes version of command
% \newcommand{\TextWithNormalTextColor}[1]{%
%   {\color{jNormalTextsFGColor} {#1}}
% }



% Beamer version of command
\newcommand{\NormalTextInMathMode}[1]{%
  {\color{jNormalTextFGColor}
    \setbeamercolor{math text}{fg=jNormalTextFGColor} \text{#1}}
}


% Article and similar classes version of command
% \newcommand{\NormalTextInMathMode}[1]{%
%   {\color{jNormalTextsFGColor} \text{#1}}
% }




% Command that sets color of some mathematical text to the same color
% that has normal text in header (?)

% Beamer version of the command
\newcommand{\MathTextFrametitleFGColor}[1]{%
  {\color{jFrametitleFGColor}
    \setbeamercolor{math text}{fg=jFrametitleFGColor} #1}
}

% Article and similar classes version of the command
% \newcommand{\MathTextWhiteColor}[1]{{\color{jFrametitleFGColor} #1}}





% Command for setting color of alert text for some text in math modestyle

% Beamer version of the command
\newcommand{\MathTextAlertColor}[1]{%
  {\color{jOrange} \setbeamercolor{math text}{fg=jOrange} #1}
}

% Article and similar classes version of the command
% \newcommand{\MathTextAlertColor}[1]{{\color{jOrange} #1}}





% Command that allow you to sets chosen color as the color of some text into
% math mode. Due to some nuances in the way that Beamer handle colors
% it not work in all cases. We hope that in the future we will improve it.

% Beamer version of the command
\newcommand{\SetMathTextsColor}[2]{%
  {\color{#1} \setbeamercolor{math text}{fg=#1} #2}
}


% Article and similar classes version of the command
% \newcommand{\SetMathTextColor}[2]{{\color{#1} #2}}










% ---------------------------------------
% Commands for setting background pictures for some slides
% ---------------------------------------
\newcommand{\TitleBackgroundPicture}
{./PresentationPictures/CommonPictures/Cute_dragon_BG_dark.png}
\newcommand{\SectionBackgroundPicture}
{./PresentationPictures/CommonPictures/Cute_dragon_small_BG_light.png}



\newcommand{\TitleSlideWithPicture}{
  \begingroup

  \usebackgroundtemplate{ % \hspace*{-11.5em}
    \includegraphics[height=\paperheight]{\TitleBackgroundPicture}}

  \maketitle

  \endgroup
}





\newcommand{\SectionSlideWithPicture}[1]{%
  \begingroup

  \usebackgroundtemplate{ % \hspace*{-11.5em}
    \includegraphics[height=\paperheight]{\SectionBackgroundPicture}}

  \setbeamercolor{titlelike}{fg=normal text.fg}

  \section{#1}

  \endgroup
}





\newcommand{\EndingSlide}[1]{%
  \begin{frame}[standout]

    \begingroup

    \color{jFrametitleFGColor}

    #1

    \endgroup

  \end{frame}
}










% ------------------------------------------------------
% BibLaTeX
% ------------------------------------------------------
% Package biblatex, with biber as its backend, allow us to handle
% bibliography entries that use Unicode symbols outside ASCII.
\usepackage[
language=polish,
backend=biber,
style=alphabetic,
url=false,
eprint=true,
]{biblatex}

\addbibresource{Podstawy-informatyki-ETC-Bibliography.bib}





% ------------------------------------------------------
% Importing packages, libraries and setting their configuration
% ------------------------------------------------------




% ------------------------------------------------------
% Local packages
% ------------------------------------------------------
% Local configuration of this particular presentation
\usepackage{./Local-packages/local-settings}

% Package containing various command useful for working with a text
\usepackage{./Local-packages/general-commands}

% Styles for arrows
% \usepackage{./Local-packages/PGF-TikZ-Arrows-styles}

% Stylef for drawing diagrams
% \usepackage{./Local-packages/PGF-TikZ-Diagram-styles}











% ------------------------------------------------------------------------------------------------------------------
\title{Systemy pozycyjne i~reprezentacja liczb w~komputerze}

\author{Kamil Ziemian \\
  \email}


% \date{}
% ------------------------------------------------------------------------------------------------------------------










% ####################################################################
% Beginning of the document
\begin{document}
% ####################################################################





% ######################################
% Text is adjusted to the left and words are broken at the end of the line.
% Number of chars: 62k+, 73k+, 43k+, 52k+, 69k+,
\RaggedRight
% ######################################





% ######################################
\maketitle
% ######################################





% ##################
\begin{frame}
  \frametitle{Spis treści}


  \tableofcontents

\end{frame}
% ##################


% ######################################
\section{Systemy pozycyjny do zapisu liczb}
% ######################################



% ##################
\begin{frame}
  \frametitle{System pozycyjny}


  Proszę powiedzieć, czy to jest na tyle dobrze znane, że możemy to pominąć
  czy nie?

  Żeby stosować systemu pozycyjny musimy ustalić kierunek w~którym czytamy
  i~piszemy. W~naszym kręgu kulturowym piszemy poziom i~od lewej do prawej,
  ludy semickie często piszą poziomo, ale od prawej do lewej. Z~kolei
  w~wschodniej Euroazji często pisze~się pionowo i~od góry do dołu.
  Nie wnikając w~te zawiłości, przedstawimy system pozycyjny który
  dostosowany jest do pisania poziomo od lewej do prawej. Zresztą,
  w~komputerach zwykle nie używa~się żadnego innego.

  Zaczniemy od systemu pozycyjnego służącego do~zapisu liczb
  naturalnych: $0, 1, 2, 3, 4, \ldots$ Każdy system pozycyjny musi posiadać
  wyróżnioną liczbę naturalną, różną od~zera i~jeden, którą nazywamy
  \textbf{podstawą systemu}. Historycznie ludzie korzystali z~systemów
  o~różnej podstawie, ale obecnie normalni ludzie chyba wszędzie używają
  systemu o~podstawie dziesięć.

\end{frame}
% ##################





% ##################
\begin{frame}
  \frametitle{System pozycyjny}


  Wybór jako podstawy systemu liczby dziesięć wynika pewnie z~tego,
  że~naturalnie człowiek ma dziesięć palców u~rąk.

  \alert{Ciekawostka.} Angielskie słowo oznaczające cyfrę to \textit{digit}.
  Pochodzi ono stąd, że~po łacinie \textit{digitus} znaczy tyle co „palec”.
  Czy więc słowo „digitalizacja” powinno~się tłumaczyć jako „upalcowienie”?
  ;) Polskie słowo „cyfra” ma pochodzić od arabskiego słowa wymawianego jako
  „sifr”, które oznacza zero. Ale nie jest ekspertem, gdy chodzi
  o~etymologię.

  Oznaczmy teraz liczbę dziesięć symbolem $A$. Wówczas możemy odczytać
  wartość liczby $1234$ jako: \\
  $\displaystyle
  1234 = 4 \cdot A^{ 0 } + 3 \cdot A^{ 1 } + 2 \cdot A^{ 2 } + 1 \cdot A^{ 3 }.$

  Podstawiając $A = 10$ łatwo sprawdzamy, że~jest to prawda. \\
  $\displaystyle
  4 \cdot A^{ 0 } + 3 \cdot A^{ 1 } + 2 \cdot A^{ 2 } + 1 \cdot A^{ 3 } =
  4 \cdot 10^{ 0 } + 3 \cdot 10^{ 1 } + 2 \cdot 10^{ 2 } + 1 \cdot 10^{ 3 } =$ \\
  $\displaystyle
  = 4 \cdot 1 + 3 \cdot 10 + 2 \cdot 100 + 1 \cdot 1 000 = 1234$

\end{frame}
% ##################





% ##################
\begin{frame}
  \frametitle{Istota systemu pozycyjnego}


  Na tym przykładzie mogliśmy zobaczyć istotę systemu pozycyjnego. Dany
  system pozycyjny jest zdefiniowany przez wybraną liczbę naturalną $b$
  (po angielsku podstawa systemu to \textit{system base}) i~$b$ symboli
  reprezentujących liczby od~$0$ do~$b - 1$. Dla systemu dziesiętnego te
  symbole to doskonale nam wszystkim znane $0$, $1$, $2$, $3$, $4$, $5$,
  $6$, $7$, $8$, $9$.

  Liczbę $n$ będziemy nazywali \textbf{podstawą systemu}, zaś zbiór $b$
  symboli $c_{ 1 }, c_{ 2 }, c_{ 3 }, \ldots, c_{ b }$ będziemy nazywali
  \textbf{cyframi}
  danego systemu pozycyjnego, które reprezentują liczby $0, 1, 2, \ldots, b - 1$.

  Dowolną liczbę naturalną możemy teraz w~systemie o~podstawie $b$ możemy
  zapisać jako ciąg cyfr. By podkreślić, że~dana liczba jest zapisany
  w~systemie o~podstawie~$n$ często używa~się następującej notacji. \\
  $\displaystyle
  ( a_{ k - 1 } a_{ k - 2 } \ldots a_{ 1 } a_{ 0 } )_{ b },$ \\
  gdzie~$k$ to pewna liczba naturalna większa lub równa~$1$: $k \geq 1$.

\end{frame}
% ##################





% ##################
\begin{frame}
  \frametitle{Istota systemu pozycyjnego}


  Wartość liczby $( a_{ k - 1 } a_{ k - 2 } \ldots a_{ 1 } a_{ 0 } )_{ b }$ zapisanej
  w~danym systemie pozycyjnym zawsze odczytujemy w~ten sam sposób: \\
  $\displaystyle
  ( a_{ k - 1 } a_{ k - 2 } \ldots a_{ 1 } a_{ 0 } )_{ b } =
  a_{ 0 } \cdot b^{ 0 } + a_{ 1 } \cdot b^{ 1 } + a_{ 2 } \cdot b^{ 2 } + \ldots +
  a_{ k - 2 } \cdot b^{ k - 2 } + a_{ k - 1 } \cdot b^{ k - 1 }.$

  Proszę zwrócić uwagę, że~jest to w~100\% zgodne z~tym, co widzieliśmy
  wcześniej: $1234 = 4 + 3 \cdot 10 + 2 \cdot 100 + 1 \cdot 1000$.

  System o~podstawie dziesięć nie jest w~żaden sposób wyróżniony przez
  matematykę, ale~ponieważ jest to jedyny system jaki normalny człowiek
  używam dzisiaj (Pytanie filozoficzne: czy informatyk to normalny
  człowiek?), jest dla każdego z~nas najbardziej naturalny ze
  wszystkich. Dlatego jeśli nie jest podane w~jakim systemie zapisana
  jest dana liczba, ani nie wynika to w~żaden sposób z~kontekstu, to
  proszę przyjąć, iż~jest zapisana w~systemie dziesiętnym.

\end{frame}
% ##################





% ##################
\begin{frame}
  \frametitle{Różne systemy pozycyjne}


  Proszę zwrócić uwagę, że~w~systemie dziesiętnym nie ma \textit{cyfry}
  na oznaczenie liczby~$10$. Oznaczamy ją za pomocą ciągu dwóch cyfr: $1$
  i~$0$. Tak samo w~skład systemu o~podstawie~$2$, nie wchodzi cyfra „$2$”.
  Liczbę dwa zapisujemy w~nim jako: \\
  $(10)_{ 2 } = 2.$ \\
  Ogólnie w~systemie o~podstawie $n$, samą liczbę $n$~zawsze zapisujemy jako
  $(10)_{ n }$. Nie powinno być trudne zrozumienie czemu tak jest.

  % Jak powszechnie wiadomo, informatyka bazuje na systemie dwójkowym
  % o~podstawie~$2$. Dlaczego akurat~$2$? Wydaje~się, że~przyczyna jest
  % czysto technologiczna. Doskonałe zrozumienie systemów o~podstawach
  % różnych od~$10$, w~tym tego o~podstawie~$2$, znajdujemy najpóźniej
  % w~pismach Gottfrieda Wilhelma Leibniza, który żył w~latach
  % $1646\text{--}1716$. W~czasach gdy zaczęto budować współczesne komputery
  % po $1900$ roku, nie była więc to żadna nowość.

\end{frame}
% ##################





% ##################
\begin{frame}
  \frametitle{Różne systemy pozycyjne}


  % Proszę zwrócić uwagę, że~w~systemie dziesiętnym nie ma \textit{cyfry}
  % na oznaczenie liczby~$10$. Oznaczamy ją za pomocą ciągu dwóch cyfr: $1$
  % i~$0$. Tak samo w~skład systemu o~podstawie~$2$, nie wchodzi cyfra „$2$”.
  % Liczbę dwa zapisujemy w~nim jako: \\
  % $(10)_{ 2 } = 2.$ \\
  % Ogólnie w~systemie o~podstawie $n$, samą liczbę $n$~zawsze zapisujemy jako
  % $(10)_{ n }$. Nie powinno być trudne zrozumienie czemu tak jest.

  Jak powszechnie wiadomo, informatyka bazuje na systemie dwójkowym
  o~podstawie~$2$. Dlaczego akurat~$2$? Wydaje~się, że~przyczyna jest
  czysto technologiczna. Doskonałe zrozumienie systemów o~podstawach
  różnych od~$10$, w~tym tego o~podstawie~$2$, znajdujemy najpóźniej
  w~pismach Gottfrieda Wilhelma Leibniza, który żył w~latach
  $1646\text{--}1716$. W~czasach gdy zaczęto budować współczesne komputery
  po $1900$ roku, nie była więc to żadna nowość.

  Dla pierwszych komputerów rozważano systemy pozycyjne o~bardzo różnych
  podstawach, takich jak $10$ czy~$5$. Po rozważeniu takich problemów jak
  to, jaka produkcja konkretnych elektronicznych części (komputer
  \alert{jest} urządzeniem elektronicznym), stabilność urządzenia
  obsługującego liczby o~danej podstawie, prądu będzie zużywać, etc.,
  stwierdzona, że~system o~podstawie~$2$, powinien być optymalny. I~tak już
  zostało.

\end{frame}
% ##################





% ##################
\begin{frame}
  \frametitle{Różne systemy pozycyjne}


  Tak samo jak system o~podstawie~$10$, tak ten o~podstawie~$2$, nie
  jest niczym wyróżniony przez matematykę. Kiedy jednak za punkt odniesienia
  weźmiemy system o~podstawie $2$, pewne inne systemy stają~się
  naturalnie wyróżnione. W~informatyce dwa systemy pochodne w~stosunku do
  tego systemu pojawiają~się szczególnie często: system ósemkowy
  o~podstawie~$8 = 2^{ 3 }$ i~szesnastkowy o~podstawie~$16 = 2^{ 4 }$.

\end{frame}
% ##################





% ##################
\begin{frame}
  \frametitle{Terminologia}


  Analogicznie, dla systemu o~podstawie $10$ wyróżnione są te o podstawie
  $100$, $1000$, etc. Proszę zwrócić uwagę na zachowanie języka polskiego.
  Liczba $99$ to „dziewięćdziesiąt dziewięć”, ale $100$ to „sto”,
  $999$ to „dziewięćset dziewięćdziesiąt dziewięć”, a~$1000$ to „tysiąc”.
  Natomiast język francuski nosi mocne ślady systemu o~podstawie $12$, które
  jakoś zachowały~się jeszcze jakoś w~języku polskim, gdzie $12$ to „tuzin”,
  a~$60 = 5 \cdot 12$ to „kopa”. Po francusku $71$ to \textit{soixante et onze},
  dosłownie „$60$ i~$11$”.

  Łacińskie słowo \textit{binarius} oznacza „złożony z~dwóch części”, stąd
  w~języku angielskim określamy system dwójkowy jako \textit{binary system},
  a~po polsku istnieje zwrot \textit{system binarny}. Co ważniejsze,
  pochodzi od~tego stosowany często w~informatyce zapis liczby w~systemie
  dwójkowym nie jako $(111)_{ 2 }$, lecz jako $0\text{b}111$. Łatwo
  sprawdzić, że \\
  $\displaystyle
  0\text{b}111 = 1 \cdot 2^{ 0 } + 1 \cdot 2^{ 1 } + 1 \cdot 2^{ 2 } = 7.$

\end{frame}
% ##################





% ##################
\begin{frame}
  \frametitle{Terminologia}


  System o~podstawie $8$ po angielsku to \textit{octal system}, od
  greckiego słowa na~liczbę $8$, czyli \textit{okto} (przybliżony zapis
  fonetyczny). Stąd w~informatyce często liczby w~tym systemie są
  zapisywane nie $(321)_{ 8 }$, lecz $0\text{o}321$. Symbol $\text{o}$ to
  mała litera „o”. Łatwo sprawdzić, że \\
  $\displaystyle
  0\text{o}321 = 1 \cdot 8^{ 0 } + 2 \cdot 8^{ 1 } + 3 \cdot 8^{ 2 } =
  1 + 2 \cdot 8 + 3 \cdot 64 = 1 + 16 + 192 = 209.$

  Po angielsku system szesnastkowy to \textit{hexadecimal system}, od
  greckiego słowa oznaczającego $16$: \textit{hekkaideka} (przybliżony
  zapis fonetyczny). Dlatego również w~polskim możne~się spotkać
  ze słowem „heksadekacymalny”. Od słowa \textit{heXadecimal} pochodzi też
  zwyczaj zapisywania liczb w~tym systemie nie jako $(321)_{ 16 }$,
  ale~$0\text{x}321$. Łatwo sprawdzić, że \\
  $0\text{x}321 = 1 + 2 \cdot 16 + 3 \cdot 16^{ 2 } = 1 + 32 + 3 \cdot 256 =
  1 + 32 + 768 = 801.$

\end{frame}
% ##################





% ##################
\begin{frame}
  \frametitle{?????}


  Ponieważ $16$ jest większe od $10$ (wielkiego odkrycie nie poczyniliśmy),
  więc potrzebujemy wprowadzić sześć nowych symboli, by móc zapisać
  wszystkie $16$ cyfry tego systemu. Standardowy wybór polega na przyjęciu
  początkowych liter alfabetu łacińskiego: $A = 10$, $B = 11$,
  $C = 12$, $D = 13$, $E = 14$ i~$F = 15$. Nie będę ukrywał,
  że~zapamiętanie co te nowe cyfry znaczą, może być dość trudne.

  Zgody do tego, czy to mają być duże litery, czy małe, czy jedne i~drugie,
  nie ma. Ja będę używał konwencji, w~której stosujemy duże litery
  łacińskie.

  Dla ćwiczeń, spróbujmy wyliczyć w~systemie dziesiętnym wartość liczby
  $0\text{xBAD}$.

\end{frame}
% ##################





% ##################
\begin{frame}
  \frametitle{Największa liczba do zapisania\ldots}


  Korzystając z~wzoru na sumę szeregu geometrycznego, łatwo pokazać,
  że~największa liczba naturalna jaką możemy zapisać w~systemie o~podstawie
  $b$ na za~pomocą $n$ cyfr wynosi: \\
  $\text{maxNumber}( b, n ) = b^{ \HorSpaceTwo n } - 1.$

  Mają dwie cyfry w~systemie dziesiętnym największą liczbą jaką możemy
  zapisać jest $10^{ 2 } - 1 = 99$. W~systemie binarnym
  $2^{ 2 } - 1 = 3 = 0\text{b}11$, w~systemie ósemkowym $8^{ 2 } - 1 =
  63 = 0\text{o}77$, a~w~szesnastkowym $16^{ 2 } - 1 = 255 = 0\text{xFF}$.

  \alert{Pytanie.} Czy mamy to przeliczyć jawnie?

\end{frame}
% ##################





% % ##################
% \jagiellonianendslide{Czy są jakieś pytania do tej części?}
% % ##################










% ######################################
\section{Reprezentacja liczb w~komputerze}
% ######################################



% ##################
\begin{frame}
  \frametitle{Reprezentacja liczb}


  Zagadnienie reprezentacji liczb w~komputerze, które teraz krótko omówimy
  nie dotyczy tylko języka~C. Jest to rzecz tak fundamentalna dla
  informatyki, że~jest taka sama dla (prawie) wszystkich języków
  programowania.

  Ponieważ komputery są oparte na systemie dwójkowym, na początku
  przyjrzyjmy~się tym potęgom dwójki, które odgrywają w~informatyce
  dużą rolę.

\end{frame}
% ##################





% ##################
\begin{frame}
  \frametitle{Wybrane potęgi liczby~$2$}


  $2^{ 0 } = 1$     \\[0.1em]
  $2^{ 1 } = 2$     \\[0.1em]
  $2^{ 2 } = 4$     \\[0.1em]
  $2^{ 3 } = 8$     \\[0.1em]
  $2^{ 4 } = 16$    \\[0.1em]
  $2^{ 5 } = 32$    \\[0.1em]
  $2^{ 6 } = 64$    \\[0.1em]
  $2^{ 7 } = 128$   \\[0.1em]
  $2^{ 8 } = 256$   \\[0.1em]
  $2^{ 9 } = 512$   \\[0.1em]
  $2^{ 10 } = 1024$ \\[0.1em]
  $2^{ 11 } = 2048$ \\[0.1em]
  $2^{ 12 } = 4096$ \\[0.1em]

\end{frame}
% ##################





% ##################
\begin{frame}
  \frametitle{Wybrane potęgi liczby~$2$}


  $2^{ 15 } = 32 \, 768$ \\[0.1em]
  $2^{ 16 } = 65 \, 536$ \\[0.1em]
  $2^{ 31 } = 2 \, 147 \, 483 \, 648$ \\[0.1em]
  $2^{ 32 } = 4 \, 294 \, 967 \, 296$

  \VerSpaceThree





  $2^{ 63 } = 9 \, 223 \, 372 \, 036 \, 854 \, 775 \, 808 \approx
  9 \cdot 10^{ 18 }$ \\
  Słownie: dziewięć trylionów, dwieście dwadzieścia trzy biliardy, trzysta
  siedemdziesiąt dwa biliony, trzydzieści sześć miliardów, osiemset
  pięćdziesiąt cztery miliony, siedemset siedemdziesiąt pięć tysięcy,
  osiemset osiem. \\
  Prościej: $9$~trylionów, $223$~biliardy, $372$~biliony,
  $36$~miliardy, $854$~miliony, $775$~tysięcy, $808$.

\end{frame}
% ##################





% ##################
\begin{frame}
  \frametitle{Wybrane potęgi liczby~$2$}


  $2^{ 63 } = 9 \, 223 \, 372 \, 036 \, 854 \, 775 \, 808 \approx
  9 \cdot 10^{ 18 }$

  $2^{ 64 } = 18 \, 446 \, 744 \, 073 \, 709 \, 551 \, 616 \approx
  18 \cdot 10^{ 18 }$ \\
  Słownie: osiemnaście trylionów, czterysta czterdzieści sześć biliardów,
  siedemset czterdzieści cztery biliony, siedemdziesiąt trzy miliardy,
  siedemset dziewięć milionów, pięćset pięćdziesiąt jeden milionów,
  sześćset szesnaście. \\
  Prościej: 18 trylionów, 446 biliardów, 744 biliony, 73 miliardy,
  709 milionów, 551 tysięcy, 616.

  Jak ktoś~się tej liczby nauczy na pamięć, to zasługuje na $5$~dodatkowych
  punktów.

\end{frame}
% ##################





% ##################
\begin{frame}
  \frametitle{Reprezentacja liczb naturalnych}




\end{frame}
% ##################





% % ##################
% \begin{frame}
%   \frametitle{Czy 1972 rok był bardzo dawno temu?}




% \end{frame}
% % ##################





% % ##################
% \begin{frame}
%   \frametitle{Przesadnie pedantyczna lista featurów języka~C}



% \end{frame}
% % ##################





% % ##################
% \begin{frame}
%   \frametitle{?????}




% \end{frame}
% % ##################





% % ##################
% \begin{frame}
%   \frametitle{Zalety języka~C}




% \end{frame}
% % ##################





% % ##################
% \begin{frame}
%   \frametitle{Wady języka~C}




% \end{frame}
% % ##################





% % ##################
% \begin{frame}
%   \frametitle{Jaki jest związek między C a~C++?}




% \end{frame}
% % ##################






% % ##################
% \jagiellonianendslide{Czy są jakieś pytania do tej części?}
% % ##################










% % ######################################
% \section{Podstawowe informacje o~języku~C}
% % ######################################



% % ######################################
% \section{Dlaczego istnieją języki takie jak~C?}
% % ######################################



% % ##################
% \begin{frame}
%   \frametitle{Dlaczego w~ogóle istnieją takie języki jak~C?}



% \end{frame}
% % ##################





% % ##################
% \begin{frame}
%   \frametitle{„Hello, World!” w~asemblerze ARM 32,
%     \parencite{Low-Level-Learning-You-Can-Learn-ARM-ETC-Ver-2020}}



% \end{frame}
% % ##################




% % ##################
% \begin{frame}
%   \frametitle{„Hello, World!” w~assemblerze x86
%     \parencite{Anonymous-Hello-World-in-x86-Assembly-Language}}




% \end{frame}
% % ##################





% ##################
\begin{frame}
  \frametitle{„Hello, World!” w~języku~C}

  Początkowe liczby naturalne zapisane w~systemie dwójkowym. \\
  $0\text{b}0 = 0$, $0\text{b}1 = 1$, $0\text{b}10 = 2$, $0\text{b}11 = 3$,
  $0\text{b}100 = 4$, $0\text{b}101 = 5$, $0\text{b}110 = 6$,
  $0\text{b}111 = 7$, $0\text{b}1000 = 8$, $0\text{b}1001 = 9$,
  $0\text{b}1010 = 10$, $0\text{b}1011 = 11$, $0\text{b}1100 = 12$,
  $0\text{b}1101 = 13$, $0\text{b}1110 = 14$, $0\text{b}1111 = 15$,
  $0\text{b}10000 = 16$.

  Początkowe liczby naturalne zapisane w~systemie ósemkowym. \\
  $0\text{o}0 = 0$, $0\text{o}1 = 1$, $0\text{o}2 = 2$, $0\text{o}3 = 3$,
  $0\text{o}4 = 4$, $0\text{o}5 = 5$, $0\text{o}6 = 6$, $0\text{o}7 = 7$,
  $0\text{o}10 = 8$, $0\text{o}11 = 9$, $0\text{o}12 = 10$,
  $0\text{o}13 = 11$, $0\text{o}14 = 12$, $0\text{o}15 = 13$,
  $0\text{o}16 = 14$, $0\text{o}17 = 15$, $0\text{o}20 = 16$.

  Początkowe liczby naturalne zapisane w~systemie szesnastkowym. \\
  $0\text{x}0 = 0$, $0\text{x}1 = 1$, $0\text{x}2 = 2$, $0\text{x}3 = 3$,
  $0\text{x}4 = 4$, $0\text{x}5 = 5$, $0\text{x}6 = 6$, $0\text{x}7 = 7$,
  $0\text{x}8 = 8$, $0\text{x}9 = 9$, $0\text{xA} = 10$,
  $0\text{xB} = 11$, $0\text{xC} = 12$, $0\text{xD} = 13$,
  $0\text{xE} = 14$, $0\text{xF} = 15$, $0\text{x}10 = 16$.

\end{frame}
% ##################





% % ##################
% \begin{frame}
%   \frametitle{„Hello, World!” w~języku~Python}




% \end{frame}
% % ##################





% % ##################
% \begin{frame}
%   \frametitle{Co z~tego wynika?}




% \end{frame}
% % ##################





% % ##################
% \begin{frame}
%   \frametitle{Problemy z~asemblerem}




% \end{frame}
% % ##################





% % ##################
% \begin{frame}
%   \frametitle{Prosty schemat działania kompilatora}




% \end{frame}
% % ##################





% % ##################
% \begin{frame}
%   \frametitle{Jakie są inne języki programowania?}




% \end{frame}
% % ##################





% % ##################
% \begin{frame}
%   \frametitle{Jakie są inne języki programowania?}




% \end{frame}
% % ##################





% % ##################
% \jagiellonianendslide{Czy są jakieś pytania do tej części?}
% % ##################










% % ######################################
% \section{Systemy pozycyjny do zapisu liczb}
% % ######################################



% % ##################
% \begin{frame}
%   \frametitle{System pozycyjny}




% \end{frame}
% % ##################





% % ##################
% \begin{frame}
%   \frametitle{System pozycyjny}



% \end{frame}
% % ##################





% % ##################
% \begin{frame}
%   \frametitle{Istota systemu pozycyjnego}




% \end{frame}
% % ##################





% % ##################
% \begin{frame}
%   \frametitle{Istota systemu pozycyjnego}




% \end{frame}
% % ##################





% % ##################
% \begin{frame}
%   \frametitle{Różne systemy pozycyjne}




% \end{frame}
% % ##################





% % ##################
% \begin{frame}
%   \frametitle{Różne systemy pozycyjne}




% \end{frame}
% % ##################





% % ##################
% \begin{frame}
%   \frametitle{Terminologia}




% \end{frame}
% % ##################





% % ##################
% \begin{frame}
%   \frametitle{Terminologia}




% \end{frame}
% % ##################





% % ##################
% \begin{frame}
%   \frametitle{?????}




% \end{frame}
% % ##################





% % ##################
% \jagiellonianendslide{Czy są jakieś pytania do tej części?}
% % ##################





% % ######################################
% \section{Informatyka i~programowanie}
% % ######################################



% % ##################
% \begin{frame}
%   \frametitle{Czym jest informatyk?}


%   \textbf{Informatyka} (\textsc{cs}, ang. \textit{computer science}) to
%   nauka o~rozwiązywaniu problemów za~pomocą komputera. Niektórzy mogą
%   uważać tą definicję, że~obrazę informatyki, według mnie jest bardzo dobra.

%   Za pomocą komputera możemy próbować rozwiązać wiele różnych problemów.
%   Np.~potrzebuję przesłać komuś informację (email), potrzebuję obliczyć
%   przybliżoną wartość pewnej liczby $\pi$ lub $\gamma$ (metody numeryczne),
%   chciałbym pograć w~coś fajnego (gry wideo), etc. Powstaje jednak pytania.
%   Czym~się różni komputer od przykładowo konsoli do gier takiej jak
%   PlayStation? Przecież ona też pozwala nam rozwiązać jeden z~tych
%   problemów.

%   Komputer to \textbf{programowalne urządzenie elektroniczne},
%   w~przeciwieństwie do PlayStation, która jest urządzeniem elektroniczny,
%   ale \textit{nie} jest programowalna. Co to jednak znaczy?

% \end{frame}
% % ##################





% % ##################
% \begin{frame}
%   \frametitle{Co to znaczy „programowalny”?}


%   Każde urządzenie elektroniczne działa w~ten sposób, że~gdy płynie przez
%   nie prąd wykonuje pewne instrukcje. To jakie są instrukcje wynika
%   z~tego jak wewnątrz urządzenia „podpięte są kable”. Jeśli jesteśmy
%   w~stanie wywołać dowolną instrukcję jaką dysponuje to urządzenie, to
%   mówimy, że~jest ono \textbf{programowalne}. Jeśli mogę wywołać tylko
%   część możliwych instrukcji, tą część którą udostępnił mi producent
%   urządzenia, to wówczas takie urządzenie \textbf{nie jest programowalne}.

%   Ta definicja nie jest bardzo prepozycyjna, ale powinna dać nam wyczucie
%   o~co chodzi. Komputer daje nam dostęp do wszystkich swoich instrukcji,
%   co objawia~się choćby tym, że mogę na nim instalować dowolne
%   oprogramowanie, podczas gdy na~PlayStation mogę mieć tylko oprogramowanie
%   jakie uprzednio zainstalowała tam firma \textsc{sony}.

% \end{frame}
% % ##################





% % ##################
% \begin{frame}
%   \frametitle{Programowalne urządzenie elektroniczne}


%   Przykładowo konsolę \textsc{sony} można zhakować, uzyskując pełną
%   kontrolę nad nią. Jak to zrobić można~się choćby dowiedzieć z~nagrania
%   kanału Low Level Learning
%   \colorhref{https://www.youtube.com/watch?v=7OwdCc81zHo}
%   {\textit{The new PS4 jailbreak is sort of hilarious}}
%   \parencite{Low-Level-Learning-The-new-PS4-jailbreak-ETC-2024}.

%   Pomijając ten i~nieskończoną ilość innych niuansów, nie powinno być
%   trudne w~zrozumieniu co to znaczy, że~komputer jest programowalnym
%   urządzeniem elektronicznym.

%   \textbf{Program komputerowy} to zestaw instrukcji danego komputera, które
%   przekazujemy mu, by je wykonał. Tutaj pojawia~się potrzeba wprowadzenia
%   języków programowania. Nie będziemy podawać definicji języka
%   programowania, zamiast tego zajmiemy~się tym jak o~nich myśleć.

% \end{frame}
% % ##################





% % ##################
% \begin{frame}
%   \frametitle{Czym jest język programowania?}


%   Dwie metafory które są wygodne w~opisie tego czym jest język
%   programowania i~które ja lubię, to metafora języka właśnie i~skrzynki
%   z~narzędziami.

%   Jeśli chcę by człowiek coś zrobił, muszę mu to przekazać w~języku który
%   on rozumiem. Przykładowo ze względu na to jak zbudowany jest język
%   angielski i~język japoński, ich wzajemna nauka dla użytkowników tych
%   języków jest niezwykle trudna. Jeśli więc osobą której chcę zlecić
%   zrobienie czegoś jest Japończyk, może~się zdarzyć, iż będę musiał
%   przekazać mu informację po japońsku.

%   Oczywiście człowiek może~się nie zgodzić, wykręcać, etc. Komputer to
%   tępa maszyna więc pewnie ślepo wykona wszystko co mu zleciliśmy. Ktoś
%   może powiedzieć, że~przecież mamy \textsc{ai}, chata\textsc{gpt}, etc.
%   To wszystko są ciekawe sprawy, ale na tym przedmiocie nie mamy
%   możliwości~się nimi zająć.

% \end{frame}
% % ##################





% % ##################
% \begin{frame}
%   \frametitle{Czym jest język programowania?}


%   Żeby komputer coś zrobił musimy więc mu przekazać polecenie w~języku,
%   który rozumie. Dla mnie pierwotny językiem jest język polski, co zaś
%   jest pierwotnym językiem komputera? Na potrzeby tego kursu przyjmiemy,
%   że~pierwotnym językiem komputera jest \textbf{język assembler}.

%   Tak jak niektórzy ludzie mają jako swój pierwotny język angielski,
%   hiszpański, japoński, niemiecki, polski, etc., tak komputer również
%   posługują~się różnymi rodzajami języka assembler, takimi jak \textsc{arm}
%   czy x86/Intel. Generalnie typ assemblera jest ustalony przez firmę, która
%   wyprodukowała procesor, bowiem assembler jest zdefiniowany przez to jak są
%   „podpięte kable” w~rzeczonym procesorze.

%   I~tutaj pojawia~się dwa poważne problem.

% \end{frame}
% % ##################





% % ##################
% \begin{frame}
%   \frametitle{Assembler, jaki problem?}


%   Większość ludzi zrobi wszystko, by tylko uniknąć pracy w~assemblerze.
%   A~nawet jeśli ktoś lubi pisać w~assemblerze, to w~praktyce unika~się
%   tego jak tylko~się da, bo w~assemblerze nawet najlepsi zbyt często
%   popełniają błędy, co może prowadzić do tragicznych konsekwencji.

%   Wynika to z~tego, że~assembler to język naturalny dla \alert{komputera},
%   ale nie dla człowieka. Ponieważ zaś, nie jest to wielkie odkrycie,
%   człowiek mocno~się różni od komputer, język naturalny dla jednego z~nich
%   jest zupełnie nienaturalny dla drugiego. Dla nas assembler jest bardzo
%   nienaturalny, komputer nie potrafi zaś sobie radzić z~angielskim, czy
%   polskim.

%   Żeby nie być gołosłownym, przedstawię teraz przykład programu, który
%   jedyne co robi to wypisuje gdzieś na ekranie tekst „Hello, World!”.
%   Podany zostanie odpowiedni kod w~assemblerze x86 w~wersji Nasm, języku~C
%   i~języku Python.

% \end{frame}
% % ##################
















% % ##################
% \begin{frame}
%   \frametitle{W~jaki sposób działają~C i~Python?}


%   Wróćmy do przykładu z~Japończykiem nieznającym angielskiego. Przyjmijmy,
%   że~ja nie znam angielskiego, a~on polskiego (po kiego Japończykowi język
%   polski?). Czy nie ma szans, żeby przekazał mu polecenie? Oczywiście,
%   że~jest. Wystarczy, że~spiszę odpowiednią informację na kartce, użyję
%   translatora by przetłumaczyć ją na japoński i~mu ją dam.

%   Tak samo postępujemy z~komputerem. W~teorii języków programowania
%   \textbf{translatorem} nazywamy program który tłumaczy między dwoma
%   \alert{dowolnymi} językami programowania. Mogę więc napisać program
%   w~języku~C następnie użyć odpowiedniego translatora by przetłumaczyć
%   go na assembler, który następnie zostanie wykonany przez komputer.

%   Koniec końców każdy język programowania, czy to~C, czy Python,
%   \alert{musi} zostać przetłumaczony na odpowiedni kod assemblera,
%   by~komputer go wykonał. To jednak wykracza poza temat zajęć.

% \end{frame}
% % ##################





% % ##################
% \begin{frame}
%   \frametitle{Typy translatorów}


%   Istnieje wiele różnych rodzai translatorów. Przykładowo translatory
%   używane dla języka~C noszą nazwę \textbf{kompilatorów}, zaś te stosowane
%   dla języka Python \textbf{interpreterów}. To jednak bardzo złożona
%   dziedzina informatyki, która od mniej więcej w~roku 2000 roku wkroczył
%   w~nową, wciąż trwającą fazę rozwoju, więc tutaj ciągle~się coś zmienia.

%   Na razie wszystko co muszą Państwo wiedzieć o~kompilatorach sprowadza~się
%   do tego jak uruchomić program napisany w~języku~C, co~można ująć w~trzech
%   punktach.

% \end{frame}
% % ##################





% % ##################
% \begin{frame}
%   \frametitle{Jak używać kompilatora?}


%   Przyjmijmy, że~program napisany w~języku~C znajduje~się w~pliku
%   \texttt{source.c} (od ang. \textit{source code}, kod źródłowy), a~program
%   wynikowy nazywa~się \texttt{progOut} (od ang. \textit{program out},
%   program wyjściowy).





%   \begin{enumerate}

%   \item W~pliku \texttt{source.c} zapisujemy odpowiedni kod w~języku~C.



%   \item Przekazujemy kompilatorowi plik \texttt{source.c}, na podstawie
%     niego tworzy on program \texttt{progOut}.



%   \item Uruchamiamy program wykonywalny \texttt{progOut}.

%   \end{enumerate}

% \end{frame}
% % ##################





% % ##################
% \begin{frame}
%   \frametitle{Jak to wygląda w~praktyce?}


%   Przyjmujemy, że~jesteśmy pod systemem GNU/Linux i~wszystkie potrzebne
%   programy~są zainstalowane.




%   \begin{enumerate}

%   \item Otwieramy powłokę \textsc{bash}.



%   \item Tworzymy plik \texttt{source.c} za pomocą komendy \\
%     \texttt{\$ gedit source.c \&}



%   \item Zapisujemy odpowiedni kod programu w~pliku \texttt{source.c}.



%   \item Kompilujemy plik \texttt{source.c} przez wpisanie w~terminalu
%     polecenia \\
%     \texttt{\$ gcc -std=c99 -pedantic source.c -o progOut}



%   \item Jeśli wystąpił błąd, poprawiamy kod w~pliku \texttt{source.c}.



%   \item Uruchamiamy program \texttt{progOut} wpisując w~terminalu \\
%     \texttt{\$ ./progOut}



%   \item Jeśli \texttt{progOut} nie działa jak trzeba wracamy do punktu~5.

%   \end{enumerate}

% \end{frame}
% % ##################





% % ##################
% \begin{frame}
%   \frametitle{„Bardzo śmieszne”}


%   Teraz sobie pewnie niektórzy z~Państwa myślą: „Bardzo śmieszne. Napisał
%   listę siedmiu punktów, z~których każdy brzmi jak magiczne zaklęcia
%   z~>>Harry’ego Pottera”. Ja od trzech godzin siedzę nad punktem piątym
%   i~powoli tracę zmysły, a~kolega obok jeszcze nie ukończył pierwszego.”.

%   Tak może być, to zupełnie normalne. Dlatego ja tu jestem by Państwu pomóc
%   zrobić te pierwsze kroki, które prawie zawsze są najtrudniejsze. Dlatego
%   jeśli coś nie działa, proszę mówić, postaram~się pomóc.

% \end{frame}
% % ##################




























% % ##################
% \begin{frame}
%   \frametitle{Komentarz odnośnie diagramów}


%   \begin{textblock}{2.8}(2,1.5)

%     \begin{tikzpicture}

%       \node[diagram block] at (0,0) {Coś robi};

%     \end{tikzpicture}

%   \end{textblock}



%   \begin{textblock}{2.8}(8,1.5)

%     \begin{tikzpicture}

%       \node[diagram rectangle block] at (0,0) {Czymś jest};

%     \end{tikzpicture}

%   \end{textblock}


%   \vspace{6em}





%   Bloki diagramu które mają kształt prostokąta z~zaokrąglonymi rogami
%   (zwykle~są koloru niebieskiego) oznaczają \textbf{aktorów}, czyli
%   taki obiekt który wykonuje jakąś czynność na zadanych obiektach
%   wejściowych. Takim obiektem może być człowiek, komputer, program
%   komputerowy, fragment programu komputerowego, etc.

%   Bloki diagramu które mają kształt prostokąta z~ostrymi rogami (zwykle~są
%   koloru karmelowego), przedstawiają rzeczy które są pobierane na~wejście
%   przez aktorów lub przez nich wytwarzane.

% \end{frame}
% % ##################



% % ##################
% \begin{frame}
%   \frametitle{Komentarz odnośnie diagramów}


%   Większość ludzi zrobi wszystko, by tylko uniknąć pracy w~assemblerze.
%   Mam nadzieję, że~nie muszę tłumaczyć dlaczego.  A~nawet jeśli
%   znajdzie~się człowieka, który lubi w~nim pisać, to unika~się jak tylko
%   można tego, by kod napisany przez niego w~assemblerze wszedł w~skład
%   danego programu.

%   Jest tak dlatego, że~nawet najlepsi programiści zbyt łatwo mylą~się
%   pisząc w~assemblerze, a~pomyłki na jego poziomie są szczególnie
%   niebezpieczne. Niestety, wyjaśnienie czemu tak jest, to temat na osobny
%   wykład. W~skrócie, błędy w~assemblerze to raj dla hakerów.

%   Jak rozwiązujemy ten problem dziś? Tworzymy język taki jak~C, który
%   następnie odpowiedni program zwany \textbf{kompilatorem} tłumaczy na
%   assembler.

% \end{frame}
% % ##################

















% % ######################################
% \section{Kilka słów ogólnie o~językach programowania}
% % ######################################


% % ##################
% \begin{frame}
%   \frametitle{Ile istnieje języków programowania?}


%   Krótko mówiąc, jest ich tyle, że~chyba nikt na świecie nie jest~się
%   w~stanie wszystkich nauczyć. Dobrze, tylko w~takim razie, na których
%   powinniśmy~się skupić?

%   Dobrym miernikiem ważność języka jest indeks \textsc{tiobe}, które
%   stara~się mierzyć popularność wszystkich możliwych języków programowania.
%   Indeks ten jest aktualizowany co miesiąc, sami zaś jego twórcy
%   zastrzegają, iż~zadanie mierzenia popularności języków jest bardzo trudne
%   i~użyte metody pozostawiają duży margines błędu. Niemniej jest to wciąż
%   narzędzie którego warto używać.

%   Więcej informacji o~indeksie \textsc{tiobe} można znaleźć na stronie \\
%   \colorhref{https://www.tiobe.com/tiobe-index/}
%   {https://www.tiobe.com/tiobe-index/} i~dostępnych tam odnośnikach.
%   My zaś przyjrzymy~się pierwszej dwudziestce języków i~ich zasięgowi.

% \end{frame}
% % ##################





% % ##################
% \begin{frame}
%   \frametitle{TIOBE indeks, lipiec 2024}


%   \begin{itemize}

%   \item[1)] Python, 16.21\%.



%   \item[2)] C++, 10.34\%.



%   \item[3)] C, 9.48\%.



%   \item[4)] Java, 8.59\%.



%   \item[5)] C\#, 6.72\%.




%   \item[6)] JavaScript, 3.79\%.




%   \item[7)] Go, 2.19\%.




%   \item[8)] Visual Basic, 2.08\%.




%   \item[9)] Fortran, 2.05\%.

%   \end{itemize}

% \end{frame}
% % ##################





% % % ##################
% % \begin{frame}
% %   \frametitle{???}



% % \end{frame}
% % % ##################





% % % ##################
% % \begin{frame}
% %   \frametitle{?????}



% % \end{frame}
% % % ##################





% % ##################
% \begin{frame}
%   \frametitle{Prosty schemat rozwiązywania problemu\ldots}


%   Poniżej prezentujemy prosty schemat, jak wygląda procedura
%   rozwiązania pewnego problemu za pomocą programu napisanego w~języku
%   programowania~X.


%   \begin{enumerate}
%     \setlength{\itemsep}{1em}

%   \item[1)] Mamy problemy który można rozwiązać za pomocą komputera.



%   \item[2)] Znajdujemy sposób jako go rozwiązać. To rozwiązanie możemy
%     wymyślić sami, znaleźć w~internecie, zapytać \textsc{ai}, etc.



%   \item[3)] Piszemy program komputerowy w~języku~X, który implementuje
%     dane rozwiązanie.



%   \item[4)] Na podstawie napisanego przez nas kodu tworzymy program
%     w~postaci wykonywalnej, zrozumiałej dla~komputera.



%   \item[5)] Uruchamiamy program utworzony w~punkcie~4 na konkretnym
%     komputerze, tak by rozwiązał nasz problem.

%   \end{enumerate}

% \end{frame}
% % ##################





% % ##################
% \begin{frame}
%   \frametitle{Przykład}


%   \begin{itemize}

%   \item[1)] Mamy 100 000 nazw użytkowników jakiegoś portalu. Potrzebujemy
%     sprawdzić, czy nazwy zgłaszających~się regularnie nowych użytkowników
%     są już zajęta czy nie. Spostrzegamy, że~można to zrobić za pomocą
%     odpowiedniego programu komputerowego.



%   \item[2)] Zauważamy, że~jeśli ustawimy nazwy dotychczasowych użytkowników
%     w~posortowany ciąg, to możemy sprawdzić czy dana nazwa w~nim jest
%     za~pomocą wydajnego przeszukiwania binarnego. Wymyślamy jak można
%     zaprowadzić relację porządku w~zbiorze tych nazw, co jest nam potrzebne
%     do~ich posortowania.



%   \item[3)] Piszemy program w~języku~X, który pobierze na wejściu nazwy
%     użytkowników, następnie je posortuje według zadanego przepisu.
%     Następnie, jeśli program ten otrzyma nową nazwę, to sprawdzi za pomocą
%     przeszukiwania binarnego, czy jest już ona wśród nazw już zajętych.

%   \end{itemize}

% \end{frame}
% % ##################





% % ##################
% \begin{frame}
%   \frametitle{Przykład}


%   \begin{itemize}

%   \item[4)] Na~podstawie kodu w~języku~X generujemy odpowiedni program
%     w~postaci wykonywalnej.



%   \item[5)] Uruchamiamy wygenerowany w~punkcie~4 program na~konkretnym
%     komputerze, by rozwiązał nasz problem.

%   \end{itemize}

% \end{frame}
% % ##################





% % ##################
% \begin{frame}
%   \frametitle{Co to jest implementacja?}


%   Często znamy ogólną, abstrakcyjną postać rozwiązania danego problemu, czy
%   też sposobu działania konkretnego algorytmu. Przykładowo „program
%   umożliwiający obliczanie średniej arytmetycznej z~miliona liczb.”, to
%   taka ogólny opis programu pewnego programu, który wykonuje potrzebną nam
%   czynność. Jeśli jest to potrzebne, możemy podać jego bardziej precyzyjną
%   formę. „Dany jest milion liczb. Wylicz ich sumę, następnie podziel ją
%   przez liczbę $1 \, 000 \, 000$. Tak otrzymana liczba jest poszukiwanym
%   rezultatem.”

%   Takie opis programu jest zrozumiały dla ludzi, ale nie dla komputera,
%   w~sensie hardwaru. Przy czym zasada ta stosuje~się też, choć w~innym
%   zakresie, do~\textsc{ai} i~\textsc{lll}, ale~to temat na osobne dyskusje.
%   Z~tego powodu musimy napisać odpowiedni kod, który będzie zrozumiały dla
%   komputera, dopiero wtedy on będzie mógł go wykonać.

% \end{frame}
% % ##################





% % ##################
% \begin{frame}
%   \frametitle{Co to jest implementacja?}


%   Jeśli mamy konkretny kod, albo program, który wykonuje dane zadanie
%   zgodnie z~ogólną idę, metodą albo algorytmem, to mówimy że~taki kod lub
%   program jest \textbf{implementacją} tej idei bądź algorytmu.

%   To wyjaśnienie nie jest bardzo precyzyjne, ale też nie musi być.
%   Implementacja nie jest pojęciem ważnym dla komputera, tylko dla ludzi,
%   stąd od precyzyjnej definicji ważniejsze jest zrozumienie ogólnej idei.

% \end{frame}
% % ##################





% % ##################
% \begin{frame}
%   \frametitle{Przykład. Postawienie problemu i~metoda rozwiązania}


%   \textbf{Problem.} Obliczyć średnią arytmetyczną zadanych $100$ liczb.

%   \textbf{Metoda rozwiązania.} Oblicz sumę danych $100$ liczb, następnie
%   podziel ją przez $100$.

% \end{frame}
% % ##################





% % ##################
% \begin{frame}
%   \frametitle{Implementacja w~języku~C}


%   \texttt{\#include <stdio.h>} \\
%   \vspace{0.8em}
%   /* Zakładamy, że~liczby który średnią arytmetyczną mamy obliczyć są
%   przechowywane w~$100$-elementowej tablicy liczby typu \texttt{double}
%   \hphantom{aa} o~nazwie \texttt{numbersArray}, zdefiniowanej
%   w~dostarczonym nam ze-
%   \hphantom{aa} wnątrz pliku\texttt{numbersProb.h}. Plik ten ma być w~tym
%   samym kata-
%   \hphantom{aa} logu co poniższy program. */ \\
%   \texttt{\#include "numbersProb.h"} \\
%   \vspace{0.8em}
%   \texttt{int main() \{ } \\
%   \hphantom{aaaa} \texttt{int i = 0;} \\
%   \hphantom{aaaa} \texttt{double suma = 0.0;} \\
%   \hphantom{aaaa} \texttt{double sredniaArytmetycza = 0.0;}

% \end{frame}
% % ##################





% % ##################
% \begin{frame}
%   \frametitle{Implementacja w~języku~C. Cd.}


%   \hphantom{aaaa} \texttt{for (i = 0; i < 100; i++) \{ } \\
%   \hphantom{aaaaaaaa} \texttt{suma += numbersArray[i];} \\
%   \hphantom{aaaa} \texttt{ \} } \\
%   \vspace{0.8em}
%   \hphantom{aaaa} \texttt{sredniaArytmetyczna = suma / 100.0;} \\
%   \vspace{0.8em}
%   \hphantom{aaaa} \texttt{prinft("sredniaArytmetyczna =
%     \%.3f.\textbackslash n", \\
%     \hphantom{aaaaaaaaa} sredniaArytmetyczna);} \\
%   \texttt{ \} }

% \end{frame}
% % ##################







% % ##################
% \begin{frame}
%   \frametitle{Implementacje w~języku Python}


%   \# Zakładamy, że~liczby których średnią arytmetyczną mamy obli- \\
%   \# czyć~są wszystkie typu \texttt{float} i~są przechowywane
%   w~$100$-elementowej \\
%   \# liście o~nazwie \texttt{numbersList}, która znajduje~się w~dostarczonym
%   \# nam module Pythona o~nazwie \texttt{numbersProg}. \\
%   \vspace{0.8em}
%   \texttt{from numbersProg import numbersList} \\
%   \vspace{0.8em}
%   \texttt{suma = 0.0} \\
%   \vspace{0.8em}
%   \texttt{for number in numbersList:} \\
%   \hphantom{aaaa} \texttt{suma += number} \\
%   \vspace{0.8em}
%   \texttt{sredniaArytmetyczna = suma / 100.0} \\
%   \texttt{answer = \textbackslash} \\
%   \hphantom{aaa}
%   \texttt{"sredniaArytmetyczna = \{\}".format(sredniaArytmetyczna)}
%   \\
%   \vspace{0.8em}
%   \texttt{print(answer)}

% \end{frame}
% % ##################





% % % ##################
% % \begin{frame}
% %   \frametitle{????}



% % \end{frame}
% % % ##################


































% % % ##################
% % \begin{frame}
% %   \frametitle{?????}



% % \end{frame}
% % % ##################





% % % ##################
% % \begin{frame}
% %   \frametitle{?????}



% % \end{frame}
% % % ##################






% % % ##################
% % \begin{frame}
% %   \frametitle{?????}




% % \end{frame}
% % % ##################





% % % ##################
% % \begin{frame}
% %   \frametitle{????}




% % \end{frame}
% % % ##################











% % % ##################
% % \begin{frame}
% %   \frametitle{????}




% % \end{frame}
% % % ##################





% % % ##################
% % \begin{frame}
% %   \frametitle{?????}



% % \end{frame}
% % % ##################





% % % ##################
% % \begin{frame}
% %   \frametitle{?????}




% % \end{frame}
% % % ##################





% % % ##################
% % \begin{frame}
% %   \frametitle{?????}



% % \end{frame}
% % % ##################





% % % ##################
% % \begin{frame}
% %   \frametitle{??????}



% % \end{frame}
% % % ##################





% % % ##################
% % \begin{frame}
% %   \frametitle{?????}



% % \end{frame}
% % % ##################





% % % ##################
% % \begin{frame}
% %   \frametitle{?????}



% % \end{frame}
% % % ##################





% % % ##################
% % \begin{frame}
% %   \frametitle{????}



% % \end{frame}
% % % ##################





% % % ##################
% % \begin{frame}
% %   \frametitle{????}



% % \end{frame}
% % % ##################





% % % ##################
% % \begin{frame}
% %   \frametitle{?????}




% % \end{frame}
% % % ##################





% % % ##################
% % \begin{frame}
% %   \frametitle{????}



% % \end{frame}
% % % ##################





% % % ##################
% % \begin{frame}
% %   \frametitle{?????}



% % \end{frame}
% % % ##################





% % % ##################
% % \begin{frame}
% %   \frametitle{?????}



% % \end{frame}
% % % ##################





% % % ##################
% % \begin{frame}
% %   \frametitle{?????}




% % \end{frame}
% % % ##################





% % % ##################
% % \begin{frame}
% %   \frametitle{?????}




% % \end{frame}
% % % ##################





% % % ##################
% % \begin{frame}
% %   \frametitle{?????}



% % \end{frame}
% % % ##################





% % % ##################
% % \begin{frame}
% %   \frametitle{????}



% % \end{frame}
% % % ##################





% % % ##################
% % \begin{frame}
% %   \frametitle{?????}



% % \end{frame}
% % % ##################





% % % ##################
% % \begin{frame}
% %   \frametitle{????}




% % \end{frame}
% % % ##################





% % % ##################
% % \begin{frame}
% %   \frametitle{?????}



% % \end{frame}
% % % ##################





% % % ##################
% % \begin{frame}
% %   \frametitle{?????}





% % \end{frame}
% % % ##################





% % % ##################
% % \begin{frame}
% %   \frametitle{?????}




% % \end{frame}
% % % ##################





% % % ##################
% % \begin{frame}
% %   \frametitle{????}




% % \end{frame}
% % % ##################










% % % ######################################
% % \appendix
% % % ######################################





% % % ##################
% % \GeometryThreeDTwoSpecialEndingSlidesEN{Questions? Thank you for your attention.}
% % % ##################



% % % % ##################
% % % \jagiellonianendslide{Dziękuję za~uwagę.}
% % % % ##################










% % % ######################################
% % \SectionSlideWithPicture{Terminological notes}
% % % ######################################



% % % ##################
% % \begin{frame}
% %   \frametitle{??????}



% % \end{frame}
% % % ##################





% % % ##################
% % \begin{frame}
% %   \frametitle{?????}



% % \end{frame}
% % ##################





% % ##################
% % \begin{frame}
% %   \frametitle{?????}



% % \end{frame}
% % ##################





% % ##################
% % \begin{frame}
% %   \frametitle{?????}



% % \end{frame}
% % ##################





% % ##################
% % \begin{frame}
% %   \frametitle{?????}



% % \end{frame}
% %  ##################







% % ######################################
% \section{Informacje wstępne}
% % ######################################



% % ##################
% \begin{frame}
%   \frametitle{Informacje wstępne}


%   Obawiam~się, że na tych konkretnych zajęciach będzie sporo przynudzania,
%   ale nie widzę sposobu, by~tego uniknąć.

%   Według mnie to zajęcia są dla studentów, nie studenci dla zajęć. Tak samo
%   ja jestem tu dla Państwa, a~nie Państwo dla mnie. Jestem tu po to, by
%   pomóc Państwu stawiać pierwsze kroki w~programowaniu w~języku~C.
%   W~związku z~tym, ja będę Państwa rozliczał tylko i~wyłącznie
%   z~umiejętności i~wiedzy, z~niczego innego. Wychodzę bowiem z~założenia,
%   że~Państwo sami najlepiej wiedzą, czemu warto poświęcić swój czas. (Choć
%   jak wiadomo, nie jeden raz potem stwierdzamy, że~nasz wybór mógł być
%   jednak lepszy.)

%   Na zajęciach nie tylko można, ale \alert{należy} zadawać pytania
%   na dowolne związane z~zajęciami zagadnienia. W~szczególności
%   \alert{należy} zadawać pytania, jeśli~się czegoś nie rozumie, lub coś
%   jest niejasne. To są podstawy informatyki, \alert{nie} zakładamy,
%   że~Państwo mają już wszystko umieć. Byłoby to bardzo niewłaściwe
%   założenie.

% \end{frame}
% % ##################





% % ##################
% \begin{frame}
%   \frametitle{Informacje wstępne}


%   Proszę pamiętać, gdy chodzi o~tematy związane z~zajęciami
%   \alert{nie} ma pytań zbyt elementarnych lub zbyt głupich. Są~tylko
%   niezadowalające odpowiedzi na pytania.

%   Jestem tutaj by Państwu pomóc w~nauce programowania w~C, pytania
%   z~Państwa strony bardzo mi to ułatwiają. Zadawania pytań nie traktuję
%   jako oznakę tego, że~ktoś czegoś nie umie, tylko że~chce~się czegoś
%   nauczyć.

%   Pytania typu „Jaki jest najfajniejszy boss w~grze \textit{Hollow
%     Knight}?” musimy jednak zostawić na czas po zajęciach.

%   % Na tych zajęciach \alert{nie} nauczymy~się jak programować. Jak dobrze
%   % pójdzie to nauczymy~się podstaw programowania w~języku~C, ale
%   % programowanie obejmuje tyle zagadnień i~wymaga tyle godzin praktyki,
%   % iż~nie ma najmniejszych szans, że~uda nam~się to wszystko zrobić.

%   % Co, na Państwa nieszczęście, nie oznacza, że~będzie mało materiału.
%   % Mogą~się Państwo wręcz czuć przytłoczeni ile tego jest. Proszę mi jednak
%   % uwierzyć, że~większość tego nie jest taka trudna, jak~się wydaje na
%   % pierwszy rzut oka.

% \end{frame}
% % ##################





% % ##################
% \begin{frame}
%   \frametitle{Informacje wstępne}


%   % Proszę pamiętać, gdy chodzi o~tematy związane z~zajęciami
%   % \alert{nie} ma pytań zbyt elementarnych lub zbyt głupich. Są~tylko
%   % niezadowalające odpowiedzi na pytania.

%   % Jestem tutaj by Państwu pomóc w~nauce programowania w~C, pytania
%   % z~Państwa strony bardzo mi to ułatwiają. Zadawania pytań nie traktuję
%   % jako oznakę tego, że~ktoś czegoś nie umie, tylko że~chce~się czegoś
%   % nauczyć.

%   % Pytania typu „Jaki jest najfajniejszy boss w~grze \textit{Hollow
%   %   Knight}?” musimy jednak zostawić na czas po zajęciach.

%   Na tych zajęciach \alert{nie} nauczymy~się jak programować. Jak dobrze
%   pójdzie to nauczymy~się podstaw programowania w~języku~C, ale
%   programowanie obejmuje tyle zagadnień i~wymaga tyle godzin praktyki,
%   iż~nie ma najmniejszych szans, że~uda nam~się to wszystko zrobić.

%   Co, na Państwa nieszczęście, nie oznacza, że~będzie mało materiału.
%   Mogą~się Państwo wręcz czuć przytłoczeni ile tego jest. Proszę mi jednak
%   uwierzyć, że~większość tego nie jest taka trudna, jak~się wydaje na
%   pierwszy rzut oka.

% \end{frame}
% % ##################





% % ##################
% \begin{frame}
%   \frametitle{Uwagi odnośnie treści zajęć}


%   Ponieważ tematyka którą poruszamy jest mimo wszystko niebanalna, więc
%   mnóstwo rzeczy będę musiał bardzo \alert{upraszczać}. Proszę mieć to na
%   uwadze w~trakcie zajęć i~studiując materiały dla Państwa przygotowane.

%   Potrzeba uproszczeń wynika z~dwóch powodów. Po pierwsze, ograniczenia
%   czasowe. Wiele z~zagadnień które poruszymy mogłoby być tematem
%   semestralnego kursu. (Co gorsza, wiele z~nich \alert{jest} tematem
%   semestralnych kursów.) Po drugie, to jest kurs \alert{podstaw}
%   informatyki, który ma położyć fundamenty pod Państwa umiejętności
%   i~wiedzę. To nie przedmiot na którym należy wnikać we wszystkie detale,
%   szczegóły, drugie, trzecie i~czwarte dno problemu.

%   Jeśli jednak ktoś chce~się bardziej zagłębić w~te temat, to służę po
%   zajęciach całą swoją osobą.

% \end{frame}
% % ##################





% % ##################
% \begin{frame}
%   \frametitle{Informacje wstępne}


%   Dlaczego zaczynamy naukę od~języka~C? Krótka odpowiedź jest taka,
%   że~pomimo tego iż język ten ma już pół wieku~(!) na karku, w~2024 roku
%   nasza infrastruktura informatyczna wciąż stoi na napisanym w~właśnie
%   w~nim. W~internecie mogą Państwo znaleźć wiele artykułów i~blogów takich
%   jak
%   \colorhref{https://wideinfo.org/c-programming-is-still-running-the-world/}{\textit{C~programming is still running the
%       world}} \parencite{Scott-C-programming-is-still-ETC-Ver-2024}
%   z~czerwca 2024 roku. Jego tytuł mówi sam za siebie.

%   Niektórzy mówią, że~C to król wszystkich języków programowania.
%   Inni, bardzo szacowni programiści, twierdzą, że każdy szanujący
%   programista musi znać~C (cf. str. ?? \parencite{Assembler}).

%   A~jak to jest w~praktyce? Na tych zajęcia potrafią uczęszczać ludzie,
%   którzy pracują zawodowo jako programiści i~nie mają pojęcia jak napisać
%   program w~języku~C. Proszę samemu wyciągnąć z~tego wnioski.

%   % Dlaczego korzystamy z~systemu GNU/Linux, a~nie z~znacznie
%   % popularniejszego wśród normalny ludzi systemu Windows? Bo~oferuje
%   % lepsze warunki do pracy z~językiem~C.

% \end{frame}
% % ##################





% % ##################
% \begin{frame}
%   \frametitle{Dlaczego język~C?}


%   % Dlaczego zaczynamy naukę od~języka~C? Krótka odpowiedź jest taka,
%   % że~pomimo tego iż język ten ma już pół wieku~(!) na karku, w~2024 roku
%   % nasza infrastruktura informatyczna wciąż stoi na języku~C. Niektórzy
%   % mówią, że~C to król wszystkich języków programowania. Inni mówią, że~każdy

%   % W~internecie mogą Państwo znaleźć wiele artykułów i~blogów takich jak
%   % \colorhref{https://wideinfo.org/c-programming-is-still-running-the-world/}{\textit{C~programming is still running the
%   %     world}}
%   % z~czerwca 2024 roku, których tytuł mówi sam za siebie
%   % \parencite{Scott-C-programming-is-still-ETC-Ver-2024}.

%   Dlaczego korzystamy z~systemu GNU/Linux, a~nie z~znacznie
%   popularniejszego wśród normalny ludzi systemu Windows? Bo~oferuje
%   znacznie lepsze warunki pracy z~językiem~C.

%   \alert{Ważne.} Jeśli mają Państwo jakiekolwiek problemy z~systemem
%   GNU/Linux to proszę o~tym \alert{mówić}. Nie przyjmujemy założenia,
%   że~Państw mają już teraz być ekspertami, w~kwestii używania tego, co by
%   tu nie mówić, często bardzo topornego systemu operacyjnego.

%   \alert{Ważne.} Wykład ma bardziej charakter teoretyczny, te laboratoria
%   zaś praktyczny. Niemniej wciąż poszukujemy optymalnej formy prowadzenia
%   tych zajęć, bo obecna jest daleka od naszych pragnień. Naprawdę
%   praktyczne zajęcia z~podstaw informatyki w~języku~C są trudne do
%   zorganizowania, z~powodów o~których będziemy mówili później.

% \end{frame}
% % ##################










% % ######################################
% \section{O~uzyskaniu zaliczenia}
% % ######################################



% % ##################
% \begin{frame}
%   \frametitle{Zaliczenie zaoczne}


%   Zaliczenie zaoczne można jak najbardziej uzyskać, np.~przedstawiając
%   swój prywatny projekt jakiegoś programu czy aplikacji. Projekt ten
%   \alert{nie musi} być napisany w~języku~C, może być w~Pythonie albo
%   JavaScripcie. Z~powodów które powinny być dla wszystkich oczywiste,
%   preferowane są jednak te stworzone w~języku C. (W~trochę mniejszym
%   stopniu, w~C++).

%   Każdy kto chce uzyskać zaliczenie zaoczne, niech zgłosi~się do mnie po
%   zajęciach lub napisze na maila \email.

% \end{frame}
% % ##################










% % ##################
% \begin{frame}
%   \frametitle{Uwagi odnośnie treści zajęć}


%   Pod koniec semestru podliczane są wszystkie punkty jakie były do
%   zdobycia. W~zależności ile procent pełnej puli Państwo zdobyli, otrzymują
%   Państwo odpowiednią ocenę.

%   \begin{itemize}

%   \item $41\%\text{--}50\%$ -- ocena dostateczna ($3.0$).

%   \item $51\%\text{--}60\%$ -- ocena plus dostateczna ($3.5$, $3+$).

%   \item $61\%\text{--}70\%$ -- ocena dobra ($4.0$).

%   \item $71\%\text{--}80\%$ -- ocena puls dobry ($4.5$, $4+$).

%   \item $81\%\text{--}100\%$ -- ocena bardzo dobry ($5.0$).

%   \end{itemize}

%   W~przypadku zaokrąglanie wyników, robione to jest zawsze na korzyść dla
%   Państwa. Czyli $40.1\%$ zaokrągla~się do $41\%$.

%   Ilość punktów do zdobycie jest będzie jawnie podana przy każdym
%   konkretnym zadaniu domowym. Za jeden test można zdobyć 10 pkt., konkretna
%   wartość zależy od zdobytej oceny. Za projekt można zdobyć do 30 pkt.
%   Ilość punktów za projekt jest tym, co obecnie najmniej mnie
%   satysfakcjonuje, ale na razie nie mam pomysłu jak to poprawić.

% \end{frame}
% % ##################





% % ##################
% \begin{frame}
%   \frametitle{Punktacja testów}


%   % Pod koniec semestru podliczane są wszystkie punkty jakie były do
%   % zdobycia. W~zależności ile procent pełnej puli Państwo zdobyli, otrzymują
%   % Państwo odpowiednią ocenę.

%   % \begin{itemize}

%   % \item $41\%\text{--}50\%$ -- ocena dostateczna ($3.0$).

%   % \item $51\%\text{--}60\%$ -- ocena plus dostateczna ($3.5$, $3+$).

%   % \item $61\%\text{--}70\%$ -- ocena dobra ($4.0$).

%   % \item $71\%\text{--}80\%$ -- ocena puls dobry ($4.5$, $4+$).

%   % \item $81\%\text{--}100\%$ -- ocena bardzo dobry ($5.0$).

%   % \end{itemize}

%   % W~przypadku zaokrąglanie wyników, robione to jest zawsze na korzyść dla
%   % Państwa. Czyli $40.1\%$ zaokrągla~się do $41\%$.

%   % Ilość punktów do zdobycie jest będzie jawnie podana przy każdym
%   % konkretnym zadaniu domowym. Za jeden test można zdobyć 10 pkt., konkretna
%   % wartość zależy od zdobytej oceny. Za projekt można zdobyć 20 pkt.
%   % Ilość punktów za projekt jest tym, co obecnie najmniej mnie
%   % satysfakcjonuje, ale na razie nie mam pomysłu jak to poprawić.

%   W~tym semestrze każdy test będzie zawierał 10 pytań jednokrotnego wyboru.
%   Progi ocen są jak dla całego przedmiotu i~regułą zaokrąglania (raczej nie
%   będzie trzeba jej używać) działa jak poprzednio.

%   \begin{itemize}

%   \item $41\%\text{--}50\%$ -- ocena dostateczna ($3.0$).

%   \item $51\%\text{--}60\%$ -- ocena plus dostateczna ($3.5$, $3+$).

%   \item $61\%\text{--}70\%$ -- ocena dobra ($4.0$).

%   \item $71\%\text{--}80\%$ -- ocena puls dobry ($4.5$, $4+$).

%   \item $81\%\text{--}100\%$ -- ocena bardzo dobry ($5.0$).

%   \end{itemize}

%   Ilość zdobytych punktów równa jest zero jeśli uzyskało~się ocenę $2.0$
%   (mniej niż $41\%$), lub $2 \cdot \text{ocena}$.

% \end{frame}
% % ##################




% % ##################
% \begin{frame}
%   \frametitle{Punktacja testów}


%   % Pod koniec semestru podliczane są wszystkie punkty jakie były do
%   % zdobycia. W~zależności ile procent pełnej puli Państwo zdobyli, otrzymują
%   % Państwo odpowiednią ocenę.

%   % \begin{itemize}

%   % \item $41\%\text{--}50\%$ -- ocena dostateczna ($3.0$).

%   % \item $51\%\text{--}60\%$ -- ocena plus dostateczna ($3.5$, $3+$).

%   % \item $61\%\text{--}70\%$ -- ocena dobra ($4.0$).

%   % \item $71\%\text{--}80\%$ -- ocena puls dobry ($4.5$, $4+$).

%   % \item $81\%\text{--}100\%$ -- ocena bardzo dobry ($5.0$).

%   % \end{itemize}

%   % W~przypadku zaokrąglanie wyników, robione to jest zawsze na korzyść dla
%   % Państwa. Czyli $40.1\%$ zaokrągla~się do $41\%$.

%   % Ilość punktów do zdobycie jest będzie jawnie podana przy każdym
%   % konkretnym zadaniu domowym. Za jeden test można zdobyć 10 pkt., konkretna
%   % wartość zależy od zdobytej oceny. Za projekt można zdobyć 20 pkt.
%   % Ilość punktów za projekt jest tym, co obecnie najmniej mnie
%   % satysfakcjonuje, ale na razie nie mam pomysłu jak to poprawić.

%   Ilość punktów do końcowej uzyskanych w~zależności od wyniku testu.

%   \begin{itemize}

%   \item $0\%\text{--}40\%$ -- $0$ pkt.

%   \item $41\%\text{--}50\%$ -- $6$ pkt.

%   \item $51\%\text{--}60\%$ -- $7$ pkt.

%   \item $61\%\text{--}70\%$ -- $8$ pkt.

%   \item $71\%\text{--}80\%$ -- $9$ pkt.

%   \item $81\%\text{--}100\%$ -- $10$ pkt.

%   \end{itemize}

%   % Ilość zdobytych punktów równa jest zero jeśli uzyskało~się ocenę $2.0$
%   % (mniej niż $40\%$), lub $2 \cdot \text{ocena}$ (czyli jeśli ma się $3.0$ lub
%   % więcej, to dostaje się tyle punktów, ile poprawnych odpowiedzi
%   % na~pytania~się udzieliło).

% \end{frame}
% % ##################










% % ######################################
% \section{Materiały do nauki}
% % ######################################



% % ##################
% \begin{frame}
%   \frametitle{Materiały do nauki}


%   Prezentacje te są dostępne w~formie plików \LaTeX a (kodu źródłowego)
%   na serwisie GitHub. Każdy kto ma na komputerze program Git i~dostęp
%   do internetu może jest zdobyć wpisując \\
%   \texttt{\$ git clone https://github.com/KZiemian/Presentation} \\
%   Znajdują~się one w~katalogu „Podstawy-informatyki-ETC-Prezentacje”.

%   Będą też dostępne w~formie \textsc{pdfów} na~Sake, wraz z~innymi
%   materiałami. Dostępna tam będzie również lista zagadnień do opanowania
%   z~tego przedmiotu, która będzie główny punktem odniesieniem przy
%   tworzeniu pytań testowych. Jak również dwa listy materiałów do nauki.
%   Jedna lista normalna, druga dla osób ambitnych.

%   Proszę zwrócić uwagę, że~ze względu na charakter tych zajęć, wystarczające
%   jest, żeby o~pewnych rzeczach wymienionych na liście zagadnień mieli
%   Państwo bardzo ogólne i~podstawowe pojęcie. Nawet w~sytuacji, gdy na
%   kursie było o~danym zagadnieniu powiedziane znacznie więcej.
%   % Dotyczy to nawet rzeczy, które,
%   % w~zależności od tego jak pójdą nam zajęcia, będą omawiane bardziej
%   % szczegółowo, ale nie należą do tematów, które są kluczowe dla tego kursu.

% \end{frame}
% % ##################





% % ##################
% \begin{frame}
%   \frametitle{Punktacja testów}



%   Przykładowo, jest wystarczające, żeby Państwo wiedzieli, że~kompilator
%   języka~C jest to program, który przetwarza kod napisany w~języku~C
%   w~program, który jest napisany w~języku zrozumiałym dla komputera.
%   Nawet jeśli na zajęciach wspomnimy czym są takie części kompilatora
%   jak lekser czy parser, nie jest wymagane by Państwo po tym kursie
%   wiedzieli o~ich istnieniu, nie mówiąc już o~znajomości tego co robią.

%   Na liście zagadnień postaramy~się wyróżnić tego typu pytania w~specjalny
%   sposób. Dodać należy, że~gdy chodzi o~pozostałe pytania, wymagana jest
%   dobra znajomość na \alert{poziomie tego co było prezentowane na kursie},
%   nie zaś taka jaka jest zawarta w~standardzie??? języka~C. Bądźmy poważni,
%   to jest tylko kurs podstaw informatyki.


%   % Pod koniec semestru podliczane są wszystkie punkty jakie były do
%   % zdobycia. W~zależności ile procent pełnej puli Państwo zdobyli, otrzymują
%   % Państwo odpowiednią ocenę.

%   % \begin{itemize}

%   % \item $41\%\text{--}50\%$ -- ocena dostateczna ($3.0$).

%   % \item $51\%\text{--}60\%$ -- ocena plus dostateczna ($3.5$, $3+$).

%   % \item $61\%\text{--}70\%$ -- ocena dobra ($4.0$).

%   % \item $71\%\text{--}80\%$ -- ocena puls dobry ($4.5$, $4+$).

%   % \item $81\%\text{--}100\%$ -- ocena bardzo dobry ($5.0$).

%   % \end{itemize}

%   % W~przypadku zaokrąglanie wyników, robione to jest zawsze na korzyść dla
%   % Państwa. Czyli $40.1\%$ zaokrągla~się do $41\%$.

%   % Ilość punktów do zdobycie jest będzie jawnie podana przy każdym
%   % konkretnym zadaniu domowym. Za jeden test można zdobyć 10 pkt., konkretna
%   % wartość zależy od zdobytej oceny. Za projekt można zdobyć 20 pkt.
%   % Ilość punktów za projekt jest tym, co obecnie najmniej mnie
%   % satysfakcjonuje, ale na razie nie mam pomysłu jak to poprawić.

%   % Ilość punktów do końcowej uzyskanych w~zależności od wyniku testu.

%   % \begin{itemize}

%   % \item $0\%\text{--}40\%$ -- $0$ pkt.

%   % \item $41\%\text{--}50\%$ -- $6$ pkt.

%   % \item $51\%\text{--}60\%$ -- $7$ pkt.

%   % \item $61\%\text{--}70\%$ -- $8$ pkt.

%   % \item $71\%\text{--}80\%$ -- $9$ pkt.

%   % \item $81\%\text{--}100\%$ -- $10$ pkt.

%   % \end{itemize}

%   % Ilość zdobytych punktów równa jest zero jeśli uzyskało~się ocenę $2.0$
%   % (mniej niż $40\%$), lub $2 \cdot \text{ocena}$ (czyli jeśli ma się $3.0$ lub
%   % więcej, to dostaje się tyle punktów, ile poprawnych odpowiedzi
%   % na~pytania~się udzieliło).

% \end{frame}
% % ##################









% % ##################
% \begin{frame}
%   \frametitle{Zgłaszanie błędu i~uwag}


%   W~razie znalezienia jakiegokolwiek błędu lub jakichkolwiek uwag
%   merytorycznych do zajęć lub dostępnych materiałów proszę zgłaszać to
%   przed lub po zajęciach lub też pisać pod adres \email. Chcemy by te
%   zajęcia i~towarzyszące im materiały były możliwie proste, łatwe
%   w~zrozumieniu i~pozbawione błędów. Proszę jednak uwierzyć, że~osiągnięcie
%   tego jest naprawdę trudne.

% \end{frame}
% % ##################










% % ######################################
% \section{Czy informatyka jest trudna?}
% % ######################################



% % ##################
% \begin{frame}
%   \frametitle{Czy informatyka jest trudna?}


%   Ten przedmiot dotyczy podstaw informatyki w~języku~C, warto~się
%   więc spytać, czy informatyka jest prosta czy trudna w~nauce?

%   Informatyka to osobna dziedzina nauki i~jeśli zabrnie~się odpowiednio
%   głęboko, to robi~się naprawdę złożona i~niebanalna. Jednak na stosunkowo
%   płytkim poziomie to czy jest on trudna czy nie, to mocno zależy od~odczuć
%   konkretnej osoby.

%   Zadam takie pytanie: czy włączenie komputera jest skomplikowane?
%   Odpowiemy na to pytanie na dwóch poziomach. Pierwszy to poziom normalnego
%   użytkownika, drugi to opis pochodzący z~książki Andrewa S.~Tanenbauma
%   \textit{Systemy operacyjne. Wydanie~III}
%   \parencite{Tannenbaum-Systemy-Operacyjne-Wydanie-III-Pub-2013}, dotyczący
%   komputera z~systemem Pentium.

% \end{frame}
% % ##################





% % ##################
% \begin{frame}
%   \frametitle{Włączanie komputera, poziom normalnego użytkownika}


%   \begin{enumerate}

%   \item Wciskamy przycisk \texttt{Power}.



%   \item Czekamy minutę albo dłużej.



%   \item Wybieramy użytkownika i~wchodzimy na swoje konto.

%   \end{enumerate}

%   Co w~tym trudnego?

% \end{frame}
% % ##################





% % ##################
% \begin{frame}
%   \frametitle{Kilka pojęcia}


%   Oczywiście, opis włączania komputera z~książki Tanenbauma jest tak
%   skomplikowany, że~trzeba wprowadzić trochę pojęć wstępnych.

%   \textbf{\textsc{rom}}, ang.~\textit{Read Only Memory}, pl.~\textit{pamięć
%     wyłącznie do~odczytu}. Pamięć komputera której zawartość została
%   zapisana przez firmę, która ten fragment pamięci wyprodukowała
%   i~użytkownik nie może zmodyfikować jej teści. Przynajmniej nie w~żaden
%   normalny sposób.

%   \textbf{\textsc{ram}}, ang.~\textit{Random Access Memory},
%   pl.~\textit{pamięć o~dostępie w~trybie losowym}. Pamięć komputera o~tej
%   własności, że~jeśli będę w~sposób losowy wybierał elementy tej pamięci,
%   to czas odczytania informacje z~każdego jej elementu będzie taki sam.
%   Inaczej mówiąc dostęp do dowolnego miejsca tej pamięci zajmuje tyle samo
%   czasu.

%   Tak naprawdę czas odczytu zależy od tego, w~jakiś sposób pamięć
%   \textsc{ram} jest odczytywana, ale jeszcze długo nie będziemy się musieli
%   tym przejmować.

% \end{frame}
% % ##################





% % ##################
% \begin{frame}
%   \frametitle{Kilka pojęcia}


%   \textbf{Pamięć ulotna}, ang.~\textit{volatile memory}. Pamięć której
%   zawartość jest tracona, gdy przestaje przez nią płynąć prąd. Typowym
%   przykładem takiej pamięci jest \textsc{ram}.

%   \textbf{Pamięć nieulotna}, ang.~\textit{non-volatile memory}. Pamięć,
%   której treść jest zachowana, gdy przez układ przestaje płynąć prąd,
%   typowym przykładem jest dysk \textsc{ssd}.

%   Żeby skomplikować życie, pamięcią nieulotną nazywa~się także tą pamięć,
%   które jest ulotna w~ścisłym sensie, ale ponieważ jest zaopatrzona
%   we~własną baterię, jej zawartość jest zachowana również po wyłączeniu
%   komputera z~prądu. Bo~niby czemu życie ma być proste?

% \end{frame}
% % ##################





% % ##################
% \begin{frame}
%   \frametitle{Kilka pojęcia}


%   \textbf{Pamięć \textsc{cmos}}, często po prostu \textbf{\textsc{cmos}}.
%   Skrót pochodzi od angielskiej nazwy technologi \textit{Complementary
%     Metal-Oxide-Semiconductor} (pl.~\textit{komplementarny półprzewodnik
%     metalowo-tlenkowy}), w~której ta pamięć jest wykonana. Musi być zasilana
%   prądem, by~zachowywała swój stan, ale ponieważ wyposażona jest w~baterię
%   klasyfikowana jest jako nieulotna.

%   \textbf{\textsc{bios}} ang.~\textit{Basic Input Output System}, pl.
%   \textit{podstawowy system wejścia, wyjścia}. Program znajdujący~się
%   na płycie głównej komputera, odpowiedzialny między innymi za odczytywanie
%   klawiatury, zapisywanie ekranu oraz operacje wejścia-wyjścia dysków.

% \end{frame}
% % ##################





% % ##################
% \begin{frame}
%   \frametitle{Uruchamianie komputera z~systemem Pentium}


%   \begin{itemize}

%   \item[1)] Wciskamy przycisk \texttt{Power}.



%   \item[2)] Z~płyty głównej ładowany jest program \textsc{bios}. Sprawdza on
%     ilość zainstalowanej pamięci \textsc{ram}, czy komputer dysponuje
%     klawiaturą i~innymi podstawowymi urządzeniami oraz sprawdza czy
%     odpowiadają one w~sposób prawidłowy. W~pierwszej kolejności skanowane
%     są magistrale \textsc{isa} (ang. \textit{Industry Standard
%       Architecture}) i~\textsc{pci} (ang.~\textit{Peripheral Component
%       Interconnect}) w~celu wykrycia podłączonych do nich urządzeń.



%   \item[3)] Jeśli do komputera podłączone są inne urządzenia, niż te które
%     były dostępne przy jego ostatni uruchomieniu, nowe urządzenia są
%     konfigurowane.



%   \item[4)] Program \textsc{bios} odczytuje listę tzw. urządzeń rozruchowych
%     z~pamięci \textsc{cmos}. Urządzenia rozruchowe to te, które zawierają
%     system operacyjny. W~przeszłości były nimi dyskietki, płyty
%     \textsc{cd}-\textsc{rom}, \textsc{dvd}, dziś choćby pendriwy
%     i~dyski~\textsc{ssd}.

%   \end{itemize}

% \end{frame}
% % ##################





% % ##################
% \begin{frame}
%   \frametitle{Uruchamianie komputera z~systemem Pentium}


%   \begin{itemize}

%   \item[5)] \textsc{bios} testuje po kolei urządzenia rozruchowe
%     z~wspomnianej wcześniej listy, aż~znajdzie pierwsze, który zawiera
%     działający system operacyjny.



%   \item[6)] \textsc{bios} wczytuje pierwszy sektor ze~znalezionego
%     w~poprzednim punkcie urządzenia rozruchowego do pamięci i~go uruchamia.




%   \item[7)] Program z~pierwszego sektora sprawdza zapisaną na jego końcu
%     listę partycji, by~ustalić która z~nich jest partycją aktywną.
%     Następnie wczytuje z~tej partycji pomocniczy program rozruchowy.



%   \item[8)] Pomocniczy program rozruchowy wczytuje system operacyjny
%     z~aktywnej partycji i~go uruchamia.



%   \item[9)] System operacyjny odczytuje informacje konfiguracyjne z~systemu
%     \textsc{bios}. Dla każdego dostępnego urządzenia sprawdza, czy posiada
%     do niego sterowniki. Jeśli nie, to prosi o~ich zainstalowanie
%     z~odpowiedniego źródła.

%   \end{itemize}

% \end{frame}
% % ##################





% % ##################
% \begin{frame}
%   \frametitle{Uruchamianie komputera z~systemem Pentium}


%   \begin{itemize}

%   \item[10)] Jeśli system operacyjny dysponuje wszystkimi sterownikami,
%     to ładuje je do jądra systemu.



%   \item[11)] System operacyjny tworzy tabele systemowe oraz procesy
%     działające w~tle.



%   \item[12)] Uruchamiane jest okno logowania.

%   \end{itemize}

% \end{frame}
% % ##################






% % ##################
% \begin{frame}
%   \frametitle{Bootowanie}


%   W~literaturze funkcjonuje termin \textbf{bootwoanie}, zwane też
%   \textbf{uruchamianiem} lub \textbf{rozruchem}. Odnosi~się ono albo do
%   całej procedury uruchamiania komputer opisanej powyżej, albo tylko
%   stawiania systemu operacyjnego, czyli od kiedy \textsc{bios} wczytał
%   pierwszy jego sektor do pamięci (punkt siedem i~dalej). Acz to pojęcie
%   nie jest specjalnie ostro zdefiniowane.

% \end{frame}
% % ##################
















% ####################################################################
% ####################################################################
% Bibliography

\printbibliography





% ############################
% End of the document

\end{document}
