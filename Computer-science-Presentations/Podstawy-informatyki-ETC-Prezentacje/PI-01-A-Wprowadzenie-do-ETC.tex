% ------------------------------------------------------------------------------------------------------------------
% Basic configuration of Beamera class and Jagiellonian theme
% ------------------------------------------------------------------------------------------------------------------
\RequirePackage[l2tabu, orthodox]{nag}



\ifx\PresentationStyle\notset
  \def\PresentationStyle{dark}
\fi



\documentclass[10pt,t]{beamer}
\mode<presentation>
\usetheme[style=\PresentationStyle]{jagiellonian}




% ------------------------------------------------------------------------------------
% Procesing configuration files of Jagiellonian theme loceted in
% the directory "preambule"
% ------------------------------------------------------------------------------------
% Configuration for polish language
% Need description
\usepackage[polish]{babel}
% Need description
\usepackage[MeX]{polski}



% ------------------------------
% Better support of polish chars in technical parts of PDF
% ------------------------------
\hypersetup{pdfencoding=auto,psdextra}

% Package "textpos" give as enviroment "textblock" which is very usefull in
% arranging text on slides.

% This is standard configuration of "textpos"
\usepackage[overlay,absolute]{textpos}

% If you need to see bounds of "textblock's" comment line above and uncomment
% one below.

% Caution! When showboxes option is on significant ammunt of space is add
% to the top of textblock and as such, everyting put in them gone down.
% We need to check how to remove this bug.

% \usepackage[showboxes,overlay,absolute]{textpos}



% Setting scale length for package "textpos"
\setlength{\TPHorizModule}{10mm}
\setlength{\TPVertModule}{\TPHorizModule}


% ---------------------------------------
% Packages written for lectures "Geometria 3D dla twórców gier wideo"
% ---------------------------------------
% \usepackage{./Geometry3DPackages/Geometry3D}
% \usepackage{./Geometry3DPackages/Geometry3DIndices}
% \usepackage{./Geometry3DPackages/Geometry3DTikZStyle}
% \usepackage{./ProgramowanieSymulacjiFizykiPaczki/ProgramowanieSymulacjiFizykiTikZStyle}
% \usepackage{./Geometry3DPackages/mathcommands}


% ---------------------------------------
% TikZ
% ---------------------------------------
% Importing TikZ libraries
\usetikzlibrary{arrows.meta}
\usetikzlibrary{positioning}





% % Configuration package "bm" that need for making bold symbols
% \newcommand{\bmmax}{0}
% \newcommand{\hmmax}{0}
% \usepackage{bm}




% ---------------------------------------
% Packages for scientific texts
% ---------------------------------------
% \let\lll\undefined  % Sometimes you must use this line to allow
% "amsmath" package to works with packages with packages for polish
% languge imported
% /preambul/LanguageSettings/JagiellonianPolishLanguageSettings.tex.
% This comments (probably) removes polish letter Ł.
\usepackage{amsmath}  % Packages from American Mathematical Society (AMS)
\usepackage{amssymb}
\usepackage{amscd}
\usepackage{amsthm}
\usepackage{siunitx}  % Package for typsetting SI units.
\usepackage{upgreek}  % Better looking greek letters.
% Example of using upgreek: pi = \uppi


\usepackage{calrsfs}  % Zmienia czcionkę kaligraficzną w \mathcal
% na ładniejszą. Może w innych miejscach robi to samo, ale o tym nic
% nie wiem.










% ---------------------------------------
% Packages written for lectures "Geometria 3D dla twórców gier wideo"
% ---------------------------------------
% \usepackage{./ProgramowanieSymulacjiFizykiPaczki/ProgramowanieSymulacjiFizyki}
% \usepackage{./ProgramowanieSymulacjiFizykiPaczki/ProgramowanieSymulacjiFizykiIndeksy}
% \usepackage{./ProgramowanieSymulacjiFizykiPaczki/ProgramowanieSymulacjiFizykiTikZStyle}





% !!!!!!!!!!!!!!!!!!!!!!!!!!!!!!
% !!!!!!!!!!!!!!!!!!!!!!!!!!!!!!
% EVIL STUFF
\if\JUlogotitle1
\edef\LogoJUPath{LogoJU_\JUlogoLang/LogoJU_\JUlogoShape_\JUlogoColor.pdf}
\titlegraphic{\hfill\includegraphics[scale=0.22]
{./JagiellonianPictures/\LogoJUPath}}
\fi
% ---------------------------------------
% Commands for handling colors
% ---------------------------------------


% Command for setting normal text color for some text in math modestyle
% Text color depend on used style of Jagiellonian

% Beamer version of command
\newcommand{\TextWithNormalTextColor}[1]{%
  {\color{jNormalTextFGColor}
    \setbeamercolor{math text}{fg=jNormalTextFGColor} {#1}}
}

% Article and similar classes version of command
% \newcommand{\TextWithNormalTextColor}[1]{%
%   {\color{jNormalTextsFGColor} {#1}}
% }



% Beamer version of command
\newcommand{\NormalTextInMathMode}[1]{%
  {\color{jNormalTextFGColor}
    \setbeamercolor{math text}{fg=jNormalTextFGColor} \text{#1}}
}


% Article and similar classes version of command
% \newcommand{\NormalTextInMathMode}[1]{%
%   {\color{jNormalTextsFGColor} \text{#1}}
% }




% Command that sets color of some mathematical text to the same color
% that has normal text in header (?)

% Beamer version of the command
\newcommand{\MathTextFrametitleFGColor}[1]{%
  {\color{jFrametitleFGColor}
    \setbeamercolor{math text}{fg=jFrametitleFGColor} #1}
}

% Article and similar classes version of the command
% \newcommand{\MathTextWhiteColor}[1]{{\color{jFrametitleFGColor} #1}}





% Command for setting color of alert text for some text in math modestyle

% Beamer version of the command
\newcommand{\MathTextAlertColor}[1]{%
  {\color{jOrange} \setbeamercolor{math text}{fg=jOrange} #1}
}

% Article and similar classes version of the command
% \newcommand{\MathTextAlertColor}[1]{{\color{jOrange} #1}}





% Command that allow you to sets chosen color as the color of some text into
% math mode. Due to some nuances in the way that Beamer handle colors
% it not work in all cases. We hope that in the future we will improve it.

% Beamer version of the command
\newcommand{\SetMathTextColor}[2]{%
  {\color{#1} \setbeamercolor{math text}{fg=#1} #2}
}


% Article and similar classes version of the command
% \newcommand{\SetMathTextColor}[2]{{\color{#1} #2}}










% ---------------------------------------
% Commands for few special slides
% ---------------------------------------
\newcommand{\EndingSlide}[1]{%
  \begin{frame}[standout]

    \begingroup

    \color{jFrametitleFGColor}

    #1

    \endgroup

  \end{frame}
}










% ---------------------------------------
% Commands for setting background pictures for some slides
% ---------------------------------------
\newcommand{\TitleBackgroundPicture}
{./JagiellonianPictures/Backgrounds/LajkonikDark.png}
\newcommand{\SectionBackgroundPicture}
{./JagiellonianPictures/Backgrounds/LajkonikLight.png}



\newcommand{\TitleSlideWithPicture}{%
  \begingroup

  \usebackgroundtemplate{%
    \includegraphics[height=\paperheight]{\TitleBackgroundPicture}}

  \maketitle

  \endgroup
}





\newcommand{\SectionSlideWithPicture}[1]{%
  \begingroup

  \usebackgroundtemplate{%
    \includegraphics[height=\paperheight]{\SectionBackgroundPicture}}

  \setbeamercolor{titlelike}{fg=normal text.fg}

  \section{#1}

  \endgroup
}










% ---------------------------------------
% Commands for lectures "Geometria 3D dla twórców gier wideo"
% Polish version
% ---------------------------------------
% Komendy teraz wykomentowane były potrzebne, gdy loga były na niebieskim
% tle, nie na białym. A są na białym bo tego chcieli w biurze projektu.
% \newcommand{\FundingLogoWhitePicturePL}
% {./PresentationPictures/CommonPictures/logotypFundusze_biale_bez_tla2.pdf}
\newcommand{\FundingLogoColorPicturePL}
{./PresentationPictures/CommonPictures/European_Funds_color_PL.pdf}
% \newcommand{\EULogoWhitePicturePL}
% {./PresentationPictures/CommonPictures/logotypUE_biale_bez_tla2.pdf}
\newcommand{\EUSocialFundLogoColorPicturePL}
{./PresentationPictures/CommonPictures/EU_Social_Fund_color_PL.pdf}
% \newcommand{\ZintegrUJLogoWhitePicturePL}
% {./PresentationPictures/CommonPictures/zintegruj-logo-white.pdf}
\newcommand{\ZintegrUJLogoColorPicturePL}
{./PresentationPictures/CommonPictures/ZintegrUJ_color.pdf}
\newcommand{\JULogoColorPicturePL}
{./JagiellonianPictures/LogoJU_PL/LogoJU_A_color.pdf}





\newcommand{\GeometryThreeDSpecialBeginningSlidePL}{%
  \begin{frame}[standout]

    \begin{textblock}{11}(1,0.7)

      \begin{flushleft}

        \mdseries

        \footnotesize

        \color{jFrametitleFGColor}

        Materiał powstał w ramach projektu współfinansowanego ze środków
        Unii Europejskiej w ramach Europejskiego Funduszu Społecznego
        POWR.03.05.00-00-Z309/17-00.

      \end{flushleft}

    \end{textblock}





    \begin{textblock}{10}(0,2.2)

      \tikz \fill[color=jBackgroundStyleLight] (0,0) rectangle (12.8,-1.5);

    \end{textblock}


    \begin{textblock}{3.2}(1,2.45)

      \includegraphics[scale=0.3]{\FundingLogoColorPicturePL}

    \end{textblock}


    \begin{textblock}{2.5}(3.7,2.5)

      \includegraphics[scale=0.2]{\JULogoColorPicturePL}

    \end{textblock}


    \begin{textblock}{2.5}(6,2.4)

      \includegraphics[scale=0.1]{\ZintegrUJLogoColorPicturePL}

    \end{textblock}


    \begin{textblock}{4.2}(8.4,2.6)

      \includegraphics[scale=0.3]{\EUSocialFundLogoColorPicturePL}

    \end{textblock}

  \end{frame}
}



\newcommand{\GeometryThreeDTwoSpecialBeginningSlidesPL}{%
  \begin{frame}[standout]

    \begin{textblock}{11}(1,0.7)

      \begin{flushleft}

        \mdseries

        \footnotesize

        \color{jFrametitleFGColor}

        Materiał powstał w ramach projektu współfinansowanego ze środków
        Unii Europejskiej w ramach Europejskiego Funduszu Społecznego
        POWR.03.05.00-00-Z309/17-00.

      \end{flushleft}

    \end{textblock}





    \begin{textblock}{10}(0,2.2)

      \tikz \fill[color=jBackgroundStyleLight] (0,0) rectangle (12.8,-1.5);

    \end{textblock}


    \begin{textblock}{3.2}(1,2.45)

      \includegraphics[scale=0.3]{\FundingLogoColorPicturePL}

    \end{textblock}


    \begin{textblock}{2.5}(3.7,2.5)

      \includegraphics[scale=0.2]{\JULogoColorPicturePL}

    \end{textblock}


    \begin{textblock}{2.5}(6,2.4)

      \includegraphics[scale=0.1]{\ZintegrUJLogoColorPicturePL}

    \end{textblock}


    \begin{textblock}{4.2}(8.4,2.6)

      \includegraphics[scale=0.3]{\EUSocialFundLogoColorPicturePL}

    \end{textblock}

  \end{frame}





  \TitleSlideWithPicture
}



\newcommand{\GeometryThreeDSpecialEndingSlidePL}{%
  \begin{frame}[standout]

    \begin{textblock}{11}(1,0.7)

      \begin{flushleft}

        \mdseries

        \footnotesize

        \color{jFrametitleFGColor}

        Materiał powstał w ramach projektu współfinansowanego ze środków
        Unii Europejskiej w~ramach Europejskiego Funduszu Społecznego
        POWR.03.05.00-00-Z309/17-00.

      \end{flushleft}

    \end{textblock}





    \begin{textblock}{10}(0,2.2)

      \tikz \fill[color=jBackgroundStyleLight] (0,0) rectangle (12.8,-1.5);

    \end{textblock}


    \begin{textblock}{3.2}(1,2.45)

      \includegraphics[scale=0.3]{\FundingLogoColorPicturePL}

    \end{textblock}


    \begin{textblock}{2.5}(3.7,2.5)

      \includegraphics[scale=0.2]{\JULogoColorPicturePL}

    \end{textblock}


    \begin{textblock}{2.5}(6,2.4)

      \includegraphics[scale=0.1]{\ZintegrUJLogoColorPicturePL}

    \end{textblock}


    \begin{textblock}{4.2}(8.4,2.6)

      \includegraphics[scale=0.3]{\EUSocialFundLogoColorPicturePL}

    \end{textblock}





    \begin{textblock}{11}(1,4)

      \begin{flushleft}

        \mdseries

        \footnotesize

        \RaggedRight

        \color{jFrametitleFGColor}

        Treść niniejszego wykładu jest udostępniona na~licencji
        Creative Commons (\textsc{cc}), z~uzna\-niem autorstwa
        (\textsc{by}) oraz udostępnianiem na tych samych warunkach
        (\textsc{sa}). Rysunki i~wy\-kresy zawarte w~wykładzie są
        autorstwa dr.~hab.~Pawła Węgrzyna et~al. i~są dostępne
        na tej samej licencji, o~ile nie wskazano inaczej.
        W~prezentacji wykorzystano temat Beamera Jagiellonian,
        oparty na~temacie Metropolis Matthiasa Vogelgesanga,
        dostępnym na licencji \LaTeX{} Project Public License~1.3c
        pod adresem: \colorhref{https://github.com/matze/mtheme}
        {https://github.com/matze/mtheme}.

        Projekt typograficzny: Iwona Grabska-Gradzińska \\
        Skład: Kamil Ziemian;
        Korekta: Wojciech Palacz \\
        Modele: Dariusz Frymus, Kamil Nowakowski \\
        Rysunki i~wykresy: Kamil Ziemian, Paweł Węgrzyn, Wojciech Palacz

      \end{flushleft}

    \end{textblock}

  \end{frame}
}



\newcommand{\GeometryThreeDTwoSpecialEndingSlidesPL}[1]{%
  \begin{frame}[standout]


    \begin{textblock}{11}(1,0.7)

      \begin{flushleft}

        \mdseries

        \footnotesize

        \color{jFrametitleFGColor}

        Materiał powstał w ramach projektu współfinansowanego ze środków
        Unii Europejskiej w~ramach Europejskiego Funduszu Społecznego
        POWR.03.05.00-00-Z309/17-00.

      \end{flushleft}

    \end{textblock}





    \begin{textblock}{10}(0,2.2)

      \tikz \fill[color=jBackgroundStyleLight] (0,0) rectangle (12.8,-1.5);

    \end{textblock}


    \begin{textblock}{3.2}(1,2.45)

      \includegraphics[scale=0.3]{\FundingLogoColorPicturePL}

    \end{textblock}


    \begin{textblock}{2.5}(3.7,2.5)

      \includegraphics[scale=0.2]{\JULogoColorPicturePL}

    \end{textblock}


    \begin{textblock}{2.5}(6,2.4)

      \includegraphics[scale=0.1]{\ZintegrUJLogoColorPicturePL}

    \end{textblock}


    \begin{textblock}{4.2}(8.4,2.6)

      \includegraphics[scale=0.3]{\EUSocialFundLogoColorPicturePL}

    \end{textblock}





    \begin{textblock}{11}(1,4)

      \begin{flushleft}

        \mdseries

        \footnotesize

        \RaggedRight

        \color{jFrametitleFGColor}

        Treść niniejszego wykładu jest udostępniona na~licencji
        Creative Commons (\textsc{cc}), z~uzna\-niem autorstwa
        (\textsc{by}) oraz udostępnianiem na tych samych warunkach
        (\textsc{sa}). Rysunki i~wy\-kresy zawarte w~wykładzie są
        autorstwa dr.~hab.~Pawła Węgrzyna et~al. i~są dostępne
        na tej samej licencji, o~ile nie wskazano inaczej.
        W~prezentacji wykorzystano temat Beamera Jagiellonian,
        oparty na~temacie Metropolis Matthiasa Vogelgesanga,
        dostępnym na licencji \LaTeX{} Project Public License~1.3c
        pod adresem: \colorhref{https://github.com/matze/mtheme}
        {https://github.com/matze/mtheme}.

        Projekt typograficzny: Iwona Grabska-Gradzińska \\
        Skład: Kamil Ziemian;
        Korekta: Wojciech Palacz \\
        Modele: Dariusz Frymus, Kamil Nowakowski \\
        Rysunki i~wykresy: Kamil Ziemian, Paweł Węgrzyn, Wojciech Palacz

      \end{flushleft}

    \end{textblock}

  \end{frame}





  \begin{frame}[standout]

    \begingroup

    \color{jFrametitleFGColor}

    #1

    \endgroup

  \end{frame}
}



\newcommand{\GeometryThreeDSpecialEndingSlideVideoPL}{%
  \begin{frame}[standout]

    \begin{textblock}{11}(1,0.7)

      \begin{flushleft}

        \mdseries

        \footnotesize

        \color{jFrametitleFGColor}

        Materiał powstał w ramach projektu współfinansowanego ze środków
        Unii Europejskiej w~ramach Europejskiego Funduszu Społecznego
        POWR.03.05.00-00-Z309/17-00.

      \end{flushleft}

    \end{textblock}





    \begin{textblock}{10}(0,2.2)

      \tikz \fill[color=jBackgroundStyleLight] (0,0) rectangle (12.8,-1.5);

    \end{textblock}


    \begin{textblock}{3.2}(1,2.45)

      \includegraphics[scale=0.3]{\FundingLogoColorPicturePL}

    \end{textblock}


    \begin{textblock}{2.5}(3.7,2.5)

      \includegraphics[scale=0.2]{\JULogoColorPicturePL}

    \end{textblock}


    \begin{textblock}{2.5}(6,2.4)

      \includegraphics[scale=0.1]{\ZintegrUJLogoColorPicturePL}

    \end{textblock}


    \begin{textblock}{4.2}(8.4,2.6)

      \includegraphics[scale=0.3]{\EUSocialFundLogoColorPicturePL}

    \end{textblock}





    \begin{textblock}{11}(1,4)

      \begin{flushleft}

        \mdseries

        \footnotesize

        \RaggedRight

        \color{jFrametitleFGColor}

        Treść niniejszego wykładu jest udostępniona na~licencji
        Creative Commons (\textsc{cc}), z~uzna\-niem autorstwa
        (\textsc{by}) oraz udostępnianiem na tych samych warunkach
        (\textsc{sa}). Rysunki i~wy\-kresy zawarte w~wykładzie są
        autorstwa dr.~hab.~Pawła Węgrzyna et~al. i~są dostępne
        na tej samej licencji, o~ile nie wskazano inaczej.
        W~prezentacji wykorzystano temat Beamera Jagiellonian,
        oparty na~temacie Metropolis Matthiasa Vogelgesanga,
        dostępnym na licencji \LaTeX{} Project Public License~1.3c
        pod adresem: \colorhref{https://github.com/matze/mtheme}
        {https://github.com/matze/mtheme}.

        Projekt typograficzny: Iwona Grabska-Gradzińska;
        Skład: Kamil Ziemian \\
        Korekta: Wojciech Palacz;
        Modele: Dariusz Frymus, Kamil Nowakowski \\
        Rysunki i~wykresy: Kamil Ziemian, Paweł Węgrzyn, Wojciech Palacz \\
        Montaż: Agencja Filmowa Film \& Television Production~-- Zbigniew
        Masklak

      \end{flushleft}

    \end{textblock}

  \end{frame}
}





\newcommand{\GeometryThreeDTwoSpecialEndingSlidesVideoPL}[1]{%
  \begin{frame}[standout]

    \begin{textblock}{11}(1,0.7)

      \begin{flushleft}

        \mdseries

        \footnotesize

        \color{jFrametitleFGColor}

        Materiał powstał w ramach projektu współfinansowanego ze środków
        Unii Europejskiej w~ramach Europejskiego Funduszu Społecznego
        POWR.03.05.00-00-Z309/17-00.

      \end{flushleft}

    \end{textblock}





    \begin{textblock}{10}(0,2.2)

      \tikz \fill[color=jBackgroundStyleLight] (0,0) rectangle (12.8,-1.5);

    \end{textblock}


    \begin{textblock}{3.2}(1,2.45)

      \includegraphics[scale=0.3]{\FundingLogoColorPicturePL}

    \end{textblock}


    \begin{textblock}{2.5}(3.7,2.5)

      \includegraphics[scale=0.2]{\JULogoColorPicturePL}

    \end{textblock}


    \begin{textblock}{2.5}(6,2.4)

      \includegraphics[scale=0.1]{\ZintegrUJLogoColorPicturePL}

    \end{textblock}


    \begin{textblock}{4.2}(8.4,2.6)

      \includegraphics[scale=0.3]{\EUSocialFundLogoColorPicturePL}

    \end{textblock}





    \begin{textblock}{11}(1,4)

      \begin{flushleft}

        \mdseries

        \footnotesize

        \RaggedRight

        \color{jFrametitleFGColor}

        Treść niniejszego wykładu jest udostępniona na~licencji
        Creative Commons (\textsc{cc}), z~uzna\-niem autorstwa
        (\textsc{by}) oraz udostępnianiem na tych samych warunkach
        (\textsc{sa}). Rysunki i~wy\-kresy zawarte w~wykładzie są
        autorstwa dr.~hab.~Pawła Węgrzyna et~al. i~są dostępne
        na tej samej licencji, o~ile nie wskazano inaczej.
        W~prezentacji wykorzystano temat Beamera Jagiellonian,
        oparty na~temacie Metropolis Matthiasa Vogelgesanga,
        dostępnym na licencji \LaTeX{} Project Public License~1.3c
        pod adresem: \colorhref{https://github.com/matze/mtheme}
        {https://github.com/matze/mtheme}.

        Projekt typograficzny: Iwona Grabska-Gradzińska;
        Skład: Kamil Ziemian \\
        Korekta: Wojciech Palacz;
        Modele: Dariusz Frymus, Kamil Nowakowski \\
        Rysunki i~wykresy: Kamil Ziemian, Paweł Węgrzyn, Wojciech Palacz \\
        Montaż: Agencja Filmowa Film \& Television Production~-- Zbigniew
        Masklak

      \end{flushleft}

    \end{textblock}

  \end{frame}





  \begin{frame}[standout]


    \begingroup

    \color{jFrametitleFGColor}

    #1

    \endgroup

  \end{frame}
}










% ---------------------------------------
% Commands for lectures "Geometria 3D dla twórców gier wideo"
% English version
% ---------------------------------------
% \newcommand{\FundingLogoWhitePictureEN}
% {./PresentationPictures/CommonPictures/logotypFundusze_biale_bez_tla2.pdf}
\newcommand{\FundingLogoColorPictureEN}
{./PresentationPictures/CommonPictures/European_Funds_color_EN.pdf}
% \newcommand{\EULogoWhitePictureEN}
% {./PresentationPictures/CommonPictures/logotypUE_biale_bez_tla2.pdf}
\newcommand{\EUSocialFundLogoColorPictureEN}
{./PresentationPictures/CommonPictures/EU_Social_Fund_color_EN.pdf}
% \newcommand{\ZintegrUJLogoWhitePictureEN}
% {./PresentationPictures/CommonPictures/zintegruj-logo-white.pdf}
\newcommand{\ZintegrUJLogoColorPictureEN}
{./PresentationPictures/CommonPictures/ZintegrUJ_color.pdf}
\newcommand{\JULogoColorPictureEN}
{./JagiellonianPictures/LogoJU_EN/LogoJU_A_color.pdf}



\newcommand{\GeometryThreeDSpecialBeginningSlideEN}{%
  \begin{frame}[standout]

    \begin{textblock}{11}(1,0.7)

      \begin{flushleft}

        \mdseries

        \footnotesize

        \color{jFrametitleFGColor}

        This content was created as part of a project co-financed by the
        European Union within the framework of the European Social Fund
        POWR.03.05.00-00-Z309/17-00.

      \end{flushleft}

    \end{textblock}





    \begin{textblock}{10}(0,2.2)

      \tikz \fill[color=jBackgroundStyleLight] (0,0) rectangle (12.8,-1.5);

    \end{textblock}


    \begin{textblock}{3.2}(0.7,2.45)

      \includegraphics[scale=0.3]{\FundingLogoColorPictureEN}

    \end{textblock}


    \begin{textblock}{2.5}(4.15,2.5)

      \includegraphics[scale=0.2]{\JULogoColorPictureEN}

    \end{textblock}


    \begin{textblock}{2.5}(6.35,2.4)

      \includegraphics[scale=0.1]{\ZintegrUJLogoColorPictureEN}

    \end{textblock}


    \begin{textblock}{4.2}(8.4,2.6)

      \includegraphics[scale=0.3]{\EUSocialFundLogoColorPictureEN}

    \end{textblock}

  \end{frame}
}



\newcommand{\GeometryThreeDTwoSpecialBeginningSlidesEN}{%
  \begin{frame}[standout]

    \begin{textblock}{11}(1,0.7)

      \begin{flushleft}

        \mdseries

        \footnotesize

        \color{jFrametitleFGColor}

        This content was created as part of a project co-financed by the
        European Union within the framework of the European Social Fund
        POWR.03.05.00-00-Z309/17-00.

      \end{flushleft}

    \end{textblock}





    \begin{textblock}{10}(0,2.2)

      \tikz \fill[color=jBackgroundStyleLight] (0,0) rectangle (12.8,-1.5);

    \end{textblock}


    \begin{textblock}{3.2}(0.7,2.45)

      \includegraphics[scale=0.3]{\FundingLogoColorPictureEN}

    \end{textblock}


    \begin{textblock}{2.5}(4.15,2.5)

      \includegraphics[scale=0.2]{\JULogoColorPictureEN}

    \end{textblock}


    \begin{textblock}{2.5}(6.35,2.4)

      \includegraphics[scale=0.1]{\ZintegrUJLogoColorPictureEN}

    \end{textblock}


    \begin{textblock}{4.2}(8.4,2.6)

      \includegraphics[scale=0.3]{\EUSocialFundLogoColorPictureEN}

    \end{textblock}

  \end{frame}





  \TitleSlideWithPicture
}



\newcommand{\GeometryThreeDSpecialEndingSlideEN}{%
  \begin{frame}[standout]

    \begin{textblock}{11}(1,0.7)

      \begin{flushleft}

        \mdseries

        \footnotesize

        \color{jFrametitleFGColor}

        This content was created as part of a project co-financed by the
        European Union within the framework of the European Social Fund
        POWR.03.05.00-00-Z309/17-00.

      \end{flushleft}

    \end{textblock}





    \begin{textblock}{10}(0,2.2)

      \tikz \fill[color=jBackgroundStyleLight] (0,0) rectangle (12.8,-1.5);

    \end{textblock}


    \begin{textblock}{3.2}(0.7,2.45)

      \includegraphics[scale=0.3]{\FundingLogoColorPictureEN}

    \end{textblock}


    \begin{textblock}{2.5}(4.15,2.5)

      \includegraphics[scale=0.2]{\JULogoColorPictureEN}

    \end{textblock}


    \begin{textblock}{2.5}(6.35,2.4)

      \includegraphics[scale=0.1]{\ZintegrUJLogoColorPictureEN}

    \end{textblock}


    \begin{textblock}{4.2}(8.4,2.6)

      \includegraphics[scale=0.3]{\EUSocialFundLogoColorPictureEN}

    \end{textblock}





    \begin{textblock}{11}(1,4)

      \begin{flushleft}

        \mdseries

        \footnotesize

        \RaggedRight

        \color{jFrametitleFGColor}

        The content of this lecture is made available under a~Creative
        Commons licence (\textsc{cc}), giving the author the credits
        (\textsc{by}) and putting an obligation to share on the same terms
        (\textsc{sa}). Figures and diagrams included in the lecture are
        authored by Paweł Węgrzyn et~al., and are available under the same
        license unless indicated otherwise.\\ The presentation uses the
        Beamer Jagiellonian theme based on Matthias Vogelgesang’s
        Metropolis theme, available under license \LaTeX{} Project
        Public License~1.3c at: \colorhref{https://github.com/matze/mtheme}
        {https://github.com/matze/mtheme}.

        Typographic design: Iwona Grabska-Gradzińska \\
        \LaTeX{} Typesetting: Kamil Ziemian \\
        Proofreading: Wojciech Palacz,
        Monika Stawicka \\
        3D Models: Dariusz Frymus, Kamil Nowakowski \\
        Figures and charts: Kamil Ziemian, Paweł Węgrzyn, Wojciech Palacz

      \end{flushleft}

    \end{textblock}

  \end{frame}
}



\newcommand{\GeometryThreeDTwoSpecialEndingSlidesEN}[1]{%
  \begin{frame}[standout]


    \begin{textblock}{11}(1,0.7)

      \begin{flushleft}

        \mdseries

        \footnotesize

        \color{jFrametitleFGColor}

        This content was created as part of a project co-financed by the
        European Union within the framework of the European Social Fund
        POWR.03.05.00-00-Z309/17-00.

      \end{flushleft}

    \end{textblock}





    \begin{textblock}{10}(0,2.2)

      \tikz \fill[color=jBackgroundStyleLight] (0,0) rectangle (12.8,-1.5);

    \end{textblock}


    \begin{textblock}{3.2}(0.7,2.45)

      \includegraphics[scale=0.3]{\FundingLogoColorPictureEN}

    \end{textblock}


    \begin{textblock}{2.5}(4.15,2.5)

      \includegraphics[scale=0.2]{\JULogoColorPictureEN}

    \end{textblock}


    \begin{textblock}{2.5}(6.35,2.4)

      \includegraphics[scale=0.1]{\ZintegrUJLogoColorPictureEN}

    \end{textblock}


    \begin{textblock}{4.2}(8.4,2.6)

      \includegraphics[scale=0.3]{\EUSocialFundLogoColorPictureEN}

    \end{textblock}





    \begin{textblock}{11}(1,4)

      \begin{flushleft}

        \mdseries

        \footnotesize

        \RaggedRight

        \color{jFrametitleFGColor}

        The content of this lecture is made available under a~Creative
        Commons licence (\textsc{cc}), giving the author the credits
        (\textsc{by}) and putting an obligation to share on the same terms
        (\textsc{sa}). Figures and diagrams included in the lecture are
        authored by Paweł Węgrzyn et~al., and are available under the same
        license unless indicated otherwise.\\ The presentation uses the
        Beamer Jagiellonian theme based on Matthias Vogelgesang’s
        Metropolis theme, available under license \LaTeX{} Project
        Public License~1.3c at: \colorhref{https://github.com/matze/mtheme}
        {https://github.com/matze/mtheme}.

        Typographic design: Iwona Grabska-Gradzińska \\
        \LaTeX{} Typesetting: Kamil Ziemian \\
        Proofreading: Wojciech Palacz,
        Monika Stawicka \\
        3D Models: Dariusz Frymus, Kamil Nowakowski \\
        Figures and charts: Kamil Ziemian, Paweł Węgrzyn, Wojciech Palacz

      \end{flushleft}

    \end{textblock}

  \end{frame}





  \begin{frame}[standout]

    \begingroup

    \color{jFrametitleFGColor}

    #1

    \endgroup

  \end{frame}
}



\newcommand{\GeometryThreeDSpecialEndingSlideVideoVerOneEN}{%
  \begin{frame}[standout]

    \begin{textblock}{11}(1,0.7)

      \begin{flushleft}

        \mdseries

        \footnotesize

        \color{jFrametitleFGColor}

        This content was created as part of a project co-financed by the
        European Union within the framework of the European Social Fund
        POWR.03.05.00-00-Z309/17-00.

      \end{flushleft}

    \end{textblock}





    \begin{textblock}{10}(0,2.2)

      \tikz \fill[color=jBackgroundStyleLight] (0,0) rectangle (12.8,-1.5);

    \end{textblock}


    \begin{textblock}{3.2}(0.7,2.45)

      \includegraphics[scale=0.3]{\FundingLogoColorPictureEN}

    \end{textblock}


    \begin{textblock}{2.5}(4.15,2.5)

      \includegraphics[scale=0.2]{\JULogoColorPictureEN}

    \end{textblock}


    \begin{textblock}{2.5}(6.35,2.4)

      \includegraphics[scale=0.1]{\ZintegrUJLogoColorPictureEN}

    \end{textblock}


    \begin{textblock}{4.2}(8.4,2.6)

      \includegraphics[scale=0.3]{\EUSocialFundLogoColorPictureEN}

    \end{textblock}





    \begin{textblock}{11}(1,4)

      \begin{flushleft}

        \mdseries

        \footnotesize

        \RaggedRight

        \color{jFrametitleFGColor}

        The content of this lecture is made available under a Creative
        Commons licence (\textsc{cc}), giving the author the credits
        (\textsc{by}) and putting an obligation to share on the same terms
        (\textsc{sa}). Figures and diagrams included in the lecture are
        authored by Paweł Węgrzyn et~al., and are available under the same
        license unless indicated otherwise.\\ The presentation uses the
        Beamer Jagiellonian theme based on Matthias Vogelgesang’s
        Metropolis theme, available under license \LaTeX{} Project
        Public License~1.3c at: \colorhref{https://github.com/matze/mtheme}
        {https://github.com/matze/mtheme}.

        Typographic design: Iwona Grabska-Gradzińska;
        \LaTeX{} Typesetting: Kamil Ziemian \\
        Proofreading: Wojciech Palacz,
        Monika Stawicka \\
        3D Models: Dariusz Frymus, Kamil Nowakowski \\
        Figures and charts: Kamil Ziemian, Paweł Węgrzyn, Wojciech
        Palacz \\
        Film editing: Agencja Filmowa Film \& Television Production~--
        Zbigniew Masklak

      \end{flushleft}

    \end{textblock}

  \end{frame}
}



\newcommand{\GeometryThreeDSpecialEndingSlideVideoVerTwoEN}{%
  \begin{frame}[standout]

    \begin{textblock}{11}(1,0.7)

      \begin{flushleft}

        \mdseries

        \footnotesize

        \color{jFrametitleFGColor}

        This content was created as part of a project co-financed by the
        European Union within the framework of the European Social Fund
        POWR.03.05.00-00-Z309/17-00.

      \end{flushleft}

    \end{textblock}





    \begin{textblock}{10}(0,2.2)

      \tikz \fill[color=jBackgroundStyleLight] (0,0) rectangle (12.8,-1.5);

    \end{textblock}


    \begin{textblock}{3.2}(0.7,2.45)

      \includegraphics[scale=0.3]{\FundingLogoColorPictureEN}

    \end{textblock}


    \begin{textblock}{2.5}(4.15,2.5)

      \includegraphics[scale=0.2]{\JULogoColorPictureEN}

    \end{textblock}


    \begin{textblock}{2.5}(6.35,2.4)

      \includegraphics[scale=0.1]{\ZintegrUJLogoColorPictureEN}

    \end{textblock}


    \begin{textblock}{4.2}(8.4,2.6)

      \includegraphics[scale=0.3]{\EUSocialFundLogoColorPictureEN}

    \end{textblock}





    \begin{textblock}{11}(1,4)

      \begin{flushleft}

        \mdseries

        \footnotesize

        \RaggedRight

        \color{jFrametitleFGColor}

        The content of this lecture is made available under a Creative
        Commons licence (\textsc{cc}), giving the author the credits
        (\textsc{by}) and putting an obligation to share on the same terms
        (\textsc{sa}). Figures and diagrams included in the lecture are
        authored by Paweł Węgrzyn et~al., and are available under the same
        license unless indicated otherwise.\\ The presentation uses the
        Beamer Jagiellonian theme based on Matthias Vogelgesang’s
        Metropolis theme, available under license \LaTeX{} Project
        Public License~1.3c at: \colorhref{https://github.com/matze/mtheme}
        {https://github.com/matze/mtheme}.

        Typographic design: Iwona Grabska-Gradzińska;
        \LaTeX{} Typesetting: Kamil Ziemian \\
        Proofreading: Wojciech Palacz,
        Monika Stawicka \\
        3D Models: Dariusz Frymus, Kamil Nowakowski \\
        Figures and charts: Kamil Ziemian, Paweł Węgrzyn, Wojciech
        Palacz \\
        Film editing: IMAVI -- Joanna Kozakiewicz, Krzysztof Magda, Nikodem
        Frodyma

      \end{flushleft}

    \end{textblock}

  \end{frame}
}



\newcommand{\GeometryThreeDSpecialEndingSlideVideoVerThreeEN}{%
  \begin{frame}[standout]

    \begin{textblock}{11}(1,0.7)

      \begin{flushleft}

        \mdseries

        \footnotesize

        \color{jFrametitleFGColor}

        This content was created as part of a project co-financed by the
        European Union within the framework of the European Social Fund
        POWR.03.05.00-00-Z309/17-00.

      \end{flushleft}

    \end{textblock}





    \begin{textblock}{10}(0,2.2)

      \tikz \fill[color=jBackgroundStyleLight] (0,0) rectangle (12.8,-1.5);

    \end{textblock}


    \begin{textblock}{3.2}(0.7,2.45)

      \includegraphics[scale=0.3]{\FundingLogoColorPictureEN}

    \end{textblock}


    \begin{textblock}{2.5}(4.15,2.5)

      \includegraphics[scale=0.2]{\JULogoColorPictureEN}

    \end{textblock}


    \begin{textblock}{2.5}(6.35,2.4)

      \includegraphics[scale=0.1]{\ZintegrUJLogoColorPictureEN}

    \end{textblock}


    \begin{textblock}{4.2}(8.4,2.6)

      \includegraphics[scale=0.3]{\EUSocialFundLogoColorPictureEN}

    \end{textblock}





    \begin{textblock}{11}(1,4)

      \begin{flushleft}

        \mdseries

        \footnotesize

        \RaggedRight

        \color{jFrametitleFGColor}

        The content of this lecture is made available under a Creative
        Commons licence (\textsc{cc}), giving the author the credits
        (\textsc{by}) and putting an obligation to share on the same terms
        (\textsc{sa}). Figures and diagrams included in the lecture are
        authored by Paweł Węgrzyn et~al., and are available under the same
        license unless indicated otherwise.\\ The presentation uses the
        Beamer Jagiellonian theme based on Matthias Vogelgesang’s
        Metropolis theme, available under license \LaTeX{} Project
        Public License~1.3c at: \colorhref{https://github.com/matze/mtheme}
        {https://github.com/matze/mtheme}.

        Typographic design: Iwona Grabska-Gradzińska;
        \LaTeX{} Typesetting: Kamil Ziemian \\
        Proofreading: Wojciech Palacz,
        Monika Stawicka \\
        3D Models: Dariusz Frymus, Kamil Nowakowski \\
        Figures and charts: Kamil Ziemian, Paweł Węgrzyn, Wojciech
        Palacz \\
        Film editing: Agencja Filmowa Film \& Television Production~--
        Zbigniew Masklak \\
        Film editing: IMAVI -- Joanna Kozakiewicz, Krzysztof Magda, Nikodem
        Frodyma

      \end{flushleft}

    \end{textblock}

  \end{frame}
}



\newcommand{\GeometryThreeDTwoSpecialEndingSlidesVideoVerOneEN}[1]{%
  \begin{frame}[standout]

    \begin{textblock}{11}(1,0.7)

      \begin{flushleft}

        \mdseries

        \footnotesize

        \color{jFrametitleFGColor}

        This content was created as part of a project co-financed by the
        European Union within the framework of the European Social Fund
        POWR.03.05.00-00-Z309/17-00.

      \end{flushleft}

    \end{textblock}





    \begin{textblock}{10}(0,2.2)

      \tikz \fill[color=jBackgroundStyleLight] (0,0) rectangle (12.8,-1.5);

    \end{textblock}


    \begin{textblock}{3.2}(0.7,2.45)

      \includegraphics[scale=0.3]{\FundingLogoColorPictureEN}

    \end{textblock}


    \begin{textblock}{2.5}(4.15,2.5)

      \includegraphics[scale=0.2]{\JULogoColorPictureEN}

    \end{textblock}


    \begin{textblock}{2.5}(6.35,2.4)

      \includegraphics[scale=0.1]{\ZintegrUJLogoColorPictureEN}

    \end{textblock}


    \begin{textblock}{4.2}(8.4,2.6)

      \includegraphics[scale=0.3]{\EUSocialFundLogoColorPictureEN}

    \end{textblock}





    \begin{textblock}{11}(1,4)

      \begin{flushleft}

        \mdseries

        \footnotesize

        \RaggedRight

        \color{jFrametitleFGColor}

        The content of this lecture is made available under a Creative
        Commons licence (\textsc{cc}), giving the author the credits
        (\textsc{by}) and putting an obligation to share on the same terms
        (\textsc{sa}). Figures and diagrams included in the lecture are
        authored by Paweł Węgrzyn et~al., and are available under the same
        license unless indicated otherwise.\\ The presentation uses the
        Beamer Jagiellonian theme based on Matthias Vogelgesang’s
        Metropolis theme, available under license \LaTeX{} Project
        Public License~1.3c at: \colorhref{https://github.com/matze/mtheme}
        {https://github.com/matze/mtheme}.

        Typographic design: Iwona Grabska-Gradzińska;
        \LaTeX{} Typesetting: Kamil Ziemian \\
        Proofreading: Wojciech Palacz,
        Monika Stawicka \\
        3D Models: Dariusz Frymus, Kamil Nowakowski \\
        Figures and charts: Kamil Ziemian, Paweł Węgrzyn,
        Wojciech Palacz \\
        Film editing: Agencja Filmowa Film \& Television Production~--
        Zbigniew Masklak

      \end{flushleft}

    \end{textblock}

  \end{frame}





  \begin{frame}[standout]


    \begingroup

    \color{jFrametitleFGColor}

    #1

    \endgroup

  \end{frame}
}



\newcommand{\GeometryThreeDTwoSpecialEndingSlidesVideoVerTwoEN}[1]{%
  \begin{frame}[standout]

    \begin{textblock}{11}(1,0.7)

      \begin{flushleft}

        \mdseries

        \footnotesize

        \color{jFrametitleFGColor}

        This content was created as part of a project co-financed by the
        European Union within the framework of the European Social Fund
        POWR.03.05.00-00-Z309/17-00.

      \end{flushleft}

    \end{textblock}





    \begin{textblock}{10}(0,2.2)

      \tikz \fill[color=jBackgroundStyleLight] (0,0) rectangle (12.8,-1.5);

    \end{textblock}


    \begin{textblock}{3.2}(0.7,2.45)

      \includegraphics[scale=0.3]{\FundingLogoColorPictureEN}

    \end{textblock}


    \begin{textblock}{2.5}(4.15,2.5)

      \includegraphics[scale=0.2]{\JULogoColorPictureEN}

    \end{textblock}


    \begin{textblock}{2.5}(6.35,2.4)

      \includegraphics[scale=0.1]{\ZintegrUJLogoColorPictureEN}

    \end{textblock}


    \begin{textblock}{4.2}(8.4,2.6)

      \includegraphics[scale=0.3]{\EUSocialFundLogoColorPictureEN}

    \end{textblock}





    \begin{textblock}{11}(1,4)

      \begin{flushleft}

        \mdseries

        \footnotesize

        \RaggedRight

        \color{jFrametitleFGColor}

        The content of this lecture is made available under a Creative
        Commons licence (\textsc{cc}), giving the author the credits
        (\textsc{by}) and putting an obligation to share on the same terms
        (\textsc{sa}). Figures and diagrams included in the lecture are
        authored by Paweł Węgrzyn et~al., and are available under the same
        license unless indicated otherwise.\\ The presentation uses the
        Beamer Jagiellonian theme based on Matthias Vogelgesang’s
        Metropolis theme, available under license \LaTeX{} Project
        Public License~1.3c at: \colorhref{https://github.com/matze/mtheme}
        {https://github.com/matze/mtheme}.

        Typographic design: Iwona Grabska-Gradzińska;
        \LaTeX{} Typesetting: Kamil Ziemian \\
        Proofreading: Wojciech Palacz,
        Monika Stawicka \\
        3D Models: Dariusz Frymus, Kamil Nowakowski \\
        Figures and charts: Kamil Ziemian, Paweł Węgrzyn,
        Wojciech Palacz \\
        Film editing: IMAVI -- Joanna Kozakiewicz, Krzysztof Magda, Nikodem
        Frodyma

      \end{flushleft}

    \end{textblock}

  \end{frame}





  \begin{frame}[standout]


    \begingroup

    \color{jFrametitleFGColor}

    #1

    \endgroup

  \end{frame}
}



\newcommand{\GeometryThreeDTwoSpecialEndingSlidesVideoVerThreeEN}[1]{%
  \begin{frame}[standout]

    \begin{textblock}{11}(1,0.7)

      \begin{flushleft}

        \mdseries

        \footnotesize

        \color{jFrametitleFGColor}

        This content was created as part of a project co-financed by the
        European Union within the framework of the European Social Fund
        POWR.03.05.00-00-Z309/17-00.

      \end{flushleft}

    \end{textblock}





    \begin{textblock}{10}(0,2.2)

      \tikz \fill[color=jBackgroundStyleLight] (0,0) rectangle (12.8,-1.5);

    \end{textblock}


    \begin{textblock}{3.2}(0.7,2.45)

      \includegraphics[scale=0.3]{\FundingLogoColorPictureEN}

    \end{textblock}


    \begin{textblock}{2.5}(4.15,2.5)

      \includegraphics[scale=0.2]{\JULogoColorPictureEN}

    \end{textblock}


    \begin{textblock}{2.5}(6.35,2.4)

      \includegraphics[scale=0.1]{\ZintegrUJLogoColorPictureEN}

    \end{textblock}


    \begin{textblock}{4.2}(8.4,2.6)

      \includegraphics[scale=0.3]{\EUSocialFundLogoColorPictureEN}

    \end{textblock}





    \begin{textblock}{11}(1,4)

      \begin{flushleft}

        \mdseries

        \footnotesize

        \RaggedRight

        \color{jFrametitleFGColor}

        The content of this lecture is made available under a Creative
        Commons licence (\textsc{cc}), giving the author the credits
        (\textsc{by}) and putting an obligation to share on the same terms
        (\textsc{sa}). Figures and diagrams included in the lecture are
        authored by Paweł Węgrzyn et~al., and are available under the same
        license unless indicated otherwise. \\ The presentation uses the
        Beamer Jagiellonian theme based on Matthias Vogelgesang’s
        Metropolis theme, available under license \LaTeX{} Project
        Public License~1.3c at: \colorhref{https://github.com/matze/mtheme}
        {https://github.com/matze/mtheme}.

        Typographic design: Iwona Grabska-Gradzińska;
        \LaTeX{} Typesetting: Kamil Ziemian \\
        Proofreading: Leszek Hadasz, Wojciech Palacz,
        Monika Stawicka \\
        3D Models: Dariusz Frymus, Kamil Nowakowski \\
        Figures and charts: Kamil Ziemian, Paweł Węgrzyn,
        Wojciech Palacz \\
        Film editing: Agencja Filmowa Film \& Television Production~--
        Zbigniew Masklak \\
        Film editing: IMAVI -- Joanna Kozakiewicz, Krzysztof Magda, Nikodem
        Frodyma


      \end{flushleft}

    \end{textblock}

  \end{frame}





  \begin{frame}[standout]


    \begingroup

    \color{jFrametitleFGColor}

    #1

    \endgroup

  \end{frame}
}











% ------------------------------------------------------
% BibLaTeX
% ------------------------------------------------------
% Package biblatex, with biber as its backend, allow us to handle
% bibliography entries that use Unicode symbols outside ASCII.
\usepackage[
language=polish,
backend=biber,
style=alphabetic,
url=false,
eprint=true,
]{biblatex}

\addbibresource{Podstawy-informatyki-ETC-Bibliography.bib}





% ------------------------------------------------------
% Importing packages, libraries and setting their configuration
% ------------------------------------------------------




% ------------------------------
% Local packages
% ------------------------------
\usepackage{./Local-packages/local-settings}





% ------------------------------------------------------
% Special configuration for this particular presentation
% ------------------------------------------------------










% ------------------------------------------------------------------------------------------------------------------
\title{Podstawy informatyki z~językiem~C}
\subtitle{Wprowadzenie do~kursu}

\author{Kamil Ziemian}


% \date{}
% ------------------------------------------------------------------------------------------------------------------










% ####################################################################
% Beginning of the document
\begin{document}
% ####################################################################





% ######################################
% Text is adjusted to the left and words are broken at the end of the line.
% Number of chars: 62k+, 73k+,
\RaggedRight
% ######################################





% ######################################
\maketitle
% ######################################





% ##################
\begin{frame}
  \frametitle{Spis treści}


  \tableofcontents

\end{frame}
% ##################





% ######################################
\section{Informacje ogólne}
% ######################################



% ##################
\begin{frame}
  \frametitle{Informacje wstępne}


  Obawiam~się, że na tych konkretnych zajęciach będzie sporo przynudzania,
  ale nie widzę sposobu, by~tego uniknąć.

  Według mnie to zajęcia są dla studentów, nie studenci dla zajęć. Tak samo
  ja jestem tu dla Państwa, a~nie Państwo dla mnie. W~związku z~tym, ja
  będę Państwa rozliczał tylko i~wyłącznie z~umiejętności i~wiedzy,
  z~niczego innego. Wychodzę bowiem z~założenia, że~Państwo sami najlepiej
  wiedzą, czemu warto poświęcić swój czas. (Choć jak wiadomo, często po
  czasie stwierdzamy, że~nasz wybór mógł być jednak lepszy.)

  Na zajęciach nie tylko można, ale \alert{trzeba i~należy} zadawać pytania
  na dowolne związane z~zajęciami zagadnienia. Nie ma tu pytań zbyt
  elementarnych lub zbyt głupich, są~tylko niezadowalające odpowiedzi.

  Pytania typu „Jaki jest najfajniejszy boss w~grze \textit{Hollow
    Knight}?” musimy jednak zostawić na czas po zajęciach.

\end{frame}
% ##################





% ##################
\begin{frame}
  \frametitle{Informacje wstępne}


  Na tych zajęciach \alert{nie} nauczymy~się jak programować. Jak dobrze
  pójdzie to nauczymy~się podstaw programowania w~języku~C, ale
  programowanie obejmuje tyle zagadnień i~wymaga tyle godzin praktyki,
  iż~nie ma najmniejszych szans, że~się tego wszystkiego tu nauczymy.
  Co, na Państwa nieszczęście, nie oznacza, że~będzie mało materiału.

  Ponieważ tematyka którą poruszamy jest niebanalna, więc mnóstwo rzeczy
  będę musiał bardzo \alert{upraszczać}. Proszę mieć to na uwadze.

  Potrzeba uproszczeń wynika z~dwóch powodów. Po pierwsze, ograniczenia
  czasowe. Wiele z~zagadnień które poruszymy mogłoby być tematem
  semestralnego kursu. (Co gorsza, wiele z~nich \alert{jest} tematem
  semestralnych kursów.) Po drugie, to jest kurs \alert{podstaw}
  informatyki, który ma położyć fundamenty pod Państwa umiejętności
  i~wiedzę. To nie przedmiot by wdawać~się we wszystkie szczegóły. Jeśli
  jednak ktoś chce~się bardziej zagłębić w~temat, to służę po zajęciach
  całą moją osobą.

\end{frame}
% ##################





% ##################
\begin{frame}
  \frametitle{Informacje wstępne}


  Dlaczego zaczynamy naukę od~języka~C? Krótka odpowiedź jest taka,
  że~pomimo tego iż język ten ma już pół wieku~(!) na karku, w~2024 roku
  nasza infrastruktura informatyczna wciąż stoi na języku~C. Niektórzy
  mówią, że~C to król wszystkich języków programowania. Jednak patrząc pod
  kątem względów pedagogicznych, to niekoniecznie jest najlepszy wybór
  na kurs podstaw informatyki, ale siła wyższa mówi, że~mamy jego używać.

  W~internecie można znaleźć wiele artykułów i~blogów jak
  \colorhref{https://wideinfo.org/c-programming-is-still-running-the-world/}{\textit{C~programming is still running the
      world}}
  z~czerwca 2024 roku, których tytuł mówi sam za siebie
  \parencite{Scott-C-programming-is-still-ETC-Ver-2024}.

  Dlaczego korzystamy z~systemu GNU/Linux, a~nie z~znacznie
  popularniejszego wśród normalny ludzi systemu Windows? Bo~oferuje
  znacznie lepsze warunki pracy z~językiem~C.

\end{frame}
% ##################





% ##################
\begin{frame}
  \frametitle{Informacje wstępne}


  Dlaczego zaczynamy naukę od~języka~C? Krótka odpowiedź jest taka,
  że~pomimo tego iż język ten ma już pół wieku~(!) na karku, w~2024 roku
  nasza infrastruktura informatyczna wciąż stoi na języku~C. Niektórzy
  mówią, że~C to król wszystkich języków programowania. Jednak patrząc pod
  kątem względów pedagogicznych, to niekoniecznie jest najlepszy wybór
  na kurs podstaw informatyki, ale siła wyższa mówi, że~mamy jego używać.

  W~internecie można znaleźć wiele artykułów i~blogów jak
  \colorhref{https://wideinfo.org/c-programming-is-still-running-the-world/}{\textit{C~programming is still running the
      world}}
  z~czerwca 2024 roku, których tytuł mówi sam za siebie
  \parencite{Scott-C-programming-is-still-ETC-Ver-2024}.

  Dlaczego korzystamy z~systemu GNU/Linux, a~nie z~znacznie
  popularniejszego wśród normalny ludzi systemu Windows? Bo~oferuje
  znacznie lepsze warunki pracy z~językiem~C.

\end{frame}
% ##################





% ##################
\begin{frame}
  \frametitle{Informacje wstępne}


  % Na tych zajęciach \alert{nie} nauczymy~się jak programować. Jak dobrze
  % pójdzie to nauczymy~się podstaw programowania w~języku~C, ale
  % programowanie obejmuje tyle zagadnień i~wymaga tyle godzin praktyki,
  % iż~nie ma najmniejszych szans, że~się tego wszystkiego tu nauczymy.
  % Co, na Państwa nieszczęście, nie oznacza, że~będzie mało materiału.

  % Ponieważ tematyka którą poruszamy jest niebanalna, więc mnóstwo rzeczy
  % będę musiał bardzo \alert{upraszczać}. Proszę mieć to na uwadze.

  % Potrzeba uproszczeń wynika z~dwóch powodów. Po pierwsze, ograniczenia
  % czasowe. Wiele z~zagadnień które poruszymy mogłoby być tematem
  % semestralnego kursu. (Co gorsza, wiele z~nich \alert{jest} tematem
  % semestralnych kursów.) Po drugie, to jest kurs \alert{podstaw}
  % informatyki, który ma położyć fundamenty pod Państwa wiedzę. To nie
  % miejsce by wdawać~się we wszystkie szczegóły.

  % Dlaczego zaczynamy naukę od~języka~C? Krótka odpowiedź jest taka,
  % że~pomimo już pół wieku~(!) na karku, w~2024 roku nasza infrastruktura
  % informatyczna wciąż stoi na języku~C. Acz ze względów pedagogicznych,
  % to niekoniecznie jest najlepszy wybór, ale siła wyższa mówi, że~mamy
  % przy nim pozostać.

  % W~internecie można znaleźć wiele artykułów i~blogów jak
  % \colorhref{https://wideinfo.org/c-programming-is-still-running-the-world/}{\textit{C~programming is still running the
  %     world}}
  %     z~czerwca 2024 roku.

  \alert{Ważne.} Jeśli mają Państwo jakiekolwiek problemy z~systemem
  GNU/Linux to proszę o~tym \alert{mówić}. Nie przyjmuję założenia,
  że~Państw mają już być ekspertami, gdy chodzi o~ten często toporny system
  operacyjny.

  \alert{Ważne.} Wykład ma bardziej charakter teoretyczny, te laboratoria
  zaś praktyczny. Niemniej wciąż poszukujemy optymalnej formy prowadzenia
  tych zajęć. Naprawdę praktyczne zajęcia z~podstaw informatyki w~języku~C
  są trudne do zorganizowania, z~powodów o~których będziemy mówili
  później.

\end{frame}
% ##################





% ##################
\begin{frame}
  \frametitle{Materiały do nauki}


  Prezentacje te są dostępne w~formie plików \LaTeX a (kodu źródłowego)
  na serwisie GitHub. Każdy kto ma na komputerze program Git i~dostęp
  do internetu może jest zdobyć wpisując \\
  \texttt{\$ git clone https://github.com/KZiemian/Presentation} \\
  Znajdują~się one w~katalogu „Podstawy-informatyki-ETC-Prezentacje”.

  Będą też dostępne w~formie PDFów na~Sake, wraz z~innymi materiałami
  do nauki.

\end{frame}
% ##################





% ##################
\begin{frame}
  \frametitle{Zgłaszanie błędu i~uwag}


  W~razie znalezienia jakiegokolwiek błędu lub jakichkolwiek uwag
  merytorycznych do zajęć lub dostępnych materiałów proszę pisać pod adres
  \email. Chcemy by te zajęcia i~towarzyszące im materiały były możliwie
  proste, łatwe w~zrozumieniu i~pozbawione błędów. Proszę jednak uwierzyć,
  że~osiągnięcie tego jest trudniejsze niżby~się wydawało.

\end{frame}
% ##################










% ######################################
\section{Czy informatyka jest trudna?}
% ######################################



% ##################
\begin{frame}
  \frametitle{Czy informatyka jest trudna?}


  Ten przedmiot dotyczy podstaw informatyki w~języku~C, warto~się
  więc spytać, czy informatyka jest prosta czy trudna w~nauce?

  Informatyka to osobna dziedzina nauki i~jeśli zabrnie~się odpowiednio
  głęboko, to robi~się naprawdę złożona i~niebanalna. Jednak na stosunkowo
  płytkim poziomie to czy jest on trudna czy nie, to mocno zależy od~odczuć
  konkretnej osoby.

  Zadam takie pytanie: czy włączenie komputera jest skomplikowane?
  Odpowiemy na to pytanie na dwóch poziomach. Pierwszy to poziom normalnego
  użytkownika, drugi to opis pochodzący z~książki Andrewa S.~Tanenbauma
  \textit{Systemy operacyjne. Wydanie~III}
  \parencite{Tannenbaum-Systemy-Operacyjne-Wydanie-III-Pub-2013}, dotyczący
  komputera z~systemem Pentium.

\end{frame}
% ##################





% ##################
\begin{frame}
  \frametitle{Włączanie komputera, poziom normalnego użytkownika}


  \begin{enumerate}

  \item Wciskamy przycisk \texttt{Power}.



  \item Czekamy minutę albo dłużej.



  \item Wybieramy użytkownika i~wchodzimy na swoje konto.

  \end{enumerate}

  Co w~tym trudnego?

\end{frame}
% ##################





% ##################
\begin{frame}
  \frametitle{Kilka pojęcia}


  Oczywiście, opis włączania komputera z~książki Tanenbauma jest tak
  skomplikowany, że~trzeba wprowadzić trochę pojęć wstępnych.

  \textbf{\textsc{rom}}, ang.~\textit{Read Only Memory}, pl.~\textit{pamięć
    wyłącznie do~odczytu}. Pamięć komputera której zawartość została
  zapisana przez firmę, która ten fragment pamięci wyprodukowała
  i~użytkownik nie może zmodyfikować jej teści. Przynajmniej nie w~żaden
  normalny sposób.

  \textbf{\textsc{ram}}, ang.~\textit{Random Access Memory},
  pl.~\textit{pamięć o~dostępie w~trybie losowym}. Pamięć komputera o~tej
  własności, że~jeśli będę w~sposób losowy wybierał elementy tej pamięci,
  to czas odczytania informacje z~każdego jej elementu będzie taki sam.
  Inaczej mówiąc dostęp do dowolnego miejsca tej pamięci zajmuje tyle samo
  czasu.

  Tak naprawdę czas odczytu zależy od tego, w~jakiś sposób pamięć
  \textsc{ram} jest odczytywana, ale jeszcze długo nie będziemy się musieli
  tym przejmować.

\end{frame}
% ##################





% ##################
\begin{frame}
  \frametitle{Kilka pojęcia}


  \textbf{Pamięć ulotna}, ang.~\textit{volatile memory}. Pamięć której
  zawartość jest tracona, gdy przestaje przez nią płynąć prąd. Typowym
  przykładem takiej pamięci jest \textsc{ram}.

  \textbf{Pamięć nieulotna}, ang.~\textit{non-volatile memory}. Pamięć,
  której treść jest zachowana, gdy przez układ przestaje płynąć prąd,
  typowym przykładem jest dysk \textsc{ssd}.

  Żeby skomplikować życie, pamięcią nieulotną nazywa~się także tą pamięć,
  które jest ulotna w~ścisłym sensie, ale ponieważ jest zaopatrzona
  we~własną baterię, jej zawartość jest zachowana również po wyłączeniu
  komputera z~prądu. Bo~niby czemu życie ma być proste?

\end{frame}
% ##################





% ##################
\begin{frame}
  \frametitle{Kilka pojęcia}


  \textbf{Pamięć \textsc{cmos}}, często po prostu \textbf{\textsc{cmos}}.
  Skrót pochodzi od angielskiej nazwy technologi \textit{Complementary
    Metal-Oxide-Semiconductor} (pl.~\textit{komplementarny półprzewodnik
    metalowo-tlenkowy}), w~której ta pamięć jest wykonana. Musi być zasilana
  prądem, by~zachowywała swój stan, ale ponieważ wyposażona jest w~baterię
  klasyfikowana jest jako nieulotna.

  \textbf{\textsc{bios}} ang.~\textit{Basic Input Output System}, pl.
  \textit{podstawowy system wejścia, wyjścia}. Program znajdujący~się
  na płycie głównej komputera, odpowiedzialny między innymi za odczytywanie
  klawiatury, zapisywanie ekranu oraz operacje wejścia-wyjścia dysków.

\end{frame}
% ##################





% ##################
\begin{frame}
  \frametitle{Uruchamianie komputera z~systemem Pentium}


  \begin{itemize}

  \item[1)] Wciskamy przycisk \texttt{Power}.



  \item[2)] Z~płyty głównej ładowany jest program \textsc{bios}. Sprawdza on
    ilość zainstalowanej pamięci \textsc{ram}, czy komputer dysponuje
    klawiaturą i~innymi podstawowymi urządzeniami oraz sprawdza czy
    odpowiadają one w~sposób prawidłowy. W~pierwszej kolejności skanowane
    są magistrale \textsc{isa} (ang. \textit{Industry Standard
      Architecture}) i~\textsc{pci} (ang.~\textit{Peripheral Component
      Interconnect}) w~celu wykrycia podłączonych do nich urządzeń.



  \item[3)] Jeśli do komputera podłączone są inne urządzenia, niż te które
    były dostępne przy jego ostatni uruchomieniu, nowe urządzenia są
    konfigurowane.



  \item[4)] Program \textsc{bios} odczytuje listę tzw. urządzeń rozruchowych
    z~pamięci \textsc{cmos}. Urządzenia rozruchowe to te, które zawierają
    system operacyjny. W~przeszłości były nimi dyskietki, płyty
    \textsc{cd}-\textsc{rom}, \textsc{dvd}, dziś choćby pendriwy
    i~dyski~\textsc{ssd}.

  \end{itemize}

\end{frame}
% ##################





% ##################
\begin{frame}
  \frametitle{Uruchamianie komputera z~systemem Pentium}


  \begin{itemize}

  \item[5)] \textsc{bios} testuje po kolei urządzenia rozruchowe
    z~wspomnianej wcześniej listy, aż~znajdzie pierwsze, który zawiera
    działający system operacyjny.



  \item[6)] \textsc{bios} wczytuje pierwszy sektor ze~znalezionego
    w~poprzednim punkcie urządzenia rozruchowego do pamięci i~go uruchamia.




  \item[7)] Program z~pierwszego sektora sprawdza zapisaną na jego końcu
    listę partycji, by~ustalić która z~nich jest partycją aktywną.
    Następnie wczytuje z~tej partycji pomocniczy program rozruchowy.



  \item[8)] Pomocniczy program rozruchowy wczytuje system operacyjny
    z~aktywnej partycji i~go uruchamia.



  \item[9)] System operacyjny odczytuje informacje konfiguracyjne z~systemu
    \textsc{bios}. Dla każdego dostępnego urządzenia sprawdza, czy posiada
    do niego sterowniki. Jeśli nie, to prosi o~ich zainstalowanie
    z~odpowiedniego źródła.

  \end{itemize}

\end{frame}
% ##################





% ##################
\begin{frame}
  \frametitle{Uruchamianie komputera z~systemem Pentium}


  \begin{itemize}

  \item[10)] Jeśli system operacyjny dysponuje wszystkimi sterownikami,
    to ładuje je do jądra systemu.



  \item[11)] System operacyjny tworzy tabele systemowe oraz procesy
    działające w~tle.



  \item[12)] Uruchamiane jest okno logowania.

  \end{itemize}

\end{frame}
% ##################






% ##################
\begin{frame}
  \frametitle{Bootowanie}


  W~literaturze funkcjonuje termin \textbf{bootwoanie}, zwane też
  \textbf{uruchamianiem} lub \textbf{rozruchem}. Odnosi~się ono albo do
  całej procedury uruchamiania komputer opisanej powyżej, albo tylko
  stawiania systemu operacyjnego, czyli od kiedy \textsc{bios} wczytał
  pierwszy jego sektor do pamięci (punkt siedem i~dalej). Acz to pojęcie
  nie jest specjalnie ostro zdefiniowane.

\end{frame}
% ##################





% ##################
\begin{frame}
  \frametitle{Czy uruchomienie komputera jest proste czy trudne?}


  Zależy jak do tego podchodzimy. I~tak jest z~większością rzeczy
  w~informatyce.

\end{frame}
% ##################










% ######################################
\section{Podstawowe informacje o~języku~C}
% ######################################



% ##################
\begin{frame}
  \frametitle{Przesadnie szczegółowa lista featurów języka~C}


  \begin{itemize}

  \item[1)] Zmienne.



  \item[2)] Typy zmiennych: \texttt{char}, \texttt{short int}, \texttt{int},
    \texttt{long int}, \texttt{float}, \ldots


  \item[3)] Operator przypisania: \texttt{=}.


  \item[3)] Operator arytmetyczne: \texttt{+}, \texttt{-}, \texttt{*},
    \texttt{/}, \texttt{\%}.
    (Jeśli Państwo myślą, że~to banalnie proste, to~się Państwo
    nieprzyjemnie zdziwią.)



  \item[4)] Operatory porównania: \texttt{==} (nie pomylić z~operatorem
    przypisania~\texttt{=}),



  \item[3)] Instrukcje warunkowe: \texttt{if \ldots{} else \ldots}, \texttt{switch}.



  \item[4)] Pętle: \texttt{while}, \texttt{for}, \texttt{do \ldots{} while}.

  \end{itemize}

\end{frame}
% ##################





% ##################
\begin{frame}
  \frametitle{?????}


  Nie potrafię teraz powiedzieć, czy zdążymy, choćby bardzo pobieżnie,
  przerobić \alert{wszystkie} featury języka~C. Proszę pamiętać, że~to jest
  kurs \alert{podstaw informatyki}, więc nie należy~się spodziewać,
  że~nauczymy~się tu wszystkiego. Nawet o~podstawach informatyki.

\end{frame}
% ##################









% ######################################
\section{Informatyka i~programowanie}
% ######################################



% ##################
\begin{frame}
  \frametitle{Czym jest informatyk?}


  \textbf{Informatyka} (\textsc{cs}, ang. \textit{computer science}) to
  nauka o~rozwiązywaniu problemów za~pomocą komputera. Niektórzy mogą
  uważać tą definicję, że~obrazę informatyki, według mnie jest bardzo dobra.

  Za pomocą komputera możemy próbować rozwiązać wiele różnych problemów.
  Np.~potrzebuję przesłać komuś informację (email), potrzebuję obliczyć
  przybliżoną wartość pewnej liczby $\pi$ lub $\gamma$ (metody numeryczne),
  chciałbym pograć w~coś fajnego (gry wideo), etc. Powstaje jednak pytania.
  Czym~się różni komputer od przykładowo konsoli do gier takiej jak
  PlayStation? Przecież ona też pozwala nam rozwiązać jeden z~tych
  problemów.

  Komputer to \textbf{programowalne urządzenie elektroniczne},
  w~przeciwieństwie do PlayStation, która jest urządzeniem elektroniczny,
  ale \textit{nie} jest programowalna. Co to jednak znaczy?

\end{frame}
% ##################





% ##################
\begin{frame}
  \frametitle{Co to znaczy „programowalny”?}


  Każde urządzenie elektroniczne działa w~ten sposób, że~gdy płynie przez
  nie prąd wykonuje pewne instrukcje. To jakie są instrukcje wynika
  z~tego jak wewnątrz urządzenia „podpięte są kable”. Jeśli jesteśmy
  w~stanie wywołać dowolną instrukcję jaką dysponuje to urządzenie, to
  mówimy, że~jest ono \textbf{programowalne}. Jeśli mogę wywołać tylko
  część możliwych instrukcji, tą część którą udostępnił mi producent
  urządzenia, to wówczas takie urządzenie \textbf{nie jest programowalne}.

  Ta definicja nie jest bardzo prepozycyjna, ale powinna dać nam wyczucie
  o~co chodzi. Komputer daje nam dostęp do wszystkich swoich instrukcji,
  co objawia~się choćby tym, że mogę na nim instalować dowolne
  oprogramowanie, podczas gdy na~PlayStation mogę mieć tylko oprogramowanie
  jakie uprzednio zainstalowała tam firma \textsc{sony}.

\end{frame}
% ##################





% ##################
\begin{frame}
  \frametitle{Programowalne urządzenie elektroniczne}


  Przykładowo konsolę \textsc{sony} można zhakować, uzyskując pełną
  kontrolę nad nią. Jak to zrobić można~się choćby dowiedzieć z~nagrania
  kanału Low Level Learning
  \colorhref{https://www.youtube.com/watch?v=7OwdCc81zHo}
  {\textit{The new PS4 jailbreak is sort of hilarious}}
  \parencite{Low-Level-Learning-The-new-PS4-jailbreak-ETC-2024}.

  Pomijając ten i~nieskończoną ilość innych niuansów, nie powinno być
  trudne w~zrozumieniu co to znaczy, że~komputer jest programowalnym
  urządzeniem elektronicznym.

  \textbf{Program komputerowy} to zestaw instrukcji danego komputera, które
  przekazujemy mu, by je wykonał. Tutaj pojawia~się potrzeba wprowadzenia
  języków programowania. Nie będziemy podawać definicji języka
  programowania, zamiast tego zajmiemy~się tym jak o~nich myśleć.

\end{frame}
% ##################





% ##################
\begin{frame}
  \frametitle{Czym jest język programowania?}


  Dwie metafory które są wygodne w~opisie tego czym jest język
  programowania i~które ja lubię, to metafora języka właśnie i~skrzynki
  z~narzędziami.

  Jeśli chcę by człowiek coś zrobił, muszę mu to przekazać w~języku który
  on rozumiem. Przykładowo ze względu na to jak zbudowany jest język
  angielski i~język japoński, ich wzajemna nauka dla użytkowników tych
  języków jest niezwykle trudna. Jeśli więc osobą której chcę zlecić
  zrobienie czegoś jest Japończyk, może~się zdarzyć, iż będę musiał
  przekazać mu informację po japońsku.

  Oczywiście człowiek może~się nie zgodzić, wykręcać, etc. Komputer to
  tępa maszyna więc pewnie ślepo wykona wszystko co mu zleciliśmy. Ktoś
  może powiedzieć, że~przecież mamy \textsc{ai}, chata\textsc{gpt}, etc.
  To wszystko są ciekawe sprawy, ale na tym przedmiocie nie mamy
  możliwości~się nimi zająć.

\end{frame}
% ##################





% ##################
\begin{frame}
  \frametitle{Czym jest język programowania?}


  Żeby komputer coś zrobił musimy więc mu przekazać polecenie w~języku,
  który rozumie. Dla mnie pierwotny językiem jest język polski, co zaś
  jest pierwotnym językiem komputera? Na potrzeby tego kursu przyjmiemy,
  że~pierwotnym językiem komputera jest \textbf{język assembler}.

  Tak jak niektórzy ludzie mają jako swój pierwotny język angielski,
  hiszpański, japoński, niemiecki, polski, etc., tak komputer również
  posługują~się różnymi rodzajami języka assembler, takimi jak \textsc{arm}
  czy x86/Intel. Generalnie typ assemblera jest ustalony przez firmę, która
  wyprodukowała procesor, bowiem assembler jest zdefiniowany przez to jak są
  „podpięte kable” w~rzeczonym procesorze.

  I~tutaj pojawia~się dwa poważne problem.

\end{frame}
% ##################





% ##################
\begin{frame}
  \frametitle{Assembler, jaki problem?}


  Większość ludzi zrobi wszystko, by tylko uniknąć pracy w~assemblerze.
  A~nawet jeśli ktoś lubi pisać w~assemblerze, to w~praktyce unika~się
  tego jak tylko~się da, bo w~assemblerze nawet najlepsi zbyt często
  popełniają błędy, co może prowadzić do tragicznych konsekwencji.

  Wynika to z~tego, że~assembler to język naturalny dla \alert{komputera},
  ale nie dla człowieka. Ponieważ zaś, nie jest to wielkie odkrycie,
  człowiek mocno~się różni od komputer, język naturalny dla jednego z~nich
  jest zupełnie nienaturalny dla drugiego. Dla nas assembler jest bardzo
  nienaturalny, komputer nie potrafi zaś sobie radzić z~angielskim, czy
  polskim.

  Żeby nie być gołosłownym, przedstawię teraz przykład programu, który
  jedyne co robi to wypisuje gdzieś na ekranie tekst „Hello, World!”.
  Podany zostanie odpowiedni kod w~assemblerze x86 w~wersji Nasm, języku~C
  i~języku Python.

\end{frame}
% ##################





% ##################
\begin{frame}
  \frametitle{„Hello, World!” w~assemblerze x86, Nasm
    \parencite{Anonymous-Hello-World-in-x86-Assembly-Language}}


  \texttt{org 0x100} \\
  \vspace{0.8em}
  \texttt{mov dx, msg} \\
  \texttt{mov ah, 9} \\
  \texttt{int 0x21} \\
  \vspace{0.8em}
  \texttt{mov ah, 0x4c} \\
  \texttt{int 0x21} \\
  \vspace{0.8em}
  \texttt{msg db 'Hello, World!', 0x0d, 0x0a, '\$'}

\end{frame}
% ##################





% ##################
\begin{frame}
  \frametitle{„Hello, World!” w~języku~C}


  \texttt{\#include <stdio.h>} \\
  \vspace{0.8em}
  \texttt{int main() \{ } \\
  \hphantom{aaaa} \texttt{printf("Hello, World!\textbackslash n");} \\
  \vspace{0.8em}
  \vspace{0.8em}
  \vspace{0.8em}
  \vspace{0.8em}
  \hphantom{aaaa} \texttt{return 0;} \\
  \texttt{ \} }

\end{frame}
% ##################





% ##################
\begin{frame}
  \frametitle{„Hello, World!” w~języku~Python}


  \texttt{print("Hello, World!")}

\end{frame}
% ##################




% ##################
\begin{frame}
  \frametitle{Co z~tego wynika?}


  Powinno być teraz dość oczywiste, że~C jest prostszy od~assemblera,
  a~Python prostszy od~C.

  Czy od tego momentu kursu możemy uznać, że~język assemblera nas nie
  interesuje? Jeśli ktoś z~Państwa chce~się dowiedzieć więcej o~assemblerze,
  to bardzo mnie to cieszy, ale musimy to przełożyć na czas po zajęciach.

  Dobrze, ale czemu w~takim wypadku ten kurs jest prowadzony w~C, a~nie
  w~Pythonie? Gdybym ja miał na to wpływ, to może byłby prowadzony
  w~Pythonie, ale ja o~tym nie decyduje.

\end{frame}
% ##################





% ##################
\begin{frame}
  \frametitle{W~jaki sposób działają~C i~Python?}


  Wróćmy do przykładu z~Japończykiem nieznającym angielskiego. Przyjmijmy,
  że~ja nie znam angielskiego, a~on polskiego (po kiego Japończykowi język
  polski?). Czy nie ma szans, żeby przekazał mu polecenie? Oczywiście,
  że~jest. Wystarczy, że~spiszę odpowiednią informację na kartce, użyję
  translatora by przetłumaczyć ją na japoński i~mu ją dam.

  Tak samo postępujemy z~komputerem. W~teorii języków programowania
  \textbf{translatorem} nazywamy program który tłumaczy między dwoma
  \alert{dowolnymi} językami programowania. Mogę więc napisać program
  w~języku~C następnie użyć odpowiedniego translatora by przetłumaczyć
  go na assembler, który następnie zostanie wykonany przez komputer.

  Koniec końców każdy język programowania, czy to~C, czy Python,
  \alert{musi} zostać przetłumaczony na odpowiedni kod assemblera,
  by~komputer go wykonał. To jednak wykracza poza temat zajęć.

\end{frame}
% ##################





% ##################
\begin{frame}
  \frametitle{Typy translatorów}


  Istnieje wiele różnych rodzai translatorów. Przykładowo translatory
  używane dla języka~C noszą nazwę \textbf{kompilatorów}, zaś te stosowane
  dla języka Python \textbf{interpreterów}. To jednak bardzo złożona
  dziedzina informatyki, która od mniej więcej w~roku 2000 roku wkroczył
  w~nową, wciąż trwającą fazę rozwoju, więc tutaj ciągle~się coś zmienia.

  Na razie wszystko co muszą Państwo wiedzieć o~kompilatorach sprowadza~się
  do tego jak uruchomić program napisany w~języku~C, co~można ująć w~trzech
  punktach.

\end{frame}
% ##################





% ##################
\begin{frame}
  \frametitle{Jak używać kompilatora?}


  Przyjmijmy, że~program napisany w~języku~C znajduje~się w~pliku
  \texttt{source.c} (od ang. \textit{source code}, kod źródłowy), a~program
  wynikowy nazywa~się \texttt{progOut} (od ang. \textit{program out},
  program wyjściowy).





  \begin{enumerate}

  \item W~pliku \texttt{source.c} zapisujemy odpowiedni kod w~języku~C.



  \item Przekazujemy kompilatorowi plik \texttt{source.c}, na podstawie
    niego tworzy on program \texttt{progOut}.



  \item Uruchamiamy program wykonywalny \texttt{progOut}.

  \end{enumerate}

\end{frame}
% ##################





% ##################
\begin{frame}
  \frametitle{Jak to wygląda w~praktyce?}


  Przyjmujemy, że~jesteśmy pod systemem GNU/Linux i~wszystkie potrzebne
  programy~są zainstalowane.




  \begin{enumerate}

  \item Otwieramy powłokę \textsc{bash}.



  \item Tworzymy plik \texttt{source.c} za pomocą komendy \\
    \texttt{\$ gedit source.c \&}



  \item Zapisujemy odpowiedni kod programu w~pliku \texttt{source.c}.



  \item Kompilujemy plik \texttt{source.c} przez wpisanie w~terminalu
    polecenia \\
    \texttt{\$ gcc -std=c99 -pedantic source.c -o progOut}



  \item Jeśli wystąpił błąd, poprawiamy kod w~pliku \texttt{source.c}.



  \item Uruchamiamy program \texttt{progOut} wpisując w~terminalu \\
    \texttt{\$ ./progOut}



  \item Jeśli \texttt{progOut} nie działa jak trzeba wracamy do punktu~5.

  \end{enumerate}

\end{frame}
% ##################





% ##################
\begin{frame}
  \frametitle{„Bardzo śmieszne”}


  Teraz sobie pewnie niektórzy z~Państwa myślą: „Bardzo śmieszne. Napisał
  listę siedmiu punktów, z~których każdy brzmi jak magiczne zaklęcia
  z~>>Harry’ego Pottera”. Ja od trzech godzin siedzę nad punktem piątym
  i~powoli tracę zmysły, a~kolega obok jeszcze nie ukończył pierwszego.”.

  Tak może być, to zupełnie normalne. Dlatego ja tu jestem by Państwu pomóc
  zrobić te pierwsze kroki, które prawie zawsze są najtrudniejsze. Dlatego
  jeśli coś nie działa, proszę mówić, postaram~się pomóc.

\end{frame}
% ##################















% % ##################
% \begin{frame}
%   \frametitle{?????}


% \end{frame}
% % ##################





% % ##################
% \begin{frame}
%   \frametitle{}



% \end{frame}
% % ##################










% % ##################
% \begin{frame}
%   \frametitle{???}




% \end{frame}
% % ##################





% % ##################
% \begin{frame}
%   \frametitle{??????}




% \end{frame}
% % ##################






% ######################################
\section{Kilka słów ogólnie o~językach programowania}
% ######################################


% ##################
\begin{frame}
  \frametitle{Ile istnieje języków programowania?}


  Krótko mówiąc, jest ich tyle, że~chyba nikt na świecie nie jest~się
  w~stanie wszystkich nauczyć. Dobrze, tylko w~takim razie, na których
  powinniśmy~się skupić?

  Dobrym miernikiem ważność języka jest indeks \textsc{tiobe}, które
  stara~się mierzyć popularność wszystkich możliwych języków programowania.
  Indeks ten jest aktualizowany co miesiąc, sami zaś jego twórcy
  zastrzegają, iż~zadanie mierzenia popularności języków jest bardzo trudne
  i~użyte metody pozostawiają duży margines błędu. Niemniej jest to wciąż
  narzędzie którego warto używać.

  Więcej informacji o~indeksie \textsc{tiobe} można znaleźć na stronie \\
  \colorhref{https://www.tiobe.com/tiobe-index/}
  {https://www.tiobe.com/tiobe-index/} i~dostępnych tam odnośnikach.
  My zaś przyjrzymy~się pierwszej dwudziestce języków i~ich zasięgowi.

\end{frame}
% ##################





% ##################
\begin{frame}
  \frametitle{TIOBE indeks, lipiec 2024}


  \begin{itemize}

  \item[1)] Python, 16.21\%.



  \item[2)] C++, 10.34\%.



  \item[3)] C, 9.48\%.



  \item[4)] Java, 8.59\%.



  \item[5)] C\#, 6.72\%.




  \item[6)] JavaScript, 3.79\%.




  \item[7)] Go, 2.19\%.




  \item[8)] Visual Basic, 2.08\%.




  \item[9)] Fortran, 2.05\%.

  \end{itemize}

\end{frame}
% ##################





% % ##################
% \begin{frame}
%   \frametitle{???}



% \end{frame}
% % ##################





% % ##################
% \begin{frame}
%   \frametitle{?????}



% \end{frame}
% % ##################





% ##################
\begin{frame}
  \frametitle{Prosty schemat rozwiązywania problemu\ldots}


  Poniżej prezentujemy prosty schemat, jak wygląda procedura
  rozwiązania pewnego problemu za pomocą programu napisanego w~języku
  programowania~X.


  \begin{enumerate}
    \setlength{\itemsep}{1em}

  \item[1)] Mamy problemy który można rozwiązać za pomocą komputera.



  \item[2)] Znajdujemy sposób jako go rozwiązać. To rozwiązanie możemy
    wymyślić sami, znaleźć w~internecie, zapytać \textsc{ai}, etc.



  \item[3)] Piszemy program komputerowy w~języku~X, który implementuje
    dane rozwiązanie.



  \item[4)] Na podstawie napisanego przez nas kodu tworzymy program
    w~postaci wykonywalnej, zrozumiałej dla~komputera.



  \item[5)] Uruchamiamy program utworzony w~punkcie~4 na konkretnym
    komputerze, tak by rozwiązał nasz problem.

  \end{enumerate}

\end{frame}
% ##################





% ##################
\begin{frame}
  \frametitle{Przykład}


  \begin{itemize}

  \item[1)] Mamy 100 000 nazw użytkowników jakiegoś portalu. Potrzebujemy
    sprawdzić, czy nazwy zgłaszających~się regularnie nowych użytkowników
    są już zajęta czy nie. Spostrzegamy, że~można to zrobić za pomocą
    odpowiedniego programu komputerowego.



  \item[2)] Zauważamy, że~jeśli ustawimy nazwy dotychczasowych użytkowników
    w~posortowany ciąg, to możemy sprawdzić czy dana nazwa w~nim jest
    za~pomocą wydajnego przeszukiwania binarnego. Wymyślamy jak można
    zaprowadzić relację porządku w~zbiorze tych nazw, co jest nam potrzebne
    do~ich posortowania.



  \item[3)] Piszemy program w~języku~X, który pobierze na wejściu nazwy
    użytkowników, następnie je posortuje według zadanego przepisu.
    Następnie, jeśli program ten otrzyma nową nazwę, to sprawdzi za pomocą
    przeszukiwania binarnego, czy jest już ona wśród nazw już zajętych.

  \end{itemize}

\end{frame}
% ##################





% ##################
\begin{frame}
  \frametitle{Przykład}


  \begin{itemize}

  \item[4)] Na~podstawie kodu w~języku~X generujemy odpowiedni program
    w~postaci wykonywalnej.



  \item[5)] Uruchamiamy wygenerowany w~punkcie~4 program na~konkretnym
    komputerze, by rozwiązał nasz problem.

  \end{itemize}

\end{frame}
% ##################





% ##################
\begin{frame}
  \frametitle{Co to jest implementacja?}


  Często znamy ogólną, abstrakcyjną postać rozwiązania danego problemu, czy
  też sposobu działania konkretnego algorytmu. Przykładowo „program
  umożliwiający obliczanie średniej arytmetycznej z~miliona liczb.”, to
  taka ogólny opis programu pewnego programu, który wykonuje potrzebną nam
  czynność. Jeśli jest to potrzebne, możemy podać jego bardziej precyzyjną
  formę. „Dany jest milion liczb. Wylicz ich sumę, następnie podziel ją
  przez liczbę $1 \, 000 \, 000$. Tak otrzymana liczba jest poszukiwanym
  rezultatem.”

  Takie opis programu jest zrozumiały dla ludzi, ale nie dla komputera,
  w~sensie hardwaru. Przy czym zasada ta stosuje~się też, choć w~innym
  zakresie, do~\textsc{ai} i~\textsc{lll}, ale~to temat na osobne dyskusje.
  Z~tego powodu musimy napisać odpowiedni kod, który będzie zrozumiały dla
  komputera, dopiero wtedy on będzie mógł go wykonać.

\end{frame}
% ##################





% ##################
\begin{frame}
  \frametitle{Co to jest implementacja?}


  Jeśli mamy konkretny kod, albo program, który wykonuje dane zadanie
  zgodnie z~ogólną idę, metodą albo algorytmem, to mówimy że~taki kod lub
  program jest \textbf{implementacją} tej idei bądź algorytmu.

  To wyjaśnienie nie jest bardzo precyzyjne, ale też nie musi być.
  Implementacja nie jest pojęciem ważnym dla komputera, tylko dla ludzi,
  stąd od precyzyjnej definicji ważniejsze jest zrozumienie ogólnej idei.

\end{frame}
% ##################





% ##################
\begin{frame}
  \frametitle{Przykład. Postawienie problemu i~metoda rozwiązania}


  \textbf{Problem.} Obliczyć średnią arytmetyczną zadanych $100$ liczb.

  \textbf{Metoda rozwiązania.} Oblicz sumę danych $100$ liczb, następnie
  podziel ją przez $100$.

\end{frame}
% ##################





% ##################
\begin{frame}
  \frametitle{Implementacja w~języku~C}


  \texttt{\#include <stdio.h>} \\
  \vspace{0.8em}
  /* Zakładamy, że~liczby który średnią arytmetyczną mamy obliczyć są
  przechowywane w~$100$-elementowej tablicy liczby typu \texttt{double}
  \hphantom{aa} o~nazwie \texttt{numbersArray}, zdefiniowanej
  w~dostarczonym nam ze-
  \hphantom{aa} wnątrz pliku\texttt{numbersProb.h}. Plik ten ma być w~tym
  samym kata-
  \hphantom{aa} logu co poniższy program. */ \\
  \texttt{\#include "numbersProb.h"} \\
  \vspace{0.8em}
  \texttt{int main() \{ } \\
  \hphantom{aaaa} \texttt{int i = 0;} \\
  \hphantom{aaaa} \texttt{double suma = 0.0;} \\
  \hphantom{aaaa} \texttt{double sredniaArytmetycza = 0.0;}

\end{frame}
% ##################





% ##################
\begin{frame}
  \frametitle{Implementacja w~języku~C. Cd.}


  \hphantom{aaaa} \texttt{for (i = 0; i < 100; i++) \{ } \\
  \hphantom{aaaaaaaa} \texttt{suma += numbersArray[i];} \\
  \hphantom{aaaa} \texttt{ \} } \\
  \vspace{0.8em}
  \hphantom{aaaa} \texttt{sredniaArytmetyczna = suma / 100.0;} \\
  \vspace{0.8em}
  \hphantom{aaaa} \texttt{prinft("sredniaArytmetyczna =
    \%.3f.\textbackslash n", \\
    \hphantom{aaaaaaaaa} sredniaArytmetyczna);} \\
  \texttt{ \} }

\end{frame}
% ##################







% ##################
\begin{frame}
  \frametitle{Implementacje w~języku Python}


  \# Zakładamy, że~liczby których średnią arytmetyczną mamy obli- \\
  \# czyć~są wszystkie typu \texttt{float} i~są przechowywane
  w~$100$-elementowej \\
  \# liście o~nazwie \texttt{numbersList}, która znajduje~się w~dostarczonym
  \# nam module Pythona o~nazwie \texttt{numbersProg}. \\
  \vspace{0.8em}
  \texttt{from numbersProg import numbersList} \\
  \vspace{0.8em}
  \texttt{suma = 0.0} \\
  \vspace{0.8em}
  \texttt{for number in numbersList:} \\
  \hphantom{aaaa} \texttt{suma += number} \\
  \vspace{0.8em}
  \texttt{sredniaArytmetyczna = suma / 100.0} \\
  \texttt{answer = \textbackslash} \\
  \hphantom{aaa}
  \texttt{"sredniaArytmetyczna = \{\}".format(sredniaArytmetyczna)}
  \\
  \vspace{0.8em}
  \texttt{print(answer)}

\end{frame}
% ##################





% % ##################
% \begin{frame}
%   \frametitle{????}



% \end{frame}
% % ##################

















% % ##################
% \begin{frame}
%   \frametitle{}



% \end{frame}
% % ##################





% % ##################
% \begin{frame}
%   \frametitle{?????}




% \end{frame}
% % ##################










% % ##################
% \begin{frame}
%   \frametitle{?????}





% \end{frame}
% % ##################





% % ##################
% \begin{frame}
%   \frametitle{?????}



% \end{frame}
% % ##################





% % ##################
% \begin{frame}
%   \frametitle{?????}



% \end{frame}
% % ##################






% % ##################
% \begin{frame}
%   \frametitle{?????}




% \end{frame}
% % ##################





% % ##################
% \begin{frame}
%   \frametitle{????}




% \end{frame}
% % ##################











% % ##################
% \begin{frame}
%   \frametitle{????}




% \end{frame}
% % ##################





% % ##################
% \begin{frame}
%   \frametitle{?????}



% \end{frame}
% % ##################





% % ##################
% \begin{frame}
%   \frametitle{?????}




% \end{frame}
% % ##################





% % ##################
% \begin{frame}
%   \frametitle{?????}



% \end{frame}
% % ##################





% % ##################
% \begin{frame}
%   \frametitle{??????}



% \end{frame}
% % ##################





% % ##################
% \begin{frame}
%   \frametitle{?????}



% \end{frame}
% % ##################





% % ##################
% \begin{frame}
%   \frametitle{?????}



% \end{frame}
% % ##################





% % ##################
% \begin{frame}
%   \frametitle{????}



% \end{frame}
% % ##################





% % ##################
% \begin{frame}
%   \frametitle{????}



% \end{frame}
% % ##################





% % ##################
% \begin{frame}
%   \frametitle{?????}




% \end{frame}
% % ##################





% % ##################
% \begin{frame}
%   \frametitle{????}



% \end{frame}
% % ##################





% % ##################
% \begin{frame}
%   \frametitle{?????}



% \end{frame}
% % ##################





% % ##################
% \begin{frame}
%   \frametitle{?????}



% \end{frame}
% % ##################





% % ##################
% \begin{frame}
%   \frametitle{?????}




% \end{frame}
% % ##################





% % ##################
% \begin{frame}
%   \frametitle{?????}




% \end{frame}
% % ##################





% % ##################
% \begin{frame}
%   \frametitle{?????}



% \end{frame}
% % ##################





% % ##################
% \begin{frame}
%   \frametitle{????}



% \end{frame}
% % ##################





% % ##################
% \begin{frame}
%   \frametitle{?????}



% \end{frame}
% % ##################





% % ##################
% \begin{frame}
%   \frametitle{????}




% \end{frame}
% % ##################





% % ##################
% \begin{frame}
%   \frametitle{?????}



% \end{frame}
% % ##################





% % ##################
% \begin{frame}
%   \frametitle{?????}





% \end{frame}
% % ##################





% % ##################
% \begin{frame}
%   \frametitle{?????}




% \end{frame}
% % ##################





% % ##################
% \begin{frame}
%   \frametitle{????}




% \end{frame}
% % ##################










% % ######################################
% \appendix
% % ######################################





% % ##################
% \GeometryThreeDTwoSpecialEndingSlidesEN{Questions? Thank you for your attention.}
% % ##################



% % % ##################
% % \jagiellonianendslide{Dziękuję za~uwagę.}
% % % ##################










% % ######################################
% \SectionSlideWithPicture{Terminological notes}
% % ######################################



% % ##################
% \begin{frame}
%   \frametitle{??????}



% \end{frame}
% % ##################





% % ##################
% \begin{frame}
%   \frametitle{?????}



% \end{frame}
% ##################





% ##################
% \begin{frame}
%   \frametitle{?????}



% \end{frame}
% ##################





% ##################
% \begin{frame}
%   \frametitle{?????}



% \end{frame}
% ##################





% ##################
% \begin{frame}
%   \frametitle{?????}



% \end{frame}
%  ##################














% ####################################################################
% ####################################################################
% Bibliography

\printbibliography





% ############################
% End of the document

\end{document}
