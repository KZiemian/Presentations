% ------------------------------------------------------------------------------------------------------------------
% Basic configuration of Beamera class and Jagiellonian theme
% ------------------------------------------------------------------------------------------------------------------
\RequirePackage[l2tabu, orthodox]{nag}



\ifx\PresentationStyle\notset
  \def\PresentationStyle{dark}
\fi



% Options: t -- align text to the top of the frame
\documentclass[10pt,t]{beamer}
\mode<presentation>
\usetheme[style=\PresentationStyle,frametitlecolorstyle=general,titleframebackgroundcolorstyle=dark]{jagiellonian}





% ------------------------------------------------------------------------------------
% Procesing configuration files of Jagiellonian theme located
% in the directory "preambule"
% ------------------------------------------------------------------------------------
% Configuration for polish language
% Need description
\usepackage[polish]{babel}
% Need description
\usepackage[MeX]{polski}



% ------------------------------
% Better support of polish chars in technical parts of PDF
% ------------------------------
\hypersetup{pdfencoding=auto,psdextra}

% Package "textpos" give as enviroment "textblock" which is very usefull in
% arranging text on slides.

% This is standard configuration of "textpos"
\usepackage[overlay,absolute]{textpos}

% If you need to see bounds of "textblock's" comment line above and uncomment
% one below.

% Caution! When showboxes option is on significant ammunt of space is add
% to the top of textblock and as such, everyting put in them gone down.
% We need to check how to remove this bug.

% \usepackage[showboxes,overlay,absolute]{textpos}



% Setting scale length for package "textpos"
\setlength{\TPHorizModule}{10mm}
\setlength{\TPVertModule}{\TPHorizModule}


% ---------------------------------------
% Packages written for lectures "Geometria 3D dla twórców gier wideo"
% ---------------------------------------
% \usepackage{./Geometry3DPackages/Geometry3D}
% \usepackage{./Geometry3DPackages/Geometry3DIndices}
% \usepackage{./Geometry3DPackages/Geometry3DTikZStyle}
% \usepackage{./ProgramowanieSymulacjiFizykiPaczki/ProgramowanieSymulacjiFizykiTikZStyle}
% \usepackage{./Geometry3DPackages/mathcommands}


% ---------------------------------------
% TikZ
% ---------------------------------------
% Importing TikZ libraries
\usetikzlibrary{arrows.meta}
\usetikzlibrary{positioning}





% % Configuration package "bm" that need for making bold symbols
% \newcommand{\bmmax}{0}
% \newcommand{\hmmax}{0}
% \usepackage{bm}




% ---------------------------------------
% Packages for scientific texts
% ---------------------------------------
% \let\lll\undefined  % Sometimes you must use this line to allow
% "amsmath" package to works with packages with packages for polish
% languge imported
% /preambul/LanguageSettings/JagiellonianPolishLanguageSettings.tex.
% This comments (probably) removes polish letter Ł.
\usepackage{amsmath}  % Packages from American Mathematical Society (AMS)
\usepackage{amssymb}
\usepackage{amscd}
\usepackage{amsthm}
\usepackage{siunitx}  % Package for typsetting SI units.
\usepackage{upgreek}  % Better looking greek letters.
% Example of using upgreek: pi = \uppi


\usepackage{calrsfs}  % Zmienia czcionkę kaligraficzną w \mathcal
% na ładniejszą. Może w innych miejscach robi to samo, ale o tym nic
% nie wiem.










% ---------------------------------------
% Packages written for lectures "Geometria 3D dla twórców gier wideo"
% ---------------------------------------
% \usepackage{./ProgramowanieSymulacjiFizykiPaczki/ProgramowanieSymulacjiFizyki}
% \usepackage{./ProgramowanieSymulacjiFizykiPaczki/ProgramowanieSymulacjiFizykiIndeksy}
% \usepackage{./ProgramowanieSymulacjiFizykiPaczki/ProgramowanieSymulacjiFizykiTikZStyle}





% !!!!!!!!!!!!!!!!!!!!!!!!!!!!!!
% !!!!!!!!!!!!!!!!!!!!!!!!!!!!!!
% EVIL STUFF
\if\JUlogotitle1
\edef\LogoJUPath{LogoJU_\JUlogoLang/LogoJU_\JUlogoShape_\JUlogoColor.pdf}
\titlegraphic{\hfill\includegraphics[scale=0.22]
{./JagiellonianPictures/\LogoJUPath}}
\fi
% ---------------------------------------
% Commands for handling colors
% ---------------------------------------


% Command for setting normal text color for some text in math modestyle
% Text color depend on used style of Jagiellonian

% Beamer version of command
\newcommand{\TextWithNormalTextColor}[1]{%
  {\color{jNormalTextFGColor}
    \setbeamercolor{math text}{fg=jNormalTextFGColor} {#1}}
}

% Article and similar classes version of command
% \newcommand{\TextWithNormalTextColor}[1]{%
%   {\color{jNormalTextsFGColor} {#1}}
% }



% Beamer version of command
\newcommand{\NormalTextInMathMode}[1]{%
  {\color{jNormalTextFGColor}
    \setbeamercolor{math text}{fg=jNormalTextFGColor} \text{#1}}
}


% Article and similar classes version of command
% \newcommand{\NormalTextInMathMode}[1]{%
%   {\color{jNormalTextsFGColor} \text{#1}}
% }




% Command that sets color of some mathematical text to the same color
% that has normal text in header (?)

% Beamer version of the command
\newcommand{\MathTextFrametitleFGColor}[1]{%
  {\color{jFrametitleFGColor}
    \setbeamercolor{math text}{fg=jFrametitleFGColor} #1}
}

% Article and similar classes version of the command
% \newcommand{\MathTextWhiteColor}[1]{{\color{jFrametitleFGColor} #1}}





% Command for setting color of alert text for some text in math modestyle

% Beamer version of the command
\newcommand{\MathTextAlertColor}[1]{%
  {\color{jOrange} \setbeamercolor{math text}{fg=jOrange} #1}
}

% Article and similar classes version of the command
% \newcommand{\MathTextAlertColor}[1]{{\color{jOrange} #1}}





% Command that allow you to sets chosen color as the color of some text into
% math mode. Due to some nuances in the way that Beamer handle colors
% it not work in all cases. We hope that in the future we will improve it.

% Beamer version of the command
\newcommand{\SetMathTextColor}[2]{%
  {\color{#1} \setbeamercolor{math text}{fg=#1} #2}
}


% Article and similar classes version of the command
% \newcommand{\SetMathTextColor}[2]{{\color{#1} #2}}










% ---------------------------------------
% Commands for few special slides
% ---------------------------------------
\newcommand{\EndingSlide}[1]{%
  \begin{frame}[standout]

    \begingroup

    \color{jFrametitleFGColor}

    #1

    \endgroup

  \end{frame}
}










% ---------------------------------------
% Commands for setting background pictures for some slides
% ---------------------------------------
\newcommand{\TitleBackgroundPicture}
{./JagiellonianPictures/Backgrounds/LajkonikDark.png}
\newcommand{\SectionBackgroundPicture}
{./JagiellonianPictures/Backgrounds/LajkonikLight.png}



\newcommand{\TitleSlideWithPicture}{%
  \begingroup

  \usebackgroundtemplate{%
    \includegraphics[height=\paperheight]{\TitleBackgroundPicture}}

  \maketitle

  \endgroup
}





\newcommand{\SectionSlideWithPicture}[1]{%
  \begingroup

  \usebackgroundtemplate{%
    \includegraphics[height=\paperheight]{\SectionBackgroundPicture}}

  \setbeamercolor{titlelike}{fg=normal text.fg}

  \section{#1}

  \endgroup
}










% ---------------------------------------
% Commands for lectures "Geometria 3D dla twórców gier wideo"
% Polish version
% ---------------------------------------
% Komendy teraz wykomentowane były potrzebne, gdy loga były na niebieskim
% tle, nie na białym. A są na białym bo tego chcieli w biurze projektu.
% \newcommand{\FundingLogoWhitePicturePL}
% {./PresentationPictures/CommonPictures/logotypFundusze_biale_bez_tla2.pdf}
\newcommand{\FundingLogoColorPicturePL}
{./PresentationPictures/CommonPictures/European_Funds_color_PL.pdf}
% \newcommand{\EULogoWhitePicturePL}
% {./PresentationPictures/CommonPictures/logotypUE_biale_bez_tla2.pdf}
\newcommand{\EUSocialFundLogoColorPicturePL}
{./PresentationPictures/CommonPictures/EU_Social_Fund_color_PL.pdf}
% \newcommand{\ZintegrUJLogoWhitePicturePL}
% {./PresentationPictures/CommonPictures/zintegruj-logo-white.pdf}
\newcommand{\ZintegrUJLogoColorPicturePL}
{./PresentationPictures/CommonPictures/ZintegrUJ_color.pdf}
\newcommand{\JULogoColorPicturePL}
{./JagiellonianPictures/LogoJU_PL/LogoJU_A_color.pdf}





\newcommand{\GeometryThreeDSpecialBeginningSlidePL}{%
  \begin{frame}[standout]

    \begin{textblock}{11}(1,0.7)

      \begin{flushleft}

        \mdseries

        \footnotesize

        \color{jFrametitleFGColor}

        Materiał powstał w ramach projektu współfinansowanego ze środków
        Unii Europejskiej w ramach Europejskiego Funduszu Społecznego
        POWR.03.05.00-00-Z309/17-00.

      \end{flushleft}

    \end{textblock}





    \begin{textblock}{10}(0,2.2)

      \tikz \fill[color=jBackgroundStyleLight] (0,0) rectangle (12.8,-1.5);

    \end{textblock}


    \begin{textblock}{3.2}(1,2.45)

      \includegraphics[scale=0.3]{\FundingLogoColorPicturePL}

    \end{textblock}


    \begin{textblock}{2.5}(3.7,2.5)

      \includegraphics[scale=0.2]{\JULogoColorPicturePL}

    \end{textblock}


    \begin{textblock}{2.5}(6,2.4)

      \includegraphics[scale=0.1]{\ZintegrUJLogoColorPicturePL}

    \end{textblock}


    \begin{textblock}{4.2}(8.4,2.6)

      \includegraphics[scale=0.3]{\EUSocialFundLogoColorPicturePL}

    \end{textblock}

  \end{frame}
}



\newcommand{\GeometryThreeDTwoSpecialBeginningSlidesPL}{%
  \begin{frame}[standout]

    \begin{textblock}{11}(1,0.7)

      \begin{flushleft}

        \mdseries

        \footnotesize

        \color{jFrametitleFGColor}

        Materiał powstał w ramach projektu współfinansowanego ze środków
        Unii Europejskiej w ramach Europejskiego Funduszu Społecznego
        POWR.03.05.00-00-Z309/17-00.

      \end{flushleft}

    \end{textblock}





    \begin{textblock}{10}(0,2.2)

      \tikz \fill[color=jBackgroundStyleLight] (0,0) rectangle (12.8,-1.5);

    \end{textblock}


    \begin{textblock}{3.2}(1,2.45)

      \includegraphics[scale=0.3]{\FundingLogoColorPicturePL}

    \end{textblock}


    \begin{textblock}{2.5}(3.7,2.5)

      \includegraphics[scale=0.2]{\JULogoColorPicturePL}

    \end{textblock}


    \begin{textblock}{2.5}(6,2.4)

      \includegraphics[scale=0.1]{\ZintegrUJLogoColorPicturePL}

    \end{textblock}


    \begin{textblock}{4.2}(8.4,2.6)

      \includegraphics[scale=0.3]{\EUSocialFundLogoColorPicturePL}

    \end{textblock}

  \end{frame}





  \TitleSlideWithPicture
}



\newcommand{\GeometryThreeDSpecialEndingSlidePL}{%
  \begin{frame}[standout]

    \begin{textblock}{11}(1,0.7)

      \begin{flushleft}

        \mdseries

        \footnotesize

        \color{jFrametitleFGColor}

        Materiał powstał w ramach projektu współfinansowanego ze środków
        Unii Europejskiej w~ramach Europejskiego Funduszu Społecznego
        POWR.03.05.00-00-Z309/17-00.

      \end{flushleft}

    \end{textblock}





    \begin{textblock}{10}(0,2.2)

      \tikz \fill[color=jBackgroundStyleLight] (0,0) rectangle (12.8,-1.5);

    \end{textblock}


    \begin{textblock}{3.2}(1,2.45)

      \includegraphics[scale=0.3]{\FundingLogoColorPicturePL}

    \end{textblock}


    \begin{textblock}{2.5}(3.7,2.5)

      \includegraphics[scale=0.2]{\JULogoColorPicturePL}

    \end{textblock}


    \begin{textblock}{2.5}(6,2.4)

      \includegraphics[scale=0.1]{\ZintegrUJLogoColorPicturePL}

    \end{textblock}


    \begin{textblock}{4.2}(8.4,2.6)

      \includegraphics[scale=0.3]{\EUSocialFundLogoColorPicturePL}

    \end{textblock}





    \begin{textblock}{11}(1,4)

      \begin{flushleft}

        \mdseries

        \footnotesize

        \RaggedRight

        \color{jFrametitleFGColor}

        Treść niniejszego wykładu jest udostępniona na~licencji
        Creative Commons (\textsc{cc}), z~uzna\-niem autorstwa
        (\textsc{by}) oraz udostępnianiem na tych samych warunkach
        (\textsc{sa}). Rysunki i~wy\-kresy zawarte w~wykładzie są
        autorstwa dr.~hab.~Pawła Węgrzyna et~al. i~są dostępne
        na tej samej licencji, o~ile nie wskazano inaczej.
        W~prezentacji wykorzystano temat Beamera Jagiellonian,
        oparty na~temacie Metropolis Matthiasa Vogelgesanga,
        dostępnym na licencji \LaTeX{} Project Public License~1.3c
        pod adresem: \colorhref{https://github.com/matze/mtheme}
        {https://github.com/matze/mtheme}.

        Projekt typograficzny: Iwona Grabska-Gradzińska \\
        Skład: Kamil Ziemian;
        Korekta: Wojciech Palacz \\
        Modele: Dariusz Frymus, Kamil Nowakowski \\
        Rysunki i~wykresy: Kamil Ziemian, Paweł Węgrzyn, Wojciech Palacz

      \end{flushleft}

    \end{textblock}

  \end{frame}
}



\newcommand{\GeometryThreeDTwoSpecialEndingSlidesPL}[1]{%
  \begin{frame}[standout]


    \begin{textblock}{11}(1,0.7)

      \begin{flushleft}

        \mdseries

        \footnotesize

        \color{jFrametitleFGColor}

        Materiał powstał w ramach projektu współfinansowanego ze środków
        Unii Europejskiej w~ramach Europejskiego Funduszu Społecznego
        POWR.03.05.00-00-Z309/17-00.

      \end{flushleft}

    \end{textblock}





    \begin{textblock}{10}(0,2.2)

      \tikz \fill[color=jBackgroundStyleLight] (0,0) rectangle (12.8,-1.5);

    \end{textblock}


    \begin{textblock}{3.2}(1,2.45)

      \includegraphics[scale=0.3]{\FundingLogoColorPicturePL}

    \end{textblock}


    \begin{textblock}{2.5}(3.7,2.5)

      \includegraphics[scale=0.2]{\JULogoColorPicturePL}

    \end{textblock}


    \begin{textblock}{2.5}(6,2.4)

      \includegraphics[scale=0.1]{\ZintegrUJLogoColorPicturePL}

    \end{textblock}


    \begin{textblock}{4.2}(8.4,2.6)

      \includegraphics[scale=0.3]{\EUSocialFundLogoColorPicturePL}

    \end{textblock}





    \begin{textblock}{11}(1,4)

      \begin{flushleft}

        \mdseries

        \footnotesize

        \RaggedRight

        \color{jFrametitleFGColor}

        Treść niniejszego wykładu jest udostępniona na~licencji
        Creative Commons (\textsc{cc}), z~uzna\-niem autorstwa
        (\textsc{by}) oraz udostępnianiem na tych samych warunkach
        (\textsc{sa}). Rysunki i~wy\-kresy zawarte w~wykładzie są
        autorstwa dr.~hab.~Pawła Węgrzyna et~al. i~są dostępne
        na tej samej licencji, o~ile nie wskazano inaczej.
        W~prezentacji wykorzystano temat Beamera Jagiellonian,
        oparty na~temacie Metropolis Matthiasa Vogelgesanga,
        dostępnym na licencji \LaTeX{} Project Public License~1.3c
        pod adresem: \colorhref{https://github.com/matze/mtheme}
        {https://github.com/matze/mtheme}.

        Projekt typograficzny: Iwona Grabska-Gradzińska \\
        Skład: Kamil Ziemian;
        Korekta: Wojciech Palacz \\
        Modele: Dariusz Frymus, Kamil Nowakowski \\
        Rysunki i~wykresy: Kamil Ziemian, Paweł Węgrzyn, Wojciech Palacz

      \end{flushleft}

    \end{textblock}

  \end{frame}





  \begin{frame}[standout]

    \begingroup

    \color{jFrametitleFGColor}

    #1

    \endgroup

  \end{frame}
}



\newcommand{\GeometryThreeDSpecialEndingSlideVideoPL}{%
  \begin{frame}[standout]

    \begin{textblock}{11}(1,0.7)

      \begin{flushleft}

        \mdseries

        \footnotesize

        \color{jFrametitleFGColor}

        Materiał powstał w ramach projektu współfinansowanego ze środków
        Unii Europejskiej w~ramach Europejskiego Funduszu Społecznego
        POWR.03.05.00-00-Z309/17-00.

      \end{flushleft}

    \end{textblock}





    \begin{textblock}{10}(0,2.2)

      \tikz \fill[color=jBackgroundStyleLight] (0,0) rectangle (12.8,-1.5);

    \end{textblock}


    \begin{textblock}{3.2}(1,2.45)

      \includegraphics[scale=0.3]{\FundingLogoColorPicturePL}

    \end{textblock}


    \begin{textblock}{2.5}(3.7,2.5)

      \includegraphics[scale=0.2]{\JULogoColorPicturePL}

    \end{textblock}


    \begin{textblock}{2.5}(6,2.4)

      \includegraphics[scale=0.1]{\ZintegrUJLogoColorPicturePL}

    \end{textblock}


    \begin{textblock}{4.2}(8.4,2.6)

      \includegraphics[scale=0.3]{\EUSocialFundLogoColorPicturePL}

    \end{textblock}





    \begin{textblock}{11}(1,4)

      \begin{flushleft}

        \mdseries

        \footnotesize

        \RaggedRight

        \color{jFrametitleFGColor}

        Treść niniejszego wykładu jest udostępniona na~licencji
        Creative Commons (\textsc{cc}), z~uzna\-niem autorstwa
        (\textsc{by}) oraz udostępnianiem na tych samych warunkach
        (\textsc{sa}). Rysunki i~wy\-kresy zawarte w~wykładzie są
        autorstwa dr.~hab.~Pawła Węgrzyna et~al. i~są dostępne
        na tej samej licencji, o~ile nie wskazano inaczej.
        W~prezentacji wykorzystano temat Beamera Jagiellonian,
        oparty na~temacie Metropolis Matthiasa Vogelgesanga,
        dostępnym na licencji \LaTeX{} Project Public License~1.3c
        pod adresem: \colorhref{https://github.com/matze/mtheme}
        {https://github.com/matze/mtheme}.

        Projekt typograficzny: Iwona Grabska-Gradzińska;
        Skład: Kamil Ziemian \\
        Korekta: Wojciech Palacz;
        Modele: Dariusz Frymus, Kamil Nowakowski \\
        Rysunki i~wykresy: Kamil Ziemian, Paweł Węgrzyn, Wojciech Palacz \\
        Montaż: Agencja Filmowa Film \& Television Production~-- Zbigniew
        Masklak

      \end{flushleft}

    \end{textblock}

  \end{frame}
}





\newcommand{\GeometryThreeDTwoSpecialEndingSlidesVideoPL}[1]{%
  \begin{frame}[standout]

    \begin{textblock}{11}(1,0.7)

      \begin{flushleft}

        \mdseries

        \footnotesize

        \color{jFrametitleFGColor}

        Materiał powstał w ramach projektu współfinansowanego ze środków
        Unii Europejskiej w~ramach Europejskiego Funduszu Społecznego
        POWR.03.05.00-00-Z309/17-00.

      \end{flushleft}

    \end{textblock}





    \begin{textblock}{10}(0,2.2)

      \tikz \fill[color=jBackgroundStyleLight] (0,0) rectangle (12.8,-1.5);

    \end{textblock}


    \begin{textblock}{3.2}(1,2.45)

      \includegraphics[scale=0.3]{\FundingLogoColorPicturePL}

    \end{textblock}


    \begin{textblock}{2.5}(3.7,2.5)

      \includegraphics[scale=0.2]{\JULogoColorPicturePL}

    \end{textblock}


    \begin{textblock}{2.5}(6,2.4)

      \includegraphics[scale=0.1]{\ZintegrUJLogoColorPicturePL}

    \end{textblock}


    \begin{textblock}{4.2}(8.4,2.6)

      \includegraphics[scale=0.3]{\EUSocialFundLogoColorPicturePL}

    \end{textblock}





    \begin{textblock}{11}(1,4)

      \begin{flushleft}

        \mdseries

        \footnotesize

        \RaggedRight

        \color{jFrametitleFGColor}

        Treść niniejszego wykładu jest udostępniona na~licencji
        Creative Commons (\textsc{cc}), z~uzna\-niem autorstwa
        (\textsc{by}) oraz udostępnianiem na tych samych warunkach
        (\textsc{sa}). Rysunki i~wy\-kresy zawarte w~wykładzie są
        autorstwa dr.~hab.~Pawła Węgrzyna et~al. i~są dostępne
        na tej samej licencji, o~ile nie wskazano inaczej.
        W~prezentacji wykorzystano temat Beamera Jagiellonian,
        oparty na~temacie Metropolis Matthiasa Vogelgesanga,
        dostępnym na licencji \LaTeX{} Project Public License~1.3c
        pod adresem: \colorhref{https://github.com/matze/mtheme}
        {https://github.com/matze/mtheme}.

        Projekt typograficzny: Iwona Grabska-Gradzińska;
        Skład: Kamil Ziemian \\
        Korekta: Wojciech Palacz;
        Modele: Dariusz Frymus, Kamil Nowakowski \\
        Rysunki i~wykresy: Kamil Ziemian, Paweł Węgrzyn, Wojciech Palacz \\
        Montaż: Agencja Filmowa Film \& Television Production~-- Zbigniew
        Masklak

      \end{flushleft}

    \end{textblock}

  \end{frame}





  \begin{frame}[standout]


    \begingroup

    \color{jFrametitleFGColor}

    #1

    \endgroup

  \end{frame}
}










% ---------------------------------------
% Commands for lectures "Geometria 3D dla twórców gier wideo"
% English version
% ---------------------------------------
% \newcommand{\FundingLogoWhitePictureEN}
% {./PresentationPictures/CommonPictures/logotypFundusze_biale_bez_tla2.pdf}
\newcommand{\FundingLogoColorPictureEN}
{./PresentationPictures/CommonPictures/European_Funds_color_EN.pdf}
% \newcommand{\EULogoWhitePictureEN}
% {./PresentationPictures/CommonPictures/logotypUE_biale_bez_tla2.pdf}
\newcommand{\EUSocialFundLogoColorPictureEN}
{./PresentationPictures/CommonPictures/EU_Social_Fund_color_EN.pdf}
% \newcommand{\ZintegrUJLogoWhitePictureEN}
% {./PresentationPictures/CommonPictures/zintegruj-logo-white.pdf}
\newcommand{\ZintegrUJLogoColorPictureEN}
{./PresentationPictures/CommonPictures/ZintegrUJ_color.pdf}
\newcommand{\JULogoColorPictureEN}
{./JagiellonianPictures/LogoJU_EN/LogoJU_A_color.pdf}



\newcommand{\GeometryThreeDSpecialBeginningSlideEN}{%
  \begin{frame}[standout]

    \begin{textblock}{11}(1,0.7)

      \begin{flushleft}

        \mdseries

        \footnotesize

        \color{jFrametitleFGColor}

        This content was created as part of a project co-financed by the
        European Union within the framework of the European Social Fund
        POWR.03.05.00-00-Z309/17-00.

      \end{flushleft}

    \end{textblock}





    \begin{textblock}{10}(0,2.2)

      \tikz \fill[color=jBackgroundStyleLight] (0,0) rectangle (12.8,-1.5);

    \end{textblock}


    \begin{textblock}{3.2}(0.7,2.45)

      \includegraphics[scale=0.3]{\FundingLogoColorPictureEN}

    \end{textblock}


    \begin{textblock}{2.5}(4.15,2.5)

      \includegraphics[scale=0.2]{\JULogoColorPictureEN}

    \end{textblock}


    \begin{textblock}{2.5}(6.35,2.4)

      \includegraphics[scale=0.1]{\ZintegrUJLogoColorPictureEN}

    \end{textblock}


    \begin{textblock}{4.2}(8.4,2.6)

      \includegraphics[scale=0.3]{\EUSocialFundLogoColorPictureEN}

    \end{textblock}

  \end{frame}
}



\newcommand{\GeometryThreeDTwoSpecialBeginningSlidesEN}{%
  \begin{frame}[standout]

    \begin{textblock}{11}(1,0.7)

      \begin{flushleft}

        \mdseries

        \footnotesize

        \color{jFrametitleFGColor}

        This content was created as part of a project co-financed by the
        European Union within the framework of the European Social Fund
        POWR.03.05.00-00-Z309/17-00.

      \end{flushleft}

    \end{textblock}





    \begin{textblock}{10}(0,2.2)

      \tikz \fill[color=jBackgroundStyleLight] (0,0) rectangle (12.8,-1.5);

    \end{textblock}


    \begin{textblock}{3.2}(0.7,2.45)

      \includegraphics[scale=0.3]{\FundingLogoColorPictureEN}

    \end{textblock}


    \begin{textblock}{2.5}(4.15,2.5)

      \includegraphics[scale=0.2]{\JULogoColorPictureEN}

    \end{textblock}


    \begin{textblock}{2.5}(6.35,2.4)

      \includegraphics[scale=0.1]{\ZintegrUJLogoColorPictureEN}

    \end{textblock}


    \begin{textblock}{4.2}(8.4,2.6)

      \includegraphics[scale=0.3]{\EUSocialFundLogoColorPictureEN}

    \end{textblock}

  \end{frame}





  \TitleSlideWithPicture
}



\newcommand{\GeometryThreeDSpecialEndingSlideEN}{%
  \begin{frame}[standout]

    \begin{textblock}{11}(1,0.7)

      \begin{flushleft}

        \mdseries

        \footnotesize

        \color{jFrametitleFGColor}

        This content was created as part of a project co-financed by the
        European Union within the framework of the European Social Fund
        POWR.03.05.00-00-Z309/17-00.

      \end{flushleft}

    \end{textblock}





    \begin{textblock}{10}(0,2.2)

      \tikz \fill[color=jBackgroundStyleLight] (0,0) rectangle (12.8,-1.5);

    \end{textblock}


    \begin{textblock}{3.2}(0.7,2.45)

      \includegraphics[scale=0.3]{\FundingLogoColorPictureEN}

    \end{textblock}


    \begin{textblock}{2.5}(4.15,2.5)

      \includegraphics[scale=0.2]{\JULogoColorPictureEN}

    \end{textblock}


    \begin{textblock}{2.5}(6.35,2.4)

      \includegraphics[scale=0.1]{\ZintegrUJLogoColorPictureEN}

    \end{textblock}


    \begin{textblock}{4.2}(8.4,2.6)

      \includegraphics[scale=0.3]{\EUSocialFundLogoColorPictureEN}

    \end{textblock}





    \begin{textblock}{11}(1,4)

      \begin{flushleft}

        \mdseries

        \footnotesize

        \RaggedRight

        \color{jFrametitleFGColor}

        The content of this lecture is made available under a~Creative
        Commons licence (\textsc{cc}), giving the author the credits
        (\textsc{by}) and putting an obligation to share on the same terms
        (\textsc{sa}). Figures and diagrams included in the lecture are
        authored by Paweł Węgrzyn et~al., and are available under the same
        license unless indicated otherwise.\\ The presentation uses the
        Beamer Jagiellonian theme based on Matthias Vogelgesang’s
        Metropolis theme, available under license \LaTeX{} Project
        Public License~1.3c at: \colorhref{https://github.com/matze/mtheme}
        {https://github.com/matze/mtheme}.

        Typographic design: Iwona Grabska-Gradzińska \\
        \LaTeX{} Typesetting: Kamil Ziemian \\
        Proofreading: Wojciech Palacz,
        Monika Stawicka \\
        3D Models: Dariusz Frymus, Kamil Nowakowski \\
        Figures and charts: Kamil Ziemian, Paweł Węgrzyn, Wojciech Palacz

      \end{flushleft}

    \end{textblock}

  \end{frame}
}



\newcommand{\GeometryThreeDTwoSpecialEndingSlidesEN}[1]{%
  \begin{frame}[standout]


    \begin{textblock}{11}(1,0.7)

      \begin{flushleft}

        \mdseries

        \footnotesize

        \color{jFrametitleFGColor}

        This content was created as part of a project co-financed by the
        European Union within the framework of the European Social Fund
        POWR.03.05.00-00-Z309/17-00.

      \end{flushleft}

    \end{textblock}





    \begin{textblock}{10}(0,2.2)

      \tikz \fill[color=jBackgroundStyleLight] (0,0) rectangle (12.8,-1.5);

    \end{textblock}


    \begin{textblock}{3.2}(0.7,2.45)

      \includegraphics[scale=0.3]{\FundingLogoColorPictureEN}

    \end{textblock}


    \begin{textblock}{2.5}(4.15,2.5)

      \includegraphics[scale=0.2]{\JULogoColorPictureEN}

    \end{textblock}


    \begin{textblock}{2.5}(6.35,2.4)

      \includegraphics[scale=0.1]{\ZintegrUJLogoColorPictureEN}

    \end{textblock}


    \begin{textblock}{4.2}(8.4,2.6)

      \includegraphics[scale=0.3]{\EUSocialFundLogoColorPictureEN}

    \end{textblock}





    \begin{textblock}{11}(1,4)

      \begin{flushleft}

        \mdseries

        \footnotesize

        \RaggedRight

        \color{jFrametitleFGColor}

        The content of this lecture is made available under a~Creative
        Commons licence (\textsc{cc}), giving the author the credits
        (\textsc{by}) and putting an obligation to share on the same terms
        (\textsc{sa}). Figures and diagrams included in the lecture are
        authored by Paweł Węgrzyn et~al., and are available under the same
        license unless indicated otherwise.\\ The presentation uses the
        Beamer Jagiellonian theme based on Matthias Vogelgesang’s
        Metropolis theme, available under license \LaTeX{} Project
        Public License~1.3c at: \colorhref{https://github.com/matze/mtheme}
        {https://github.com/matze/mtheme}.

        Typographic design: Iwona Grabska-Gradzińska \\
        \LaTeX{} Typesetting: Kamil Ziemian \\
        Proofreading: Wojciech Palacz,
        Monika Stawicka \\
        3D Models: Dariusz Frymus, Kamil Nowakowski \\
        Figures and charts: Kamil Ziemian, Paweł Węgrzyn, Wojciech Palacz

      \end{flushleft}

    \end{textblock}

  \end{frame}





  \begin{frame}[standout]

    \begingroup

    \color{jFrametitleFGColor}

    #1

    \endgroup

  \end{frame}
}



\newcommand{\GeometryThreeDSpecialEndingSlideVideoVerOneEN}{%
  \begin{frame}[standout]

    \begin{textblock}{11}(1,0.7)

      \begin{flushleft}

        \mdseries

        \footnotesize

        \color{jFrametitleFGColor}

        This content was created as part of a project co-financed by the
        European Union within the framework of the European Social Fund
        POWR.03.05.00-00-Z309/17-00.

      \end{flushleft}

    \end{textblock}





    \begin{textblock}{10}(0,2.2)

      \tikz \fill[color=jBackgroundStyleLight] (0,0) rectangle (12.8,-1.5);

    \end{textblock}


    \begin{textblock}{3.2}(0.7,2.45)

      \includegraphics[scale=0.3]{\FundingLogoColorPictureEN}

    \end{textblock}


    \begin{textblock}{2.5}(4.15,2.5)

      \includegraphics[scale=0.2]{\JULogoColorPictureEN}

    \end{textblock}


    \begin{textblock}{2.5}(6.35,2.4)

      \includegraphics[scale=0.1]{\ZintegrUJLogoColorPictureEN}

    \end{textblock}


    \begin{textblock}{4.2}(8.4,2.6)

      \includegraphics[scale=0.3]{\EUSocialFundLogoColorPictureEN}

    \end{textblock}





    \begin{textblock}{11}(1,4)

      \begin{flushleft}

        \mdseries

        \footnotesize

        \RaggedRight

        \color{jFrametitleFGColor}

        The content of this lecture is made available under a Creative
        Commons licence (\textsc{cc}), giving the author the credits
        (\textsc{by}) and putting an obligation to share on the same terms
        (\textsc{sa}). Figures and diagrams included in the lecture are
        authored by Paweł Węgrzyn et~al., and are available under the same
        license unless indicated otherwise.\\ The presentation uses the
        Beamer Jagiellonian theme based on Matthias Vogelgesang’s
        Metropolis theme, available under license \LaTeX{} Project
        Public License~1.3c at: \colorhref{https://github.com/matze/mtheme}
        {https://github.com/matze/mtheme}.

        Typographic design: Iwona Grabska-Gradzińska;
        \LaTeX{} Typesetting: Kamil Ziemian \\
        Proofreading: Wojciech Palacz,
        Monika Stawicka \\
        3D Models: Dariusz Frymus, Kamil Nowakowski \\
        Figures and charts: Kamil Ziemian, Paweł Węgrzyn, Wojciech
        Palacz \\
        Film editing: Agencja Filmowa Film \& Television Production~--
        Zbigniew Masklak

      \end{flushleft}

    \end{textblock}

  \end{frame}
}



\newcommand{\GeometryThreeDSpecialEndingSlideVideoVerTwoEN}{%
  \begin{frame}[standout]

    \begin{textblock}{11}(1,0.7)

      \begin{flushleft}

        \mdseries

        \footnotesize

        \color{jFrametitleFGColor}

        This content was created as part of a project co-financed by the
        European Union within the framework of the European Social Fund
        POWR.03.05.00-00-Z309/17-00.

      \end{flushleft}

    \end{textblock}





    \begin{textblock}{10}(0,2.2)

      \tikz \fill[color=jBackgroundStyleLight] (0,0) rectangle (12.8,-1.5);

    \end{textblock}


    \begin{textblock}{3.2}(0.7,2.45)

      \includegraphics[scale=0.3]{\FundingLogoColorPictureEN}

    \end{textblock}


    \begin{textblock}{2.5}(4.15,2.5)

      \includegraphics[scale=0.2]{\JULogoColorPictureEN}

    \end{textblock}


    \begin{textblock}{2.5}(6.35,2.4)

      \includegraphics[scale=0.1]{\ZintegrUJLogoColorPictureEN}

    \end{textblock}


    \begin{textblock}{4.2}(8.4,2.6)

      \includegraphics[scale=0.3]{\EUSocialFundLogoColorPictureEN}

    \end{textblock}





    \begin{textblock}{11}(1,4)

      \begin{flushleft}

        \mdseries

        \footnotesize

        \RaggedRight

        \color{jFrametitleFGColor}

        The content of this lecture is made available under a Creative
        Commons licence (\textsc{cc}), giving the author the credits
        (\textsc{by}) and putting an obligation to share on the same terms
        (\textsc{sa}). Figures and diagrams included in the lecture are
        authored by Paweł Węgrzyn et~al., and are available under the same
        license unless indicated otherwise.\\ The presentation uses the
        Beamer Jagiellonian theme based on Matthias Vogelgesang’s
        Metropolis theme, available under license \LaTeX{} Project
        Public License~1.3c at: \colorhref{https://github.com/matze/mtheme}
        {https://github.com/matze/mtheme}.

        Typographic design: Iwona Grabska-Gradzińska;
        \LaTeX{} Typesetting: Kamil Ziemian \\
        Proofreading: Wojciech Palacz,
        Monika Stawicka \\
        3D Models: Dariusz Frymus, Kamil Nowakowski \\
        Figures and charts: Kamil Ziemian, Paweł Węgrzyn, Wojciech
        Palacz \\
        Film editing: IMAVI -- Joanna Kozakiewicz, Krzysztof Magda, Nikodem
        Frodyma

      \end{flushleft}

    \end{textblock}

  \end{frame}
}



\newcommand{\GeometryThreeDSpecialEndingSlideVideoVerThreeEN}{%
  \begin{frame}[standout]

    \begin{textblock}{11}(1,0.7)

      \begin{flushleft}

        \mdseries

        \footnotesize

        \color{jFrametitleFGColor}

        This content was created as part of a project co-financed by the
        European Union within the framework of the European Social Fund
        POWR.03.05.00-00-Z309/17-00.

      \end{flushleft}

    \end{textblock}





    \begin{textblock}{10}(0,2.2)

      \tikz \fill[color=jBackgroundStyleLight] (0,0) rectangle (12.8,-1.5);

    \end{textblock}


    \begin{textblock}{3.2}(0.7,2.45)

      \includegraphics[scale=0.3]{\FundingLogoColorPictureEN}

    \end{textblock}


    \begin{textblock}{2.5}(4.15,2.5)

      \includegraphics[scale=0.2]{\JULogoColorPictureEN}

    \end{textblock}


    \begin{textblock}{2.5}(6.35,2.4)

      \includegraphics[scale=0.1]{\ZintegrUJLogoColorPictureEN}

    \end{textblock}


    \begin{textblock}{4.2}(8.4,2.6)

      \includegraphics[scale=0.3]{\EUSocialFundLogoColorPictureEN}

    \end{textblock}





    \begin{textblock}{11}(1,4)

      \begin{flushleft}

        \mdseries

        \footnotesize

        \RaggedRight

        \color{jFrametitleFGColor}

        The content of this lecture is made available under a Creative
        Commons licence (\textsc{cc}), giving the author the credits
        (\textsc{by}) and putting an obligation to share on the same terms
        (\textsc{sa}). Figures and diagrams included in the lecture are
        authored by Paweł Węgrzyn et~al., and are available under the same
        license unless indicated otherwise.\\ The presentation uses the
        Beamer Jagiellonian theme based on Matthias Vogelgesang’s
        Metropolis theme, available under license \LaTeX{} Project
        Public License~1.3c at: \colorhref{https://github.com/matze/mtheme}
        {https://github.com/matze/mtheme}.

        Typographic design: Iwona Grabska-Gradzińska;
        \LaTeX{} Typesetting: Kamil Ziemian \\
        Proofreading: Wojciech Palacz,
        Monika Stawicka \\
        3D Models: Dariusz Frymus, Kamil Nowakowski \\
        Figures and charts: Kamil Ziemian, Paweł Węgrzyn, Wojciech
        Palacz \\
        Film editing: Agencja Filmowa Film \& Television Production~--
        Zbigniew Masklak \\
        Film editing: IMAVI -- Joanna Kozakiewicz, Krzysztof Magda, Nikodem
        Frodyma

      \end{flushleft}

    \end{textblock}

  \end{frame}
}



\newcommand{\GeometryThreeDTwoSpecialEndingSlidesVideoVerOneEN}[1]{%
  \begin{frame}[standout]

    \begin{textblock}{11}(1,0.7)

      \begin{flushleft}

        \mdseries

        \footnotesize

        \color{jFrametitleFGColor}

        This content was created as part of a project co-financed by the
        European Union within the framework of the European Social Fund
        POWR.03.05.00-00-Z309/17-00.

      \end{flushleft}

    \end{textblock}





    \begin{textblock}{10}(0,2.2)

      \tikz \fill[color=jBackgroundStyleLight] (0,0) rectangle (12.8,-1.5);

    \end{textblock}


    \begin{textblock}{3.2}(0.7,2.45)

      \includegraphics[scale=0.3]{\FundingLogoColorPictureEN}

    \end{textblock}


    \begin{textblock}{2.5}(4.15,2.5)

      \includegraphics[scale=0.2]{\JULogoColorPictureEN}

    \end{textblock}


    \begin{textblock}{2.5}(6.35,2.4)

      \includegraphics[scale=0.1]{\ZintegrUJLogoColorPictureEN}

    \end{textblock}


    \begin{textblock}{4.2}(8.4,2.6)

      \includegraphics[scale=0.3]{\EUSocialFundLogoColorPictureEN}

    \end{textblock}





    \begin{textblock}{11}(1,4)

      \begin{flushleft}

        \mdseries

        \footnotesize

        \RaggedRight

        \color{jFrametitleFGColor}

        The content of this lecture is made available under a Creative
        Commons licence (\textsc{cc}), giving the author the credits
        (\textsc{by}) and putting an obligation to share on the same terms
        (\textsc{sa}). Figures and diagrams included in the lecture are
        authored by Paweł Węgrzyn et~al., and are available under the same
        license unless indicated otherwise.\\ The presentation uses the
        Beamer Jagiellonian theme based on Matthias Vogelgesang’s
        Metropolis theme, available under license \LaTeX{} Project
        Public License~1.3c at: \colorhref{https://github.com/matze/mtheme}
        {https://github.com/matze/mtheme}.

        Typographic design: Iwona Grabska-Gradzińska;
        \LaTeX{} Typesetting: Kamil Ziemian \\
        Proofreading: Wojciech Palacz,
        Monika Stawicka \\
        3D Models: Dariusz Frymus, Kamil Nowakowski \\
        Figures and charts: Kamil Ziemian, Paweł Węgrzyn,
        Wojciech Palacz \\
        Film editing: Agencja Filmowa Film \& Television Production~--
        Zbigniew Masklak

      \end{flushleft}

    \end{textblock}

  \end{frame}





  \begin{frame}[standout]


    \begingroup

    \color{jFrametitleFGColor}

    #1

    \endgroup

  \end{frame}
}



\newcommand{\GeometryThreeDTwoSpecialEndingSlidesVideoVerTwoEN}[1]{%
  \begin{frame}[standout]

    \begin{textblock}{11}(1,0.7)

      \begin{flushleft}

        \mdseries

        \footnotesize

        \color{jFrametitleFGColor}

        This content was created as part of a project co-financed by the
        European Union within the framework of the European Social Fund
        POWR.03.05.00-00-Z309/17-00.

      \end{flushleft}

    \end{textblock}





    \begin{textblock}{10}(0,2.2)

      \tikz \fill[color=jBackgroundStyleLight] (0,0) rectangle (12.8,-1.5);

    \end{textblock}


    \begin{textblock}{3.2}(0.7,2.45)

      \includegraphics[scale=0.3]{\FundingLogoColorPictureEN}

    \end{textblock}


    \begin{textblock}{2.5}(4.15,2.5)

      \includegraphics[scale=0.2]{\JULogoColorPictureEN}

    \end{textblock}


    \begin{textblock}{2.5}(6.35,2.4)

      \includegraphics[scale=0.1]{\ZintegrUJLogoColorPictureEN}

    \end{textblock}


    \begin{textblock}{4.2}(8.4,2.6)

      \includegraphics[scale=0.3]{\EUSocialFundLogoColorPictureEN}

    \end{textblock}





    \begin{textblock}{11}(1,4)

      \begin{flushleft}

        \mdseries

        \footnotesize

        \RaggedRight

        \color{jFrametitleFGColor}

        The content of this lecture is made available under a Creative
        Commons licence (\textsc{cc}), giving the author the credits
        (\textsc{by}) and putting an obligation to share on the same terms
        (\textsc{sa}). Figures and diagrams included in the lecture are
        authored by Paweł Węgrzyn et~al., and are available under the same
        license unless indicated otherwise.\\ The presentation uses the
        Beamer Jagiellonian theme based on Matthias Vogelgesang’s
        Metropolis theme, available under license \LaTeX{} Project
        Public License~1.3c at: \colorhref{https://github.com/matze/mtheme}
        {https://github.com/matze/mtheme}.

        Typographic design: Iwona Grabska-Gradzińska;
        \LaTeX{} Typesetting: Kamil Ziemian \\
        Proofreading: Wojciech Palacz,
        Monika Stawicka \\
        3D Models: Dariusz Frymus, Kamil Nowakowski \\
        Figures and charts: Kamil Ziemian, Paweł Węgrzyn,
        Wojciech Palacz \\
        Film editing: IMAVI -- Joanna Kozakiewicz, Krzysztof Magda, Nikodem
        Frodyma

      \end{flushleft}

    \end{textblock}

  \end{frame}





  \begin{frame}[standout]


    \begingroup

    \color{jFrametitleFGColor}

    #1

    \endgroup

  \end{frame}
}



\newcommand{\GeometryThreeDTwoSpecialEndingSlidesVideoVerThreeEN}[1]{%
  \begin{frame}[standout]

    \begin{textblock}{11}(1,0.7)

      \begin{flushleft}

        \mdseries

        \footnotesize

        \color{jFrametitleFGColor}

        This content was created as part of a project co-financed by the
        European Union within the framework of the European Social Fund
        POWR.03.05.00-00-Z309/17-00.

      \end{flushleft}

    \end{textblock}





    \begin{textblock}{10}(0,2.2)

      \tikz \fill[color=jBackgroundStyleLight] (0,0) rectangle (12.8,-1.5);

    \end{textblock}


    \begin{textblock}{3.2}(0.7,2.45)

      \includegraphics[scale=0.3]{\FundingLogoColorPictureEN}

    \end{textblock}


    \begin{textblock}{2.5}(4.15,2.5)

      \includegraphics[scale=0.2]{\JULogoColorPictureEN}

    \end{textblock}


    \begin{textblock}{2.5}(6.35,2.4)

      \includegraphics[scale=0.1]{\ZintegrUJLogoColorPictureEN}

    \end{textblock}


    \begin{textblock}{4.2}(8.4,2.6)

      \includegraphics[scale=0.3]{\EUSocialFundLogoColorPictureEN}

    \end{textblock}





    \begin{textblock}{11}(1,4)

      \begin{flushleft}

        \mdseries

        \footnotesize

        \RaggedRight

        \color{jFrametitleFGColor}

        The content of this lecture is made available under a Creative
        Commons licence (\textsc{cc}), giving the author the credits
        (\textsc{by}) and putting an obligation to share on the same terms
        (\textsc{sa}). Figures and diagrams included in the lecture are
        authored by Paweł Węgrzyn et~al., and are available under the same
        license unless indicated otherwise. \\ The presentation uses the
        Beamer Jagiellonian theme based on Matthias Vogelgesang’s
        Metropolis theme, available under license \LaTeX{} Project
        Public License~1.3c at: \colorhref{https://github.com/matze/mtheme}
        {https://github.com/matze/mtheme}.

        Typographic design: Iwona Grabska-Gradzińska;
        \LaTeX{} Typesetting: Kamil Ziemian \\
        Proofreading: Leszek Hadasz, Wojciech Palacz,
        Monika Stawicka \\
        3D Models: Dariusz Frymus, Kamil Nowakowski \\
        Figures and charts: Kamil Ziemian, Paweł Węgrzyn,
        Wojciech Palacz \\
        Film editing: Agencja Filmowa Film \& Television Production~--
        Zbigniew Masklak \\
        Film editing: IMAVI -- Joanna Kozakiewicz, Krzysztof Magda, Nikodem
        Frodyma


      \end{flushleft}

    \end{textblock}

  \end{frame}





  \begin{frame}[standout]


    \begingroup

    \color{jFrametitleFGColor}

    #1

    \endgroup

  \end{frame}
}











% ------------------------------------------------------
% BibLaTeX
% ------------------------------------------------------
% Package biblatex, with biber as its backend, allow us to handle
% bibliography entries that use Unicode symbols outside ASCII.
\usepackage[
language=polish,
backend=biber,
style=alphabetic,
url=false,
eprint=true,
]{biblatex}

\addbibresource{Podstawy-informatyki-ETC-Bibliography.bib}





% ------------------------------------------------------
% Importing packages, libraries and setting their configuration
% ------------------------------------------------------




% ------------------------------------------------------
% Local packages
% ------------------------------------------------------
% Local configuration of this particular presentation
\usepackage{./Local-packages/local-settings}










% ------------------------------------------------------------------------------------------------------------------
\title{Podstawy informatyki z~językiem~C}
\subtitle{Wprowadzenie do~kursu}

\author{Kamil Ziemian \\
  \email}


% \date{}
% ------------------------------------------------------------------------------------------------------------------










% ####################################################################
% Beginning of the document
\begin{document}
% ####################################################################





% ######################################
% Text is adjusted to the left and words are broken at the end of the line.
\RaggedRight
% Number of chars: 23k+,
% ######################################





% ######################################
\maketitle
% ######################################





% ##################
\begin{frame}
  \frametitle{Spis treści}


  \tableofcontents

\end{frame}
% ##################





% ######################################
\section{Informacje wstępne}
% ######################################



% ##################
\begin{frame}
  \frametitle{Informacje wstępne}


  Obawiam~się, że na tych konkretnych zajęciach będzie sporo przynudzania,
  ale nie widzę sposobu, by~tego uniknąć.

  Według mnie to zajęcia są dla studentów, nie studenci dla zajęć. Tak samo
  ja jestem tu dla Państwa, a~nie Państwo dla mnie. Jestem tu po to, by
  pomóc Państwu stawiać pierwsze kroki w~programowaniu w~języku~C.
  W~związku z~tym, ja będę Państwa rozliczał tylko i~wyłącznie
  z~umiejętności i~wiedzy, z~niczego innego. Wychodzę bowiem z~założenia,
  że~Państwo sami najlepiej wiedzą, czemu warto poświęcić swój czas. (Choć
  jak wiadomo, nie jeden raz potem stwierdzamy, że~nasz wybór mógł być
  jednak lepszy.)

  Na zajęciach nie tylko można, ale \alert{należy} zadawać pytania
  na dowolne związane z~zajęciami zagadnienia. W~szczególności
  \alert{należy} zadawać pytania, jeśli~się czegoś nie rozumie, lub coś
  jest niejasne. To są podstawy informatyki, \alert{nie} zakładamy,
  że~Państwo mają już wszystko umieć. Byłoby to bardzo niewłaściwe
  założenie.

\end{frame}
% ##################





% ##################
\begin{frame}
  \frametitle{Informacje wstępne}


  Proszę pamiętać, gdy chodzi o~tematy związane z~zajęciami
  \alert{nie} ma pytań zbyt elementarnych lub zbyt głupich. Są~tylko
  niezadowalające odpowiedzi na~Państwa pytania.

  Jestem tutaj by Państwu pomóc w~nauce programowania w~C, pytania
  z~Państwa strony bardzo mi to ułatwiają. Zadawania pytań nie traktuję
  jako oznakę tego, że~ktoś czegoś nie umie, tylko że~chce~się czegoś
  nauczyć.

  Pytania typu „Jaki jest najfajniejszy boss w~grze \textit{Hollow
    Knight}?” musimy jednak zostawić na czas po zajęciach.

\end{frame}
% ##################





% ##################
\begin{frame}
  \frametitle{Informacje wstępne}


  Na tych zajęciach \alert{nie} nauczymy~się jak programować. Jak dobrze
  pójdzie to nauczymy~się podstaw programowania w~języku~C, ale
  programowanie obejmuje tyle zagadnień i~wymaga tyle godzin praktyki,
  iż~nie ma najmniejszych szans, że~uda nam~się to wszystko zrobić.

  Co, na Państwa nieszczęście, nie oznacza, że~będzie mało materiału.
  Mogą~się Państwo wręcz czuć przytłoczeni ile tego jest. Proszę mi jednak
  uwierzyć, że~większość tego nie jest taka trudna, jak~się wydaje na
  pierwszy rzut oka.

\end{frame}
% ##################





% ##################
\begin{frame}
  \frametitle{Uwagi odnośnie treści zajęć}


  Ponieważ tematyka którą poruszamy jest mimo wszystko niebanalna, więc
  mnóstwo rzeczy będę musiał bardzo \alert{upraszczać}. Proszę mieć to na
  uwadze w~trakcie zajęć i~studiując materiały dla Państwa przygotowane.

  Potrzeba uproszczeń wynika z~dwóch powodów. Po pierwsze, ograniczenia
  czasowe. Wiele z~zagadnień które poruszymy mogłoby być tematem
  semestralnego kursu. (Co gorsza, wiele z~nich \alert{jest} tematem
  semestralnych kursów.) Po drugie, to jest kurs \alert{podstaw}
  informatyki, który ma położyć fundamenty pod Państwa umiejętności
  i~wiedzę. To nie przedmiot na którym należy wnikać we wszystkie detale,
  szczegóły, drugie, trzecie i~czwarte dno problemu.

  Jeśli jednak ktoś chce~się bardziej zagłębić w~te temat, to służę po
  zajęciach całą swoją osobą.

\end{frame}
% ##################





% ##################
\begin{frame}
  \frametitle{Informacje wstępne}


  Dlaczego zaczynamy naukę od~języka~C? Krótka odpowiedź jest taka,
  że~pomimo tego iż język ten ma już pół wieku~(!) na karku, w~2024 roku
  nasza infrastruktura informatyczna wciąż stoi na napisanym w~właśnie
  w~nim. W~internecie mogą Państwo znaleźć wiele artykułów i~blogów takich
  jak
  \colorhref{https://wideinfo.org/c-programming-is-still-running-the-world/}{\textit{C~programming is still running the
      world}} \parencite{Scott-C-programming-is-still-ETC-Ver-2024}
  z~czerwca 2024 roku. Jego tytuł mówi sam za siebie.

  Niektórzy mówią, że~C to król wszystkich języków programowania.
  Inni, bardzo szacowni programiści, twierdzą, że każdy szanujący
  programista musi znać~C (cf. str. ?? \parencite{Assembler}).

  A~jak to jest w~praktyce? Na tych zajęcia potrafią uczęszczać ludzie,
  którzy pracują zawodowo jako programiści i~nie mają pojęcia jak napisać
  program w~języku~C. Proszę samemu wyciągnąć z~tego wnioski.

\end{frame}
% ##################





% ##################
\begin{frame}
  \frametitle{Dlaczego język~C?}


  Dlaczego korzystamy z~systemu GNU/Linux, a~nie z~znacznie
  popularniejszego wśród normalny ludzi systemu Windows? Bo~oferuje
  znacznie lepsze warunki pracy z~językiem~C.

  \alert{Ważne.} Jeśli mają Państwo jakiekolwiek problemy z~systemem
  GNU/Linux to proszę o~tym \alert{mówić}. Nie przyjmujemy założenia,
  że~Państw mają już teraz być ekspertami, w~kwestii używania tego, co by
  tu nie mówić, często bardzo topornego systemu operacyjnego.

  \alert{Ważne.} Wykład ma bardziej charakter teoretyczny, te laboratoria
  zaś praktyczny. Niemniej wciąż poszukujemy optymalnej formy prowadzenia
  tych zajęć, bo obecna jest daleka od naszych pragnień. Naprawdę
  praktyczne zajęcia z~podstaw informatyki w~języku~C są trudne do
  zorganizowania, z~powodów o~których będziemy mówili później.

\end{frame}
% ##################










% ######################################
\section{O~uzyskaniu zaliczenia}
% ######################################



% ##################
\begin{frame}
  \frametitle{Zaliczenie zaoczne}


  Zaliczenie zaoczne można jak najbardziej uzyskać, np.~przedstawiając
  swój prywatny projekt jakiegoś programu czy aplikacji. Projekt ten
  \alert{nie musi} być napisany w~języku~C, może być w~Pythonie albo
  JavaScripcie. Z~powodów które powinny być dla wszystkich oczywiste,
  preferowane są jednak te stworzone w~języku C. (W~trochę mniejszym
  stopniu, w~C++).

  Każdy kto chce uzyskać zaliczenie zaoczne, niech zgłosi~się do mnie po
  zajęciach lub napisze na maila \email.

\end{frame}
% ##################





% ##################
\begin{frame}
  \frametitle{Uzyskanie zaliczenia}


  Zaliczenie i~ocenę uzyskują Państwo na podstawie trzech rzeczy.

  \begin{itemize}

  \item Zadania domowe.

  \item Dwa testy jednokrotnego wyboru.

  \item Jeden większy projekt.

  \end{itemize}

  Za każdy z~tych rzeczy przyznawana jest pewna ilość punktów. Niestety
  system przyznawania punktów wciąż jest daleki od ideału, ale mam nadzieję,
  że~nie będzie to dla Państwa zbyt wielkim obciążeniem.

  Jeśli ktoś ma uwagi do tego systemu, co można zmienić, co poprawić, to
  proszę powiedzieć to mi po zajęciach lub napisać pod adres \email.

\end{frame}
% ##################





% ##################
\begin{frame}
  \frametitle{Uwagi odnośnie treści zajęć}


  Pod koniec semestru podliczane są wszystkie punkty jakie były do
  zdobycia. W~zależności ile procent pełnej puli Państwo zdobyli, otrzymują
  Państwo odpowiednią ocenę.

  \begin{itemize}

  \item $41\%\text{--}50\%$ -- ocena dostateczna ($3.0$).

  \item $51\%\text{--}60\%$ -- ocena plus dostateczna ($3.5$, $3+$).

  \item $61\%\text{--}70\%$ -- ocena dobra ($4.0$).

  \item $71\%\text{--}80\%$ -- ocena puls dobry ($4.5$, $4+$).

  \item $81\%\text{--}100\%$ -- ocena bardzo dobry ($5.0$).

  \end{itemize}

  W~przypadku zaokrąglanie wyników, robione to jest zawsze na korzyść dla
  Państwa. Czyli $40.1\%$ zaokrągla~się do $41\%$.

  Ilość punktów do zdobycie jest będzie jawnie podana przy każdym
  konkretnym zadaniu domowym. Za jeden test można zdobyć 10 pkt., konkretna
  wartość zależy od zdobytej oceny. Za projekt można zdobyć do 30 pkt.
  Ilość punktów za projekt jest tym, co obecnie najmniej mnie
  satysfakcjonuje, ale na razie nie mam pomysłu jak to poprawić.

\end{frame}
% ##################





% ##################
\begin{frame}
  \frametitle{Punktacja testów}


  W~tym semestrze każdy test będzie zawierał 10 pytań jednokrotnego wyboru.
  Progi ocen są jak dla całego przedmiotu i~regułą zaokrąglania (raczej nie
  będzie trzeba jej używać) działa jak poprzednio.

  \begin{itemize}

  \item $41\%\text{--}50\%$ -- ocena dostateczna ($3.0$).

  \item $51\%\text{--}60\%$ -- ocena plus dostateczna ($3.5$, $3+$).

  \item $61\%\text{--}70\%$ -- ocena dobra ($4.0$).

  \item $71\%\text{--}80\%$ -- ocena puls dobry ($4.5$, $4+$).

  \item $81\%\text{--}100\%$ -- ocena bardzo dobry ($5.0$).

  \end{itemize}

  Ilość zdobytych punktów równa jest zero jeśli uzyskało~się ocenę $2.0$
  (mniej niż $41\%$), lub $2 \cdot \text{ocena}$.

\end{frame}
% ##################




% ##################
\begin{frame}
  \frametitle{Punktacja testów}


  Ilość punktów do końcowej uzyskanych w~zależności od wyniku testu.

  \begin{itemize}

  \item $0\%\text{--}40\%$ -- $0$ pkt.

  \item $41\%\text{--}50\%$ -- $6$ pkt.

  \item $51\%\text{--}60\%$ -- $7$ pkt.

  \item $61\%\text{--}70\%$ -- $8$ pkt.

  \item $71\%\text{--}80\%$ -- $9$ pkt.

  \item $81\%\text{--}100\%$ -- $10$ pkt.

  \end{itemize}

\end{frame}
% ##################










% ######################################
\section{Materiały do nauki}
% ######################################



% ##################
\begin{frame}
  \frametitle{Materiały do nauki}


  Prezentacje te są dostępne w~formie plików \LaTeX a (kodu źródłowego)
  na serwisie GitHub. Każdy kto ma na komputerze program Git i~dostęp
  do internetu może jest zdobyć wpisując \\
  \texttt{\$ git clone https://github.com/KZiemian/Presentation} \\
  Znajdują~się one w~katalogu „Podstawy-informatyki-ETC-Prezentacje”.

  Będą też dostępne w~formie \textsc{pdfów} na~Sake, wraz z~innymi
  materiałami. Dostępna tam będzie również lista zagadnień do opanowania
  z~tego przedmiotu, która będzie główny punktem odniesieniem przy
  tworzeniu pytań testowych. Jak również dwa listy materiałów do nauki.
  Jedna lista normalna, druga dla osób ambitnych.

  Proszę zwrócić uwagę, że~ze względu na charakter tych zajęć, wystarczające
  jest, żeby o~pewnych rzeczach wymienionych na liście zagadnień mieli
  Państwo bardzo ogólne i~podstawowe pojęcie. Nawet w~sytuacji, gdy na
  kursie było o~danym zagadnieniu powiedziane znacznie więcej.

\end{frame}
% ##################





% ##################
\begin{frame}
  \frametitle{Punktacja testów}


  Przykładowo, jest wystarczające, żeby Państwo wiedzieli, że~kompilator
  języka~C jest to program, który przetwarza kod napisany w~języku~C
  w~program, który jest napisany w~języku zrozumiałym dla komputera.
  Nawet jeśli na zajęciach wspomnimy czym są takie części kompilatora
  jak lekser czy parser, nie jest wymagane by Państwo po tym kursie
  wiedzieli o~ich istnieniu, nie mówiąc już o~znajomości tego co robią.

  Na liście zagadnień postaramy~się wyróżnić tego typu pytania w~specjalny
  sposób. Dodać należy, że~gdy chodzi o~pozostałe pytania, wymagana jest
  dobra znajomość na \alert{poziomie tego co było prezentowane na kursie},
  nie zaś taka jaka jest zawarta w~standardzie??? języka~C. Bądźmy poważni,
  to jest tylko kurs podstaw informatyki.

\end{frame}
% ##################









% ##################
\begin{frame}
  \frametitle{Zgłaszanie błędu i~uwag}


  W~razie znalezienia jakiegokolwiek błędu lub jakichkolwiek uwag
  merytorycznych do zajęć lub dostępnych materiałów proszę zgłaszać to
  przed lub po zajęciach lub też pisać pod adres \email. Chcemy by te
  zajęcia i~towarzyszące im materiały były możliwie proste, łatwe
  w~zrozumieniu i~pozbawione błędów. Proszę jednak uwierzyć, że~osiągnięcie
  tego jest naprawdę trudne.

\end{frame}
% ##################










% ######################################
\section{Czy informatyka jest trudna?}
% ######################################



% ##################
\begin{frame}
  \frametitle{Czy informatyka jest trudna?}


  Ten przedmiot dotyczy podstaw informatyki w~języku~C, warto~się
  więc spytać, czy informatyka jest prosta czy trudna w~nauce?

  Informatyka to osobna dziedzina nauki i~jeśli zabrnie~się odpowiednio
  głęboko, to robi~się naprawdę złożona i~niebanalna. Jednak na stosunkowo
  płytkim poziomie to czy jest on trudna czy nie, to mocno zależy od~odczuć
  konkretnej osoby.

  Zadam takie pytanie: czy włączenie komputera jest skomplikowane?
  Odpowiemy na to pytanie na dwóch poziomach. Pierwszy to poziom normalnego
  użytkownika, drugi to opis pochodzący z~książki Andrewa S.~Tanenbauma
  \textit{Systemy operacyjne. Wydanie~III}
  \parencite{Tannenbaum-Systemy-Operacyjne-Wydanie-III-Pub-2013}, dotyczący
  komputera z~systemem Pentium.

\end{frame}
% ##################





% ##################
\begin{frame}
  \frametitle{Włączanie komputera, poziom normalnego użytkownika}


  \begin{enumerate}

  \item Wciskamy przycisk \texttt{Power}.

  \item Czekamy minutę albo dłużej.

  \item Wybieramy użytkownika i~wchodzimy na swoje konto.

  \end{enumerate}

  Co w~tym trudnego?

\end{frame}
% ##################





% ##################
\begin{frame}
  \frametitle{Kilka pojęcia}


  Oczywiście, opis włączania komputera z~książki Tanenbauma jest tak
  skomplikowany, że~trzeba wprowadzić trochę pojęć wstępnych.

  \textbf{\textsc{rom}}, ang.~\textit{Read Only Memory}, pl.~\textit{pamięć
    wyłącznie do~odczytu}. Pamięć komputera której zawartość została
  zapisana przez firmę, która ten fragment pamięci wyprodukowała
  i~użytkownik nie może zmodyfikować jej teści. Przynajmniej nie w~żaden
  normalny sposób.

  \textbf{\textsc{ram}}, ang.~\textit{Random Access Memory},
  pl.~\textit{pamięć o~dostępie w~trybie losowym}. Pamięć komputera o~tej
  własności, że~jeśli będę w~sposób losowy wybierał elementy tej pamięci,
  to czas odczytania informacje z~każdego jej elementu będzie taki sam.
  Inaczej mówiąc dostęp do dowolnego miejsca tej pamięci zajmuje tyle samo
  czasu.

  Tak naprawdę czas odczytu zależy od tego, w~jakiś sposób pamięć
  \textsc{ram} jest odczytywana, ale jeszcze długo nie będziemy się musieli
  tym przejmować.

\end{frame}
% ##################





% ##################
\begin{frame}
  \frametitle{Kilka pojęcia}


  \textbf{Pamięć ulotna}, ang.~\textit{volatile memory}. Pamięć której
  zawartość jest tracona, gdy przestaje przez nią płynąć prąd. Typowym
  przykładem takiej pamięci jest \textsc{ram}.

  \textbf{Pamięć nieulotna}, ang.~\textit{non-volatile memory}. Pamięć,
  której treść jest zachowana, gdy przez układ przestaje płynąć prąd,
  typowym przykładem jest dysk \textsc{ssd}.

  Żeby skomplikować życie, pamięcią nieulotną nazywa~się także tą pamięć,
  które jest ulotna w~ścisłym sensie, ale ponieważ jest zaopatrzona
  we~własną baterię, jej zawartość jest zachowana również po wyłączeniu
  komputera z~prądu. Bo~niby czemu życie ma być proste?

\end{frame}
% ##################





% ##################
\begin{frame}
  \frametitle{Kilka pojęcia}


  \textbf{Pamięć \textsc{cmos}}, często po prostu \textbf{\textsc{cmos}}.
  Skrót pochodzi od angielskiej nazwy technologi \textit{Complementary
    Metal-Oxide-Semiconductor} (pl.~\textit{komplementarny półprzewodnik
    metalowo-tlenkowy}), w~której ta pamięć jest wykonana. Musi być zasilana
  prądem, by~zachowywała swój stan, ale ponieważ wyposażona jest w~baterię
  klasyfikowana jest jako nieulotna.

  \textbf{\textsc{bios}} ang.~\textit{Basic Input Output System}, pl.
  \textit{podstawowy system wejścia, wyjścia}. Program znajdujący~się
  na płycie głównej komputera, odpowiedzialny między innymi za odczytywanie
  klawiatury, zapisywanie ekranu oraz operacje wejścia-wyjścia dysków.

\end{frame}
% ##################





% ##################
\begin{frame}
  \frametitle{Uruchamianie komputera z~systemem Pentium}


  \begin{itemize}

  \item[1)] Wciskamy przycisk \texttt{Power}.

  \item[2)] Z~płyty głównej ładowany jest program \textsc{bios}. Sprawdza on
    ilość zainstalowanej pamięci \textsc{ram}, czy komputer dysponuje
    klawiaturą i~innymi podstawowymi urządzeniami oraz sprawdza czy
    odpowiadają one w~sposób prawidłowy. W~pierwszej kolejności skanowane
    są magistrale \textsc{isa} (ang. \textit{Industry Standard
      Architecture}) i~\textsc{pci} (ang.~\textit{Peripheral Component
      Interconnect}) w~celu wykrycia podłączonych do nich urządzeń.

  \item[3)] Jeśli do komputera podłączone są inne urządzenia, niż te które
    były dostępne przy jego ostatni uruchomieniu, nowe urządzenia są
    konfigurowane.

  \item[4)] Program \textsc{bios} odczytuje listę tzw. urządzeń rozruchowych
    z~pamięci \textsc{cmos}. Urządzenia rozruchowe to te, które zawierają
    system operacyjny. W~przeszłości były nimi dyskietki, płyty
    \textsc{cd}-\textsc{rom}, \textsc{dvd}, dziś choćby pendriwy
    i~dyski~\textsc{ssd}.

  \end{itemize}

\end{frame}
% ##################





% ##################
\begin{frame}
  \frametitle{Uruchamianie komputera z~systemem Pentium}


  \begin{itemize}

  \item[5)] \textsc{bios} testuje po kolei urządzenia rozruchowe
    z~wspomnianej wcześniej listy, aż~znajdzie pierwsze, który zawiera
    działający system operacyjny.

  \item[6)] \textsc{bios} wczytuje pierwszy sektor ze~znalezionego
    w~poprzednim punkcie urządzenia rozruchowego do pamięci i~go uruchamia.

  \item[7)] Program z~pierwszego sektora sprawdza zapisaną na jego końcu
    listę partycji, by~ustalić która z~nich jest partycją aktywną.
    Następnie wczytuje z~tej partycji pomocniczy program rozruchowy.

  \item[8)] Pomocniczy program rozruchowy wczytuje system operacyjny
    z~aktywnej partycji i~go uruchamia.

  \item[9)] System operacyjny odczytuje informacje konfiguracyjne z~systemu
    \textsc{bios}. Dla każdego dostępnego urządzenia sprawdza, czy posiada
    do niego sterowniki. Jeśli nie, to prosi o~ich zainstalowanie
    z~odpowiedniego źródła.

  \end{itemize}

\end{frame}
% ##################





% ##################
\begin{frame}
  \frametitle{Uruchamianie komputera z~systemem Pentium}


  \begin{itemize}

  \item[10)] Jeśli system operacyjny dysponuje wszystkimi sterownikami,
    to ładuje je do jądra systemu.

  \item[11)] System operacyjny tworzy tabele systemowe oraz procesy
    działające w~tle.

  \item[12)] Uruchamiane jest okno logowania.

  \end{itemize}

\end{frame}
% ##################






% ##################
\begin{frame}
  \frametitle{Bootowanie}


  W~literaturze funkcjonuje termin \textbf{bootwoanie}, zwane też
  \textbf{uruchamianiem} lub \textbf{rozruchem}. Odnosi~się ono albo do
  całej procedury uruchamiania komputer opisanej powyżej, albo tylko
  stawiania systemu operacyjnego, czyli od kiedy \textsc{bios} wczytał
  pierwszy jego sektor do pamięci (punkt siedem i~dalej). Acz to pojęcie
  nie jest specjalnie ostro zdefiniowane.

\end{frame}
% ##################





% ##################
\begin{frame}
  \frametitle{Czy uruchomienie komputera jest proste czy trudne?}


  Zależy jak do tego podchodzimy. I~tak jest z~większością rzeczy
  w~informatyce.

\end{frame}
% ##################







% % ##################
% \jagiellonianendslide{Czy są jakieś pytania do tej części?}
% % ##################










% ######################################
\section{Uwagi dla początkujących}
% ######################################



% ##################
\begin{frame}
  \frametitle{Uwagi dla początkujący}


  Rozszerzenie pliku (kropka i~to co po niej w~jego nazwie) informuje
  programy z~jakim typem pliku mają do czynienia, dlatego każdy język
  programowania w~jakim języku mają do czynienia. Z~tego powodu każdy
  język programowania ma \alert{własne} rozszerzenie. Pliki napisane
  w~języku~C mają mieć rozszerzenie \texttt{.c}.

  Jak ktoś mi napisze program w~języku~C, który \alert{nie} ma
  rozszerzenia~\texttt{.c}, będzie to musiał przy mnie poprawić.
  (Chyba, że~będę musiał od razu biec do kogoś innego. Ale proszę nie
  liczyć, że~zapomnę o~tej zniewadze. ;))

  Brak tego rozszerzenia sprawia, że~możemy zostać pozbawienie części
  wsparcia w~pracy z~C, jakie dają nam przeróżne programy. A~do tego
  \alert{nie} możemy dopuścić.

\end{frame}
% ##################























% ####################################################################
% ####################################################################
% Bibliography

\printbibliography





% ############################
% End of the document

\end{document}
