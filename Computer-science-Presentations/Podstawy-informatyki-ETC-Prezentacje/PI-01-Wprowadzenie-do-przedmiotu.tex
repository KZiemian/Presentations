% ------------------------------------------------------------------------------------------------------------------
% Basic configuration of Beamera class and Jagiellonian theme
% ------------------------------------------------------------------------------------------------------------------
\RequirePackage[l2tabu, orthodox]{nag}



\ifx\PresentationStyle\notset
  \def\PresentationStyle{dark}
\fi



% Options: t -- align text to the top of the frame
\documentclass[10pt,t]{beamer}
\mode<presentation>
\usetheme[style=\PresentationStyle]{jagiellonian}





% ------------------------------------------------------------------------------------
% Procesing configuration files of Jagiellonian theme located
% in the directory "preambule"
% ------------------------------------------------------------------------------------
% Configuration for polish language
% Need description
\usepackage[polish]{babel}
% Need description
\usepackage[MeX]{polski}



% ------------------------------
% Better support of polish chars in technical parts of PDF
% ------------------------------
\hypersetup{pdfencoding=auto,psdextra}

% Package "textpos" give as enviroment "textblock" which is very usefull in
% arranging text on slides.

% This is standard configuration of "textpos"
\usepackage[overlay,absolute]{textpos}

% If you need to see bounds of "textblock's" comment line above and uncomment
% one below.

% Caution! When showboxes option is on significant ammunt of space is add
% to the top of textblock and as such, everyting put in them gone down.
% We need to check how to remove this bug.

% \usepackage[showboxes,overlay,absolute]{textpos}



% Setting scale length for package "textpos"
\setlength{\TPHorizModule}{10mm}
\setlength{\TPVertModule}{\TPHorizModule}


% ---------------------------------------
% Packages written for lectures "Geometria 3D dla twórców gier wideo"
% ---------------------------------------
% \usepackage{./Geometry3DPackages/Geometry3D}
% \usepackage{./Geometry3DPackages/Geometry3DIndices}
% \usepackage{./Geometry3DPackages/Geometry3DTikZStyle}
% \usepackage{./ProgramowanieSymulacjiFizykiPaczki/ProgramowanieSymulacjiFizykiTikZStyle}
% \usepackage{./Geometry3DPackages/mathcommands}


% ---------------------------------------
% TikZ
% ---------------------------------------
% Importing TikZ libraries
\usetikzlibrary{arrows.meta}
\usetikzlibrary{positioning}





% % Configuration package "bm" that need for making bold symbols
% \newcommand{\bmmax}{0}
% \newcommand{\hmmax}{0}
% \usepackage{bm}




% ---------------------------------------
% Packages for scientific texts
% ---------------------------------------
% \let\lll\undefined  % Sometimes you must use this line to allow
% "amsmath" package to works with packages with packages for polish
% languge imported
% /preambul/LanguageSettings/JagiellonianPolishLanguageSettings.tex.
% This comments (probably) removes polish letter Ł.
\usepackage{amsmath}  % Packages from American Mathematical Society (AMS)
\usepackage{amssymb}
\usepackage{amscd}
\usepackage{amsthm}
\usepackage{siunitx}  % Package for typsetting SI units.
\usepackage{upgreek}  % Better looking greek letters.
% Example of using upgreek: pi = \uppi


\usepackage{calrsfs}  % Zmienia czcionkę kaligraficzną w \mathcal
% na ładniejszą. Może w innych miejscach robi to samo, ale o tym nic
% nie wiem.










% ---------------------------------------
% Packages written for lectures "Geometria 3D dla twórców gier wideo"
% ---------------------------------------
% \usepackage{./ProgramowanieSymulacjiFizykiPaczki/ProgramowanieSymulacjiFizyki}
% \usepackage{./ProgramowanieSymulacjiFizykiPaczki/ProgramowanieSymulacjiFizykiIndeksy}
% \usepackage{./ProgramowanieSymulacjiFizykiPaczki/ProgramowanieSymulacjiFizykiTikZStyle}





% !!!!!!!!!!!!!!!!!!!!!!!!!!!!!!
% !!!!!!!!!!!!!!!!!!!!!!!!!!!!!!
% EVIL STUFF
\if\JUlogotitle1
\edef\LogoJUPath{LogoJU_\JUlogoLang/LogoJU_\JUlogoShape_\JUlogoColor.pdf}
\titlegraphic{\hfill\includegraphics[scale=0.22]
{./JagiellonianPictures/\LogoJUPath}}
\fi
% ---------------------------------------
% Commands for handling colors
% ---------------------------------------


% Command for setting normal text color for some text in math modestyle
% Text color depend on used style of Jagiellonian

% Beamer version of command
\newcommand{\TextWithNormalTextColor}[1]{%
  {\color{jNormalTextFGColor}
    \setbeamercolor{math text}{fg=jNormalTextFGColor} {#1}}
}

% Article and similar classes version of command
% \newcommand{\TextWithNormalTextColor}[1]{%
%   {\color{jNormalTextsFGColor} {#1}}
% }



% Beamer version of command
\newcommand{\NormalTextInMathMode}[1]{%
  {\color{jNormalTextFGColor}
    \setbeamercolor{math text}{fg=jNormalTextFGColor} \text{#1}}
}


% Article and similar classes version of command
% \newcommand{\NormalTextInMathMode}[1]{%
%   {\color{jNormalTextsFGColor} \text{#1}}
% }




% Command that sets color of some mathematical text to the same color
% that has normal text in header (?)

% Beamer version of the command
\newcommand{\MathTextFrametitleFGColor}[1]{%
  {\color{jFrametitleFGColor}
    \setbeamercolor{math text}{fg=jFrametitleFGColor} #1}
}

% Article and similar classes version of the command
% \newcommand{\MathTextWhiteColor}[1]{{\color{jFrametitleFGColor} #1}}





% Command for setting color of alert text for some text in math modestyle

% Beamer version of the command
\newcommand{\MathTextAlertColor}[1]{%
  {\color{jOrange} \setbeamercolor{math text}{fg=jOrange} #1}
}

% Article and similar classes version of the command
% \newcommand{\MathTextAlertColor}[1]{{\color{jOrange} #1}}





% Command that allow you to sets chosen color as the color of some text into
% math mode. Due to some nuances in the way that Beamer handle colors
% it not work in all cases. We hope that in the future we will improve it.

% Beamer version of the command
\newcommand{\SetMathTextColor}[2]{%
  {\color{#1} \setbeamercolor{math text}{fg=#1} #2}
}


% Article and similar classes version of the command
% \newcommand{\SetMathTextColor}[2]{{\color{#1} #2}}










% ---------------------------------------
% Commands for few special slides
% ---------------------------------------
\newcommand{\EndingSlide}[1]{%
  \begin{frame}[standout]

    \begingroup

    \color{jFrametitleFGColor}

    #1

    \endgroup

  \end{frame}
}










% ---------------------------------------
% Commands for setting background pictures for some slides
% ---------------------------------------
\newcommand{\TitleBackgroundPicture}
{./JagiellonianPictures/Backgrounds/LajkonikDark.png}
\newcommand{\SectionBackgroundPicture}
{./JagiellonianPictures/Backgrounds/LajkonikLight.png}



\newcommand{\TitleSlideWithPicture}{%
  \begingroup

  \usebackgroundtemplate{%
    \includegraphics[height=\paperheight]{\TitleBackgroundPicture}}

  \maketitle

  \endgroup
}





\newcommand{\SectionSlideWithPicture}[1]{%
  \begingroup

  \usebackgroundtemplate{%
    \includegraphics[height=\paperheight]{\SectionBackgroundPicture}}

  \setbeamercolor{titlelike}{fg=normal text.fg}

  \section{#1}

  \endgroup
}










% ---------------------------------------
% Commands for lectures "Geometria 3D dla twórców gier wideo"
% Polish version
% ---------------------------------------
% Komendy teraz wykomentowane były potrzebne, gdy loga były na niebieskim
% tle, nie na białym. A są na białym bo tego chcieli w biurze projektu.
% \newcommand{\FundingLogoWhitePicturePL}
% {./PresentationPictures/CommonPictures/logotypFundusze_biale_bez_tla2.pdf}
\newcommand{\FundingLogoColorPicturePL}
{./PresentationPictures/CommonPictures/European_Funds_color_PL.pdf}
% \newcommand{\EULogoWhitePicturePL}
% {./PresentationPictures/CommonPictures/logotypUE_biale_bez_tla2.pdf}
\newcommand{\EUSocialFundLogoColorPicturePL}
{./PresentationPictures/CommonPictures/EU_Social_Fund_color_PL.pdf}
% \newcommand{\ZintegrUJLogoWhitePicturePL}
% {./PresentationPictures/CommonPictures/zintegruj-logo-white.pdf}
\newcommand{\ZintegrUJLogoColorPicturePL}
{./PresentationPictures/CommonPictures/ZintegrUJ_color.pdf}
\newcommand{\JULogoColorPicturePL}
{./JagiellonianPictures/LogoJU_PL/LogoJU_A_color.pdf}





\newcommand{\GeometryThreeDSpecialBeginningSlidePL}{%
  \begin{frame}[standout]

    \begin{textblock}{11}(1,0.7)

      \begin{flushleft}

        \mdseries

        \footnotesize

        \color{jFrametitleFGColor}

        Materiał powstał w ramach projektu współfinansowanego ze środków
        Unii Europejskiej w ramach Europejskiego Funduszu Społecznego
        POWR.03.05.00-00-Z309/17-00.

      \end{flushleft}

    \end{textblock}





    \begin{textblock}{10}(0,2.2)

      \tikz \fill[color=jBackgroundStyleLight] (0,0) rectangle (12.8,-1.5);

    \end{textblock}


    \begin{textblock}{3.2}(1,2.45)

      \includegraphics[scale=0.3]{\FundingLogoColorPicturePL}

    \end{textblock}


    \begin{textblock}{2.5}(3.7,2.5)

      \includegraphics[scale=0.2]{\JULogoColorPicturePL}

    \end{textblock}


    \begin{textblock}{2.5}(6,2.4)

      \includegraphics[scale=0.1]{\ZintegrUJLogoColorPicturePL}

    \end{textblock}


    \begin{textblock}{4.2}(8.4,2.6)

      \includegraphics[scale=0.3]{\EUSocialFundLogoColorPicturePL}

    \end{textblock}

  \end{frame}
}



\newcommand{\GeometryThreeDTwoSpecialBeginningSlidesPL}{%
  \begin{frame}[standout]

    \begin{textblock}{11}(1,0.7)

      \begin{flushleft}

        \mdseries

        \footnotesize

        \color{jFrametitleFGColor}

        Materiał powstał w ramach projektu współfinansowanego ze środków
        Unii Europejskiej w ramach Europejskiego Funduszu Społecznego
        POWR.03.05.00-00-Z309/17-00.

      \end{flushleft}

    \end{textblock}





    \begin{textblock}{10}(0,2.2)

      \tikz \fill[color=jBackgroundStyleLight] (0,0) rectangle (12.8,-1.5);

    \end{textblock}


    \begin{textblock}{3.2}(1,2.45)

      \includegraphics[scale=0.3]{\FundingLogoColorPicturePL}

    \end{textblock}


    \begin{textblock}{2.5}(3.7,2.5)

      \includegraphics[scale=0.2]{\JULogoColorPicturePL}

    \end{textblock}


    \begin{textblock}{2.5}(6,2.4)

      \includegraphics[scale=0.1]{\ZintegrUJLogoColorPicturePL}

    \end{textblock}


    \begin{textblock}{4.2}(8.4,2.6)

      \includegraphics[scale=0.3]{\EUSocialFundLogoColorPicturePL}

    \end{textblock}

  \end{frame}





  \TitleSlideWithPicture
}



\newcommand{\GeometryThreeDSpecialEndingSlidePL}{%
  \begin{frame}[standout]

    \begin{textblock}{11}(1,0.7)

      \begin{flushleft}

        \mdseries

        \footnotesize

        \color{jFrametitleFGColor}

        Materiał powstał w ramach projektu współfinansowanego ze środków
        Unii Europejskiej w~ramach Europejskiego Funduszu Społecznego
        POWR.03.05.00-00-Z309/17-00.

      \end{flushleft}

    \end{textblock}





    \begin{textblock}{10}(0,2.2)

      \tikz \fill[color=jBackgroundStyleLight] (0,0) rectangle (12.8,-1.5);

    \end{textblock}


    \begin{textblock}{3.2}(1,2.45)

      \includegraphics[scale=0.3]{\FundingLogoColorPicturePL}

    \end{textblock}


    \begin{textblock}{2.5}(3.7,2.5)

      \includegraphics[scale=0.2]{\JULogoColorPicturePL}

    \end{textblock}


    \begin{textblock}{2.5}(6,2.4)

      \includegraphics[scale=0.1]{\ZintegrUJLogoColorPicturePL}

    \end{textblock}


    \begin{textblock}{4.2}(8.4,2.6)

      \includegraphics[scale=0.3]{\EUSocialFundLogoColorPicturePL}

    \end{textblock}





    \begin{textblock}{11}(1,4)

      \begin{flushleft}

        \mdseries

        \footnotesize

        \RaggedRight

        \color{jFrametitleFGColor}

        Treść niniejszego wykładu jest udostępniona na~licencji
        Creative Commons (\textsc{cc}), z~uzna\-niem autorstwa
        (\textsc{by}) oraz udostępnianiem na tych samych warunkach
        (\textsc{sa}). Rysunki i~wy\-kresy zawarte w~wykładzie są
        autorstwa dr.~hab.~Pawła Węgrzyna et~al. i~są dostępne
        na tej samej licencji, o~ile nie wskazano inaczej.
        W~prezentacji wykorzystano temat Beamera Jagiellonian,
        oparty na~temacie Metropolis Matthiasa Vogelgesanga,
        dostępnym na licencji \LaTeX{} Project Public License~1.3c
        pod adresem: \colorhref{https://github.com/matze/mtheme}
        {https://github.com/matze/mtheme}.

        Projekt typograficzny: Iwona Grabska-Gradzińska \\
        Skład: Kamil Ziemian;
        Korekta: Wojciech Palacz \\
        Modele: Dariusz Frymus, Kamil Nowakowski \\
        Rysunki i~wykresy: Kamil Ziemian, Paweł Węgrzyn, Wojciech Palacz

      \end{flushleft}

    \end{textblock}

  \end{frame}
}



\newcommand{\GeometryThreeDTwoSpecialEndingSlidesPL}[1]{%
  \begin{frame}[standout]


    \begin{textblock}{11}(1,0.7)

      \begin{flushleft}

        \mdseries

        \footnotesize

        \color{jFrametitleFGColor}

        Materiał powstał w ramach projektu współfinansowanego ze środków
        Unii Europejskiej w~ramach Europejskiego Funduszu Społecznego
        POWR.03.05.00-00-Z309/17-00.

      \end{flushleft}

    \end{textblock}





    \begin{textblock}{10}(0,2.2)

      \tikz \fill[color=jBackgroundStyleLight] (0,0) rectangle (12.8,-1.5);

    \end{textblock}


    \begin{textblock}{3.2}(1,2.45)

      \includegraphics[scale=0.3]{\FundingLogoColorPicturePL}

    \end{textblock}


    \begin{textblock}{2.5}(3.7,2.5)

      \includegraphics[scale=0.2]{\JULogoColorPicturePL}

    \end{textblock}


    \begin{textblock}{2.5}(6,2.4)

      \includegraphics[scale=0.1]{\ZintegrUJLogoColorPicturePL}

    \end{textblock}


    \begin{textblock}{4.2}(8.4,2.6)

      \includegraphics[scale=0.3]{\EUSocialFundLogoColorPicturePL}

    \end{textblock}





    \begin{textblock}{11}(1,4)

      \begin{flushleft}

        \mdseries

        \footnotesize

        \RaggedRight

        \color{jFrametitleFGColor}

        Treść niniejszego wykładu jest udostępniona na~licencji
        Creative Commons (\textsc{cc}), z~uzna\-niem autorstwa
        (\textsc{by}) oraz udostępnianiem na tych samych warunkach
        (\textsc{sa}). Rysunki i~wy\-kresy zawarte w~wykładzie są
        autorstwa dr.~hab.~Pawła Węgrzyna et~al. i~są dostępne
        na tej samej licencji, o~ile nie wskazano inaczej.
        W~prezentacji wykorzystano temat Beamera Jagiellonian,
        oparty na~temacie Metropolis Matthiasa Vogelgesanga,
        dostępnym na licencji \LaTeX{} Project Public License~1.3c
        pod adresem: \colorhref{https://github.com/matze/mtheme}
        {https://github.com/matze/mtheme}.

        Projekt typograficzny: Iwona Grabska-Gradzińska \\
        Skład: Kamil Ziemian;
        Korekta: Wojciech Palacz \\
        Modele: Dariusz Frymus, Kamil Nowakowski \\
        Rysunki i~wykresy: Kamil Ziemian, Paweł Węgrzyn, Wojciech Palacz

      \end{flushleft}

    \end{textblock}

  \end{frame}





  \begin{frame}[standout]

    \begingroup

    \color{jFrametitleFGColor}

    #1

    \endgroup

  \end{frame}
}



\newcommand{\GeometryThreeDSpecialEndingSlideVideoPL}{%
  \begin{frame}[standout]

    \begin{textblock}{11}(1,0.7)

      \begin{flushleft}

        \mdseries

        \footnotesize

        \color{jFrametitleFGColor}

        Materiał powstał w ramach projektu współfinansowanego ze środków
        Unii Europejskiej w~ramach Europejskiego Funduszu Społecznego
        POWR.03.05.00-00-Z309/17-00.

      \end{flushleft}

    \end{textblock}





    \begin{textblock}{10}(0,2.2)

      \tikz \fill[color=jBackgroundStyleLight] (0,0) rectangle (12.8,-1.5);

    \end{textblock}


    \begin{textblock}{3.2}(1,2.45)

      \includegraphics[scale=0.3]{\FundingLogoColorPicturePL}

    \end{textblock}


    \begin{textblock}{2.5}(3.7,2.5)

      \includegraphics[scale=0.2]{\JULogoColorPicturePL}

    \end{textblock}


    \begin{textblock}{2.5}(6,2.4)

      \includegraphics[scale=0.1]{\ZintegrUJLogoColorPicturePL}

    \end{textblock}


    \begin{textblock}{4.2}(8.4,2.6)

      \includegraphics[scale=0.3]{\EUSocialFundLogoColorPicturePL}

    \end{textblock}





    \begin{textblock}{11}(1,4)

      \begin{flushleft}

        \mdseries

        \footnotesize

        \RaggedRight

        \color{jFrametitleFGColor}

        Treść niniejszego wykładu jest udostępniona na~licencji
        Creative Commons (\textsc{cc}), z~uzna\-niem autorstwa
        (\textsc{by}) oraz udostępnianiem na tych samych warunkach
        (\textsc{sa}). Rysunki i~wy\-kresy zawarte w~wykładzie są
        autorstwa dr.~hab.~Pawła Węgrzyna et~al. i~są dostępne
        na tej samej licencji, o~ile nie wskazano inaczej.
        W~prezentacji wykorzystano temat Beamera Jagiellonian,
        oparty na~temacie Metropolis Matthiasa Vogelgesanga,
        dostępnym na licencji \LaTeX{} Project Public License~1.3c
        pod adresem: \colorhref{https://github.com/matze/mtheme}
        {https://github.com/matze/mtheme}.

        Projekt typograficzny: Iwona Grabska-Gradzińska;
        Skład: Kamil Ziemian \\
        Korekta: Wojciech Palacz;
        Modele: Dariusz Frymus, Kamil Nowakowski \\
        Rysunki i~wykresy: Kamil Ziemian, Paweł Węgrzyn, Wojciech Palacz \\
        Montaż: Agencja Filmowa Film \& Television Production~-- Zbigniew
        Masklak

      \end{flushleft}

    \end{textblock}

  \end{frame}
}





\newcommand{\GeometryThreeDTwoSpecialEndingSlidesVideoPL}[1]{%
  \begin{frame}[standout]

    \begin{textblock}{11}(1,0.7)

      \begin{flushleft}

        \mdseries

        \footnotesize

        \color{jFrametitleFGColor}

        Materiał powstał w ramach projektu współfinansowanego ze środków
        Unii Europejskiej w~ramach Europejskiego Funduszu Społecznego
        POWR.03.05.00-00-Z309/17-00.

      \end{flushleft}

    \end{textblock}





    \begin{textblock}{10}(0,2.2)

      \tikz \fill[color=jBackgroundStyleLight] (0,0) rectangle (12.8,-1.5);

    \end{textblock}


    \begin{textblock}{3.2}(1,2.45)

      \includegraphics[scale=0.3]{\FundingLogoColorPicturePL}

    \end{textblock}


    \begin{textblock}{2.5}(3.7,2.5)

      \includegraphics[scale=0.2]{\JULogoColorPicturePL}

    \end{textblock}


    \begin{textblock}{2.5}(6,2.4)

      \includegraphics[scale=0.1]{\ZintegrUJLogoColorPicturePL}

    \end{textblock}


    \begin{textblock}{4.2}(8.4,2.6)

      \includegraphics[scale=0.3]{\EUSocialFundLogoColorPicturePL}

    \end{textblock}





    \begin{textblock}{11}(1,4)

      \begin{flushleft}

        \mdseries

        \footnotesize

        \RaggedRight

        \color{jFrametitleFGColor}

        Treść niniejszego wykładu jest udostępniona na~licencji
        Creative Commons (\textsc{cc}), z~uzna\-niem autorstwa
        (\textsc{by}) oraz udostępnianiem na tych samych warunkach
        (\textsc{sa}). Rysunki i~wy\-kresy zawarte w~wykładzie są
        autorstwa dr.~hab.~Pawła Węgrzyna et~al. i~są dostępne
        na tej samej licencji, o~ile nie wskazano inaczej.
        W~prezentacji wykorzystano temat Beamera Jagiellonian,
        oparty na~temacie Metropolis Matthiasa Vogelgesanga,
        dostępnym na licencji \LaTeX{} Project Public License~1.3c
        pod adresem: \colorhref{https://github.com/matze/mtheme}
        {https://github.com/matze/mtheme}.

        Projekt typograficzny: Iwona Grabska-Gradzińska;
        Skład: Kamil Ziemian \\
        Korekta: Wojciech Palacz;
        Modele: Dariusz Frymus, Kamil Nowakowski \\
        Rysunki i~wykresy: Kamil Ziemian, Paweł Węgrzyn, Wojciech Palacz \\
        Montaż: Agencja Filmowa Film \& Television Production~-- Zbigniew
        Masklak

      \end{flushleft}

    \end{textblock}

  \end{frame}





  \begin{frame}[standout]


    \begingroup

    \color{jFrametitleFGColor}

    #1

    \endgroup

  \end{frame}
}










% ---------------------------------------
% Commands for lectures "Geometria 3D dla twórców gier wideo"
% English version
% ---------------------------------------
% \newcommand{\FundingLogoWhitePictureEN}
% {./PresentationPictures/CommonPictures/logotypFundusze_biale_bez_tla2.pdf}
\newcommand{\FundingLogoColorPictureEN}
{./PresentationPictures/CommonPictures/European_Funds_color_EN.pdf}
% \newcommand{\EULogoWhitePictureEN}
% {./PresentationPictures/CommonPictures/logotypUE_biale_bez_tla2.pdf}
\newcommand{\EUSocialFundLogoColorPictureEN}
{./PresentationPictures/CommonPictures/EU_Social_Fund_color_EN.pdf}
% \newcommand{\ZintegrUJLogoWhitePictureEN}
% {./PresentationPictures/CommonPictures/zintegruj-logo-white.pdf}
\newcommand{\ZintegrUJLogoColorPictureEN}
{./PresentationPictures/CommonPictures/ZintegrUJ_color.pdf}
\newcommand{\JULogoColorPictureEN}
{./JagiellonianPictures/LogoJU_EN/LogoJU_A_color.pdf}



\newcommand{\GeometryThreeDSpecialBeginningSlideEN}{%
  \begin{frame}[standout]

    \begin{textblock}{11}(1,0.7)

      \begin{flushleft}

        \mdseries

        \footnotesize

        \color{jFrametitleFGColor}

        This content was created as part of a project co-financed by the
        European Union within the framework of the European Social Fund
        POWR.03.05.00-00-Z309/17-00.

      \end{flushleft}

    \end{textblock}





    \begin{textblock}{10}(0,2.2)

      \tikz \fill[color=jBackgroundStyleLight] (0,0) rectangle (12.8,-1.5);

    \end{textblock}


    \begin{textblock}{3.2}(0.7,2.45)

      \includegraphics[scale=0.3]{\FundingLogoColorPictureEN}

    \end{textblock}


    \begin{textblock}{2.5}(4.15,2.5)

      \includegraphics[scale=0.2]{\JULogoColorPictureEN}

    \end{textblock}


    \begin{textblock}{2.5}(6.35,2.4)

      \includegraphics[scale=0.1]{\ZintegrUJLogoColorPictureEN}

    \end{textblock}


    \begin{textblock}{4.2}(8.4,2.6)

      \includegraphics[scale=0.3]{\EUSocialFundLogoColorPictureEN}

    \end{textblock}

  \end{frame}
}



\newcommand{\GeometryThreeDTwoSpecialBeginningSlidesEN}{%
  \begin{frame}[standout]

    \begin{textblock}{11}(1,0.7)

      \begin{flushleft}

        \mdseries

        \footnotesize

        \color{jFrametitleFGColor}

        This content was created as part of a project co-financed by the
        European Union within the framework of the European Social Fund
        POWR.03.05.00-00-Z309/17-00.

      \end{flushleft}

    \end{textblock}





    \begin{textblock}{10}(0,2.2)

      \tikz \fill[color=jBackgroundStyleLight] (0,0) rectangle (12.8,-1.5);

    \end{textblock}


    \begin{textblock}{3.2}(0.7,2.45)

      \includegraphics[scale=0.3]{\FundingLogoColorPictureEN}

    \end{textblock}


    \begin{textblock}{2.5}(4.15,2.5)

      \includegraphics[scale=0.2]{\JULogoColorPictureEN}

    \end{textblock}


    \begin{textblock}{2.5}(6.35,2.4)

      \includegraphics[scale=0.1]{\ZintegrUJLogoColorPictureEN}

    \end{textblock}


    \begin{textblock}{4.2}(8.4,2.6)

      \includegraphics[scale=0.3]{\EUSocialFundLogoColorPictureEN}

    \end{textblock}

  \end{frame}





  \TitleSlideWithPicture
}



\newcommand{\GeometryThreeDSpecialEndingSlideEN}{%
  \begin{frame}[standout]

    \begin{textblock}{11}(1,0.7)

      \begin{flushleft}

        \mdseries

        \footnotesize

        \color{jFrametitleFGColor}

        This content was created as part of a project co-financed by the
        European Union within the framework of the European Social Fund
        POWR.03.05.00-00-Z309/17-00.

      \end{flushleft}

    \end{textblock}





    \begin{textblock}{10}(0,2.2)

      \tikz \fill[color=jBackgroundStyleLight] (0,0) rectangle (12.8,-1.5);

    \end{textblock}


    \begin{textblock}{3.2}(0.7,2.45)

      \includegraphics[scale=0.3]{\FundingLogoColorPictureEN}

    \end{textblock}


    \begin{textblock}{2.5}(4.15,2.5)

      \includegraphics[scale=0.2]{\JULogoColorPictureEN}

    \end{textblock}


    \begin{textblock}{2.5}(6.35,2.4)

      \includegraphics[scale=0.1]{\ZintegrUJLogoColorPictureEN}

    \end{textblock}


    \begin{textblock}{4.2}(8.4,2.6)

      \includegraphics[scale=0.3]{\EUSocialFundLogoColorPictureEN}

    \end{textblock}





    \begin{textblock}{11}(1,4)

      \begin{flushleft}

        \mdseries

        \footnotesize

        \RaggedRight

        \color{jFrametitleFGColor}

        The content of this lecture is made available under a~Creative
        Commons licence (\textsc{cc}), giving the author the credits
        (\textsc{by}) and putting an obligation to share on the same terms
        (\textsc{sa}). Figures and diagrams included in the lecture are
        authored by Paweł Węgrzyn et~al., and are available under the same
        license unless indicated otherwise.\\ The presentation uses the
        Beamer Jagiellonian theme based on Matthias Vogelgesang’s
        Metropolis theme, available under license \LaTeX{} Project
        Public License~1.3c at: \colorhref{https://github.com/matze/mtheme}
        {https://github.com/matze/mtheme}.

        Typographic design: Iwona Grabska-Gradzińska \\
        \LaTeX{} Typesetting: Kamil Ziemian \\
        Proofreading: Wojciech Palacz,
        Monika Stawicka \\
        3D Models: Dariusz Frymus, Kamil Nowakowski \\
        Figures and charts: Kamil Ziemian, Paweł Węgrzyn, Wojciech Palacz

      \end{flushleft}

    \end{textblock}

  \end{frame}
}



\newcommand{\GeometryThreeDTwoSpecialEndingSlidesEN}[1]{%
  \begin{frame}[standout]


    \begin{textblock}{11}(1,0.7)

      \begin{flushleft}

        \mdseries

        \footnotesize

        \color{jFrametitleFGColor}

        This content was created as part of a project co-financed by the
        European Union within the framework of the European Social Fund
        POWR.03.05.00-00-Z309/17-00.

      \end{flushleft}

    \end{textblock}





    \begin{textblock}{10}(0,2.2)

      \tikz \fill[color=jBackgroundStyleLight] (0,0) rectangle (12.8,-1.5);

    \end{textblock}


    \begin{textblock}{3.2}(0.7,2.45)

      \includegraphics[scale=0.3]{\FundingLogoColorPictureEN}

    \end{textblock}


    \begin{textblock}{2.5}(4.15,2.5)

      \includegraphics[scale=0.2]{\JULogoColorPictureEN}

    \end{textblock}


    \begin{textblock}{2.5}(6.35,2.4)

      \includegraphics[scale=0.1]{\ZintegrUJLogoColorPictureEN}

    \end{textblock}


    \begin{textblock}{4.2}(8.4,2.6)

      \includegraphics[scale=0.3]{\EUSocialFundLogoColorPictureEN}

    \end{textblock}





    \begin{textblock}{11}(1,4)

      \begin{flushleft}

        \mdseries

        \footnotesize

        \RaggedRight

        \color{jFrametitleFGColor}

        The content of this lecture is made available under a~Creative
        Commons licence (\textsc{cc}), giving the author the credits
        (\textsc{by}) and putting an obligation to share on the same terms
        (\textsc{sa}). Figures and diagrams included in the lecture are
        authored by Paweł Węgrzyn et~al., and are available under the same
        license unless indicated otherwise.\\ The presentation uses the
        Beamer Jagiellonian theme based on Matthias Vogelgesang’s
        Metropolis theme, available under license \LaTeX{} Project
        Public License~1.3c at: \colorhref{https://github.com/matze/mtheme}
        {https://github.com/matze/mtheme}.

        Typographic design: Iwona Grabska-Gradzińska \\
        \LaTeX{} Typesetting: Kamil Ziemian \\
        Proofreading: Wojciech Palacz,
        Monika Stawicka \\
        3D Models: Dariusz Frymus, Kamil Nowakowski \\
        Figures and charts: Kamil Ziemian, Paweł Węgrzyn, Wojciech Palacz

      \end{flushleft}

    \end{textblock}

  \end{frame}





  \begin{frame}[standout]

    \begingroup

    \color{jFrametitleFGColor}

    #1

    \endgroup

  \end{frame}
}



\newcommand{\GeometryThreeDSpecialEndingSlideVideoVerOneEN}{%
  \begin{frame}[standout]

    \begin{textblock}{11}(1,0.7)

      \begin{flushleft}

        \mdseries

        \footnotesize

        \color{jFrametitleFGColor}

        This content was created as part of a project co-financed by the
        European Union within the framework of the European Social Fund
        POWR.03.05.00-00-Z309/17-00.

      \end{flushleft}

    \end{textblock}





    \begin{textblock}{10}(0,2.2)

      \tikz \fill[color=jBackgroundStyleLight] (0,0) rectangle (12.8,-1.5);

    \end{textblock}


    \begin{textblock}{3.2}(0.7,2.45)

      \includegraphics[scale=0.3]{\FundingLogoColorPictureEN}

    \end{textblock}


    \begin{textblock}{2.5}(4.15,2.5)

      \includegraphics[scale=0.2]{\JULogoColorPictureEN}

    \end{textblock}


    \begin{textblock}{2.5}(6.35,2.4)

      \includegraphics[scale=0.1]{\ZintegrUJLogoColorPictureEN}

    \end{textblock}


    \begin{textblock}{4.2}(8.4,2.6)

      \includegraphics[scale=0.3]{\EUSocialFundLogoColorPictureEN}

    \end{textblock}





    \begin{textblock}{11}(1,4)

      \begin{flushleft}

        \mdseries

        \footnotesize

        \RaggedRight

        \color{jFrametitleFGColor}

        The content of this lecture is made available under a Creative
        Commons licence (\textsc{cc}), giving the author the credits
        (\textsc{by}) and putting an obligation to share on the same terms
        (\textsc{sa}). Figures and diagrams included in the lecture are
        authored by Paweł Węgrzyn et~al., and are available under the same
        license unless indicated otherwise.\\ The presentation uses the
        Beamer Jagiellonian theme based on Matthias Vogelgesang’s
        Metropolis theme, available under license \LaTeX{} Project
        Public License~1.3c at: \colorhref{https://github.com/matze/mtheme}
        {https://github.com/matze/mtheme}.

        Typographic design: Iwona Grabska-Gradzińska;
        \LaTeX{} Typesetting: Kamil Ziemian \\
        Proofreading: Wojciech Palacz,
        Monika Stawicka \\
        3D Models: Dariusz Frymus, Kamil Nowakowski \\
        Figures and charts: Kamil Ziemian, Paweł Węgrzyn, Wojciech
        Palacz \\
        Film editing: Agencja Filmowa Film \& Television Production~--
        Zbigniew Masklak

      \end{flushleft}

    \end{textblock}

  \end{frame}
}



\newcommand{\GeometryThreeDSpecialEndingSlideVideoVerTwoEN}{%
  \begin{frame}[standout]

    \begin{textblock}{11}(1,0.7)

      \begin{flushleft}

        \mdseries

        \footnotesize

        \color{jFrametitleFGColor}

        This content was created as part of a project co-financed by the
        European Union within the framework of the European Social Fund
        POWR.03.05.00-00-Z309/17-00.

      \end{flushleft}

    \end{textblock}





    \begin{textblock}{10}(0,2.2)

      \tikz \fill[color=jBackgroundStyleLight] (0,0) rectangle (12.8,-1.5);

    \end{textblock}


    \begin{textblock}{3.2}(0.7,2.45)

      \includegraphics[scale=0.3]{\FundingLogoColorPictureEN}

    \end{textblock}


    \begin{textblock}{2.5}(4.15,2.5)

      \includegraphics[scale=0.2]{\JULogoColorPictureEN}

    \end{textblock}


    \begin{textblock}{2.5}(6.35,2.4)

      \includegraphics[scale=0.1]{\ZintegrUJLogoColorPictureEN}

    \end{textblock}


    \begin{textblock}{4.2}(8.4,2.6)

      \includegraphics[scale=0.3]{\EUSocialFundLogoColorPictureEN}

    \end{textblock}





    \begin{textblock}{11}(1,4)

      \begin{flushleft}

        \mdseries

        \footnotesize

        \RaggedRight

        \color{jFrametitleFGColor}

        The content of this lecture is made available under a Creative
        Commons licence (\textsc{cc}), giving the author the credits
        (\textsc{by}) and putting an obligation to share on the same terms
        (\textsc{sa}). Figures and diagrams included in the lecture are
        authored by Paweł Węgrzyn et~al., and are available under the same
        license unless indicated otherwise.\\ The presentation uses the
        Beamer Jagiellonian theme based on Matthias Vogelgesang’s
        Metropolis theme, available under license \LaTeX{} Project
        Public License~1.3c at: \colorhref{https://github.com/matze/mtheme}
        {https://github.com/matze/mtheme}.

        Typographic design: Iwona Grabska-Gradzińska;
        \LaTeX{} Typesetting: Kamil Ziemian \\
        Proofreading: Wojciech Palacz,
        Monika Stawicka \\
        3D Models: Dariusz Frymus, Kamil Nowakowski \\
        Figures and charts: Kamil Ziemian, Paweł Węgrzyn, Wojciech
        Palacz \\
        Film editing: IMAVI -- Joanna Kozakiewicz, Krzysztof Magda, Nikodem
        Frodyma

      \end{flushleft}

    \end{textblock}

  \end{frame}
}



\newcommand{\GeometryThreeDSpecialEndingSlideVideoVerThreeEN}{%
  \begin{frame}[standout]

    \begin{textblock}{11}(1,0.7)

      \begin{flushleft}

        \mdseries

        \footnotesize

        \color{jFrametitleFGColor}

        This content was created as part of a project co-financed by the
        European Union within the framework of the European Social Fund
        POWR.03.05.00-00-Z309/17-00.

      \end{flushleft}

    \end{textblock}





    \begin{textblock}{10}(0,2.2)

      \tikz \fill[color=jBackgroundStyleLight] (0,0) rectangle (12.8,-1.5);

    \end{textblock}


    \begin{textblock}{3.2}(0.7,2.45)

      \includegraphics[scale=0.3]{\FundingLogoColorPictureEN}

    \end{textblock}


    \begin{textblock}{2.5}(4.15,2.5)

      \includegraphics[scale=0.2]{\JULogoColorPictureEN}

    \end{textblock}


    \begin{textblock}{2.5}(6.35,2.4)

      \includegraphics[scale=0.1]{\ZintegrUJLogoColorPictureEN}

    \end{textblock}


    \begin{textblock}{4.2}(8.4,2.6)

      \includegraphics[scale=0.3]{\EUSocialFundLogoColorPictureEN}

    \end{textblock}





    \begin{textblock}{11}(1,4)

      \begin{flushleft}

        \mdseries

        \footnotesize

        \RaggedRight

        \color{jFrametitleFGColor}

        The content of this lecture is made available under a Creative
        Commons licence (\textsc{cc}), giving the author the credits
        (\textsc{by}) and putting an obligation to share on the same terms
        (\textsc{sa}). Figures and diagrams included in the lecture are
        authored by Paweł Węgrzyn et~al., and are available under the same
        license unless indicated otherwise.\\ The presentation uses the
        Beamer Jagiellonian theme based on Matthias Vogelgesang’s
        Metropolis theme, available under license \LaTeX{} Project
        Public License~1.3c at: \colorhref{https://github.com/matze/mtheme}
        {https://github.com/matze/mtheme}.

        Typographic design: Iwona Grabska-Gradzińska;
        \LaTeX{} Typesetting: Kamil Ziemian \\
        Proofreading: Wojciech Palacz,
        Monika Stawicka \\
        3D Models: Dariusz Frymus, Kamil Nowakowski \\
        Figures and charts: Kamil Ziemian, Paweł Węgrzyn, Wojciech
        Palacz \\
        Film editing: Agencja Filmowa Film \& Television Production~--
        Zbigniew Masklak \\
        Film editing: IMAVI -- Joanna Kozakiewicz, Krzysztof Magda, Nikodem
        Frodyma

      \end{flushleft}

    \end{textblock}

  \end{frame}
}



\newcommand{\GeometryThreeDTwoSpecialEndingSlidesVideoVerOneEN}[1]{%
  \begin{frame}[standout]

    \begin{textblock}{11}(1,0.7)

      \begin{flushleft}

        \mdseries

        \footnotesize

        \color{jFrametitleFGColor}

        This content was created as part of a project co-financed by the
        European Union within the framework of the European Social Fund
        POWR.03.05.00-00-Z309/17-00.

      \end{flushleft}

    \end{textblock}





    \begin{textblock}{10}(0,2.2)

      \tikz \fill[color=jBackgroundStyleLight] (0,0) rectangle (12.8,-1.5);

    \end{textblock}


    \begin{textblock}{3.2}(0.7,2.45)

      \includegraphics[scale=0.3]{\FundingLogoColorPictureEN}

    \end{textblock}


    \begin{textblock}{2.5}(4.15,2.5)

      \includegraphics[scale=0.2]{\JULogoColorPictureEN}

    \end{textblock}


    \begin{textblock}{2.5}(6.35,2.4)

      \includegraphics[scale=0.1]{\ZintegrUJLogoColorPictureEN}

    \end{textblock}


    \begin{textblock}{4.2}(8.4,2.6)

      \includegraphics[scale=0.3]{\EUSocialFundLogoColorPictureEN}

    \end{textblock}





    \begin{textblock}{11}(1,4)

      \begin{flushleft}

        \mdseries

        \footnotesize

        \RaggedRight

        \color{jFrametitleFGColor}

        The content of this lecture is made available under a Creative
        Commons licence (\textsc{cc}), giving the author the credits
        (\textsc{by}) and putting an obligation to share on the same terms
        (\textsc{sa}). Figures and diagrams included in the lecture are
        authored by Paweł Węgrzyn et~al., and are available under the same
        license unless indicated otherwise.\\ The presentation uses the
        Beamer Jagiellonian theme based on Matthias Vogelgesang’s
        Metropolis theme, available under license \LaTeX{} Project
        Public License~1.3c at: \colorhref{https://github.com/matze/mtheme}
        {https://github.com/matze/mtheme}.

        Typographic design: Iwona Grabska-Gradzińska;
        \LaTeX{} Typesetting: Kamil Ziemian \\
        Proofreading: Wojciech Palacz,
        Monika Stawicka \\
        3D Models: Dariusz Frymus, Kamil Nowakowski \\
        Figures and charts: Kamil Ziemian, Paweł Węgrzyn,
        Wojciech Palacz \\
        Film editing: Agencja Filmowa Film \& Television Production~--
        Zbigniew Masklak

      \end{flushleft}

    \end{textblock}

  \end{frame}





  \begin{frame}[standout]


    \begingroup

    \color{jFrametitleFGColor}

    #1

    \endgroup

  \end{frame}
}



\newcommand{\GeometryThreeDTwoSpecialEndingSlidesVideoVerTwoEN}[1]{%
  \begin{frame}[standout]

    \begin{textblock}{11}(1,0.7)

      \begin{flushleft}

        \mdseries

        \footnotesize

        \color{jFrametitleFGColor}

        This content was created as part of a project co-financed by the
        European Union within the framework of the European Social Fund
        POWR.03.05.00-00-Z309/17-00.

      \end{flushleft}

    \end{textblock}





    \begin{textblock}{10}(0,2.2)

      \tikz \fill[color=jBackgroundStyleLight] (0,0) rectangle (12.8,-1.5);

    \end{textblock}


    \begin{textblock}{3.2}(0.7,2.45)

      \includegraphics[scale=0.3]{\FundingLogoColorPictureEN}

    \end{textblock}


    \begin{textblock}{2.5}(4.15,2.5)

      \includegraphics[scale=0.2]{\JULogoColorPictureEN}

    \end{textblock}


    \begin{textblock}{2.5}(6.35,2.4)

      \includegraphics[scale=0.1]{\ZintegrUJLogoColorPictureEN}

    \end{textblock}


    \begin{textblock}{4.2}(8.4,2.6)

      \includegraphics[scale=0.3]{\EUSocialFundLogoColorPictureEN}

    \end{textblock}





    \begin{textblock}{11}(1,4)

      \begin{flushleft}

        \mdseries

        \footnotesize

        \RaggedRight

        \color{jFrametitleFGColor}

        The content of this lecture is made available under a Creative
        Commons licence (\textsc{cc}), giving the author the credits
        (\textsc{by}) and putting an obligation to share on the same terms
        (\textsc{sa}). Figures and diagrams included in the lecture are
        authored by Paweł Węgrzyn et~al., and are available under the same
        license unless indicated otherwise.\\ The presentation uses the
        Beamer Jagiellonian theme based on Matthias Vogelgesang’s
        Metropolis theme, available under license \LaTeX{} Project
        Public License~1.3c at: \colorhref{https://github.com/matze/mtheme}
        {https://github.com/matze/mtheme}.

        Typographic design: Iwona Grabska-Gradzińska;
        \LaTeX{} Typesetting: Kamil Ziemian \\
        Proofreading: Wojciech Palacz,
        Monika Stawicka \\
        3D Models: Dariusz Frymus, Kamil Nowakowski \\
        Figures and charts: Kamil Ziemian, Paweł Węgrzyn,
        Wojciech Palacz \\
        Film editing: IMAVI -- Joanna Kozakiewicz, Krzysztof Magda, Nikodem
        Frodyma

      \end{flushleft}

    \end{textblock}

  \end{frame}





  \begin{frame}[standout]


    \begingroup

    \color{jFrametitleFGColor}

    #1

    \endgroup

  \end{frame}
}



\newcommand{\GeometryThreeDTwoSpecialEndingSlidesVideoVerThreeEN}[1]{%
  \begin{frame}[standout]

    \begin{textblock}{11}(1,0.7)

      \begin{flushleft}

        \mdseries

        \footnotesize

        \color{jFrametitleFGColor}

        This content was created as part of a project co-financed by the
        European Union within the framework of the European Social Fund
        POWR.03.05.00-00-Z309/17-00.

      \end{flushleft}

    \end{textblock}





    \begin{textblock}{10}(0,2.2)

      \tikz \fill[color=jBackgroundStyleLight] (0,0) rectangle (12.8,-1.5);

    \end{textblock}


    \begin{textblock}{3.2}(0.7,2.45)

      \includegraphics[scale=0.3]{\FundingLogoColorPictureEN}

    \end{textblock}


    \begin{textblock}{2.5}(4.15,2.5)

      \includegraphics[scale=0.2]{\JULogoColorPictureEN}

    \end{textblock}


    \begin{textblock}{2.5}(6.35,2.4)

      \includegraphics[scale=0.1]{\ZintegrUJLogoColorPictureEN}

    \end{textblock}


    \begin{textblock}{4.2}(8.4,2.6)

      \includegraphics[scale=0.3]{\EUSocialFundLogoColorPictureEN}

    \end{textblock}





    \begin{textblock}{11}(1,4)

      \begin{flushleft}

        \mdseries

        \footnotesize

        \RaggedRight

        \color{jFrametitleFGColor}

        The content of this lecture is made available under a Creative
        Commons licence (\textsc{cc}), giving the author the credits
        (\textsc{by}) and putting an obligation to share on the same terms
        (\textsc{sa}). Figures and diagrams included in the lecture are
        authored by Paweł Węgrzyn et~al., and are available under the same
        license unless indicated otherwise. \\ The presentation uses the
        Beamer Jagiellonian theme based on Matthias Vogelgesang’s
        Metropolis theme, available under license \LaTeX{} Project
        Public License~1.3c at: \colorhref{https://github.com/matze/mtheme}
        {https://github.com/matze/mtheme}.

        Typographic design: Iwona Grabska-Gradzińska;
        \LaTeX{} Typesetting: Kamil Ziemian \\
        Proofreading: Leszek Hadasz, Wojciech Palacz,
        Monika Stawicka \\
        3D Models: Dariusz Frymus, Kamil Nowakowski \\
        Figures and charts: Kamil Ziemian, Paweł Węgrzyn,
        Wojciech Palacz \\
        Film editing: Agencja Filmowa Film \& Television Production~--
        Zbigniew Masklak \\
        Film editing: IMAVI -- Joanna Kozakiewicz, Krzysztof Magda, Nikodem
        Frodyma


      \end{flushleft}

    \end{textblock}

  \end{frame}





  \begin{frame}[standout]


    \begingroup

    \color{jFrametitleFGColor}

    #1

    \endgroup

  \end{frame}
}











% ------------------------------------------------------
% BibLaTeX
% ------------------------------------------------------
% Package biblatex, with biber as its backend, allow us to handle
% bibliography entries that use Unicode symbols outside ASCII.
\usepackage[
language=polish,
backend=biber,
style=alphabetic,
url=false,
eprint=true,
]{biblatex}

\addbibresource{Podstawy-informatyki-ETC-Bibliography.bib}





% ------------------------------------------------------
% Importing packages, libraries and setting their configuration
% ------------------------------------------------------




% ------------------------------------------------------
% Local packages
% ------------------------------------------------------
% Local configuration of this particular presentation
\usepackage{./Local-packages/local-settings}










% ------------------------------------------------------------------------------------------------------------------
\title{Podstawy informatyki z~językiem~C}
\subtitle{1. Wprowadzenie do~przedmiotu}

\author{Kamil Ziemian \\
  \email}


% \date{}
% ------------------------------------------------------------------------------------------------------------------










% ####################################################################
% Beginning of the document
\begin{document}
% ####################################################################





% ######################################
% Number of chars: 23k+, 51k+,
% Text is adjusted to the left and words are broken at the end of the line.
\RaggedRight
% ######################################





% ######################################
\maketitle
% ######################################





% ##################
\begin{frame}
  \frametitle{Spis treści}


  \tableofcontents

\end{frame}
% ##################





% ######################################
\section{Informacje wstępne}
% ######################################



% ##################
\begin{frame}
  \frametitle{Informacje wstępne}


  Obawiam~się, że na pierwszych zajęciach będzie sporo przynudzania,
  ale nie widzę sposobu, by~tego uniknąć. Do tego to jest kurs
  \textit{Podstawy informatyki}, więc przyjmujemy, że~Państwo dopiero
  zaczynają swoją przygodę z~tą specyficzną dziedziną
  teoretyczno-praktyczną jaką jest informatyka.

  Według mnie to zajęcia są dla studentów, nie studenci dla zajęć. Tak samo
  ja jestem tu dla Państwa, a~nie Państwo dla mnie. Jestem tu po to, by
  pomóc Państwu stawiać pierwsze kroki w~programowaniu w~języku~C.
  W~związku z~tym, ja będę Państwa rozliczał tylko i~wyłącznie
  z~umiejętności i~wiedzy, i~z~niczego innego. Wychodzę bowiem z~założenia,
  że~Państwo sami najlepiej wiedzą, czemu warto poświęcić swój czas. (Choć
  jak wiadomo, nie jeden raz potem stwierdzamy, że~nasz wybór mógł być
  jednak lepszy.)

\end{frame}
% ##################





% ##################
\begin{frame}
  \frametitle{Informacje wstępne}


  Na zajęciach nie tylko można, ale \alert{należy} zadawać pytania
  na dowolne związane z~nimi tematy. W~szczególności
  \alert{należy} zadawać pytania, jeśli czegoś~się nie rozumie, lub coś
  jest niejasno przedstawione. To są podstawy informatyki, \alert{nie}
  zakładamy, że~Państwo mają już wszystko umieć. Byłoby to bardzo
  niewłaściwe założenie.

  Proszę pamiętać, że~gdy chodzi o~tematy związane z~zajęciami
  \alert{nie} ma pytań zbyt elementarnych lub zbyt głupich. Są~tylko
  niezadowalające odpowiedzi na nie. Jestem tutaj by Państwu
  pomóc w~nauce programowania w~C, pytania z~Państwa strony bardzo mi to
  zadanie ułatwiają. Zadawania pytań nie jest traktowane, jako oznaka tego,
  że~ktoś czegoś nie umie, tylko że~chce~się czegoś nauczyć.

  Pytania typu „Jaki jest najfajniejszy boss w~grze \textit{Hollow
    Knight}?” musimy jednak zostawić na czas po zajęciach.

\end{frame}
% ##################





% ##################
\begin{frame}
  \frametitle{Informacje wstępne}


  Na tych zajęciach \alert{nie} nauczymy~się jak programować. Jak dobrze
  pójdzie to nauczymy~się podstaw programowania w~języku~C, ale
  programowanie obejmuje tyle zagadnień i~wymaga tyle godzin praktyki,
  iż~nie ma najmniejszych szans, że~uda nam~się to wszystko zrobić.

  Co, na Państwa nieszczęście, nie oznacza, że~będzie mało materiału.
  Mogą~się Państwo wręcz czuć przytłoczeni jego ilością. Proszę mi jednak
  uwierzyć, że~większość tych zagadnień nie jest taka trudna, jak~się
  może wydawać na pierwszy rzut oka.

  Przyznam~się, że~sam staję przed duży dylematem. Czy to mają być
  \alert{podstawy informatyki} z~językiem~C, czy \alert{nauka
    programowania w~języku~C} z~podstawami informatyki. Na którą część
  położyć większy nacisk, to jest problem przed którym stoję. W~obecnej
  chwili zdecydowałem~się iść tą drugą drogą. A~jaka jest Państwa opinia
  w~tej sprawie?

\end{frame}
% ##################





% ##################
\begin{frame}
  \frametitle{Informacje wstępne}


  Wedle mojej obserwacji większość osób na tych studiach jest nastawiona
  na praktyczną stronę informatyki: jak stworzyć program który działa?
  Biorąc to za punkt, staram~się prowadzić te zajęcia tak, aby najpierw
  mieli Państwo okazję oswoić~się z~tworzeniem programów w~C, nawet jeśli
  są one bardzo proste. Zagadnienia teoretyczne i~bardziej techniczne,
  które dobry programista musi znać, będę w~miarę możliwości omawiane
  na późniejszych zajęciach.

  Minus tego podejścia jest następujący. W~języku~C cała maszyneria
  komputera jest delikatnie schowana pod dość cienką warstwą tego języka
  (w~dialekcie informatyków powiedzielibyśmy „pod cienką warstwą
  abstrakcji”) i~nawet początkujący ciągle natrafia na drugie dno problemu.
  Doświadczenie pokazuje, że~wyjaśnianie czemu tak jest wymaga sporo czasu
  i~mocno zaburza nakierowany na praktykę programistyczną plan zajęć.

\end{frame}
% ##################





% ##################
\begin{frame}
  \frametitle{Informacje wstępne}


  Dlatego jak jestem \alert{bardzo} za tym by zadawali mi Państwo pytania,
  to w~wielu wypadkach nie będę mógł na nie od razu odpowiadać, bo
  dobre wyjaśnienie zajęłoby za dużo czasu. Zawsze można~się do mnie
  zgłosić z~takimi pytaniami po zajęciach, wtedy odpowiem najlepiej
  jak umiem. W~trakcie zajęć będę jednak często musiał mówić „Proszę zrobić
  tak i~tak. Dlaczego, wyjaśnimy to potem.”. Proszę mieć dla mnie
  wyrozumiałość w~tym względzie.

  Jeśli to o~czym mówię jest tak proste, że~omawianie tego jest stratą
  Państwa czasu, proszę mnie o~tym poinformować, przejdziemy do następnego
  zagadnienia. Jest bardzo trudno sprawą wyczucie, co jest dla Państwa
  proste, a~co trudniejsze.

  Proszę mieć dla mnie pewną wyrozumiałość, pamiętać o~tych zastrzeżeniach
  i~próbować mi pomóc w~takim prowadzeniu zajęć, by Państwo wynieśli z~nich
  jak najwięcej. Jak mówiłem, te~zajęcia są dla Państwa.

\end{frame}
% ##################





% ##################
\begin{frame}
  \frametitle{Dlaczego na tych zajęciach nie nauczymy~się
    programować?}


  Rozpatrzmy problem liczb z~częścią dziesiętną, takich jak $0.1$, $2.71$,
  czy $3.14$. Formalna ich nazwa w~informatyce to \textbf{liczby
    zmiennoprzecinkowe} (ang. \textit{floating-point numbers}). Klasyczny
  artykuł Davida Goldberga
  \colorhref{https://dl.acm.org/doi/10.1145/103162.103163}
  {\textit{Co każdym informatyk musi wiedzieć o~arytmetyce liczb
      zmiennoprzecinkowych}} z~1991 roku, ma w~wersji \textsc{pdf}
  $44$~strony długości. I~nie jest to tekst, który~się szybko czyta.

  Mówiąc inaczej, gdyby ten przedmiot prowadził zawodowy
  zmiennoprzecinkowiec to mógłby \alert{wszystkie} (powtórzmy to:
  wszystkie) nasze spotkania poświęcić tylko i~wyłącznie operacjom
  arytmetycznym w~komputerze. I~tak pewnie zabrakło by mu czasu, by zrobić
  wszystko co uważa za ważne.

  My tego nie będziemy robić, bo trzeba jakoś omówić również liczby
  całkowite, stringi, zmienne, typy zmiennych, konwersje typów, algebrę
  Boole’a, operatory logiczne, instrukcje warunkowe, pętle, struktur, etc.
  Nie wiem, czy i~tak uda nam~się to wszystko przerobić.

\end{frame}
% ##################





% ##################
\begin{frame}
  \frametitle{Tak to wygląda}

  \vspace{-0.5em}


  \begin{figure}

    \label{fig:Learning-any-language-at-100-procent}

    \centering


    \includegraphics[scale=0.195]
    {./Presentations-pictures/Learning-language-at-100.jpg}

  \end{figure}

\end{frame}
% ##################





% ##################
\begin{frame}
  \frametitle{Uwagi odnośnie treści zajęć}


  Ponieważ tematyka którą poruszamy jest mimo wszystko niebanalna, więc
  mnóstwo rzeczy będę musiał bardzo \alert{upraszczać}. Proszę mieć to na
  uwadze w~trakcie zajęć i~studiując materiały do nich.

  Potrzeba uproszczeń wynika z~dwóch powodów. Po pierwsze, ograniczenia
  czasowe. Wiele z~zagadnień które poruszymy mogłoby być tematem
  semestralnego kursu. Co gorsza, wiele z~nich \alert{jest} tematem
  semestralnych kursów, przykładowo: praca z~liczbami zmiennoprzecinkowymi.
  Po drugie, to jest kurs \alert{podstaw} informatyki, który ma położyć
  fundamenty pod Państwa umiejętności i~wiedzę w~zakresie informatyki. To
  nie przedmiot na którym należy wnikać we wszystkie detale, szczegóły,
  drugie, trzecie i~czwarte dno problemu.

  Jeśli jednak ktoś chce~się bardziej zagłębić w~te temat, to służę po
  zajęciach całą swoją osobą.

\end{frame}
% ##################





% ##################
\begin{frame}
  \frametitle{Dlaczego język~C?}


  Dlaczego zaczynamy naukę od~języka~C? Krótka odpowiedź jest taka,
  że~pomimo tego iż język ten ma już pół wieku~(!) na karku, w~2024 roku
  nasza infrastruktura informatyczna wciąż stoi na kodzie źródłowym
  napisanym właśnie w~nim. W~internecie mogą Państwo znaleźć wiele
  artykułów i~blogów takich jak
  \colorhref{https://wideinfo.org/c-programming-is-still-running-the-world/}
  {\textit{C~programming is still running the world}}, datowany na styczeń
  $2025$ roku, którego tytuł mówi sam za siebie
  \parencite{Scott-C-programming-is-still-ETC-Ver-2025}.

  Niektórzy mówią, że~C to król wszystkich języków programowania.
  Inni, bardzo dobrzy informatycy, twierdzą, że każdy szanujący
  programista musi znać~C (cf. str.~17
  \parencite{Hoey-Programowanie-w-asemblerze-x64-ETC-Pub-2024}).

  A~jak to jest w~praktyce? Na te zajęcia potrafią uczęszczać ludzie,
  którzy pracują zawodowo jako programiści i~nie mają pojęcia jak napisać
  program w~języku~C. Proszę samemu wyciągnąć z~tego wnioski.

\end{frame}
% ##################





% ##################
\begin{frame}
  \frametitle{Dlaczego system GNU/Linux?}


  Dlaczego korzystamy z~systemu GNU/Linux, a~nie z~znacznie
  popularniejszego wśród normalny ludzi systemu Windows? Bo~oferuje
  znacznie lepsze warunki pracy z~językiem~C.

  \alert{Ważne.} Jeśli mają Państwo jakiekolwiek problemy z~systemem
  GNU/Linux to proszę o~tym \alert{mówić}. Nie przyjmujemy założenia,
  że~Państw mają już teraz być ekspertami, w~kwestii używania tego, co by
  tu nie mówić, często bardzo topornego systemu operacyjnego.

  Jak bowiem głosi mądrość internetu „GNU/Linux jest darmowy, tylko jeśli
  twój czas nie ma wartości.”

\end{frame}
% ##################





% ##################
\begin{frame}
  \frametitle{Rozwój GNU/Linuxa}

  \vspace{-0.5em}


  \begin{figure}

    \label{fig:Evolution-of-OS}

    \centering


    \includegraphics[scale=0.3]
    {./Presentations-pictures/Evolution-of-operating-systems.jpg}

  \end{figure}

\end{frame}
% ##################





% ##################
\begin{frame}
  \frametitle{O~zajęciach i~konsultacjach}


  \alert{Ważne.} Wykład ma bardziej charakter teoretyczny, te laboratoria
  zaś praktyczny. Niemniej wciąż poszukujemy optymalnej formy prowadzenia
  tych zajęć, bo obecna jest daleka od ideału. Naprawdę
  praktyczne zajęcia z~podstaw informatyki w~języku~C są trudne do
  zorganizowania. Jedną z~przyczyn takiego stanu rzeczy, jest natura
  języka~C, który pozwala pisać praktyczne programy, dopiero gdy opanuje~się
  naprawdę duży zakres jest funkcjonalności.

  Z~mojego doświadczenia wynika, że~ustalanie jednego terminu na konsultacje
  to nie jest dobry pomysł. W~zasadzie nikt wtedy nie przychodzi, a~ja
  wyznaję zasadę, że~konsultacje są dla Państwa, nie dla mnie. Jeśli
  Państwo chcą bym ustalił konkretne terminy na konsultacje, to proszę jako
  grupa wybrać jeden taki i~poinformować mnie o~tym mailowo, pisząc na
  adres \email. W~tytule maila proszę napisać „Termin konsultacji” lub
  coś podobnego, byśmy wiedzieli, że~na ten mail należy możliwie szybko
  odpowiedzieć.

\end{frame}
% ##################





% ##################
\begin{frame}
  \frametitle{Bardzo ważne}


  W~przeciwnym razie, jeśli ktoś z~Państwa ma problem i~chce zasięgnąć
  mojej pomocy, proszę do mnie podejść po zajęciach lub napisać, na
  wspomniany już adres \email, kiedy, gdzie i~w~jakiej formie chcą Państwo
  uczestniczyć w~konsultacjach. Mogą one być zarówno w~świecie rzeczywistym
  (niekoniecznie w~budynku \textsc{wsz}i\textsc{b}u), online lub
  telefonicznie.

  Ponawiam prośbę o~nadanie takiemu e-mailowi tytułu typu „Termin
  konsultacji”, bo w~przy ilości e-maili jakie trafiają na tę skrzynkę
  pocztową łatwo jeden przegapić.

  Jeśli ktoś będzie pisał w~sprawie konsultacji, to będę wdzięczny
  za~napisanie z~czym konkretnie mają Państwo problem. Rozumiem jedna,
  że~często wskazanie czy nazwanie tego co sprawia komuś problem, samo nie
  jest łatwe. Sam przez to przechodziłem.

\end{frame}
% ##################





% ##################
\begin{frame}
  \frametitle{Nagrania zajęć}


  Będę~się starał nagrywać na \textsc{ms}~Teamsach każde naszych spotkań.
  Proszę mi o~tym ciągle przypominać, bo jestem roztrzepany i~któregoś
  razu o~tym zapomnę.

  Proszę mi też zwracać uwagę, że~na ekranie czegoś nie widać,
  że~czcionka za mała, że~kolory kłują w~oczy, że~nagrany dźwięk
  jest niskiej jakości,~etc. Zajęcia są dla Państwa, naszym obowiązkiem jest
  dostarczyć Państwu najlepszej jakości materiały do nauki jakie jesteśmy
  w~stanie stworzyć.

  Niestety, jakość dźwięku to coś, na co mamy mały wpływ. Mogę~się starać
  mówić możliwie blisko mikrofonu, ale nie wiem co więcej mogę zrobić.
  Poza tym, na~pewno nie wyjdzie zbyt dobrze, bo te zajęcia często wymagają
  bym~się poruszał, poza tym w~którymś momencie na pewno o~tym zapomnę.
  Swoje uwagi na temat jakości nagrań proszę kierować do ludzi
  odpowiedzialnych za~sprawy studenckie na \textsc{wsz}i\textsc{b}ie.

\end{frame}
% ##################





% ##################
\begin{frame}
  \frametitle{Nie lubię zbyt poważnych zajęć, ale\ldots}

  \vspace{-0.5em}


  \begin{figure}

    \label{fig:Jak-to-bywa-na-zajeciach}

    \centering


    \includegraphics[scale=0.42]
    {./Presentations-pictures/Jak-to-bywa-na-zajeciach.jpeg}

  \end{figure}

\end{frame}
% ##################










% ######################################
\section{O~uzyskaniu zaliczenia}
% ######################################



% ##################
\begin{frame}
  \frametitle{Zaliczenie zaoczne}


  Zaliczenie zaoczne można jak najbardziej uzyskać, np.~przedstawiając
  kod jakiegoś swojego programu czy aplikacji. Projekt ten
  \alert{nie musi} być napisany w~języku~C, może być w~Pythonie albo
  JavaScripcie. Z~powodów które powinny być dla wszystkich oczywiste,
  preferowane są jednak te stworzone w~języku C. W~trochę mniejszym
  stopniu, w~te w języku~C++.

  Każdy kto chce uzyskać zaliczenie zaoczne, niech zgłosi~się do mnie po
  zajęciach lub napisze na maila \email. W~tytule proszę wpisać
  „Pytanie o~zaliczenie zaoczne” lub coś podobnego, inaczej mail może
  na~długo zaginąć na mojej skrzynce pocztowej.

\end{frame}
% ##################





% ##################
\begin{frame}
  \frametitle{Uzyskanie zaliczenia}


  Zaliczenie i~ocenę uzyskują Państwo na podstawie trzech rzeczy.

  \vspace{-0.3em}



  \begin{itemize}

  \item Zadania domowe.

  \item Dwa testy jednokrotnego wyboru.

  \item Jeden większy projekt.

  \end{itemize}

  \vspace{-0.3em}

  Za każdy z~tych rzeczy przyznawana jest pewna ilość punktów. Niestety
  system przyznawania punktów wciąż jest daleki od ideału, ale mam nadzieję,
  że~nie będzie to dla Państwa zbyt wielkim obciążeniem.

  Podział na ćwiczenia i~laboratoria jest bardzo sztuczny, obecnie
  to bardziej kwestia administracji, niż praktyki. Dlatego pod koniec
  semestru otrzymują Państwo \alert{jedną} ocenę łączną, która
  jest wpisywana jako jednocześnie jako ocena z~ćwiczeń i~laboratoriów.

\end{frame}
% ##################





% ##################
\begin{frame}
  \frametitle{Naprawdę cenimy Państwa zadanie}


  Jeśli ktoś ma uwagi do tego systemu, propozycję co można zmienić,
  co~poprawić, to proszę powiedzieć to mi po zajęciach lub napisać pod
  adres \email. Ponownie prosimy by w~tytule emaila napisać „Uwagi do zajęć”
  lub coś podobnego, bo inaczej może zniknąć na długo w~skrzynce.

  W~chwili obecnej nie mamy złudzeń, że~nasz system jest choćby bliski
  ideałowi, próbujemy jednak go~ciągle ulepszać. Państwa zdanie jest dla
  nas naprawdę ważne, choć oczywiście nie zawsze~się z~nim zgadzamy.
  To jak ten przedmiot wygląda obecnie wynika częściowo z~tego co
  w~przeszłości zaproponowali studenci.

\end{frame}
% ##################





% ##################
\begin{frame}
  \frametitle{Uwagi odnośnie treści zajęć}


  Pod koniec semestru podliczane są wszystkie punkty jakie były do
  zdobycia. W~zależności ile procent pełnej puli Państwo zdobyli, otrzymują
  Państwo odpowiednią ocenę.

  \vspace{-0.3em}



  \begin{itemize}

  \item $41\%\text{--}50\%$~-- ocena dostateczna ($3.0$).

  \item $51\%\text{--}60\%$ -- ocena plus dostateczna ($3.5$, $3+$).

  \item $61\%\text{--}70\%$ -- ocena dobra ($4.0$).

  \item $71\%\text{--}80\%$ -- ocena puls dobry ($4.5$, $4+$).

  \item $81\%\text{--}100\%$ -- ocena bardzo dobry ($5.0$).

  \end{itemize}

  \vspace{-0.3em}



  W~przypadku zaokrąglanie wyników, robione to jest zawsze na korzyść dla
  Państwa. Czyli $40.1\%$ zaokrągla~się do $41\%$.

  Ilość punktów do zdobycie jest będzie jawnie podana przy każdym
  konkretnym zadaniu domowym. Za jeden test można zdobyć $10$ pkt.,
  konkretna wartość zależy od zdobytej oceny. Za projekt można zdobyć
  do~$30$ pkt. Ilość punktów za projekt jest tym, co obecnie najmniej mnie
  satysfakcjonuje, ale na razie nie mamy pomysłu jak to poprawić.

\end{frame}
% ##################





% ##################
\begin{frame}
  \frametitle{Punktacja testów}


  W~tym semestrze każdy test będzie zawierał 10 pytań jednokrotnego wyboru.
  Progi ocen są takie same jak dla całego przedmiotu, stosowana jest też ta
  sama reguła zaokrąglania (wątpię by była potrzeba korzystania z~niej
  w~praktyce).

  \vspace{-0.3em}



  \begin{itemize}

  \item $41\%\text{--}50\%$ -- ocena dostateczna ($3.0$).

  \item $51\%\text{--}60\%$ -- ocena plus dostateczna ($3.5$, $3+$).

  \item $61\%\text{--}70\%$ -- ocena dobra ($4.0$).

  \item $71\%\text{--}80\%$ -- ocena puls dobry ($4.5$, $4+$).

  \item $81\%\text{--}100\%$ -- ocena bardzo dobry ($5.0$).

  \end{itemize}

  \vspace{-0.3em}



  Ilość zdobytych punktów równa jest zero jeśli uzyskało~się ocenę $2.0$
  (mniej niż $41\%$), lub $2 \cdot \text{uzyskana ocena}$.

\end{frame}
% ##################




% ##################
\begin{frame}
  \frametitle{Punktacja testów}


  Liczba punktów z~testów wliczanych do końcowej puli, w~zależności od
  wyniku testu, przedstawia~się następująco.

  \vspace{-0.3em}



  \begin{itemize}

  \item $0\%\text{--}40\%$ -- $0$ pkt.

  \item $41\%\text{--}50\%$ -- $6$ pkt.

  \item $51\%\text{--}60\%$ -- $7$ pkt.

  \item $61\%\text{--}70\%$ -- $8$ pkt.

  \item $71\%\text{--}80\%$ -- $9$ pkt.

  \item $81\%\text{--}100\%$ -- $10$ pkt.

  \end{itemize}

  \vspace{-0.3em}




  Następujące zasady powinny być dla każdego oczywiste, ale lepiej
  przedstawić je jawnie, by nie było żadnych niedomówień.

  Wszelkie sprawdziany i~testy przeprowadzane na zajęciach, lub w~trakcie
  zajęć online, mają Państwo rozwiązać \alert{samemu}, bez żadnej pomocy
  ze strony innych ludzi, internetu czy programów takich jak
  Chat\textsc{gpt}.

\end{frame}
% ##################





% ##################
\begin{frame}
  \frametitle{O~pracach domowych}


  Z~naszego punktu widzenia, zadania i~projekt, które rozwiązują Państwo
  w~domu są najważniejsze przy wyliczaniu oceny. Testy są, ale staramy~się
  by miały one mniejszą wagę. Możliwie, że~trzeba będzie zmniejszyć
  obecną punktacje testów oraz przemyśleć ich miejsce w~przedmiocie.

  Proszę przy tym nie traktować zadań domowych, jako nieprzyjemnego
  obowiązku do~odhaczenia, ale okazję do nauczenia~się czegoś o~pisaniu
  programów komputerowych. Aby zostać dobrym informatykiem trzeba sporo
  czasu spędzić pisząc, testując i~debugując własne programy, te zadania
  temu przede wszystkim mają służyć. To że~otrzymują Państwo za nie punkty,
  to w~porównaniu z~tym sprawa drugorzędna.

  Wątpię by rozwiązywanie i~oddawanie tych zadań było przyjemnością,
  ale~można o~tym mimo wszystko myśleć jako o~wartościowo spędzonym czasie.

\end{frame}
% ##################





% ##################
\begin{frame}
  \frametitle{O~pracach domowych}


  Ile będzie zestawów zadań? Ciężko powiedzieć. W~obecnej chwili wychodzi
  tak jeden zestaw na dwa tygodnie, ale staram~się to zoptymalizować.
  Proszę~się spodziewać około siedmiu zestawów.

  Początkowo zadania są bardzo prosto, jak na ten przedmiot, z~czasem ich
  trudność, jak i~punkty do zdobycia za nie rosną. Układanie dobrych
  zestawów zadań to trudna sztuka, stąd będę wdzięczny za~wszelkie uwagi
  jak można je poprawić. Przypominam, że~o tych zestawach lepiej myśleć
  jako o~sposobie nauczenia~się tworzenia programów, a~nie nieprzyjemnym
  obowiązku do odhaczenia. Easier said than done, wiemy o~tym.

  Jeśli dla kogoś zadania domowe to będzie za mało, co jest wspaniałą
  wiadomością, to na listach materiałów do nauki będą zamieszczone źródła,
  z~większą liczbą zadań do~rozwiązania.

\end{frame}
% ##################





% ##################
\begin{frame}
  \frametitle{O~rozwiązywaniu prac domowych}


  Proszę przyjąć zasadę, że~\alert{najprostszy program} który rozwiązuje
  zadanie zgodnie z~wytycznymi jest najlepszym jaki można oddać. Niekiedy
  studenci oddają bardzo skomplikowane programy jako rozwiązanie naprawdę
  prostych zadań, co jest zupełnie zbędne.

  Oczywiście, nikt nie straci punktów za oddanie skomplikowanego programu,
  ale jest wiele powodów, by poprzestać na pisaniu prostego kodu.
  Po~pierwsze, to zwykle oszczędza czas spędzony na~rozwiązywaniu zadań
  na~ten przedmiot. Po~drugie, prostota i~czytelność kodu są naprawdę ważne
  przy tworzeniu programów na~każdym poziomie.

  Należy bowiem wiedzieć, że~pisanie programów, wbrew pozorom,
  \alert{nie} jest trudne. \alert{Trudne jest ich debugowanie.}
  Dokładniej, obowiązuje prawo \colorhref{}{Kernighana}. \\
  \textit{Debugowanie programu jest dwa razy trudniejsze, niż jego
    napisanie. Jeśli więc napiszesz program w~najbardziej pomysłowy sposób
    jaki możesz, to nie jesteś wystarczająco inteligentny by go zdebugować.}

  % Jeżeli ktoś oddaje trudne rozwiązywania zadań, bo chce spróbować swoich
  % sił w~programowaniu, to oczywiście, naszym obowiązkiem jest poprawić
  % ten zestaw bez marudzenia. Wszystkie te uwagi nie są po to byśmy
  % my mieli prostszą pracę przy poprawianiu zdań, tylko by Państwo nie
  % trudzili~się niepotrzebnie.

  % Proszę też pamiętać, że~obowiązuje prawo Kernighana. \\
  % \textit{Debugowanie programu jest dwa razy trudniejsze, niż jego
  %   napisanie. Jeśli więc napiszesz program w~najbardziej pomysłowy sposób
  %   jaki możesz, to nie jesteś wystarczająco inteligentny by go zdebugować.}

  % Ostrzegam, ja bardzo \alert{nie lubię} ściągania. Student który
  % oblał~$20$ poprawkę nie będzie miał u~mnie takich problemów, jak osoba
  % którą złapię na ściąganiu. Moja zasada jest taka, że~jak kogoś na tym
  % nakryję na tego typu oszustwach, to więcej niż $3.0$ u~mnie
  % \alert{nie dostanie}.

  % Ściągania nie da~się całkiem wyeliminować, ale można je ograniczyć.
  % A~jak ktoś ściąga tak, żeby ślepy zauważył, to może mieć pretensje tylko
  % i~wyłącznie do siebie. ;)

  % Gdy chodzi o~zadania domowe i~projekt, to należy próbować rozwiązać je
  % możliwie samodzielnie. W~razie napotkania problemów nie tylko można,
  % ale i~należy prosić o~pomoc kolegów, korzystać z~materiałów w~internecie
  % i~używać programów takich jak Chat\textsc{gpt}.

  % Na zajęciach, poza testami, też można z~tych pomocy korzystać, ale nie do
  % tego stopnia, by czyjś wkład własny został zredukowany do metody
  % Copy’ego-Pejsta.

\end{frame}
% ##################





% ##################
\begin{frame}
  \frametitle{O~rozwiązywaniu prac domowych}


  Jeśli ktoś nie wierzy, że~trudną częścią programowania nie jest pisanie
  programów, tylko ich debugowanie, niech zajrzy na świetny kanał
  \colorhref{}{Low Level} i~posłucha jakie błędy są znajdowane w~kodzie
  tworzony przez zawodowych programistów.

  Jeżeli ktoś oddaje trudne rozwiązywania zadań, bo chce spróbować swoich
  sił w~programowaniu, to oczywiście, naszym obowiązkiem jest poprawić
  ten zestaw bez marudzenia. Wszystkie te uwagi nie są po to byśmy
  my mieli prostszą pracę przy poprawianiu zdań, tylko by Państwo nie
  trudzili~się niepotrzebnie.

  Proszę jednak mieć w~głowie następującą mądrość informatyków. \\
  \textit{Program z~którego usunęliśmy jeden błąd, stał~się tym samym
    $10$~razy lepszy.}

  Należy mieć świadomość, że~w~praktyce brak błędów (a~przynajmniej
  zmniejszenie ich liczby) jest ważniejszy niż szybkość programu.

\end{frame}
% ##################





% ##################
\begin{frame}
  \frametitle{O~rozwiązywaniu prac domowych}


  Błędy zwiększają kilkunastokrotnie możliwość, że~haker włamie~się do
  danego komputera, a~to w~czasach internetu rzecz kluczowa. W~okresie po
  ??? roku, gdy
  odkryto luki w~bezpieczeństwie \colorhref{}{Specter}
  i~\colorhref{}{Meltdown}
  w~imię bezpieczeństwa zabiliśmy $50$\% (!!!) prędkości pewnych programów,
  by~się zabezpieczyć przed atakami hakerów.


  % Początkowo zadania są bardzo prosto, jak na ten przedmiot, z~czasem ich
  % trudność, jak i~punkty do zdobycia za nie rosną. Układanie dobrych
  % zestawów zadań to trudna sztuka, stąd będę wdzięczny za~wszelkie uwagi
  % jak można je poprawić. Przypominam, że~o tych zestawach lepiej myśleć
  % jako o~sposobie nauczenia~się tworzenia programów, a~nie nieprzyjemnym
  % obowiązku do odhaczenia. Easier said than done, wiemy o~tym.

  % Proszę przyjąć zasadę, że~\alert{najprostszy program} który rozwiązuje
  % zadanie zgodnie z~wytycznymi jest najlepszym jaki można oddać. Niekiedy
  % studenci oddają bardzo skomplikowane programy jako rozwiązanie naprawdę
  % prostych zadań, co jest zupełnie zbędne.

  % Ostrzegam, ja bardzo \alert{nie lubię} ściągania. Student który
  % oblał~$20$ poprawkę nie będzie miał u~mnie takich problemów, jak osoba
  % którą złapię na ściąganiu. Moja zasada jest taka, że~jak kogoś na tym
  % nakryję na tego typu oszustwach, to więcej niż $3.0$ u~mnie
  % \alert{nie dostanie}.

  % Ściągania nie da~się całkiem wyeliminować, ale można je ograniczyć.
  % A~jak ktoś ściąga tak, żeby ślepy zauważył, to może mieć pretensje tylko
  % i~wyłącznie do siebie. ;)

  % Gdy chodzi o~zadania domowe i~projekt, to należy próbować rozwiązać je
  % możliwie samodzielnie. W~razie napotkania problemów nie tylko można,
  % ale i~należy prosić o~pomoc kolegów, korzystać z~materiałów w~internecie
  % i~używać programów takich jak Chat\textsc{gpt}.

  % Na zajęciach, poza testami, też można z~tych pomocy korzystać, ale nie do
  % tego stopnia, by czyjś wkład własny został zredukowany do metody
  % Copy’ego-Pejsta.

  % Oczywiście, nikt nie straci punktów za oddanie skomplikowanego programu,
  % ale jest wiele powodów, by poprzestać na pisaniu prostego kodu.
  % Po~pierwsze, to zwykle oszczędza czas spędzony na~rozwiązywaniu zadań
  % na~ten przedmiot. Po~drugie, prostota i~czytelność kodu są naprawdę ważne
  % przy tworzeniu programów na~każdym poziomie.

  % Jeżeli ktoś oddaje trudne rozwiązywania zadań, bo chce spróbować swoich
  % sił w~programowaniu, to oczywiście, naszym obowiązkiem jest poprawić
  % ten zestaw bez marudzenia. Wszystkie te uwagi nie są po to byśmy
  % my mieli prostszą pracę przy poprawianiu zdań, tylko by Państwo nie
  % trudzili~się niepotrzebnie.

  % Proszę też pamiętać, że~obowiązuje prawo Kernighana. \\
  % \textit{Debugowanie programu jest dwa razy trudniejsze, niż jego
  %   napisanie. Jeśli więc napiszesz program w~najbardziej pomysłowy sposób
  %   jaki możesz, to nie jesteś wystarczająco inteligentny by go zdebugować.}

  % Ostrzegam, ja bardzo \alert{nie lubię} ściągania. Student który
  % oblał~$20$ poprawkę nie będzie miał u~mnie takich problemów, jak osoba
  % którą złapię na ściąganiu. Moja zasada jest taka, że~jak kogoś na tym
  % nakryję na tego typu oszustwach, to więcej niż $3.0$ u~mnie
  % \alert{nie dostanie}.

  % Ściągania nie da~się całkiem wyeliminować, ale można je ograniczyć.
  % A~jak ktoś ściąga tak, żeby ślepy zauważył, to może mieć pretensje tylko
  % i~wyłącznie do siebie. ;)

  % Gdy chodzi o~zadania domowe i~projekt, to należy próbować rozwiązać je
  % możliwie samodzielnie. W~razie napotkania problemów nie tylko można,
  % ale i~należy prosić o~pomoc kolegów, korzystać z~materiałów w~internecie
  % i~używać programów takich jak Chat\textsc{gpt}.

  % Na zajęciach, poza testami, też można z~tych pomocy korzystać, ale nie do
  % tego stopnia, by czyjś wkład własny został zredukowany do metody
  % Copy’ego-Pejsta.

\end{frame}
% ##################





% ##################
\begin{frame}
  \frametitle{Pomoce przy rozwiązywaniu prac domowych}


  % Jeżeli ktoś oddaje trudne rozwiązywania zadań, bo chce spróbować swoich
  % sił w~programowaniu, to oczywiście, naszym obowiązkiem jest poprawić
  % ten zestaw bez marudzenia. Wszystkie te uwagi nie są po to byśmy
  % my mieli prostszą pracę przy poprawianiu zdań, tylko by Państwo nie
  % trudzili~się niepotrzebnie.

  % Proszę jednak mieć w~głowie następującą mądrość informatyków. \\
  % \textit{Program z~którego usunęliśmy jeden błąd, stał~się tym samym
  %   $10$~razy lepszy.}

  % Proszę uwierzyć, że~w~praktyce brak błędów (a~przynajmniej ich
  % zmniejszenie) jest ważniejszy niż szybkość programu. Błędy zwiększają
  % kilkunastokrotnie możliwość, że~haker włamie~się do danego komputera,
  % a~to w~czasach internetu rzecz kluczowa. W~okresie po ??? roku, gdy
  % odkryto luki w~bezpieczeństwie \colorhref{}{Specter}
  % i~\colorhref{}{Meltdown}
  % w~imię bezpieczeństwa zabiliśmy $50$\% (!!!) prędkości pewnych programów,
  % by~się zabezpieczyć przed atakami hakerów.


  % Początkowo zadania są bardzo prosto, jak na ten przedmiot, z~czasem ich
  % trudność, jak i~punkty do zdobycia za nie rosną. Układanie dobrych
  % zestawów zadań to trudna sztuka, stąd będę wdzięczny za~wszelkie uwagi
  % jak można je poprawić. Przypominam, że~o tych zestawach lepiej myśleć
  % jako o~sposobie nauczenia~się tworzenia programów, a~nie nieprzyjemnym
  % obowiązku do odhaczenia. Easier said than done, wiemy o~tym.

  Ponieważ zadania są głównie po to, by mogli Państwo nauczyć~się
  programować, jeśli Państwo tempo kopiują rozwiązanie kogoś innego,
  to oszukują Państwo głównie samych siebie.

  Ostrzegam, ja bardzo \alert{nie lubię} ściągania. Student który
  oblał~$20$ poprawkę nie będzie miał u~mnie takich problemów, jak osoba
  którą złapię na ściąganiu. Moja zasada jest taka, że~jak kogoś na tym
  nakryję na tego typu oszustwach, to więcej niż $3.0$ u~mnie
  \alert{nie dostanie}. Chyba, że~przymkniemy oko i~udamy, że~tego nie było.
  Wszystko zależy.

  Ściągania nie da~się całkiem wyeliminować, ale można je ograniczyć.
  A~jak ktoś ściąga tak, żeby ślepy zauważył, to może mieć pretensje tylko
  i~wyłącznie do siebie. ;)

  Gdy chodzi o~zadania domowe i~projekt, to należy próbować rozwiązać je
  możliwie samodzielnie. W~razie napotkania problemów nie tylko można,
  ale i~należy prosić o~pomoc kolegów, korzystać z~materiałów w~internecie
  i~używać programów takich jak Chat\textsc{gpt}.

  % Na zajęciach, poza testami, też można z~tych pomocy korzystać, ale nie do
  % tego stopnia, by czyjś wkład własny został zredukowany do metody
  % Copy’ego-Pejsta.

  % Oczywiście, nikt nie straci punktów za oddanie skomplikowanego programu,
  % ale jest wiele powodów, by poprzestać na pisaniu prostego kodu.
  % Po~pierwsze, to zwykle oszczędza czas spędzony na~rozwiązywaniu zadań
  % na~ten przedmiot. Po~drugie, prostota i~czytelność kodu są naprawdę ważne
  % przy tworzeniu programów na~każdym poziomie.

  % Ostrzegam, ja bardzo \alert{nie lubię} ściągania. Student który
  % oblał~$20$ poprawkę nie będzie miał u~mnie takich problemów, jak osoba
  % którą złapię na ściąganiu. Moja zasada jest taka, że~jak kogoś na tym
  % nakryję na tego typu oszustwach, to więcej niż $3.0$ u~mnie
  % \alert{nie dostanie}.

  % Ściągania nie da~się całkiem wyeliminować, ale można je ograniczyć.
  % A~jak ktoś ściąga tak, żeby ślepy zauważył, to może mieć pretensje tylko
  % i~wyłącznie do siebie. ;)

  % Gdy chodzi o~zadania domowe i~projekt, to należy próbować rozwiązać je
  % możliwie samodzielnie. W~razie napotkania problemów nie tylko można,
  % ale i~należy prosić o~pomoc kolegów, korzystać z~materiałów w~internecie
  % i~używać programów takich jak Chat\textsc{gpt}.

  % Na zajęciach, poza testami, też można z~tych pomocy korzystać, ale nie do
  % tego stopnia, by czyjś wkład własny został zredukowany do metody
  % Copy’ego-Pejsta.

\end{frame}
% ##################





% ##################
\begin{frame}
  \frametitle{Pomoce przy rozwiązywaniu prac domowych}


  % Ponieważ zadania są głównie po to, by mogli Państwo nauczyć~się
  % programować, jeśli Państwo tempo kopiują rozwiązanie kogoś innego,
  % to oszukują Państwo głównie samych siebie.

  % Ostrzegam, ja bardzo \alert{nie lubię} ściągania. Student który
  % oblał~$20$ poprawkę nie będzie miał u~mnie takich problemów, jak osoba
  % którą złapię na ściąganiu. Moja zasada jest taka, że~jak kogoś na tym
  % nakryję na tego typu oszustwach, to więcej niż $3.0$ u~mnie
  % \alert{nie dostanie}. Chyba, że~przymkniemy oko i~udamy, że~tego nie było.
  % Wszystko zależy.

  % Ściągania nie da~się całkiem wyeliminować, ale można je ograniczyć.
  % A~jak ktoś ściąga tak, żeby ślepy zauważył, to może mieć pretensje tylko
  % i~wyłącznie do siebie. ;)

  % Gdy chodzi o~zadania domowe i~projekt, to należy próbować rozwiązać je
  % możliwie samodzielnie. W~razie napotkania problemów nie tylko można,
  % ale i~należy prosić o~pomoc kolegów, korzystać z~materiałów w~internecie
  % i~używać programów takich jak Chat\textsc{gpt}.

  Na zajęciach, poza testami, też można z~tych pomocy korzystać, ale nie do
  tego stopnia, by czyjś wkład własny został zredukowany do metody
  Copy’ego-Pejsta.

  \vspace{-0.5em}





  \begin{figure}

    \label{fig:The-best-salute}

    \centering


    \includegraphics[scale=0.17]
    {./Presentations-pictures/The-best-salute.jpg}

  \end{figure}

  % Oczywiście, nikt nie straci punktów za oddanie skomplikowanego programu,
  % ale jest wiele powodów, by poprzestać na pisaniu prostego kodu.
  % Po~pierwsze, to zwykle oszczędza czas spędzony na~rozwiązywaniu zadań
  % na~ten przedmiot. Po~drugie, prostota i~czytelność kodu są naprawdę ważne
  % przy tworzeniu programów na~każdym poziomie.

  % Jeżeli ktoś oddaje trudne rozwiązywania zadań, bo chce spróbować swoich
  % sił w~programowaniu, to oczywiście, naszym obowiązkiem jest poprawić
  % ten zestaw bez marudzenia. Wszystkie te uwagi nie są po to byśmy
  % my mieli prostszą pracę przy poprawianiu zdań, tylko by Państwo nie
  % trudzili~się niepotrzebnie.

  % Proszę też pamiętać, że~obowiązuje prawo Kernighana. \\
  % \textit{Debugowanie programu jest dwa razy trudniejsze, niż jego
  %   napisanie. Jeśli więc napiszesz program w~najbardziej pomysłowy sposób
  %   jaki możesz, to nie jesteś wystarczająco inteligentny by go zdebugować.}

  % Ostrzegam, ja bardzo \alert{nie lubię} ściągania. Student który
  % oblał~$20$ poprawkę nie będzie miał u~mnie takich problemów, jak osoba
  % którą złapię na ściąganiu. Moja zasada jest taka, że~jak kogoś na tym
  % nakryję na tego typu oszustwach, to więcej niż $3.0$ u~mnie
  % \alert{nie dostanie}.

  % Ściągania nie da~się całkiem wyeliminować, ale można je ograniczyć.
  % A~jak ktoś ściąga tak, żeby ślepy zauważył, to może mieć pretensje tylko
  % i~wyłącznie do siebie. ;)

  % Gdy chodzi o~zadania domowe i~projekt, to należy próbować rozwiązać je
  % możliwie samodzielnie. W~razie napotkania problemów nie tylko można,
  % ale i~należy prosić o~pomoc kolegów, korzystać z~materiałów w~internecie
  % i~używać programów takich jak Chat\textsc{gpt}.

  % Na zajęciach, poza testami, też można z~tych pomocy korzystać, ale nie do
  % tego stopnia, by czyjś wkład własny został zredukowany do metody
  % Copy’ego-Pejsta.

\end{frame}
% ##################





% ##################
\begin{frame}
  \frametitle{O~oddawaniu prac domowych}


  Prosimy oddawać jako rozwiązania pliki o~nazwie \\
  \texttt{Imię-nazwisko-Zestaw-XX-Zad-YY.c} \\
  gdzie za \texttt{XX} i~\texttt{YY} trzeba oczywiście wstawić odpowiednie
  numery zestawu i~zadania. Nie jest to $100$\% obowiązkowe, ale gdy~się
  dostaje $50$~rozwiązań jednego zestawu zadań, to naprawdę nam pomaga.

  Ocenie podlega tylko kod źródłowy języka~C, proszę nie przesyłać nic
  więcej. Pliki zawierające ten kod prawie zawsze kończą~się na
  \texttt{.c}. Jeśli będzie trzeba przesłać innego typu, jak plik
  nagłówkowy (ang.~\textit{header}) kończący~się na~\texttt{.h}, to będzie
  to jawnie zaznaczone w~zadaniu. Acz na dzień dzisiejszy jest mała szansa,
  że~takie zadanie w~ogóle~się pojawi.

\end{frame}
% ##################





% ##################
\begin{frame}
  \frametitle{O~oddawaniu prac domowych}


  Jeśli plik kończy~się na~\texttt{.c} to prawie na pewno jest to plik
  z~kodem źródłowy w~języku~C. Jeśli plik kończy~się choćby
  na~\texttt{.cbp} czy \texttt{.layout} to na $99.99$\% plik zupełnie
  innego typu, których nie należy przesyłać. Doświadczenie mówi,
  że~większość osób na tym kursie jest początkującymi informatykami, więc
  nic dziwnego, że~niekiedy Państwo~się mylą i~przesyłając nie ten plik
  co~trzeba.

  Kilka drobny uwag. Plik pozbawiony rozszerzenia (brak kropki w~nazwie) to
  nie jest dobry pomysł. Plik ze spacją w~nazwie to proszenie~się
  o~problemy, gdy trzeba pracować w~systemie GNU/Linux (\textsc{bash}
  i~inne sprawy). Plik który ma~dwie lub~więcej kropek w~nazwie to wymysł
  Szatana i~produkt piekieł. Z~tego powodu do walki z~nim należy wysłać
  choćby Doom Marina
  (\colorhref{}{wersja z~$2016$ roku} jest całkiem dobra, w~nowsze
  nie grałem).

  %   Bardzo proszę nie skazywać mnie na obcowanie z~tymi diabelskimi tworami.
  % Acz za przesyłanie mi tych wytworów czeluści piekielnych nie stracą
  % Państwo punktów.

\end{frame}
% ##################





% ##################
\begin{frame}
  \frametitle{O~oddawaniu prac domowych}


  % Jeśli plik kończy~się na~\texttt{.c} to prawie na pewno jest to plik
  % z~kodem źródłowy w~języku~C. Jeśli plik kończy~się choćby
  % na~\texttt{.cbp} czy \texttt{.layout} to na $99.99$\% plik zupełnie
  % innego typu, których nie należy przesyłać. Doświadczenie mówi,
  % że~większość osób na tym kursie jest początkującymi informatykami, więc
  % nic dziwnego, że~niekiedy Państwo~się mylą i~przesyłając nie ten plik
  % co~trzeba.

  % Kilka drobny uwag. Plik pozbawiony rozszerzenia (brak kropki w~nazwie) to
  % nie jest dobry pomysł. Plik ze spacją w~nazwie to proszenie~się
  % o~problemy, gdy trzeba pracować w~systemie GNU/Linux (\textsc{bash}
  % i~inne sprawy). Plik który ma~dwie lub~więcej kropek w~nazwie to wymysł
  % Szatana i~produkt piekieł. Z~tego powodu do walki z~nim należy wysłać
  % choćby Doom Marina
  % (\colorhref{}{wersja z~$2016$ roku} jest całkiem dobra, w~nowsze
  % nie grałem).

    Bardzo proszę nie skazywać mnie na obcowanie z~tymi diabelskimi tworami.
  Acz za przesyłanie mi tych wytworów czeluści piekielnych nie stracą
  Państwo punktów.

\end{frame}
% ##################










% % ##################
% \begin{frame}
%   \frametitle{Rozwiązywanie zadań}

%   \vspace{-0.5em}




% \end{frame}
% % ##################





% ##################
\begin{frame}
  \frametitle{A~co~z~AI?}

  \vspace{-0.5em}


  \begin{figure}

    \label{fig:Coping-for-others}

    \centering


    \includegraphics[scale=0.415]
    {./Presentations-pictures/Copying-from-others.jpg}

  \end{figure}

\end{frame}
% ##################





% ##################
\begin{frame}
  \frametitle{A~co~z~AI?}

  \vspace{-0.5em}


  \begin{figure}

    \label{fig:Impact-of-ChatGPT-One-view}

    \centering


    \includegraphics[scale=0.38]
    {./Presentations-pictures/Impact-of-ChatGPT-One-view.jpg}

  \end{figure}

\end{frame}
% ##################






% % ##################
% \begin{frame}
%   \frametitle{Prawo Kernighana}




% \end{frame}
% % ##################










% ######################################
\section{Czego~się nauczymy, a~czego nie}
% ######################################


% ##################
\begin{frame}
  \frametitle{Semestr z~językiem~C i~co~dalej?}



\end{frame}
% ##################










% ######################################
\section{Wymagania i~materiały do nauki}
% ######################################


% ##################
\begin{frame}
  \frametitle{Materiały do nauki, błędy i~uwagi}


  Zarówno w~materiałach z~wykładu, jak i~w~tych do ćwiczeń, były
  i~obawiam~się, że~wciąż są błędy. Pragniemy by te zajęcia i~towarzyszące
  im materiały były możliwie merytoryczne, proste, łatwe w~zrozumieniu
  i~pozbawione błędów, proszę jednak uwierzyć, że~osiągnięcie tego jest
  naprawdę trudne.

  W~razie znalezienia jakiegokolwiek błędu lub jakichkolwiek uwag
  merytorycznych do zajęć lub dostępnych materiałów proszę zgłaszać to
  przed lub po zajęciach lub też pisać pod adres \email. Proszę w~tytule
  napisać coś w~stylu „Błędy w~materiałach”, „Uwagi do zajęć”, inaczej
  email może na długo zaginąć wśród dużej ilości tych, które dostaję od
  Państwa z~rozwiązanymi zadaniami.

  % Do tego na rok $2024$ priorytet ma dla mnie przygotowanie
  % materiałów, które pokrywają cały zakres tego przedmiotu. Zaś~usuwanie,
  % z~nich różnorakich niekonsekwencji jest sprawą drugorzędną.

\end{frame}
% ##################





% ##################
\begin{frame}
  \frametitle{Styl pisania w~języku~C}


  Można i~należy zgłaszać też uwagi do materiałów z~wykładów, bezpośrednio
  do~prowadzącego albo do mnie. Mój wpływ na stan materiałów do wykładów
  jest oczywiście skromniejszy, niż~na moje własne materiały.
  Mimo tego, będę próbował nakłonić prowadzącego, by poprawiał znaleziony
  błędy i~wprowadzał zasadne zmiany.

  Mówi~się, że~w~pewnych językach programowania istnieje tylko jeden sposób
  zrobienia czegoś, w~innych (prawie) każdą rzecz można zrobić na kilka
  sposobów. Język~C należy do tej drugiej grupy. Ja na zajęciach będę
  prezentował styl i~podejście do korzystania z~języka~C, który ja uważam
  za optymalny.

  Doświadczenie poprzednich lat uczy, że~ja i~prowadzący mamy bardzo różne
  opinie na temat tego jak wygląda dobry styl pisania kodu w~C. Wiele
  z~przykładów kodu które w~poprzednich latach pojawiły~się na wykładzie,
  ja uważam za~przykład złego stylu i~złych praktyk. O~tym jeszcze za
  chwilę.

\end{frame}
% ##################





% ##################
\begin{frame}
  \frametitle{Materiały do nauki}


  Na Sake będzie dostępna w~formacie \textsc{pdf} \alert{lista zagadnień do
    opanowania z~tego przedmiotu}, która będzie główny punktem odniesieniem
  przy tworzeniu pytań testowych. Jak również dwa listy materiałów do nauki.
  Jedna lista normalna, druga dla bardzo ambitnych osób.

  Będą tam również dostępne te prezentacje w~formacie \textsc{pdf}ów.
  W~formie źródłowej (plików \LaTeX a) są dostępna na serwisie GitHub.
  Każdy kto ma na komputerze program Git i~dostęp do internetu może je
  pobrać wpisując \\
  \texttt{\$ git clone https://github.com/KZiemian/Presentation} \\
  Znajdują~się one w~katalogu „Podstawy-informatyki-ETC-Prezentacje”.

  Można też obejrzeć to repozytorium jak normalny człowiek. Czyli
  w~przeglądarce: \\
  \colorhref{https://github.com/KZiemian/Presentation}
  {https://github.com/KZiemian/Presentation}.

\end{frame}
% ##################





% ##################
\begin{frame}
  \frametitle{Materiały do nauki}


  W~serwisie Sake są dostępne pliki zawierające kod~C wraz oraz komentarze
  objaśniające jak on działa. Będziemy~się też starali przygotować możliwie
  duży zestaw programów, które ilustrują, w~naszej ocenie, dobry styl
  pisania programów w~języku~C. Jednak to drugie zadanie jest ze względu
  na ilość możliwych wyborów, dość trudne do wykonania.

\end{frame}
% ##################





% ##################
\begin{frame}
  \frametitle{Czego od Państwa oczekujemy?}


  Proszę zwrócić uwagę, że~ze względu na charakter tych zajęć, wystarczające
  jest, żeby o~pewnych rzeczach wymienionych na liście zagadnień mieli
  Państwo bardzo ogólne i~podstawowe pojęcie. Nawet w~sytuacji, gdy na
  kursie było o~danym zagadnieniu powiedziane znacznie więcej.

  Przykładowo, jest wystarczające, żeby Państwo wiedzieli, że~kompilator
  języka~C jest to program, który przetwarza kod napisany w~języku~C
  w~program, który jest napisany w~języku zrozumiałym dla komputera.
  Nawet jeśli na zajęciach wspomnimy czym są takie części kompilatora
  jak lekser czy parser, nie jest wymagane by Państwo po tym kursie
  wiedzieli, że takie rzeczy istnieją, nie mówiąc już o~znajomości tego
  co robią.

\end{frame}
% ##################





% ##################
\begin{frame}
  \frametitle{Czego od Państwa oczekujemy?}


  Na liście zagadnień postaramy~się wyróżnić tego typu „powierzchowne”
  zagadnienia. Dodać należy, że~gdy chodzi o~pozostałe pytania, wymagana
  jest dobra znajomość na \alert{poziomie tego co było prezentowane na
    kursie}, nie zaś taka jaka jest zawarta w~standardzie języka~C.
  Bądźmy poważni, to jest tylko kurs podstaw informatyki.

  Przykładowo, jak ktoś będzie umiał napisać program który dodaje $2.71$
  do~$3.14$ to na \alert{ten} kurs wystarczy. Przynajmniej by dostać dobrą
  ocenę z~laboratoriów, ocena z~\alert{całego} kursu to inna sprawa.
  Nie musicie Państwo znać całego artykułu
  \colorhref{https://dl.acm.org/doi/10.1145/103162.103163}{\textit{Co każdy
      informatyk musi wiedzieć o~arytmetyce zmiennoprzecinkowej}}.

  % Na liście zagadnień postaramy~się wyróżnić tego typu pytania w~specjalny
  % sposób. Dodać należy, że~gdy chodzi o~pozostałe pytania, wymagana jest
  % dobra znajomość na \alert{poziomie tego co było prezentowane na kursie},
  % nie zaś taka jaka jest zawarta w~standardzie??? języka~C. Bądźmy poważni,
  % to jest tylko kurs podstaw informatyki.

  % Przykładowo, jak ktoś będzie umiał napisać program który dodaje $2.71$
  % do~$3.14$ to na \alert{ten} kurs wystarczy. Przynajmniej by dostać dobrą
  % ocenę z~laboratoriów, ocena z~\alert{całego} kursu to inna sprawa.
  % Nie musicie Państwo znać całego
  % \colorhref{https://dl.acm.org/doi/10.1145/103162.103163}{\textit{Co każdy
  %     informatyk musi wiedzieć o~arytmetyce zmiennoprzecinkowej}}.

\end{frame}
% ##################





























% % ##################
% \jagiellonianendslide{Czy są jakieś pytania do tej części?}
% % ##################

































% ####################################################################
% ####################################################################
% Bibliography

\printbibliography





% ############################
% End of the document

\end{document}
