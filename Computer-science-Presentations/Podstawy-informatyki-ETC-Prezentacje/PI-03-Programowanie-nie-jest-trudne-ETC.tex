% ------------------------------------------------------------------------------------------------------------------
% Basic configuration of Beamera class and Jagiellonian theme
% ------------------------------------------------------------------------------------------------------------------
\RequirePackage[l2tabu, orthodox]{nag}



\ifx\PresentationStyle\notset
  \def\PresentationStyle{dark}
\fi



% Options: t -- align text to the top of the frame
\documentclass[10pt,t]{beamer}
\mode<presentation>
\usetheme[style=\PresentationStyle]{jagiellonian}





% ------------------------------------------------------------------------------------
% Procesing configuration files of Jagiellonian theme located in
% the directory "preambule"
% ------------------------------------------------------------------------------------
% Configuration for polish language
% Need description
\usepackage[polish]{babel}
% Need description
\usepackage[MeX]{polski}



% ------------------------------
% Better support of polish chars in technical parts of PDF
% ------------------------------
\hypersetup{pdfencoding=auto,psdextra}

% Package "textpos" give as enviroment "textblock" which is very usefull in
% arranging text on slides.

% This is standard configuration of "textpos"
\usepackage[overlay,absolute]{textpos}

% If you need to see bounds of "textblock's" comment line above and uncomment
% one below.

% Caution! When showboxes option is on significant ammunt of space is add
% to the top of textblock and as such, everyting put in them gone down.
% We need to check how to remove this bug.

% \usepackage[showboxes,overlay,absolute]{textpos}



% Setting scale length for package "textpos"
\setlength{\TPHorizModule}{10mm}
\setlength{\TPVertModule}{\TPHorizModule}


% ---------------------------------------
% Packages written for lectures "Geometria 3D dla twórców gier wideo"
% ---------------------------------------
% \usepackage{./Geometry3DPackages/Geometry3D}
% \usepackage{./Geometry3DPackages/Geometry3DIndices}
% \usepackage{./Geometry3DPackages/Geometry3DTikZStyle}
% \usepackage{./ProgramowanieSymulacjiFizykiPaczki/ProgramowanieSymulacjiFizykiTikZStyle}
% \usepackage{./Geometry3DPackages/mathcommands}


% ---------------------------------------
% TikZ
% ---------------------------------------
% Importing TikZ libraries
\usetikzlibrary{arrows.meta}
\usetikzlibrary{positioning}





% % Configuration package "bm" that need for making bold symbols
% \newcommand{\bmmax}{0}
% \newcommand{\hmmax}{0}
% \usepackage{bm}




% ---------------------------------------
% Packages for scientific texts
% ---------------------------------------
% \let\lll\undefined  % Sometimes you must use this line to allow
% "amsmath" package to works with packages with packages for polish
% languge imported
% /preambul/LanguageSettings/JagiellonianPolishLanguageSettings.tex.
% This comments (probably) removes polish letter Ł.
\usepackage{amsmath}  % Packages from American Mathematical Society (AMS)
\usepackage{amssymb}
\usepackage{amscd}
\usepackage{amsthm}
\usepackage{siunitx}  % Package for typsetting SI units.
\usepackage{upgreek}  % Better looking greek letters.
% Example of using upgreek: pi = \uppi


\usepackage{calrsfs}  % Zmienia czcionkę kaligraficzną w \mathcal
% na ładniejszą. Może w innych miejscach robi to samo, ale o tym nic
% nie wiem.










% ---------------------------------------
% Packages written for lectures "Geometria 3D dla twórców gier wideo"
% ---------------------------------------
% \usepackage{./ProgramowanieSymulacjiFizykiPaczki/ProgramowanieSymulacjiFizyki}
% \usepackage{./ProgramowanieSymulacjiFizykiPaczki/ProgramowanieSymulacjiFizykiIndeksy}
% \usepackage{./ProgramowanieSymulacjiFizykiPaczki/ProgramowanieSymulacjiFizykiTikZStyle}





% !!!!!!!!!!!!!!!!!!!!!!!!!!!!!!
% !!!!!!!!!!!!!!!!!!!!!!!!!!!!!!
% EVIL STUFF
\if\JUlogotitle1
\edef\LogoJUPath{LogoJU_\JUlogoLang/LogoJU_\JUlogoShape_\JUlogoColor.pdf}
\titlegraphic{\hfill\includegraphics[scale=0.22]
{./JagiellonianPictures/\LogoJUPath}}
\fi
% ---------------------------------------
% Commands for handling colors
% ---------------------------------------


% Command for setting normal text color for some text in math modestyle
% Text color depend on used style of Jagiellonian

% Beamer version of command
\newcommand{\TextWithNormalTextColor}[1]{%
  {\color{jNormalTextFGColor}
    \setbeamercolor{math text}{fg=jNormalTextFGColor} {#1}}
}

% Article and similar classes version of command
% \newcommand{\TextWithNormalTextColor}[1]{%
%   {\color{jNormalTextsFGColor} {#1}}
% }



% Beamer version of command
\newcommand{\NormalTextInMathMode}[1]{%
  {\color{jNormalTextFGColor}
    \setbeamercolor{math text}{fg=jNormalTextFGColor} \text{#1}}
}


% Article and similar classes version of command
% \newcommand{\NormalTextInMathMode}[1]{%
%   {\color{jNormalTextsFGColor} \text{#1}}
% }




% Command that sets color of some mathematical text to the same color
% that has normal text in header (?)

% Beamer version of the command
\newcommand{\MathTextFrametitleFGColor}[1]{%
  {\color{jFrametitleFGColor}
    \setbeamercolor{math text}{fg=jFrametitleFGColor} #1}
}

% Article and similar classes version of the command
% \newcommand{\MathTextWhiteColor}[1]{{\color{jFrametitleFGColor} #1}}





% Command for setting color of alert text for some text in math modestyle

% Beamer version of the command
\newcommand{\MathTextAlertColor}[1]{%
  {\color{jOrange} \setbeamercolor{math text}{fg=jOrange} #1}
}

% Article and similar classes version of the command
% \newcommand{\MathTextAlertColor}[1]{{\color{jOrange} #1}}





% Command that allow you to sets chosen color as the color of some text into
% math mode. Due to some nuances in the way that Beamer handle colors
% it not work in all cases. We hope that in the future we will improve it.

% Beamer version of the command
\newcommand{\SetMathTextsColor}[2]{%
  {\color{#1} \setbeamercolor{math text}{fg=#1} #2}
}


% Article and similar classes version of the command
% \newcommand{\SetMathTextColor}[2]{{\color{#1} #2}}










% ---------------------------------------
% Commands for setting background pictures for some slides
% ---------------------------------------
\newcommand{\TitleBackgroundPicture}
{./PresentationPictures/CommonPictures/Cute_dragon_BG_dark.png}
\newcommand{\SectionBackgroundPicture}
{./PresentationPictures/CommonPictures/Cute_dragon_small_BG_light.png}



\newcommand{\TitleSlideWithPicture}{
  \begingroup

  \usebackgroundtemplate{ % \hspace*{-11.5em}
    \includegraphics[height=\paperheight]{\TitleBackgroundPicture}}

  \maketitle

  \endgroup
}





\newcommand{\SectionSlideWithPicture}[1]{%
  \begingroup

  \usebackgroundtemplate{ % \hspace*{-11.5em}
    \includegraphics[height=\paperheight]{\SectionBackgroundPicture}}

  \setbeamercolor{titlelike}{fg=normal text.fg}

  \section{#1}

  \endgroup
}





\newcommand{\EndingSlide}[1]{%
  \begin{frame}[standout]

    \begingroup

    \color{jFrametitleFGColor}

    #1

    \endgroup

  \end{frame}
}










% ------------------------------------------------------
% BibLaTeX
% ------------------------------------------------------
% Package biblatex, with biber as its backend, allow us to handle
% bibliography entries that use Unicode symbols outside ASCII.
\usepackage[
language=polish,
backend=biber,
style=alphabetic,
url=false,
eprint=true,
]{biblatex}

\addbibresource{Podstawy-informatyki-ETC-Bibliography.bib}





% ------------------------------------------------------
% Importing packages, libraries and setting their configuration
% ------------------------------------------------------
% Library improving positioning of nodes in graphs
% \usetikzlibrary{positioning}





% ------------------------------------------------------
% Local packages
% ------------------------------------------------------
% Local configuration of this particular presentation.
\usepackage{./Local-packages/local-settings}

% Package containing various command useful for working with a text.
\usepackage{./Local-packages/general-commands}

% % Styles for drawing diagrams
% \usepackage{./Local-packages/PGF-TikZ-Diagram-styles}

% % Jagiellonian theme's colors
% \usepackage{./Local-packages/jagiellonian-theme-colors}










% ------------------------------------------------------------------------------------------------------------------
\title{Podstawy informatyki z~językiem~C}
\subtitle{Programowanie nie jest trudne, debugowanie już tak}

\author{Kamil Ziemian \\
  \email}


% \date{}
% ------------------------------------------------------------------------------------------------------------------










% ####################################################################
% Beginning of the document
\begin{document}
% ####################################################################





% ######################################
% Number of chars: 7k+,
% Text is adjusted to the left and words are broken at the end of the line.
\RaggedRight
% ######################################





% ######################################
\maketitle
% ######################################





% ##################
\begin{frame}
  \frametitle{Spis treści}


  \tableofcontents

\end{frame}
% ##################










% ######################################
\section{Programowanie nie jest trudne. Debugowanie już tak}
% ######################################


% ##################
\begin{frame}
  \frametitle{Programowanie nie jest trudne. Debugowanie już
    tak}


  \textit{Debugowanie programu jest dwa razy trudniejsze, niż jego
    napisanie. Jeśli więc napiszesz program w~najbardziej inteligentny
    sposób jaki możesz, to nie jesteś wystarczająco mądry by go
    zdebugować.} \\
  Prawo
  \colorhref{https://en.wikipedia.org/wiki/Brian\_Kernighan}{Kernighana}.

  Proszę pamiętać, że~pisanie programów nie jest trudne, ich debugowanie już
  tak. Na tych zajęciach będziemy próbować wspólnie stawiać pierwsze kroki
  na drodze nauki debugowania programów.

  Przypomnijmy jak działa język~C. Piszemy program z~kodem źródłowym
  w~języku~C, który ma być zawarty w~pliku, którego nazwa \alert{kończy~się
    na~\texttt{.c}}. Następnie program zwany kompilatorem na podstawie
  tego kodu źródłowego generuje program wykonywalny \texttt{prog.out}.
  Kompilator to też program komputerowy, więc to my, ludzie, ustalamy
  jego zasady działania.

\end{frame}
% ##################





% ##################
\begin{frame}
  \frametitle{Błędy}


  Twórcy języka~C przyjęli, że~jeśli coś jest nie tak, to kompilator może
  zwrócić jedną z~dwóch rodzai wiadomości: \textbf{błąd}
  (ang.~\textit{error}) i~\textbf{ostrzeżenie} (ang.~\textit{warning}).
  Różnica między nimi jest następująca.

  Zwrócenie błędu przez kompilator można zrozumieć, jako następująca
  informację od niego. „Człowieku, ten kod który napisałeś w~linii
  numer~X (mniej więcej) jest tak zły, że~ja nie wygenerują ci pliku
  wykonywalnego. Napraw ten błąd i~spróbuj jeszcze raz.”

  Ze względu na to jak skomplikowany jest proces kompilacji oraz na~wiek
  języka~C, informacje o~błędach są często bardzo trudne w~zrozumieniu
  i~niezbyt precyzyjne. To naprawdę skomplikowany temat. Jedną z~rzeczy,
  które należy rozumieć jest to, że~jeśli kompilator twierdzi, że~błąd
  jest w~linii $20$, to on zwykle w~tej linii jest, ale kompilator
  może~się mylić i~błąd znajduje~się wcześniej albo później. Bo dlaczego
  życie miałoby być proste?

\end{frame}
% ##################





% ##################
\begin{frame}
  \frametitle{Ostrzeżenia}


  Jeśli kompilator zwróci ostrzeżenie, to można to rozumieć jakby mówił:
  „Kod w~tym miejscu wygląda na zły, ale skoro tak chcesz, to wygeneruję
  program wykonalny zgodnie z~tym kodem. Jeśli samolot spadnie przez to
  z~nieba, to ja nie biorę za to odpowiedzialności.”

  Warto dodać, że~twórcy języka Go, stwierdzili, że~kompilator nie
  powinien zwracać ostrzeżeń, bo dzieje~się dokładnie to czego można~się
  było spodziewać: programiści ignorują ostrzeżenia. Bo czy program~się
  uruchamia? Tak. To czym~się przejmować? Skutkiem tego istniejący
  kod jest pełen podejrzanych miejsc, których programistą nie chciało~się
  zwyczajnie poprawić. Z~tego powodu w~języku Go są tylko błędy, więc
  programista musi je wszystkie poprawić.

  Na tym przedmiocie obowiązuje zasada, że~każdy program ma zostać
  doprowadzony do stanu, w~którym kompilator podczas jego przetwarzania
  nie zgłasza żadnego ostrzeżenia. Państwo na pewno rozumieją, dlaczego
  takie zasady przyjęliśmy.

\end{frame}
% ##################





% ##################
\begin{frame}
  \frametitle{Błąd? Ostrzeżenie? Who cares?}


  Niestety, jedną z~problemów w~pracy z~językiem~C jest to, że~czasem
  dostajemy od kompilatora ostrzeżenie, a~jak~się przyjrzymy, to okazuje~się,
  że~wystąpił błąd. Czyli widzimy, że~kompilator pisze „Ostrzegam ciebie\ldots”
  i~myślimy sobie „Okej, uruchomię plik i~zobaczę jak działa.”, a~pliku nie
  ma. Bo czemu życie miałoby być proste?

  Można by powiedzieć pewne rzeczy na usprawiedliwienie kompilatora,
  ale~to temat na inny przedmiot. Przyjmijmy po prostu, że~kompilator
  to dość uciążliwy przyjaciel.

\end{frame}
% ##################










% ######################################
\section{Typy zmiennych}
% ######################################


% ##################
\begin{frame}
  \frametitle{Czym są typy zmiennych?}


  Widzieliśmy już kilka typów zmiennych, takich jak liczby całkowite
  i~liczby zmiennoprzecinkowe: \\
  \texttt{int intVar1 = 0;} \\
  \texttt{int float64Var1 = 0.0;} \\
  Czym jednak dokładniej jest typ zmiennej? To wbrew pozorom jest bardzo
  głęboki problem, który pojawił~się w~rozważaniach
  filozoficzno-matematycznych kilka dekad wcześniej, nim ludzkość
  skonstruowała pierwszy komputer z~prawdziwego zdarzenia. Powinno być
  oczywiste, że~na takim przedmiocie jak \textit{Podstawy informatyki}
  możemy omówić ten temat tylko na intuicyjnym poziomie.

  Dzisiejsze komputery buduje~się w~ten sposób, że~najmniejszą porcją pamięci
  jakiej możemy używać jest jeden bajt, czyli $8$~bitów. Techniczny
  termin na taką najmniejszą jednostkę pamięci nazywamy
  \textbf{słowem} (ang.~\textit{word}).

\end{frame}
% ##################





% ##################
\begin{frame}
  \frametitle{Abstrakcja w~informatyce}


  Historycznie istniały komputery, których najmniejszą jednostką było
  powiedzmy $6$~bitów, ale to nie są zajęcia z~historii informatyki.

  Moglibyśmy korzystać komputera operując tylko na bajtach jako takich,
  ale byłaby to męka nie do pojęcia. Jeśli ktoś z~Państwa pisał kiedyś
  w~jakimś dialekcie assemblera, wie o~co chodzi.

  By to zilustrować, proszę sobie wyobrazić, że~zamiast zdjęcia na dysku,
  musimy operować ręcznie na milionie~(!) bajtów które taki rysunek tworzą.
  Kopiowanie zdjęcia wyglądałoby wtedy mniej więcej tak. Weź bajty
  o~numerach od $1 \, 555 \, 000$ do $2 \, 554 \, 999$ i~skopiuj
  zawartość każdego z~nich do bajtów o~numerach $11 \, 000 \, 073$
  do~$12 \, 000 \, 072$. Koszmar na~jawie!

  By tego uniknąć wprowadzamy coś, co w~informatyce określamy terminem
  \textbf{abstrakcja}.

\end{frame}
% ##################





% ##################
\begin{frame}
  \frametitle{Abstrakcja w~informatyce}


  Zdjęcie na komputerze to pewna abstrakcja, która zwalnia nas z~myślenia
  o~nim jako o~milionie bajtów, które tworzą ten rysunek na~dysku. Inny
  przykładem tego typu zachowania jest to, że zamiast mówić „To jest
  zwierzę, które jest ssakiem, ma cztery nogi, ogon, futro, jest
  drapieżnikiem,\ldots” powiemy „To jest pies.”. Takie dzielenie rzeczy na różne
  kategorie i~operować tak wprowadzonymi kategoriami okazuje~się być bardzo
  wygodne. Mamy nadzieję, że~takie wyjaśnienie koncepcji abstrakcji
  w~informatyce jest dla Państwa wystarczająca.

  Typy zmiennych istnieją z~tego samego powodu, zamiast myśleć „Tu mam cztery
  bajty.”, prościej jest myśleć „Tu mam liczbę całkowitą”. Poza tym, typy
  zmiennych pozwalają komputerowi lepiej kontrolować działanie programów,
  to jest jednak temat rzeka, zbyt obszerny na ramy tego przedmiotu.

\end{frame}
% ##################





% ##################
\begin{frame}
  \frametitle{Pożytek z~typów}


  W~poprzednim paragrafie napisaliśmy, że~grupa czterech bajtów reprezentuje
  liczbę całkowitą i~tak rzeczywiście często jest w~komputerach. Pojedyncza
  liczba całkowita często zajmuje dokładnie cztery bajty w~pamięci, acz
  nie jest to teraz super ważne.

  Rozpatrzmy następujący, mocno uproszczony przykład. Powiedzmy, że~używamy
  dwóch rodzajów plików: graficznych oraz tekstowych. Dzięki typom, nasz
  program może rozpoznać, że~chcemy zapisać coś do~pliku graficznego, ale
  otwarliśmy plik tekstowy i~zgłosi błąd, co jest bardzo pożądanym
  zachowaniem. Jeśli Państwo jeszcze tego nie widzą, to myślę, że~na tym
  przedmiocie napotkamy dostatecznie wiele przykładów, które pokazując,
  czemu jest to dobra rzecz.

\end{frame}
% ##################





% ##################
\begin{frame}
  \frametitle{Pożytek z~typów}


  Należy też dodać, że~w~różnych językach granica między typami jest mniej
  lub bardziej ostra. Przykładowo, czy w~miejscu, gdzie program oczekuje
  liczby zmiennoprzecinkowej, mogę użyć liczby całkowitej~$1$? Wedle
  niektórych języków, tak, bo wszak wiadomo jak zamienić liczbę
  całkowitą~$1$ na liczbę zmiennoprzecinkową~$1.0$. Wedle innych języków,
  nie, bo~taka zmiana jednego typu liczbowego na drugi, może prowadzić do
  błędów numerycznych, których programista może nie być świadomy pisząc
  program.

  W~tym drugim typie~;) języków, programista musi jawnie poprosić
  o~\textbf{konwersje typów}. Czyli musi napisać kod, który mówi coś
  w~stylu „Wiem, że~to jest liczba całkowita~$1$ i~w~tym miejscu masz
  ją zamienić na liczbę zmiennoprzecinkową~$1.0$. Ja biorę odpowiedzialność
  za to co~się stanie.” O~konwersji typów powiem w~swoim czasie.

  % Język~C stosuje podejście mieszane. Bo czemu życie miałoby być proste?
  % Przykładowo, liczby całkowite zamienia na zmiennoprzecinkowe bez
  % pytania~się programisty o~pozwolenie, przy innych typach musimy~się sami
  % napracować. Wolelibyśmy uniknąć szczegółowej dyskusji tego typu zagadnień,
  % gdyż są ważniejsze rzeczy, którym należy poświęcić czas na~przedmiocie
  % \textit{Podstawy informatyki}.

\end{frame}
% ##################





% ##################
\begin{frame}
  \frametitle{Pożytek z~typów}


  Język~C stosuje podejście mieszane. Bo czemu życie miałoby być proste?
  Przykładowo, liczby całkowite zamienia na zmiennoprzecinkowe bez
  pytania~się programisty o~pozwolenie, przy innych typach musimy~się sami
  napracować. Wolelibyśmy uniknąć szczegółowej dyskusji tego typu zagadnień,
  gdyż są ważniejsze rzeczy, którym należy poświęcić czas na~przedmiocie
  \textit{Podstawy informatyki}.

\end{frame}
% ##################










% ######################################
\section{Zapisywanie liczb w~komputerze}
% ######################################


% ##################
\begin{frame}
  \frametitle{Komputer to maszyna do liczenia}


  W~samej angielskiej nazwie \textit{computer} zawiera~się informacja,
  że~jest to maszyna przeznaczona do liczenia, więc siłą rzeczy należy
  coś powiedzieć o~tym jak komputer zarządza liczbami. Nie będziemy przy
  tym omawiali wszystkich zawiłości jakie dotyczą tego problemu. Na tym
  przedmiocie chcemy przekazać tylko tyle informacji, ile jest
  naszym zdaniem potrzebne do wyrobienia sobie podstawowych intuicji.
  Za głębsze wniknięcie w~naturę rzeczy odpowiedzialne są inne przedmioty.

  Zaczniemy od pewnie dość nudnego dla większości osób przyjrzenia~się
  potęgom liczby~$2$. W~informatyce bardzo lubimy liczbę~$2$. By być wiernym
  duchowni języka~C równość dwóch liczb oznaczamy symbolem $==$. \\
  $2^{ \HorSpaceTwo 0 } == 1$ \\[0.1em]
  $2^{ \HorSpaceTwo 1 } == 2$ \\[0.1em]
  $2^{ \HorSpaceTwo 2 } == 4$

\end{frame}
% ##################





% ##################
\begin{frame}
  \frametitle{Wybrane potęgi liczby~$2$}


  $2^{ \HorSpaceTwo 3 } == 8$ \\[0.1em]
  $2^{ \HorSpaceTwo 4 } == 16$ \\[0.1em]
  $2^{ \HorSpaceTwo 5 } == 32$ \\[0.1em]
  $2^{ \HorSpaceTwo 6 } == 64$ \\[0.1em]
  $2^{ \HorSpaceTwo 7 } == 128$ \\[0.1em]
  $2^{ \HorSpaceTwo 8 } == 256$ \\[0.1em]
  $2^{ \HorSpaceTwo 9 } == 512$ \\[0.1em]
  $2^{ \HorSpaceTwo 10 } == 1 \; 024$ \\[0.1em]
  $2^{ \HorSpaceTwo 11 } == 2 \; 048$ \\[0.1em]
  $2^{ \HorSpaceTwo 12 } == 4 \; 096$ \\[0.1em]
  $2^{ \HorSpaceTwo 15 } == 32 \; 768$ \\[0.1em]
  $2^{ \HorSpaceTwo 16 } == 65 \; 536$

\end{frame}
% ##################





% ##################
\begin{frame}
  \frametitle{Wybrane potęgi liczby~$2$}


  Liczby $2^{ \HorSpaceTwo 16 } == 65 \; 536$
  i~$2^{ \HorSpaceTwo 16 } - 1 == 65 \; 535$ pojawiają~się
  często w~różnych działach techniki, gdyż często zachodzi potrzeba
  przesyłania informacji, której podstawową jednostką są dwa bajty,
  czyli $16$~bitów.

  $2^{ \HorSpaceTwo 17 } == 131 \; 072$ \\[0.1em]
  $2^{ \HorSpaceTwo 18 } == 262 \; 144$ \\[0.1em]
  $2^{ \HorSpaceTwo 19 } == 524 \; 288$ \\[0.1em]
  $2^{ \HorSpaceTwo 20 } == 1 \; 048 \; 576$ \\[0.1em]
  $2^{ \HorSpaceTwo 21 } == 2 \; 097 \; 152$ \\[0.1em]
  $2^{ \HorSpaceTwo 22 } == 4 \; 194 \; 304$ \\[0.1em]
  $2^{ \HorSpaceTwo 23 } == 8 \; 388 \; 608$ \\[0.1em]
  $2^{ \HorSpaceTwo 24 } == 16 \; 777 \; 216$ \\[0.1em]
  $2^{ \HorSpaceTwo 25 } == 33 \; 554 \; 432$ \\[0.1em]
  $2^{ \HorSpaceTwo 26 } == 67 \; 108 \; 864$ \\[0.1em]
  $2^{ \HorSpaceTwo 27 } == 134 \; 217 \; 728$

\end{frame}
% ##################





% ##################
\begin{frame}
  \frametitle{Wybrane potęgi liczby~$2$}


  $2^{ \HorSpaceTwo 28 } == 268 \; 435 \; 456$ \\[0.1em]
  $2^{ \HorSpaceTwo 29 } == 536 \; 870 \; 912$ \\[0.1em]
  $2^{ \HorSpaceTwo 30 } == 1 \; 073 \; 741 \; 824$ \\[0.1em]
  $2^{ \HorSpaceTwo 31 } == 2 \; 147 \; 483 \; 648$ \\[0.1em]
  $2^{ \HorSpaceTwo 32 } == 4 \; 294 \; 967 \; 296$

  Liczba $2^{ \HorSpaceTwo 32 } == 4 \; 294 \; 967 \; 296$ jest bardzo ważna
  dla wszystkich \\
  systemów i~programów $32$-bitowych, czyli takich w~których
  naturalną jednostką przesyłu informacji są właśnie $32$ bity ($4$~bajty).

  $2^{ \HorSpaceTwo 63 } == 9 \; 223 \; 372 \; 036 \; 854 \; 775 \; 808 \approx
  9 \cdot 10^{ 18 }$ \\
  Słownie: dziewięć trylionów, dwieście dwadzieścia trzy biliardy, trzysta
  siedemdziesiąt dwa biliony, trzydzieści sześć miliardów, osiemset
  pięćdziesiąt cztery miliony, siedemset siedemdziesiąt pięć tysięcy,
  osiemset osiem. \\
  Prościej: $9$~trylionów, $223$~biliardy, $372$~biliony,
  $36$~miliardy, $854$~miliony, $775$~tysięcy, $808$.

\end{frame}
% ##################





% ##################
\begin{frame}
  \frametitle{Wybrane potęgi liczby~$2$}


  $2^{ \HorSpaceTwo 64 } == 18 \; 446 \; 744 \; 073 \; 709 \; 551 \; 616 \approx
  18 \cdot 10^{ 18 }$ \\
  Słownie: osiemnaście trylionów, czterysta czterdzieści sześć biliardów,
  siedemset czterdzieści cztery biliony, siedemdziesiąt trzy miliardy,
  siedemset dziewięć milionów, pięćset pięćdziesiąt jeden milionów,
  sześćset szesnaście. \\
  Prościej: $18$ trylionów, $446$ biliardów, $744$ biliony, $73$ miliardy,
  $709$ milionów, $551$ tysięcy, $616$.

  Jak ktoś~się tej liczby nauczy na pamięć, to zasługuje na $5$~dodatkowych
  punktów.

  Liczba ta jest oczywiście bardzo ważna, dla systemów $64$-bitowych.

\end{frame}
% ##################





% ##################
\begin{frame}
  \frametitle{Systemy pozycyjny zapisu liczb naturalnych}


  Sposoby zapisu liczb w~komputerze to naprawdę temat na osobny wykład,
  taki jak metody numeryczne. My zaczniemy od~prostej dyskusji jak komputery
  zapisują liczby naturalne, w~miarę potrzeby omawiając bardziej zaawansowane
  tematy.

  Zacznijmy od krótkiego odświeżenia najlepiej nam znanego systemu
  pozycyjnego, czyli tego opartego o~liczbę~$10$. Liczby w~tym systemie
  czytamy od~prawej do lewej w~następujący sposób. \\
  $123 == 3 \cdot 10^{ \HorSpaceTwo 0 } + 2 \cdot 10^{ \HorSpaceTwo 1 } +
  1 \cdot 10^{ \HorSpaceTwo 2 } == 3 \cdot 1 + 2 \cdot 10 + 1 \cdot 100 == 3 + 20 + 100$ \\
  $456 == 6 \cdot 10^{ \HorSpaceTwo 0 } + 5 \cdot 10^{ \HorSpaceTwo 1 } +
  4 \cdot 10^{ \HorSpaceTwo 2 } == 6 \cdot 1 + 5 \cdot 10 + 4 \cdot 100 == 6 + 50 + 400$ \\
  $2345 == 5 \cdot 10^{ \HorSpaceTwo 0 } + 4 \cdot 10^{ \HorSpaceTwo 1 } +
  3 \cdot 10^{ \HorSpaceTwo 2 } + 2 \cdot 10^{ \HorSpaceTwo 3 } == 5 \cdot 1 + 4 \cdot 10 +
  3 \cdot 100 \, +$ \\
  $+ \, 2 \cdot 1000 == 5 + 40 + 300 + 2000$

  Wzorując~się na języku Go, liczby binarne będziemy zapisywać jako
  $0\text{b}1$, $0\text{b}01$, $0\text{b}111$,~etc. Tak jak w~systemie
  dziesiętnym, czytamy je od prawej do lewej. Prześledźmy teraz
  kilka przykładów.

\end{frame}
% ##################





% ##################
\begin{frame}
  \frametitle{System binarny zapisu liczb naturalnych}


  $0\text{b}0 == 0 \cdot 2^{ \HorSpaceTwo 0 } == 0 \cdot 1 == 0$ \\
  $0\text{b}1 == 1 \cdot 2^{ \HorSpaceTwo 0 } == 1 \cdot 1 == 1$ \\
  $0\text{b}10 == 0 \cdot 2^{ \HorSpaceTwo 0 } + 1 \cdot 2^{ \HorSpaceTwo 1 } ==
  0 \cdot 1 + 1 \cdot 2 == 2$ \\
  $0\text{b}11 == 1 \cdot 2^{ \HorSpaceTwo 0 } + 1 \cdot 2^{ \HorSpaceTwo 1 } ==
  1 \cdot 1 + 1 \cdot 2 == 1 + 2 == 3$ \\
  $0\text{b}111 == 1 \cdot 2^{ \HorSpaceTwo 0 } + 1 \cdot 2^{ \HorSpaceTwo 1 } +
  1 \cdot 2^{ \HorSpaceTwo 2 } == 1 \cdot 1 + 1 \cdot 2 + 1 \cdot 4 == 1 + 2 + 4 == 7$ \\
  $0\text{b}10101 == 1 \cdot 2^{ \HorSpaceTwo 0 } + 0 \cdot 2^{ \HorSpaceTwo 1 } +
  1 \cdot 2^{ \HorSpaceTwo 2 } + 0 \cdot 2^{ \HorSpaceTwo 3 } + 1 \cdot 2^{ \HorSpaceTwo 4 } ==
  1 \cdot 1 + 0 \cdot 2 + 1 \cdot 4 \, +$ \\
  $+ \, 0 \cdot 8 + 1 \cdot 16 == 1 + 4 + 16 == 21$

  Jak dobrze wszyscy wiemy, jednostkę która może przechowywać wartość
  $0$ albo $1$ nazywamy \textbf{bitem}. Bity grupujemy dziś w~\textbf{bajty},
  jeden bajt to $2^{ \HorSpaceTwo 3 } == 8$ bitów. W~tym, że~bajt zawiera
  $2^{ \HorSpaceTwo 3 }$ bitów widać upodobanie informatyki do potęgi
  liczby~$2$. Skąd jednak jest to upodobanie.

  Wydaje~się, że~podstawą tego są kwestie techniczne. Zbudowanie urządzenia,
  które może przechowywać dwie różne wartości ($0$ lub $1$), jest prostsze,
  niż takiego, które przechowywałoby trzy wartości ($0$, $1$ lub $2$).
  Warto powiedzieć o~tym kilka słów więcej.

\end{frame}
% ##################





% ##################
\begin{frame}
  \frametitle{System binarny zapisu liczb naturalnych}


  Pozostając na bardzo ogólnym poziomie, w~XX wieku eksperymentowano
  z~układami elektronicznymi, których podstawowa jednostka pamięci, była
  w~stanie przechowywać jedną z~trzech wartości ($0$, $1$ lub $2$), jedną
  z~czterech ($0$, $1$, $2$ lub $3$),~etc. Jednostki te odpowiadały temu,
  czemu w~dzisiejszym komputerze odpowiada bit. Gdyby ludzkość
  zdecydowała~się budować komputery w systemie trójkowym, pewnie dzisiaj
  operowalibyśmy „bajtami” o~długości $3^{ 2 } == 9$ „bitów”.

  Różne względy, które są bardziej odpowiednie na bardzo obszerny wykład
  z~elektroniki, sprawiły, iż~zdecydowaliśmy~się oprzeć pamięć komputera
  o~jednostki dwustanowe, czyli właśnie dobrze nam znane bity. Czy tak już
  pozostanie? Przewidywanie przyszłości jest trudne.

  Zadajmy sobie teraz pytanie, ile liczb możemy zapisać za pomocą jednego
  bitu? Dwie $0\text{b}0 == 0$ i~$0\text{b}1 == 1$. Ile zaś za pomocą
  dwóch bitów? Cztery: $0\text{b}00 == 0$, $0\text{b}01 == 1$,
  $0\text{b}10 == 2$ i~$0\text{b}11 == 3$.

\end{frame}
% ##################





% ##################
\begin{frame}
  \frametitle{System binarny zapisu liczb naturalnych}


  Zadajmy sobie jeszcze pytanie, ile liczb jesteśmy w~stanie zapisać
  na trzech bitach? Odpowiedź to~$8$: $0\text{b}000 == 0$,
  $0\text{b}001 == 1$, $0\text{b}010 == 2$, $0\text{b}011 == 3$,
  $0\text{b}100 == 4$, $0\text{b}101 == 5$, $0\text{b}110 == 6$
  i~$0\text{b}111 == 7$.

  Te przykłady powinny zilustrować następujący fakt. Mając dostępne
  $n$~bitów możemy za ich pomocą zapisać $2^{ \HorSpaceTwo n }$ stanów. Jest
  to fakt ogólny, którego prosto dowodzi~się matematycznie, jednak
  zajmowanie~się matematycznymi podstawami informatyki, nie należy do zakresu
  tych zajęć. Naszym celem jest pomóc wyrobić sobie Państwu odpowiednią
  intuicję, pozostawiając innym przedmiotom dokładniejsze omówienie tych
  spraw.

  Zwróćmy uwagę na jeszcze jeden fakt. W~tym systemie notacji liczb
  naturalnych, mając do dyspozycji $n$~bitów jeden stan zużywamy na zapisanie
  liczby $0 == 0\text{b}00\ldots0$, pozostałe $2^{ \HorSpaceTwo n - 1 }$ zużywamy
  na zapisanie kolejny liczb naturalnych. Stąd, w~tym konkretnym
  systemie, największą liczbą jaką możemy zapisać na $n$~bitach
  jest $2^{ \HorSpaceTwo n } - 1$.

\end{frame}
% ##################





% ##################
\begin{frame}
  \frametitle{System binarny zapisu liczb naturalnych}


  Maksymalna liczba naturalna, jaką w~prezentowanym systemie można zapisać
  na $n$~bitach to $2^{ \HorSpaceTwo n - 1 }$. Ponieważ
  $2^{ \HorSpaceOne 8 } - 1 == 256 - 1 == 255$, to wyjaśnia czemu w~wielu
  miejscach musimy podawać liczby w~zakresie od~$0$ do~$255$. Po porostu
  komputer na zapisanie tej liczby poświęca dokładnie jeden bajt ($8$~bitów).

  System który omówiliśmy nie pozwala zapisywać choć liczb całkowitych
  ujemnych, musi więc zostać odpowiednio rozszerzony. Jednak my nie będziemy
  wnikać w~tego typu zagadnienia, gdyż to co powiedzieliśmy, jest na nasze
  potrzeby wystarczające.

\end{frame}
% ##################










% ######################################
\section{Liczby i~cała reszta}
% ######################################


% ##################
\begin{frame}
  \frametitle{Liczby i~cała reszta}


  Do tej pory powiedzieliśmy sobie jak komputer może zapisywać liczby,
  ale przecież komputer może oprócz liczb zapisywać również wiele innych
  rzeczy, choć litery alfabetu, którymi ta prezentacja jest zapisana. Myśląc
  o~takich sprawach wkraczamy na temat kodowania w~komputerze.

  Zacznijmy od bardzo prostego przykładu, który może~się wydawać
  dziwny, ale jest wbrew pozorom bardzo praktyczny. Jak można zapamiętać
  angielskie słowo „Hello” w~komputerze? Możemy to zrobić w~następujący
  sposób. Wiemy, że~komputer umie zapamiętywać liczby naturalne, więc
  ponumerujmy litery alfabetu (angielskiego) kolejnymi liczbami
  w~następujący sposób. \\
  $\text{A} \to 1$, $\text{B} \to 2$, $\text{C} \to 3$, \ldots, $\text{Z} \to 26$,
  $\text{a} \to 27$, $\text{b} \to 28$, $\text{c} \to 29$, \ldots,
  $\text{z} \to 52$. \\
  Taka odpowiedniość między literami alfabetu angielskiego, a~liczbami
  naturalnymi nazywamy \textbf{kodowaniem}. Dokładniej, zakodowaniem
  znaków jako liczb naturalnych.

\end{frame}
% ##################





% ##################
\begin{frame}
  \frametitle{Kodowanie liter}


  Możemy teraz zakodować słowo „Hello” jako ciąg liczb
  $8 \; 31 \; 38 \; 38 \; 41$. Niedługo zobaczymy takie kodowanie w~praktyce.
  Teraz możemy komputerowi powiedzieć „Przełącz~się na tryb, w~którym
  każdą liczbę od $1$ do~$52$ rozumiesz jako pewną literę alfabetu
  angielskiego. Jeśli ją napotkasz, to na ekranie wyświetl odpowiedni znak.”

  To wbrew pozorom dobrze opisuje, jak komputery działają, co więcej,
  język~C działa tak cały czas. Jednak zanim powiemy sobie jak język~C
  sobie z~tym radzi, to musimy powiedzieć coś o~systemie kodowania
  \colorhref{https://en.wikipedia.org/wiki/ASCII}{\textsc{ascii}}, który
  w~$2025$ wciąż jest jednym z~fundamentów działania naszych komputerów.

  Akronim \colorhref{https://en.wikipedia.org/wiki/ASCII}{\textsc{ascii}}
  pochodzi od~\textit{American Standard Code for~Information Interchange},
  czyli \textit{Standardowy Amerykański Kod do Wymiany Informacji}.
  Jak sama nazwa wskazuje, został on stworzony przez Amerykanów na ich
  potrzeby, co wyjaśnia dlaczego opiera~się na alfabecie angielskim.
  Dlatego też niektórzy uważają, że~powinno~się go nazywać
  \textsc{us-ascii}.

\end{frame}
% ##################





% ##################
\begin{frame}
  \frametitle{Historia kodu ASCII}


  Powstanie kodu \textsc{ascii} zaczyna~się w~$1961$ roku. Należy sobie
  uświadomić jak wówczas wyglądała świat komputerów. Komputery potrafiły
  zajmować całe pokoje, internet nie był nawet marzeniem, a~przenoszenie
  danych z~jednego komputera na drugi polegało na nagraniu danych na kartki
  papieru lub dysk, przeniesienia ich do budynku, gdzie znajduje~się drugi
  komputer i~wgraniu ich do niego.

  I~tu pojawia~się problem. Co jeśli na jednym komputerze zapisywaliśmy
  literę „A” jako liczbę~$1$, ale na drugim komputerze, ktoś przypisał jej
  wartość~$33$? Wbrew pozorom takie sytuacje~się zdarzały dość często
  i~doprowadzały, jak łatwo~się domyślić, użytkowników komputerów do
  rozpaczy.

  Rozwiązanie wydawało~się dość proste. Utwórzmy standard kodowania, którym
  będziemy~się wszyscy posługiwać i~dzięki któremu na każdym komputerze
  literze~„A”, będzie odpowiadała ta sama liczba.

\end{frame}
% ##################





% ##################
\begin{frame}
  \frametitle{Historia kodu ASCII}


  Wprowadzanie tej idei w~życie rozpoczął pracujący wówczas dla firmy
  \textsc{ibm} \colorhref{https://en.wikipedia.org/wiki/Bob\_Bemer}{Bob
    Bemer}, który w~maju $1961$~roku złożył projekt takiego standardu
  do organizacji, która dziś nosi nazwę
  \colorhref{https://en.wikipedia.org/wiki/American\_National\_Standards\_Institute}
  {American National Standards Institute}, w~skrócie \textsc{ansi}.

  W~tamtych czasach pamięć była droga, każdy bit~się liczył, zaś bajt liczący
  osiem bitów nie był jeszcze powszechnie przyjętym standardem. Po ustaleniu,
  że~potrzeba im standardu dla około $100$ znaków, wzięto najmniejszą
  liczbę bitów, która zapewnia taką liczbę stanów. Liczbą tą jest~$7$,
  bo~$2^{ \HorSpaceTwo 7 } = 128$. Tym samy kodowanie \textsc{ascii} zawiera
  $128$ znaków, przypisane liczbom od~$0$ do~$127$ (proszę zauważyć,
  że~tych liczb jest dokładnie~$128$).

  Pełną tabelę znaków \textsc{ascii} można znaleźć choćby
  \colorhref{https://www.ascii-code.com/}{tutaj}. Pierwsze $32$ znaki tego
  kodu to tzw. znaki kontrolne (ang. \textit{control characters}). Nie są to
  znaki w~sensie języka codziennego, ile raczej symbole służące do kontroli
  pracy komputera.

\end{frame}
% ##################





% ##################
\begin{frame}
  \frametitle{Struktura kodu ASCII}


  Liczbie $0$ przypisany jest symbol null, służący do~informowania komputera,
  że~coś uległo zakończeniu. Opowiemy sobie o~nim więcej, gdy będziemy
  mówić o~stringach (napisach) w~języku~C. Symbol o~numerze~$2$ informuje
  komputer, że~właśnie rozpoczyna~się pewien tekst, symbol o~numerze~$3$,
  że~ten tekst~się właśnie kończy, symbol~$4$ oznacza zakończenie przesyłu
  danych,~etc. Nie muszą Państwo tego pamiętać, to jest ważne tylko by
  wyjaśnić, czemu \textsc{ascii} ma taki kształt, a~nie inny. Musiał on
  bowiem wziąć pod uwagę wiele zagadnień technicznych, przed którymi
  stanęli jego twórcy w~latach $60$-tych XX~wieku.

  Ostatecznie, literom alfabetu angielskiego przyporządkowano następujące
  liczby. $\text{A} \to 65$, $\text{B} \to 66$, $\text{C} \to 67$, \ldots,
  $\text{Z} \to 90$, $\text{a} \to 97$, $\text{b} \to 98$, $\text{c} \to 99$,
  \ldots, $\text{z} \to 122$. Oprócz tego znakowi przecinka przyporządkowano
  liczbę~$44$, kropce $46$, średnikowi $59$,~etc.

\end{frame}
% ##################






% ##################
\begin{frame}
  \frametitle{Struktura kodu ASCII}


  Ponieważ \textsc{ascii} stworzyli amerykanie, więc zawiera tylko litery
  alfabetu angielskiego. W~jaki sposób kodowane są dziś litery takie jak
  „ą”, czy „ć” to temat na zupełnie inny wykład.

  Nie będziemy od Państwa wymagać znajomości tego wszystkiego. Nasza opowieść
  o~\textsc{ascii} zawiera pewne uproszczenia, gdyż jej główny celem było
  dostarczenie Państwu informacji, pozwalające zrozumieć pewne cechy
  języka~C, które bez tego mogłyby~się wydawać, zupełnie bezsensowne.
  Teraz zaś możemy przejść bezpośrednio do kodowania w~języku~C.

\end{frame}
% ##################






% % ############################
% \jagiellonianendslide{Czy są jakieś pytania do tej części?}
% % ############################































% ####################################################################
% ####################################################################
% Bibliography

\printbibliography





% ############################
% End of the document

\end{document}
