% ------------------------------------------------------------------------------------------------------------------
% Basic configuration of Beamera class and Jagiellonian theme
% ------------------------------------------------------------------------------------------------------------------
\RequirePackage[l2tabu, orthodox]{nag}



\ifx\PresentationStyle\notset
  \def\PresentationStyle{dark}
\fi



% Options: t -- align text to the top of the frame
\documentclass[10pt,t]{beamer}
\mode<presentation>
\usetheme[style=\PresentationStyle]{jagiellonian}





% ------------------------------------------------------------------------------------
% Procesing configuration files of Jagiellonian theme located in
% the directory "preambule"
% ------------------------------------------------------------------------------------
% Configuration for polish language
% Need description
\usepackage[polish]{babel}
% Need description
\usepackage[MeX]{polski}



% ------------------------------
% Better support of polish chars in technical parts of PDF
% ------------------------------
\hypersetup{pdfencoding=auto,psdextra}

% Package "textpos" give as enviroment "textblock" which is very usefull in
% arranging text on slides.

% This is standard configuration of "textpos"
\usepackage[overlay,absolute]{textpos}

% If you need to see bounds of "textblock's" comment line above and uncomment
% one below.

% Caution! When showboxes option is on significant ammunt of space is add
% to the top of textblock and as such, everyting put in them gone down.
% We need to check how to remove this bug.

% \usepackage[showboxes,overlay,absolute]{textpos}



% Setting scale length for package "textpos"
\setlength{\TPHorizModule}{10mm}
\setlength{\TPVertModule}{\TPHorizModule}


% ---------------------------------------
% Packages written for lectures "Geometria 3D dla twórców gier wideo"
% ---------------------------------------
% \usepackage{./Geometry3DPackages/Geometry3D}
% \usepackage{./Geometry3DPackages/Geometry3DIndices}
% \usepackage{./Geometry3DPackages/Geometry3DTikZStyle}
% \usepackage{./ProgramowanieSymulacjiFizykiPaczki/ProgramowanieSymulacjiFizykiTikZStyle}
% \usepackage{./Geometry3DPackages/mathcommands}


% ---------------------------------------
% TikZ
% ---------------------------------------
% Importing TikZ libraries
\usetikzlibrary{arrows.meta}
\usetikzlibrary{positioning}





% % Configuration package "bm" that need for making bold symbols
% \newcommand{\bmmax}{0}
% \newcommand{\hmmax}{0}
% \usepackage{bm}




% ---------------------------------------
% Packages for scientific texts
% ---------------------------------------
% \let\lll\undefined  % Sometimes you must use this line to allow
% "amsmath" package to works with packages with packages for polish
% languge imported
% /preambul/LanguageSettings/JagiellonianPolishLanguageSettings.tex.
% This comments (probably) removes polish letter Ł.
\usepackage{amsmath}  % Packages from American Mathematical Society (AMS)
\usepackage{amssymb}
\usepackage{amscd}
\usepackage{amsthm}
\usepackage{siunitx}  % Package for typsetting SI units.
\usepackage{upgreek}  % Better looking greek letters.
% Example of using upgreek: pi = \uppi


\usepackage{calrsfs}  % Zmienia czcionkę kaligraficzną w \mathcal
% na ładniejszą. Może w innych miejscach robi to samo, ale o tym nic
% nie wiem.










% ---------------------------------------
% Packages written for lectures "Geometria 3D dla twórców gier wideo"
% ---------------------------------------
% \usepackage{./ProgramowanieSymulacjiFizykiPaczki/ProgramowanieSymulacjiFizyki}
% \usepackage{./ProgramowanieSymulacjiFizykiPaczki/ProgramowanieSymulacjiFizykiIndeksy}
% \usepackage{./ProgramowanieSymulacjiFizykiPaczki/ProgramowanieSymulacjiFizykiTikZStyle}





% !!!!!!!!!!!!!!!!!!!!!!!!!!!!!!
% !!!!!!!!!!!!!!!!!!!!!!!!!!!!!!
% EVIL STUFF
\if\JUlogotitle1
\edef\LogoJUPath{LogoJU_\JUlogoLang/LogoJU_\JUlogoShape_\JUlogoColor.pdf}
\titlegraphic{\hfill\includegraphics[scale=0.22]
{./JagiellonianPictures/\LogoJUPath}}
\fi
% ---------------------------------------
% Commands for handling colors
% ---------------------------------------


% Command for setting normal text color for some text in math modestyle
% Text color depend on used style of Jagiellonian

% Beamer version of command
\newcommand{\TextWithNormalTextColor}[1]{%
  {\color{jNormalTextFGColor}
    \setbeamercolor{math text}{fg=jNormalTextFGColor} {#1}}
}

% Article and similar classes version of command
% \newcommand{\TextWithNormalTextColor}[1]{%
%   {\color{jNormalTextsFGColor} {#1}}
% }



% Beamer version of command
\newcommand{\NormalTextInMathMode}[1]{%
  {\color{jNormalTextFGColor}
    \setbeamercolor{math text}{fg=jNormalTextFGColor} \text{#1}}
}


% Article and similar classes version of command
% \newcommand{\NormalTextInMathMode}[1]{%
%   {\color{jNormalTextsFGColor} \text{#1}}
% }




% Command that sets color of some mathematical text to the same color
% that has normal text in header (?)

% Beamer version of the command
\newcommand{\MathTextFrametitleFGColor}[1]{%
  {\color{jFrametitleFGColor}
    \setbeamercolor{math text}{fg=jFrametitleFGColor} #1}
}

% Article and similar classes version of the command
% \newcommand{\MathTextWhiteColor}[1]{{\color{jFrametitleFGColor} #1}}





% Command for setting color of alert text for some text in math modestyle

% Beamer version of the command
\newcommand{\MathTextAlertColor}[1]{%
  {\color{jOrange} \setbeamercolor{math text}{fg=jOrange} #1}
}

% Article and similar classes version of the command
% \newcommand{\MathTextAlertColor}[1]{{\color{jOrange} #1}}





% Command that allow you to sets chosen color as the color of some text into
% math mode. Due to some nuances in the way that Beamer handle colors
% it not work in all cases. We hope that in the future we will improve it.

% Beamer version of the command
\newcommand{\SetMathTextsColor}[2]{%
  {\color{#1} \setbeamercolor{math text}{fg=#1} #2}
}


% Article and similar classes version of the command
% \newcommand{\SetMathTextColor}[2]{{\color{#1} #2}}










% ---------------------------------------
% Commands for setting background pictures for some slides
% ---------------------------------------
\newcommand{\TitleBackgroundPicture}
{./PresentationPictures/CommonPictures/Cute_dragon_BG_dark.png}
\newcommand{\SectionBackgroundPicture}
{./PresentationPictures/CommonPictures/Cute_dragon_small_BG_light.png}



\newcommand{\TitleSlideWithPicture}{
  \begingroup

  \usebackgroundtemplate{ % \hspace*{-11.5em}
    \includegraphics[height=\paperheight]{\TitleBackgroundPicture}}

  \maketitle

  \endgroup
}





\newcommand{\SectionSlideWithPicture}[1]{%
  \begingroup

  \usebackgroundtemplate{ % \hspace*{-11.5em}
    \includegraphics[height=\paperheight]{\SectionBackgroundPicture}}

  \setbeamercolor{titlelike}{fg=normal text.fg}

  \section{#1}

  \endgroup
}





\newcommand{\EndingSlide}[1]{%
  \begin{frame}[standout]

    \begingroup

    \color{jFrametitleFGColor}

    #1

    \endgroup

  \end{frame}
}










% ------------------------------------------------------
% BibLaTeX
% ------------------------------------------------------
% Package biblatex, with biber as its backend, allow us to handle
% bibliography entries that use Unicode symbols outside ASCII.
\usepackage[
language=polish,
backend=biber,
style=alphabetic,
url=false,
eprint=true,
]{biblatex}

\addbibresource{Podstawy-informatyki-ETC-Bibliography.bib}





% ------------------------------------------------------
% Importing packages, libraries and setting their configuration
% ------------------------------------------------------
% Library improving positioning of nodes in graphs
% \usetikzlibrary{positioning}





% ------------------------------------------------------
% Local packages
% ------------------------------------------------------
% Local configuration of this particular presentation
\usepackage{./Local-packages/local-settings}

% % Styles for arrows
% \usepackage{./Local-packages/PGF-TikZ-Arrows-styles}

% % Styles for drawing diagrams
% \usepackage{./Local-packages/PGF-TikZ-Diagram-styles}

% % Jagiellonian theme's colors
% \usepackage{./Local-packages/jagiellonian-theme-colors}










% ------------------------------------------------------------------------------------------------------------------
\title{Podstawy informatyki z~językiem~C}
\subtitle{Programowanie nie jest trudne, debugowanie już tak}

\author{Kamil Ziemian \\
  \email}


% \date{}
% ------------------------------------------------------------------------------------------------------------------










% ####################################################################
% Beginning of the document
\begin{document}
% ####################################################################





% ######################################
% Number of chars: 7k+,
% Text is adjusted to the left and words are broken at the end of the line.
\RaggedRight
% ######################################





% ######################################
\maketitle
% ######################################





% ##################
\begin{frame}
  \frametitle{Spis treści}


  \tableofcontents

\end{frame}
% ##################










% ######################################
\section{Programowanie nie jest trudne. Debugowanie już tak}
% ######################################


% ##################
\begin{frame}
  \frametitle{Programowanie nie jest trudne. Debugowanie już
    tak}


  \textit{Debugowanie programu jest dwa razy trudniejsze, niż jego
    napisanie. Jeśli więc napiszesz program w~najbardziej inteligentny
    sposób jaki możesz, to nie jesteś wystarczająco mądry by go
    zdebugować.} \\
  Prawo
  \colorhref{https://en.wikipedia.org/wiki/Brian\_Kernighan}{Kernighana}.

  Proszę pamiętać, że~pisanie programów nie jest trudne, ich debugowanie już
  tak. Pomimo tego postaram~się postawić wspólnie kilka kroków na trudnej
  ścieżce nauki debugowania programów.

  Przypomnijmy jak działa język~C. Piszemy program z~kodem źródłowym
  w~języku~C, który ma być zawarty w~pliku, którego nazwa \alert{kończy~się
    na~\texttt{.c}}. Następnie program zwany kompilatorem na podstawie
  tego kodu źródłowego generuje program wykonywalny \texttt{prog.out}.
  Kompilator to też program komputerowy, więc to my, ludzie, ustalamy
  jego zasady pracy.

\end{frame}
% ##################





% ##################
\begin{frame}
  \frametitle{Błędy}


  Twórcy języka~C przyjęli, że~kompilator może zwrócić nam dwa rodzaje
  wiadomości, że~coś jest nie tak: \textbf{błędy} (ang.~\textit{errors})
  i~\textbf{ostrzeżenia} (ang.~\textit{warnings}). Różnica między nimi
  jest następująca.

  Zwrócenie błędu przez kompilator można zrozumieć, jako następująca
  informację od niego. „Człowieku, ten kod który napisałeś w~linii
  numer X (mniej więcej) jest tak zły, że~ja nie wygenerują ci pliku
  wykonywalnego. Napraw ten błąd i~spróbuj jeszcze raz.”

  Ze względu na to jak skomplikowany jest proces kompilacji oraz wieku
  języka~C, informacje o~błędach często są bardzo trudne w~zrozumieniu
  i~niezbyt precyzyjne. To naprawdę skomplikowany temat. Jedną z~rzeczy,
  które należy rozumieć jest to, że~jeśli kompilator twierdzi, że~błąd
  jest w~linii $20$, to on zwykle w~tej linii jest, ale kompilator
  może~się mylić i~błąd znajduje~się wcześniej albo później. Bo dlaczego
  życie miałoby być proste?

\end{frame}
% ##################





% ##################
\begin{frame}
  \frametitle{Ostrzeżenia}


  Jeśli kompilator zwróci ostrzeżenie, to można to rozumieć jak mówił:
  „Kod w~tym miejscu wygląda na potencjalnie wadliwy, ale skoro tak chcesz,
  to wygeneruję program zgodnie z~nim. Jeśli samolot spadnie przez to
  z~nieba, to ja nie biorę za to odpowiedzialności.”

  Warto dodać, że~twórcy języka Go, stwierdzili, że~kompilator nie
  powinien zwracać ostrzeżeń, bo dzieje~się dokładnie to czego można~się
  byłoby spodziewać. Programiści ignorują ostrzeżenia, bo czy program został
  wygenerowany? Tak. To czym~się przejmować? Skutkiem tego istniejący
  kod jest pełen podejrzanych miejsc, których programistą nie chciało~się
  zwyczajnie poprawić. Z~tego powodu w~języku Go są tylko błędy, więc
  programista musi je wszystkie poprawić.

  Na tym przedmiocie obowiązuje zasada, że~każdy program ma zostać
  doprowadzony do stanu, w~którym kompilator podczas jego przetwarzania
  nie zgłasza żadnego ostrzeżenia. Państwo rozumieją, dlaczego takie zasady
  przyjęliśmy.

\end{frame}
% ##################





% ##################
\begin{frame}
  \frametitle{Plugawe nazwy plików}


  W~nazwie pliku powinna być \alert{jedna i~tylko jedna kropka}, ta
  poprzedzająca nazwę rozszerzenia pliku, czyli dla nas \texttt{.c}.
  Jeśli ta zasada nie jest prawdą wszędzie na świecie, to na pewno jest
  prawdą na tych zajęciach i~proszę o~tym pamiętać.

  Jeśli ktoś prześle mi plik, którego nazwa zawiera więcej lub mniej niż
  \alert{jedną} kropkę, to zastrzegam sobie prawo odesłania mu pliku
  z~prośbą o~zmianę nazwy na~poprawniejszą. Dla mnie to nie jest sposób
  karania, tylko metoda wyrabiania w~ludziach dobrych nawyków.

\end{frame}
% ##################










% % ############################
% \jagiellonianendslide{Czy są jakieś pytania do tej części?}
% % ############################































% ####################################################################
% ####################################################################
% Bibliography

\printbibliography





% ############################
% End of the document

\end{document}
