% ------------------------------------------------------------------------------------------------------------------
% Basic configuration of Beamera class and Jagiellonian theme
% ------------------------------------------------------------------------------------------------------------------
\RequirePackage[l2tabu, orthodox]{nag}



\ifx\PresentationStyle\notset
  \def\PresentationStyle{dark}
\fi



% Options: t -- align text to the top of the frame
\documentclass[10pt,t]{beamer}
\mode<presentation>
\usetheme[style=\PresentationStyle]{jagiellonian}





% ------------------------------------------------------------------------------------
% Procesing configuration files of Jagiellonian theme located
% in the directory "preambule"
% ------------------------------------------------------------------------------------
% Configuration for polish language
% Need description
\usepackage[polish]{babel}
% Need description
\usepackage[MeX]{polski}



% ------------------------------
% Better support of polish chars in technical parts of PDF
% ------------------------------
\hypersetup{pdfencoding=auto,psdextra}

% Package "textpos" give as enviroment "textblock" which is very usefull in
% arranging text on slides.

% This is standard configuration of "textpos"
\usepackage[overlay,absolute]{textpos}

% If you need to see bounds of "textblock's" comment line above and uncomment
% one below.

% Caution! When showboxes option is on significant ammunt of space is add
% to the top of textblock and as such, everyting put in them gone down.
% We need to check how to remove this bug.

% \usepackage[showboxes,overlay,absolute]{textpos}



% Setting scale length for package "textpos"
\setlength{\TPHorizModule}{10mm}
\setlength{\TPVertModule}{\TPHorizModule}


% ---------------------------------------
% Packages written for lectures "Geometria 3D dla twórców gier wideo"
% ---------------------------------------
% \usepackage{./Geometry3DPackages/Geometry3D}
% \usepackage{./Geometry3DPackages/Geometry3DIndices}
% \usepackage{./Geometry3DPackages/Geometry3DTikZStyle}
% \usepackage{./ProgramowanieSymulacjiFizykiPaczki/ProgramowanieSymulacjiFizykiTikZStyle}
% \usepackage{./Geometry3DPackages/mathcommands}


% ---------------------------------------
% TikZ
% ---------------------------------------
% Importing TikZ libraries
\usetikzlibrary{arrows.meta}
\usetikzlibrary{positioning}





% % Configuration package "bm" that need for making bold symbols
% \newcommand{\bmmax}{0}
% \newcommand{\hmmax}{0}
% \usepackage{bm}




% ---------------------------------------
% Packages for scientific texts
% ---------------------------------------
% \let\lll\undefined  % Sometimes you must use this line to allow
% "amsmath" package to works with packages with packages for polish
% languge imported
% /preambul/LanguageSettings/JagiellonianPolishLanguageSettings.tex.
% This comments (probably) removes polish letter Ł.
\usepackage{amsmath}  % Packages from American Mathematical Society (AMS)
\usepackage{amssymb}
\usepackage{amscd}
\usepackage{amsthm}
\usepackage{siunitx}  % Package for typsetting SI units.
\usepackage{upgreek}  % Better looking greek letters.
% Example of using upgreek: pi = \uppi


\usepackage{calrsfs}  % Zmienia czcionkę kaligraficzną w \mathcal
% na ładniejszą. Może w innych miejscach robi to samo, ale o tym nic
% nie wiem.










% ---------------------------------------
% Packages written for lectures "Geometria 3D dla twórców gier wideo"
% ---------------------------------------
% \usepackage{./ProgramowanieSymulacjiFizykiPaczki/ProgramowanieSymulacjiFizyki}
% \usepackage{./ProgramowanieSymulacjiFizykiPaczki/ProgramowanieSymulacjiFizykiIndeksy}
% \usepackage{./ProgramowanieSymulacjiFizykiPaczki/ProgramowanieSymulacjiFizykiTikZStyle}





% !!!!!!!!!!!!!!!!!!!!!!!!!!!!!!
% !!!!!!!!!!!!!!!!!!!!!!!!!!!!!!
% EVIL STUFF
\if\JUlogotitle1
\edef\LogoJUPath{LogoJU_\JUlogoLang/LogoJU_\JUlogoShape_\JUlogoColor.pdf}
\titlegraphic{\hfill\includegraphics[scale=0.22]
{./JagiellonianPictures/\LogoJUPath}}
\fi
% ---------------------------------------
% Commands for handling colors
% ---------------------------------------


% Command for setting normal text color for some text in math modestyle
% Text color depend on used style of Jagiellonian

% Beamer version of command
\newcommand{\TextWithNormalTextColor}[1]{%
  {\color{jNormalTextFGColor}
    \setbeamercolor{math text}{fg=jNormalTextFGColor} {#1}}
}

% Article and similar classes version of command
% \newcommand{\TextWithNormalTextColor}[1]{%
%   {\color{jNormalTextsFGColor} {#1}}
% }



% Beamer version of command
\newcommand{\NormalTextInMathMode}[1]{%
  {\color{jNormalTextFGColor}
    \setbeamercolor{math text}{fg=jNormalTextFGColor} \text{#1}}
}


% Article and similar classes version of command
% \newcommand{\NormalTextInMathMode}[1]{%
%   {\color{jNormalTextsFGColor} \text{#1}}
% }




% Command that sets color of some mathematical text to the same color
% that has normal text in header (?)

% Beamer version of the command
\newcommand{\MathTextFrametitleFGColor}[1]{%
  {\color{jFrametitleFGColor}
    \setbeamercolor{math text}{fg=jFrametitleFGColor} #1}
}

% Article and similar classes version of the command
% \newcommand{\MathTextWhiteColor}[1]{{\color{jFrametitleFGColor} #1}}





% Command for setting color of alert text for some text in math modestyle

% Beamer version of the command
\newcommand{\MathTextAlertColor}[1]{%
  {\color{jOrange} \setbeamercolor{math text}{fg=jOrange} #1}
}

% Article and similar classes version of the command
% \newcommand{\MathTextAlertColor}[1]{{\color{jOrange} #1}}





% Command that allow you to sets chosen color as the color of some text into
% math mode. Due to some nuances in the way that Beamer handle colors
% it not work in all cases. We hope that in the future we will improve it.

% Beamer version of the command
\newcommand{\SetMathTextColor}[2]{%
  {\color{#1} \setbeamercolor{math text}{fg=#1} #2}
}


% Article and similar classes version of the command
% \newcommand{\SetMathTextColor}[2]{{\color{#1} #2}}










% ---------------------------------------
% Commands for few special slides
% ---------------------------------------
\newcommand{\EndingSlide}[1]{%
  \begin{frame}[standout]

    \begingroup

    \color{jFrametitleFGColor}

    #1

    \endgroup

  \end{frame}
}










% ---------------------------------------
% Commands for setting background pictures for some slides
% ---------------------------------------
\newcommand{\TitleBackgroundPicture}
{./JagiellonianPictures/Backgrounds/LajkonikDark.png}
\newcommand{\SectionBackgroundPicture}
{./JagiellonianPictures/Backgrounds/LajkonikLight.png}



\newcommand{\TitleSlideWithPicture}{%
  \begingroup

  \usebackgroundtemplate{%
    \includegraphics[height=\paperheight]{\TitleBackgroundPicture}}

  \maketitle

  \endgroup
}





\newcommand{\SectionSlideWithPicture}[1]{%
  \begingroup

  \usebackgroundtemplate{%
    \includegraphics[height=\paperheight]{\SectionBackgroundPicture}}

  \setbeamercolor{titlelike}{fg=normal text.fg}

  \section{#1}

  \endgroup
}










% ---------------------------------------
% Commands for lectures "Geometria 3D dla twórców gier wideo"
% Polish version
% ---------------------------------------
% Komendy teraz wykomentowane były potrzebne, gdy loga były na niebieskim
% tle, nie na białym. A są na białym bo tego chcieli w biurze projektu.
% \newcommand{\FundingLogoWhitePicturePL}
% {./PresentationPictures/CommonPictures/logotypFundusze_biale_bez_tla2.pdf}
\newcommand{\FundingLogoColorPicturePL}
{./PresentationPictures/CommonPictures/European_Funds_color_PL.pdf}
% \newcommand{\EULogoWhitePicturePL}
% {./PresentationPictures/CommonPictures/logotypUE_biale_bez_tla2.pdf}
\newcommand{\EUSocialFundLogoColorPicturePL}
{./PresentationPictures/CommonPictures/EU_Social_Fund_color_PL.pdf}
% \newcommand{\ZintegrUJLogoWhitePicturePL}
% {./PresentationPictures/CommonPictures/zintegruj-logo-white.pdf}
\newcommand{\ZintegrUJLogoColorPicturePL}
{./PresentationPictures/CommonPictures/ZintegrUJ_color.pdf}
\newcommand{\JULogoColorPicturePL}
{./JagiellonianPictures/LogoJU_PL/LogoJU_A_color.pdf}





\newcommand{\GeometryThreeDSpecialBeginningSlidePL}{%
  \begin{frame}[standout]

    \begin{textblock}{11}(1,0.7)

      \begin{flushleft}

        \mdseries

        \footnotesize

        \color{jFrametitleFGColor}

        Materiał powstał w ramach projektu współfinansowanego ze środków
        Unii Europejskiej w ramach Europejskiego Funduszu Społecznego
        POWR.03.05.00-00-Z309/17-00.

      \end{flushleft}

    \end{textblock}





    \begin{textblock}{10}(0,2.2)

      \tikz \fill[color=jBackgroundStyleLight] (0,0) rectangle (12.8,-1.5);

    \end{textblock}


    \begin{textblock}{3.2}(1,2.45)

      \includegraphics[scale=0.3]{\FundingLogoColorPicturePL}

    \end{textblock}


    \begin{textblock}{2.5}(3.7,2.5)

      \includegraphics[scale=0.2]{\JULogoColorPicturePL}

    \end{textblock}


    \begin{textblock}{2.5}(6,2.4)

      \includegraphics[scale=0.1]{\ZintegrUJLogoColorPicturePL}

    \end{textblock}


    \begin{textblock}{4.2}(8.4,2.6)

      \includegraphics[scale=0.3]{\EUSocialFundLogoColorPicturePL}

    \end{textblock}

  \end{frame}
}



\newcommand{\GeometryThreeDTwoSpecialBeginningSlidesPL}{%
  \begin{frame}[standout]

    \begin{textblock}{11}(1,0.7)

      \begin{flushleft}

        \mdseries

        \footnotesize

        \color{jFrametitleFGColor}

        Materiał powstał w ramach projektu współfinansowanego ze środków
        Unii Europejskiej w ramach Europejskiego Funduszu Społecznego
        POWR.03.05.00-00-Z309/17-00.

      \end{flushleft}

    \end{textblock}





    \begin{textblock}{10}(0,2.2)

      \tikz \fill[color=jBackgroundStyleLight] (0,0) rectangle (12.8,-1.5);

    \end{textblock}


    \begin{textblock}{3.2}(1,2.45)

      \includegraphics[scale=0.3]{\FundingLogoColorPicturePL}

    \end{textblock}


    \begin{textblock}{2.5}(3.7,2.5)

      \includegraphics[scale=0.2]{\JULogoColorPicturePL}

    \end{textblock}


    \begin{textblock}{2.5}(6,2.4)

      \includegraphics[scale=0.1]{\ZintegrUJLogoColorPicturePL}

    \end{textblock}


    \begin{textblock}{4.2}(8.4,2.6)

      \includegraphics[scale=0.3]{\EUSocialFundLogoColorPicturePL}

    \end{textblock}

  \end{frame}





  \TitleSlideWithPicture
}



\newcommand{\GeometryThreeDSpecialEndingSlidePL}{%
  \begin{frame}[standout]

    \begin{textblock}{11}(1,0.7)

      \begin{flushleft}

        \mdseries

        \footnotesize

        \color{jFrametitleFGColor}

        Materiał powstał w ramach projektu współfinansowanego ze środków
        Unii Europejskiej w~ramach Europejskiego Funduszu Społecznego
        POWR.03.05.00-00-Z309/17-00.

      \end{flushleft}

    \end{textblock}





    \begin{textblock}{10}(0,2.2)

      \tikz \fill[color=jBackgroundStyleLight] (0,0) rectangle (12.8,-1.5);

    \end{textblock}


    \begin{textblock}{3.2}(1,2.45)

      \includegraphics[scale=0.3]{\FundingLogoColorPicturePL}

    \end{textblock}


    \begin{textblock}{2.5}(3.7,2.5)

      \includegraphics[scale=0.2]{\JULogoColorPicturePL}

    \end{textblock}


    \begin{textblock}{2.5}(6,2.4)

      \includegraphics[scale=0.1]{\ZintegrUJLogoColorPicturePL}

    \end{textblock}


    \begin{textblock}{4.2}(8.4,2.6)

      \includegraphics[scale=0.3]{\EUSocialFundLogoColorPicturePL}

    \end{textblock}





    \begin{textblock}{11}(1,4)

      \begin{flushleft}

        \mdseries

        \footnotesize

        \RaggedRight

        \color{jFrametitleFGColor}

        Treść niniejszego wykładu jest udostępniona na~licencji
        Creative Commons (\textsc{cc}), z~uzna\-niem autorstwa
        (\textsc{by}) oraz udostępnianiem na tych samych warunkach
        (\textsc{sa}). Rysunki i~wy\-kresy zawarte w~wykładzie są
        autorstwa dr.~hab.~Pawła Węgrzyna et~al. i~są dostępne
        na tej samej licencji, o~ile nie wskazano inaczej.
        W~prezentacji wykorzystano temat Beamera Jagiellonian,
        oparty na~temacie Metropolis Matthiasa Vogelgesanga,
        dostępnym na licencji \LaTeX{} Project Public License~1.3c
        pod adresem: \colorhref{https://github.com/matze/mtheme}
        {https://github.com/matze/mtheme}.

        Projekt typograficzny: Iwona Grabska-Gradzińska \\
        Skład: Kamil Ziemian;
        Korekta: Wojciech Palacz \\
        Modele: Dariusz Frymus, Kamil Nowakowski \\
        Rysunki i~wykresy: Kamil Ziemian, Paweł Węgrzyn, Wojciech Palacz

      \end{flushleft}

    \end{textblock}

  \end{frame}
}



\newcommand{\GeometryThreeDTwoSpecialEndingSlidesPL}[1]{%
  \begin{frame}[standout]


    \begin{textblock}{11}(1,0.7)

      \begin{flushleft}

        \mdseries

        \footnotesize

        \color{jFrametitleFGColor}

        Materiał powstał w ramach projektu współfinansowanego ze środków
        Unii Europejskiej w~ramach Europejskiego Funduszu Społecznego
        POWR.03.05.00-00-Z309/17-00.

      \end{flushleft}

    \end{textblock}





    \begin{textblock}{10}(0,2.2)

      \tikz \fill[color=jBackgroundStyleLight] (0,0) rectangle (12.8,-1.5);

    \end{textblock}


    \begin{textblock}{3.2}(1,2.45)

      \includegraphics[scale=0.3]{\FundingLogoColorPicturePL}

    \end{textblock}


    \begin{textblock}{2.5}(3.7,2.5)

      \includegraphics[scale=0.2]{\JULogoColorPicturePL}

    \end{textblock}


    \begin{textblock}{2.5}(6,2.4)

      \includegraphics[scale=0.1]{\ZintegrUJLogoColorPicturePL}

    \end{textblock}


    \begin{textblock}{4.2}(8.4,2.6)

      \includegraphics[scale=0.3]{\EUSocialFundLogoColorPicturePL}

    \end{textblock}





    \begin{textblock}{11}(1,4)

      \begin{flushleft}

        \mdseries

        \footnotesize

        \RaggedRight

        \color{jFrametitleFGColor}

        Treść niniejszego wykładu jest udostępniona na~licencji
        Creative Commons (\textsc{cc}), z~uzna\-niem autorstwa
        (\textsc{by}) oraz udostępnianiem na tych samych warunkach
        (\textsc{sa}). Rysunki i~wy\-kresy zawarte w~wykładzie są
        autorstwa dr.~hab.~Pawła Węgrzyna et~al. i~są dostępne
        na tej samej licencji, o~ile nie wskazano inaczej.
        W~prezentacji wykorzystano temat Beamera Jagiellonian,
        oparty na~temacie Metropolis Matthiasa Vogelgesanga,
        dostępnym na licencji \LaTeX{} Project Public License~1.3c
        pod adresem: \colorhref{https://github.com/matze/mtheme}
        {https://github.com/matze/mtheme}.

        Projekt typograficzny: Iwona Grabska-Gradzińska \\
        Skład: Kamil Ziemian;
        Korekta: Wojciech Palacz \\
        Modele: Dariusz Frymus, Kamil Nowakowski \\
        Rysunki i~wykresy: Kamil Ziemian, Paweł Węgrzyn, Wojciech Palacz

      \end{flushleft}

    \end{textblock}

  \end{frame}





  \begin{frame}[standout]

    \begingroup

    \color{jFrametitleFGColor}

    #1

    \endgroup

  \end{frame}
}



\newcommand{\GeometryThreeDSpecialEndingSlideVideoPL}{%
  \begin{frame}[standout]

    \begin{textblock}{11}(1,0.7)

      \begin{flushleft}

        \mdseries

        \footnotesize

        \color{jFrametitleFGColor}

        Materiał powstał w ramach projektu współfinansowanego ze środków
        Unii Europejskiej w~ramach Europejskiego Funduszu Społecznego
        POWR.03.05.00-00-Z309/17-00.

      \end{flushleft}

    \end{textblock}





    \begin{textblock}{10}(0,2.2)

      \tikz \fill[color=jBackgroundStyleLight] (0,0) rectangle (12.8,-1.5);

    \end{textblock}


    \begin{textblock}{3.2}(1,2.45)

      \includegraphics[scale=0.3]{\FundingLogoColorPicturePL}

    \end{textblock}


    \begin{textblock}{2.5}(3.7,2.5)

      \includegraphics[scale=0.2]{\JULogoColorPicturePL}

    \end{textblock}


    \begin{textblock}{2.5}(6,2.4)

      \includegraphics[scale=0.1]{\ZintegrUJLogoColorPicturePL}

    \end{textblock}


    \begin{textblock}{4.2}(8.4,2.6)

      \includegraphics[scale=0.3]{\EUSocialFundLogoColorPicturePL}

    \end{textblock}





    \begin{textblock}{11}(1,4)

      \begin{flushleft}

        \mdseries

        \footnotesize

        \RaggedRight

        \color{jFrametitleFGColor}

        Treść niniejszego wykładu jest udostępniona na~licencji
        Creative Commons (\textsc{cc}), z~uzna\-niem autorstwa
        (\textsc{by}) oraz udostępnianiem na tych samych warunkach
        (\textsc{sa}). Rysunki i~wy\-kresy zawarte w~wykładzie są
        autorstwa dr.~hab.~Pawła Węgrzyna et~al. i~są dostępne
        na tej samej licencji, o~ile nie wskazano inaczej.
        W~prezentacji wykorzystano temat Beamera Jagiellonian,
        oparty na~temacie Metropolis Matthiasa Vogelgesanga,
        dostępnym na licencji \LaTeX{} Project Public License~1.3c
        pod adresem: \colorhref{https://github.com/matze/mtheme}
        {https://github.com/matze/mtheme}.

        Projekt typograficzny: Iwona Grabska-Gradzińska;
        Skład: Kamil Ziemian \\
        Korekta: Wojciech Palacz;
        Modele: Dariusz Frymus, Kamil Nowakowski \\
        Rysunki i~wykresy: Kamil Ziemian, Paweł Węgrzyn, Wojciech Palacz \\
        Montaż: Agencja Filmowa Film \& Television Production~-- Zbigniew
        Masklak

      \end{flushleft}

    \end{textblock}

  \end{frame}
}





\newcommand{\GeometryThreeDTwoSpecialEndingSlidesVideoPL}[1]{%
  \begin{frame}[standout]

    \begin{textblock}{11}(1,0.7)

      \begin{flushleft}

        \mdseries

        \footnotesize

        \color{jFrametitleFGColor}

        Materiał powstał w ramach projektu współfinansowanego ze środków
        Unii Europejskiej w~ramach Europejskiego Funduszu Społecznego
        POWR.03.05.00-00-Z309/17-00.

      \end{flushleft}

    \end{textblock}





    \begin{textblock}{10}(0,2.2)

      \tikz \fill[color=jBackgroundStyleLight] (0,0) rectangle (12.8,-1.5);

    \end{textblock}


    \begin{textblock}{3.2}(1,2.45)

      \includegraphics[scale=0.3]{\FundingLogoColorPicturePL}

    \end{textblock}


    \begin{textblock}{2.5}(3.7,2.5)

      \includegraphics[scale=0.2]{\JULogoColorPicturePL}

    \end{textblock}


    \begin{textblock}{2.5}(6,2.4)

      \includegraphics[scale=0.1]{\ZintegrUJLogoColorPicturePL}

    \end{textblock}


    \begin{textblock}{4.2}(8.4,2.6)

      \includegraphics[scale=0.3]{\EUSocialFundLogoColorPicturePL}

    \end{textblock}





    \begin{textblock}{11}(1,4)

      \begin{flushleft}

        \mdseries

        \footnotesize

        \RaggedRight

        \color{jFrametitleFGColor}

        Treść niniejszego wykładu jest udostępniona na~licencji
        Creative Commons (\textsc{cc}), z~uzna\-niem autorstwa
        (\textsc{by}) oraz udostępnianiem na tych samych warunkach
        (\textsc{sa}). Rysunki i~wy\-kresy zawarte w~wykładzie są
        autorstwa dr.~hab.~Pawła Węgrzyna et~al. i~są dostępne
        na tej samej licencji, o~ile nie wskazano inaczej.
        W~prezentacji wykorzystano temat Beamera Jagiellonian,
        oparty na~temacie Metropolis Matthiasa Vogelgesanga,
        dostępnym na licencji \LaTeX{} Project Public License~1.3c
        pod adresem: \colorhref{https://github.com/matze/mtheme}
        {https://github.com/matze/mtheme}.

        Projekt typograficzny: Iwona Grabska-Gradzińska;
        Skład: Kamil Ziemian \\
        Korekta: Wojciech Palacz;
        Modele: Dariusz Frymus, Kamil Nowakowski \\
        Rysunki i~wykresy: Kamil Ziemian, Paweł Węgrzyn, Wojciech Palacz \\
        Montaż: Agencja Filmowa Film \& Television Production~-- Zbigniew
        Masklak

      \end{flushleft}

    \end{textblock}

  \end{frame}





  \begin{frame}[standout]


    \begingroup

    \color{jFrametitleFGColor}

    #1

    \endgroup

  \end{frame}
}










% ---------------------------------------
% Commands for lectures "Geometria 3D dla twórców gier wideo"
% English version
% ---------------------------------------
% \newcommand{\FundingLogoWhitePictureEN}
% {./PresentationPictures/CommonPictures/logotypFundusze_biale_bez_tla2.pdf}
\newcommand{\FundingLogoColorPictureEN}
{./PresentationPictures/CommonPictures/European_Funds_color_EN.pdf}
% \newcommand{\EULogoWhitePictureEN}
% {./PresentationPictures/CommonPictures/logotypUE_biale_bez_tla2.pdf}
\newcommand{\EUSocialFundLogoColorPictureEN}
{./PresentationPictures/CommonPictures/EU_Social_Fund_color_EN.pdf}
% \newcommand{\ZintegrUJLogoWhitePictureEN}
% {./PresentationPictures/CommonPictures/zintegruj-logo-white.pdf}
\newcommand{\ZintegrUJLogoColorPictureEN}
{./PresentationPictures/CommonPictures/ZintegrUJ_color.pdf}
\newcommand{\JULogoColorPictureEN}
{./JagiellonianPictures/LogoJU_EN/LogoJU_A_color.pdf}



\newcommand{\GeometryThreeDSpecialBeginningSlideEN}{%
  \begin{frame}[standout]

    \begin{textblock}{11}(1,0.7)

      \begin{flushleft}

        \mdseries

        \footnotesize

        \color{jFrametitleFGColor}

        This content was created as part of a project co-financed by the
        European Union within the framework of the European Social Fund
        POWR.03.05.00-00-Z309/17-00.

      \end{flushleft}

    \end{textblock}





    \begin{textblock}{10}(0,2.2)

      \tikz \fill[color=jBackgroundStyleLight] (0,0) rectangle (12.8,-1.5);

    \end{textblock}


    \begin{textblock}{3.2}(0.7,2.45)

      \includegraphics[scale=0.3]{\FundingLogoColorPictureEN}

    \end{textblock}


    \begin{textblock}{2.5}(4.15,2.5)

      \includegraphics[scale=0.2]{\JULogoColorPictureEN}

    \end{textblock}


    \begin{textblock}{2.5}(6.35,2.4)

      \includegraphics[scale=0.1]{\ZintegrUJLogoColorPictureEN}

    \end{textblock}


    \begin{textblock}{4.2}(8.4,2.6)

      \includegraphics[scale=0.3]{\EUSocialFundLogoColorPictureEN}

    \end{textblock}

  \end{frame}
}



\newcommand{\GeometryThreeDTwoSpecialBeginningSlidesEN}{%
  \begin{frame}[standout]

    \begin{textblock}{11}(1,0.7)

      \begin{flushleft}

        \mdseries

        \footnotesize

        \color{jFrametitleFGColor}

        This content was created as part of a project co-financed by the
        European Union within the framework of the European Social Fund
        POWR.03.05.00-00-Z309/17-00.

      \end{flushleft}

    \end{textblock}





    \begin{textblock}{10}(0,2.2)

      \tikz \fill[color=jBackgroundStyleLight] (0,0) rectangle (12.8,-1.5);

    \end{textblock}


    \begin{textblock}{3.2}(0.7,2.45)

      \includegraphics[scale=0.3]{\FundingLogoColorPictureEN}

    \end{textblock}


    \begin{textblock}{2.5}(4.15,2.5)

      \includegraphics[scale=0.2]{\JULogoColorPictureEN}

    \end{textblock}


    \begin{textblock}{2.5}(6.35,2.4)

      \includegraphics[scale=0.1]{\ZintegrUJLogoColorPictureEN}

    \end{textblock}


    \begin{textblock}{4.2}(8.4,2.6)

      \includegraphics[scale=0.3]{\EUSocialFundLogoColorPictureEN}

    \end{textblock}

  \end{frame}





  \TitleSlideWithPicture
}



\newcommand{\GeometryThreeDSpecialEndingSlideEN}{%
  \begin{frame}[standout]

    \begin{textblock}{11}(1,0.7)

      \begin{flushleft}

        \mdseries

        \footnotesize

        \color{jFrametitleFGColor}

        This content was created as part of a project co-financed by the
        European Union within the framework of the European Social Fund
        POWR.03.05.00-00-Z309/17-00.

      \end{flushleft}

    \end{textblock}





    \begin{textblock}{10}(0,2.2)

      \tikz \fill[color=jBackgroundStyleLight] (0,0) rectangle (12.8,-1.5);

    \end{textblock}


    \begin{textblock}{3.2}(0.7,2.45)

      \includegraphics[scale=0.3]{\FundingLogoColorPictureEN}

    \end{textblock}


    \begin{textblock}{2.5}(4.15,2.5)

      \includegraphics[scale=0.2]{\JULogoColorPictureEN}

    \end{textblock}


    \begin{textblock}{2.5}(6.35,2.4)

      \includegraphics[scale=0.1]{\ZintegrUJLogoColorPictureEN}

    \end{textblock}


    \begin{textblock}{4.2}(8.4,2.6)

      \includegraphics[scale=0.3]{\EUSocialFundLogoColorPictureEN}

    \end{textblock}





    \begin{textblock}{11}(1,4)

      \begin{flushleft}

        \mdseries

        \footnotesize

        \RaggedRight

        \color{jFrametitleFGColor}

        The content of this lecture is made available under a~Creative
        Commons licence (\textsc{cc}), giving the author the credits
        (\textsc{by}) and putting an obligation to share on the same terms
        (\textsc{sa}). Figures and diagrams included in the lecture are
        authored by Paweł Węgrzyn et~al., and are available under the same
        license unless indicated otherwise.\\ The presentation uses the
        Beamer Jagiellonian theme based on Matthias Vogelgesang’s
        Metropolis theme, available under license \LaTeX{} Project
        Public License~1.3c at: \colorhref{https://github.com/matze/mtheme}
        {https://github.com/matze/mtheme}.

        Typographic design: Iwona Grabska-Gradzińska \\
        \LaTeX{} Typesetting: Kamil Ziemian \\
        Proofreading: Wojciech Palacz,
        Monika Stawicka \\
        3D Models: Dariusz Frymus, Kamil Nowakowski \\
        Figures and charts: Kamil Ziemian, Paweł Węgrzyn, Wojciech Palacz

      \end{flushleft}

    \end{textblock}

  \end{frame}
}



\newcommand{\GeometryThreeDTwoSpecialEndingSlidesEN}[1]{%
  \begin{frame}[standout]


    \begin{textblock}{11}(1,0.7)

      \begin{flushleft}

        \mdseries

        \footnotesize

        \color{jFrametitleFGColor}

        This content was created as part of a project co-financed by the
        European Union within the framework of the European Social Fund
        POWR.03.05.00-00-Z309/17-00.

      \end{flushleft}

    \end{textblock}





    \begin{textblock}{10}(0,2.2)

      \tikz \fill[color=jBackgroundStyleLight] (0,0) rectangle (12.8,-1.5);

    \end{textblock}


    \begin{textblock}{3.2}(0.7,2.45)

      \includegraphics[scale=0.3]{\FundingLogoColorPictureEN}

    \end{textblock}


    \begin{textblock}{2.5}(4.15,2.5)

      \includegraphics[scale=0.2]{\JULogoColorPictureEN}

    \end{textblock}


    \begin{textblock}{2.5}(6.35,2.4)

      \includegraphics[scale=0.1]{\ZintegrUJLogoColorPictureEN}

    \end{textblock}


    \begin{textblock}{4.2}(8.4,2.6)

      \includegraphics[scale=0.3]{\EUSocialFundLogoColorPictureEN}

    \end{textblock}





    \begin{textblock}{11}(1,4)

      \begin{flushleft}

        \mdseries

        \footnotesize

        \RaggedRight

        \color{jFrametitleFGColor}

        The content of this lecture is made available under a~Creative
        Commons licence (\textsc{cc}), giving the author the credits
        (\textsc{by}) and putting an obligation to share on the same terms
        (\textsc{sa}). Figures and diagrams included in the lecture are
        authored by Paweł Węgrzyn et~al., and are available under the same
        license unless indicated otherwise.\\ The presentation uses the
        Beamer Jagiellonian theme based on Matthias Vogelgesang’s
        Metropolis theme, available under license \LaTeX{} Project
        Public License~1.3c at: \colorhref{https://github.com/matze/mtheme}
        {https://github.com/matze/mtheme}.

        Typographic design: Iwona Grabska-Gradzińska \\
        \LaTeX{} Typesetting: Kamil Ziemian \\
        Proofreading: Wojciech Palacz,
        Monika Stawicka \\
        3D Models: Dariusz Frymus, Kamil Nowakowski \\
        Figures and charts: Kamil Ziemian, Paweł Węgrzyn, Wojciech Palacz

      \end{flushleft}

    \end{textblock}

  \end{frame}





  \begin{frame}[standout]

    \begingroup

    \color{jFrametitleFGColor}

    #1

    \endgroup

  \end{frame}
}



\newcommand{\GeometryThreeDSpecialEndingSlideVideoVerOneEN}{%
  \begin{frame}[standout]

    \begin{textblock}{11}(1,0.7)

      \begin{flushleft}

        \mdseries

        \footnotesize

        \color{jFrametitleFGColor}

        This content was created as part of a project co-financed by the
        European Union within the framework of the European Social Fund
        POWR.03.05.00-00-Z309/17-00.

      \end{flushleft}

    \end{textblock}





    \begin{textblock}{10}(0,2.2)

      \tikz \fill[color=jBackgroundStyleLight] (0,0) rectangle (12.8,-1.5);

    \end{textblock}


    \begin{textblock}{3.2}(0.7,2.45)

      \includegraphics[scale=0.3]{\FundingLogoColorPictureEN}

    \end{textblock}


    \begin{textblock}{2.5}(4.15,2.5)

      \includegraphics[scale=0.2]{\JULogoColorPictureEN}

    \end{textblock}


    \begin{textblock}{2.5}(6.35,2.4)

      \includegraphics[scale=0.1]{\ZintegrUJLogoColorPictureEN}

    \end{textblock}


    \begin{textblock}{4.2}(8.4,2.6)

      \includegraphics[scale=0.3]{\EUSocialFundLogoColorPictureEN}

    \end{textblock}





    \begin{textblock}{11}(1,4)

      \begin{flushleft}

        \mdseries

        \footnotesize

        \RaggedRight

        \color{jFrametitleFGColor}

        The content of this lecture is made available under a Creative
        Commons licence (\textsc{cc}), giving the author the credits
        (\textsc{by}) and putting an obligation to share on the same terms
        (\textsc{sa}). Figures and diagrams included in the lecture are
        authored by Paweł Węgrzyn et~al., and are available under the same
        license unless indicated otherwise.\\ The presentation uses the
        Beamer Jagiellonian theme based on Matthias Vogelgesang’s
        Metropolis theme, available under license \LaTeX{} Project
        Public License~1.3c at: \colorhref{https://github.com/matze/mtheme}
        {https://github.com/matze/mtheme}.

        Typographic design: Iwona Grabska-Gradzińska;
        \LaTeX{} Typesetting: Kamil Ziemian \\
        Proofreading: Wojciech Palacz,
        Monika Stawicka \\
        3D Models: Dariusz Frymus, Kamil Nowakowski \\
        Figures and charts: Kamil Ziemian, Paweł Węgrzyn, Wojciech
        Palacz \\
        Film editing: Agencja Filmowa Film \& Television Production~--
        Zbigniew Masklak

      \end{flushleft}

    \end{textblock}

  \end{frame}
}



\newcommand{\GeometryThreeDSpecialEndingSlideVideoVerTwoEN}{%
  \begin{frame}[standout]

    \begin{textblock}{11}(1,0.7)

      \begin{flushleft}

        \mdseries

        \footnotesize

        \color{jFrametitleFGColor}

        This content was created as part of a project co-financed by the
        European Union within the framework of the European Social Fund
        POWR.03.05.00-00-Z309/17-00.

      \end{flushleft}

    \end{textblock}





    \begin{textblock}{10}(0,2.2)

      \tikz \fill[color=jBackgroundStyleLight] (0,0) rectangle (12.8,-1.5);

    \end{textblock}


    \begin{textblock}{3.2}(0.7,2.45)

      \includegraphics[scale=0.3]{\FundingLogoColorPictureEN}

    \end{textblock}


    \begin{textblock}{2.5}(4.15,2.5)

      \includegraphics[scale=0.2]{\JULogoColorPictureEN}

    \end{textblock}


    \begin{textblock}{2.5}(6.35,2.4)

      \includegraphics[scale=0.1]{\ZintegrUJLogoColorPictureEN}

    \end{textblock}


    \begin{textblock}{4.2}(8.4,2.6)

      \includegraphics[scale=0.3]{\EUSocialFundLogoColorPictureEN}

    \end{textblock}





    \begin{textblock}{11}(1,4)

      \begin{flushleft}

        \mdseries

        \footnotesize

        \RaggedRight

        \color{jFrametitleFGColor}

        The content of this lecture is made available under a Creative
        Commons licence (\textsc{cc}), giving the author the credits
        (\textsc{by}) and putting an obligation to share on the same terms
        (\textsc{sa}). Figures and diagrams included in the lecture are
        authored by Paweł Węgrzyn et~al., and are available under the same
        license unless indicated otherwise.\\ The presentation uses the
        Beamer Jagiellonian theme based on Matthias Vogelgesang’s
        Metropolis theme, available under license \LaTeX{} Project
        Public License~1.3c at: \colorhref{https://github.com/matze/mtheme}
        {https://github.com/matze/mtheme}.

        Typographic design: Iwona Grabska-Gradzińska;
        \LaTeX{} Typesetting: Kamil Ziemian \\
        Proofreading: Wojciech Palacz,
        Monika Stawicka \\
        3D Models: Dariusz Frymus, Kamil Nowakowski \\
        Figures and charts: Kamil Ziemian, Paweł Węgrzyn, Wojciech
        Palacz \\
        Film editing: IMAVI -- Joanna Kozakiewicz, Krzysztof Magda, Nikodem
        Frodyma

      \end{flushleft}

    \end{textblock}

  \end{frame}
}



\newcommand{\GeometryThreeDSpecialEndingSlideVideoVerThreeEN}{%
  \begin{frame}[standout]

    \begin{textblock}{11}(1,0.7)

      \begin{flushleft}

        \mdseries

        \footnotesize

        \color{jFrametitleFGColor}

        This content was created as part of a project co-financed by the
        European Union within the framework of the European Social Fund
        POWR.03.05.00-00-Z309/17-00.

      \end{flushleft}

    \end{textblock}





    \begin{textblock}{10}(0,2.2)

      \tikz \fill[color=jBackgroundStyleLight] (0,0) rectangle (12.8,-1.5);

    \end{textblock}


    \begin{textblock}{3.2}(0.7,2.45)

      \includegraphics[scale=0.3]{\FundingLogoColorPictureEN}

    \end{textblock}


    \begin{textblock}{2.5}(4.15,2.5)

      \includegraphics[scale=0.2]{\JULogoColorPictureEN}

    \end{textblock}


    \begin{textblock}{2.5}(6.35,2.4)

      \includegraphics[scale=0.1]{\ZintegrUJLogoColorPictureEN}

    \end{textblock}


    \begin{textblock}{4.2}(8.4,2.6)

      \includegraphics[scale=0.3]{\EUSocialFundLogoColorPictureEN}

    \end{textblock}





    \begin{textblock}{11}(1,4)

      \begin{flushleft}

        \mdseries

        \footnotesize

        \RaggedRight

        \color{jFrametitleFGColor}

        The content of this lecture is made available under a Creative
        Commons licence (\textsc{cc}), giving the author the credits
        (\textsc{by}) and putting an obligation to share on the same terms
        (\textsc{sa}). Figures and diagrams included in the lecture are
        authored by Paweł Węgrzyn et~al., and are available under the same
        license unless indicated otherwise.\\ The presentation uses the
        Beamer Jagiellonian theme based on Matthias Vogelgesang’s
        Metropolis theme, available under license \LaTeX{} Project
        Public License~1.3c at: \colorhref{https://github.com/matze/mtheme}
        {https://github.com/matze/mtheme}.

        Typographic design: Iwona Grabska-Gradzińska;
        \LaTeX{} Typesetting: Kamil Ziemian \\
        Proofreading: Wojciech Palacz,
        Monika Stawicka \\
        3D Models: Dariusz Frymus, Kamil Nowakowski \\
        Figures and charts: Kamil Ziemian, Paweł Węgrzyn, Wojciech
        Palacz \\
        Film editing: Agencja Filmowa Film \& Television Production~--
        Zbigniew Masklak \\
        Film editing: IMAVI -- Joanna Kozakiewicz, Krzysztof Magda, Nikodem
        Frodyma

      \end{flushleft}

    \end{textblock}

  \end{frame}
}



\newcommand{\GeometryThreeDTwoSpecialEndingSlidesVideoVerOneEN}[1]{%
  \begin{frame}[standout]

    \begin{textblock}{11}(1,0.7)

      \begin{flushleft}

        \mdseries

        \footnotesize

        \color{jFrametitleFGColor}

        This content was created as part of a project co-financed by the
        European Union within the framework of the European Social Fund
        POWR.03.05.00-00-Z309/17-00.

      \end{flushleft}

    \end{textblock}





    \begin{textblock}{10}(0,2.2)

      \tikz \fill[color=jBackgroundStyleLight] (0,0) rectangle (12.8,-1.5);

    \end{textblock}


    \begin{textblock}{3.2}(0.7,2.45)

      \includegraphics[scale=0.3]{\FundingLogoColorPictureEN}

    \end{textblock}


    \begin{textblock}{2.5}(4.15,2.5)

      \includegraphics[scale=0.2]{\JULogoColorPictureEN}

    \end{textblock}


    \begin{textblock}{2.5}(6.35,2.4)

      \includegraphics[scale=0.1]{\ZintegrUJLogoColorPictureEN}

    \end{textblock}


    \begin{textblock}{4.2}(8.4,2.6)

      \includegraphics[scale=0.3]{\EUSocialFundLogoColorPictureEN}

    \end{textblock}





    \begin{textblock}{11}(1,4)

      \begin{flushleft}

        \mdseries

        \footnotesize

        \RaggedRight

        \color{jFrametitleFGColor}

        The content of this lecture is made available under a Creative
        Commons licence (\textsc{cc}), giving the author the credits
        (\textsc{by}) and putting an obligation to share on the same terms
        (\textsc{sa}). Figures and diagrams included in the lecture are
        authored by Paweł Węgrzyn et~al., and are available under the same
        license unless indicated otherwise.\\ The presentation uses the
        Beamer Jagiellonian theme based on Matthias Vogelgesang’s
        Metropolis theme, available under license \LaTeX{} Project
        Public License~1.3c at: \colorhref{https://github.com/matze/mtheme}
        {https://github.com/matze/mtheme}.

        Typographic design: Iwona Grabska-Gradzińska;
        \LaTeX{} Typesetting: Kamil Ziemian \\
        Proofreading: Wojciech Palacz,
        Monika Stawicka \\
        3D Models: Dariusz Frymus, Kamil Nowakowski \\
        Figures and charts: Kamil Ziemian, Paweł Węgrzyn,
        Wojciech Palacz \\
        Film editing: Agencja Filmowa Film \& Television Production~--
        Zbigniew Masklak

      \end{flushleft}

    \end{textblock}

  \end{frame}





  \begin{frame}[standout]


    \begingroup

    \color{jFrametitleFGColor}

    #1

    \endgroup

  \end{frame}
}



\newcommand{\GeometryThreeDTwoSpecialEndingSlidesVideoVerTwoEN}[1]{%
  \begin{frame}[standout]

    \begin{textblock}{11}(1,0.7)

      \begin{flushleft}

        \mdseries

        \footnotesize

        \color{jFrametitleFGColor}

        This content was created as part of a project co-financed by the
        European Union within the framework of the European Social Fund
        POWR.03.05.00-00-Z309/17-00.

      \end{flushleft}

    \end{textblock}





    \begin{textblock}{10}(0,2.2)

      \tikz \fill[color=jBackgroundStyleLight] (0,0) rectangle (12.8,-1.5);

    \end{textblock}


    \begin{textblock}{3.2}(0.7,2.45)

      \includegraphics[scale=0.3]{\FundingLogoColorPictureEN}

    \end{textblock}


    \begin{textblock}{2.5}(4.15,2.5)

      \includegraphics[scale=0.2]{\JULogoColorPictureEN}

    \end{textblock}


    \begin{textblock}{2.5}(6.35,2.4)

      \includegraphics[scale=0.1]{\ZintegrUJLogoColorPictureEN}

    \end{textblock}


    \begin{textblock}{4.2}(8.4,2.6)

      \includegraphics[scale=0.3]{\EUSocialFundLogoColorPictureEN}

    \end{textblock}





    \begin{textblock}{11}(1,4)

      \begin{flushleft}

        \mdseries

        \footnotesize

        \RaggedRight

        \color{jFrametitleFGColor}

        The content of this lecture is made available under a Creative
        Commons licence (\textsc{cc}), giving the author the credits
        (\textsc{by}) and putting an obligation to share on the same terms
        (\textsc{sa}). Figures and diagrams included in the lecture are
        authored by Paweł Węgrzyn et~al., and are available under the same
        license unless indicated otherwise.\\ The presentation uses the
        Beamer Jagiellonian theme based on Matthias Vogelgesang’s
        Metropolis theme, available under license \LaTeX{} Project
        Public License~1.3c at: \colorhref{https://github.com/matze/mtheme}
        {https://github.com/matze/mtheme}.

        Typographic design: Iwona Grabska-Gradzińska;
        \LaTeX{} Typesetting: Kamil Ziemian \\
        Proofreading: Wojciech Palacz,
        Monika Stawicka \\
        3D Models: Dariusz Frymus, Kamil Nowakowski \\
        Figures and charts: Kamil Ziemian, Paweł Węgrzyn,
        Wojciech Palacz \\
        Film editing: IMAVI -- Joanna Kozakiewicz, Krzysztof Magda, Nikodem
        Frodyma

      \end{flushleft}

    \end{textblock}

  \end{frame}





  \begin{frame}[standout]


    \begingroup

    \color{jFrametitleFGColor}

    #1

    \endgroup

  \end{frame}
}



\newcommand{\GeometryThreeDTwoSpecialEndingSlidesVideoVerThreeEN}[1]{%
  \begin{frame}[standout]

    \begin{textblock}{11}(1,0.7)

      \begin{flushleft}

        \mdseries

        \footnotesize

        \color{jFrametitleFGColor}

        This content was created as part of a project co-financed by the
        European Union within the framework of the European Social Fund
        POWR.03.05.00-00-Z309/17-00.

      \end{flushleft}

    \end{textblock}





    \begin{textblock}{10}(0,2.2)

      \tikz \fill[color=jBackgroundStyleLight] (0,0) rectangle (12.8,-1.5);

    \end{textblock}


    \begin{textblock}{3.2}(0.7,2.45)

      \includegraphics[scale=0.3]{\FundingLogoColorPictureEN}

    \end{textblock}


    \begin{textblock}{2.5}(4.15,2.5)

      \includegraphics[scale=0.2]{\JULogoColorPictureEN}

    \end{textblock}


    \begin{textblock}{2.5}(6.35,2.4)

      \includegraphics[scale=0.1]{\ZintegrUJLogoColorPictureEN}

    \end{textblock}


    \begin{textblock}{4.2}(8.4,2.6)

      \includegraphics[scale=0.3]{\EUSocialFundLogoColorPictureEN}

    \end{textblock}





    \begin{textblock}{11}(1,4)

      \begin{flushleft}

        \mdseries

        \footnotesize

        \RaggedRight

        \color{jFrametitleFGColor}

        The content of this lecture is made available under a Creative
        Commons licence (\textsc{cc}), giving the author the credits
        (\textsc{by}) and putting an obligation to share on the same terms
        (\textsc{sa}). Figures and diagrams included in the lecture are
        authored by Paweł Węgrzyn et~al., and are available under the same
        license unless indicated otherwise. \\ The presentation uses the
        Beamer Jagiellonian theme based on Matthias Vogelgesang’s
        Metropolis theme, available under license \LaTeX{} Project
        Public License~1.3c at: \colorhref{https://github.com/matze/mtheme}
        {https://github.com/matze/mtheme}.

        Typographic design: Iwona Grabska-Gradzińska;
        \LaTeX{} Typesetting: Kamil Ziemian \\
        Proofreading: Leszek Hadasz, Wojciech Palacz,
        Monika Stawicka \\
        3D Models: Dariusz Frymus, Kamil Nowakowski \\
        Figures and charts: Kamil Ziemian, Paweł Węgrzyn,
        Wojciech Palacz \\
        Film editing: Agencja Filmowa Film \& Television Production~--
        Zbigniew Masklak \\
        Film editing: IMAVI -- Joanna Kozakiewicz, Krzysztof Magda, Nikodem
        Frodyma


      \end{flushleft}

    \end{textblock}

  \end{frame}





  \begin{frame}[standout]


    \begingroup

    \color{jFrametitleFGColor}

    #1

    \endgroup

  \end{frame}
}











% ------------------------------------------------------
% BibLaTeX
% ------------------------------------------------------
% Package biblatex, with biber as its backend, allow us to handle
% bibliography entries that use Unicode symbols outside ASCII.
\usepackage[
language=polish,
backend=biber,
style=alphabetic,
url=false,
eprint=true,
]{biblatex}

\addbibresource{Podstawy-informatyki-ETC-Bibliography.bib}





% ------------------------------------------------------
% Importing packages, libraries and setting their configuration
% ------------------------------------------------------




% ------------------------------------------------------
% Local packages
% ------------------------------------------------------
% Local configuration of this particular presentation
\usepackage{./Local-packages/local-settings}










% ------------------------------------------------------------------------------------------------------------------
\title{Podstawy informatyki z~językiem~C}
\subtitle{1. Sprawy organizacyjne}

\author{Kamil Ziemian \\
  \email}


% \date{}
% ------------------------------------------------------------------------------------------------------------------










% ####################################################################
% Beginning of the document
\begin{document}
% ####################################################################





% ######################################
% Number of chars: 23k+, 51k+,
% Text is adjusted to the left and words are broken at the end of the line.
\RaggedRight
% ######################################





% ######################################
\maketitle
% ######################################





% ##################
\begin{frame}
  \frametitle{Spis treści}


  \tableofcontents

\end{frame}
% ##################





% ######################################
\section{Informacje wstępne}
% ######################################



% ##################
\begin{frame}
  \frametitle{Informacje wstępne}


  Obawiam~się, że na pierwszych zajęciach będzie sporo przynudzania,
  ale nie widzę sposobu, by~tego uniknąć. Do tego to jest kurs
  \textit{Podstawy informatyki}, więc przyjmujemy, że~Państwo dopiero
  zaczynają swoją przygodę z~tą specyficzną dziedziną
  teoretyczno-praktyczną jaką jest informatyka.

  Według mnie to zajęcia są dla studentów, nie studenci dla zajęć. Tak samo
  ja jestem tu dla Państwa, a~nie Państwo dla mnie. Jesteśmy tu po to, by
  pomóc Państwu stawiać pierwsze kroki w~informatyce i~języku~C.
  W~związku z~tym, ja będziemy Państwa rozliczać tylko i~wyłącznie
  z~umiejętności i~wiedzy, i~z~niczego innego. Wychodzę bowiem z~założenia,
  że~Państwo sami najlepiej wiedzą, czemu warto poświęcić swój czas. (Choć
  sami potem często żałujemy tych wyborów.)

\end{frame}
% ##################





% ##################
\begin{frame}
  \frametitle{Pytania~są mile widziane}


  Na zajęciach nie tylko można, ale \alert{należy} zadawać pytania
  na dowolne związane z~nimi tematy. W~szczególności
  \alert{należy} zadawać pytania, jeśli czegoś~się nie rozumie, lub coś
  jest niejasno przedstawione. To są podstawy informatyki, \alert{nie}
  zakładamy, że~Państwo już wszystko umieją. Byłoby to bardzo
  niewłaściwe założenie.

  Proszę pamiętać, że~gdy chodzi o~tematy związane z~zajęciami
  \alert{nie} ma pytań zbyt elementarnych lub zbyt głupich. Są~tylko
  niezadowalające odpowiedzi na nie. Jesteśmy tutaj by pomóc
  Państwu w~nauce programowania w~C, pytania z~Państwa strony bardzo nam to
  zadanie ułatwiają. Zadawania pytań nie jest traktowane jako oznaka tego,
  że~ktoś czegoś nie umie, tylko że~chce~się czegoś nauczyć.

  Pytania typu „Jaki jest najfajniejszy boss w~grze \textit{Hollow
    Knight}?” musimy jednak zostawić na czas po zajęciach.

\end{frame}
% ##################





% ##################
\begin{frame}
  \frametitle{Dobre pytani}


  W~szczególności, dobrymi pytaniami są
  \begin{itemize}

  \item Co to jest komputer?

  \item Co to jest program komputerowy?

  \item Czym jest programowanie?

  \item Co to jest język programowania?

  \item Co to jest plik?

  \item Co to jest rozszerzenie pliku?

  \item etc.

  \end{itemize}

  Wszystkie te pytania są bardzo dobre. Pytanie, które jest proste
  w~postawieniu, nie jest z~tego powodu w~żadnym razie głupie. Prawdziwym
  problemem jest to, że~na~takie proste pytania, często bardzo trudno
  udzielić satysfakcjonującej odpowiedzi. Takie rzeczy to większość
  informatyków bardziej czuje, niż umie wysłowić.

\end{frame}
% ##################





% ##################
\begin{frame}
  \frametitle{Co zrobimy, a~czego nie}


  Na tych zajęciach \alert{nie} nauczymy~się programować. Jak dobrze
  pójdzie to nauczymy~się podstaw programowania w~języku~C, ale
  programowanie obejmuje tyle zagadnień i~wymaga tyle godzin praktyki,
  iż~nie ma najmniejszych szans, że~uda nam~się to \alert{wszystko} zrobić.

  Co, na Państwa nieszczęście, nie oznacza, że~będzie mało materiału.
  Mogą~się Państwo wręcz czuć przytłoczeni jego ilością. Proszę mi jednak
  uwierzyć, że~większość tych zagadnień nie jest taka trudna, jak~się
  może wydawać na pierwszy rzut oka. Rzeczy nowe jawią~się często jako dużo
  trudniejsze niż są naprawdę.

  Przyznam~się, że~sam staję w~tym momencie przed duży dylematem. Czy to
  mają być \alert{podstawy informatyki} z~wykorzystaniem języka~C, czy
  \alert{nauka programowania w~języku~C} z~podstawami informatyki. Na którą
  część położyć większy nacisk, to jest problem przed którym stoję.
  W~obecnej chwili zdecydowaliśmy~się iść drogą nauki języka~C. A~jaka
  jest Państwa opinia w~tej sprawie?

\end{frame}
% ##################





% ##################
\begin{frame}
  \frametitle{Staram~się podchodzić do przedmiotu praktycznie}


  Wedle mojej obserwacji większość osób na tych studiach jest nastawiona
  na praktyczną stronę informatyki: jak stworzyć program który działa i~to
  w~pożądany sposób? Biorąc takie podejście za punkt odniesienia,
  staram~się prowadzić te zajęcia tak, aby najpierw mieli Państwo okazję
  oswoić~się z~tworzeniem programów w~języku~C, nawet jeśli są one
  bardzo proste i~nie robią nic ciekawego. Zagadnienia teoretyczne
  i~bardziej techniczne, które dobry programista musi znać, będę w~miarę
  możliwości omawiane na późniejszych zajęciach.

  Minus tego podejścia jest następujący. W~języku~C cała maszyneria
  komputera jest delikatnie schowana pod dość cienką warstwą tego języka
  (w~dialekcie informatyków powiedzielibyśmy „pod cienką warstwą
  abstrakcji”) i~nawet początkujący ciągle natrafia na drugie dno problemu.
  Doświadczenie pokazuje, że~wyjaśnianie czemu tak jest wymaga sporo czasu
  i~mocno zaburza nakierowany na praktykę programistyczną plan zajęć.

\end{frame}
% ##################





% ##################
\begin{frame}
  \frametitle{Niektóre pytania idą zbyt głęboko}


  Dlatego jak najbardziej jesteśmy \alert{bardzo} za tym by zadawali nam
  Państwo pytania, ale w~wielu wypadkach nie będziemy mogli na nie od razu
  odpowiadać, bo
  dobre wyjaśnienie zajęłoby za dużo czasu. Zawsze można~się do nas
  zgłosić z~takimi pytaniami po zajęciach, wtedy odpowiemy najlepiej
  jak umiemy. W~trakcie zajęć będziemy jednak często musieli powiedzieć
  „Proszę zrobić tak i~tak. Dlaczego, wyjaśnimy to potem.”. Proszę mieć dla
  nas wyrozumiałość w~tym względzie.

  Jeśli to o~czym mówimy jest tak proste, że~omawianie tego jest stratą
  Państwa czasu, proszę nas o~tym poinformować, przejdziemy do następnego
  zagadnienia. Jest bardzo trudno sprawą wyczucie, co jest dla Państwa
  proste, a~co trudniejsze.

  Proszę mieć dla nas pewną wyrozumiałość, pamiętać o~tych zastrzeżeniach
  i~próbować pomóc nam w~takim prowadzeniu zajęć, by Państwo wynieśli z~nich
  jak najwięcej. Jak mówiliśmy, te~zajęcia są dla Państwa.

\end{frame}
% ##################





% ##################
\begin{frame}
  \frametitle{Będziemy wiele upraszczać}


  Ponieważ tematyka którą poruszamy jest mimo wszystko niebanalna, więc
  mnóstwo rzeczy będziemy musieli \alert{upraszczać}. Proszę mieć to na
  uwadze w~trakcie zajęć i~studiując materiały do nich.

  Potrzeba uproszczeń wynika z~dwóch powodów. Po pierwsze, ograniczenia
  czasowe. Wiele z~zagadnień które poruszymy mogłoby być tematem
  semestralnego kursu. Co gorsza, wiele z~nich \alert{jest} tematem
  semestralnych kursów, przykładowo: praca z~liczbami zmiennoprzecinkowymi
  i~powiązane z~tym metody numeryczne. Po drugie, to jest kurs
  \alert{podstaw} informatyki, który ma położyć fundamenty pod Państwa
  umiejętności i~wiedzę w~zakresie informatyki. To~nie przedmiot na którym
  należy wnikać we wszystkie detale, szczegóły, drugie, trzecie i~czwarte
  dno problemu.

  Jeśli jednak ktoś chce~się bardziej zagłębić w~te temat, to po zajęciach
  służymy całą swoją osobą.

\end{frame}
% ##################





% ##################
\begin{frame}
  \frametitle{Naprawdę cenimy Państwa zadanie}


  Jeśli ktoś ma uwagi do tych zajęć, propozycję co można zmienić,
  co~poprawić, to proszę powiedzieć to mi po zajęciach lub napisać pod
  adres \email. Ponownie prosimy by w~tytule emaila napisać
  \textit{Uwagi do zajęć} lub coś podobnego, bo inaczej może zniknąć na
  długo w~skrzynce.

  W~chwili obecnej nie mamy złudzeń, że~nasz system jest choćby bliski
  ideałowi, próbujemy jednak go~ciągle ulepszać. Państwa zdanie jest dla
  nas naprawdę \alert{ważne}, choć oczywiście nie zawsze~się z~nim zgadzamy.
  To jak ten przedmiot wygląda obecnie wynika częściowo z~tego co
  w~przeszłości zaproponowali studenci.

  Choć trudno w~to uwierzyć, opinie typu \textit{Te zajęcia to jest dno!}
  nie wpływają negatywnie na ocenę. Acz Państwa to raczej nie przekona
  do~wypowiadania~się ;).

\end{frame}
% ##################










% ######################################
\section{Kilka pytań i~odpowiedzi}
% ######################################


% ##################
\begin{frame}
  \frametitle{Dlaczego na tych zajęciach nie nauczymy~się
    programować?}


  Rozpatrzmy problem liczb z~częścią dziesiętną, takich jak $0.1$, $2.71$,
  czy $3.14$. Formalna ich nazwa w~informatyce to \textbf{liczby
    zmiennoprzecinkowe} (ang. \textit{floating-point numbers}). Klasyczny
  artykuł Davida Goldberga
  \colorhref{https://dl.acm.org/doi/10.1145/103162.103163}
  {\textit{Co każdym informatyk musi wiedzieć o~arytmetyce liczb
      zmiennoprzecinkowych}} z~1991 roku, ma w~wersji \textsc{pdf}
  $44$~strony długości. I~nie jest to tekst, który~się szybko czyta.

  Mówiąc inaczej, gdyby ten przedmiot prowadził zawodowy
  zmiennoprzecinkowiec to mógłby \alert{wszystkie} (powtórzmy to:
  wszystkie) nasze spotkania poświęcić tylko i~wyłącznie operacjom
  arytmetycznym w~komputerze. I~tak pewnie zabrakło by mu czasu, by zrobić
  wszystko co uważa za ważne.

  My tego nie będziemy robić, bo trzeba jakoś omówić również liczby
  całkowite, stringi, zmienne, typy zmiennych, konwersje typów, algebrę
  Boole’a, operatory logiczne, instrukcje warunkowe, pętle, struktur, etc.
  Nie jest pewne, czy i~tak uda nam~się to wszystko zrobić.

\end{frame}
% ##################





% ##################
\begin{frame}
  \frametitle{Tak jest prawda}

  \vspace{-0.5em}


  \begin{figure}

    \label{fig:Learning-any-language-at-100-procent}

    \centering


    \includegraphics[scale=0.195]
    {./Presentations-pictures/Learning-language-at-100.jpg}

  \end{figure}

\end{frame}
% ##################





% ##################
\begin{frame}
  \frametitle{Dlaczego język~C?}


  Dlaczego zaczynamy naukę od~języka~C? Krótka odpowiedź jest taka,
  że~pomimo tego iż język ten ma już pół wieku~(!) na karku, w~2025 roku
  nasza infrastruktura informatyczna wciąż stoi na kodzie źródłowym
  napisanym właśnie w~nim. W~internecie mogą Państwo znaleźć wiele
  artykułów i~blogów takich jak
  \colorhref{https://wideinfo.org/c-programming-is-still-running-the-world/}
  {\textit{C~programming is still running the world}}, datowany na styczeń
  $2025$ roku, którego tytuł mówi sam za siebie
  \parencite{Scott-C-programming-is-still-ETC-Ver-2025}.

  Niektórzy mówią, że~C to król wszystkich języków programowania.
  Inni, bardzo dobrzy informatycy, twierdzą, że każdy szanujący
  programista musi znać~C (cf. str.~17
  \parencite{Hoey-Programowanie-w-asemblerze-x64-ETC-Pub-2024}).

  A~jak to jest w~praktyce? Na te zajęcia potrafią uczęszczać ludzie,
  którzy pracują zawodowo jako programiści i~nie mają pojęcia jak napisać
  najprostszy program w~języku~C. Proszę samemu wyciągnąć z~tego wnioski.

\end{frame}
% ##################





% ##################
\begin{frame}
  \frametitle{Dlaczego język~C?}


  Ponieważ nauka informatyki nie może~się ograniczać tylko do znajomości
  jednego języka programowania, możemy sobie zadać dwa pytania. Czy warto
  znać język~C? Czy C~jest właściwym językiem dla kursu \textit{Podstawy
    informatyki}?

  W~naszej ocenie warto znać język~C i~cały ten semestr będziemy~się
  starali pokazać Państwu dlaczego. Czy nam~się to uda, to inna sprawa.

  Gdy chodzi o~drugie pytanie, to odpowiedź zależy od~zapytanego. Prowadzący
  ten kursu uważają, że~C jest właściwym językiem do tego kursu, my
  wybralibyśmy bardziej przyjazny dla początkujących
  język~\colorhref{https://go.dev/}{Go}. Jeśli Państwo chcą by ten kurs był
  prowadzony w~innym języku, to muszą~się Państwo zwrócić z~prośbą do
  prowadzących cały kurs, bo tylko oni mogą taką zmianę wprowadzić.

\end{frame}
% ##################





% ##################
\begin{frame}
  \frametitle{Co można napisać w~języku~C?}


  Wszystko. Powtórzmy to. W~języku~C można napisać wszystko. W~tym języku ze
  względu na pewne jego zalety, napisano już program każdego typu i~czy
  ktoś jest tego świadomy, czy nie, każdego dnia korzysta jakoś z~programów
  napisanych w~języku~C. Są dwa główne powody. Po pierwsze, dobre programy
  napisane w~C są bardzo szybkie i~wydajne. Po drugie, kod napisany w~C
  można uruchomić wszędzie: od pralki do superkomputera i~to nie jest
  metafora.

  Jeśli jednak ktoś chce napisać jakiś program komputerowy, to pytanie
  „Czy powinienem go napisać w~C?” nie ma prostej odpowiedzi. Zależy od tego
  co ten program ma robić, na jakim sprzęcie ma działać, jakich programistów
  mamy do dyspozycji (bardzo ważny czynnik), jak ma być szybki,~etc.

\end{frame}
% ##################





% ##################
\begin{frame}
  \frametitle{Dlaczego system GNU/Linux?}


  Dlaczego korzystamy z~systemu GNU/Linux, a~nie z~znacznie
  popularniejszego wśród normalny ludzi systemu Windows? Bo~oferuje
  znacznie lepsze warunki pracy z~językiem~C.

  \alert{Ważne.} Jeśli mają Państwo jakiekolwiek problemy z~systemem
  GNU/Linux to proszę o~tym \alert{mówić}. Nie przyjmujemy założenia,
  że~Państw mają już teraz być ekspertami, w~kwestii używania tego, co by
  tu nie mówić, często bardzo topornego systemu operacyjnego.

  Jak bowiem głosi mądrość internetu „GNU/Linux jest darmowy, tylko jeśli
  twój czas nie ma wartości.”

\end{frame}
% ##################





% ##################
\begin{frame}
  \frametitle{Rozwój GNU/Linuxa}

  \vspace{-0.5em}


  \begin{figure}

    \label{fig:Evolution-of-OS}

    \centering


    \includegraphics[scale=0.3]
    {./Presentations-pictures/Evolution-of-operating-systems.jpg}

  \end{figure}

\end{frame}
% ##################





% ##################
\begin{frame}
  \frametitle{Dlaczego na tych zajęciach używamy~X?}


  Na w~zasadzie wszystkie pytania typu „Czemu na tych zajęciach
  używamy~X?” odpowiedź jest taka sama. Bo my uważamy, że~jest to dla tego
  kursu optymalny wybór, zaś inni prowadzący zapewne użyliby czegoś innego.

  Celem tego przedmiotu jest zaznajomienie Państwa z~podstawami
  programowania w~języku~C, wybór narzędzi do realizacji tego zadania jest
  sprawą drugorzędną, więc każdy prowadzący wybiera te, które on uważa za
  optymalne, kierując~się przy tym wybranymi przez siebie kryteriami.

\end{frame}
% ##################





% ##################
\begin{frame}
  \frametitle{Dlaczego programy, które piszemy nie są ciekawe?}


  Ciekawe programy są z~reguły długie, już stosunkowo skomplikowane, podatne
  na~różnorakie błędy ze strony programisty, mówiąc krótko, bardzo trudne
  dla początkujących. Stąd na tych zajęciach pojawiają~się programy
  możliwie proste, a~to często skutkuje nudą. Próbujemy w~miarę naszych
  możliwości ograniczać poziom nudy, ale to nie jest proste, ani łatwe.

  Jeśli będą~się Państwo zmagać z~tym, że~program z~zajęć jest nudny,
  ale już zbyt skomplikowany, by go zrozumieć lub działać poprawnie, proszę
  pamiętać, że~ciekawe programy, są jeszcze bardziej skomplikowane i~jeszcze
  bardziej podatne na błędy.

  Co jest ciekawe, a~co nudne, oczywiście mocno zależy od danego człowieka.
  Biorąc również to pod uwagą, proszę~się dodatkowo uzbroić na tych
  zajęciach w~cierpliwość. Nauka programowania jest tego warta, choć nie
  jest to wcale łatwo pokazać na~zajęciach.

\end{frame}
% ##################










% ######################################
\section{Uzyskanie zaliczenia}
% ######################################


% ##################
\begin{frame}
  \frametitle{Zaliczenie zaoczne}


  Zaliczenie zaoczne można jak najbardziej uzyskać, np.~przedstawiając
  kod jakiegoś swojego programu czy aplikacji wraz w~wyjaśnieniem jak
  działa. Projekt ten \alert{nie musi} być napisany w~języku~C, może być
  w~Pythonie albo JavaScripcie. Z~powodów które powinny być dla wszystkich
  oczywiste, preferowane są jednak te stworzone w~języku C. W~trochę
  mniejszym stopniu, te w~języku~C++.

  Każdy kto chce uzyskać zaliczenie zaoczne, niech zgłosi~się do nas po
  zajęciach lub napisze na e-maila \email. W~tytule proszę wpisać
  \textit{Pytanie o~zaliczenie zaoczne} lub coś podobnego, inaczej e-mail
  może na~długo zaginąć pośród innych.

\end{frame}
% ##################





% ##################
\begin{frame}
  \frametitle{Uzyskanie zaliczenia}


  Zaliczenie i~ocenę uzyskują Państwo na podstawie dwóch rzeczy:

  \vspace{-0.3em}



  \begin{itemize}

  \item prac domowych;

  \item dwóch sprawdzianów praktycznych na~ćwiczeniach.

  \end{itemize}

  \vspace{-0.3em}


  Za prace domowe wystawiana jest jedna ocena, za~dwa sprawdziany
  dwie oceny, przy czym ocena za~prace domowe, stanowi
  \alert{połowę} wkładu do oceny końcowej. Dokładniej, ocena końcowa
  jest liczona jako średnia ważona: \\[0.3em]
  $\displaystyle
  \text{ocenaKońcowa} =
  \frac{ \text{ocenaSpraw1} + \text{ocenaSpraw2} +
    2 \cdot \text{ocenaPraceDom} }{ 4 }$,

  \vspace{0.5em}


  gdzie $\text{ocenaSpraw1}$ i~$\text{ocenaSpraw1}$ to dwie oceny ze
  sprawdzianów, zaś $\text{ocenaPraceDom}$ to ocena z~prac domowych.


  Podział na ćwiczenia i~laboratoria jest bardzo sztuczny, obecnie
  to bardziej kwestia administracji, niż praktyki. Dlatego pod koniec
  semestru otrzymują Państwo \alert{jedną} ocenę, która
  wpisywana jest jako ocena z~ćwiczeń i~laboratoriów.

\end{frame}
% ##################





% ##################
\begin{frame}
  \frametitle{O~pracach domowych}


  Z~naszego punktu widzenia prace domowe~są najważniejszym czynnikiem
  przy wyliczaniu oceny, gdyż pozwalają Państwu najwięcej~się nauczyć,
  poprzez rozwiązywanie konkretnego problemu. Proszę nie traktować prac
  domowych, jako nieprzyjemnego obowiązku do~odhaczenia, ale właśnie jako
  okazję do~nauczenia~się czegoś o~pisaniu programów komputerowych.

  Aby zostać dobrym informatykiem trzeba sporo czasu spędzić pisząc,
  testując i~debugując własne programy, te zadania temu przede wszystkim
  mają służyć. To że~otrzymują Państwo za nie punkty, to w~porównaniu z~tym
  sprawa drugorzędna.

  Wątpię by rozwiązywanie i~oddawanie tych prac było przyjemnością,
  ale~można spróbować o~nich myśleć jako o~wartościowym spędzaniu czasu.
  Łatwo nam to mówić, ale warto próbować.

\end{frame}
% ##################





% ##################
\begin{frame}
  \frametitle{O~pracach domowych}


  Ile będzie zestawów zadań? Prawdopodobnie koło siedmiu.

  Początkowo zadania są bardzo prosto, jak na ten przedmiot, z~czasem ich
  trudność, jak i~punkty do zdobycia za nie rosną. Układanie dobrych
  zestawów zadań to trudna sztuka, stąd będziemy wdzięczni za~wszystkie
  uwagi jak można je poprawić. Przypominam, że~o tych zestawach lepiej
  myśleć jako o~sposobie nauczenia~się tworzenia programów, a~nie
  nieprzyjemnym obowiązku do odhaczenia. Easier said than done, wiemy o~tym.

  Jeśli dla kogoś zadania domowe to będzie za mało, co jest wspaniałą
  wiadomością, to na listach materiałów do nauki będą zamieszczone źródła,
  z~większą liczbą zadań do~rozwiązania.

\end{frame}
% ##################





% ##################
\begin{frame}
  \frametitle{O~rozwiązywaniu prac domowych}


  Proszę przyjąć zasadę, że~\alert{najprostszy program} który rozwiązuje
  zadanie zgodnie z~wytycznymi jest najlepszym jaki można oddać. Niekiedy
  studenci oddają bardzo skomplikowane programy jako rozwiązanie naprawdę
  prostych zadań, co jest zupełnie niepotrzebne.

  Oczywiście, nikt nie straci punktów za oddanie skomplikowanego programu,
  ale jest wiele powodów, by poprzestać na pisaniu prostego kodu.
  Po~pierwsze, to zwykle oszczędza czas spędzony na~rozwiązywaniu zadań
  na~ten przedmiot. Po~drugie, prostota i~czytelność kodu są naprawdę ważne
  przy tworzeniu programów na~każdym poziomie.

  Należy bowiem wiedzieć, że~pisanie programów, wbrew pozorom,
  \alert{nie} jest trudne. \alert{Trudne jest ich debugowanie.}
  Dokładniej, obowiązuje prawo
  \colorhref{https://en.wikipedia.org/wiki/Brian_Kernighan}{Kernighana}. \\
  \textit{Debugowanie programu jest dwa razy trudniejsze, niż jego
    napisanie. Jeśli więc napiszesz program w~najbardziej inteligentny
    sposób jaki znasz, to nie jesteś wystarczająco mądry by go
    zdebugować.}

\end{frame}
% ##################





% ##################
\begin{frame}
  \frametitle{Pomoc przy rozwiązywaniu prac domowych}


  Ponieważ zadania są głównie po to, by mogli Państwo nauczyć~się
  programować, jeśli Państwo tempo kopiują rozwiązanie kogoś innego,
  to oszukują Państwo głównie samych siebie.

  Ostrzegam, ja bardzo \alert{nie lubię} ściągania. Student który
  oblał~$20$ poprawkę nie będzie miał u~mnie takich problemów, jak osoba
  którą złapię na ściąganiu. Moja zasada jest taka, że~jak kogoś na tym
  nakryję na tego typu oszustwach, to więcej niż $3.0$ u~mnie
  \alert{nie dostanie}. Chyba, że~przymkniemy oko i~udamy, że~tego nie było.
  Wszystko zależy.

  Ściągania nie da~się całkiem wyeliminować, ale można je ograniczyć.
  A~jak ktoś ściąga tak, żeby ślepy zauważył, to może mieć pretensje tylko
  i~wyłącznie do siebie. ;)

  Gdy chodzi o~zadania domowe i~projekt, to należy próbować rozwiązać je
  możliwie samodzielnie. W~razie napotkania problemów nie tylko można,
  ale i~\alert{należy} prosić o~pomoc kolegów, korzystać z~materiałów
  w~internecie i~używać programów takich jak Chat\textsc{gpt}.

\end{frame}
% ##################





% ##################
\begin{frame}
  \frametitle{Pomoc przy rozwiązywaniu prac domowych}


  % Gdy chodzi o~zadania domowe i~projekt, to należy próbować rozwiązać je
  % możliwie samodzielnie. W~razie napotkania problemów nie tylko można,
  % ale i~należy prosić o~pomoc kolegów, korzystać z~materiałów w~internecie
  % i~używać programów takich jak Chat\textsc{gpt}.

  Ważne jest tylko to by rozwiązanie prac domowych nie ograniczyło~się
  do zastosowania metody Copy’ego-Pejsta.

  \vspace{-0.5em}





  \begin{figure}

    \label{fig:The-best-salute}

    \centering


    \includegraphics[scale=0.19]
    {./Presentations-pictures/The-best-salute.jpg}

  \end{figure}

  % Oczywiście, nikt nie straci punktów za oddanie skomplikowanego programu,
  % ale jest wiele powodów, by poprzestać na pisaniu prostego kodu.
  % Po~pierwsze, to zwykle oszczędza czas spędzony na~rozwiązywaniu zadań
  % na~ten przedmiot. Po~drugie, prostota i~czytelność kodu są naprawdę ważne
  % przy tworzeniu programów na~każdym poziomie.

  % Jeżeli ktoś oddaje trudne rozwiązywania zadań, bo chce spróbować swoich
  % sił w~programowaniu, to oczywiście, naszym obowiązkiem jest poprawić
  % ten zestaw bez marudzenia. Wszystkie te uwagi nie są po to byśmy
  % my mieli prostszą pracę przy poprawianiu zdań, tylko by Państwo nie
  % trudzili~się niepotrzebnie.

  % Proszę też pamiętać, że~obowiązuje prawo Kernighana. \\
  % \textit{Debugowanie programu jest dwa razy trudniejsze, niż jego
  %   napisanie. Jeśli więc napiszesz program w~najbardziej pomysłowy sposób
  %   jaki możesz, to nie jesteś wystarczająco inteligentny by go zdebugować.}

  % Ostrzegam, ja bardzo \alert{nie lubię} ściągania. Student który
  % oblał~$20$ poprawkę nie będzie miał u~mnie takich problemów, jak osoba
  % którą złapię na ściąganiu. Moja zasada jest taka, że~jak kogoś na tym
  % nakryję na tego typu oszustwach, to więcej niż $3.0$ u~mnie
  % \alert{nie dostanie}.

  % Ściągania nie da~się całkiem wyeliminować, ale można je ograniczyć.
  % A~jak ktoś ściąga tak, żeby ślepy zauważył, to może mieć pretensje tylko
  % i~wyłącznie do siebie. ;)

  % Gdy chodzi o~zadania domowe i~projekt, to należy próbować rozwiązać je
  % możliwie samodzielnie. W~razie napotkania problemów nie tylko można,
  % ale i~należy prosić o~pomoc kolegów, korzystać z~materiałów w~internecie
  % i~używać programów takich jak Chat\textsc{gpt}.

\end{frame}
% ##################





% ##################
\begin{frame}
  \frametitle{O~oddawaniu prac domowych}


  Można też prosić nas o~pomoc pisząc e-mail o~tytule typu
  \textit{Prośba o~pomoc z~zadaniami} pod adres \email. W~razie pytania,
  na które należy udzielić bardzo szybkiej proszę napisać w~tej sprawie pod
  numer $883 \, 010 \, 779$. Odradzam dzwonić, bo z~reguły unikamy
  odbierania telefonów od nieznanych numerów. Z~jakiegoś powodu do tej pory
  mało osób z~tego korzystało. Dziwne ;).

  Prosimy oddawać jako rozwiązania pliki o~nazwie \\
  \texttt{Imię-Nazwisko-Zestaw-XX-Zad-YY.c} \\
  gdzie za \texttt{XX} i~\texttt{YY} trzeba oczywiście wstawić odpowiednie
  numery zestawu i~zadania oraz przesyłać je pod adres \email. Nie jest to
  $100$\% obowiązkowe, ale gdy~się dostaje $50$~rozwiązań jednego zestawu
  zadań, to naprawdę nam pomaga.

  Ocenie podlega tylko kod źródłowy języka~C, proszę nie przesyłać nic
  więcej, chyba że~w~zestawie zadań napisano inaczej. Pliki zawierające
  kod źródłowy~C prawie zawsze kończą~się na \texttt{.c}.

\end{frame}
% ##################





% ##################
\begin{frame}
  \frametitle{O~oddawaniu prac domowych}


  Jeśli będzie trzeba przesłać innego typu, jak plik nagłówkowy
  (ang.~\textit{header}) kończący~się na~\texttt{.h}, to będzie to jawnie
  zaznaczone w~zadaniu. Acz na dzień dzisiejszy jest mała szansa, że~takie
  zadanie w~ogóle~się pojawi.

  Jeśli plik kończy~się na~\texttt{.c} to prawie na pewno jest to plik
  z~kodem źródłowy w~języku~C. Jeśli plik kończy~się choćby
  na~\texttt{.cbp} czy \texttt{.layout} to na $99.99$\% plik zupełnie
  innego typu, którego nie należy przesyłać. Doświadczenie mówi,
  że~większość osób na tym kursie jest początkującymi informatykami, więc
  nic dziwnego, że~niekiedy Państwo~się mylą i~przesyłając nie ten plik
  co~trzeba.

  Mała dygresja. Plik pozbawiony rozszerzenia (brak kropki w~nazwie) to
  nie jest dobry pomysł. Plik ze spacją w~nazwie to proszenie~się
  o~problemy, gdy trzeba pracować w~systemie GNU/Linux (\textsc{bash}
  i~inne sprawy).

\end{frame}
% ##################





% ##################
\begin{frame}
  \frametitle{O~oddawaniu prac domowych}


  Plik który ma~w~nazwie dwie lub~więcej kropek, to wymysł Szatana i~wytwór
  piekieł, z~tego powodu do walki z~nim należy wysłać choćby Doom Marina
  (\colorhref{https://www.youtube.com/watch?v=I5mRwzVvFGE}{wersja
    z~$2016$ roku} jest całkiem dobra, w~nowsze nie grałem).
  Bardzo proszę nie skazywać mnie na obcowanie z~tymi diabelskimi tworami.
  Acz za przesyłanie mi tych wytworów czeluści piekielnych nie stracą
  Państwo punktów.

\end{frame}
% ##################





% ##################
\begin{frame}
  \frametitle{A~co~z~AI?}

  \vspace{-0.5em}


  \begin{figure}

    \label{fig:Coping-for-others}

    \centering


    \includegraphics[scale=0.415]
    {./Presentations-pictures/Copying-from-others.jpg}

  \end{figure}

\end{frame}
% ##################





% ##################
\begin{frame}
  \frametitle{A~co~z~AI?}

  \vspace{-0.5em}


  \begin{figure}

    \label{fig:Impact-of-ChatGPT-One-view}

    \centering


    \includegraphics[scale=0.38]
    {./Presentations-pictures/Impact-of-ChatGPT-One-view.jpg}

  \end{figure}

\end{frame}
% ##################





% ##################
\begin{frame}
  \frametitle{A~co~z~AI?}


  Jak była mowa wcześniej, dla nas nie jest problemem, jeśli ktoś rozsądnie
  korzysta z~\textsc{ai}. To czy ta technologi w~pracy informatyka
  bardziej pomaga, czy szkodzi, to temat budzący wiele emocji w~środowisku
  i~przedmiot bardzo ostrych debat, więc bardzo trudno podać wyważoną
  opinię na jego temat.

\end{frame}
% ##################





% ##################
\begin{frame}
  \frametitle{Skala oceniania prac domowych}


  Przy każdym zadaniu z~prac domowych, podana jest liczba punktów możliwych
  za nie do zdobycia. Po koniec semestru podliczana jest liczba wszystkich
  punktów jakie Państwo zdobyli i~w~zależności jaki procent wszystkich
  punktów one stanowią, wystawiana jest ocena według poniższej skali.

  \vspace{-0.3em}



  \begin{itemize}

  \item $41\%\text{--}50\%$~-- ocena dostateczna ($3.0$).

  \item $51\%\text{--}60\%$ -- ocena plus dostateczna ($3.5$, $3+$).

  \item $61\%\text{--}80\%$ -- ocena dobra ($4.0$).

  \item $81\%\text{--}90\%$ -- ocena puls dobry ($4.5$, $4+$).

  \item $91\%\text{--}100\%$ -- ocena bardzo dobry ($5.0$).

  \end{itemize}

  \vspace{-0.3em}



  W~przypadku zaokrąglanie wyników, robione to jest zawsze na Państwa
  korzyść. Czyli $40.1\%$ zaokrągla~się do $41\%$.

\end{frame}
% ##################






% ##################
\begin{frame}
  \frametitle{Sprawdziany}


  Dwa razy w~semestrze, prawdopodobnie zaraz po $15$~listopada
  i~około~$15$-tego stycznia, odbędą~się dwa sprawdziany. Ich przebieg jest
  prosty. Przychodzą Państwo do sali, gdzie odbywają~się zajęcia, dostają
  Państwo jedno lub więcej zadań i~muszą je samodzielnie rozwiązać. Bez
  pomocy kolegów, bez internetu i~bez Chata\textsc{gpt}.

  Obecność na tym sprawdzianie jest obowiązkowa, chyba, że~ktoś przyniesie
  usprawiedliwienie, typu „Muszę zostać w~pracy, bo brak zastępstwa.”,
  „Bardzo ważne sprawa prywatna~X.”,~etc. Jeśli ktoś nie przyjdzie na
  sprawdzian i~\alert{nie} prześle usprawiedliwienia z~miejsca traci tyle
  punktów, że~dostanie najwyżej~$3.0$.

  Tak wiemy, że~czasem \alert{nie ma zajęć stacjonarnych}. Wtedy oczywiście
  piszemy zdalnie, bo trzeba sobie jakoś radzić. Jestem bardzo przeciwny
  temu, żeby pewne grupy miały tylko zajęcia zdalne, ale~niewiele mogę
  w~tej sprawie zrobić.

\end{frame}
% ##################





% ##################
\begin{frame}
  \frametitle{Sprawdziany}


  Potrzeba takie sprawdzianu na żywo wynika z~wielu czynników, przed
  wszystkim z~tego, że~obecnie, również dzięki~\textsc{ai}, miarodajność
  prac domowych jest wątpliwa. Do~tego sytuacja ta krzywdzi tych, co sami
  rozwiązują zadania i~w~sposób naturalny popełniają przy tym błędy, co
  jest zupełnie niedopuszczalne. Ci co sami pracują, mają prawo być
  oceniani lepiej, niż~ci, którzy przesyłają nie swoje rozwiązanie,
  a~nie na odwrót.

  Na sprawdzianie może być jedno zadanie, może być ich kilka. Przy czym jest
  możliwość, że~zadanie będzie polegało na napisaniu samemu, programu,
  który już napisaliśmy wspólnie na zajęciach. Doświadczenie pokazuje,
  że~dawanie już przerobionych zadań na sprawdzianach, wcale nie
  prowadzi do tego, iż~wszyscy dostają piątki.

\end{frame}
% ##################





% ##################
\begin{frame}
  \frametitle{Skala oceniania sprawdzianów}


  Przy każdym zadaniu będzie podana liczba punktów do zdobycia, ocena
  za~sprawdzian jest wystawiana według tej samej skali, co przy pracach
  domowych.

  \vspace{-0.3em}



  \begin{itemize}

  \item $41\%\text{--}50\%$~-- ocena dostateczna ($3.0$).

  \item $51\%\text{--}60\%$ -- ocena plus dostateczna ($3.5$, $3+$).

  \item $61\%\text{--}80\%$ -- ocena dobra ($4.0$).

  \item $81\%\text{--}90\%$ -- ocena puls dobry ($4.5$, $4+$).

  \item $91\%\text{--}100\%$ -- ocena bardzo dobry ($5.0$).

  \end{itemize}

  \vspace{-0.3em}



  Tak jak poprzednio, w~przypadku zaokrąglanie wyników, robione to jest
  zawsze na Państwa korzyść. Czyli $40.1\%$ zaokrągla~się do $41\%$.

\end{frame}
% ##################










% ######################################
\section{Konsultacje}
% ######################################


% ##################
\begin{frame}
  \frametitle{O~zajęciach i~konsultacjach}


  Z~naszego doświadczenia wynika, że~ustalanie jednego terminu na
  konsultacje to nie jest dobry pomysł. W~zasadzie nikt wtedy nie
  przychodzi, a~ja wyznaję zasadę, że~konsultacje są dla Państwa, nie dla
  mnie. Jeśli Państwo chcą bym ustalił konkretne terminy na konsultacje, to
  proszę jako grupa wybrać jeden taki i~poinformować mnie o~tym mailowo,
  pisząc na adres \email. W~tytule maila proszę napisać \textit{Termin
    konsultacji} lub coś podobnego, byśmy wiedzieli, że~na ten mail należy
  możliwie szybko odpowiedzieć.

  W~przeciwnym razie, jeśli ktoś z~Państwa ma problem i~chce zasięgnąć
  naszej pomocy, proszę do nas podejść po zajęciach lub napisać na
  wspomniany już adres \email, kiedy, gdzie i~w~jakiej formie chcą Państwo
  uczestniczyć w~konsultacjach. Mogą one być zarówno w~świecie rzeczywistym
  (niekoniecznie w~budynku \textsc{wsz}i\textsc{b}u), online lub
  telefonicznie.

\end{frame}
% ##################





% ##################
\begin{frame}
  \frametitle{Bardzo ważne}


  Ponawiamy prośbę o~nadanie takiemu e-mailowi tytułu typu \textit{Termin
    konsultacji}, bo w~przy ilości e-maili jakie trafiają na tę skrzynkę
  pocztową łatwo jeden przegapić.

  Można też napisać pod numer $883 \, 010 \, 779$. Przypominamy,
  że~wiadomości tekstowe czytamy, a~telefony od nieznanych numerów, raczej
  nie odbieramy.

  Jeśli ktoś będzie pisał w~sprawie konsultacji, to będziemy wdzięczni
  za~napisanie z~czym konkretnie mają Państwo problem. Rozumiem jedna,
  że~często wskazanie czy nazwanie tego co sprawia komuś problem, samo nie
  jest łatwe. Sami przez to przechodziliśmy.

\end{frame}
% ##################










% ####################################################################
% ####################################################################
% Bibliography

\printbibliography





% ############################
% End of the document

\end{document}
