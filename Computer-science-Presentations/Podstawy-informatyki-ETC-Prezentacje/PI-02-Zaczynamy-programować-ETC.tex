% ------------------------------------------------------------------------------------------------------------------
% Basic configuration of Beamera class and Jagiellonian theme
% ------------------------------------------------------------------------------------------------------------------
\RequirePackage[l2tabu, orthodox]{nag}



\ifx\PresentationStyle\notset
  \def\PresentationStyle{dark}
\fi



% Options: t -- align text to the top of the frame
\documentclass[10pt,t]{beamer}
\mode<presentation>
\usetheme[style=\PresentationStyle]{jagiellonian}





% ------------------------------------------------------------------------------------
% Procesing configuration files of Jagiellonian theme located
% in the directory "preambule".
% ------------------------------------------------------------------------------------
% Configuration for polish language
% Need description
\usepackage[polish]{babel}
% Need description
\usepackage[MeX]{polski}



% ------------------------------
% Better support of polish chars in technical parts of PDF
% ------------------------------
\hypersetup{pdfencoding=auto,psdextra}

% Package "textpos" give as enviroment "textblock" which is very usefull in
% arranging text on slides.

% This is standard configuration of "textpos"
\usepackage[overlay,absolute]{textpos}

% If you need to see bounds of "textblock's" comment line above and uncomment
% one below.

% Caution! When showboxes option is on significant ammunt of space is add
% to the top of textblock and as such, everyting put in them gone down.
% We need to check how to remove this bug.

% \usepackage[showboxes,overlay,absolute]{textpos}



% Setting scale length for package "textpos"
\setlength{\TPHorizModule}{10mm}
\setlength{\TPVertModule}{\TPHorizModule}


% ---------------------------------------
% Packages written for lectures "Geometria 3D dla twórców gier wideo"
% ---------------------------------------
% \usepackage{./Geometry3DPackages/Geometry3D}
% \usepackage{./Geometry3DPackages/Geometry3DIndices}
% \usepackage{./Geometry3DPackages/Geometry3DTikZStyle}
% \usepackage{./ProgramowanieSymulacjiFizykiPaczki/ProgramowanieSymulacjiFizykiTikZStyle}
% \usepackage{./Geometry3DPackages/mathcommands}


% ---------------------------------------
% TikZ
% ---------------------------------------
% Importing TikZ libraries
\usetikzlibrary{arrows.meta}
\usetikzlibrary{positioning}





% % Configuration package "bm" that need for making bold symbols
% \newcommand{\bmmax}{0}
% \newcommand{\hmmax}{0}
% \usepackage{bm}




% ---------------------------------------
% Packages for scientific texts
% ---------------------------------------
% \let\lll\undefined  % Sometimes you must use this line to allow
% "amsmath" package to works with packages with packages for polish
% languge imported
% /preambul/LanguageSettings/JagiellonianPolishLanguageSettings.tex.
% This comments (probably) removes polish letter Ł.
\usepackage{amsmath}  % Packages from American Mathematical Society (AMS)
\usepackage{amssymb}
\usepackage{amscd}
\usepackage{amsthm}
\usepackage{siunitx}  % Package for typsetting SI units.
\usepackage{upgreek}  % Better looking greek letters.
% Example of using upgreek: pi = \uppi


\usepackage{calrsfs}  % Zmienia czcionkę kaligraficzną w \mathcal
% na ładniejszą. Może w innych miejscach robi to samo, ale o tym nic
% nie wiem.










% ---------------------------------------
% Packages written for lectures "Geometria 3D dla twórców gier wideo"
% ---------------------------------------
% \usepackage{./ProgramowanieSymulacjiFizykiPaczki/ProgramowanieSymulacjiFizyki}
% \usepackage{./ProgramowanieSymulacjiFizykiPaczki/ProgramowanieSymulacjiFizykiIndeksy}
% \usepackage{./ProgramowanieSymulacjiFizykiPaczki/ProgramowanieSymulacjiFizykiTikZStyle}





% !!!!!!!!!!!!!!!!!!!!!!!!!!!!!!
% !!!!!!!!!!!!!!!!!!!!!!!!!!!!!!
% EVIL STUFF
\if\JUlogotitle1
\edef\LogoJUPath{LogoJU_\JUlogoLang/LogoJU_\JUlogoShape_\JUlogoColor.pdf}
\titlegraphic{\hfill\includegraphics[scale=0.22]
{./JagiellonianPictures/\LogoJUPath}}
\fi
% ---------------------------------------
% Commands for handling colors
% ---------------------------------------


% Command for setting normal text color for some text in math modestyle
% Text color depend on used style of Jagiellonian

% Beamer version of command
\newcommand{\TextWithNormalTextColor}[1]{%
  {\color{jNormalTextFGColor}
    \setbeamercolor{math text}{fg=jNormalTextFGColor} {#1}}
}

% Article and similar classes version of command
% \newcommand{\TextWithNormalTextColor}[1]{%
%   {\color{jNormalTextsFGColor} {#1}}
% }



% Beamer version of command
\newcommand{\NormalTextInMathMode}[1]{%
  {\color{jNormalTextFGColor}
    \setbeamercolor{math text}{fg=jNormalTextFGColor} \text{#1}}
}


% Article and similar classes version of command
% \newcommand{\NormalTextInMathMode}[1]{%
%   {\color{jNormalTextsFGColor} \text{#1}}
% }




% Command that sets color of some mathematical text to the same color
% that has normal text in header (?)

% Beamer version of the command
\newcommand{\MathTextFrametitleFGColor}[1]{%
  {\color{jFrametitleFGColor}
    \setbeamercolor{math text}{fg=jFrametitleFGColor} #1}
}

% Article and similar classes version of the command
% \newcommand{\MathTextWhiteColor}[1]{{\color{jFrametitleFGColor} #1}}





% Command for setting color of alert text for some text in math modestyle

% Beamer version of the command
\newcommand{\MathTextAlertColor}[1]{%
  {\color{jOrange} \setbeamercolor{math text}{fg=jOrange} #1}
}

% Article and similar classes version of the command
% \newcommand{\MathTextAlertColor}[1]{{\color{jOrange} #1}}





% Command that allow you to sets chosen color as the color of some text into
% math mode. Due to some nuances in the way that Beamer handle colors
% it not work in all cases. We hope that in the future we will improve it.

% Beamer version of the command
\newcommand{\SetMathTextsColor}[2]{%
  {\color{#1} \setbeamercolor{math text}{fg=#1} #2}
}


% Article and similar classes version of the command
% \newcommand{\SetMathTextColor}[2]{{\color{#1} #2}}










% ---------------------------------------
% Commands for setting background pictures for some slides
% ---------------------------------------
\newcommand{\TitleBackgroundPicture}
{./PresentationPictures/CommonPictures/Cute_dragon_BG_dark.png}
\newcommand{\SectionBackgroundPicture}
{./PresentationPictures/CommonPictures/Cute_dragon_small_BG_light.png}



\newcommand{\TitleSlideWithPicture}{
  \begingroup

  \usebackgroundtemplate{ % \hspace*{-11.5em}
    \includegraphics[height=\paperheight]{\TitleBackgroundPicture}}

  \maketitle

  \endgroup
}





\newcommand{\SectionSlideWithPicture}[1]{%
  \begingroup

  \usebackgroundtemplate{ % \hspace*{-11.5em}
    \includegraphics[height=\paperheight]{\SectionBackgroundPicture}}

  \setbeamercolor{titlelike}{fg=normal text.fg}

  \section{#1}

  \endgroup
}





\newcommand{\EndingSlide}[1]{%
  \begin{frame}[standout]

    \begingroup

    \color{jFrametitleFGColor}

    #1

    \endgroup

  \end{frame}
}










% ------------------------------------------------------
% BibLaTeX
% ------------------------------------------------------
% Package biblatex, with biber as its backend, allow us to handle
% bibliography entries that use Unicode symbols outside ASCII.
\usepackage[
language=polish,
backend=biber,
style=alphabetic,
url=false,
eprint=true,
]{biblatex}

\addbibresource{Podstawy-informatyki-ETC-Bibliography.bib}





% ------------------------------------------------------
% Importing packages, libraries and setting their configuration
% ------------------------------------------------------
% Library improving positioning of nodes in graphs
\usetikzlibrary{positioning}





% ------------------------------------------------------
% Local packages
% ------------------------------------------------------
% Local configuration of this particular presentation
\usepackage{./Local-packages/local-settings}










% ------------------------------------------------------------------------------------------------------------------
\title{Podstawy informatyki z~językiem~C}
\subtitle{2. Zaczynamy programować w~języku~C}

\author{Kamil Ziemian \\
  \email}


% \date{}
% ------------------------------------------------------------------------------------------------------------------










% ####################################################################
% Beginning of the document
\begin{document}
% ####################################################################





% ######################################
% Number of chars: 37k+,
% Text is adjusted to the left and words are broken at the end of the line.
\RaggedRight
% ######################################





% ######################################
\maketitle
% ######################################





% ##################
\begin{frame}
  \frametitle{Spis treści}


  \tableofcontents

\end{frame}
% ##################










% ######################################
\section{Zanim zaczniemy}
% ######################################


% ##################
\begin{frame}
  \frametitle{Nagrywanie zajęć}


  Będziemy~się starać nagrywać każde nasze spotkanie
  na~\textsc{ms}~Teamsach. Proszę nam o~tym ciągle przypominać, bo kiedyś
  na pewno zapomnimy o~włączeniu nagrywania.

  Proszę zwracać nam też uwagę, że~na ekranie czegoś nie widać,
  że~czcionka jest za mała, że~kolory kłują w~oczy, że~nagrany dźwięk
  jest niskiej jakości,~etc. Zajęcia są dla Państwa, naszym obowiązkiem jest
  dostarczyć Państwu najlepszej jakości materiały do nauki jakie jesteśmy
  w~stanie stworzyć.

  Niestety, jakość dźwięku to coś, na co mamy mały wpływ. Możemy~się starać
  mówić możliwie blisko mikrofonu, ale raczej niewiele więcej.
  Poza tym, to na~pewno nie wyjdzie zbyt dobrze, bo te zajęcia  wymagają
  byśmy chodzili po sali, a~w~którymś momencie na pewno zapomnimy,
  że~spotkanie~się nagrywa. Swoje uwagi na temat jakości nagrań proszę
  kierować do ludzi odpowiedzialnych za~sprawy studenckie na
  \textsc{wsz}i\textsc{b}ie.

\end{frame}
% ##################





% ##################
\begin{frame}
  \frametitle{Nie chcemy by zajęcia były zbyt poważne, ale\ldots}

  \vspace{-0.5em}


  \begin{figure}

    \label{fig:Jak-to-bywa-na-zajeciach}

    \centering


    \includegraphics[scale=0.42]
    {./Presentations-pictures/Jak-to-bywa-na-zajeciach.jpeg}

  \end{figure}

\end{frame}
% ##################





% ##################
\begin{frame}
  \frametitle{Naprawdę cenimy Państwa zadanie}


  Jeśli ktoś ma uwagi do tych zajęć, propozycję co można zmienić,
  co~poprawić, to proszę powiedzieć to nam po zajęciach lub napisać pod
  adres \email. Ponownie prosimy by w~tytule emaila napisać
  \textit{Uwagi do zajęć} lub coś podobnego, bo inaczej może on zniknąć na
  długo w~skrzynce pocztowej.

  W~chwili obecnej nie mamy złudzeń, że~forma tych zajęć jest choćby bliski
  ideałowi, próbujemy jednak ciągle ją ulepszać. Państwa zdanie jest dla
  nas naprawdę \alert{ważne}, choć oczywiście nie zawsze~się z~nim zgadzamy.
  To jak ten przedmiot wygląda obecnie wynika częściowo z~tego, co
  w~przeszłości zaproponowali studenci.

  Choć trudno w~to uwierzyć, opinie typu „Te zajęcia to jest dno!”
  nie wpływają negatywnie na ocenę. Acz Państwa to raczej nie przekona
  do~komentowania ich w~e-mailach ;).

\end{frame}
% ##################










% ######################################
\section{Czym jest programowanie?}
% ######################################


% ##################
\begin{frame}
  \frametitle{Czym jest programowanie?}


  \textit{Programowanie to radzenie sobie ze złożonością: złożoność
    problemu, który chcemy rozwiązać, potęguje złożoność maszyny, na której
    pracujemy. Właśnie z~uwagi na złożoność większość projektów kończy~się
    niepowodzeniem.} \\
  Bruce Eckel \textit{Thinking in~Java. Edycja polska, wydanie~III},
  str.~$22$
  \parencite{Eckel-Thinking-in-Java-Ed-polska-Wyd-III-Pub-2003}.

  Istotą informatyki i~programowania jest rozwiązywanie problemów z~pomocą
  komputera lub za pomocą komputera i~ten cytat w~dużej mierze wyjaśnia,
  czemu informatyka może być trudna. Problemy które chcemy rozwiązać są
  złożone, zaś do ich rozwiązania używamy złożonych maszyn. I~nagle mamy dwa
  problemy, zamiast jednego ;).

  Będziemy~się upierać, że~rozwiązywanie problemu „z~pomocą komputera”
  i~„za pomocą komputera”, to są dwie różne rzeczy i~warto sobie zdawać
  z~tego sprawę. Gdy mówimy o~tym, że~jakiś problem rozwiązujemy „z~pomocą
  komputera”, to chodzi nam o~to, że~komputer jest narzędziem, które
  pomaga nam rozwiązać dany problem.

\end{frame}
% ##################





% ##################
\begin{frame}
  \frametitle{Rozwiązywanie problemów i~informatyka}


  Przykładowo, potrzebuję stworzyć pewien rysunek techniczny i~komputer
  może mi pomóc to zrobić, ale ja też będę~się musiał przy tym dużo
  napracować.

  Gdy chodzi o~problem, który można rozwiązać „za pomocą komputera”, jest to
  taki problem, który komputer może rozwiązać bez specjalnego wkładu
  z~naszej strony. Przykładowo, muszę obliczyć sumę jakiś $100$ liczb.
  Podaję je komputerowi i~on je sam zsumuje.

  Należy też pamiętać, że~nie każdy problem należy próbować rozwiązać
  za~pomocą komputera. Pomimo rozwoju \textsc{ai} jedną z~najważniejszy
  umiejętności, przynajmniej w~naszej ocenie, które dobry informatyk musi
  posiąść, jest rozeznanie, czy dany problem da~się rozwiązać za~pomocą
  komputera, czy lepiej dać sobie z~tym spokój? Odpowiedź na to pytanie nie
  zawsze będzie łatwa, prosta, ani krótka.

\end{frame}
% ##################





% ##################
\begin{frame}
  \frametitle{Rozwiązywanie problemów i~informatyka}


  Przykładowo, jeśli chodzi o~pytanie „Czy za pomocą komputera, można
  wygrywać mecze szachowe z~zawodowcami?”, to udzielenie na nie odpowiedzi
  zajęło kilka dekad. Pierwsze mecze szachowe człowiek-komputer to lata
  $50$-te XX wieku, a~mecz
  \colorhref{https://en.wikipedia.org/wiki/Deep_Blue_versus_Garry_Kasparov}
  {Deep Blue vs. Garri Kasparow}, który Kasparow przegrał to~$1997$~rok.

  Czy to wszystko jest takie ważne? Na tym przedmiocie, będziemy poznawali
  problemy, które da~się rozwiązać korzystając z~komputera, „z~i~za”, więc
  chyba dobrze jest mieć świadomość, że~nie każdy problem jest tego typu.

  Gdy chodzi o~\textsc{ai} to ona wygeneruje odpowiedź na wszystko, co do
  tego nie ma wątpliwości. Czy ta odpowiedź będzie cokolwiek warta, to
  jest prawdziwe pytanie. Wciąż próbujemy ocenić wartość korzystania
  z~\textsc{ai} (tego konkretnego typu). Acz zadawanie \textsc{ai} pytań
  typu „Czy \textit{Lalka} Prusa to jest cringe?”, jest na pewno bardzo
  zabawne~:D.

\end{frame}
% ##################





% ##################
\begin{frame}
  \frametitle{Rozwiązywanie problemów i~informatyka}


  Spróbujmy podsumować, to co powiedzieliśmy. Istotą informatyki jest
  rozwiązywanie pewnych problemów z~wykorzystaniem komputera. Programowanie
  polega na tym, by przekazać komputerowi zestaw instrukcji, które ten ma
  wykonać, by rozwiązać dany problem. Te stwierdzenia można rozwijać
  na~$1000$ sposobów, na początek to jednak zupełnie wystarczy.

  Tym co najbardziej chcielibyśmy Państwu przekazać na tych zajęciach,
  to umiejętność rozwiązywania pewnych problemów, za pomocą pisania
  programu, który używa zmiennych, instrukcji warunkowych i~pętli.
  Jeśli nie rozumieją Państwo o~co chodzi, to cały semestr będziemy
  próbowali to wyjaśnić. Państwo ocenią, czy to się nam udało, czy nie.

\end{frame}
% ##################










% ######################################
\section{Kilka uwag wstępnych}
% ######################################


% ##################
\begin{frame}
  \frametitle{Ostrzeżenia}


  Jednym z~podstawowych uproszczeń, które będziemy stosować na tym kursie,
  jest mówienie, że~język~C coś robi. Co to tak naprawdę znaczy, to temat
  na osobne spotkanie. Mamy nadzieję, że~ten sposób mówienia, będzie dla
  Państwa wystarczająco zrozumiały.

  Słowa takie jak „litość” czy „miłosierdzie” nie są komputerom specjalnie
  dobrze znane. Z~tego powodu, jeśli coś nie działa, to proszę sprawdzić,
  że~zrobili Państwo \alert{wszystko} dokładnie tak jak zostało podane
  i~wszystkie polecenia mają \alert{dokładnie} taką samą postać jak tu.
  Nawet najmniejszy błąd może wszystko zepsuć, bo komputer nie wie co to
  litość. Choć są programy mnie i~bardziej bezlitosne. Proszę też pamiętać,
  że~w~razie dowolnych trudności, jesteśmy do Państwa dyspozycji
  (również e-mailowo: \email).

  W~tych materiałach też mogą być jakieś błędy, bo jak wszyscy ludzie
  mylimy~się. Acz nie powinno być ich zbyt dużo. W~przypadku ich
  znalezienia, proszę napisać o~tym pod adres~\email.

\end{frame}
% ##################





% ##################
\begin{frame}
  \frametitle{Kilka uwag}


  W~naszej ocenie o~języku~C można myśleć jako o~lokalny osiłku. Jest
  gburowaty, humorzasty, trudny do zrozumienia, jeśli mu~się narazisz, może
  ci ostro dołożyć. Jednak warto mieć go po swojej stronie jako przyjaciela,
  choć nawiązanie z~nim przyjaźni nie jest łatwe.

  Powtórzmy to jeszcze raz. Najważniejszą rzeczą, jaką chcielibyśmy Państwu
  przekazać na tym kursie jest umiejętność rozwiązywania pewnych problemów,
  za pomocą pisania programów operujących na zmiennych, instrukcjach
  sterujących i~pętlach. Co to naprawdę znaczy, cały semestr spróbujemy
  poświęcić wyjaśnieniu tego.

\end{frame}
% ##################










% ######################################
\section{Zaczynamy programowanie w~języku~C}
% ######################################


% ##################
\begin{frame}
  \frametitle{Pisanie programów pod systemem GNU/Linux}


  Jak wyjaśniliśmy wcześniej, będziemy pracować pod systemem GNU/Linux,
  bo daje nam lepsze wsparcie przy programowaniu w~języku~C, niż system
  Windows. Poniżej podamy instrukcje jak utworzyć i~uruchomić nasz pierwszy
  program w~języku~C pod tym właśnie systemem, dla osób korzystających
  z~systemu Windows podamy pewne informacje trochę dalej.

  Zakładamy, że~obok systemu GNU/Linux na komputerze zainstalowane są też
  wszystkie potrzebne nam programy.

  Zaczniemy od podanie listy kroków, które należy wykonać, by utworzyć
  i~uruchomić prosty program napisany w~języku~C.

  \begin{enumerate}

  \item Otwarcie powłoki.

  \item Utworzenie katalogu \texttt{Podstawy-informatyki-Zajęcia/}.

  \item Wejście do katalogu \texttt{Podstawy-informatyki-Zajęcia/}.

  \item Utworzenie pliku \texttt{Hello-World.c} za pomocą edytora.

  \end{enumerate}

\end{frame}
% ##################





% ##################
\begin{frame}
  \frametitle{Pisanie programów pod systemem GNU/Linux}


  \begin{enumerate}

  \item[5.] Napisanie odpowiedniego kodu języka~C w~pliku
    \texttt{Hello-World.c}.

  \item[6.] Zapisanie pliku \texttt{Hello-World.c} w~edytorze.
    Proszę uwierzyć, nie zrobienie tego potrafi narobić problemów.

  \item[7.] Skompilowanie pliku \texttt{Hello-World.c} w~celu utworzenia
    pliku \texttt{prog.out}.

  \item[8.] Uruchomienie pliku \texttt{prog.out}.

  \end{enumerate}

  Teraz omówimy powoli wszystkie punkty tej procedury.

\end{frame}
% ##################





% ##################
\begin{frame}
  \frametitle{1. Otwarcie powłoki}


  By uruchomić powłokę w~systemie GNU/Linux, proszę najpierw wcisnąć
  \texttt{Ctrl-Alt-t} i~zobaczyć, czy~się otworzy. Jeśli to nie zadziała,
  to proszę pochodzić kursorem po ekranie, aż~znajdziemy okienko „Szukaj”,
  „Wyszukaj” lub jeszcze jakaś inna nazwa. Proszę tam wpisać „Terminal”,
  „Konsola” lub „Consol” i~zobaczyć, czy wyświetli~się odpowiednia ikona.

\end{frame}
% ##################





% ##################
\begin{frame}
  \frametitle{1.~Otwarcie powłoki}


  \begin{figure}

    \centering


    \includegraphics[scale=0.23]
    {./Presentations-pictures/BASH-shell.png}


    \caption{Przykładowy wygląd włączonej powłoki \textsc{bash},
      podstawowej powłoki w~systemie GNU/Linux.}


    \label{fig:BASH-shell}

  \end{figure}

\end{frame}
% ##################





% ##################
\begin{frame}
  \frametitle{Punkty 2 i~3}


  Przyjmujemy konwencję, że~jeśli linia zaczyna~się symbolem
  \texttt{\$} to oznacza ona polecenie (ang.~\textit{command}), które
  należy wpisać w~powłoce. Treść polecenia następuje po symbolu dolara,
  zaś wykonujemy je poprzez wciśnięcie klawisza \texttt{Enter}.

  Symbolu „\texttt{\$}” \alert{nie} przepisujemy do powłoki. Zwykle powłoka
  sama wyświetla go przed znakiem kursora, stąd pochodzi używana przez
  nas konwencja.

  By~utworzyć katalog o~nazwie \texttt{Podstawy-informatyki-Zajęcia/}
  w~powłoce wykonujemy polecenie \\
  \texttt{\$ mkdir Podstawy-informatyki-Zajęcia/} \\
  Katalog ten będzie istniał, dopóki ktoś go nie usunie, więc ponowne
  wykonywanie powyższego polecenia jest nie tylko zbyteczne, ale
  i~szkodliwe. A~przynajmniej irytujące dla użytkownika.

  % Aby wejść do omawianego katalogu używamy polecenia \\
  % \texttt{\$ cd Podstawy-informatyki-Zajęcia/} \\
  % Od tego momentu, wszystkie pliki które będziemy tworzyć na tych zajęciach,
  % powinny znajdować~się właśnie w~katalogu
  % \texttt{Podstawy-informatyki-Zajęcia/}. Dość logiczne.

\end{frame}
% ##################





% ##################
\begin{frame}
  \frametitle{Punkty 2 i~3}


  % Przyjmujemy konwencję, że~jeśli linia zaczyna~się symbolem
  % \texttt{\$} to oznacza ona polecenie (ang.~\textit{command}), które
  % należy wpisać w~powłoce. Treść polecenia następuje po symbolu dolara,
  % zaś wykonujemy je poprzez wciśnięcie klawisza \texttt{Enter}.

  % Symbolu „\texttt{\$}” \alert{nie} wpisujemy do powłoki. Zwykle powłoka
  % sama wyświetla go przed znakiem kursora, stąd pochodzi używana przez
  % nas konwencja.

  % By~utworzyć katalog o~nazwie \texttt{Podstawy-informatyki-Zajęcia/}
  % w~powłoce wykonujemy polecenie \\
  % \texttt{\$ mkdir Podstawy-informatyki-Zajęcia/} \\
  % Katalog ten będzie istniał, dopóki ktoś go nie usunie, więc ponowne
  % wykonywanie powyższego polecenia jest nie tylko zbyteczne, ale
  % i~szkodliwe. A~przynajmniej irytujące dla użytkownika.

  Aby wejść do omawianego katalogu używamy polecenia \\
  \texttt{\$ cd Podstawy-informatyki-Zajęcia/} \\
  Od tego momentu, wszystkie pliki które będziemy tworzyć na tych zajęciach,
  powinny znajdować~się właśnie w~katalogu
  \texttt{Podstawy-informatyki-Zajęcia/}. Dość logiczne.

\end{frame}
% ##################





% ##################
\begin{frame}
  \frametitle{Postęp technologiczny}

  \vspace{-0.5em}


  \begin{figure}

    \label{fig:Evolution-of-OS}

    \centering


    \includegraphics[scale=0.3]
    {./Presentations-pictures/Evolution-of-operating-systems.jpg}

  \end{figure}

\end{frame}
% ##################





% ##################
\begin{frame}
  \frametitle{4.~Utworzenie pliku \texttt{Hello-world.c}
    za pomocą\ldots}


  Utworzymy teraz plik \texttt{Hello-World.c} za pomocą edytora
  \colorhref{https://gedit-text-editor.org/}{gedit} lub
  \colorhref{https://kate-editor.org/}{kate}. By to zrobić w przypadku
  \colorhref{https://gedit-text-editor.org/}{gedit} wpisujemy w~powłoce \\
  \texttt{\$ gedit Hello-World.c \&} \\
  W przypadku edytora \colorhref{https://kate-editor.org/}{kate},
  wpisujemy \\
  \texttt{\$ kate Hello-World.c \&}

  Proszę pamiętać o~dodaniu na końcu polecenia znaku
  \alert{ampersand}:~„\&”. Inaczej edytor zablokuje nam możliwość wpisywania
  kolejnych poleceń do powłoki i~by ją odzyskać, będziemy musieli
  go wyłączyć albo użyć kombinacji poleceń \texttt{jobs} i~\texttt{bg}.

  Jeśli wszystko poszło dobrze, to na ekranie zobaczymy okienko
  \colorhref{https://gedit-text-editor.org/}{gedita}, które wygląda mniej
  więcej tak jak to na~następnym slajdzie \eqref{fig:Gedit-window}.
  Odpowiednie okienko dla \colorhref{https://kate-editor.org/}{kate}
  jest pokazane na slajdzie???.

\end{frame}
% ##################





% ##################
\begin{frame}
  \frametitle{Okienko gedita}


  \begin{figure}

    \centering

    \includegraphics[scale=0.18]
    {./Presentations-pictures/gedit-window.png}

    \caption{Otwarte okienko programu gedit.}


    \label{fig:Gedit-window}

  \end{figure}

\end{frame}
% ##################





% ##################
\begin{frame}
  \frametitle{5.~Napisanie odpowiedniego kodu języka~C\ldots}


  W~pliku \texttt{Hello-World.c} umieszczamy następujący tekst. Proszę
  przepisać go bardzo \alert{dokładnie}, bo język~C jest dość humorzasty
  i~nawet drobny błąd może uniemożliwić wykonanie tego programu.

  \vspace{2em}



  \texttt{\#include <stdio.h>} \\
  \vspace{0.8em}
  \texttt{int main() \{ } \\
  \hphantom{aaaa} \texttt{printf("Hello, World!\textbackslash n");} \\
  \vspace{0.8em}
  \vspace{0.8em}
  \vspace{0.8em}
  \vspace{0.8em}
  \hphantom{aaaa} \texttt{return 0;} \\
  \texttt{ \} }

\end{frame}
% ##################





% ##################
\begin{frame}
  \frametitle{6.~Zapisanie pliku \texttt{Hello-World.c}
    w~programie\ldots}


  Niezapisywanie właśnie utworzonych lub zmienionych plików, to
  nadspodziewanie częsty błąd, proszę więc być na niego wyczulonym.

  Plik w~naszych edytorach zapisujemy wciskając skrót~\texttt{Ctrl-s} lub
  wciskając przycisk zapisu. Dodatkowo, jeśli nie zapisaliśmy zmian
  wprowadzonych do pliku, to edytor umieści przy nazwie pliku widocznej na
  górze okienka symbol gwiazdki: „\hspace{0.1em}*”. Proszę~się upewnić, iż
  symbol ten jest tam nieobecny.

\end{frame}
% ##################





% ##################
\begin{frame}
  \frametitle{7.~Skompilowanie pliku\ldots}


  Na poziomie koncepcyjnym, kompilacja może być najtrudniejszą do
  zrozumienia częścią pisania programu w~C, dlatego zatrzymajmy~się nad
  nią na chwilę. W~naszym przypadku kompilacja polega na tym, że~uruchamiamy
  program zwany kompilatorem, który na podstawie zawartości pliku
  \texttt{Hello-World.c} wytwarza plik \texttt{prog.out}, który możemy
  uruchomić na komputerze.

  Nie możemy i~nie chcemy wnikać w~złożone kwestie techniczne, które
  przekraczają zakres \textit{Podstaw informatyki}, jednak kilka słów
  wyjaśnienia jest wskazanych. Plik \texttt{Hello-World.c} zawiera kod,
  który nazywamy \textbf{kodem źródłowym}, którego komputer nie potrafi
  bezpośrednio wykonać. Kompilator ma za zadanie wziąć ten plik i~utworzyć
  plik \texttt{prog.out}, który komputer już będzie w~stanie uruchomić.

  Nazwa „\texttt{prog.out}” pochodzi od angielskich słów \textit{program}
  i~\textit{output}, czyli „program wyjściowy”. Sama postać nazwa jest
  sprawą drugorzędną, ważne tylko by była zrozumiała dla użytkownika.

\end{frame}
% ##################




% ##################
\begin{frame}
  \frametitle{7.~Skompilowanie pliku\ldots}


  By skompilować plik w~powłoce wpisujemy polecenie \\
  \texttt{\$ gcc -ansi --std=c99 -pedantic Hello-World.c -o
    prog.out} \\
  Ten tekst może wyglądać jak magiczne zaklęcie napisane w~jakimś
  niestworzonym języku, takim jak reformowany egipski
  \colorhref{https://en.wikipedia.org/wiki/Joseph\_Smith}{Josepha
    Smitha~Jr.} Proszę jednak nam uwierzyć, że~poprawne wykonanie tego
  „zaklęcia” zrobi co trzeba.

  Powłoka \textsc{bash} już w~ogóle nie wie co to litość, więc proszę
  przepisać „zaklęcie” bardzo dokładnie. Jeśli ktoś napotkał
  tu jakikolwiek problem, proszę nas o~tym poinformować, pomożemy go
  rozwiązać.

  Jeśli wszystko~się udało, to na dysku powinien istnieć plik
  \texttt{prog.out}. W~systemie GNU/Linux by to sprawdzić możemy w~powłoce
  wpisać polecenie \texttt{ls}: \\
  \texttt{\$ ls} \\
  Oto co prawdopodobnie zobaczymy: \\
  \texttt{Hello-World.c \quad prog.out} \\

\end{frame}
% ##################





% ##################
\begin{frame}
  \frametitle{7.~Skompilowanie pliku\ldots}


  Jeśli po wpisaniu polecenia \texttt{ls} nie widzimy pliku
  \texttt{prog.out}, to znaczy, że~coś poszło nie tak. W~programowaniu
  zawsze~$100$ może pójść nie tak, nie można więc z~góry powiedzieć,
  co tym razem było źle. Proszę nas poinformować o~tym, pomożemy
  rozwiązać ten problem.

\end{frame}
% ##################





% ##################
\begin{frame}
  \frametitle{8.~Uruchomienie pliku \texttt{prog.out}}


  By uruchomić program \texttt{prog.out} w~powłoce wpisujemy \\
  \texttt{\$ ./prog.out} \\
  Jeśli wszystko poszło dobrze, to w~powłoce zobaczymy tekst \\
  \texttt{Hello, World!}

\end{frame}
% ##################










% ######################################
\section{Programowanie w~języku~C pod systemem Windows}
% ######################################



% ##################
\begin{frame}
  \frametitle{Programowanie w~języku~C pod systemem Windows}


  W~systemie Windows istnieje kilka środowisk, umożliwiających programowanie
  w~języku~C. Jednym z~nich jest
  \colorhref{https://www.codeblocks.org}{Code::Blocks}, którego instrukcje
  instalacji można znaleźć na początku tego
  \colorhref{https://www.youtube.com/watch?v=KJgsSFOSQv0}{widea}
  \parencite{freeCodeCamp-org-C-Programming-Tutorial-for-ETC-Ver-2018}.

  Istnieje też możliwość zdalnego zalogowania~się na komputer uczelniany
  i~zdalne programowanie pod GNU/Linuxem, dzięki systemowi Artemis.
  Chcielibyśmy bardzo podziękować panu Tomaszowi Molęckiemu z~działu
  \textsc{it} \textsc{wsz}i\textsc{b}u za~udzielenie nam informacji na temat
  procedury logowania~się do Artemisa.

  Pierwszy krok polega na pobraniu programu
  \colorhref{https://winscp.net/eng/download.php}{Win\textsc{scp}}. Gdy
  już mamy go dostępnego, znajdujemy okno logowania, które wypełniamy
  zgodnie ze~zdjęciem widocznym na następnym slajdzie. Dane logowania
  wpisujemy takie same jak dla systemu \textsc{suszi}. Gdy to jest gotowe,
  klikamy na przycisk logowania. W~razie problemów proszę pisać do nas
  (\email) lub pana Molęckiego.

\end{frame}
% ##################





% ##################
\begin{frame}
  \frametitle{Korzystanie z~systemu Artemis}

  \vspace{-0.5em}


  \begin{figure}

    \centering


    \includegraphics[scale=0.45]
    {./Presentations-pictures/Artemis.jpg}


    \caption{Logowanie~się do systemu Artemis za pomocą programu
      Win\textsc{scp}.}


    \label{fig:Artemis-system}

  \end{figure}

\end{frame}
% ##################










% ######################################
\section{Kilka ważnych informacji o~programowaniu
  w~języku~C}
% ######################################


% ##################
\begin{frame}
  \frametitle{Kilka uwag o~kompilacji}


  Wyjaśnimy tutaj pewien problem, na który często natykają~się
  początkujący programiści. Powiedzmy, że~chcemy, aby właśnie omówiony
  program zamiast wypisywać na ekranie \\
  \texttt{Hello, World!} \\
  wypisywał \\
  \texttt{Hello, World! What a~nice day.} \\
  W~tym celu musimy linię \\
  \texttt{printf("Hello, World!\textbackslash n");} \\
  zastąpić linią \\
  \texttt{printf("Hello, World! What a~nice day.\textbackslash n");}

  Robimy to, uruchamiamy nasz program \\
  \texttt{\$ ./prog.out} \\
  i~widzimy \\
  \texttt{Hello, World!} \\

\end{frame}
% ##################





% ##################
\begin{frame}
  \frametitle{Kilka uwag o~kompilacji}


  Problem polega na tym, że~\texttt{prog.out} powstał w~wyniku kompilacji
  poprzedniej wersji pliku \texttt{Hello-World.c}. Przypomnijmy,
  że~kompilacja polega na wzięciu pliku \texttt{Hello-World.c} i~utworzeniu
  na jego podstawie pliku \texttt{prog.out}, który komputer potrafi
  uruchomić.

  W~tej chwili potrzebujemy skompilować nową wersję pliku
  \texttt{Hello-World.c}, w~wyniku czego stary plik \texttt{prog.out}
  zostanie skasowany, a~powstanie nowy o~tej samej nazwie, wypisujący nowy
  tekst na ekranie. W~tym celu wpisujemy w~powłoce \\
  \texttt{\$ gcc -ansi --std=c99 -pedantic Hello-World.c -o
    prog.out} \\
  Wykonujemy \alert{nowy} program \\
  \texttt{\$ ./prog.out} \\
  i~na ekranie widzimy \\
  \texttt{Hello, World! What a~nice day.}

\end{frame}
% ##################





% ##################
\begin{frame}
  \frametitle{Kilka uwag o~kompilacji}


  Wobec tego, jeśli nasz program z~niezrozumiałych powodów nie chce
  działać poprawnie, to należy sprawdzić, czy nie zapomnieliśmy
  skompilować nowego kodu źródłowego i~przypadkiem nie uruchamiamy wciąż
  starego programu. Nigdy też nie zaszkodzi sprawdzić, czy nie
  zapomnieliśmy zapisać kodu w~edytorze.

  Kompilację i~wykonanie programu można wykonać za pomocą pojedynczego
  polecenia powłoki: \\
  \texttt{\$ time gcc -ansi --std=c99 -pedantic Hello-World.c} \\
  \hphantom{aaaa} \texttt{-o prog.out; ./prog.out} \\
  Tekst ten należy wpisać w~\alert{jednym} wierszu powłoki. My rozbiliśmy go
  na~dwa wiersze, tylko po to by był bardziej czytelny.

  Jeśli chcemy skompilować plik o~innej nazwie, powiedzmy, że~jest nią
  \texttt{Fajny-program.c}, to używamy polecenia \\
  \texttt{\$ gcc -ansi --std=c99 -pedantic Fajny-program.c -o prog.out}

\end{frame}
% ##################





% ##################
\begin{frame}
  \frametitle{Komentarze w~języku C}


  Język~C jak każdy normalny język programowania, pozwala pisać komentarze.
  \textbf{Komentarz} to tekst przeznaczony dla człowieka, który język~C
  całkowicie ignoruje. W~języku C komentarz ma postać \\
  \texttt{/* Informacje dla człowieka. */} \\
  Komentarz może zawierać jedną lub więcej linii tekstu: \\
  \texttt{/* Dość długi tekst z informacją dla człowieka, dlatego} \\
  \hspace{0.2em} \texttt{*} \hspace{0.12em} \texttt{został podzielony na
    dwie linie. */}

  Mogą~się Państwo spotkać z~komentarzem postaci \\
  \texttt{// Informacje dla człowieka.} \\
  Nie ma jednak gwarancji, że~program zawierający taki komentarz będzie
  działał. Wynika to z~tego, że~komentarz zaczynający~się od~„\texttt{//}”
  jest poprawny w~języku~C++, blisko związanym z~C. Jednak dokument
  organizacji standardów \textsc{iso} stwierdza, że~jedynym poprawny
  komentarz w~języku C ma postać \\
  \texttt{/* Treść komentarza. */}

\end{frame}
% ##################





% ##################
\begin{frame}
  \frametitle{Komentarze w~języku~C}


  W~praktyce, C często przymyka oko i~akceptuje komentarz, który zaczyna~się
  od symbolu~„\texttt{//}”, \alert{nie} ma jednak gwarancji, że~tak zrobi.
  Z~tego powodu, na tych zajęciach dopuszczony jest tylko i~wyłącznie
  komentarz postaci \\
  \texttt{/* Treść komentarza. */}

  Proszę zwrócić uwagę, że komentarz \\
  \texttt{/* Informacje dla człowieka. */} \\
  jest poprawny. Za to komentarz \\
  \texttt{/ * Informacje dla człowieka. */} \\
  jest błędem, który zepsuje cały program.

  Proszę zwrócić uwagę na różnicę między „\texttt{/*}” a~„\texttt{/ *}”.
  Jedna głupia spacja. Bo czemu życie miałoby być proste?

\end{frame}
% ##################










% ######################################
\section{Co może pójść nie tak w~języku~C?}
% ######################################


% ##################
\begin{frame}
  \frametitle{Co może pójść nie tak?}


  \textit{Kompilatory mają ostrzegać o~najczęstszych typach błędów,
    wyeliminowano także automatyczne przekształcenia danych o~niezgodnych
    typach. Niemniej jednak język~C zachował swoją zasadniczą filozofię,
    według której programista wie co robi; wymaga~się jedynie, aby jawnie
    sformułował swoje zamiary.} \\
  \colorhref{https://en.wikipedia.org/wiki/Brian\_Kernighan}{Brian
    W.~Kernighan},
  \colorhref{https://en.wikipedia.org/wiki/Dennis\_Ritchie}{Dennis
    M.~Ritchie}, \textit{Język ANSI~C}, popularnie
  \textit{K\&R}, str.~20,
  \parencite{Kernighan-Ritchie-Jezyk-ANSI-C-Pub-2004}.

  Przetłumaczy te słowa mistrzów języka~C, na język zrozumiały dla
  normalnego człowieka. \\
  --~Co w~języku~C może pójść nie tak? \\
  --~W~skrócie, wszystko.

  Dokładniej, program w~C może~się nie chcieć dać w~ogóle uruchomić,
  z~następujących powodów.

  \vspace{-0.5em}



  \begin{itemize}

  \item Brak średnika.

  \item Obecność spacji w~pewnym miejscu.

  \item Brak spacji w~pewnym miejscu.

  \end{itemize}

\end{frame}
% ##################





% ##################
\begin{frame}
  \frametitle{Co może pójść nie tak?}


  \begin{itemize}

  \item Duża litera zamiast małej.

  \item Mała litera zamiast dużej.

  \item Przecinek zamiast średnika.

  \item Średnika zamiast przecinka.

  \item Przecinek zamiast kropki.

  \item Kropka zamiast przecinka.

  \item Jeden nawias za mało.

  \item Jeden nawias za dużo.

  \item Źle umieszczone nawiasy.

  \item etc.

  \end{itemize}



  Gdy zaczynamy naukę programowania, w~C albo innym języku, to wiele czasu
  spędzamy na uczeniu~się unikaniu tego typu błędów oraz nauce korzystania
  z~narzędzi, które nam walkę z~nimi ułatwiają. Przykładowo, edytory
  tekstu takie jak \colorhref{https://gedit-text-editor.org/}{gedit}
  czy \colorhref{https://kate-editor.org/}{kate}, dostarczają nam od razu
  narzędzi takich jak kolorowanie składni.

\end{frame}
% ##################





% ##################
\begin{frame}
  \frametitle{Nie należy tracić nadziei}


  Nie należy~się zniechęcać tym, że~na początku nic nie chce nam działać
  z~wyżej wymienionych powodów i~innych im podobnych. Im więcej mamy
  doświadczenia w~programowaniu, tym rzadziej napotykamy tego typu
  problemy, aż~stają~się one dla nas prawie nieistotne. Zmaganie~się z~nimi
  to niestety nieunikniona część nauki programowania w~C i~dowolnym innym
  języku.

  Język~C zostawia nam pewną swobodę, w~wyborze tego jak dokładnie piszemy
  nasz kod, dopuszczając różne \alert{style} jego tworzenia. Na tych
  zajęciach prezentujemy styl, który uważamy za~jeden z~lepszych, acz jest
  to kwestia gustu.

  Niestety, kryteria decydujące o~tym, co jest kwestią stylu programowania,
  a~co jest błędem, który całkowicie położy nasz program, nie są całkiem
  oczywiste. Będziemy starali~się wyjaśniać to powoli wraz z~kolejnymi
  zajęciami.

\end{frame}
% ##################









% ######################################
\section{Kilka ważnych informacji}
% ######################################


% ##################
\begin{frame}
  \frametitle{Rozszerzenia nazw plików}


  Z~pojęciem rozszerzenia nazwy pliku na pewno już~się Państwo spotkali.
  Dobrze nazwany plik zawiera w~swojej nazwie jedną i~tylko jedną kropkę,
  zaś rozszerzenie nazwy tego pliku, to ta właśnie kropka wraz z~całym
  tekstem następującym po~niej. Wobec tego jeśli plik nosi nazwę
  \texttt{Opowiadanie.txt}, to rozszerzeniem nazwy tego pliku jest
  \texttt{.txt}. Analogicznie dla pliku o~nazwie \texttt{Hello-World.c}
  jego rozszerzenie to~\texttt{.c}.

  Rozszerzenie nazwy pliku informuje programy z~jakim rodzajem pliku mają do
  czynienia, dlatego każdy język programowania posiada własne, właściwe
  tylko jemu rozszerzenie. Jak już widzieliśmy, pliki napisane w~języku~C
  mają mieć rozszerzenie~\texttt{.c}. Analogicznie, każdy program w~języku
  C++ ma się kończyć na \texttt{.cpp} (\texttt{Hello-World.cpp}),
  program napisany w~Pythonie na~\texttt{.py} (\texttt{Hello-World.py}),
  w~\textsc{bash}u \texttt{.sh} (\texttt{Hello-World.sh}),~etc.

\end{frame}
% ##################





% ##################
\begin{frame}
  \frametitle{Rozszerzenia nazw plików}


  Rozszerzenie nazwy pliku to nie jest jedyny sposób sprawdzenia z~jakiego
  rodzajem pliku mamy do czynienia, ale jest jednym z~najważniejszych.
  W~szczególności, wiele programów jak
  \colorhref{https://gedit-text-editor.org/}{gedit} czy
  \colorhref{https://kate-editor.org/}{kate} na podstawie rozszerzenia
  nazwy pliku włączają nam odpowiednie narzędzia pomocnicze do programowania
  w~języku takim jak~C. Najbardziej widocznym tego typu narzędziem
  jest kolorowanie składni.

  Nie trzeba chyba dodawać, że~pozbawianie~się tego typu pomocy przy
  pisaniu programów w~języku~C jest wielkim błędem. Dlatego jeśli
  ktoś na zajęciach napisze program w~języku~C, którego plik \alert{nie} ma
  rozszerzenia~\texttt{.c}, będzie to musiał przy mnie poprawić.
  Chyba, że~ograniczenia czasowe sprawią, że nie będę mógł stać nad taką
  osobą i~nadzorować jak to robi. Ale proszę nie liczyć, że~zapomnę o~tej
  zniewadze~;).

  Przypadek plików nagłówkowych w~C omówimy później.

\end{frame}
% ##################





% ##################
\begin{frame}
  \frametitle{Plugawe nazwy plików}


  To wyjaśnia, czemu nazwy plików typu \\
  \texttt{plik-numer.1.c} \\
  to wymysł Szatana i~wytwór piekieł. Programy z~których korzystamy
  mogą bowiem błędnie rozpoznać \texttt{.1.c} jako rozszerzenie pliku
  i~pozbawić nas wsparcia przy pisaniu programów w~C. To samo zresztą
  stosuje~się do wszystkich innych języków programowania.

  Podsumujmy. W~nazwie pliku powinna być \alert{jedna i~tylko jedna
    kropka}, ta poprzedzająca nazwę rozszerzenia pliku, czyli dla nas
  \texttt{.c}. Jeśli ta zasada nie jest prawdą wszędzie na świecie, to na
  pewno jest prawdą na tych zajęciach i~proszę o~tym pamiętać.

  Jeśli ktoś prześle nam plik, którego nazwa zawiera więcej lub mniej niż
  \alert{jedną} kropkę, to zastrzegamy sobie prawo odesłania mu tego pliku
  z~prośbą o~zmianę nazwy na~poprawniejszą. To nie jest forma karania
  takiej osoby, tylko metoda wyrabiania w~ludziach dobrych nawyków.

\end{frame}
% ##################





% ##################
\begin{frame}
  \frametitle{Nieszczęsna funkcja \texttt{scanf()}}


  W~języku C dostępna jest funkcja \texttt{scanf()}, która pozwala wczytywać
  dane z~klawiatury. Funkcji tej można używać ucząc~się języka~C, ale poza
  tym wyjątkiem \alert{nigdy}, \alert{pod żadnym pozorem}, \alert{nie wolno}
  jej używać pisząc jakikolwiek praktyczny program. Zwłaszcza, jeśli
  program ten ma działać na komputerze podłączonym do~internetu.

  Problem z~funkcją \texttt{scanf()} jest taki, że~pozwala ona hakerom
  w~prosty sposób włamać~się do programu, za pomocą ataku przez
  przeładowania bufora. O~tym na czym polega ten atak, powiemy sobie trochę
  później, zainteresowani mogą już teraz znaleźć jego objaśnienie
  w~kilku wideach z~kanału Low Level:
  \colorhref{https://www.youtube.com/watch?v=fjMrDDj47E8}{\textit{Why
      do hackers\ldots}}
  \parencite{Low-Level-Why-do-hackers-love-strings-Ver-2022},
  \colorhref{https://www.youtube.com/watch?v=qpyRz5lkRjE}
  {\textit{\textit{How do hackers\ldots}}}
  \parencite{Low-Level-How-do-hackers-exploit-buffers-that-are-too-ETC-Ver-2022},
  \colorhref{https://www.youtube.com/watch?v=z6gdQt8mjn4}
  {\textit{What ever happened\ldots}}
  \parencite{Low-Level-What-ever-happened-to-buffer-overflows-Ver-2023}.

  Prawdopodobnie każdy z~nas dzisiaj korzystał z~kilku programów napisanych
  w~języku~C, które korzystają z~\texttt{scanf()} i~to jest bardzo
  poważny \alert{problem}. Należy dodać, że~jest \alert{więcej} funkcji
  w~języku~C, który \alert{nie wolno} używać. Jest to efekt tego, jak stary
  jest język~C.

\end{frame}
% ##################





% ##################
\begin{frame}
  \frametitle{C a~C++}


  Jaka jest różnica między C a~C++? Jak między lodami czekoladowymi
  a~lodołamaczem. Oba mają do czynienia z~lodem, ale w~różny sposób.
  Tak samo C i~C++ mają ze sobą wiele wspólnego, ale też różnią~się między
  sobą w~ogromnym stopniu.

  Dla porównania, w~pliku \texttt{PI-05-G-Hello-World.cpp} mamy program
  „Hello, World!” napisany w~języku C++. Proszę go porównać z~tym samym
  programem napisanym w~języku~C. By go uruchomić proszę w~powłoce wpisać \\
  \texttt{\$ time g++ PI-05-G-Hello-World.cpp -o prog.out; ./prog.out}

\end{frame}
% ##################










% % ##################
% \jagiellonianendslide{Czy są jakieś pytania do tej części?}
% % ##################










% ####################################################################
% ####################################################################
% Bibliography

\printbibliography





% ############################
% End of the document

\end{document}
