% ------------------------------------------------------------------------------------------------------------------
% Basic configuration of Beamera class and Jagiellonian theme
% ------------------------------------------------------------------------------------------------------------------
\RequirePackage[l2tabu, orthodox]{nag}



\ifx\PresentationStyle\notset
  \def\PresentationStyle{dark}
\fi



\documentclass[10pt,t]{beamer}
\mode<presentation>
\usetheme[style=\PresentationStyle]{jagiellonian}




% ------------------------------------------------------------------------------------
% Procesing configuration files of Jagiellonian theme loceted in
% the directory "preambule"
% ------------------------------------------------------------------------------------
% Configuration for polish language
% Need description
\usepackage[polish]{babel}
% Need description
\usepackage[MeX]{polski}



% ------------------------------
% Better support of polish chars in technical parts of PDF
% ------------------------------
\hypersetup{pdfencoding=auto,psdextra}

% Package "textpos" give as enviroment "textblock" which is very usefull in
% arranging text on slides.

% This is standard configuration of "textpos"
\usepackage[overlay,absolute]{textpos}

% If you need to see bounds of "textblock's" comment line above and uncomment
% one below.

% Caution! When showboxes option is on significant ammunt of space is add
% to the top of textblock and as such, everyting put in them gone down.
% We need to check how to remove this bug.

% \usepackage[showboxes,overlay,absolute]{textpos}



% Setting scale length for package "textpos"
\setlength{\TPHorizModule}{10mm}
\setlength{\TPVertModule}{\TPHorizModule}


% ---------------------------------------
% Packages written for lectures "Geometria 3D dla twórców gier wideo"
% ---------------------------------------
% \usepackage{./Geometry3DPackages/Geometry3D}
% \usepackage{./Geometry3DPackages/Geometry3DIndices}
% \usepackage{./Geometry3DPackages/Geometry3DTikZStyle}
% \usepackage{./ProgramowanieSymulacjiFizykiPaczki/ProgramowanieSymulacjiFizykiTikZStyle}
% \usepackage{./Geometry3DPackages/mathcommands}


% ---------------------------------------
% TikZ
% ---------------------------------------
% Importing TikZ libraries
\usetikzlibrary{arrows.meta}
\usetikzlibrary{positioning}





% % Configuration package "bm" that need for making bold symbols
% \newcommand{\bmmax}{0}
% \newcommand{\hmmax}{0}
% \usepackage{bm}




% ---------------------------------------
% Packages for scientific texts
% ---------------------------------------
% \let\lll\undefined  % Sometimes you must use this line to allow
% "amsmath" package to works with packages with packages for polish
% languge imported
% /preambul/LanguageSettings/JagiellonianPolishLanguageSettings.tex.
% This comments (probably) removes polish letter Ł.
\usepackage{amsmath}  % Packages from American Mathematical Society (AMS)
\usepackage{amssymb}
\usepackage{amscd}
\usepackage{amsthm}
\usepackage{siunitx}  % Package for typsetting SI units.
\usepackage{upgreek}  % Better looking greek letters.
% Example of using upgreek: pi = \uppi


\usepackage{calrsfs}  % Zmienia czcionkę kaligraficzną w \mathcal
% na ładniejszą. Może w innych miejscach robi to samo, ale o tym nic
% nie wiem.










% ---------------------------------------
% Packages written for lectures "Geometria 3D dla twórców gier wideo"
% ---------------------------------------
% \usepackage{./ProgramowanieSymulacjiFizykiPaczki/ProgramowanieSymulacjiFizyki}
% \usepackage{./ProgramowanieSymulacjiFizykiPaczki/ProgramowanieSymulacjiFizykiIndeksy}
% \usepackage{./ProgramowanieSymulacjiFizykiPaczki/ProgramowanieSymulacjiFizykiTikZStyle}





% !!!!!!!!!!!!!!!!!!!!!!!!!!!!!!
% !!!!!!!!!!!!!!!!!!!!!!!!!!!!!!
% EVIL STUFF
\if\JUlogotitle1
\edef\LogoJUPath{LogoJU_\JUlogoLang/LogoJU_\JUlogoShape_\JUlogoColor.pdf}
\titlegraphic{\hfill\includegraphics[scale=0.22]
{./JagiellonianPictures/\LogoJUPath}}
\fi
% ---------------------------------------
% Commands for handling colors
% ---------------------------------------


% Command for setting normal text color for some text in math modestyle
% Text color depend on used style of Jagiellonian

% Beamer version of command
\newcommand{\TextWithNormalTextColor}[1]{%
  {\color{jNormalTextFGColor}
    \setbeamercolor{math text}{fg=jNormalTextFGColor} {#1}}
}

% Article and similar classes version of command
% \newcommand{\TextWithNormalTextColor}[1]{%
%   {\color{jNormalTextsFGColor} {#1}}
% }



% Beamer version of command
\newcommand{\NormalTextInMathMode}[1]{%
  {\color{jNormalTextFGColor}
    \setbeamercolor{math text}{fg=jNormalTextFGColor} \text{#1}}
}


% Article and similar classes version of command
% \newcommand{\NormalTextInMathMode}[1]{%
%   {\color{jNormalTextsFGColor} \text{#1}}
% }




% Command that sets color of some mathematical text to the same color
% that has normal text in header (?)

% Beamer version of the command
\newcommand{\MathTextFrametitleFGColor}[1]{%
  {\color{jFrametitleFGColor}
    \setbeamercolor{math text}{fg=jFrametitleFGColor} #1}
}

% Article and similar classes version of the command
% \newcommand{\MathTextWhiteColor}[1]{{\color{jFrametitleFGColor} #1}}





% Command for setting color of alert text for some text in math modestyle

% Beamer version of the command
\newcommand{\MathTextAlertColor}[1]{%
  {\color{jOrange} \setbeamercolor{math text}{fg=jOrange} #1}
}

% Article and similar classes version of the command
% \newcommand{\MathTextAlertColor}[1]{{\color{jOrange} #1}}





% Command that allow you to sets chosen color as the color of some text into
% math mode. Due to some nuances in the way that Beamer handle colors
% it not work in all cases. We hope that in the future we will improve it.

% Beamer version of the command
\newcommand{\SetMathTextsColor}[2]{%
  {\color{#1} \setbeamercolor{math text}{fg=#1} #2}
}


% Article and similar classes version of the command
% \newcommand{\SetMathTextColor}[2]{{\color{#1} #2}}










% ---------------------------------------
% Commands for setting background pictures for some slides
% ---------------------------------------
\newcommand{\TitleBackgroundPicture}
{./PresentationPictures/CommonPictures/Cute_dragon_BG_dark.png}
\newcommand{\SectionBackgroundPicture}
{./PresentationPictures/CommonPictures/Cute_dragon_small_BG_light.png}



\newcommand{\TitleSlideWithPicture}{
  \begingroup

  \usebackgroundtemplate{ % \hspace*{-11.5em}
    \includegraphics[height=\paperheight]{\TitleBackgroundPicture}}

  \maketitle

  \endgroup
}





\newcommand{\SectionSlideWithPicture}[1]{%
  \begingroup

  \usebackgroundtemplate{ % \hspace*{-11.5em}
    \includegraphics[height=\paperheight]{\SectionBackgroundPicture}}

  \setbeamercolor{titlelike}{fg=normal text.fg}

  \section{#1}

  \endgroup
}





\newcommand{\EndingSlide}[1]{%
  \begin{frame}[standout]

    \begingroup

    \color{jFrametitleFGColor}

    #1

    \endgroup

  \end{frame}
}










% ------------------------------------------------------
% BibLaTeX
% ------------------------------------------------------
% Package biblatex, with biber as its backend, allow us to handle
% bibliography entries that use Unicode symbols outside ASCII.
\usepackage[
language=polish,
backend=biber,
style=alphabetic,
url=false,
eprint=true,
]{biblatex}

\addbibresource{Podstawy-informatyki-ETC-Bibliography.bib}





% ------------------------------------------------------
% Importing packages, libraries and setting their configuration
% ------------------------------------------------------





% ------------------------------------------------------
% Special configuration for this particular presentation
% ------------------------------------------------------










% ------------------------------------------------------------------------------------------------------------------
\title{Podstawy informatyki z~językiem~C}
\subtitle{Informacje i~opinie o~języku~C}

\author{Kamil Ziemian}


% \date{}
% ------------------------------------------------------------------------------------------------------------------










% ####################################################################
% Beginning of the document
\begin{document}
% ####################################################################





% ######################################
% Text is adjusted to the left and words are broken at the end of the line.
\RaggedRight
% Number of chars: 62k+, 15k+,
% ######################################





% ######################################
\maketitle
% ######################################





% ##################
\begin{frame}
  \frametitle{Spis treści}


  \tableofcontents

\end{frame}
% ##################





% ######################################
\section{Informacje ogólne}
% ######################################








% ##################
\begin{frame}
  \frametitle{Kilka uwag}


  Na tych zajęciach \textit{nie} nauczymy~się jak programować. Jak dobrze
  pójdzie to nauczymy~się podstaw programowania w~języku~C, ale
  programowanie obejmuje tyle zagadnień i~wymaga tyle praktyki, iż~nie
  ma najmniejszych szans, że~się tego wszystkiego tu nauczymy.

  Co, na Państwa nieszczęście, nie oznacza, że~będzie mało materiału.

  Ponieważ tematyka którą poruszamy jest niebanalna, więc mnóstwo rzeczy
  będę musiał bardzo upraszczać. Proszę mieć to na uwadze.

\end{frame}
% ##################










% ######################################
\section{O~języku~C}
% ######################################



% ##################
\begin{frame}
  \frametitle{Skąd~się wziął i~czym jest język~C?}


  \textbf{Uwaga.} Tekst książka Kernighana i~Ritchiego (K\&R) została
  pochodzi z~roku 1989, od tego czasu wiele~się w~informatyce i~na świecie
  zmieniło. Proszę mieć to na uwadze.

  \vspace{1em}





  Język~C jest językiem ogólnego stosowania. Był on ściśle związany
  z~systemem Unix, dla którego został opracowany, ponieważ zarówno sam
  system operacyjny, jak i~większość programów działających pod jego
  kontrolą napisano w~C. Jednak~C nie jest przywiązany do~żadnego
  systemu operacyjnego lub maszyny. Wprawdzie nazwano go~„językiem
  programowania systemowego”, jest bowiem wygodnym narzędziem do
  konstruowania kompilatorów i~systemów operacyjnych, ale okazał~się,
  że~nadawał~się równie dobrze do napisania ważniejszych programów
  z~wielu różnych dziedzin.

  Brian W.~Kernighan, Dennis M.~Ritchie, \textit{Język ANSI C}, K\&R,
  str.~17, \parencite{Kernighan-Ritchie-Jezyk-ANSI-C-Pub-2004}.

  % \vspace{2em}

\end{frame}
% ##################





% ##################
\begin{frame}
  \frametitle{C, język niskiego czy wysokiego poziomu?}


  C~jest językiem względnie „niskiego poziomu”. Określenie to nie zawiera
  w~sobie nic negatywnego; po prostu znaczy tyle, że~C posługuje~się tym
  rodzajem obiektów, co~większość komputerów, a~więc znakami, liczbami
  i~adresami. Obiekty te mogą być łączone lub przemieszczane za~pomocą
  zwykłych operacji arytmetycznych i~logicznych dostępnych na~istniejących
  maszynach.

  Brian W.~Kernighan, Dennis M.~Ritchie, \textit{Język ANSI C}, K\&R,
  str.~18, \parencite{Kernighan-Ritchie-Jezyk-ANSI-C-Pub-2004}.

\end{frame}
% ##################





% ##################
\begin{frame}
  \frametitle{Czego w~języku C nie ma?}


  W~języku~C nie istnieją operacje na~takich obiektach złożonych, jak~ciągi
  znaków, zbiory, listy czy tablice. Nie ma operacji do~manipulowania całą
  tablicą czy napisem, chociaż struktury mogą być kopiowane w~całości.
  Język nie daje innych możliwości przydziału pamięci niż definicja
  statyczna i~obsługa stosu realizowana przez zmienne lokalne funkcji; nie
  ma takich pojęć, jak sterta (ang. \textit{heap}) czy odśmiecania
  (ang. \textit{garbage collection}). I~wreszcie, sam język~C nie dostarcza
  narzędzi wejścia-wyjścia: nie ma instrukcji READ (czytaj) lub WRITE (pisz)
  ani wbudowanych metod dostępu do~pliku. Są~to mechanizmy wyższego poziomu
  i~muszą być wprowadzone do~języka przez jawnie wywoływane funkcje.
  W~większości instalacji~C występuje sensowny standardowy zestaw takich
  funkcji.

  Brian W.~Kernighan, Dennis M.~Ritchie, \textit{Język ANSI~C}, K\&R,
  str.~18, \parencite{Kernighan-Ritchie-Jezyk-ANSI-C-Pub-2004}.

\end{frame}
% ##################





% ##################
\begin{frame}
  \frametitle{Czego w~języku C nie ma?}


  Podobnie, języka~C oferuje jedynie proste, jednościeżkowe konstrukcje
  sterujące; instrukcje warunkowe, pętle, grupowanie instrukcji
  i~podprogram; nie zapewnia wieloprogramowości, współbieżności,
  synchronizacji czy~współprogramów.

  Jakkolwiek brak kilku takich cech można wydawać~się poważną usterką
  („Sugerujecie, że~dla porównania dwóch ciągów znaków będę musiały wywołać
  jakąś funkcję?”), to sprowadzenie języka do tak skromnych rozmiarów
  przyniosło wymierne korzyści. Ponieważ~C jest względnie mały, jego opis
  zajmuje mało miejsce i~można go szybko przyswoić. Wolno więc rozsądnie
  założyć, że~programista zna, rozumie i~rzeczywiście regularnie używa
  pełnej mocy języka.

  Brian W.~Kernighan, Dennis M.~Ritchie, \textit{Język ANSI~C}, K\&R,
  str.~18, \parencite{Kernighan-Ritchie-Jezyk-ANSI-C-Pub-2004}.

\end{frame}
% ##################





% ##################
\begin{frame}
  \frametitle{W~jakim sensie~C jest językiem przenośnym?}


  Chociaż język~C odzwierciedla możliwości wielu komputerów, nie jest
  zależny od architektury żadnej konkretnej maszyny. Dzięki temu~--
  z~pewną dozą ostrożności~-- łatwo można napisać programy „przenośne”,
  co znaczy tyle, że~mogą być uruchamiane bez zmian na różnorodnym sprzęcie.

  Brian W.~Kernighan, Dennis M.~Ritchie, \textit{Język ANSI~C}, K\&R,
  str.~19, \parencite{Kernighan-Ritchie-Jezyk-ANSI-C-Pub-2004}.

\end{frame}
% ##################





% ##################
\begin{frame}
  \frametitle{Bardzo ważny problem}


  Kompilatory mają ostrzegać o~najczęstszych typach błędów, wyeliminowano
  także automatyczne przekształcenia danych o~niezgodnych typach.
  Niemniej jednak język~C zachował swoją zasadniczą filozofię, według
  której programista wie co robi; wymaga~się jedynie, aby jawnie sformułował
  swoje zamiary.

  Brian W.~Kernighan, Dennis M.~Ritchie, \textit{Język ANSI~C}, K\&R,
  str.~20, \parencite{Kernighan-Ritchie-Jezyk-ANSI-C-Pub-2004}.

  % \VerSpaceFour





  Dzisiaj przyjmuje~się inne podejście, z~którym ja~się zgadzam. Mianowicie,
  że~programistę \alert{trzeba} bronić przed nim samym.

\end{frame}
% ##################





% ##################
\begin{frame}
  \frametitle{Zdrowa samokrytyka}


  Na~koniec~C, jak każdy inny język, ma swoje wady. Kilka operatorów
  ma~źle określony priorytet; pewne części składni mogłyby być lepsze. Mimo
  to~C udowodnił, że~jest niezwykle sprawnym i~wyrazistym językiem dla
  szerokiej gamy zastosowań.

  Brian W.~Kernighan, Dennis M.~Ritchie, \textit{Język ANSI~C}, K\&R,
  str.~20, \parencite{Kernighan-Ritchie-Jezyk-ANSI-C-Pub-2004}.

\end{frame}
% ##################





% ##################
\begin{frame}
  \frametitle{Bardzo ważne}


  Program w~języku~C, niezależnie od rozmiaru, jest zbudowany
  z~\alert{funkcji} i~\alert{zmiennych}. Funkcja zawiera \alert{instrukcje}
  określające, jakie operacje procesu obliczeniowego należy wykonać,
  zmienne zaś przechowują wartości używane podczas tego procesu.

  Brian W.~Kernighan, Dennis M.~Ritchie, \textit{Język ANSI~C}, K\&R,
  str.~20, \parencite{Kernighan-Ritchie-Jezyk-ANSI-C-Pub-2004}.

\end{frame}
% ##################





% ##################
\begin{frame}
  \frametitle{Natura C}


  Od tego czasu [powstania języka~C w~okolicach roku 1972] wiele~się
  zmieniło, ale~C nadal jest językiem, który powinien znać każdy
  szanujący~się programista. Większość „nowoczesnych” i~„wyrafinowanych”
  języków programowania ma korzenie w~C. Niekiedy~C jest nazywany
  „przenośnym asemblerem”, więc jako początkujący programista asemblera,
  powinieneś go poznać.

  Jo Van Hoey \textit{Programowanie w~asemblerze x64}, str.~17,
  \parencite{}

\end{frame}
% ##################






% % ##################
% \begin{frame}
%   \frametitle{Bootowanie}




% \end{frame}
% % ##################





% % ##################
% \begin{frame}
%   \frametitle{Czy uruchomienie komputera jest proste czy trudne?}




% \end{frame}
% % ##################










% % ##################
% \begin{frame}
%   \frametitle{Czym jest informatyk?}



% \end{frame}
% % ##################





% % ##################
% \begin{frame}
%   \frametitle{Po co języki programowania?}




% \end{frame}
% % ##################





% % ##################
% \begin{frame}
%   \frametitle{Języki programowania}




% \end{frame}
% % ##################





% % ##################
% \begin{frame}
%   \frametitle{„Hello, World!” w~assemblerze ???}



% \end{frame}
% % ##################





% % ##################
% \begin{frame}
%   \frametitle{Prosty schemat rozwiązywania problemu\ldots}




% \end{frame}
% % ##################





% % ##################
% \begin{frame}
%   \frametitle{Przykład}




% \end{frame}
% % ##################





% % ##################
% \begin{frame}
%   \frametitle{Przykład}



% \end{frame}
% % ##################





% % ##################
% \begin{frame}
%   \frametitle{Co to jest implementacja?}




% \end{frame}
% % ##################





% % ##################
% \begin{frame}
%   \frametitle{Co to jest implementacja?}




% \end{frame}
% % ##################





% % ##################
% \begin{frame}
%   \frametitle{Przykład. Postawienie problemu i~metoda rozwiązania}




% \end{frame}
% % ##################





% % ##################
% \begin{frame}
%   \frametitle{Implementacja w~języku~C}



% \end{frame}
% % ##################





% % ##################
% \begin{frame}
%   \frametitle{Implementacja w~języku~C. Cd.}




% \end{frame}
% % ##################







% % ##################
% \begin{frame}
%   \frametitle{Implementacje w~języku Python}




% \end{frame}
% % ##################





% % ##################
% \begin{frame}
%   \frametitle{????}



% \end{frame}
% % ##################

















% % ##################
% \begin{frame}
%   \frametitle{}



% \end{frame}
% % ##################





% % ##################
% \begin{frame}
%   \frametitle{?????}




% \end{frame}
% % ##################





% % ##################
% \begin{frame}
%   \frametitle{????}



% \end{frame}
% % ##################





% % ##################
% \begin{frame}
%   \frametitle{?????}


% \end{frame}
% % ##################





% % ##################
% \begin{frame}
%   \frametitle{}



% \end{frame}
% % ##################










% % ##################
% \begin{frame}
%   \frametitle{???}




% \end{frame}
% % ##################





% % ##################
% \begin{frame}
%   \frametitle{??????}




% \end{frame}
% % ##################





% % ##################
% \begin{frame}
%   \frametitle{?????}





% \end{frame}
% % ##################





% % ##################
% \begin{frame}
%   \frametitle{?????}



% \end{frame}
% % ##################





% % ##################
% \begin{frame}
%   \frametitle{?????}



% \end{frame}
% % ##################






% % ##################
% \begin{frame}
%   \frametitle{?????}




% \end{frame}
% % ##################





% % ##################
% \begin{frame}
%   \frametitle{????}




% \end{frame}
% % ##################





% % ##################
% \begin{frame}
%   \frametitle{??????}




% \end{frame}
% % ##################





% % ##################
% \begin{frame}
%   \frametitle{???}



% \end{frame}
% % ##################





% % ##################
% \begin{frame}
%   \frametitle{?????}



% \end{frame}
% % ##################






% % ##################
% \begin{frame}
%   \frametitle{????}




% \end{frame}
% % ##################





% % ##################
% \begin{frame}
%   \frametitle{?????}



% \end{frame}
% % ##################





% % ##################
% \begin{frame}
%   \frametitle{?????}




% \end{frame}
% % ##################





% % ##################
% \begin{frame}
%   \frametitle{?????}



% \end{frame}
% % ##################





% % ##################
% \begin{frame}
%   \frametitle{??????}



% \end{frame}
% % ##################





% % ##################
% \begin{frame}
%   \frametitle{?????}



% \end{frame}
% % ##################





% % ##################
% \begin{frame}
%   \frametitle{?????}



% \end{frame}
% % ##################





% % ##################
% \begin{frame}
%   \frametitle{????}



% \end{frame}
% % ##################





% % ##################
% \begin{frame}
%   \frametitle{????}



% \end{frame}
% % ##################





% % ##################
% \begin{frame}
%   \frametitle{?????}




% \end{frame}
% % ##################





% % ##################
% \begin{frame}
%   \frametitle{????}



% \end{frame}
% % ##################





% % ##################
% \begin{frame}
%   \frametitle{?????}



% \end{frame}
% % ##################





% % ##################
% \begin{frame}
%   \frametitle{?????}



% \end{frame}
% % ##################





% % ##################
% \begin{frame}
%   \frametitle{?????}




% \end{frame}
% % ##################





% % ##################
% \begin{frame}
%   \frametitle{?????}




% \end{frame}
% % ##################





% % ##################
% \begin{frame}
%   \frametitle{?????}



% \end{frame}
% % ##################





% % ##################
% \begin{frame}
%   \frametitle{????}



% \end{frame}
% % ##################





% % ##################
% \begin{frame}
%   \frametitle{?????}



% \end{frame}
% % ##################





% % ##################
% \begin{frame}
%   \frametitle{????}




% \end{frame}
% % ##################





% % ##################
% \begin{frame}
%   \frametitle{?????}



% \end{frame}
% % ##################





% % ##################
% \begin{frame}
%   \frametitle{?????}





% \end{frame}
% % ##################





% % ##################
% \begin{frame}
%   \frametitle{?????}




% \end{frame}
% % ##################





% % ##################
% \begin{frame}
%   \frametitle{????}




% \end{frame}
% % ##################










% % ######################################
% \appendix
% % ######################################





% % ##################
% \GeometryThreeDTwoSpecialEndingSlidesEN{Questions? Thank you for your attention.}
% % ##################



% % % ##################
% % \jagiellonianendslide{Dziękuję za~uwagę.}
% % % ##################










% % ######################################
% \SectionSlideWithPicture{Terminological notes}
% % ######################################



% % ##################
% \begin{frame}
%   \frametitle{??????}



% \end{frame}
% % ##################





% % ##################
% \begin{frame}
%   \frametitle{?????}



% \end{frame}
% ##################





% ##################
% \begin{frame}
%   \frametitle{?????}



% \end{frame}
% ##################





% ##################
% \begin{frame}
%   \frametitle{?????}



% \end{frame}
% ##################





% ##################
% \begin{frame}
%   \frametitle{?????}



% \end{frame}
%  ##################









% ####################################################################
% ####################################################################
% Bibliography

\printbibliography





% ####################################################################
% End of the document

\end{document}
