% ------------------------------------------------------------------------------------------------------------------
% Basic configuration of Beamera class and Jagiellonian theme
% ------------------------------------------------------------------------------------------------------------------
\RequirePackage[l2tabu, orthodox]{nag}



\ifx\PresentationStyle\notset
  \def\PresentationStyle{dark}
\fi



% Options: t -- align text to the top of the frame
\documentclass[10pt,t]{beamer}
\mode<presentation>
\usetheme[style=\PresentationStyle]{jagiellonian}





% ------------------------------------------------------------------------------------
% Procesing configuration files of Jagiellonian theme located in
% the directory "preambule"
% ------------------------------------------------------------------------------------
% Configuration for polish language
% Need description
\usepackage[polish]{babel}
% Need description
\usepackage[MeX]{polski}



% ------------------------------
% Better support of polish chars in technical parts of PDF
% ------------------------------
\hypersetup{pdfencoding=auto,psdextra}

% Package "textpos" give as enviroment "textblock" which is very usefull in
% arranging text on slides.

% This is standard configuration of "textpos"
\usepackage[overlay,absolute]{textpos}

% If you need to see bounds of "textblock's" comment line above and uncomment
% one below.

% Caution! When showboxes option is on significant ammunt of space is add
% to the top of textblock and as such, everyting put in them gone down.
% We need to check how to remove this bug.

% \usepackage[showboxes,overlay,absolute]{textpos}



% Setting scale length for package "textpos"
\setlength{\TPHorizModule}{10mm}
\setlength{\TPVertModule}{\TPHorizModule}


% ---------------------------------------
% Packages written for lectures "Geometria 3D dla twórców gier wideo"
% ---------------------------------------
% \usepackage{./Geometry3DPackages/Geometry3D}
% \usepackage{./Geometry3DPackages/Geometry3DIndices}
% \usepackage{./Geometry3DPackages/Geometry3DTikZStyle}
% \usepackage{./ProgramowanieSymulacjiFizykiPaczki/ProgramowanieSymulacjiFizykiTikZStyle}
% \usepackage{./Geometry3DPackages/mathcommands}


% ---------------------------------------
% TikZ
% ---------------------------------------
% Importing TikZ libraries
\usetikzlibrary{arrows.meta}
\usetikzlibrary{positioning}





% % Configuration package "bm" that need for making bold symbols
% \newcommand{\bmmax}{0}
% \newcommand{\hmmax}{0}
% \usepackage{bm}




% ---------------------------------------
% Packages for scientific texts
% ---------------------------------------
% \let\lll\undefined  % Sometimes you must use this line to allow
% "amsmath" package to works with packages with packages for polish
% languge imported
% /preambul/LanguageSettings/JagiellonianPolishLanguageSettings.tex.
% This comments (probably) removes polish letter Ł.
\usepackage{amsmath}  % Packages from American Mathematical Society (AMS)
\usepackage{amssymb}
\usepackage{amscd}
\usepackage{amsthm}
\usepackage{siunitx}  % Package for typsetting SI units.
\usepackage{upgreek}  % Better looking greek letters.
% Example of using upgreek: pi = \uppi


\usepackage{calrsfs}  % Zmienia czcionkę kaligraficzną w \mathcal
% na ładniejszą. Może w innych miejscach robi to samo, ale o tym nic
% nie wiem.










% ---------------------------------------
% Packages written for lectures "Geometria 3D dla twórców gier wideo"
% ---------------------------------------
% \usepackage{./ProgramowanieSymulacjiFizykiPaczki/ProgramowanieSymulacjiFizyki}
% \usepackage{./ProgramowanieSymulacjiFizykiPaczki/ProgramowanieSymulacjiFizykiIndeksy}
% \usepackage{./ProgramowanieSymulacjiFizykiPaczki/ProgramowanieSymulacjiFizykiTikZStyle}





% !!!!!!!!!!!!!!!!!!!!!!!!!!!!!!
% !!!!!!!!!!!!!!!!!!!!!!!!!!!!!!
% EVIL STUFF
\if\JUlogotitle1
\edef\LogoJUPath{LogoJU_\JUlogoLang/LogoJU_\JUlogoShape_\JUlogoColor.pdf}
\titlegraphic{\hfill\includegraphics[scale=0.22]
{./JagiellonianPictures/\LogoJUPath}}
\fi
% ---------------------------------------
% Commands for handling colors
% ---------------------------------------


% Command for setting normal text color for some text in math modestyle
% Text color depend on used style of Jagiellonian

% Beamer version of command
\newcommand{\TextWithNormalTextColor}[1]{%
  {\color{jNormalTextFGColor}
    \setbeamercolor{math text}{fg=jNormalTextFGColor} {#1}}
}

% Article and similar classes version of command
% \newcommand{\TextWithNormalTextColor}[1]{%
%   {\color{jNormalTextsFGColor} {#1}}
% }



% Beamer version of command
\newcommand{\NormalTextInMathMode}[1]{%
  {\color{jNormalTextFGColor}
    \setbeamercolor{math text}{fg=jNormalTextFGColor} \text{#1}}
}


% Article and similar classes version of command
% \newcommand{\NormalTextInMathMode}[1]{%
%   {\color{jNormalTextsFGColor} \text{#1}}
% }




% Command that sets color of some mathematical text to the same color
% that has normal text in header (?)

% Beamer version of the command
\newcommand{\MathTextFrametitleFGColor}[1]{%
  {\color{jFrametitleFGColor}
    \setbeamercolor{math text}{fg=jFrametitleFGColor} #1}
}

% Article and similar classes version of the command
% \newcommand{\MathTextWhiteColor}[1]{{\color{jFrametitleFGColor} #1}}





% Command for setting color of alert text for some text in math modestyle

% Beamer version of the command
\newcommand{\MathTextAlertColor}[1]{%
  {\color{jOrange} \setbeamercolor{math text}{fg=jOrange} #1}
}

% Article and similar classes version of the command
% \newcommand{\MathTextAlertColor}[1]{{\color{jOrange} #1}}





% Command that allow you to sets chosen color as the color of some text into
% math mode. Due to some nuances in the way that Beamer handle colors
% it not work in all cases. We hope that in the future we will improve it.

% Beamer version of the command
\newcommand{\SetMathTextColor}[2]{%
  {\color{#1} \setbeamercolor{math text}{fg=#1} #2}
}


% Article and similar classes version of the command
% \newcommand{\SetMathTextColor}[2]{{\color{#1} #2}}










% ---------------------------------------
% Commands for few special slides
% ---------------------------------------
\newcommand{\EndingSlide}[1]{%
  \begin{frame}[standout]

    \begingroup

    \color{jFrametitleFGColor}

    #1

    \endgroup

  \end{frame}
}










% ---------------------------------------
% Commands for setting background pictures for some slides
% ---------------------------------------
\newcommand{\TitleBackgroundPicture}
{./JagiellonianPictures/Backgrounds/LajkonikDark.png}
\newcommand{\SectionBackgroundPicture}
{./JagiellonianPictures/Backgrounds/LajkonikLight.png}



\newcommand{\TitleSlideWithPicture}{%
  \begingroup

  \usebackgroundtemplate{%
    \includegraphics[height=\paperheight]{\TitleBackgroundPicture}}

  \maketitle

  \endgroup
}





\newcommand{\SectionSlideWithPicture}[1]{%
  \begingroup

  \usebackgroundtemplate{%
    \includegraphics[height=\paperheight]{\SectionBackgroundPicture}}

  \setbeamercolor{titlelike}{fg=normal text.fg}

  \section{#1}

  \endgroup
}










% ---------------------------------------
% Commands for lectures "Geometria 3D dla twórców gier wideo"
% Polish version
% ---------------------------------------
% Komendy teraz wykomentowane były potrzebne, gdy loga były na niebieskim
% tle, nie na białym. A są na białym bo tego chcieli w biurze projektu.
% \newcommand{\FundingLogoWhitePicturePL}
% {./PresentationPictures/CommonPictures/logotypFundusze_biale_bez_tla2.pdf}
\newcommand{\FundingLogoColorPicturePL}
{./PresentationPictures/CommonPictures/European_Funds_color_PL.pdf}
% \newcommand{\EULogoWhitePicturePL}
% {./PresentationPictures/CommonPictures/logotypUE_biale_bez_tla2.pdf}
\newcommand{\EUSocialFundLogoColorPicturePL}
{./PresentationPictures/CommonPictures/EU_Social_Fund_color_PL.pdf}
% \newcommand{\ZintegrUJLogoWhitePicturePL}
% {./PresentationPictures/CommonPictures/zintegruj-logo-white.pdf}
\newcommand{\ZintegrUJLogoColorPicturePL}
{./PresentationPictures/CommonPictures/ZintegrUJ_color.pdf}
\newcommand{\JULogoColorPicturePL}
{./JagiellonianPictures/LogoJU_PL/LogoJU_A_color.pdf}





\newcommand{\GeometryThreeDSpecialBeginningSlidePL}{%
  \begin{frame}[standout]

    \begin{textblock}{11}(1,0.7)

      \begin{flushleft}

        \mdseries

        \footnotesize

        \color{jFrametitleFGColor}

        Materiał powstał w ramach projektu współfinansowanego ze środków
        Unii Europejskiej w ramach Europejskiego Funduszu Społecznego
        POWR.03.05.00-00-Z309/17-00.

      \end{flushleft}

    \end{textblock}





    \begin{textblock}{10}(0,2.2)

      \tikz \fill[color=jBackgroundStyleLight] (0,0) rectangle (12.8,-1.5);

    \end{textblock}


    \begin{textblock}{3.2}(1,2.45)

      \includegraphics[scale=0.3]{\FundingLogoColorPicturePL}

    \end{textblock}


    \begin{textblock}{2.5}(3.7,2.5)

      \includegraphics[scale=0.2]{\JULogoColorPicturePL}

    \end{textblock}


    \begin{textblock}{2.5}(6,2.4)

      \includegraphics[scale=0.1]{\ZintegrUJLogoColorPicturePL}

    \end{textblock}


    \begin{textblock}{4.2}(8.4,2.6)

      \includegraphics[scale=0.3]{\EUSocialFundLogoColorPicturePL}

    \end{textblock}

  \end{frame}
}



\newcommand{\GeometryThreeDTwoSpecialBeginningSlidesPL}{%
  \begin{frame}[standout]

    \begin{textblock}{11}(1,0.7)

      \begin{flushleft}

        \mdseries

        \footnotesize

        \color{jFrametitleFGColor}

        Materiał powstał w ramach projektu współfinansowanego ze środków
        Unii Europejskiej w ramach Europejskiego Funduszu Społecznego
        POWR.03.05.00-00-Z309/17-00.

      \end{flushleft}

    \end{textblock}





    \begin{textblock}{10}(0,2.2)

      \tikz \fill[color=jBackgroundStyleLight] (0,0) rectangle (12.8,-1.5);

    \end{textblock}


    \begin{textblock}{3.2}(1,2.45)

      \includegraphics[scale=0.3]{\FundingLogoColorPicturePL}

    \end{textblock}


    \begin{textblock}{2.5}(3.7,2.5)

      \includegraphics[scale=0.2]{\JULogoColorPicturePL}

    \end{textblock}


    \begin{textblock}{2.5}(6,2.4)

      \includegraphics[scale=0.1]{\ZintegrUJLogoColorPicturePL}

    \end{textblock}


    \begin{textblock}{4.2}(8.4,2.6)

      \includegraphics[scale=0.3]{\EUSocialFundLogoColorPicturePL}

    \end{textblock}

  \end{frame}





  \TitleSlideWithPicture
}



\newcommand{\GeometryThreeDSpecialEndingSlidePL}{%
  \begin{frame}[standout]

    \begin{textblock}{11}(1,0.7)

      \begin{flushleft}

        \mdseries

        \footnotesize

        \color{jFrametitleFGColor}

        Materiał powstał w ramach projektu współfinansowanego ze środków
        Unii Europejskiej w~ramach Europejskiego Funduszu Społecznego
        POWR.03.05.00-00-Z309/17-00.

      \end{flushleft}

    \end{textblock}





    \begin{textblock}{10}(0,2.2)

      \tikz \fill[color=jBackgroundStyleLight] (0,0) rectangle (12.8,-1.5);

    \end{textblock}


    \begin{textblock}{3.2}(1,2.45)

      \includegraphics[scale=0.3]{\FundingLogoColorPicturePL}

    \end{textblock}


    \begin{textblock}{2.5}(3.7,2.5)

      \includegraphics[scale=0.2]{\JULogoColorPicturePL}

    \end{textblock}


    \begin{textblock}{2.5}(6,2.4)

      \includegraphics[scale=0.1]{\ZintegrUJLogoColorPicturePL}

    \end{textblock}


    \begin{textblock}{4.2}(8.4,2.6)

      \includegraphics[scale=0.3]{\EUSocialFundLogoColorPicturePL}

    \end{textblock}





    \begin{textblock}{11}(1,4)

      \begin{flushleft}

        \mdseries

        \footnotesize

        \RaggedRight

        \color{jFrametitleFGColor}

        Treść niniejszego wykładu jest udostępniona na~licencji
        Creative Commons (\textsc{cc}), z~uzna\-niem autorstwa
        (\textsc{by}) oraz udostępnianiem na tych samych warunkach
        (\textsc{sa}). Rysunki i~wy\-kresy zawarte w~wykładzie są
        autorstwa dr.~hab.~Pawła Węgrzyna et~al. i~są dostępne
        na tej samej licencji, o~ile nie wskazano inaczej.
        W~prezentacji wykorzystano temat Beamera Jagiellonian,
        oparty na~temacie Metropolis Matthiasa Vogelgesanga,
        dostępnym na licencji \LaTeX{} Project Public License~1.3c
        pod adresem: \colorhref{https://github.com/matze/mtheme}
        {https://github.com/matze/mtheme}.

        Projekt typograficzny: Iwona Grabska-Gradzińska \\
        Skład: Kamil Ziemian;
        Korekta: Wojciech Palacz \\
        Modele: Dariusz Frymus, Kamil Nowakowski \\
        Rysunki i~wykresy: Kamil Ziemian, Paweł Węgrzyn, Wojciech Palacz

      \end{flushleft}

    \end{textblock}

  \end{frame}
}



\newcommand{\GeometryThreeDTwoSpecialEndingSlidesPL}[1]{%
  \begin{frame}[standout]


    \begin{textblock}{11}(1,0.7)

      \begin{flushleft}

        \mdseries

        \footnotesize

        \color{jFrametitleFGColor}

        Materiał powstał w ramach projektu współfinansowanego ze środków
        Unii Europejskiej w~ramach Europejskiego Funduszu Społecznego
        POWR.03.05.00-00-Z309/17-00.

      \end{flushleft}

    \end{textblock}





    \begin{textblock}{10}(0,2.2)

      \tikz \fill[color=jBackgroundStyleLight] (0,0) rectangle (12.8,-1.5);

    \end{textblock}


    \begin{textblock}{3.2}(1,2.45)

      \includegraphics[scale=0.3]{\FundingLogoColorPicturePL}

    \end{textblock}


    \begin{textblock}{2.5}(3.7,2.5)

      \includegraphics[scale=0.2]{\JULogoColorPicturePL}

    \end{textblock}


    \begin{textblock}{2.5}(6,2.4)

      \includegraphics[scale=0.1]{\ZintegrUJLogoColorPicturePL}

    \end{textblock}


    \begin{textblock}{4.2}(8.4,2.6)

      \includegraphics[scale=0.3]{\EUSocialFundLogoColorPicturePL}

    \end{textblock}





    \begin{textblock}{11}(1,4)

      \begin{flushleft}

        \mdseries

        \footnotesize

        \RaggedRight

        \color{jFrametitleFGColor}

        Treść niniejszego wykładu jest udostępniona na~licencji
        Creative Commons (\textsc{cc}), z~uzna\-niem autorstwa
        (\textsc{by}) oraz udostępnianiem na tych samych warunkach
        (\textsc{sa}). Rysunki i~wy\-kresy zawarte w~wykładzie są
        autorstwa dr.~hab.~Pawła Węgrzyna et~al. i~są dostępne
        na tej samej licencji, o~ile nie wskazano inaczej.
        W~prezentacji wykorzystano temat Beamera Jagiellonian,
        oparty na~temacie Metropolis Matthiasa Vogelgesanga,
        dostępnym na licencji \LaTeX{} Project Public License~1.3c
        pod adresem: \colorhref{https://github.com/matze/mtheme}
        {https://github.com/matze/mtheme}.

        Projekt typograficzny: Iwona Grabska-Gradzińska \\
        Skład: Kamil Ziemian;
        Korekta: Wojciech Palacz \\
        Modele: Dariusz Frymus, Kamil Nowakowski \\
        Rysunki i~wykresy: Kamil Ziemian, Paweł Węgrzyn, Wojciech Palacz

      \end{flushleft}

    \end{textblock}

  \end{frame}





  \begin{frame}[standout]

    \begingroup

    \color{jFrametitleFGColor}

    #1

    \endgroup

  \end{frame}
}



\newcommand{\GeometryThreeDSpecialEndingSlideVideoPL}{%
  \begin{frame}[standout]

    \begin{textblock}{11}(1,0.7)

      \begin{flushleft}

        \mdseries

        \footnotesize

        \color{jFrametitleFGColor}

        Materiał powstał w ramach projektu współfinansowanego ze środków
        Unii Europejskiej w~ramach Europejskiego Funduszu Społecznego
        POWR.03.05.00-00-Z309/17-00.

      \end{flushleft}

    \end{textblock}





    \begin{textblock}{10}(0,2.2)

      \tikz \fill[color=jBackgroundStyleLight] (0,0) rectangle (12.8,-1.5);

    \end{textblock}


    \begin{textblock}{3.2}(1,2.45)

      \includegraphics[scale=0.3]{\FundingLogoColorPicturePL}

    \end{textblock}


    \begin{textblock}{2.5}(3.7,2.5)

      \includegraphics[scale=0.2]{\JULogoColorPicturePL}

    \end{textblock}


    \begin{textblock}{2.5}(6,2.4)

      \includegraphics[scale=0.1]{\ZintegrUJLogoColorPicturePL}

    \end{textblock}


    \begin{textblock}{4.2}(8.4,2.6)

      \includegraphics[scale=0.3]{\EUSocialFundLogoColorPicturePL}

    \end{textblock}





    \begin{textblock}{11}(1,4)

      \begin{flushleft}

        \mdseries

        \footnotesize

        \RaggedRight

        \color{jFrametitleFGColor}

        Treść niniejszego wykładu jest udostępniona na~licencji
        Creative Commons (\textsc{cc}), z~uzna\-niem autorstwa
        (\textsc{by}) oraz udostępnianiem na tych samych warunkach
        (\textsc{sa}). Rysunki i~wy\-kresy zawarte w~wykładzie są
        autorstwa dr.~hab.~Pawła Węgrzyna et~al. i~są dostępne
        na tej samej licencji, o~ile nie wskazano inaczej.
        W~prezentacji wykorzystano temat Beamera Jagiellonian,
        oparty na~temacie Metropolis Matthiasa Vogelgesanga,
        dostępnym na licencji \LaTeX{} Project Public License~1.3c
        pod adresem: \colorhref{https://github.com/matze/mtheme}
        {https://github.com/matze/mtheme}.

        Projekt typograficzny: Iwona Grabska-Gradzińska;
        Skład: Kamil Ziemian \\
        Korekta: Wojciech Palacz;
        Modele: Dariusz Frymus, Kamil Nowakowski \\
        Rysunki i~wykresy: Kamil Ziemian, Paweł Węgrzyn, Wojciech Palacz \\
        Montaż: Agencja Filmowa Film \& Television Production~-- Zbigniew
        Masklak

      \end{flushleft}

    \end{textblock}

  \end{frame}
}





\newcommand{\GeometryThreeDTwoSpecialEndingSlidesVideoPL}[1]{%
  \begin{frame}[standout]

    \begin{textblock}{11}(1,0.7)

      \begin{flushleft}

        \mdseries

        \footnotesize

        \color{jFrametitleFGColor}

        Materiał powstał w ramach projektu współfinansowanego ze środków
        Unii Europejskiej w~ramach Europejskiego Funduszu Społecznego
        POWR.03.05.00-00-Z309/17-00.

      \end{flushleft}

    \end{textblock}





    \begin{textblock}{10}(0,2.2)

      \tikz \fill[color=jBackgroundStyleLight] (0,0) rectangle (12.8,-1.5);

    \end{textblock}


    \begin{textblock}{3.2}(1,2.45)

      \includegraphics[scale=0.3]{\FundingLogoColorPicturePL}

    \end{textblock}


    \begin{textblock}{2.5}(3.7,2.5)

      \includegraphics[scale=0.2]{\JULogoColorPicturePL}

    \end{textblock}


    \begin{textblock}{2.5}(6,2.4)

      \includegraphics[scale=0.1]{\ZintegrUJLogoColorPicturePL}

    \end{textblock}


    \begin{textblock}{4.2}(8.4,2.6)

      \includegraphics[scale=0.3]{\EUSocialFundLogoColorPicturePL}

    \end{textblock}





    \begin{textblock}{11}(1,4)

      \begin{flushleft}

        \mdseries

        \footnotesize

        \RaggedRight

        \color{jFrametitleFGColor}

        Treść niniejszego wykładu jest udostępniona na~licencji
        Creative Commons (\textsc{cc}), z~uzna\-niem autorstwa
        (\textsc{by}) oraz udostępnianiem na tych samych warunkach
        (\textsc{sa}). Rysunki i~wy\-kresy zawarte w~wykładzie są
        autorstwa dr.~hab.~Pawła Węgrzyna et~al. i~są dostępne
        na tej samej licencji, o~ile nie wskazano inaczej.
        W~prezentacji wykorzystano temat Beamera Jagiellonian,
        oparty na~temacie Metropolis Matthiasa Vogelgesanga,
        dostępnym na licencji \LaTeX{} Project Public License~1.3c
        pod adresem: \colorhref{https://github.com/matze/mtheme}
        {https://github.com/matze/mtheme}.

        Projekt typograficzny: Iwona Grabska-Gradzińska;
        Skład: Kamil Ziemian \\
        Korekta: Wojciech Palacz;
        Modele: Dariusz Frymus, Kamil Nowakowski \\
        Rysunki i~wykresy: Kamil Ziemian, Paweł Węgrzyn, Wojciech Palacz \\
        Montaż: Agencja Filmowa Film \& Television Production~-- Zbigniew
        Masklak

      \end{flushleft}

    \end{textblock}

  \end{frame}





  \begin{frame}[standout]


    \begingroup

    \color{jFrametitleFGColor}

    #1

    \endgroup

  \end{frame}
}










% ---------------------------------------
% Commands for lectures "Geometria 3D dla twórców gier wideo"
% English version
% ---------------------------------------
% \newcommand{\FundingLogoWhitePictureEN}
% {./PresentationPictures/CommonPictures/logotypFundusze_biale_bez_tla2.pdf}
\newcommand{\FundingLogoColorPictureEN}
{./PresentationPictures/CommonPictures/European_Funds_color_EN.pdf}
% \newcommand{\EULogoWhitePictureEN}
% {./PresentationPictures/CommonPictures/logotypUE_biale_bez_tla2.pdf}
\newcommand{\EUSocialFundLogoColorPictureEN}
{./PresentationPictures/CommonPictures/EU_Social_Fund_color_EN.pdf}
% \newcommand{\ZintegrUJLogoWhitePictureEN}
% {./PresentationPictures/CommonPictures/zintegruj-logo-white.pdf}
\newcommand{\ZintegrUJLogoColorPictureEN}
{./PresentationPictures/CommonPictures/ZintegrUJ_color.pdf}
\newcommand{\JULogoColorPictureEN}
{./JagiellonianPictures/LogoJU_EN/LogoJU_A_color.pdf}



\newcommand{\GeometryThreeDSpecialBeginningSlideEN}{%
  \begin{frame}[standout]

    \begin{textblock}{11}(1,0.7)

      \begin{flushleft}

        \mdseries

        \footnotesize

        \color{jFrametitleFGColor}

        This content was created as part of a project co-financed by the
        European Union within the framework of the European Social Fund
        POWR.03.05.00-00-Z309/17-00.

      \end{flushleft}

    \end{textblock}





    \begin{textblock}{10}(0,2.2)

      \tikz \fill[color=jBackgroundStyleLight] (0,0) rectangle (12.8,-1.5);

    \end{textblock}


    \begin{textblock}{3.2}(0.7,2.45)

      \includegraphics[scale=0.3]{\FundingLogoColorPictureEN}

    \end{textblock}


    \begin{textblock}{2.5}(4.15,2.5)

      \includegraphics[scale=0.2]{\JULogoColorPictureEN}

    \end{textblock}


    \begin{textblock}{2.5}(6.35,2.4)

      \includegraphics[scale=0.1]{\ZintegrUJLogoColorPictureEN}

    \end{textblock}


    \begin{textblock}{4.2}(8.4,2.6)

      \includegraphics[scale=0.3]{\EUSocialFundLogoColorPictureEN}

    \end{textblock}

  \end{frame}
}



\newcommand{\GeometryThreeDTwoSpecialBeginningSlidesEN}{%
  \begin{frame}[standout]

    \begin{textblock}{11}(1,0.7)

      \begin{flushleft}

        \mdseries

        \footnotesize

        \color{jFrametitleFGColor}

        This content was created as part of a project co-financed by the
        European Union within the framework of the European Social Fund
        POWR.03.05.00-00-Z309/17-00.

      \end{flushleft}

    \end{textblock}





    \begin{textblock}{10}(0,2.2)

      \tikz \fill[color=jBackgroundStyleLight] (0,0) rectangle (12.8,-1.5);

    \end{textblock}


    \begin{textblock}{3.2}(0.7,2.45)

      \includegraphics[scale=0.3]{\FundingLogoColorPictureEN}

    \end{textblock}


    \begin{textblock}{2.5}(4.15,2.5)

      \includegraphics[scale=0.2]{\JULogoColorPictureEN}

    \end{textblock}


    \begin{textblock}{2.5}(6.35,2.4)

      \includegraphics[scale=0.1]{\ZintegrUJLogoColorPictureEN}

    \end{textblock}


    \begin{textblock}{4.2}(8.4,2.6)

      \includegraphics[scale=0.3]{\EUSocialFundLogoColorPictureEN}

    \end{textblock}

  \end{frame}





  \TitleSlideWithPicture
}



\newcommand{\GeometryThreeDSpecialEndingSlideEN}{%
  \begin{frame}[standout]

    \begin{textblock}{11}(1,0.7)

      \begin{flushleft}

        \mdseries

        \footnotesize

        \color{jFrametitleFGColor}

        This content was created as part of a project co-financed by the
        European Union within the framework of the European Social Fund
        POWR.03.05.00-00-Z309/17-00.

      \end{flushleft}

    \end{textblock}





    \begin{textblock}{10}(0,2.2)

      \tikz \fill[color=jBackgroundStyleLight] (0,0) rectangle (12.8,-1.5);

    \end{textblock}


    \begin{textblock}{3.2}(0.7,2.45)

      \includegraphics[scale=0.3]{\FundingLogoColorPictureEN}

    \end{textblock}


    \begin{textblock}{2.5}(4.15,2.5)

      \includegraphics[scale=0.2]{\JULogoColorPictureEN}

    \end{textblock}


    \begin{textblock}{2.5}(6.35,2.4)

      \includegraphics[scale=0.1]{\ZintegrUJLogoColorPictureEN}

    \end{textblock}


    \begin{textblock}{4.2}(8.4,2.6)

      \includegraphics[scale=0.3]{\EUSocialFundLogoColorPictureEN}

    \end{textblock}





    \begin{textblock}{11}(1,4)

      \begin{flushleft}

        \mdseries

        \footnotesize

        \RaggedRight

        \color{jFrametitleFGColor}

        The content of this lecture is made available under a~Creative
        Commons licence (\textsc{cc}), giving the author the credits
        (\textsc{by}) and putting an obligation to share on the same terms
        (\textsc{sa}). Figures and diagrams included in the lecture are
        authored by Paweł Węgrzyn et~al., and are available under the same
        license unless indicated otherwise.\\ The presentation uses the
        Beamer Jagiellonian theme based on Matthias Vogelgesang’s
        Metropolis theme, available under license \LaTeX{} Project
        Public License~1.3c at: \colorhref{https://github.com/matze/mtheme}
        {https://github.com/matze/mtheme}.

        Typographic design: Iwona Grabska-Gradzińska \\
        \LaTeX{} Typesetting: Kamil Ziemian \\
        Proofreading: Wojciech Palacz,
        Monika Stawicka \\
        3D Models: Dariusz Frymus, Kamil Nowakowski \\
        Figures and charts: Kamil Ziemian, Paweł Węgrzyn, Wojciech Palacz

      \end{flushleft}

    \end{textblock}

  \end{frame}
}



\newcommand{\GeometryThreeDTwoSpecialEndingSlidesEN}[1]{%
  \begin{frame}[standout]


    \begin{textblock}{11}(1,0.7)

      \begin{flushleft}

        \mdseries

        \footnotesize

        \color{jFrametitleFGColor}

        This content was created as part of a project co-financed by the
        European Union within the framework of the European Social Fund
        POWR.03.05.00-00-Z309/17-00.

      \end{flushleft}

    \end{textblock}





    \begin{textblock}{10}(0,2.2)

      \tikz \fill[color=jBackgroundStyleLight] (0,0) rectangle (12.8,-1.5);

    \end{textblock}


    \begin{textblock}{3.2}(0.7,2.45)

      \includegraphics[scale=0.3]{\FundingLogoColorPictureEN}

    \end{textblock}


    \begin{textblock}{2.5}(4.15,2.5)

      \includegraphics[scale=0.2]{\JULogoColorPictureEN}

    \end{textblock}


    \begin{textblock}{2.5}(6.35,2.4)

      \includegraphics[scale=0.1]{\ZintegrUJLogoColorPictureEN}

    \end{textblock}


    \begin{textblock}{4.2}(8.4,2.6)

      \includegraphics[scale=0.3]{\EUSocialFundLogoColorPictureEN}

    \end{textblock}





    \begin{textblock}{11}(1,4)

      \begin{flushleft}

        \mdseries

        \footnotesize

        \RaggedRight

        \color{jFrametitleFGColor}

        The content of this lecture is made available under a~Creative
        Commons licence (\textsc{cc}), giving the author the credits
        (\textsc{by}) and putting an obligation to share on the same terms
        (\textsc{sa}). Figures and diagrams included in the lecture are
        authored by Paweł Węgrzyn et~al., and are available under the same
        license unless indicated otherwise.\\ The presentation uses the
        Beamer Jagiellonian theme based on Matthias Vogelgesang’s
        Metropolis theme, available under license \LaTeX{} Project
        Public License~1.3c at: \colorhref{https://github.com/matze/mtheme}
        {https://github.com/matze/mtheme}.

        Typographic design: Iwona Grabska-Gradzińska \\
        \LaTeX{} Typesetting: Kamil Ziemian \\
        Proofreading: Wojciech Palacz,
        Monika Stawicka \\
        3D Models: Dariusz Frymus, Kamil Nowakowski \\
        Figures and charts: Kamil Ziemian, Paweł Węgrzyn, Wojciech Palacz

      \end{flushleft}

    \end{textblock}

  \end{frame}





  \begin{frame}[standout]

    \begingroup

    \color{jFrametitleFGColor}

    #1

    \endgroup

  \end{frame}
}



\newcommand{\GeometryThreeDSpecialEndingSlideVideoVerOneEN}{%
  \begin{frame}[standout]

    \begin{textblock}{11}(1,0.7)

      \begin{flushleft}

        \mdseries

        \footnotesize

        \color{jFrametitleFGColor}

        This content was created as part of a project co-financed by the
        European Union within the framework of the European Social Fund
        POWR.03.05.00-00-Z309/17-00.

      \end{flushleft}

    \end{textblock}





    \begin{textblock}{10}(0,2.2)

      \tikz \fill[color=jBackgroundStyleLight] (0,0) rectangle (12.8,-1.5);

    \end{textblock}


    \begin{textblock}{3.2}(0.7,2.45)

      \includegraphics[scale=0.3]{\FundingLogoColorPictureEN}

    \end{textblock}


    \begin{textblock}{2.5}(4.15,2.5)

      \includegraphics[scale=0.2]{\JULogoColorPictureEN}

    \end{textblock}


    \begin{textblock}{2.5}(6.35,2.4)

      \includegraphics[scale=0.1]{\ZintegrUJLogoColorPictureEN}

    \end{textblock}


    \begin{textblock}{4.2}(8.4,2.6)

      \includegraphics[scale=0.3]{\EUSocialFundLogoColorPictureEN}

    \end{textblock}





    \begin{textblock}{11}(1,4)

      \begin{flushleft}

        \mdseries

        \footnotesize

        \RaggedRight

        \color{jFrametitleFGColor}

        The content of this lecture is made available under a Creative
        Commons licence (\textsc{cc}), giving the author the credits
        (\textsc{by}) and putting an obligation to share on the same terms
        (\textsc{sa}). Figures and diagrams included in the lecture are
        authored by Paweł Węgrzyn et~al., and are available under the same
        license unless indicated otherwise.\\ The presentation uses the
        Beamer Jagiellonian theme based on Matthias Vogelgesang’s
        Metropolis theme, available under license \LaTeX{} Project
        Public License~1.3c at: \colorhref{https://github.com/matze/mtheme}
        {https://github.com/matze/mtheme}.

        Typographic design: Iwona Grabska-Gradzińska;
        \LaTeX{} Typesetting: Kamil Ziemian \\
        Proofreading: Wojciech Palacz,
        Monika Stawicka \\
        3D Models: Dariusz Frymus, Kamil Nowakowski \\
        Figures and charts: Kamil Ziemian, Paweł Węgrzyn, Wojciech
        Palacz \\
        Film editing: Agencja Filmowa Film \& Television Production~--
        Zbigniew Masklak

      \end{flushleft}

    \end{textblock}

  \end{frame}
}



\newcommand{\GeometryThreeDSpecialEndingSlideVideoVerTwoEN}{%
  \begin{frame}[standout]

    \begin{textblock}{11}(1,0.7)

      \begin{flushleft}

        \mdseries

        \footnotesize

        \color{jFrametitleFGColor}

        This content was created as part of a project co-financed by the
        European Union within the framework of the European Social Fund
        POWR.03.05.00-00-Z309/17-00.

      \end{flushleft}

    \end{textblock}





    \begin{textblock}{10}(0,2.2)

      \tikz \fill[color=jBackgroundStyleLight] (0,0) rectangle (12.8,-1.5);

    \end{textblock}


    \begin{textblock}{3.2}(0.7,2.45)

      \includegraphics[scale=0.3]{\FundingLogoColorPictureEN}

    \end{textblock}


    \begin{textblock}{2.5}(4.15,2.5)

      \includegraphics[scale=0.2]{\JULogoColorPictureEN}

    \end{textblock}


    \begin{textblock}{2.5}(6.35,2.4)

      \includegraphics[scale=0.1]{\ZintegrUJLogoColorPictureEN}

    \end{textblock}


    \begin{textblock}{4.2}(8.4,2.6)

      \includegraphics[scale=0.3]{\EUSocialFundLogoColorPictureEN}

    \end{textblock}





    \begin{textblock}{11}(1,4)

      \begin{flushleft}

        \mdseries

        \footnotesize

        \RaggedRight

        \color{jFrametitleFGColor}

        The content of this lecture is made available under a Creative
        Commons licence (\textsc{cc}), giving the author the credits
        (\textsc{by}) and putting an obligation to share on the same terms
        (\textsc{sa}). Figures and diagrams included in the lecture are
        authored by Paweł Węgrzyn et~al., and are available under the same
        license unless indicated otherwise.\\ The presentation uses the
        Beamer Jagiellonian theme based on Matthias Vogelgesang’s
        Metropolis theme, available under license \LaTeX{} Project
        Public License~1.3c at: \colorhref{https://github.com/matze/mtheme}
        {https://github.com/matze/mtheme}.

        Typographic design: Iwona Grabska-Gradzińska;
        \LaTeX{} Typesetting: Kamil Ziemian \\
        Proofreading: Wojciech Palacz,
        Monika Stawicka \\
        3D Models: Dariusz Frymus, Kamil Nowakowski \\
        Figures and charts: Kamil Ziemian, Paweł Węgrzyn, Wojciech
        Palacz \\
        Film editing: IMAVI -- Joanna Kozakiewicz, Krzysztof Magda, Nikodem
        Frodyma

      \end{flushleft}

    \end{textblock}

  \end{frame}
}



\newcommand{\GeometryThreeDSpecialEndingSlideVideoVerThreeEN}{%
  \begin{frame}[standout]

    \begin{textblock}{11}(1,0.7)

      \begin{flushleft}

        \mdseries

        \footnotesize

        \color{jFrametitleFGColor}

        This content was created as part of a project co-financed by the
        European Union within the framework of the European Social Fund
        POWR.03.05.00-00-Z309/17-00.

      \end{flushleft}

    \end{textblock}





    \begin{textblock}{10}(0,2.2)

      \tikz \fill[color=jBackgroundStyleLight] (0,0) rectangle (12.8,-1.5);

    \end{textblock}


    \begin{textblock}{3.2}(0.7,2.45)

      \includegraphics[scale=0.3]{\FundingLogoColorPictureEN}

    \end{textblock}


    \begin{textblock}{2.5}(4.15,2.5)

      \includegraphics[scale=0.2]{\JULogoColorPictureEN}

    \end{textblock}


    \begin{textblock}{2.5}(6.35,2.4)

      \includegraphics[scale=0.1]{\ZintegrUJLogoColorPictureEN}

    \end{textblock}


    \begin{textblock}{4.2}(8.4,2.6)

      \includegraphics[scale=0.3]{\EUSocialFundLogoColorPictureEN}

    \end{textblock}





    \begin{textblock}{11}(1,4)

      \begin{flushleft}

        \mdseries

        \footnotesize

        \RaggedRight

        \color{jFrametitleFGColor}

        The content of this lecture is made available under a Creative
        Commons licence (\textsc{cc}), giving the author the credits
        (\textsc{by}) and putting an obligation to share on the same terms
        (\textsc{sa}). Figures and diagrams included in the lecture are
        authored by Paweł Węgrzyn et~al., and are available under the same
        license unless indicated otherwise.\\ The presentation uses the
        Beamer Jagiellonian theme based on Matthias Vogelgesang’s
        Metropolis theme, available under license \LaTeX{} Project
        Public License~1.3c at: \colorhref{https://github.com/matze/mtheme}
        {https://github.com/matze/mtheme}.

        Typographic design: Iwona Grabska-Gradzińska;
        \LaTeX{} Typesetting: Kamil Ziemian \\
        Proofreading: Wojciech Palacz,
        Monika Stawicka \\
        3D Models: Dariusz Frymus, Kamil Nowakowski \\
        Figures and charts: Kamil Ziemian, Paweł Węgrzyn, Wojciech
        Palacz \\
        Film editing: Agencja Filmowa Film \& Television Production~--
        Zbigniew Masklak \\
        Film editing: IMAVI -- Joanna Kozakiewicz, Krzysztof Magda, Nikodem
        Frodyma

      \end{flushleft}

    \end{textblock}

  \end{frame}
}



\newcommand{\GeometryThreeDTwoSpecialEndingSlidesVideoVerOneEN}[1]{%
  \begin{frame}[standout]

    \begin{textblock}{11}(1,0.7)

      \begin{flushleft}

        \mdseries

        \footnotesize

        \color{jFrametitleFGColor}

        This content was created as part of a project co-financed by the
        European Union within the framework of the European Social Fund
        POWR.03.05.00-00-Z309/17-00.

      \end{flushleft}

    \end{textblock}





    \begin{textblock}{10}(0,2.2)

      \tikz \fill[color=jBackgroundStyleLight] (0,0) rectangle (12.8,-1.5);

    \end{textblock}


    \begin{textblock}{3.2}(0.7,2.45)

      \includegraphics[scale=0.3]{\FundingLogoColorPictureEN}

    \end{textblock}


    \begin{textblock}{2.5}(4.15,2.5)

      \includegraphics[scale=0.2]{\JULogoColorPictureEN}

    \end{textblock}


    \begin{textblock}{2.5}(6.35,2.4)

      \includegraphics[scale=0.1]{\ZintegrUJLogoColorPictureEN}

    \end{textblock}


    \begin{textblock}{4.2}(8.4,2.6)

      \includegraphics[scale=0.3]{\EUSocialFundLogoColorPictureEN}

    \end{textblock}





    \begin{textblock}{11}(1,4)

      \begin{flushleft}

        \mdseries

        \footnotesize

        \RaggedRight

        \color{jFrametitleFGColor}

        The content of this lecture is made available under a Creative
        Commons licence (\textsc{cc}), giving the author the credits
        (\textsc{by}) and putting an obligation to share on the same terms
        (\textsc{sa}). Figures and diagrams included in the lecture are
        authored by Paweł Węgrzyn et~al., and are available under the same
        license unless indicated otherwise.\\ The presentation uses the
        Beamer Jagiellonian theme based on Matthias Vogelgesang’s
        Metropolis theme, available under license \LaTeX{} Project
        Public License~1.3c at: \colorhref{https://github.com/matze/mtheme}
        {https://github.com/matze/mtheme}.

        Typographic design: Iwona Grabska-Gradzińska;
        \LaTeX{} Typesetting: Kamil Ziemian \\
        Proofreading: Wojciech Palacz,
        Monika Stawicka \\
        3D Models: Dariusz Frymus, Kamil Nowakowski \\
        Figures and charts: Kamil Ziemian, Paweł Węgrzyn,
        Wojciech Palacz \\
        Film editing: Agencja Filmowa Film \& Television Production~--
        Zbigniew Masklak

      \end{flushleft}

    \end{textblock}

  \end{frame}





  \begin{frame}[standout]


    \begingroup

    \color{jFrametitleFGColor}

    #1

    \endgroup

  \end{frame}
}



\newcommand{\GeometryThreeDTwoSpecialEndingSlidesVideoVerTwoEN}[1]{%
  \begin{frame}[standout]

    \begin{textblock}{11}(1,0.7)

      \begin{flushleft}

        \mdseries

        \footnotesize

        \color{jFrametitleFGColor}

        This content was created as part of a project co-financed by the
        European Union within the framework of the European Social Fund
        POWR.03.05.00-00-Z309/17-00.

      \end{flushleft}

    \end{textblock}





    \begin{textblock}{10}(0,2.2)

      \tikz \fill[color=jBackgroundStyleLight] (0,0) rectangle (12.8,-1.5);

    \end{textblock}


    \begin{textblock}{3.2}(0.7,2.45)

      \includegraphics[scale=0.3]{\FundingLogoColorPictureEN}

    \end{textblock}


    \begin{textblock}{2.5}(4.15,2.5)

      \includegraphics[scale=0.2]{\JULogoColorPictureEN}

    \end{textblock}


    \begin{textblock}{2.5}(6.35,2.4)

      \includegraphics[scale=0.1]{\ZintegrUJLogoColorPictureEN}

    \end{textblock}


    \begin{textblock}{4.2}(8.4,2.6)

      \includegraphics[scale=0.3]{\EUSocialFundLogoColorPictureEN}

    \end{textblock}





    \begin{textblock}{11}(1,4)

      \begin{flushleft}

        \mdseries

        \footnotesize

        \RaggedRight

        \color{jFrametitleFGColor}

        The content of this lecture is made available under a Creative
        Commons licence (\textsc{cc}), giving the author the credits
        (\textsc{by}) and putting an obligation to share on the same terms
        (\textsc{sa}). Figures and diagrams included in the lecture are
        authored by Paweł Węgrzyn et~al., and are available under the same
        license unless indicated otherwise.\\ The presentation uses the
        Beamer Jagiellonian theme based on Matthias Vogelgesang’s
        Metropolis theme, available under license \LaTeX{} Project
        Public License~1.3c at: \colorhref{https://github.com/matze/mtheme}
        {https://github.com/matze/mtheme}.

        Typographic design: Iwona Grabska-Gradzińska;
        \LaTeX{} Typesetting: Kamil Ziemian \\
        Proofreading: Wojciech Palacz,
        Monika Stawicka \\
        3D Models: Dariusz Frymus, Kamil Nowakowski \\
        Figures and charts: Kamil Ziemian, Paweł Węgrzyn,
        Wojciech Palacz \\
        Film editing: IMAVI -- Joanna Kozakiewicz, Krzysztof Magda, Nikodem
        Frodyma

      \end{flushleft}

    \end{textblock}

  \end{frame}





  \begin{frame}[standout]


    \begingroup

    \color{jFrametitleFGColor}

    #1

    \endgroup

  \end{frame}
}



\newcommand{\GeometryThreeDTwoSpecialEndingSlidesVideoVerThreeEN}[1]{%
  \begin{frame}[standout]

    \begin{textblock}{11}(1,0.7)

      \begin{flushleft}

        \mdseries

        \footnotesize

        \color{jFrametitleFGColor}

        This content was created as part of a project co-financed by the
        European Union within the framework of the European Social Fund
        POWR.03.05.00-00-Z309/17-00.

      \end{flushleft}

    \end{textblock}





    \begin{textblock}{10}(0,2.2)

      \tikz \fill[color=jBackgroundStyleLight] (0,0) rectangle (12.8,-1.5);

    \end{textblock}


    \begin{textblock}{3.2}(0.7,2.45)

      \includegraphics[scale=0.3]{\FundingLogoColorPictureEN}

    \end{textblock}


    \begin{textblock}{2.5}(4.15,2.5)

      \includegraphics[scale=0.2]{\JULogoColorPictureEN}

    \end{textblock}


    \begin{textblock}{2.5}(6.35,2.4)

      \includegraphics[scale=0.1]{\ZintegrUJLogoColorPictureEN}

    \end{textblock}


    \begin{textblock}{4.2}(8.4,2.6)

      \includegraphics[scale=0.3]{\EUSocialFundLogoColorPictureEN}

    \end{textblock}





    \begin{textblock}{11}(1,4)

      \begin{flushleft}

        \mdseries

        \footnotesize

        \RaggedRight

        \color{jFrametitleFGColor}

        The content of this lecture is made available under a Creative
        Commons licence (\textsc{cc}), giving the author the credits
        (\textsc{by}) and putting an obligation to share on the same terms
        (\textsc{sa}). Figures and diagrams included in the lecture are
        authored by Paweł Węgrzyn et~al., and are available under the same
        license unless indicated otherwise. \\ The presentation uses the
        Beamer Jagiellonian theme based on Matthias Vogelgesang’s
        Metropolis theme, available under license \LaTeX{} Project
        Public License~1.3c at: \colorhref{https://github.com/matze/mtheme}
        {https://github.com/matze/mtheme}.

        Typographic design: Iwona Grabska-Gradzińska;
        \LaTeX{} Typesetting: Kamil Ziemian \\
        Proofreading: Leszek Hadasz, Wojciech Palacz,
        Monika Stawicka \\
        3D Models: Dariusz Frymus, Kamil Nowakowski \\
        Figures and charts: Kamil Ziemian, Paweł Węgrzyn,
        Wojciech Palacz \\
        Film editing: Agencja Filmowa Film \& Television Production~--
        Zbigniew Masklak \\
        Film editing: IMAVI -- Joanna Kozakiewicz, Krzysztof Magda, Nikodem
        Frodyma


      \end{flushleft}

    \end{textblock}

  \end{frame}





  \begin{frame}[standout]


    \begingroup

    \color{jFrametitleFGColor}

    #1

    \endgroup

  \end{frame}
}











% ------------------------------------------------------
% BibLaTeX
% ------------------------------------------------------
% Package biblatex, with biber as its backend, allow us to handle
% bibliography entries that use Unicode symbols outside ASCII.
\usepackage[
language=polish,
backend=biber,
style=alphabetic,
url=false,
eprint=true,
]{biblatex}

\addbibresource{Podstawy-informatyki-ETC-Bibliography.bib}





% ------------------------------------------------------
% Importing packages, libraries and setting their configuration
% ------------------------------------------------------
% Library improving positioning of nodes in graphs
% \usetikzlibrary{positioning}





% ------------------------------------------------------
% Local packages
% ------------------------------------------------------
% Local configuration of this particular presentation
\usepackage{./Local-packages/local-settings}

% Styles for arrows
\usepackage{./Local-packages/PGF-TikZ-Arrows-styles}

% Styles for drawing diagrams
\usepackage{./Local-packages/PGF-TikZ-Diagram-styles}

% Jagiellonian theme's colros
\usepackage{./Local-packages/jagiellonian-theme-colors}










% ------------------------------------------------------------------------------------------------------------------
\title{Podstawy informatyki z~językiem~C}
\subtitle{Wprowadzenie do~programowania}

\author{Kamil Ziemian \\
  \email}


% \date{}
% ------------------------------------------------------------------------------------------------------------------










% ####################################################################
% Beginning of the document
\begin{document}
% ####################################################################





% ######################################
% Number of chars: 62k+, 73k+, 43k+, 52k+, 69k+, 77k+, 32k+, 33k+, 62k+,
% Text is adjusted to the left and words are broken at the end of the line.
\RaggedRight
% ######################################





% ######################################
\maketitle
% ######################################





% ##################
\begin{frame}
  \frametitle{Spis treści}


  \tableofcontents

\end{frame}
% ##################










% ##################
\begin{frame}
  \frametitle{Charakter tych zajęć}


  Niekiedy wejdziemy w~pewne zagadnienia teoretyczne, bo~są sytuacje, gdy
  niewiedza teoretyczna powoduje poważne problemy praktyczne. Jeśli ktoś
  nie wie, że~proch strzelniczy wybucha pod wpływem temperatury, to może
  nieopatrznie postawić pudełko naboi koło gorącego pieca. W~informatyce
  też zdarzają~się takie sytuacje, gdy brak wiedzy teoretycznej jest zbyt
  groźny, by można było pominąć dyskusję danego problemu.

  Jako kwestie praktyczne traktujemy też dobre praktyki programisty.
  Czyli jeśli nawet dany kod działa, ale w~dłuższej perspektywie może
  sprawiać problemy, to będziemy naciskali na jego zmianę. Wbrew
  pozorom~są to bardzo praktyczne sprawy.

  Proszę jednak pamiętać, że~ze względu na to treść tego kursu jest wysoce
  selektywna. O~wielu rzeczach z~premedytacją nie powiemy, uznając je
  za~zbyt teoretyczne.

\end{frame}
% ##################





% ##################
\begin{frame}
  \frametitle{Charakter tych zajęć}


  Może~się Państwu wydawać, że~ilość zagadnień teoretycznych na tym kursie
  i~tak jest bardzo duża. To zależy w~sporej mierze od perspektywy,
  bo~naprawdę wiele zagadnień teoretycznych zdecydowaliśmy~się pominąć.

  Jeśli oczywiście ktoś chce~się zagłębić w~te bardziej teoretyczne
  zagadnienia, to oczywiście służymy całą naszą wiedzą. Można
  nas o~to pytać po zajęciach i~pisać na e-maila \email.
  Prosimy takim e-mailom nadawać tytuły typu „Pytanie o~X”, tak by nie
  zaginęły one wśród kilkudziesięciu innych e-maili jakie zwykle dostajemy
  każdego miesiąca.

\end{frame}
% ##################





% ##################
\begin{frame}
  \frametitle{O~nazwach plików}


  Powtórzmy kilka rzeczy. Rozszerzenie pliku, kropka i~to co następuje po
  niej w~nazwie tego pliku, informuje używane przez nas programy z~jakim
  typem pliku mają do czynienia. Dlatego każdy język programowania posiada
  własne, właściwe tylko jemu rozszerzenie. Pliki napisane w~języku~C mają
  mieć rozszerzenie~\texttt{.c}. Analogicznie, każdy program w~języku Python
  powinien~się kończyć na~\texttt{.py}, pliki w~języku C++ \texttt{.cpp},
  w~\textsc{bash}u \texttt{.sh},~etc.

  Nadanie plikowi nazwy bez rozszerzenia to nie jest dobry pomysł. Nazwa
  zawierająca spacje to proszenie~się o~problem, gdy pracujemy w~systemie
  GNU/Linux (\textsc{bash} i~inne sprawy). Nazwa pliku która zawiera
  dwie lub więcej kropek to wymysł Szatana i~wytwór piekieł. Proszę mi
  oszczędzić styczności z~tymi plugawymi tworami.

\end{frame}
% ##################


% ##################
\begin{frame}
  \frametitle{Tak to wygląda}

  \vspace{-0.5em}


  \begin{figure}

    \label{fig:Learning-any-language-at-100-procent}

    \centering


    \includegraphics[scale=0.195]
    {./Presentations-pictures/Learning-language-at-100.jpg}

  \end{figure}

\end{frame}
% ##################


% ##################
\begin{frame}
  \frametitle{Rozszerzenia plików}


  Jak ktoś na zajęciach napisze program w~języku~C, który \alert{nie} ma
  rozszerzenia~\texttt{.c}, będzie to musiał przy mnie poprawić.
  Chyba, że~będę musiał od razu biec do kogoś innego. Ale proszę nie
  liczyć, że~zapomnę o~tej zniewadze. ;)

  Brak tego rozszerzenia sprawia, że~możemy zostać pozbawienie części
  wsparcia w~pracy z~C, jaką dają nam przeróżne programy. A~do tego
  \alert{nie} możemy dopuścić.

  Załóżmy, że~ktoś wymyśli taką szatańską nazwę \\
  \texttt{plik-numer.1.c} \\
  co jest absolutnym plugastwem. Programy których używamy mogą błędnie
  rozpoznać „\texttt{.1.c}” jako rozszerzenie pliku i~wtedy demony mają
  prawa wylecieć nam
  \colorhref{http://www.catb.org/jargon/html/N/nasal-demons.html}{z~nosa}.

  % Jak ktoś na zajęciach napisze program w~języku~C, który \alert{nie} ma
  % rozszerzenia~\texttt{.c}, będzie to musiał przy mnie poprawić.
  % Chyba, że~będę musiał od razu biec do kogoś innego. Ale proszę nie
  % liczyć, że~zapomnę o~tej zniewadze. ;)

\end{frame}
% ##################





% ##################
\begin{frame}
  \frametitle{Plugawe nazwy plików}


  W~nazwie pliku powinna być \alert{jedna i~tylko jedna kropka}, ta
  poprzedzająca nazwę rozszerzenia pliku, czyli dla nas \texttt{.c}.
  Jeśli ta zasada nie jest prawdą wszędzie na świecie, to na pewno jest
  prawdą na tych zajęciach i~proszę o~tym pamiętać.

  Jeśli ktoś prześle mi plik, którego nazwa zawiera więcej lub mniej niż
  \alert{jedną} kropkę, to zastrzegam sobie prawo odesłania mu pliku
  z~prośbą o~zmianę nazwy na~poprawniejszą. Dla mnie to nie jest sposób
  karania, tylko metoda wyrabiania w~ludziach dobrych nawyków.

\end{frame}
% ##################





% ##################
\begin{frame}
  \frametitle{Te zajęcia są prowadzone w~C, ale\ldots}


  Czy ja powinienem~się skupić na nauce języka~C? To bardzo dobre
  pytanie i~każdy z~Państwa powinien je sobie zadać.

  Każdy dobry informatyk powinien dobrze znać co najmniej jeden język
  programowania, acz niekoniecznie musi to być język~C. Zupełnie zrozumiałe
  jest, że~ktoś z~Państwa zdecyduje, iż~językiem który chce poznać jest
  jakiś inny język niż~C i~będzie~się uczył~C tylko tyle ile wymaga ten
  przedmiot. Co~więcej, w~naszej ocenie jest to bardzo dobre podejście do
  nauki.

  Jeśli nie wiedzą Państwo, na nauce którego języka należy~się teraz skupić,
  proponuję sprawdzić~$20$ najpopularniejszych języków programowania
  według indeksu
  \colorhref{https://www.tiobe.com/tiobe-index/}{\textsc{tiobe}},
  poczytać o~każdym z~nich i~wybrać ten który Państwo uważają za~najlepszy.

\end{frame}
% ##################





% ##################
\begin{frame}
  \frametitle{Te zajęcia są prowadzone w~C, ale\ldots}


  W~grudniu $2024$ roku indeks
  \colorhref{https://www.tiobe.com/tiobe-index/}{\textsc{tiobe}}
  przedstawia~się w~następujący sposób: Python, C++, Java, C, C\#,
  JavaScript, Go, \textsc{sql}, Visual Basic, Fortran, Delphi/Object
  Pascal, \textsc{php}, Scratch, Rust, \textsc{matlab}, R, język asemblera,
  Ruby, \textsc{cobol}, Swift.

  Decyzję o~tym na nauce którego języka~się Państwo skupią, muszą
  oczywiście podjąć Państwo sami. Jeśli ktoś chce uzyskać od nas jakieś
  wskazówki, to służymy całą swoją osobą. Jeśli chcą Państwo napisać emaila
  w~tej sprawie to proszę go nazwać „Pytanie o~wybór języka do nauki”, czy
  coś podobnego.

  Bardzo~się ucieszymy, jeśli ktoś z~Państwa zdecyduje~się skupić na
  języku~C, ale to w~żadnym razie nie jest wymagane. Nie o~to chodzi w~nauce
  informatyki.

\end{frame}
% ##################











% ##################
\begin{frame}
  \frametitle{Kilka rad praktycznych}


  Edytor \textsc{gnu} \colorhref{https://www.nano-editor.org/}{nano} to nie
  jest, w~mojej ocenie, dobry edytor dla osób początkujących. Zbyt dużo
  czasu tracimy na przełączaniu~się między nim, a~powłoką, zamiast na
  próbie zrozumienia jak należy pisać programy komputerowe. Jeśli ktoś jest
  już zaawansowany programistą~C (czy jakiegokolwiek innego języka) lub
  wyjątkowo lubi nano, to oczywiście, może i~powinien z~niego korzystać.
  Po prostu uważam, że~dla osób początkujących to~nie jest zbyt dobry wybór.

  Dla tych którzy nigdy wcześniej nie programowali polecam edytor
  \colorhref{https://gedit-text-editor.org/}{gedit}, gdyż jest prosty
  w~obsłudze i~nie trzeba go samemu ustawiać. Potem warto rozważyć użycie
  potężniejszej technologii.

  Gdy chodzi o~edytory to ja normalnie używam
  \colorhref{https://www.gnu.org/software/emacs/}{\textsc{gnu} Emacsa},
  który jest niezwykle fajny, ale ogromnie skomplikowany w~użyciu.
  Tak trudny, że~aby go dobrze opanować, trzeba~się nauczyć nowego języka
  programowania o~nazwie \colorhref
  {https://www.gnu.org/software/emacs/manual/html_node/eintr/}{Emacs Lisp}
  (w~skrócie: elisp). I~to wcale nie jest żart. Oprócz tego legendarny
  \colorhref{https://www.vim.org/}{vim} wciąż ma wierną rzeszę fanów.

\end{frame}
% ##################





% ##################
\begin{frame}
  \frametitle{Kilka rad praktycznych}


  \colorhref{https://www.vim.org/}{Vima} warto znać, bo jest praktycznie
  wszędzie, gdzie jest system GNU/Linux, a~takich edytorów nie ma wiele.
  Trzeba jednak wiedzieć, że~jest on dość trudny w~obsłudze.

  Odkładając na bok edytory, warto zwrócić uwagę na zintegrowane środowiska
  programistyczne (\textsc{ide}, ang.~\textit{Integrated Development
    Environment}). Dla języków C i~C++ jest dedykowane \textsc{ide}
  \colorhref{https://www.jetbrains.com/clion/promo/}{CLion} od firmy
  \colorhref{https://www.jetbrains.com/}{JetBrains}, z~którego mogą Państwo
  korzystać dzięki licencji uczelnianej. Warto~się mu przyjrzeć, jeśli
  ktoś chce programować w~C z~wykorzystaniem \textsc{ide}.

  Wybór oczywiście pozostawiamy Państwu. Jeśli ktoś chce więcej porad
  od nas, to oczywiście służymy całą naszą wiedzą.

\end{frame}
% ##################










% ######################################
\section{Kilka rad odnośnie języka~C}
% ######################################


% ##################
\begin{frame}
  \frametitle{Kilka rad odnośnie języka~C}


  Jeśli chcesz użyć \texttt{\#define} zastanów~się, czy nie można
  tego zrobić korzystając z~\texttt{const}. Jeśli można, to skorzystaj
  z~\texttt{const}.

  Jeśli nie wiesz czy użyć \texttt{float} czy \texttt{double}, użyj
  \texttt{double}.

  Jeśli masz dobry powód by użyć \texttt{float}, użyj \texttt{float}.

  Jeśli nic Państwo z~tego nie rozumieją, to proszę~się nie martwić.
  Będziemy o~tym mówić w~trakcie tego kursu.

\end{frame}
% ##################










% ######################################
\section{Kilka informacji o~języku~C}
% ######################################



% ##################
\begin{frame}
  \frametitle{Jak długo~C jest z~nami?}


  Język~C został stworzony przez Dennis Ritchie ($1941\text{--}2011$), by
  ułatwić Kenowi Thompsonowi (ur.~$1943$), pracę nad systemem operacyjnym
  \textsc{unix}. Pierwsza jego wersja ukazała~się w~$1972$ roku.

  Język~C w $2025$ rok ma $53$ (!) lata. Jego wiek
  niestety często czuć, bo wiele funkcjonalności które dziś uważamy
  za~niezbędne w~pracy programisty albo są w~nim nieobecne, albo są,
  ale w~dość koślawej wersji. Niektórych rzeczy nie da~się w~sposób
  elegancki dodać do języka z~lat 70-tych XX wieku.

  Do tego język~C został pomyślany jako mały język, więc on z~premedytacją
  \alert{nie} wprowadził pewnych rzeczy znanych w~$1972$~roku. To
  niestety też nie ułatwia nam życia. Co gorsza, dla początkujących~C i~tak
  może sprawiać wrażenie ogromnego języka. On jest mały w~porównaniu
  z~innymi językami, ale to niewiele zmienia, gdy~się dopiera zaczyna swoją
  przygodę z~informatyką.

\end{frame}
% ##################





% ##################
\begin{frame}
  \frametitle{Czy 1972 rok był dawno temu?}


    \begin{figure}

    \centering


    \includegraphics[scale=0.3]
    {./Presentations-pictures/Meeting-01/Dennis-Ritchie-Ken-Thompson-PDP-11.jpg}

    \caption{Dennis Ritchie (stojący) i~Ken Thompson pracują na komputerze
      PDP-11 firmy Digital Equipment Corporation.}

    \label{fig:Ritchie-Thompson-PDP-11}

  \end{figure}

\end{frame}
% ##################





% ##################
\begin{frame}
  \frametitle{Zalety języka~C}


  \begin{itemize}

  \item Pozwala tworzyć bardzo szybkie programy. \alert{Bardzo szybkie}
      programy.

  \item Jeśli poznamy~C, to nauka języka który na nim bazuje będzie
    już prostsza. A~60\%, albo więcej, najważniejszych języków jakie
    istnieją, w~jakimś stopniu bazuje na~C.

  \item Na świecie jest tyle kodu napisanego w~języku~C, że~jeśli umie~się
    go dobrze, można dość łatwo znaleźć naprawdę dobrze płatną pracę. I~na
    razie nic nie wskazuje, by ta praca zniknęła przez następne $20$ lat
    (do $2044$~roku). Chat\textsc{gpt} też nie.

  \item Pozwala lepiej poznać jak komputer działa na niskim poziomie.
    Zdania uczonych są podzielone, czy to jest wada czy zaleta. Ja
    ostatnio~się skłaniam do stwierdzenia, że~to jest zaleta, ale rozumiem,
    jeśli według Państwa mylę się~totalnie.

  \item Jawne zarządzanie wskaźnikami.

  \end{itemize}

\end{frame}
% ##################







% ##################
\begin{frame}
  \frametitle{Wady języka~C}


  \begin{itemize}

  \item Pisane w~C nie jest wcale proste.

  \item Wersji języka~C, różniących~się drugorzędnymi kwestiami, jest
    zbyt dużo i~łatwo~się w~tym pogubić.

  \item C~oferuje nam bardzo mało zabezpieczeń przed nami samymi, co
    skutkuje tym, że~bardzo łatwo stworzyć kod w~którym kryją~się
    koszmarnie niebezpieczne rzeczy. To jednak temat na inny przedmiot.

  \item Jawne zarządzanie wskaźnikami. Tak, to było też w~dziale
    \textit{Zalety języka~C}.

  \item Arytmetyka wskaźników. To relikt minionej epoki.

  \item Metody operowania na stringach. A~raczej brak takich metod.

  \item Wiele standardowych funkcji języka~C \alert{nie} powinno być
    używane. Powtarzam, wielu standardowych funkcji dostępnych w~języku~C
    \alert{nie należy} używać pod żadnym pozorem.

  \end{itemize}

\end{frame}
% ##################





% ##################
\begin{frame}
  \frametitle{Wady języka~C}


  \begin{itemize}

  \item Trudne do zapamiętania nazewnictwo. Czasem nazwy w~C są zwyczajnie
    głupie, jak \texttt{unsigned char} (pl.~\textit{znak bez znaku}) na
    oznaczenie pojedynczego bajtu.

  \item  Sama nazwa języka, „C”, też jest w~sumie dość głupia. Dzisiaj
    staramy~się nadawać im nazwy będące pełnymi słowami.

  \item Brak normalnego typu boolowskiego. Nawet \texttt{stdbool.h} nie
    rozwiązuje tego problemu.

  \item Domyślna konwersja typów. To naprawdę nie pasuje do takiego
    języka jak~C.

  \item Obsługa błędów. Jak obsługa błędów?

  \item Toporny i~niewygodny model kompilacji: pliki nagłówkowe,
    preprocesor, etc.

  \item „The more I~C, the less I see.” Przykład: złożone typy danych.

  % \item Brak modułowości.

  % \item I~miliona innych udogodnień, które daje ci dobry, nowoczesny język
  %   programowania (1972 rok był dawno temu).

  \end{itemize}

\end{frame}
% ##################





% ##################
\begin{frame}
  \frametitle{The more I~C, the less I~see}


  Jak głosi mądrość internetu, języki~C nie jest trudny.

  \texttt{void (* (*f[])())()}

  Poprawna kod w~języku~C, który definiuje~\texttt{f} jako tablicę
  o~nieokreślonej długości, zawierającej wskaźniki do funkcji, które
  to funkcje zwracając wskaźniki do~funkcji, które to funkcje~są typu
  \texttt{void}.

  Taaaaak, nic trudnego.

\end{frame}
% ##################





% ##################
\begin{frame}
  \frametitle{Wady języka~C}


  \begin{itemize}

  \item Brak modułowości.

  \item I~miliona innych udogodnień, które daje ci dobry, nowoczesny język
    programowania (rok $1972$ to było kawał czasu temu).

  \end{itemize}

\end{frame}
% ##################





% ##################
\begin{frame}
  \frametitle{Jaki jest związek między C a~C++?}


  Taki jak między krzesłem, a~krzesłem elektrycznym. Bardzie trafne byłoby
  powiedzenie, że~jak, między mechaniką i~mechaniką kwantową, ale mniej
  osób rozumie tą różnicę.

  Mimo wszystko to są dwa bardzo różne języki programowania, które
  ze~względów historycznych mają pewnie wspólny rdzeń.

  Jeśli~C to taka skrzynka z~narzędziami, to C++ to cały Leroy
  Merlin. Naprawdę, ogarnięcie wszystkich featurów C++, jest zupełnie nie do
  ogarnięcia. A~przynajmniej ja w~możliwość tego osiągnięcia nie wierzę.
  Jak zapytałem osobą prowadzącą kurs C++, czy ktoś zna cały ten język
  to odpowiedziała „Podobno Bjarne Stroustrup [twórca C++]. Ale to tylko
  plotki.”

  Języki takie jak C\#, to zupełnie osobna sprawa.

\end{frame}
% ##################













% % ##################
% \begin{frame}
%   \frametitle{AI, ChatGPT,\ldots}


%   Żeby nie pozostać za bardzo w~tyle, staram~się wprowadzić do zajęć
%   praktyki związane z~korzystanie ze~sztucznej inteligencji. Niemniej
%   nawet taki gigant programowania jak Linus Torvalds (czyta się:
%   \textit{turwalds}),
%   \colorhref{https://www.youtube.com/watch?v=VHHT6W-N0ak}{zapytany na
%     przełomie} $2023$ i~$2024$ roku o~rolę \textsc{ai} w~praktyce
%   programisty, odpowiedział w~sposób, który zdradzał jego wątpliwości
%   na temat tego, jak to będzie działać w~praktyce.

%   Ja zaś~$5$ września $2024$ przeprowadziłem z~Chatem\textsc{gpt} taką
%   rozmowę.

%   \textsc{kz:} Czy w~$2024$ roku warto~się uczyć języka~C?

%   Chat\textsc{gtp:} Tak, uczenie~się C~w~roku $2024$ wciąż jest dobrym
%   pomysłem. Oto dlaczego.

%   \textbf{1.~Podstawy informatyki.} C~pozwala nam dogłębnie zrozumieć
%   architekturę komputera i~zarządzania pamięcią, które~są kluczowe
%   dla przyswojenia sobie bardziej abstrakcyjnych koncepcji z~programowania.

%   % \textbf{2.~Duża wydajność.} C~jest znany ze swej szybkości
%   % i~niskopoziomowego dostępu do sprzętu, czyniąc go idealny kandydatem
%   % do programowania systemów (np. systemów operacyjnych, systemów
%   % wbudowanych, etc.). Wciąż jest szeroko stosowany w~przemyśle lotniczym,
%   % samochodowym i~telekomunikacji, gdzie wydajność i~kontrola nad sprzętem
%   % są kluczowe.

% \end{frame}
% % ##################







% % ##################
% \begin{frame}
%   \frametitle{AI, ChatGPT,\ldots}


%   % Żeby nie pozostać za bardzo w~tyle, staram~się wprowadzić do zajęć
%   % praktyki związane z~korzystanie ze~sztucznej inteligencji. Niemniej
%   % nawet taki gigant programowania jak Linus Torvalds (czyta się:
%   % \textit{turwalds}),
%   % \colorhref{https://www.youtube.com/watch?v=VHHT6W-N0ak}{zapytany na
%   %   przełomie} $2023$ i~$2024$ roku o~rolę \textsc{ai} w~praktyce
%   % programisty, odpowiedział w~sposób, który zdradzał jego wątpliwości
%   % na temat tego, jak to będzie działać w~praktyce.

%   % Ja zaś~$5$ września $2024$ przeprowadziłem z~Chatem\textsc{gpt} taką
%   % rozmowę.

%   % \textsc{kz:} Czy w~$2024$ roku warto~się uczyć języka~C?

%   % Chat\textsc{gtp:} Tak, uczenie~się C~w~roku $2024$ wciąż jest dobrym
%   % pomysłem. Oto dlaczego.

%   % \textbf{1.~Podstawy informatyki.} C~pozwala nam dogłębnie zrozumieć
%   % architekturę komputera i~zarządzania pamięcią, które~są kluczowe
%   % dla przyswojenia sobie bardziej abstrakcyjnych koncepcji z~programowania.

%   \textbf{2.~Duża wydajność.} C~jest znany ze swej szybkości
%   i~niskopoziomowego dostępu do sprzętu, czyniąc go idealny kandydatem
%   do programowania systemów (np. systemów operacyjnych, systemów
%   wbudowanych, etc.). Wciąż jest szeroko stosowany w~przemyśle lotniczym,
%   samochodowym i~telekomunikacji, gdzie wydajność i~kontrola nad sprzętem
%   są kluczowe.



% \end{frame}
% % ##################





% % ##################
% \begin{frame}
%   \frametitle{AI, ChatGPT,\ldots}


%   Żeby nie pozostać za bardzo w~tyle, staram~się wprowadzić do zajęć
%   praktyki związane z~korzystanie ze~sztucznej inteligencji. Niemniej
%   nawet taki gigant programowania jak Linus Torvalds (czyta się:
%   \textit{turwalds}),
%   \colorhref{https://www.youtube.com/watch?v=VHHT6W-N0ak}{zapytany na
%     przełomie} $2023$ i~$2024$ roku o~rolę \textsc{ai} w~praktyce
%   programisty, odpowiedział w~sposób, który zdradzał jego wątpliwości
%   na temat tego, jak to będzie działać w~praktyce.

%   Ja zaś $5$ września $2024$ przeprowadziłem z~Chatem\textsc{gpt} taką
%   rozmowę.

%   \textsc{kz:} Czy w~$2024$ roku warto~się uczyć języka~C?

%   Chat\textsc{gtp:} Tak, uczenie~się C~w~roku $2024$ wciąż jest dobrym
%   pomysłem. Oto dlaczego.

%   \textbf{1.~Podstawy informatyki.} C~pozwala nam dogłębnie zrozumieć
%   architekturę komputera i~zarządzania pamięcią, które~są kluczowe
%   dla przyswojenia sobie bardziej abstrakcyjnych koncepcji z~programowania.

%   \textbf{2.~Duża wydajność.} C~jest znany ze swej szybkości
%   i~niskopoziomowego dostępu do sprzętu, czyniąc go idealny kandydatem
%   do programowania systemów (np. systemów operacyjnych, systemów
%   wbudowanych, etc.). Wciąż jest szeroko stosowany w~przemyśle lotniczym,
%   samochodowym i~telekomunikacji, gdzie wydajność i~kontrola nad sprzętem
%   są kluczowe.


% \end{frame}
% % ##################



% % ##################
% \begin{frame}
%   \frametitle{AI, ChatGPT,\ldots}


%   \textbf{\textsc{kz}:} Czy Chat\textsc{gpt} jest dobry w~generowaniu kodu
%   w~C?

%   \textbf{Chat\textsc{gpt}:} Chat\textsc{gpt} jest stosunkowo dobry
%   w~generowaniu kodu w~C, zwłaszcza gdy kod ten dotyczy typowych zadań
%   i~problemów. Jednak, jakość i~dokładność generowanego kodu może zależeć
%   od złożoności zadań i~dokładności przekazanych instrukcji. Tutaj
%   jest kilka czynników które należy rozważyć.

%   1.~\textbf{Składnia i~biblioteka standardowa.} Chat\textsc{gpt} potrafi
%   dobrze generować fragmenty kodu, które są zgodne ze~składnią~C oraz
%   odwołują do standardowych bibliotek. Na przykład, może pisać kod
%   dla podstawowych operacji, struktur kontrolnych i~przetwarzania danych,
%   jak również używać standardowych funkcji z~bibliotek takich jak
%   \texttt{stdio.h}, \texttt{stdlib.h}, \texttt{string.h}, etc.

%   2.~\textbf{Powszechne schematy i~idiomy.} Potrafi rozpoznawać i~powielać
%   schemat obecne w~programach w~C, takie jak pętle dla iteracji,
%   \texttt{if-else} dla logiki warunkowej, czy \texttt{switch-case} dla
%   rozgałęzień. Potrafi również obsługiwać popularne idiomy~C, takie jak
%   używanie wskaźników, dynamiczna alokacja pamięci za pomocą \texttt{malloc}
%   i~\texttt{free} i~operacje bitowe.

%   % 3. \textbf{Rozumienie konstrukcji właściwych dla C.} Chat\textsc{gpt}
%   % zwykle dobrze sobie radzi z~konstrukcjami właściwymi dla języka C, takimi
%   % jak wskaźniki, struktury, unie, czy zarządzanie pamięcią. Potrafi również
%   % generować kod do obsługi plików \textsc{i/o}, zarządzania procesami
%   % poprzez funkcje \texttt{fork()} i~\texttt{exec()} oraz obsługujący
%   % podstawową komunikację między procesami (\textsc{ipc}, ang.
%   % \textit{Inter-Process Communication}), za pomocą takich mechanizmów jak
%   % rury (ang. \textit{pipes}) czy współdzielona pamięć.

% \end{frame}
% % ##################




% % ##################
% \begin{frame}
%   \frametitle{AI, ChatGPT,\ldots}


%   % 1.~\textbf{Składnia i~biblioteka standardowa.} Chat\textsc{gpt} potrafi
%   % dobrze generować fragmenty kodu, które są zgodne ze~składnią~C oraz
%   % odwołują do standardowych bibliotek. Na przykład, może pisać kod
%   % dla podstawowych operacji, struktur kontrolnych i~przetwarzania danych,
%   % jak również używać standardowych funkcji z~bibliotek takich jak
%   % \texttt{stdio.h}, \texttt{stdlib.h}, \texttt{string.h}, etc.

%   % 2.~\textbf{Powszechne schematy i~idiomy.} Potrafi rozpoznawać i~powielać
%   % schemat obecne w~programach w~C, takie jak pętle dla iteracji,
%   % \texttt{if-else} dla logiki warunkowej, czy \texttt{switch-case} dla
%   % rozgałęzień. Potrafi również obsługiwać popularne idiomy~C, takie jak
%   % używanie wskaźników, dynamiczna alokacja pamięci za pomocą \texttt{malloc}
%   % i~\texttt{free} i~operacje bitowe.

%   3. \textbf{Rozumienie konstrukcji właściwych dla C.} Chat\textsc{gpt}
%   zwykle dobrze sobie radzi z~konstrukcjami właściwymi dla języka C, takimi
%   jak wskaźniki, struktury, unie, czy zarządzanie pamięcią. Potrafi również
%   generować kod do obsługi plików \textsc{i/o}, zarządzania procesami
%   poprzez funkcje \texttt{fork()} i~\texttt{exec()} oraz obsługujący
%   podstawową komunikację między procesami (\textsc{ipc}, ang.
%   \textit{Inter-Process Communication}), za pomocą takich mechanizmów jak
%   rury (ang. \textit{pipes}) czy współdzielona pamięć.

%   4.~\textbf{Obsługa błędów.} Chat\textsc{gpt} może zawrzeć w~plikach
%   podstawową obsługę błędów (np. sprawdzenie, czy \texttt{malloc} zwrócił
%   \texttt{NULL}), bardziej zaawansowane zarządzanie błędami może wymagać
%   dodatkowych ulepszeń pochodzących od użytkownika.

%   5. **Advanced Topics**: For more advanced topics, like implementing network protocols, writing multi-threaded applications, or interfacing with hardware, the generated code might need more careful review and modification. ChatGPT can provide a solid starting point, but may not always capture the nuances required for robust, production-quality code.

% 6. **Code Quality**: The code generated by ChatGPT may lack certain best practices, such as efficient memory usage, proper resource management, and comprehensive error handling. Therefore, it is advisable for a human programmer to review and refine the code, especially for critical applications.

% In summary, while ChatGPT can generate functional and syntactically correct C code for a wide range of scenarios, it may not always produce optimal or fully correct solutions, particularly for more complex tasks. It's a useful tool for generating initial code snippets, learning concepts, and exploring ideas, but it should be complemented with thorough testing and review by an experienced C programmer.

% \end{frame}
% % ##################





% ######################################
\section{Czy informatyka jest trudna?}
% ######################################



% ##################
\begin{frame}
  \frametitle{Czy informatyka jest trudna?}


  Ten przedmiot dotyczy podstaw informatyki w~języku~C, warto~się
  więc spytać, czy informatyka jest prosta czy trudna w~nauce?

  Informatyka to osobna dziedzina nauki i~jeśli zabrnie~się odpowiednio
  głęboko, to robi~się naprawdę złożona i~niebanalna. Jednak na stosunkowo
  płytkim poziomie to czy jest on trudna czy nie, to mocno zależy od~odczuć
  konkretnej osoby.

  Zadam takie pytanie: czy włączenie komputera jest skomplikowane?
  Odpowiemy na to pytanie na dwóch poziomach. Pierwszy to poziom normalnego
  użytkownika, drugi to opis pochodzący z~książki Andrewa S.~Tanenbauma
  \textit{Systemy operacyjne. Wydanie~III}
  \parencite{Tannenbaum-Systemy-Operacyjne-Wydanie-III-Pub-2013}
  i~dotyczy komputera z~systemem Pentium.

\end{frame}
% ##################





% ##################
\begin{frame}
  \frametitle{Włączanie komputera, poziom normalnego użytkownika}


  \begin{enumerate}

  \item Wciskamy przycisk \texttt{Power}.

  \item Czekamy minutę albo dłużej.

  \item W~oknie logowania wybieramy użytkownika i~wchodzimy na swoje konto.

  \end{enumerate}

  Co w~tym trudnego?

\end{frame}
% ##################





% ##################
\begin{frame}
  \frametitle{Kilka pojęcia}


  Oczywiście, opis włączania komputera z~książki Tanenbauma jest tak
  skomplikowany, że~najpierw trzeba wprowadzić trochę pojęć.

  \textbf{\textsc{rom}}, ang.~\textit{Read Only Memory}, pl.~\textit{pamięć
    wyłącznie do~odczytu}. Pamięć komputera której zawartość została
  zapisana przez firmę, która ten fragment pamięci wyprodukowała
  i~użytkownik nie może zmodyfikować jej zawartości. Przynajmniej nie
  w~żaden normalny sposób.

  \textbf{\textsc{ram}}, ang.~\textit{Random Access Memory},
  pl.~\textit{pamięć o~dostępie w~trybie losowym}. Pamięć komputera o~tej
  własności, że~jeśli będę w~sposób losowy wybierał elementy tej pamięci,
  to czas odczytania informacji z~każdego z~tych elementów będzie taki sam.
  Inaczej mówiąc dostęp do dowolnego miejsca tej pamięci zajmuje tyle samo
  czasu.

  Tak naprawdę czas odczytu zależy od tego, w~jakiś sposób pamięć
  \textsc{ram} jest odczytywana, ale jeszcze długo nie będziemy się musieli
  tym przejmować.

\end{frame}
% ##################





% ##################
\begin{frame}
  \frametitle{Kilka pojęcia}


  \textbf{Pamięć ulotna}, ang.~\textit{volatile memory}. Pamięć której
  zawartość jest tracona, gdy przestaje przez nią płynąć prąd. Typowym
  przykładem takiej pamięci jest \textsc{ram}.

  \textbf{Pamięć nieulotna}, ang.~\textit{non-volatile memory}. Pamięć,
  której treść jest zachowana, gdy przez układ przestaje płynąć prąd.
  Typowym przykładem jest dysk \textsc{ssd}.

  Żeby skomplikować życie, pamięcią nieulotną nazywa~się także tą pamięć,
  które jest ulotna w~ścisłym sensie, ale ponieważ jest zaopatrzona
  we~własną baterię, jej zawartość jest zachowana również po wyłączeniu
  komputera z~prądu. Bo~niby czemu życie ma być proste?

\end{frame}
% ##################





% ##################
\begin{frame}
  \frametitle{Kilka pojęcia}


  \textbf{Pamięć \textsc{cmos}}, często po prostu \textbf{\textsc{cmos}}.
  Skrót pochodzi od angielskiej nazwy technologi \textit{Complementary
    Metal-Oxide-Semiconductor} (pl.~\textit{komplementarny półprzewodnik
    metalowo-tlenkowy}), w~której ta pamięć jest wykonana. Musi być zasilana
  prądem, by~zachowywała swój stan, ale ponieważ wyposażona jest w~baterię,
  klasyfikowana jest jako nieulotna.

  \textbf{\textsc{bios}} ang.~\textit{Basic Input Output System}, pl.
  \textit{podstawowy system wejścia, wyjścia}. Program znajdujący~się
  na płycie głównej komputera, odpowiedzialny między innymi za odczytywanie
  danych z~klawiatury, obsługę ekranu oraz operacje wejścia-wyjścia dysków.

\end{frame}
% ##################





% ##################
\begin{frame}
  \frametitle{Uruchamianie komputera z~systemem Pentium}


  \begin{itemize}

  \item[1)] Wciskamy przycisk \texttt{Power}.

  \item[2)] Z~płyty głównej ładowany jest program \textsc{bios}. Sprawdza on
    ilość zainstalowanej pamięci \textsc{ram}, czy komputer dysponuje
    klawiaturą i~innymi podstawowymi urządzeniami oraz sprawdza czy
    urządzenia te odpowiadają w~sposób prawidłowy. W~pierwszej
    kolejności skanowane są magistrale \textsc{isa}
    (ang. \textit{Industry Standard Architecture}) i~\textsc{pci}
    (ang.~\textit{Peripheral Component Interconnect}) w~celu wykrycia
    podłączonych do nich urządzeń.

  \item[3)] Jeśli do komputera podłączone są inne urządzenia, niż te które
    były dostępne przy jego ostatni uruchomieniu, nowe urządzenia są
    konfigurowane.

  \item[4)] Program \textsc{bios} odczytuje listę tzw. urządzeń rozruchowych
    z~pamięci \textsc{cmos}. Urządzenia rozruchowe to te, które mogą
    zawierać system operacyjny. W~przeszłości były nimi dyskietki, płyty
    \textsc{cd}-\textsc{rom}, \textsc{dvd}, dziś choćby pendriwy
    i~dyski~\textsc{ssd}.

  \end{itemize}

\end{frame}
% ##################





% ##################
\begin{frame}
  \frametitle{Uruchamianie komputera z~systemem Pentium}


  \begin{itemize}

  \item[5)] \textsc{bios} testuje po kolei urządzenia rozruchowe
    z~wspomnianej wcześniej listy, aż~znajdzie pierwsze, który zawiera
    działający system operacyjny.

  \item[6)] \textsc{bios} wczytuje pierwszy sektor ze~znalezionego
    w~poprzednim punkcie urządzenia rozruchowego do pamięci i~go uruchamia.

  \item[7)] Program wczytany z~pierwszego sektora sprawdza zapisaną na jego
    końcu listę partycji, by~ustalić która z~nich jest partycją aktywną.
    Następnie wczytuje z~tej partycji pomocniczy program rozruchowy.

  \item[8)] Pomocniczy program rozruchowy wczytuje system operacyjny
    z~aktywnej partycji i~go uruchamia.

  \item[9)] System operacyjny odczytuje informacje konfiguracyjne z~systemu
    \textsc{bios}. Dla każdego dostępnego urządzenia sprawdza, czy posiada
    jego sterowniki. Jeśli nie, to prosi o~ich zainstalowanie
    z~odpowiedniego źródła.

  \end{itemize}

\end{frame}
% ##################





% ##################
\begin{frame}
  \frametitle{Uruchamianie komputera z~systemem Pentium}


  \begin{itemize}

  \item[10)] Jeśli system operacyjny dysponuje wszystkimi sterownikami,
    to ładuje je do jądra systemu.

  \item[11)] System operacyjny tworzy tabele systemowe oraz procesy
    działające w~tle.

  \item[12)] Uruchamiane jest okno logowania.

  \end{itemize}

\end{frame}
% ##################






% ##################
\begin{frame}
  \frametitle{Bootowanie}


  W~literaturze funkcjonuje termin \textbf{bootwoanie}, zwane też
  \textbf{uruchamianiem} lub \textbf{rozruchem}. Odnosi~się ono albo do
  całej procedury uruchamiania komputer opisanej powyżej, albo tylko
  stawiania systemu operacyjnego, czyli od kiedy \textsc{bios} wczytał
  pierwszy jego sektor do pamięci (punkt siedem i~dalej). Acz to pojęcie
  nie jest specjalnie ostro zdefiniowane.

\end{frame}
% ##################





% ##################
\begin{frame}
  \frametitle{Czy uruchomienie komputera jest proste czy trudne?}


  Zależy jak do tego podchodzimy. I~tak jest z~większością rzeczy
  w~informatyce.

\end{frame}
% ##################








% ######################################
\section{Dygresja o~diagramach występujących w~tych
  prezentacjach}
% ######################################



% ##################
\begin{frame}
  \frametitle{Konwencja}


  \begin{textblock}{2.8}(2,1.5)

    \begin{tikzpicture}

      \node[diagram block] at (0,0) {Coś robi};

    \end{tikzpicture}

  \end{textblock}



  \begin{textblock}{2.8}(8,1.5)

    \begin{tikzpicture}

      \node[diagram rectangle block] at (0,0) {Czymś jest};

    \end{tikzpicture}

  \end{textblock}


  \vspace{6em}





  Bloki diagramu które mają kształt prostokąta z~zaokrąglonymi rogami
  (zwykle~są koloru niebieskiego) oznaczają \textbf{aktorów}, czyli
  taki obiekt który wykonuje jakąś czynność na zadanych obiektach
  wejściowych. Takim obiektem może być człowiek, komputer, program
  komputerowy, fragment programu komputerowego, etc.

  Bloki diagramu które mają kształt prostokąta z~ostrymi rogami (zwykle~są
  koloru karmelowego), przedstawiają rzeczy które są pobierane na~wejście
  przez aktorów lub przez nich wytwarzane.

\end{frame}
% ##################





% ##################
\begin{frame}
  \frametitle{Przykładowy diagram}


  Konwencję tą ilustruje poniższy, zrobionym z~przymrużeniem oka, diagram.
  Przedstawia on proces tworzenia przez człowieka za pomocą klawiatury
  i~komputera programu w~języku programowania~C.





  \begin{figure}

    \label{fig:Creating-code-in-C}


    \begin{tikzpicture}

      \node[diagram block] (Man) at (0,0) {Człowiek};

      \node[diagram block,right=of Man] (Keyboard) {Klawiatura};

      \draw[thick diagram arrow] (Man) -- (Keyboard);



      \node[diagram block,right=of Keyboard] (Computer) {Komputer};

      \draw[thick diagram arrow] (Keyboard) -- (Computer);



      \node[diagram rectangle block,below=of Computer] (Source code)
      {Kod źródłowy w~języku~C};

      \draw[thick diagram arrow] (Computer) -- (Source code);

    \end{tikzpicture}

    \caption{Diagram ilustrujący tworzenie kodu w~języku~C.}


  \end{figure}

\end{frame}
% ##################





% ##################
\begin{frame}
  \frametitle{Inny diagram}


  Czy poniższy diagram jest lepszy czy gorszy od poprzedniego? Zależy
  od~tego jakie informacje ma dany diagram przekazywać. Proszę~się więc nie
  dziwić, jeśli na danym diagramie czegoś nie ma, a~nawet, że~nie ma czegoś
  co było na poprzedniej wersji diagramu. Zawsze trzeba dokonać selekcji co
  na diagramie zostanie umieszczone, a~co nie i~ten wybór często zależy
  od kontekstu.

  Jeśli uważają Państwo, że~na diagramie nie ma czegoś, co być powinno,
  albo coś innego jest nie w~porządku, \alert{proszę} to mi powiedzieć.
  Całkiem możliwe, że~popełniłem błąd rysując dany diagram.






  \begin{figure}

    \label{fig:Also-creating-code-in-C}


    \begin{tikzpicture}

      \node[diagram block] (Man) at (0,0) {Człowiek};

      \node[diagram rectangle block,right=of Man] (Source code)
      {Kod źródłowy w~języku~C};

      \draw[thick diagram arrow] (Man) -- (Source code);

    \end{tikzpicture}

    \caption{Inna wersja diagramu ilustrującego tworzenie kodu w~języku~C.}


  \end{figure}

\end{frame}
% ##################





% % ##################
% \jagiellonianendslide{Czy są jakieś pytania do tej części?}
% % ##################










% ######################################
\section{Podstawowe informacje o~języku~C}
% ######################################








% ##################
\begin{frame}
  \frametitle{Przesadnie pedantyczna lista featurów języka~C}


  \begin{itemize}

  \item[1)] Zmienne.

  \item[2)] Typy zmiennych: \texttt{char}, \texttt{short int}, \texttt{int},
    \texttt{long int}, \texttt{float}, \ldots

  \item[3)] Operator przypisania: \texttt{=}.

  \item[4)] Operator arytmetyczne: \texttt{+}, \texttt{-}, \texttt{*},
    \texttt{/}, \texttt{\%}.
    (Jeśli Państwo myślą, że~to banalnie proste, to~się Państwo
    nieprzyjemnie zdziwią.)

  \item[5)] Operatory porównania: \texttt{==} (nie pomylić z~operatorem
    przypisania~\texttt{=}),

  \item[6)] Instrukcje warunkowe: \texttt{if \ldots{} else \ldots}, \texttt{switch}.

  \item[7)] Pętle: \texttt{while}, \texttt{for}, \texttt{do \ldots{} while}.

  \end{itemize}

\end{frame}
% ##################





% ##################
\begin{frame}
  \frametitle{?????}


  Nie potrafię teraz powiedzieć, czy zdążymy, choćby bardzo pobieżnie,
  przerobić \alert{wszystkie} featury języka~C. Proszę pamiętać, że~to jest
  kurs \alert{podstaw informatyki}, więc nie należy~się spodziewać,
  że~nauczymy~się tu wszystkiego. Nawet o~podstawach informatyki.

  Lubię myśleć o~języku programowania jako o~skrzynce z~narzędziami.
  Im lepiej znasz swoją skrzynkę, tym prościej ci wykonać dane zadanie.
  Przykładowo, można wyciągnąć śrubę ze~ściany obcęgami, ale lepiej
  wiedzieć, że w~skrzynce jest śrubokręt.

  Tak samo proszę myśleć o~naszej nauce języka~C. Na początku będziemy
  poznawali pojedyncze „narzędzia” jakie są w~tej „skrzynce” są, co niestety
  dla wielu osób jest nudne. Dopiero jednak jak będziemy znali ich
  odpowiednią liczbę ma sens zabieranie~się

\end{frame}
% ##################
















% % ##################
% \jagiellonianendslide{Czy są jakieś pytania do tej części?}
% % ##################










% ######################################
\section{C~to teraźniejszość. A~co jest przyszłością?}
% ######################################



% ##################
\begin{frame}
  \frametitle{C ma ponad 50~lat, co dalej?}


  Jak powiedział amerykański filozof i~bejsbolista Yogi Berra
  (właśc. Lawrence Peter Berra, $1925 \text{--} 2015$) „Przewidywanie jest
  bardzo trudne. Zwłaszcza jeśli próbuje~się przewidywać przyszłość.”
  Dlatego muszę zastrzec, że~to co mówią odnośnie przyszłości, są to moje
  osobiste refleksje. Nie mogę dać żadnej gwarancji, że~się sprawdzą.

  Język~C stworzony w~$1972$ roku, ma już ponad $50$~lat, więc to jasne,
  że~pojawiają~się pomysły na języki które mają go zastąpić. W~pewnym sensie
  C++ jest takim językiem. Jednak od momentu pojawienia~się C++, za taką
  datę możemy przyjąć $1985$~rok, minęło sporo czasu, zamiast zastąpić~C,
  język ten żyje obok niego. Ale to bardziej temat na zajęcia o~C++.

  Proszę pamiętać, że~relacja między~C, a~C++, jest jak między krzesłem,
  a~krzesłem elektrycznym.

\end{frame}
% ##################





% ##################
\begin{frame}
  \frametitle{C~nie będzie wieczny, ale\ldots}


  Przez ponad pół wieku~(!) w~C napisano niewiarygodne ilość rewelacyjnie
  działającego kodu (co nie znaczy, że~kod ten jest pozbawiony błędów
  i~wad). Ponieważ utrzymywanie starego kodu jest zwykle prostsze niż
  pisanie nowego, który robi to samo, więc część z~tego kody będzie
  utrzymywana tak długo jak tylko~się będzie dało. Przykładowo, jądro
  systemu operacyjnego Linux jest napisane w~$98\%$ w~języku~C (zobacz
  statystki jego \colorhref{https://github.com/torvalds/linux}{repozytorium}
  na~GitHubie). A~bez tego jądra internet pewnie~się zawali.

  Trzeba też pamiętać, że~cała masa innych języków programowania zależna
  jest od kodu napisanego w~dużej części w~C. Przykładowo, szeroko stosowana
  w~obliczeniach numerycznych biblioteka \texttt{numpy} języka Python,
  w~35\% jest napisana w~języku~C (zob. jego
  \colorhref{https://github.com/numpy/numpy}{repozytorium} na~GitHubie).

  Przyczyna tego jest zwykle prosta: C~pozwala tworzyć bardzo szybki kod.
  Kod napisany w~czystym Pythonie jest niestety relatywnie wolny.

\end{frame}
% ##################





% ##################
\begin{frame}
  \frametitle{Kilka pojęcia}


  Dobrze, ale co z~kodem napisanym w~C, który chcemy zastąpić nowym oraz
  nowo tworzonym kodem? Czy tutaj język~C jest dalej niezastąpiony? Krótka
  i~bardzo uproszczona odpowiedź jest taka: C~jest tam już zastępowany
  nowymi językami, acz sprawa toczy~się stosunkowo wolno. W~tym momencie
  powiedziałbym, że~C wciąż ma przyszłość.

  Patrząc trochę w~przyszłość, moja opinia na dzień dzisiejszy jest taka,
  że~języka~C \alert{nie} zastąpi jeden język, tylko kilka. Każdy z~nich
  będzie dopasowany do konkretnej dziedziny
  zastosowań, którą obecnie obejmuje (przynajmniej częściowo) język~C.

  W~mojej ocenie, 80\% dotychczasowego pola zastosowań~C przejmą trzy
  języki: \colorhref{https://go.dev/}{Go},
  \colorhref{https://www.rust-lang.org/}{Rust}
  i~\colorhref{https://ziglang.org/}{Zig}. Pozostałe 20\% przyjmie kilka
  innych, wysoce wyspecjalizowanych języków, ale tutaj nie mam odwagi
  zgadywać, jakie to będą.

\end{frame}
% ##################





% ##################
\begin{frame}
  \frametitle{????}


  Jeśli ktoś chce rozszerzyć swoją wiedzę o~jeden z~wymienionych wcześniej
  języków, Go, Rust i~Zig, to proponuję zacząć od~języka~Go. Jest on
  stosunkowo mały, wygodny w~użyciu i~najbardziej z~całej trójki zbliżony
  do~C, więc studiowanie go równolegle z~C, powinno wymagać najmniej pracy.

  Instrukcje jak pobrać i~zainstalować język~Go, można znaleźć
  \colorhref{https://go.dev/learn/}{tutaj}.

  Gdy chodzi o~materiały do nauki, to początkującym polecam zacząć od
  \colorhref{https://www.youtube.com/watch?v=akosxcqJorU}{\textit{Go for
      Absolute Beginners~-- Tutorial}} kanału freeCodeCamp.org
  i~kursu
  \colorhref{https://www.youtube.com/watch?v=WVjc_wl17FA}{\textit{Go
      tutorial for beginners}} Thomasa Langhorsta. Osoby bardziej
  zaawansowane mogą sięgnąć od razu po oficjalne wprowadzenie do~języka
  \colorhref{https://go.dev/tour/welcome/1}{\textit{A~Tour~of Go}}
  (większość tego wprowadzenia jest dostępna również w~języku polskim).

\end{frame}
% ##################










% ######################################
\section{Dlaczego istnieją języki takie jak~C?}
% ######################################



% ##################
\begin{frame}
  \frametitle{Dlaczego w~ogóle istnieją takie języki jak~C?}

  \pause


  Żeby komputer coś zrobił musimy więc mu przekazać polecenie w~języku,
  który rozumie. Dla mnie pierwotny językiem jest język polski, co zaś
  jest pierwotnym językiem komputera? Na potrzeby tego kursu przyjmiemy,
  że~pierwotnym językiem komputera jest \textbf{język asembler}
  (ang. \textit{assembly language}).

  Tak jak niektórzy ludzie mają jako swój pierwotny język angielski,
  hiszpański, japoński, niemiecki, polski, etc., tak komputer również
  posługują~się różnymi rodzajami języka asembler, takimi jak \textsc{arm}
  czy x86/Intel. Typ asemblera jest ustalony przez firmę, która
  wyprodukowała procesor, bowiem asembler jest zdefiniowany przez to jak są
  „podpięte kable” w~rzeczonym procesorze.

  Tutaj pojawia~się pewien problem. Przyjrzyjmy~się teraz programowi
  „Hello, World!” napisanemu w~dialektach asemblera \textsc{arm}~32,
  AArch64, \textsc{risc-v}, x86 i~\textsc{nasm} oraz językach~C
  i~Python.

\end{frame}
% ##################





% ##################
\begin{frame}
  \frametitle{„Hello, World!” w~asemblerze ARM~32,
    \parencite{Low-Level-Learning-You-Can-Learn-ARM-ETC-Ver-2020}}


  \texttt{.global \_start} \\
  \texttt{.section .text} \\



  \texttt{\_start:} \\[-0.2em]
  \hphantom{aaaaaaaa} \texttt{mov r7, \#0x4} \\
  \hphantom{aaaaaaaa} \texttt{mov r0, \#1} \\
  \hphantom{aaaaaaaa} \texttt{ldr r1, =message} \\
  \hphantom{aaaaaaaa} \texttt{mov r2, \#14} \\

  \hphantom{aaaaaaaa} \texttt{svc 0} \\

  \hphantom{aaaaaaaa} \texttt{mov x8, \#0x5d} \\
  \hphantom{aaaaaaaa} \texttt{mov x0, \#0x41} \\

  \hphantom{aaaaaaaa} \texttt{svc 0} \\



  \texttt{.section .data} \\
  \texttt{message:} \\
  \hphantom{aaaaaaaa} \texttt{.ascii "Hello, World!\textbackslash n"}

\end{frame}
% ##################





% ##################
\begin{frame}
  \frametitle{„Hello, World!” w~asemblerze AArch64,
    \parencite{Low-Level-Learning-You-Can-Learn-AArch64-ETC-Ver-2020}}


  \texttt{.global \_start} \\
  \texttt{.section .text} \\



  \texttt{\_start:} \\[-0.2em]
  \hphantom{aaaaaaaa} \texttt{mov x8, \#64} \\
  \hphantom{aaaaaaaa} \texttt{mov x0, \#1} \\
  \hphantom{aaaaaaaa} \texttt{ldr x1, =message} \\
  \hphantom{aaaaaaaa} \texttt{mov x2, \#14} \\

  \hphantom{aaaaaaaa} \texttt{swi 0} \\

  \hphantom{aaaaaaaa} \texttt{mov r7, \#0x1} \\
  \hphantom{aaaaaaaa} \texttt{mov r0, \#65} \\

  \hphantom{aaaaaaaa} \texttt{swi 0} \\



  \texttt{.section .data} \\
  \texttt{message:} \\
  \hphantom{aaaaaaaa} \texttt{.ascii "Hello, World!\textbackslash n"}

\end{frame}
% ##################





% ##################
\begin{frame}
  \frametitle{„Hello, World!” w~assemblerze RISC-V
    \parencite{Low-Level-Learning-You-Can-Learn-RISC-V-ETC-Ver-2021}}


  \texttt{.global \_start} \\
  \texttt{\_start:} \\
  \hphantom{aaaa} \texttt{addi a7, zero, 64} \\
  \hphantom{aaaa} \texttt{addi a0, zero, 1} \\
  \hphantom{aaaa} \texttt{la a1, helloworld} \\
  \hphantom{aaaa} \texttt{addi a2, zero, 14} \\
  \hphantom{aaaa} \texttt{ecall}
  \vspace{0.8em}

  \hphantom{aaaa} \texttt{addi a7, zero, 93} \\
  \hphantom{aaaa} \texttt{addi a0, zero, 13} \\
  \hphantom{aaaa} \texttt{ecall}
  \vspace{0.8em}

  \texttt{helloworld:} \\
  \hphantom{aaaa} \texttt{.ascii "Hello, World!\textbackslash n"}

\end{frame}
% ##################





% ##################
\begin{frame}
  \frametitle{„Hello, World!” w~assemblerze NASM
    \parencite{Toal-NASM-Tutorial-Ver-2024}}


  \hphantom{aaaaaaaaa} \texttt{global} \hphantom{aa} \texttt{\_start} \\
  \vspace{0.8em}

  \hphantom{aaaaaaaaa} \texttt{section} \hphantom{a} \texttt{.text} \\
  \texttt{\_start:} \hphantom{a} \hspace{-0.15em}
  \texttt{mov} \hphantom{aaaaaa} \texttt{rax, 1} \\
  \hphantom{aaaaaaaaa} \texttt{mov} \hphantom{aaaaaa} \texttt{rdi, 1} \\
  \hphantom{aaaaaaaaa} \texttt{mov} \hphantom{aaaaaa}
  \texttt{rsi, message} \\
  \hphantom{aaaaaaaaa} \texttt{mov} \hphantom{aaaaaa} \texttt{rdx, 14} \\
  \vspace{0.8em}

  \hphantom{aaaaaaaaa} \texttt{syscall} \\
  \vspace{0.8em}

  \hphantom{aaaaaaaaa} \texttt{mov} \hphantom{aaaaaa} \texttt{rax, 60} \\
  \hphantom{aaaaaaaaa} \texttt{xor} \hphantom{aaaaaa} \texttt{rdi, rdi} \\
  \vspace{0.8em}

  \hphantom{aaaaaaaaa} \texttt{syscall} \\
  \vspace{0.8em}

  \hphantom{aaaaaaaaa} \texttt{section .data} \\
  \vspace{0.8em}

  \texttt{message: db} \hphantom{aaaaa} \texttt{"Hello, World!", 10}

\end{frame}
% ##################





% ##################
\begin{frame}
  \frametitle{„Hello, World!” w~assemblerze x86
    \parencite{Anonymous-Hello-World-in-x86-Assembly-Language}}


  \texttt{org 0x100} \\
  \vspace{0.8em}

  \texttt{mov dx, msg} \\
  \texttt{mov ah, 9} \\
  \texttt{int 0x21} \\
  \vspace{0.8em}

  \texttt{mov ah, 0x4c} \\
  \texttt{int 0x21} \\
  \vspace{0.8em}

  \texttt{msg db 'Hello, World!', 0x0d, 0x0a, '\$'}

\end{frame}
% ##################





% ##################
\begin{frame}
  \frametitle{„Hello, World!” w~języku~C}


  \texttt{\#include <stdio.h>} \\
  \vspace{0.8em}
  \texttt{int main() \{ } \\
  \hphantom{aaaa} \texttt{printf("Hello, World!\textbackslash n");} \\
  \vspace{0.8em}
  \vspace{0.8em}
  \vspace{0.8em}
  \vspace{0.8em}
  \hphantom{aaaa} \texttt{return 0;} \\
  \texttt{ \} }

\end{frame}
% ##################





% ##################
\begin{frame}
  \frametitle{„Hello, World!” w~języku~Python}


  \texttt{print("Hello, World!")}

\end{frame}
% ##################





% ##################
\begin{frame}
  \frametitle{Co z~tego wynika?}


  Na pierwszy rzut oka powinno być jasne, że~C jest prostszy od~asemblera,
  a~Python prostszy od~C.

  Większość ludzi zrobi wszystko, by tylko uniknąć pracy w~asemblerze.
  Mam nadzieję, że~nie muszę tłumaczyć dlaczego.  A~nawet jeśli
  znajdzie~się człowieka, który lubi w~nim pisać, to unika~się jak tylko
  można tego, by kod napisany przez niego w~asemblerze wszedł w~skład
  danego programu.

  Jest tak dlatego, że~nawet najlepsi programiści zbyt łatwo mylą~się
  pisząc w~asemblerze, a~pomyłki na jego poziomie są szczególnie
  niebezpieczne. Niestety, wyjaśnienie czemu tak jest, to temat na osobny
  wykład. W~skrócie, błędy w~asemblerze to raj dla hakerów.

  \alert{Ważne.} Czy od tego momentu kursu możemy uznać, że~język
  asemblera sam w~sobie nas zupełnie nie interesuje?

\end{frame}
% ##################





% ##################
\begin{frame}
  \frametitle{Jak sobie z~tym poradzić?}


  Jeśli jednak ktoś z~Państwa chce~się dowiedzieć więcej o~asemblerze,
  to bardzo mnie to cieszy, ale musimy to przełożyć na czas po zajęciach.
  Wtedy służę Państwu całą moją wiedzą.

  Może zadają sobie Państwo następujące pytanie. Jeśli Python jest prostszy
  od~C, to czemu kurs \textit{Podstawy informatyki w~języku~C} jest
  prowadzony w~C, a~nie w~Pythonie? Gdybym ja miał na to wpływ, to może
  byłby prowadzony w~Pythonie, ale nie ja o~tym decyduje.

  Jak rozwiązujemy ten problem dziś? Tworzymy język taki jak~C, piszemy
  w~nich programy, a~następnie odpowiedni program zwany
  \textbf{kompilatorem} tłumaczy ten program na język asembler.

\end{frame}
% ##################





% ##################
\begin{frame}
  \frametitle{Prosty schemat działania kompilatora}


  \begin{figure}

    \begin{tikzpicture}[node distance=0.7]

      \node[diagram rectangle block] (C source code) at (0,0)
      {Kod programu w~języku C};

      \node[diagram block,below=of C source code] (Compiler)
      {Magia (kompilator)};

      \draw[thick diagram arrow] (C source code) -- (Compiler);



      \node[diagram rectangle block,below=of Compiler] (Assembly code)
      {Kod programu w~asemblerze};

      \draw[thick diagram arrow] (Compiler) -- (Assembly code);

    \end{tikzpicture}

    \caption{Ilustracja działania kompilatora~;).}


  \end{figure}

\end{frame}
% ##################





% ##################
\begin{frame}
  \frametitle{Program „Hello, World!” skompilowany z~C\ldots}


  Kompilator tłumaczy kod w~języku źródłowym, w~naszym przypadku jest nim
  język~C, na kod w~języku wynikowym, u~nas jest to odpowiedni dialekt
  asemblera. Kompilator którego będziemy używać, to~\texttt{gcc}, akronim
  od~angielskiego \textit{\textsc{gnu} Compiler Collection}, „zbiór
  kompilatorów projektu \textsc{gnu}”.

  Żeby zobaczyć jaki kod w~języku asembler tworzy kompilator \texttt{gcc}
  należy wywołać polecenie: \\
  \texttt{\$ gcc -S program.c -o program-w-jezyku-asemblera.asm}

  Kod języka asemblera jaki zostanie utworzony zależy od miliarda rzeczy
  (typu procesora, wersji kompilatora, etc.). W~każdym razie na moim
  komputerze program „Hello, World!” kompiluje~się do następującego kodu
  języka asemblera.

\end{frame}
% ##################





% ##################
\begin{frame}
  \frametitle{Z~C do~asemblera, część~I}


  \hphantom{aaaaa} \texttt{.file} \hphantom{aaa}
  \texttt{"PI-01-A-Program-pokazowy-01.c"} \\
  \hphantom{aaaaa} \texttt{.text} \\
  \hphantom{aaaaa} \texttt{.section} \hphantom{aaaaaa} \texttt{.rodata} \\
  \texttt{.LC0:} \\
  \hphantom{aaaaa} \texttt{.string "Hello, World!"} \\
  \hphantom{aaaaa} \texttt{.text} \\
  \hphantom{aaaaa} \texttt{.globl} \hphantom{aa} \texttt{main} \\
  \hphantom{aaaaa} \texttt{.type} \hphantom{aaa}
  \texttt{main, @function} \\
  \texttt{main:} \\
  \texttt{.LFB0:} \\
  \hphantom{aaaaa} \texttt{.cfi\_startproc} \\
  \hphantom{aaaaa} \texttt{endbr64} \\
  \hphantom{aaaaa} \texttt{pushq} \hphantom{aaa} \texttt{\%rbp} \\
  \hphantom{aaaaa} \texttt{.cfi\_def\_cfa\_offset 16} \\
  \hphantom{aaaaa} \texttt{.cfi\_offset 6, -16} \\

\end{frame}
% ##################





% ##################
\begin{frame}
  \frametitle{Z~C do~asemblera, część~II}


  \hphantom{aaaaa} \texttt{movq} \hphantom{aaa} \texttt{\%rsp, \%rbp} \\
  \hphantom{aaaaa} \texttt{.cfi\_def\_cfa\_regitser 6} \\
  \hphantom{aaaaa} \texttt{leaq} \hphantom{aaa}
  \texttt{.LC0(\%rip), \%rax} \\
  \hphantom{aaaaa} \texttt{movq} \hphantom{aaa} \texttt{\%rax, \%rdi} \\
  \hphantom{aaaaa} \texttt{call} \hphantom{aaa} \texttt{puts@PLT} \\
  \hphantom{aaaaa} \texttt{movl} \hphantom{aaa} \texttt{\$0, \%eax} \\
  \hphantom{aaaaa} \texttt{popq} \hphantom{aaa} \texttt{\%rbp} \\
  \hphantom{aaaaa} \texttt{.cfi\_def\_cfa 7, 8} \\
  \hphantom{aaaaa} \texttt{ret} \\
  \hphantom{aaaaa} \texttt{.cfi\_endproc} \\
  \texttt{.LFE0:} \\
  \hphantom{aaaaa} \texttt{.size} \hphantom{aaa} \texttt{main, .-main} \\
  \hphantom{aaaaa} \texttt{.ident} \hphantom{aa}
  \texttt{"GCC: (Ubuntu 11.4.0-1ubuntu1-22.04) 11.4.0"} \\
  \hphantom{aaaaa} \texttt{.section} \hphantom{aaaaaa}
  \texttt{.note.GNU-stack,"",@progbits} \\
  \hphantom{aaaaa} \texttt{.section} \hphantom{aaaaaa}
  \texttt{.note.gnu.property,"a"}

\end{frame}
% ##################





% ##################
\begin{frame}
  \frametitle{Z~C do~asemblera, część~III}


  \hphantom{aaaaa} \texttt{.align 8} \\
  \hphantom{aaaaa} \texttt{.long} \hphantom{aaa} \texttt{1f - 0f} \\
  \hphantom{aaaaa} \texttt{.long} \hphantom{aaa} \texttt{4f - 1f} \\
  \hphantom{aaaaa} \texttt{.long} \hphantom{aaa} \texttt{5} \\
  \texttt{0:} \\
  \hphantom{aaaaa} \texttt{.string "GNU"} \\
  \texttt{1:} \\
  \hphantom{aaaaa} \texttt{.align 8} \\
  \hphantom{aaaaa} \texttt{.long} \hphantom{aaa} \texttt{0xc0000002} \\
  \hphantom{aaaaa} \texttt{.long} \hphantom{aaa} \texttt{3f - 2f} \\
  \texttt{2:} \\
  \hphantom{aaaaa} \texttt{.long} \hphantom{aaa} \texttt{0x3} \\
  \texttt{3:} \\
  \hphantom{aaaaa} \texttt{.align 8} \\
  \texttt{4:}

\end{frame}
% ##################





% ##################
\begin{frame}
  \frametitle{Program „Hello, World!” skompilowany z~C\ldots}


  \alert{Ważne.} Czy od tego momentu, możemy przyjąć, że~nie interesuje
  nas jak dokładnie kompilator działa?

  Jeśli jednak ktoś jest bardziej zainteresowany tą tematyką, to po
  zajęciach służę całą moją wiedzą.

\end{frame}
% ##################










% ######################################
\section{Fazy uruchamiania programu w~języku~C}
% ######################################



% ##################
\begin{frame}
  \frametitle{Trzy fazy tworzenie programu w~języku~C}


  By stworzyć program w~języku~C musimy przejść trzy fazy.

  \begin{enumerate}

  \item Faza pisania programu.

  \item Faza kompilacji.

  \item Faza wykonywania.

  \end{enumerate}



  \alert{Faza pisania programu.} Jest dość oczywista. Otwieramy plik
  z~kodem źródłowym poleceniem: \\
  \texttt{\$ gedit helloWorld.c \&} \\
  wpisujemy w~nim \\
  \texttt{\#include <stdio.h>} \\
  \texttt{int main() \{ } \\
  \hphantom{aaaa} \texttt{printf("Hello, World!\textbackslash n");} \\
  \hphantom{aaaa} \texttt{return 0;} \\
  \texttt{ \} } \\
  Następnie zapisujemy plik na~dysku (\texttt{Ctrl-s}). Koniec.



  % \begin{figure}

  %   \label{fig:Translator-04}

  %   \begin{tikzpicture}

  %     \node[diagram rectangle block] (Source program) at (0,0)
  %     {Program źródłowy (język~C)};

  %     \node[diagram block,right=of Source program] (Compiler)
  %     {Kompilator};

  %     \draw[thick diagram arrow] (Source program) -- (Compiler);



  %     \node[diagram rectangle block,right=of Compiler]
  %     (Target program)
  %     {Program wynikowy (dia. asemblera)};

  %     \draw[thick diagram arrow] (Translator) -- (Target program);

  %   \end{tikzpicture}

  %   \caption{Translator jako czarna skrzynka}


  % \end{figure}

\end{frame}
% ##################





% ##################
\begin{frame}
  \frametitle{Faza kompilacji i~wykonania}


  \alert{Faza kompilacji.} Kompilator analizuj niezrozumiały dla komputera
  kod w~języku~C, następnie tłumaczy go na zrozumiał dla danej maszyny kod
  w~odpowiedni dialekcie asemblera. Wynik tego tłumaczenia jest zapisywany
  w~osobnym pliku.

  Kompilator \alert{tylko} przekłada kod z~jednego języka na drugi
  i~zapisuje wynik w~odpowiednim pliku, ale \alert{nie} uruchamia
  otrzymanego programu. Używając prostego przykładu, kompilator
  przetłumaczy „Eat the soup.” na „Zjedz zupę.”, ale nie powie nikomu,
  że~ma zjeść zupę.%  Kompilator dokonuje \alert{tylko} przekładu z~jednego
  % języka na drugi i~jego zapisu do pliku, \alert{nie} uruchamia jednak tego
  % programu. Używając prostego przykładu, kompilator przetłumaczy „Eat the
  % soup.” na „Zjedz zupę.”, ale nie powie nikomu, że~ma zjeść zupę.

  % Przez wywołanie polecenia \\
  % \texttt{\$ gcc --std=c99 -pedantic helloWorld.c -o outProg} \\
  % wykonujemy fazę kompilacji kodu źródłowego \texttt{helloWorld.c}
  % dostając w~rezultacie program wynikowy \texttt{outProg}.

\end{frame}
% ##################






% ##################
\begin{frame}
  \frametitle{Jak to wygląda w~praktyce?}


  Przez wywołanie polecenia \\
  \texttt{\$ gcc --std=c99 -pedantic helloWorld.c -o outProg} \\
  wykonujemy fazę kompilacji kodu źródłowego \texttt{helloWorld.c}
  dostając w~rezultacie program wynikowy \texttt{outProg}.

  By bardziej to zobrazować wywołajmy teraz polecenie \\
  \texttt{\$ file helloWorld.c} \\
  Zakładając, że~na dysku istnieje już plik \texttt{outProg}, wprowadźmy
  podobne do poprzedniego polecenie: \\
  \texttt{\$ file outProg} \\
  Można się czuć zagubionym wśród \textsc{elf}ów, ciast (\textsc{pie})
  i~tym podobnych tworów, ale da~się z~tym żyć.

  % \alert{Faza wykonania.} Otrzymany program uruchamiamy wywołując
  % polecenie \\
  % \texttt{\$ ./outProg}

\end{frame}
% ##################





% ##################
\begin{frame}
  \frametitle{Jak to wygląda w~praktyce?}


  % Przez wywołanie polecenia \\
  % \texttt{\$ gcc --std=c99 -pedantic helloWorld.c -o outProg} \\
  % wykonujemy fazę kompilacji kodu źródłowego \texttt{helloWorld.c}
  % dostając w~rezultacie program wynikowy \texttt{outProg}.

  % By bardziej to zobrazować wywołajmy teraz polecenie \\
  % \texttt{\$ file helloWorld.c} \\
  % Zakładając, że~na dysku istnieje już plik \texttt{outProg}, wprowadźmy
  % podobne do poprzedniego polecenie: \\
  % \texttt{\$ file outProg} \\
  % Można się czuć zagubionym wśród \textsc{elf}ów, ciast (\textsc{pie})
  % i~tym podobnych tworów, ale da~się z~tym żyć.

  \alert{Faza wykonania.} Wkraczamy w~nią uruchamiając otrzymany w~fazie
  kompilacji program \texttt{outProg} za pomocą polecenia \\
  \texttt{\$ ./outProg} \\
  W~jego rezultacie system operacyjny przekierowuje kod asemblera
  zawarty w~pliku \texttt{outProg} na procesor, który następnie go wykonuje
  i~w~rezultacie na ekranie widzimy napis „Hello, World!”.

  Jeśli komuś brak jeszcze mocnych wrażeń albo chce mocno odlecieć, niech
  wywoła polecenie \\
  \texttt{\$ xxd outProg outProg\_hex.txt} \\
  a~następnie otworzy plik \texttt{outProg\_hex.txt} wywołując \\
  \texttt{\$ gedit outProg\_hex.txt}

\end{frame}
% ##################





% ##################
\begin{frame}
  \frametitle{Faza kompilacji i~faza wykonania}


  \begin{figure}

    \begin{tikzpicture}[node distance=0.5]

      \node[diagram rectangle block] (Source code) at (0,0) {Kod źródłowy};

      \node[diagram block,below=of Source code] (Compiler) {Kompilator};

      \draw[thick diagram arrow] (Source code) -- (Compiler);



      \node[diagram rectangle block,below=of Compiler] (Target program)
      {Program wynikowy};

      \draw[thick diagram arrow] (Compiler) -- (Target program);



      \node[below=0.1em of Target program] {a) Faza kompilacji};






      \begin{scope}[xshift=5cm]


        \node[diagram rectangle block] (Input) at (0,0) {Wejście};

        \node[diagram block,below=of Input] (Running target program)
        {Uruchomiony program wynikowy};

        \draw[thick diagram arrow] (Input) -- (Running target program);



        \node[diagram rectangle block,below=of Running target program]
        (Results) {Wyniki działania programu};

        \draw[thick diagram arrow] (Running target program) --
        (Results);



        \node[below=0.1em of Results] {b) Faza wykonania};

      \end{scope}

    \end{tikzpicture}

    \caption{Ilustracja podziału na fazę kompilacji i~wykonania programu}


  \end{figure}

\end{frame}
% ##################





% ##################
\begin{frame}
  \frametitle{W~naszym wypadku}


  \begin{figure}

    \begin{tikzpicture}[node distance=0.5]

      \node[diagram rectangle block] (Source code) at (0,0)
      {\texttt{helloWorld.c}};

      \node[diagram block,below=of Source code] (Compiler)
      {Kompilator \texttt{gcc}};

      \draw[thick diagram arrow] (Source code) -- (Compiler);



      \node[diagram rectangle block,below=of Compiler] (Target program)
      {\texttt{outProg}};

      \draw[thick diagram arrow] (Compiler) -- (Target program);



      \node[below=0.1em of Target program] {a) Faza kompilacji};






      \begin{scope}[xshift=5cm]


        \node[diagram rectangle block] (Input) at (0,0)
        {\texttt{outProg}};

        \node[diagram block,below=of Input] (Running target program)
        {\texttt{\$ ./outProg}};

        \draw[thick diagram arrow] (Input) -- (Running target program);



        \node[diagram rectangle block,below=of Running target program]
        (Results) {Hello, World!};

        \draw[thick diagram arrow] (Running target program) --
        (Results);



        \node[below=0.1em of Results] {b) Faza wykonania};

      \end{scope}

    \end{tikzpicture}

    \caption{Kompilacja i~wykonanie programu „Hello, World!”.}


  \end{figure}

\end{frame}
% ##################










% ######################################
\section{Jakie są inne języki programowania?}
% ######################################


% ##################
\begin{frame}
  \frametitle{Jakie są inne języki programowania?}


  Wymieńmy 20 najpopularniejszych języków programowania według indeksu
  \colorhref{https://www.tiobe.com/tiobe-index/}{\textsc{tiobe}}. Poniżej
  prezentowany jest stan tego indeksu na lipiec 2024. Podane wielkości
  procentowe przedstawiają wyliczoną przez twórców indeksu popularność
  języka.



  \begin{itemize}

  \item[1)] \colorhref{https://www.python.org/}{Python}, 16.12\%.

  \item[2)] C++, 10.34\%.

  \item[3)] C, 9.48\%.

  \item[4)] \colorhref{https://www.java.com/pl/}{Java}, 8.59\%.

  \item[5)]
    \colorhref{https://learn.microsoft.com/en-us/dotnet/csharp/}{C\#},
    6.72\%.

  \item[6)] JavaScript, 3.79\%.

  \item[7)] \colorhref{https://go.dev/}{Go}, 2.19\%.

  \item[8)] Visual Basic, 2.08\%.

  \end{itemize}

\end{frame}
% ##################





% ##################
\begin{frame}
  \frametitle{Jakie są inne języki programowania?}


  \begin{itemize}

  \item[9)] Fortran, 2.05\%.

  \item[10)] \textsc{sql}, 2.04\%.

  \item[11)] Delphi/Object Pascal, 1.89\%.

  \item[12)] \textsc{matlab}, 1.34\%.

  \item[13)] \colorhref{https://www.rust-lang.org/}{Rust}, 1.18\%.

  \item[14)] \colorhref{https://www.ruby-lang.org/en/}{Ruby}, 1.16\%.

  \item[15)] Scratch, 1.15\%.

  \item[16)] \textsc{php}, 1.15\%.

  \item[17)] \colorhref{https://www.swift.org/}{Swift}, 1.13\%.

  \item[18)] Assembly language, 1.11\%.

  \item[19)] \textsc{cobol}, 1.08\%.

  \item[20)] \colorhref{https://kotlinlang.org/}{Kotlin}, 1.05\%.

  \end{itemize}

\end{frame}
% ##################

% ##################
\begin{frame}
  \frametitle{Punktacja testów}


  Liczba punktów z~testów wliczanych do końcowej puli, w~zależności od
  wyniku testu, przedstawia~się następująco.

  \vspace{-0.3em}



  \begin{itemize}

  \item $0\%\text{--}40\%$ -- $0$ pkt.

  \item $41\%\text{--}50\%$ -- $6$ pkt.

  \item $51\%\text{--}60\%$ -- $7$ pkt.

  \item $61\%\text{--}70\%$ -- $8$ pkt.

  \item $71\%\text{--}80\%$ -- $9$ pkt.

  \item $81\%\text{--}100\%$ -- $10$ pkt.

  \end{itemize}

  \vspace{-0.3em}




  Następujące zasady powinny być dla każdego oczywiste, ale lepiej
  przedstawić je jawnie, by nie było żadnych niedomówień.

  Wszelkie sprawdziany i~testy przeprowadzane na zajęciach, lub w~trakcie
  zajęć online, mają Państwo rozwiązać \alert{samemu}, bez żadnej pomocy
  ze strony innych ludzi, internetu czy programów takich jak
  Chat\textsc{gpt}.

\end{frame}
% ##################





% % ############################
% \jagiellonianendslide{Czy są jakieś pytania do tej części?}
% % ############################


























% ######################################
\section{Charakter tych zajęć}
% ######################################


% ##################
\begin{frame}
  \frametitle{Charakter tych zajęć}


  Jest to jedna z~tych rzeczy o~której będziemy przypominali do~znudzenia.
  Ze względu na oczekiwania większości studentów, na tych zajęciach
  \alert{praktyka bierze górę nad teorią}. Czyli ważniejsze jest dla nas jak
  stworzyć program który działa i~coś robi, niż zrozumienie wszystkich
  elementów, które sprawiają, że~on działa.

  Jest to nasza świadoma decyzja, która znajduje odbicie na wszystkich
  poziomach tych zajęć, od wyboru tematów które omawiamy, do~tego jaki
  charakter mają tworzone przez nas materiały. Teorii nie można całkowicie
  wyeliminować, można jednak ograniczać jej zakres i~tą drogą idziemy.

  Możliwe jest więc, że~przez cały semestr nie wspomnimy o~kilku liniach
  kodu które ciągle będą pojawiać~się w~naszym programie, bo uznamy,
  iż~wystarczające jest przyjęcie, że~program ich potrzebuje, bez
  tłumaczenia co robią. Nasze doświadczenie mówi, że~Państwu taka sytuacja
  nie sprawi problemu.

\end{frame}
% ##################





% ##################
\begin{frame}
  \frametitle{Charakter tych zajęć}


  Niekiedy wejdziemy w~pewne zagadnienia teoretyczne, bo~są sytuacje, gdy
  niewiedza teoretyczna powoduje poważne problemy praktyczne. Jeśli ktoś
  nie wie, że~proch strzelniczy wybucha pod wpływem temperatury, to może
  nieopatrznie postawić pudełko naboi koło gorącego pieca. W~informatyce
  też zdarzają~się takie sytuacje, gdy brak wiedzy teoretycznej jest zbyt
  groźny, by można było pominąć dyskusję danego problemu.

  Jako kwestie praktyczne traktujemy też dobre praktyki programisty.
  Czyli jeśli nawet dany kod działa, ale w~dłuższej perspektywie może
  sprawiać problemy, to będziemy naciskali na jego zmianę. Wbrew
  pozorom~są to bardzo praktyczne sprawy.

  Proszę jednak pamiętać, że~ze względu na to treść tego kursu jest wysoce
  selektywna. O~wielu rzeczach z~premedytacją nie powiemy, uznając je
  za~zbyt teoretyczne.

\end{frame}
% ##################





% ##################
\begin{frame}
  \frametitle{Charakter tych zajęć}


  Może~się Państwu wydawać, że~ilość zagadnień teoretycznych na tym kursie
  i~tak jest bardzo duża. To zależy w~sporej mierze od perspektywy,
  bo~naprawdę wiele zagadnień zdecydowaliśmy~się pominąć. Poza tym,
  uważamy, że~jest bardziej praktyczną rzeczą spędzenie minuty na
  czytaniu, że~czegoś nie należy robić, niż spędzenie $10$ godziny nad
  programem, który nie działa, by zrozumieć samemu, iż~tego nie należy
  robić.

\end{frame}
% ##################





% ##################
\begin{frame}
  \frametitle{Charakter tych zajęć}




  Jeśli oczywiście ktoś chce~się zagłębić w~te bardziej teoretyczne
  zagadnienia, to oczywiście służymy całą naszą wiedzą. Można
  nas o~to pytać po zajęciach i~pisać na e-maila \email.
  Prosimy takim e-mailom nadawać tytuły typu \textit{Pytanie o~X}, tak by
  nie zaginęły one wśród kilkudziesięciu innych e-maili jakie zwykle
  dostajemy każdego miesiąca.

\end{frame}
% ##################
















% % ##################
% \begin{frame}
%   \frametitle{Prawo Kernighana}




% \end{frame}
% % ##################










% ######################################
\section{O~nauce programowania i~języku~C}
% ######################################


% ##################
\begin{frame}
  \frametitle{Semestr z~językiem~C i~co~dalej?}


  W~programie studiów jest to chyba jedyny przedmiot poświęcony nauce
  języka~C i~ktoś może na podstawie tego wyciągnąć wniosek, że~w~takim
  razie po jego zakończeniu nie jest potrzebna dalsza nauka tego języka.
  Jednak to jest błędny wniosek

  Zacznijmy od stwierdzenie, że~każdy informatyk powinien umieć programować
  w~jakimś języku, ale to \alert{nie} musi być język~C. Każdy z~Państwa
  musi sam zdecydować na nauce jakiego języka programowania~się skupi, do
  tego wrócimy za chwilę. Na razie przyjrzyjmy~się jak to wygląda
  dla języka~C.

  Wbrew pozorom~C nie jest dużym językiem. Początkujący programista może~się
  poczuć przytłoczony tym ile rzeczy z~języka~C musi~się nauczyć, to jest
  zupełnie zrozumiałe. Jednak w~porównaniu z~większością innych
  „przemysłowych” języków~C jest stosunkowo małym językiem. Mimo tego nie
  widzę szans na przerobienie go w~ciągu jednego semestru.

\end{frame}
% ##################





% ##################
\begin{frame}
  \frametitle{Semestr z~językiem~C i~co~dalej?}


  Patrząc~się jakim tempem szedł ten przedmiot w~poprzednich latach,
  szacuję, że~na przyzwoitą naukę języka~C potrzebowalibyśmy co najmniej
  dwóch semestrów. Moglibyśmy iść szybciej, gdybym część rzeczy nie robił
  na zajęciach, tylko zlecał do samodzielnego przerobienia z~domu, to
  jednak kłóci~się z~koncepcją przedmiotu \textit{\alert{Podstawy}
    informatyki}. Myślę, że~Państwo~się ze mną zgodzą.

  Proszę pomyśleć o~języku~C jako o~skrzynce z~narzędziami. Najpierw
  musimy~się dowiedzieć, że~w~skrzynce mamy metr do mierzenia,
  młotek do wbijania gwoździ, obcęgi do wyciągania gwoździ, śrubokręt do
  wkręcania śrub, etc. Kiedy już~się zapoznamy z~tymi narzędziami możemy
  przystąpić do budowania krzeseł, czy półek na książki.

  Staramy~się ułożyć plan tego przedmiotu, żeby najpierw pokazać Państwu
  czym dysponuje język~C, potem zaś za~pomocą tego budować programy, które
  coś sensownego, nawet jeśli bardzo prostego, robią.

\end{frame}
% ##################





% ##################
\begin{frame}
  \frametitle{Semestr z~językiem~C i~co~dalej?}


  Jeśli tak na to spojrzymy, to powinno być dość oczywiste, że~jak sama
  nauka narzędzi jakie daj nam język~C trochę zajmuje, to budowanie z~ich
  pomocą programów zajmie jeszcze więcej czasu. I~dopiero kiedy umiem
  zbudować kilka „krzeseł” możemy powiedzieć, że~naprawdę umiemy dany język.

  To według mnie jedyny sensowny wprowadzenia w~to czym jest programowanie
  i~by zrobić to wszystko jeden semestr to stanowczo za mało. Do tego, chcę
  żeby w~miarę możliwości na zajęciach jak najwięcej programowali Państwo
  sami, bo samodzielne pisanie programów, gdy ma~się do pomocy w~miarę
  doświadczoną osobę, to jeden z~najlepszych sposobów nauki jaki istniej.

  Mimo tych wszystkich trudność staram~się znaleźć dobry balans między nauką
  „narzędzi”, a~budowanie czegoś z~ich pomocą. Będziemy bardzo wdzięczni
  za~wszystkie uwagi jak możemy ten system poprawić (przypominam adres:
  \email).

\end{frame}
% ##################





% ##################
\begin{frame}
  \frametitle{Dobrze, ale jak się uczyć programować?}


  Układanie planu studiów to paskudnie trudny temat, więc ja nie będę
  narzekał, że~mamy tylko jeden semestr na naukę języka~C. Mówię to bez
  ironii, bo~wiem, że~sam nie ułożyłby dobrego planu studiów.

  Najlepsza radę jaką mogę Państwu dać jest taka. Proszę wybrać jeden,
  konkretny język programowania, niezależnie od tego czy jest on przerabiany
  w~planie studiów i~spróbować nauczyć~się go najlepiej jak Państwo mogą.
  Tym językiem nie musi być~C (choć bardzo nas ucieszy, jeśli ktoś go
  wybierze).

  Język~C będą Państwo musieli oczywiście trochę poznać, by zdać ten
  przedmiot, ale~nie będziemy w~żadnym razie naciskać, żeby Państwo
  go poznawali bardziej niż minimum wymagane przez ten przedmiot. Każdy
  z~Państwa musi sam podjąć decyzję, który język programowania uczyni swoim
  asem w~rękawie.

\end{frame}
% ##################





% ##################
\begin{frame}
  \frametitle{Dobrze, ale jak się uczyć programować?}


  Jeśli chodzi o~wybór języka na którym się Państwo skupią, to najlepsze co
  mogę dziś poradzić, to przyjrzenie~się indeksowi
  \colorhref{https://www.tiobe.com/tiobe-index/}{\textsc{tiobe}}
  i~wybranie jednego języka z~pierwszych dwudziestu.

  Indeks ten stara~się mierzyć popularność języków programowania, co jest
  oczywiście bardzo trudnym zadaniem, jego twórcy to wiedzą. Niemniej
  pierwsza dwudziestka to naprawdę dobre miejsce by zacząć
  szukać języka dla siebie. Do tego indeks ten jest uaktualniany
  co~miesiąc, więc dane są zawsze bardzo aktualne. W~styczniu $2025$
  pierwsza dwudziestka według niego to: Python, C++, Java, C, C\#,
  JavaScript, Go, \textsc{sql}, Visual Basic, Fortran, Delphi/Object Pascal,
  Scratch, \textsc{php}, Rust, \textsc{matlab}, Ruby, język assemblera,
  R, Swift, \textsc{cobol}.

  Jeśli chcą Państwo poznać naszą opinię na temat którego języka warto~się
  uczyć, to oczywiście służymy całą naszą wiedzą.

\end{frame}
% ##################





% ##################
\begin{frame}
  \frametitle{Praktyka czyni mistrza}


  Niezależnie od tego, który język Państwo wybiorą, to polecamy spróbować
  pisać w~nim nawet bardzo małe program, które służą do rozwiązywanie
  codziennych problemów. Gdy ja~się uczę jakiegoś języka, to jak potrzebuję
  np.~obliczyć sumę planowych wydatków, to piszę w~nim prosty program, który
  wykonuje te obliczenia. Owszem zajmuje to więcej czasu, niż zrobienie
  tego samego za pomocą gotowego programu (kalkulatora), ale też
  naprawdę dużo daje, gdy chodzi o~umiejętność programowania.

\end{frame}
% ##################










% ######################################
\section{Wymagania i~materiały do nauki}
% ######################################


% ##################
\begin{frame}
  \frametitle{Materiały do nauki, błędy i~uwagi}


  Zarówno w~materiałach z~wykładu, jak i~w~tych do ćwiczeń, były
  i~obawiam~się, że~wciąż są błędy. Pragniemy by te zajęcia i~towarzyszące
  im materiały były możliwie merytoryczne, proste, łatwe w~zrozumieniu
  i~pozbawione błędów, proszę jednak uwierzyć, że~osiągnięcie tego jest
  naprawdę trudne.

  W~razie znalezienia jakiegokolwiek błędu lub jakichkolwiek uwag
  merytorycznych do zajęć lub dostępnych materiałów proszę zgłaszać to
  przed lub po zajęciach lub też pisać pod adres \email. Proszę w~tytule
  napisać coś w~stylu „Błędy w~materiałach”, „Uwagi do zajęć”, inaczej
  email może na długo zaginąć wśród dużej ilości tych, które dostaję od
  Państwa z~rozwiązanymi zadaniami.

\end{frame}
% ##################





% ##################
\begin{frame}
  \frametitle{Styl pisania w~języku~C}


  Można i~należy zgłaszać też uwagi do materiałów z~wykładów, bezpośrednio
  do~prowadzącego albo do mnie. Mój wpływ na stan materiałów do wykładów
  jest oczywiście skromniejszy, niż~na moje własne materiały.
  Mimo tego, będę próbował nakłonić prowadzącego, by poprawiał znaleziony
  błędy i~wprowadzał zasadne zmiany.

  Mówi~się, że~w~pewnych językach programowania istnieje tylko jeden sposób
  zrobienia czegoś, w~innych (prawie) każdą rzecz można zrobić na kilka
  sposobów. Język~C należy do tej drugiej grupy. Ja na zajęciach będę
  prezentował styl i~podejście do korzystania z~języka~C, który ja uważam
  za optymalny.

  Doświadczenie poprzednich lat uczy, że~ja i~prowadzący mamy bardzo różne
  opinie na temat tego jak wygląda dobry styl pisania kodu w~C. Wiele
  z~przykładów kodu które w~poprzednich latach pojawiły~się na wykładzie,
  ja uważam za~przykład złego stylu i~złych praktyk. O~tym jeszcze za
  chwilę.

\end{frame}
% ##################





% ##################
\begin{frame}
  \frametitle{Materiały do nauki}


  Na Sake będzie dostępna w~formacie \textsc{pdf} \alert{lista zagadnień do
    opanowania z~tego przedmiotu}, która będzie główny punktem odniesieniem
  przy tworzeniu pytań testowych. Jak również dwa listy materiałów do nauki.
  Jedna lista normalna, druga dla bardzo ambitnych osób.

  Będą tam również dostępne te prezentacje w~formacie \textsc{pdf}ów.
  W~formie źródłowej (plików \LaTeX a) są dostępna na serwisie GitHub.
  Każdy kto ma na komputerze program Git i~dostęp do internetu może je
  pobrać wpisując \\
  \texttt{\$ git clone https://github.com/KZiemian/Presentation} \\
  Znajdują~się one w~katalogu „Podstawy-informatyki-ETC-Prezentacje”.
  Należy dodać, że~gałąź \texttt{main} jest zwykle w~tyle
  za~\texttt{Gałąź-robocza}.

  Można też obejrzeć to repozytorium jak normalny człowiek. Czyli
  w~przeglądarce: \\
  \colorhref{https://github.com/KZiemian/Presentation}
  {https://github.com/KZiemian/Presentation}.

\end{frame}
% ##################





% ##################
\begin{frame}
  \frametitle{Materiały do nauki}


  W~serwisie Sake są dostępne pliki zawierające kod~C wraz oraz komentarze
  objaśniające jak on działa. Będziemy~się też starali przygotować możliwie
  duży zestaw programów, które ilustrują, w~naszej ocenie, dobry styl
  pisania programów w~języku~C. Jednak to drugie zadanie jest ze względu
  na ilość możliwych wyborów, dość trudne do wykonania.

\end{frame}
% ##################





% ##################
\begin{frame}
  \frametitle{Czego od Państwa oczekujemy?}


  Proszę zwrócić uwagę, że~ze względu na charakter tych zajęć, wystarczające
  jest, żeby o~pewnych rzeczach wymienionych na liście zagadnień mieli
  Państwo bardzo ogólne i~podstawowe pojęcie. Nawet w~sytuacji, gdy na
  kursie było o~danym zagadnieniu powiedziane znacznie więcej.

  Przykładowo, jest wystarczające, żeby Państwo wiedzieli, że~kompilator
  języka~C jest to program, który przetwarza kod napisany w~języku~C
  w~program, który jest napisany w~języku zrozumiałym dla komputera.
  Nawet jeśli na zajęciach wspomnimy czym są takie części kompilatora
  jak lekser czy parser, nie jest wymagane by Państwo po tym kursie
  wiedzieli, że takie rzeczy istnieją, nie mówiąc już o~znajomości tego
  co robią.

\end{frame}
% ##################





% ##################
\begin{frame}
  \frametitle{Czego od Państwa oczekujemy?}


  Na liście zagadnień postaramy~się wyróżnić tego typu „powierzchowne”
  zagadnienia. Dodać należy, że~gdy chodzi o~pozostałe pytania, wymagana
  jest dobra znajomość na \alert{poziomie tego co było prezentowane na
    kursie}, nie zaś taka jaka jest zawarta w~standardzie języka~C.
  Bądźmy poważni, to jest tylko kurs podstaw informatyki.

  Przykładowo, jak ktoś będzie umiał napisać program który dodaje $2.71$
  do~$3.14$ to na \alert{ten} kurs wystarczy. Przynajmniej by dostać dobrą
  ocenę z~laboratoriów, ocena z~\alert{całego} kursu to inna sprawa.
  Nie musicie Państwo znać całego artykułu
  \colorhref{https://dl.acm.org/doi/10.1145/103162.103163}{\textit{Co każdy
      informatyk musi wiedzieć o~arytmetyce zmiennoprzecinkowej}}.

  % Na liście zagadnień postaramy~się wyróżnić tego typu pytania w~specjalny
  % sposób. Dodać należy, że~gdy chodzi o~pozostałe pytania, wymagana jest
  % dobra znajomość na \alert{poziomie tego co było prezentowane na kursie},
  % nie zaś taka jaka jest zawarta w~standardzie??? języka~C. Bądźmy poważni,
  % to jest tylko kurs podstaw informatyki.

\end{frame}
% ##################





% ##################
\begin{frame}
  \frametitle{Charakter tych zajęć}


  Niekiedy wejdziemy w~pewne zagadnienia teoretyczne, bo~są sytuacje, gdy
  niewiedza teoretyczna powoduje poważne problemy praktyczne. Jeśli ktoś
  nie wie, że~proch strzelniczy wybucha pod wpływem temperatury, to może
  nieopatrznie postawić pudełko naboi koło gorącego pieca. W~informatyce
  też zdarzają~się takie sytuacje, gdy brak wiedzy teoretycznej jest zbyt
  groźny, by można było pominąć dyskusję danego problemu.

  Jako kwestie praktyczne traktujemy też dobre praktyki programisty.
  Czyli jeśli nawet dany kod działa, ale w~dłuższej perspektywie może
  sprawiać problemy, to będziemy naciskali na jego zmianę. Wbrew
  pozorom~są to bardzo praktyczne sprawy.

  Proszę jednak pamiętać, że~ze względu na to treść tego kursu jest wysoce
  selektywna. O~wielu rzeczach z~premedytacją nie powiemy, uznając je
  za~zbyt teoretyczne.

\end{frame}
% ##################





% ##################
\begin{frame}
  \frametitle{Charakter tych zajęć}


  Może~się Państwu wydawać, że~ilość zagadnień teoretycznych na tym kursie
  i~tak jest bardzo duża. To zależy w~sporej mierze od perspektywy,
  bo~naprawdę wiele zagadnień teoretycznych zdecydowaliśmy~się pominąć.

  Jeśli oczywiście ktoś chce~się zagłębić w~te bardziej teoretyczne
  zagadnienia, to oczywiście służymy całą naszą wiedzą. Można
  nas o~to pytać po zajęciach i~pisać na e-maila \email.
  Prosimy takim e-mailom nadawać tytuły typu „Pytanie o~X”, tak by nie
  zaginęły one wśród kilkudziesięciu innych e-maili jakie zwykle dostajemy
  każdego miesiąca.

\end{frame}
% ##################










% ######################################
\section{Informacje dla początkujących programistów}
% ######################################



% ##################
\begin{frame}
  \frametitle{O~nazwach plików}


    \textit{Programowanie to radzenie sobie ze złożonością: złożoność
    problemu, który chcemy rozwiązać, potęguje złożoność maszyny, na które
    pracujemy. Właśnie z~uwagi na złożoność większość projektów kończy~się
    niepowodzeniem.} Bruce Eckel
  str.~$22$ \parencite{Eckel-Thinnking-in-Java-Ed-polska-Wyd-III-Pub-2003}.

  Powtórzmy kilka rzeczy. Rozszerzenie pliku, kropka i~to co następuje po
  niej w~nazwie tego pliku, informuje używane przez nas programy z~jakim
  typem pliku mają do czynienia. Dlatego każdy język programowania posiada
  własne, właściwe tylko jemu rozszerzenie. Pliki napisane w~języku~C mają
  mieć rozszerzenie~\texttt{.c}. Analogicznie, każdy program w~języku Python
  powinien~się kończyć na~\texttt{.py}, pliki w~języku C++ \texttt{.cpp},
  w~\textsc{bash}u \texttt{.sh},~etc.

  Nadanie plikowi nazwy bez rozszerzenia to nie jest dobry pomysł. Nazwa
  zawierająca spacje to proszenie~się o~problem, gdy pracujemy w~systemie
  GNU/Linux (\textsc{bash} i~inne sprawy). Nazwa pliku która zawiera
  dwie lub więcej kropek to wymysł Szatana i~wytwór piekieł. Proszę mi
  oszczędzić styczności z~tymi plugawymi tworami.

\end{frame}
% ##################





% ##################
\begin{frame}
  \frametitle{Rozszerzenia plików}


  Jak ktoś na zajęciach napisze program w~języku~C, który \alert{nie} ma
  rozszerzenia~\texttt{.c}, będzie to musiał przy mnie poprawić.
  Chyba, że~będę musiał od razu biec do kogoś innego. Ale proszę nie
  liczyć, że~zapomnę o~tej zniewadze. ;)

  Brak tego rozszerzenia sprawia, że~możemy zostać pozbawienie części
  wsparcia w~pracy z~C, jaką dają nam przeróżne programy. A~do tego
  \alert{nie} możemy dopuścić.

  Załóżmy, że~ktoś wymyśli taką szatańską nazwę \\
  \texttt{plik-numer.1.c} \\
  co jest absolutnym plugastwem. Programy których używamy mogą błędnie
  rozpoznać „\texttt{.1.c}” jako rozszerzenie pliku i~wtedy demony mają
  prawa wylecieć nam
  \colorhref{http://www.catb.org/jargon/html/N/nasal-demons.html}{z~nosa}.

  % Jak ktoś na zajęciach napisze program w~języku~C, który \alert{nie} ma
  % rozszerzenia~\texttt{.c}, będzie to musiał przy mnie poprawić.
  % Chyba, że~będę musiał od razu biec do kogoś innego. Ale proszę nie
  % liczyć, że~zapomnę o~tej zniewadze. ;)

\end{frame}
% ##################





% ##################
\begin{frame}
  \frametitle{Plugawe nazwy plików}


  W~nazwie pliku powinna być \alert{jedna i~tylko jedna kropka}, ta
  poprzedzająca nazwę rozszerzenia pliku, czyli dla nas \texttt{.c}.
  Jeśli ta zasada nie jest prawdą wszędzie na świecie, to na pewno jest
  prawdą na tych zajęciach i~proszę o~tym pamiętać.

  Jeśli ktoś prześle mi plik, którego nazwa zawiera więcej lub mniej niż
  \alert{jedną} kropkę, to zastrzegam sobie prawo odesłania mu pliku
  z~prośbą o~zmianę nazwy na~poprawniejszą. Dla mnie to nie jest sposób
  karania, tylko metoda wyrabiania w~ludziach dobrych nawyków.

\end{frame}
% ##################





% ##################
\begin{frame}
  \frametitle{Te zajęcia są prowadzone w~C, ale\ldots}


  Czy ja powinienem~się skupić na nauce języka~C? To bardzo dobre
  pytanie i~każdy z~Państwa powinien je sobie zadać.

  Każdy dobry informatyk powinien dobrze znać co najmniej jeden język
  programowania, acz niekoniecznie musi to być język~C. Zupełnie zrozumiałe
  jest, że~ktoś z~Państwa zdecyduje, iż~językiem który chce poznać jest
  jakiś inny język niż~C i~będzie~się uczył~C tylko tyle ile wymaga ten
  przedmiot. Co~więcej, w~naszej ocenie jest to bardzo dobre podejście do
  nauki.

  Jeśli nie wiedzą Państwo, na nauce którego języka należy~się teraz skupić,
  proponuję sprawdzić~$20$ najpopularniejszych języków programowania
  według indeksu
  \colorhref{https://www.tiobe.com/tiobe-index/}{\textsc{tiobe}},
  poczytać o~każdym z~nich i~wybrać ten który Państwo uważają za~najlepszy.

\end{frame}
% ##################





% ##################
\begin{frame}
  \frametitle{Te zajęcia są prowadzone w~C, ale\ldots}


  W~grudniu $2024$ roku indeks
  \colorhref{https://www.tiobe.com/tiobe-index/}{\textsc{tiobe}}
  przedstawia~się w~następujący sposób: Python, C++, Java, C, C\#,
  JavaScript, Go, \textsc{sql}, Visual Basic, Fortran, Delphi/Object
  Pascal, \textsc{php}, Scratch, Rust, \textsc{matlab}, R, język asemblera,
  Ruby, \textsc{cobol}, Swift.

  Decyzję o~tym na nauce którego języka~się Państwo skupią, muszą
  oczywiście podjąć Państwo sami. Jeśli ktoś chce uzyskać od nas jakieś
  wskazówki, to służymy całą swoją osobą. Jeśli chcą Państwo napisać emaila
  w~tej sprawie to proszę go nazwać „Pytanie o~wybór języka do nauki”, czy
  coś podobnego.

  Bardzo~się ucieszymy, jeśli ktoś z~Państwa zdecyduje~się skupić na
  języku~C, ale to w~żadnym razie nie jest wymagane. Nie o~to chodzi w~nauce
  informatyki.

\end{frame}
% ##################





% ##################
\begin{frame}
  \frametitle{Tak to wygląda}

  \vspace{-0.5em}


  \begin{figure}

    \label{fig:Learning-any-language-at-100-procent}

    \centering


    \includegraphics[scale=0.195]
    {./Presentations-pictures/Learning-language-at-100.jpg}

  \end{figure}

\end{frame}
% ##################





% ##################
\begin{frame}
  \frametitle{Kilka rad praktycznych}


  Edytor \textsc{gnu} \colorhref{https://www.nano-editor.org/}{nano} to nie
  jest, w~mojej ocenie, dobry edytor dla osób początkujących. Zbyt dużo
  czasu tracimy na przełączaniu~się między nim, a~powłoką, zamiast na
  próbie zrozumienia jak należy pisać programy komputerowe. Jeśli ktoś jest
  już zaawansowany programistą~C (czy jakiegokolwiek innego języka) lub
  wyjątkowo lubi nano, to oczywiście, może i~powinien z~niego korzystać.
  Po prostu uważam, że~dla osób początkujących to~nie jest zbyt dobry wybór.

  Dla tych którzy nigdy wcześniej nie programowali polecam edytor
  \colorhref{https://gedit-text-editor.org/}{gedit}, gdyż jest prosty
  w~obsłudze i~nie trzeba go samemu ustawiać. Potem warto rozważyć użycie
  potężniejszej technologii.

  Gdy chodzi o~edytory to ja normalnie używam
  \colorhref{https://www.gnu.org/software/emacs/}{\textsc{gnu} Emacsa},
  który jest niezwykle fajny, ale ogromnie skomplikowany w~użyciu.
  Tak trudny, że~aby go dobrze opanować, trzeba~się nauczyć nowego języka
  programowania o~nazwie \colorhref
  {https://www.gnu.org/software/emacs/manual/html_node/eintr/}{Emacs Lisp}
  (w~skrócie: elisp). I~to wcale nie jest żart. Oprócz tego legendarny
  \colorhref{https://www.vim.org/}{vim} wciąż ma wierną rzeszę fanów.

\end{frame}
% ##################





% ##################
\begin{frame}
  \frametitle{Kilka rad praktycznych}


  \colorhref{https://www.vim.org/}{Vima} warto znać, bo jest praktycznie
  wszędzie, gdzie jest system GNU/Linux, a~takich edytorów nie ma wiele.
  Trzeba jednak wiedzieć, że~jest on dość trudny w~obsłudze.

  Odkładając na bok edytory, warto zwrócić uwagę na zintegrowane środowiska
  programistyczne (\textsc{ide}, ang.~\textit{Integrated Development
    Environment}). Dla języków C i~C++ jest dedykowane \textsc{ide}
  \colorhref{https://www.jetbrains.com/clion/promo/}{CLion} od firmy
  \colorhref{https://www.jetbrains.com/}{JetBrains}, z~którego mogą Państwo
  korzystać dzięki licencji uczelnianej. Warto~się mu przyjrzeć, jeśli
  ktoś chce programować w~C z~wykorzystaniem \textsc{ide}.

  Wybór oczywiście pozostawiamy Państwu. Jeśli ktoś chce więcej porad
  od nas, to oczywiście służymy całą naszą wiedzą.

\end{frame}
% ##################










% ######################################
\section{Kilka rad odnośnie języka~C}
% ######################################


% ##################
\begin{frame}
  \frametitle{Kilka rad odnośnie języka~C}


  Jeśli chcesz użyć \texttt{\#define} zastanów~się, czy nie można
  tego zrobić korzystając z~\texttt{const}. Jeśli można, to skorzystaj
  z~\texttt{const}.

  Jeśli nie wiesz czy użyć \texttt{float} czy \texttt{double}, użyj
  \texttt{double}.

  Jeśli masz dobry powód by użyć \texttt{float}, użyj \texttt{float}.

  Jeśli nic Państwo z~tego nie rozumieją, to proszę~się nie martwić.
  Będziemy o~tym mówić w~trakcie tego kursu.

\end{frame}
% ##################










% ######################################
\section{Kilka informacji o~języku~C}
% ######################################



% ##################
\begin{frame}
  \frametitle{Jak długo~C jest z~nami?}


  Język~C został stworzony przez Dennis Ritchie ($1941\text{--}2011$), by
  ułatwić Kenowi Thompsonowi (ur.~$1943$), pracę nad systemem operacyjnym
  \textsc{unix}. Pierwsza jego wersja ukazała~się w~$1972$ roku.

  Język~C w $2025$ rok ma $53$ (!) lata. Jego wiek
  niestety często czuć, bo wiele funkcjonalności które dziś uważamy
  za~niezbędne w~pracy programisty albo są w~nim nieobecne, albo są,
  ale w~dość koślawej wersji. Niektórych rzeczy nie da~się w~sposób
  elegancki dodać do języka z~lat 70-tych XX wieku.

  Do tego język~C został pomyślany jako mały język, więc on z~premedytacją
  \alert{nie} wprowadził pewnych rzeczy znanych w~$1972$~roku. To
  niestety też nie ułatwia nam życia. Co gorsza, dla początkujących~C i~tak
  może sprawiać wrażenie ogromnego języka. On jest mały w~porównaniu
  z~innymi językami, ale to niewiele zmienia, gdy~się dopiera zaczyna swoją
  przygodę z~informatyką.

\end{frame}
% ##################





% ##################
\begin{frame}
  \frametitle{Czy 1972 rok był dawno temu?}


    \begin{figure}

    \centering


    \includegraphics[scale=0.3]
    {./Presentations-pictures/Meeting-01/Dennis-Ritchie-Ken-Thompson-PDP-11.jpg}

    \caption{Dennis Ritchie (stojący) i~Ken Thompson pracują na komputerze
      PDP-11 firmy Digital Equipment Corporation.}

    \label{fig:Ritchie-Thompson-PDP-11}

  \end{figure}

\end{frame}
% ##################





% ##################
\begin{frame}
  \frametitle{Zalety języka~C}


  \begin{itemize}

  \item Pozwala tworzyć bardzo szybkie programy. \alert{Bardzo szybkie}
      programy.

  \item Jeśli poznamy~C, to nauka języka który na nim bazuje będzie
    już prostsza. A~60\%, albo więcej, najważniejszych języków jakie
    istnieją, w~jakimś stopniu bazuje na~C.

  \item Na świecie jest tyle kodu napisanego w~języku~C, że~jeśli umie~się
    go dobrze, można dość łatwo znaleźć naprawdę dobrze płatną pracę. I~na
    razie nic nie wskazuje, by ta praca zniknęła przez następne $20$ lat
    (do $2044$~roku). Chat\textsc{gpt} też nie.

  \item Pozwala lepiej poznać jak komputer działa na niskim poziomie.
    Zdania uczonych są podzielone, czy to jest wada czy zaleta. Ja
    ostatnio~się skłaniam do stwierdzenia, że~to jest zaleta, ale rozumiem,
    jeśli według Państwa mylę się~totalnie.

  \item Jawne zarządzanie wskaźnikami.

  \end{itemize}

\end{frame}
% ##################







% ##################
\begin{frame}
  \frametitle{Wady języka~C}


  \begin{itemize}

  \item Pisane w~C nie jest wcale proste.

  \item Wersji języka~C, różniących~się drugorzędnymi kwestiami, jest
    zbyt dużo i~łatwo~się w~tym pogubić.

  \item C~oferuje nam bardzo mało zabezpieczeń przed nami samymi, co
    skutkuje tym, że~bardzo łatwo stworzyć kod w~którym kryją~się
    koszmarnie niebezpieczne rzeczy. To jednak temat na inny przedmiot.

  \item Jawne zarządzanie wskaźnikami. Tak, to było też w~dziale
    \textit{Zalety języka~C}.

  \item Arytmetyka wskaźników. To relikt minionej epoki.

  \item Metody operowania na stringach. A~raczej brak takich metod.

  \item Wiele standardowych funkcji języka~C \alert{nie} powinno być
    używane. Powtarzam, wielu standardowych funkcji dostępnych w~języku~C
    \alert{nie należy} używać pod żadnym pozorem.

  \end{itemize}

\end{frame}
% ##################





% ##################
\begin{frame}
  \frametitle{Wady języka~C}


  \begin{itemize}

  \item Trudne do zapamiętania nazewnictwo. Czasem nazwy w~C są zwyczajnie
    głupie, jak \texttt{unsigned char} (pl.~\textit{znak bez znaku}) na
    oznaczenie pojedynczego bajtu.

  \item  Sama nazwa języka, „C”, też jest w~sumie dość głupia. Dzisiaj
    staramy~się nadawać im nazwy będące pełnymi słowami.

  \item Brak normalnego typu boolowskiego. Nawet \texttt{stdbool.h} nie
    rozwiązuje tego problemu.

  \item Domyślna konwersja typów. To naprawdę nie pasuje do takiego
    języka jak~C.

  \item Obsługa błędów. Jak obsługa błędów?

  \item Toporny i~niewygodny model kompilacji: pliki nagłówkowe,
    preprocesor, etc.

  \item „The more I~C, the less I see.” Przykład: złożone typy danych.

  % \item Brak modułowości.

  % \item I~miliona innych udogodnień, które daje ci dobry, nowoczesny język
  %   programowania (1972 rok był dawno temu).

  \end{itemize}

\end{frame}
% ##################





% ##################
\begin{frame}
  \frametitle{The more I~C, the less I~see}


  Jak głosi mądrość internetu, języki~C nie jest trudny.

  \texttt{void (* (*f[])())()}

  Poprawna kod w~języku~C, który definiuje~\texttt{f} jako tablicę
  o~nieokreślonej długości, zawierającej wskaźniki do funkcji, które
  to funkcje zwracając wskaźniki do~funkcji, które to funkcje~są typu
  \texttt{void}.

  Taaaaak, nic trudnego.

\end{frame}
% ##################





% ##################
\begin{frame}
  \frametitle{Wady języka~C}


  \begin{itemize}

  \item Brak modułowości.

  \item I~miliona innych udogodnień, które daje ci dobry, nowoczesny język
    programowania (rok $1972$ to było kawał czasu temu).

  \end{itemize}

\end{frame}
% ##################





% ##################
\begin{frame}
  \frametitle{Jaki jest związek między C a~C++?}


  Taki jak między krzesłem, a~krzesłem elektrycznym. Bardzie trafne byłoby
  powiedzenie, że~jak, między mechaniką i~mechaniką kwantową, ale mniej
  osób rozumie tą różnicę.

  Mimo wszystko to są dwa bardzo różne języki programowania, które
  ze~względów historycznych mają pewnie wspólny rdzeń.

  Jeśli~C to taka skrzynka z~narzędziami, to C++ to cały Leroy
  Merlin. Naprawdę, ogarnięcie wszystkich featurów C++, jest zupełnie nie do
  ogarnięcia. A~przynajmniej ja w~możliwość tego osiągnięcia nie wierzę.
  Jak zapytałem osobą prowadzącą kurs C++, czy ktoś zna cały ten język
  to odpowiedziała „Podobno Bjarne Stroustrup [twórca C++]. Ale to tylko
  plotki.”

  Języki takie jak C\#, to zupełnie osobna sprawa.

\end{frame}
% ##################













% % ##################
% \begin{frame}
%   \frametitle{AI, ChatGPT,\ldots}


%   Żeby nie pozostać za bardzo w~tyle, staram~się wprowadzić do zajęć
%   praktyki związane z~korzystanie ze~sztucznej inteligencji. Niemniej
%   nawet taki gigant programowania jak Linus Torvalds (czyta się:
%   \textit{turwalds}),
%   \colorhref{https://www.youtube.com/watch?v=VHHT6W-N0ak}{zapytany na
%     przełomie} $2023$ i~$2024$ roku o~rolę \textsc{ai} w~praktyce
%   programisty, odpowiedział w~sposób, który zdradzał jego wątpliwości
%   na temat tego, jak to będzie działać w~praktyce.

%   Ja zaś~$5$ września $2024$ przeprowadziłem z~Chatem\textsc{gpt} taką
%   rozmowę.

%   \textsc{kz:} Czy w~$2024$ roku warto~się uczyć języka~C?

%   Chat\textsc{gtp:} Tak, uczenie~się C~w~roku $2024$ wciąż jest dobrym
%   pomysłem. Oto dlaczego.

%   \textbf{1.~Podstawy informatyki.} C~pozwala nam dogłębnie zrozumieć
%   architekturę komputera i~zarządzania pamięcią, które~są kluczowe
%   dla przyswojenia sobie bardziej abstrakcyjnych koncepcji z~programowania.

%   % \textbf{2.~Duża wydajność.} C~jest znany ze swej szybkości
%   % i~niskopoziomowego dostępu do sprzętu, czyniąc go idealny kandydatem
%   % do programowania systemów (np. systemów operacyjnych, systemów
%   % wbudowanych, etc.). Wciąż jest szeroko stosowany w~przemyśle lotniczym,
%   % samochodowym i~telekomunikacji, gdzie wydajność i~kontrola nad sprzętem
%   % są kluczowe.

% \end{frame}
% % ##################







% % ##################
% \begin{frame}
%   \frametitle{AI, ChatGPT,\ldots}


%   % Żeby nie pozostać za bardzo w~tyle, staram~się wprowadzić do zajęć
%   % praktyki związane z~korzystanie ze~sztucznej inteligencji. Niemniej
%   % nawet taki gigant programowania jak Linus Torvalds (czyta się:
%   % \textit{turwalds}),
%   % \colorhref{https://www.youtube.com/watch?v=VHHT6W-N0ak}{zapytany na
%   %   przełomie} $2023$ i~$2024$ roku o~rolę \textsc{ai} w~praktyce
%   % programisty, odpowiedział w~sposób, który zdradzał jego wątpliwości
%   % na temat tego, jak to będzie działać w~praktyce.

%   % Ja zaś~$5$ września $2024$ przeprowadziłem z~Chatem\textsc{gpt} taką
%   % rozmowę.

%   % \textsc{kz:} Czy w~$2024$ roku warto~się uczyć języka~C?

%   % Chat\textsc{gtp:} Tak, uczenie~się C~w~roku $2024$ wciąż jest dobrym
%   % pomysłem. Oto dlaczego.

%   % \textbf{1.~Podstawy informatyki.} C~pozwala nam dogłębnie zrozumieć
%   % architekturę komputera i~zarządzania pamięcią, które~są kluczowe
%   % dla przyswojenia sobie bardziej abstrakcyjnych koncepcji z~programowania.

%   \textbf{2.~Duża wydajność.} C~jest znany ze swej szybkości
%   i~niskopoziomowego dostępu do sprzętu, czyniąc go idealny kandydatem
%   do programowania systemów (np. systemów operacyjnych, systemów
%   wbudowanych, etc.). Wciąż jest szeroko stosowany w~przemyśle lotniczym,
%   samochodowym i~telekomunikacji, gdzie wydajność i~kontrola nad sprzętem
%   są kluczowe.



% \end{frame}
% % ##################





% % ##################
% \begin{frame}
%   \frametitle{AI, ChatGPT,\ldots}


%   Żeby nie pozostać za bardzo w~tyle, staram~się wprowadzić do zajęć
%   praktyki związane z~korzystanie ze~sztucznej inteligencji. Niemniej
%   nawet taki gigant programowania jak Linus Torvalds (czyta się:
%   \textit{turwalds}),
%   \colorhref{https://www.youtube.com/watch?v=VHHT6W-N0ak}{zapytany na
%     przełomie} $2023$ i~$2024$ roku o~rolę \textsc{ai} w~praktyce
%   programisty, odpowiedział w~sposób, który zdradzał jego wątpliwości
%   na temat tego, jak to będzie działać w~praktyce.

%   Ja zaś $5$ września $2024$ przeprowadziłem z~Chatem\textsc{gpt} taką
%   rozmowę.

%   \textsc{kz:} Czy w~$2024$ roku warto~się uczyć języka~C?

%   Chat\textsc{gtp:} Tak, uczenie~się C~w~roku $2024$ wciąż jest dobrym
%   pomysłem. Oto dlaczego.

%   \textbf{1.~Podstawy informatyki.} C~pozwala nam dogłębnie zrozumieć
%   architekturę komputera i~zarządzania pamięcią, które~są kluczowe
%   dla przyswojenia sobie bardziej abstrakcyjnych koncepcji z~programowania.

%   \textbf{2.~Duża wydajność.} C~jest znany ze swej szybkości
%   i~niskopoziomowego dostępu do sprzętu, czyniąc go idealny kandydatem
%   do programowania systemów (np. systemów operacyjnych, systemów
%   wbudowanych, etc.). Wciąż jest szeroko stosowany w~przemyśle lotniczym,
%   samochodowym i~telekomunikacji, gdzie wydajność i~kontrola nad sprzętem
%   są kluczowe.


% \end{frame}
% % ##################



% % ##################
% \begin{frame}
%   \frametitle{AI, ChatGPT,\ldots}


%   \textbf{\textsc{kz}:} Czy Chat\textsc{gpt} jest dobry w~generowaniu kodu
%   w~C?

%   \textbf{Chat\textsc{gpt}:} Chat\textsc{gpt} jest stosunkowo dobry
%   w~generowaniu kodu w~C, zwłaszcza gdy kod ten dotyczy typowych zadań
%   i~problemów. Jednak, jakość i~dokładność generowanego kodu może zależeć
%   od złożoności zadań i~dokładności przekazanych instrukcji. Tutaj
%   jest kilka czynników które należy rozważyć.

%   1.~\textbf{Składnia i~biblioteka standardowa.} Chat\textsc{gpt} potrafi
%   dobrze generować fragmenty kodu, które są zgodne ze~składnią~C oraz
%   odwołują do standardowych bibliotek. Na przykład, może pisać kod
%   dla podstawowych operacji, struktur kontrolnych i~przetwarzania danych,
%   jak również używać standardowych funkcji z~bibliotek takich jak
%   \texttt{stdio.h}, \texttt{stdlib.h}, \texttt{string.h}, etc.

%   2.~\textbf{Powszechne schematy i~idiomy.} Potrafi rozpoznawać i~powielać
%   schemat obecne w~programach w~C, takie jak pętle dla iteracji,
%   \texttt{if-else} dla logiki warunkowej, czy \texttt{switch-case} dla
%   rozgałęzień. Potrafi również obsługiwać popularne idiomy~C, takie jak
%   używanie wskaźników, dynamiczna alokacja pamięci za pomocą \texttt{malloc}
%   i~\texttt{free} i~operacje bitowe.

%   % 3. \textbf{Rozumienie konstrukcji właściwych dla C.} Chat\textsc{gpt}
%   % zwykle dobrze sobie radzi z~konstrukcjami właściwymi dla języka C, takimi
%   % jak wskaźniki, struktury, unie, czy zarządzanie pamięcią. Potrafi również
%   % generować kod do obsługi plików \textsc{i/o}, zarządzania procesami
%   % poprzez funkcje \texttt{fork()} i~\texttt{exec()} oraz obsługujący
%   % podstawową komunikację między procesami (\textsc{ipc}, ang.
%   % \textit{Inter-Process Communication}), za pomocą takich mechanizmów jak
%   % rury (ang. \textit{pipes}) czy współdzielona pamięć.

% \end{frame}
% % ##################




% % ##################
% \begin{frame}
%   \frametitle{AI, ChatGPT,\ldots}


%   % 1.~\textbf{Składnia i~biblioteka standardowa.} Chat\textsc{gpt} potrafi
%   % dobrze generować fragmenty kodu, które są zgodne ze~składnią~C oraz
%   % odwołują do standardowych bibliotek. Na przykład, może pisać kod
%   % dla podstawowych operacji, struktur kontrolnych i~przetwarzania danych,
%   % jak również używać standardowych funkcji z~bibliotek takich jak
%   % \texttt{stdio.h}, \texttt{stdlib.h}, \texttt{string.h}, etc.

%   % 2.~\textbf{Powszechne schematy i~idiomy.} Potrafi rozpoznawać i~powielać
%   % schemat obecne w~programach w~C, takie jak pętle dla iteracji,
%   % \texttt{if-else} dla logiki warunkowej, czy \texttt{switch-case} dla
%   % rozgałęzień. Potrafi również obsługiwać popularne idiomy~C, takie jak
%   % używanie wskaźników, dynamiczna alokacja pamięci za pomocą \texttt{malloc}
%   % i~\texttt{free} i~operacje bitowe.

%   3. \textbf{Rozumienie konstrukcji właściwych dla C.} Chat\textsc{gpt}
%   zwykle dobrze sobie radzi z~konstrukcjami właściwymi dla języka C, takimi
%   jak wskaźniki, struktury, unie, czy zarządzanie pamięcią. Potrafi również
%   generować kod do obsługi plików \textsc{i/o}, zarządzania procesami
%   poprzez funkcje \texttt{fork()} i~\texttt{exec()} oraz obsługujący
%   podstawową komunikację między procesami (\textsc{ipc}, ang.
%   \textit{Inter-Process Communication}), za pomocą takich mechanizmów jak
%   rury (ang. \textit{pipes}) czy współdzielona pamięć.

%   4.~\textbf{Obsługa błędów.} Chat\textsc{gpt} może zawrzeć w~plikach
%   podstawową obsługę błędów (np. sprawdzenie, czy \texttt{malloc} zwrócił
%   \texttt{NULL}), bardziej zaawansowane zarządzanie błędami może wymagać
%   dodatkowych ulepszeń pochodzących od użytkownika.

%   5. **Advanced Topics**: For more advanced topics, like implementing network protocols, writing multi-threaded applications, or interfacing with hardware, the generated code might need more careful review and modification. ChatGPT can provide a solid starting point, but may not always capture the nuances required for robust, production-quality code.

% 6. **Code Quality**: The code generated by ChatGPT may lack certain best practices, such as efficient memory usage, proper resource management, and comprehensive error handling. Therefore, it is advisable for a human programmer to review and refine the code, especially for critical applications.

% In summary, while ChatGPT can generate functional and syntactically correct C code for a wide range of scenarios, it may not always produce optimal or fully correct solutions, particularly for more complex tasks. It's a useful tool for generating initial code snippets, learning concepts, and exploring ideas, but it should be complemented with thorough testing and review by an experienced C programmer.

% \end{frame}
% % ##################





% ######################################
\section{Czy informatyka jest trudna?}
% ######################################



% ##################
\begin{frame}
  \frametitle{Czy informatyka jest trudna?}


  Ten przedmiot dotyczy podstaw informatyki w~języku~C, warto~się
  więc spytać, czy informatyka jest prosta czy trudna w~nauce?

  Informatyka to osobna dziedzina nauki i~jeśli zabrnie~się odpowiednio
  głęboko, to robi~się naprawdę złożona i~niebanalna. Jednak na stosunkowo
  płytkim poziomie to czy jest on trudna czy nie, to mocno zależy od~odczuć
  konkretnej osoby.

  Zadam takie pytanie: czy włączenie komputera jest skomplikowane?
  Odpowiemy na to pytanie na dwóch poziomach. Pierwszy to poziom normalnego
  użytkownika, drugi to opis pochodzący z~książki Andrewa S.~Tanenbauma
  \textit{Systemy operacyjne. Wydanie~III}
  \parencite{Tannenbaum-Systemy-Operacyjne-Wydanie-III-Pub-2013}
  i~dotyczy komputera z~systemem Pentium.

\end{frame}
% ##################





% ##################
\begin{frame}
  \frametitle{Włączanie komputera, poziom normalnego użytkownika}


  \begin{enumerate}

  \item Wciskamy przycisk \texttt{Power}.

  \item Czekamy minutę albo dłużej.

  \item W~oknie logowania wybieramy użytkownika i~wchodzimy na swoje konto.

  \end{enumerate}

  Co w~tym trudnego?

\end{frame}
% ##################





% ##################
\begin{frame}
  \frametitle{Kilka pojęcia}


  Oczywiście, opis włączania komputera z~książki Tanenbauma jest tak
  skomplikowany, że~najpierw trzeba wprowadzić trochę pojęć.

  \textbf{\textsc{rom}}, ang.~\textit{Read Only Memory}, pl.~\textit{pamięć
    wyłącznie do~odczytu}. Pamięć komputera której zawartość została
  zapisana przez firmę, która ten fragment pamięci wyprodukowała
  i~użytkownik nie może zmodyfikować jej zawartości. Przynajmniej nie
  w~żaden normalny sposób.

  \textbf{\textsc{ram}}, ang.~\textit{Random Access Memory},
  pl.~\textit{pamięć o~dostępie w~trybie losowym}. Pamięć komputera o~tej
  własności, że~jeśli będę w~sposób losowy wybierał elementy tej pamięci,
  to czas odczytania informacji z~każdego z~tych elementów będzie taki sam.
  Inaczej mówiąc dostęp do dowolnego miejsca tej pamięci zajmuje tyle samo
  czasu.

  Tak naprawdę czas odczytu zależy od tego, w~jakiś sposób pamięć
  \textsc{ram} jest odczytywana, ale jeszcze długo nie będziemy się musieli
  tym przejmować.

\end{frame}
% ##################





% ##################
\begin{frame}
  \frametitle{Kilka pojęcia}


  \textbf{Pamięć ulotna}, ang.~\textit{volatile memory}. Pamięć której
  zawartość jest tracona, gdy przestaje przez nią płynąć prąd. Typowym
  przykładem takiej pamięci jest \textsc{ram}.

  \textbf{Pamięć nieulotna}, ang.~\textit{non-volatile memory}. Pamięć,
  której treść jest zachowana, gdy przez układ przestaje płynąć prąd.
  Typowym przykładem jest dysk \textsc{ssd}.

  Żeby skomplikować życie, pamięcią nieulotną nazywa~się także tą pamięć,
  które jest ulotna w~ścisłym sensie, ale ponieważ jest zaopatrzona
  we~własną baterię, jej zawartość jest zachowana również po wyłączeniu
  komputera z~prądu. Bo~niby czemu życie ma być proste?

\end{frame}
% ##################





% ##################
\begin{frame}
  \frametitle{Kilka pojęcia}


  \textbf{Pamięć \textsc{cmos}}, często po prostu \textbf{\textsc{cmos}}.
  Skrót pochodzi od angielskiej nazwy technologi \textit{Complementary
    Metal-Oxide-Semiconductor} (pl.~\textit{komplementarny półprzewodnik
    metalowo-tlenkowy}), w~której ta pamięć jest wykonana. Musi być zasilana
  prądem, by~zachowywała swój stan, ale ponieważ wyposażona jest w~baterię,
  klasyfikowana jest jako nieulotna.

  \textbf{\textsc{bios}} ang.~\textit{Basic Input Output System}, pl.
  \textit{podstawowy system wejścia, wyjścia}. Program znajdujący~się
  na płycie głównej komputera, odpowiedzialny między innymi za odczytywanie
  danych z~klawiatury, obsługę ekranu oraz operacje wejścia-wyjścia dysków.

\end{frame}
% ##################





% ##################
\begin{frame}
  \frametitle{Uruchamianie komputera z~systemem Pentium}


  \begin{itemize}

  \item[1)] Wciskamy przycisk \texttt{Power}.

  \item[2)] Z~płyty głównej ładowany jest program \textsc{bios}. Sprawdza on
    ilość zainstalowanej pamięci \textsc{ram}, czy komputer dysponuje
    klawiaturą i~innymi podstawowymi urządzeniami oraz sprawdza czy
    urządzenia te odpowiadają w~sposób prawidłowy. W~pierwszej
    kolejności skanowane są magistrale \textsc{isa}
    (ang. \textit{Industry Standard Architecture}) i~\textsc{pci}
    (ang.~\textit{Peripheral Component Interconnect}) w~celu wykrycia
    podłączonych do nich urządzeń.

  \item[3)] Jeśli do komputera podłączone są inne urządzenia, niż te które
    były dostępne przy jego ostatni uruchomieniu, nowe urządzenia są
    konfigurowane.

  \item[4)] Program \textsc{bios} odczytuje listę tzw. urządzeń rozruchowych
    z~pamięci \textsc{cmos}. Urządzenia rozruchowe to te, które mogą
    zawierać system operacyjny. W~przeszłości były nimi dyskietki, płyty
    \textsc{cd}-\textsc{rom}, \textsc{dvd}, dziś choćby pendriwy
    i~dyski~\textsc{ssd}.

  \end{itemize}

\end{frame}
% ##################





% ##################
\begin{frame}
  \frametitle{Uruchamianie komputera z~systemem Pentium}


  \begin{itemize}

  \item[5)] \textsc{bios} testuje po kolei urządzenia rozruchowe
    z~wspomnianej wcześniej listy, aż~znajdzie pierwsze, który zawiera
    działający system operacyjny.

  \item[6)] \textsc{bios} wczytuje pierwszy sektor ze~znalezionego
    w~poprzednim punkcie urządzenia rozruchowego do pamięci i~go uruchamia.

  \item[7)] Program wczytany z~pierwszego sektora sprawdza zapisaną na jego
    końcu listę partycji, by~ustalić która z~nich jest partycją aktywną.
    Następnie wczytuje z~tej partycji pomocniczy program rozruchowy.

  \item[8)] Pomocniczy program rozruchowy wczytuje system operacyjny
    z~aktywnej partycji i~go uruchamia.

  \item[9)] System operacyjny odczytuje informacje konfiguracyjne z~systemu
    \textsc{bios}. Dla każdego dostępnego urządzenia sprawdza, czy posiada
    jego sterowniki. Jeśli nie, to prosi o~ich zainstalowanie
    z~odpowiedniego źródła.

  \end{itemize}

\end{frame}
% ##################





% ##################
\begin{frame}
  \frametitle{Uruchamianie komputera z~systemem Pentium}


  \begin{itemize}

  \item[10)] Jeśli system operacyjny dysponuje wszystkimi sterownikami,
    to ładuje je do jądra systemu.

  \item[11)] System operacyjny tworzy tabele systemowe oraz procesy
    działające w~tle.

  \item[12)] Uruchamiane jest okno logowania.

  \end{itemize}

\end{frame}
% ##################






% ##################
\begin{frame}
  \frametitle{Bootowanie}


  W~literaturze funkcjonuje termin \textbf{bootwoanie}, zwane też
  \textbf{uruchamianiem} lub \textbf{rozruchem}. Odnosi~się ono albo do
  całej procedury uruchamiania komputer opisanej powyżej, albo tylko
  stawiania systemu operacyjnego, czyli od kiedy \textsc{bios} wczytał
  pierwszy jego sektor do pamięci (punkt siedem i~dalej). Acz to pojęcie
  nie jest specjalnie ostro zdefiniowane.

\end{frame}
% ##################





% ##################
\begin{frame}
  \frametitle{Czy uruchomienie komputera jest proste czy trudne?}


  Zależy jak do tego podchodzimy. I~tak jest z~większością rzeczy
  w~informatyce.

\end{frame}
% ##################








% ######################################
\section{Dygresja o~diagramach występujących w~tych
  prezentacjach}
% ######################################



% ##################
\begin{frame}
  \frametitle{Konwencja}


  \begin{textblock}{2.8}(2,1.5)

    \begin{tikzpicture}

      \node[diagram block] at (0,0) {Coś robi};

    \end{tikzpicture}

  \end{textblock}



  \begin{textblock}{2.8}(8,1.5)

    \begin{tikzpicture}

      \node[diagram rectangle block] at (0,0) {Czymś jest};

    \end{tikzpicture}

  \end{textblock}


  \vspace{6em}





  Bloki diagramu które mają kształt prostokąta z~zaokrąglonymi rogami
  (zwykle~są koloru niebieskiego) oznaczają \textbf{aktorów}, czyli
  taki obiekt który wykonuje jakąś czynność na zadanych obiektach
  wejściowych. Takim obiektem może być człowiek, komputer, program
  komputerowy, fragment programu komputerowego, etc.

  Bloki diagramu które mają kształt prostokąta z~ostrymi rogami (zwykle~są
  koloru karmelowego), przedstawiają rzeczy które są pobierane na~wejście
  przez aktorów lub przez nich wytwarzane.

\end{frame}
% ##################





% ##################
\begin{frame}
  \frametitle{Przykładowy diagram}


  Konwencję tą ilustruje poniższy, zrobionym z~przymrużeniem oka, diagram.
  Przedstawia on proces tworzenia przez człowieka za pomocą klawiatury
  i~komputera programu w~języku programowania~C.





  \begin{figure}

    \label{fig:Creating-code-in-C}


    \begin{tikzpicture}

      \node[diagram block] (Man) at (0,0) {Człowiek};

      \node[diagram block,right=of Man] (Keyboard) {Klawiatura};

      \draw[thick diagram arrow] (Man) -- (Keyboard);



      \node[diagram block,right=of Keyboard] (Computer) {Komputer};

      \draw[thick diagram arrow] (Keyboard) -- (Computer);



      \node[diagram rectangle block,below=of Computer] (Source code)
      {Kod źródłowy w~języku~C};

      \draw[thick diagram arrow] (Computer) -- (Source code);

    \end{tikzpicture}

    \caption{Diagram ilustrujący tworzenie kodu w~języku~C.}


  \end{figure}

\end{frame}
% ##################





% ##################
\begin{frame}
  \frametitle{Inny diagram}


  Czy poniższy diagram jest lepszy czy gorszy od poprzedniego? Zależy
  od~tego jakie informacje ma dany diagram przekazywać. Proszę~się więc nie
  dziwić, jeśli na danym diagramie czegoś nie ma, a~nawet, że~nie ma czegoś
  co było na poprzedniej wersji diagramu. Zawsze trzeba dokonać selekcji co
  na diagramie zostanie umieszczone, a~co nie i~ten wybór często zależy
  od kontekstu.

  Jeśli uważają Państwo, że~na diagramie nie ma czegoś, co być powinno,
  albo coś innego jest nie w~porządku, \alert{proszę} to mi powiedzieć.
  Całkiem możliwe, że~popełniłem błąd rysując dany diagram.






  \begin{figure}

    \label{fig:Also-creating-code-in-C}


    \begin{tikzpicture}

      \node[diagram block] (Man) at (0,0) {Człowiek};

      \node[diagram rectangle block,right=of Man] (Source code)
      {Kod źródłowy w~języku~C};

      \draw[thick diagram arrow] (Man) -- (Source code);

    \end{tikzpicture}

    \caption{Inna wersja diagramu ilustrującego tworzenie kodu w~języku~C.}


  \end{figure}

\end{frame}
% ##################





% % ##################
% \jagiellonianendslide{Czy są jakieś pytania do tej części?}
% % ##################










% ######################################
\section{Podstawowe informacje o~języku~C}
% ######################################








% ##################
\begin{frame}
  \frametitle{Przesadnie pedantyczna lista featurów języka~C}


  \begin{itemize}

  \item[1)] Zmienne.

  \item[2)] Typy zmiennych: \texttt{char}, \texttt{short int}, \texttt{int},
    \texttt{long int}, \texttt{float}, \ldots

  \item[3)] Operator przypisania: \texttt{=}.

  \item[4)] Operator arytmetyczne: \texttt{+}, \texttt{-}, \texttt{*},
    \texttt{/}, \texttt{\%}.
    (Jeśli Państwo myślą, że~to banalnie proste, to~się Państwo
    nieprzyjemnie zdziwią.)

  \item[5)] Operatory porównania: \texttt{==} (nie pomylić z~operatorem
    przypisania~\texttt{=}),

  \item[6)] Instrukcje warunkowe: \texttt{if \ldots{} else \ldots}, \texttt{switch}.

  \item[7)] Pętle: \texttt{while}, \texttt{for}, \texttt{do \ldots{} while}.

  \end{itemize}

\end{frame}
% ##################





% ##################
\begin{frame}
  \frametitle{?????}


  Nie potrafię teraz powiedzieć, czy zdążymy, choćby bardzo pobieżnie,
  przerobić \alert{wszystkie} featury języka~C. Proszę pamiętać, że~to jest
  kurs \alert{podstaw informatyki}, więc nie należy~się spodziewać,
  że~nauczymy~się tu wszystkiego. Nawet o~podstawach informatyki.

  Lubię myśleć o~języku programowania jako o~skrzynce z~narzędziami.
  Im lepiej znasz swoją skrzynkę, tym prościej ci wykonać dane zadanie.
  Przykładowo, można wyciągnąć śrubę ze~ściany obcęgami, ale lepiej
  wiedzieć, że w~skrzynce jest śrubokręt.

  Tak samo proszę myśleć o~naszej nauce języka~C. Na początku będziemy
  poznawali pojedyncze „narzędzia” jakie są w~tej „skrzynce” są, co niestety
  dla wielu osób jest nudne. Dopiero jednak jak będziemy znali ich
  odpowiednią liczbę ma sens zabieranie~się

\end{frame}
% ##################
















% % ##################
% \jagiellonianendslide{Czy są jakieś pytania do tej części?}
% % ##################










% ######################################
\section{C~to teraźniejszość. A~co jest przyszłością?}
% ######################################



% ##################
\begin{frame}
  \frametitle{C ma ponad 50~lat, co dalej?}


  Jak powiedział amerykański filozof i~bejsbolista Yogi Berra
  (właśc. Lawrence Peter Berra, $1925 \text{--} 2015$) „Przewidywanie jest
  bardzo trudne. Zwłaszcza jeśli próbuje~się przewidywać przyszłość.”
  Dlatego muszę zastrzec, że~to co mówią odnośnie przyszłości, są to moje
  osobiste refleksje. Nie mogę dać żadnej gwarancji, że~się sprawdzą.

  Język~C stworzony w~$1972$ roku, ma już ponad $50$~lat, więc to jasne,
  że~pojawiają~się pomysły na języki które mają go zastąpić. W~pewnym sensie
  C++ jest takim językiem. Jednak od momentu pojawienia~się C++, za taką
  datę możemy przyjąć $1985$~rok, minęło sporo czasu, zamiast zastąpić~C,
  język ten żyje obok niego. Ale to bardziej temat na zajęcia o~C++.

  Proszę pamiętać, że~relacja między~C, a~C++, jest jak między krzesłem,
  a~krzesłem elektrycznym.

\end{frame}
% ##################





% ##################
\begin{frame}
  \frametitle{C~nie będzie wieczny, ale\ldots}


  Przez ponad pół wieku~(!) w~C napisano niewiarygodne ilość rewelacyjnie
  działającego kodu (co nie znaczy, że~kod ten jest pozbawiony błędów
  i~wad). Ponieważ utrzymywanie starego kodu jest zwykle prostsze niż
  pisanie nowego, który robi to samo, więc część z~tego kody będzie
  utrzymywana tak długo jak tylko~się będzie dało. Przykładowo, jądro
  systemu operacyjnego Linux jest napisane w~$98\%$ w~języku~C (zobacz
  statystki jego \colorhref{https://github.com/torvalds/linux}{repozytorium}
  na~GitHubie). A~bez tego jądra internet pewnie~się zawali.

  Trzeba też pamiętać, że~cała masa innych języków programowania zależna
  jest od kodu napisanego w~dużej części w~C. Przykładowo, szeroko stosowana
  w~obliczeniach numerycznych biblioteka \texttt{numpy} języka Python,
  w~35\% jest napisana w~języku~C (zob. jego
  \colorhref{https://github.com/numpy/numpy}{repozytorium} na~GitHubie).

  Przyczyna tego jest zwykle prosta: C~pozwala tworzyć bardzo szybki kod.
  Kod napisany w~czystym Pythonie jest niestety relatywnie wolny.

\end{frame}
% ##################





% ##################
\begin{frame}
  \frametitle{Kilka pojęcia}


  Dobrze, ale co z~kodem napisanym w~C, który chcemy zastąpić nowym oraz
  nowo tworzonym kodem? Czy tutaj język~C jest dalej niezastąpiony? Krótka
  i~bardzo uproszczona odpowiedź jest taka: C~jest tam już zastępowany
  nowymi językami, acz sprawa toczy~się stosunkowo wolno. W~tym momencie
  powiedziałbym, że~C wciąż ma przyszłość.

  Patrząc trochę w~przyszłość, moja opinia na dzień dzisiejszy jest taka,
  że~języka~C \alert{nie} zastąpi jeden język, tylko kilka. Każdy z~nich
  będzie dopasowany do konkretnej dziedziny
  zastosowań, którą obecnie obejmuje (przynajmniej częściowo) język~C.

  W~mojej ocenie, 80\% dotychczasowego pola zastosowań~C przejmą trzy
  języki: \colorhref{https://go.dev/}{Go},
  \colorhref{https://www.rust-lang.org/}{Rust}
  i~\colorhref{https://ziglang.org/}{Zig}. Pozostałe 20\% przyjmie kilka
  innych, wysoce wyspecjalizowanych języków, ale tutaj nie mam odwagi
  zgadywać, jakie to będą.

\end{frame}
% ##################





% ##################
\begin{frame}
  \frametitle{????}


  Jeśli ktoś chce rozszerzyć swoją wiedzę o~jeden z~wymienionych wcześniej
  języków, Go, Rust i~Zig, to proponuję zacząć od~języka~Go. Jest on
  stosunkowo mały, wygodny w~użyciu i~najbardziej z~całej trójki zbliżony
  do~C, więc studiowanie go równolegle z~C, powinno wymagać najmniej pracy.

  Instrukcje jak pobrać i~zainstalować język~Go, można znaleźć
  \colorhref{https://go.dev/learn/}{tutaj}.

  Gdy chodzi o~materiały do nauki, to początkującym polecam zacząć od
  \colorhref{https://www.youtube.com/watch?v=akosxcqJorU}{\textit{Go for
      Absolute Beginners~-- Tutorial}} kanału freeCodeCamp.org
  i~kursu
  \colorhref{https://www.youtube.com/watch?v=WVjc_wl17FA}{\textit{Go
      tutorial for beginners}} Thomasa Langhorsta. Osoby bardziej
  zaawansowane mogą sięgnąć od razu po oficjalne wprowadzenie do~języka
  \colorhref{https://go.dev/tour/welcome/1}{\textit{A~Tour~of Go}}
  (większość tego wprowadzenia jest dostępna również w~języku polskim).

\end{frame}
% ##################










% ######################################
\section{Dlaczego istnieją języki takie jak~C?}
% ######################################



% ##################
\begin{frame}
  \frametitle{Dlaczego w~ogóle istnieją takie języki jak~C?}

  \pause


  Żeby komputer coś zrobił musimy więc mu przekazać polecenie w~języku,
  który rozumie. Dla mnie pierwotny językiem jest język polski, co zaś
  jest pierwotnym językiem komputera? Na potrzeby tego kursu przyjmiemy,
  że~pierwotnym językiem komputera jest \textbf{język asembler}
  (ang. \textit{assembly language}).

  Tak jak niektórzy ludzie mają jako swój pierwotny język angielski,
  hiszpański, japoński, niemiecki, polski, etc., tak komputer również
  posługują~się różnymi rodzajami języka asembler, takimi jak \textsc{arm}
  czy x86/Intel. Typ asemblera jest ustalony przez firmę, która
  wyprodukowała procesor, bowiem asembler jest zdefiniowany przez to jak są
  „podpięte kable” w~rzeczonym procesorze.

  Tutaj pojawia~się pewien problem. Przyjrzyjmy~się teraz programowi
  „Hello, World!” napisanemu w~dialektach asemblera \textsc{arm}~32,
  AArch64, \textsc{risc-v}, x86 i~\textsc{nasm} oraz językach~C
  i~Python.

\end{frame}
% ##################





% ##################
\begin{frame}
  \frametitle{„Hello, World!” w~asemblerze ARM~32,
    \parencite{Low-Level-Learning-You-Can-Learn-ARM-ETC-Ver-2020}}


  \texttt{.global \_start} \\
  \texttt{.section .text} \\



  \texttt{\_start:} \\[-0.2em]
  \hphantom{aaaaaaaa} \texttt{mov r7, \#0x4} \\
  \hphantom{aaaaaaaa} \texttt{mov r0, \#1} \\
  \hphantom{aaaaaaaa} \texttt{ldr r1, =message} \\
  \hphantom{aaaaaaaa} \texttt{mov r2, \#14} \\

  \hphantom{aaaaaaaa} \texttt{svc 0} \\

  \hphantom{aaaaaaaa} \texttt{mov x8, \#0x5d} \\
  \hphantom{aaaaaaaa} \texttt{mov x0, \#0x41} \\

  \hphantom{aaaaaaaa} \texttt{svc 0} \\



  \texttt{.section .data} \\
  \texttt{message:} \\
  \hphantom{aaaaaaaa} \texttt{.ascii "Hello, World!\textbackslash n"}

\end{frame}
% ##################





% ##################
\begin{frame}
  \frametitle{„Hello, World!” w~asemblerze AArch64,
    \parencite{Low-Level-Learning-You-Can-Learn-AArch64-ETC-Ver-2020}}


  \texttt{.global \_start} \\
  \texttt{.section .text} \\



  \texttt{\_start:} \\[-0.2em]
  \hphantom{aaaaaaaa} \texttt{mov x8, \#64} \\
  \hphantom{aaaaaaaa} \texttt{mov x0, \#1} \\
  \hphantom{aaaaaaaa} \texttt{ldr x1, =message} \\
  \hphantom{aaaaaaaa} \texttt{mov x2, \#14} \\

  \hphantom{aaaaaaaa} \texttt{swi 0} \\

  \hphantom{aaaaaaaa} \texttt{mov r7, \#0x1} \\
  \hphantom{aaaaaaaa} \texttt{mov r0, \#65} \\

  \hphantom{aaaaaaaa} \texttt{swi 0} \\



  \texttt{.section .data} \\
  \texttt{message:} \\
  \hphantom{aaaaaaaa} \texttt{.ascii "Hello, World!\textbackslash n"}

\end{frame}
% ##################





% ##################
\begin{frame}
  \frametitle{„Hello, World!” w~assemblerze RISC-V
    \parencite{Low-Level-Learning-You-Can-Learn-RISC-V-ETC-Ver-2021}}


  \texttt{.global \_start} \\
  \texttt{\_start:} \\
  \hphantom{aaaa} \texttt{addi a7, zero, 64} \\
  \hphantom{aaaa} \texttt{addi a0, zero, 1} \\
  \hphantom{aaaa} \texttt{la a1, helloworld} \\
  \hphantom{aaaa} \texttt{addi a2, zero, 14} \\
  \hphantom{aaaa} \texttt{ecall}
  \vspace{0.8em}

  \hphantom{aaaa} \texttt{addi a7, zero, 93} \\
  \hphantom{aaaa} \texttt{addi a0, zero, 13} \\
  \hphantom{aaaa} \texttt{ecall}
  \vspace{0.8em}

  \texttt{helloworld:} \\
  \hphantom{aaaa} \texttt{.ascii "Hello, World!\textbackslash n"}

\end{frame}
% ##################





% ##################
\begin{frame}
  \frametitle{„Hello, World!” w~assemblerze NASM
    \parencite{Toal-NASM-Tutorial-Ver-2024}}


  \hphantom{aaaaaaaaa} \texttt{global} \hphantom{aa} \texttt{\_start} \\
  \vspace{0.8em}

  \hphantom{aaaaaaaaa} \texttt{section} \hphantom{a} \texttt{.text} \\
  \texttt{\_start:} \hphantom{a} \hspace{-0.15em}
  \texttt{mov} \hphantom{aaaaaa} \texttt{rax, 1} \\
  \hphantom{aaaaaaaaa} \texttt{mov} \hphantom{aaaaaa} \texttt{rdi, 1} \\
  \hphantom{aaaaaaaaa} \texttt{mov} \hphantom{aaaaaa}
  \texttt{rsi, message} \\
  \hphantom{aaaaaaaaa} \texttt{mov} \hphantom{aaaaaa} \texttt{rdx, 14} \\
  \vspace{0.8em}

  \hphantom{aaaaaaaaa} \texttt{syscall} \\
  \vspace{0.8em}

  \hphantom{aaaaaaaaa} \texttt{mov} \hphantom{aaaaaa} \texttt{rax, 60} \\
  \hphantom{aaaaaaaaa} \texttt{xor} \hphantom{aaaaaa} \texttt{rdi, rdi} \\
  \vspace{0.8em}

  \hphantom{aaaaaaaaa} \texttt{syscall} \\
  \vspace{0.8em}

  \hphantom{aaaaaaaaa} \texttt{section .data} \\
  \vspace{0.8em}

  \texttt{message: db} \hphantom{aaaaa} \texttt{"Hello, World!", 10}

\end{frame}
% ##################





% ##################
\begin{frame}
  \frametitle{„Hello, World!” w~assemblerze x86
    \parencite{Anonymous-Hello-World-in-x86-Assembly-Language}}


  \texttt{org 0x100} \\
  \vspace{0.8em}

  \texttt{mov dx, msg} \\
  \texttt{mov ah, 9} \\
  \texttt{int 0x21} \\
  \vspace{0.8em}

  \texttt{mov ah, 0x4c} \\
  \texttt{int 0x21} \\
  \vspace{0.8em}

  \texttt{msg db 'Hello, World!', 0x0d, 0x0a, '\$'}

\end{frame}
% ##################





% ##################
\begin{frame}
  \frametitle{„Hello, World!” w~języku~C}


  \texttt{\#include <stdio.h>} \\
  \vspace{0.8em}
  \texttt{int main() \{ } \\
  \hphantom{aaaa} \texttt{printf("Hello, World!\textbackslash n");} \\
  \vspace{0.8em}
  \vspace{0.8em}
  \vspace{0.8em}
  \vspace{0.8em}
  \hphantom{aaaa} \texttt{return 0;} \\
  \texttt{ \} }

\end{frame}
% ##################





% ##################
\begin{frame}
  \frametitle{„Hello, World!” w~języku~Python}


  \texttt{print("Hello, World!")}

\end{frame}
% ##################





% ##################
\begin{frame}
  \frametitle{Co z~tego wynika?}


  Na pierwszy rzut oka powinno być jasne, że~C jest prostszy od~asemblera,
  a~Python prostszy od~C.

  Większość ludzi zrobi wszystko, by tylko uniknąć pracy w~asemblerze.
  Mam nadzieję, że~nie muszę tłumaczyć dlaczego.  A~nawet jeśli
  znajdzie~się człowieka, który lubi w~nim pisać, to unika~się jak tylko
  można tego, by kod napisany przez niego w~asemblerze wszedł w~skład
  danego programu.

  Jest tak dlatego, że~nawet najlepsi programiści zbyt łatwo mylą~się
  pisząc w~asemblerze, a~pomyłki na jego poziomie są szczególnie
  niebezpieczne. Niestety, wyjaśnienie czemu tak jest, to temat na osobny
  wykład. W~skrócie, błędy w~asemblerze to raj dla hakerów.

  \alert{Ważne.} Czy od tego momentu kursu możemy uznać, że~język
  asemblera sam w~sobie nas zupełnie nie interesuje?

\end{frame}
% ##################





% ##################
\begin{frame}
  \frametitle{Jak sobie z~tym poradzić?}


  Jeśli jednak ktoś z~Państwa chce~się dowiedzieć więcej o~asemblerze,
  to bardzo mnie to cieszy, ale musimy to przełożyć na czas po zajęciach.
  Wtedy służę Państwu całą moją wiedzą.

  Może zadają sobie Państwo następujące pytanie. Jeśli Python jest prostszy
  od~C, to czemu kurs \textit{Podstawy informatyki w~języku~C} jest
  prowadzony w~C, a~nie w~Pythonie? Gdybym ja miał na to wpływ, to może
  byłby prowadzony w~Pythonie, ale nie ja o~tym decyduje.

  Jak rozwiązujemy ten problem dziś? Tworzymy język taki jak~C, piszemy
  w~nich programy, a~następnie odpowiedni program zwany
  \textbf{kompilatorem} tłumaczy ten program na język asembler.

\end{frame}
% ##################





% ##################
\begin{frame}
  \frametitle{Prosty schemat działania kompilatora}


  \begin{figure}

    \begin{tikzpicture}[node distance=0.7]

      \node[diagram rectangle block] (C source code) at (0,0)
      {Kod programu w~języku C};

      \node[diagram block,below=of C source code] (Compiler)
      {Magia (kompilator)};

      \draw[thick diagram arrow] (C source code) -- (Compiler);



      \node[diagram rectangle block,below=of Compiler] (Assembly code)
      {Kod programu w~asemblerze};

      \draw[thick diagram arrow] (Compiler) -- (Assembly code);

    \end{tikzpicture}

    \caption{Ilustracja działania kompilatora~;).}


  \end{figure}

\end{frame}
% ##################





% ##################
\begin{frame}
  \frametitle{Program „Hello, World!” skompilowany z~C\ldots}


  Kompilator tłumaczy kod w~języku źródłowym, w~naszym przypadku jest nim
  język~C, na kod w~języku wynikowym, u~nas jest to odpowiedni dialekt
  asemblera. Kompilator którego będziemy używać, to~\texttt{gcc}, akronim
  od~angielskiego \textit{\textsc{gnu} Compiler Collection}, „zbiór
  kompilatorów projektu \textsc{gnu}”.

  Żeby zobaczyć jaki kod w~języku asembler tworzy kompilator \texttt{gcc}
  należy wywołać polecenie: \\
  \texttt{\$ gcc -S program.c -o program-w-jezyku-asemblera.asm}

  Kod języka asemblera jaki zostanie utworzony zależy od miliarda rzeczy
  (typu procesora, wersji kompilatora, etc.). W~każdym razie na moim
  komputerze program „Hello, World!” kompiluje~się do następującego kodu
  języka asemblera.

\end{frame}
% ##################





% ##################
\begin{frame}
  \frametitle{Z~C do~asemblera, część~I}


  \hphantom{aaaaa} \texttt{.file} \hphantom{aaa}
  \texttt{"PI-01-A-Program-pokazowy-01.c"} \\
  \hphantom{aaaaa} \texttt{.text} \\
  \hphantom{aaaaa} \texttt{.section} \hphantom{aaaaaa} \texttt{.rodata} \\
  \texttt{.LC0:} \\
  \hphantom{aaaaa} \texttt{.string "Hello, World!"} \\
  \hphantom{aaaaa} \texttt{.text} \\
  \hphantom{aaaaa} \texttt{.globl} \hphantom{aa} \texttt{main} \\
  \hphantom{aaaaa} \texttt{.type} \hphantom{aaa}
  \texttt{main, @function} \\
  \texttt{main:} \\
  \texttt{.LFB0:} \\
  \hphantom{aaaaa} \texttt{.cfi\_startproc} \\
  \hphantom{aaaaa} \texttt{endbr64} \\
  \hphantom{aaaaa} \texttt{pushq} \hphantom{aaa} \texttt{\%rbp} \\
  \hphantom{aaaaa} \texttt{.cfi\_def\_cfa\_offset 16} \\
  \hphantom{aaaaa} \texttt{.cfi\_offset 6, -16} \\

\end{frame}
% ##################





% ##################
\begin{frame}
  \frametitle{Z~C do~asemblera, część~II}


  \hphantom{aaaaa} \texttt{movq} \hphantom{aaa} \texttt{\%rsp, \%rbp} \\
  \hphantom{aaaaa} \texttt{.cfi\_def\_cfa\_regitser 6} \\
  \hphantom{aaaaa} \texttt{leaq} \hphantom{aaa}
  \texttt{.LC0(\%rip), \%rax} \\
  \hphantom{aaaaa} \texttt{movq} \hphantom{aaa} \texttt{\%rax, \%rdi} \\
  \hphantom{aaaaa} \texttt{call} \hphantom{aaa} \texttt{puts@PLT} \\
  \hphantom{aaaaa} \texttt{movl} \hphantom{aaa} \texttt{\$0, \%eax} \\
  \hphantom{aaaaa} \texttt{popq} \hphantom{aaa} \texttt{\%rbp} \\
  \hphantom{aaaaa} \texttt{.cfi\_def\_cfa 7, 8} \\
  \hphantom{aaaaa} \texttt{ret} \\
  \hphantom{aaaaa} \texttt{.cfi\_endproc} \\
  \texttt{.LFE0:} \\
  \hphantom{aaaaa} \texttt{.size} \hphantom{aaa} \texttt{main, .-main} \\
  \hphantom{aaaaa} \texttt{.ident} \hphantom{aa}
  \texttt{"GCC: (Ubuntu 11.4.0-1ubuntu1-22.04) 11.4.0"} \\
  \hphantom{aaaaa} \texttt{.section} \hphantom{aaaaaa}
  \texttt{.note.GNU-stack,"",@progbits} \\
  \hphantom{aaaaa} \texttt{.section} \hphantom{aaaaaa}
  \texttt{.note.gnu.property,"a"}

\end{frame}
% ##################





% ##################
\begin{frame}
  \frametitle{Z~C do~asemblera, część~III}


  \hphantom{aaaaa} \texttt{.align 8} \\
  \hphantom{aaaaa} \texttt{.long} \hphantom{aaa} \texttt{1f - 0f} \\
  \hphantom{aaaaa} \texttt{.long} \hphantom{aaa} \texttt{4f - 1f} \\
  \hphantom{aaaaa} \texttt{.long} \hphantom{aaa} \texttt{5} \\
  \texttt{0:} \\
  \hphantom{aaaaa} \texttt{.string "GNU"} \\
  \texttt{1:} \\
  \hphantom{aaaaa} \texttt{.align 8} \\
  \hphantom{aaaaa} \texttt{.long} \hphantom{aaa} \texttt{0xc0000002} \\
  \hphantom{aaaaa} \texttt{.long} \hphantom{aaa} \texttt{3f - 2f} \\
  \texttt{2:} \\
  \hphantom{aaaaa} \texttt{.long} \hphantom{aaa} \texttt{0x3} \\
  \texttt{3:} \\
  \hphantom{aaaaa} \texttt{.align 8} \\
  \texttt{4:}

\end{frame}
% ##################





% ##################
\begin{frame}
  \frametitle{Program „Hello, World!” skompilowany z~C\ldots}


  \alert{Ważne.} Czy od tego momentu, możemy przyjąć, że~nie interesuje
  nas jak dokładnie kompilator działa?

  Jeśli jednak ktoś jest bardziej zainteresowany tą tematyką, to po
  zajęciach służę całą moją wiedzą.

\end{frame}
% ##################



% ##################
\begin{frame}
  \frametitle{O~rozwiązywaniu prac domowych}


  Jeśli ktoś nie wierzy, że~trudną częścią programowania nie jest pisanie
  programów, tylko ich debugowanie, niech zajrzy na świetny kanał
  \colorhref{https://www.youtube.com/@LowLevelTV}{Low Level} i~posłucha
  jakie błędy są znajdowane w~kodzie tworzony przez zawodowych programistów.

  Jeżeli ktoś oddaje trudne rozwiązywania zadań, bo chce spróbować swoich
  sił w~programowaniu, to oczywiście, naszym obowiązkiem jest poprawić
  ten zestaw bez marudzenia. Wszystkie te uwagi nie są po to byśmy
  my mieli prostszą pracę przy poprawianiu zdań, tylko by Państwo nie
  trudzili~się niepotrzebnie.

  Proszę jednak mieć w~głowie następującą mądrość informatyków. \\
  \textit{Program z~którego usunęliśmy jeden błąd, stał~się tym samym
    $10$~razy lepszy.}

  Należy mieć świadomość, że~w~praktyce brak błędów (a~przynajmniej
  zmniejszenie ich liczby) jest ważniejszy niż szybkość programu.

\end{frame}
% ##################





% ##################
\begin{frame}
  \frametitle{O~rozwiązywaniu prac domowych}


  Błędy zwiększają kilkunastokrotnie możliwość, że~haker włamie~się do
  danego komputera, a~to w~czasach internetu rzecz kluczowa. W~okresie po
  $2018$ roku, gdy odkryto luki w~bezpieczeństwie
  \colorhref{https://www.youtube.com/watch?v=I5mRwzVvFGE}{Specter}
  i~\colorhref{https://www.youtube.com/watch?v=I5mRwzVvFGE}{Meltdown}
  w~imię bezpieczeństwa zabiliśmy $50$\% (!!!) prędkości pewnych programów,
  by~się zabezpieczyć przed atakami hakerów.


  % Początkowo zadania są bardzo prosto, jak na ten przedmiot, z~czasem ich
  % trudność, jak i~punkty do zdobycia za nie rosną. Układanie dobrych
  % zestawów zadań to trudna sztuka, stąd będę wdzięczny za~wszelkie uwagi
  % jak można je poprawić. Przypominam, że~o tych zestawach lepiej myśleć
  % jako o~sposobie nauczenia~się tworzenia programów, a~nie nieprzyjemnym
  % obowiązku do odhaczenia. Easier said than done, wiemy o~tym.

  % Proszę przyjąć zasadę, że~\alert{najprostszy program} który rozwiązuje
  % zadanie zgodnie z~wytycznymi jest najlepszym jaki można oddać. Niekiedy
  % studenci oddają bardzo skomplikowane programy jako rozwiązanie naprawdę
  % prostych zadań, co jest zupełnie zbędne.

  % Ostrzegam, ja bardzo \alert{nie lubię} ściągania. Student który
  % oblał~$20$ poprawkę nie będzie miał u~mnie takich problemów, jak osoba
  % którą złapię na ściąganiu. Moja zasada jest taka, że~jak kogoś na tym
  % nakryję na tego typu oszustwach, to więcej niż $3.0$ u~mnie
  % \alert{nie dostanie}.

  % Ściągania nie da~się całkiem wyeliminować, ale można je ograniczyć.
  % A~jak ktoś ściąga tak, żeby ślepy zauważył, to może mieć pretensje tylko
  % i~wyłącznie do siebie. ;)

  % Gdy chodzi o~zadania domowe i~projekt, to należy próbować rozwiązać je
  % możliwie samodzielnie. W~razie napotkania problemów nie tylko można,
  % ale i~należy prosić o~pomoc kolegów, korzystać z~materiałów w~internecie
  % i~używać programów takich jak Chat\textsc{gpt}.

  % Na zajęciach, poza testami, też można z~tych pomocy korzystać, ale nie do
  % tego stopnia, by czyjś wkład własny został zredukowany do metody
  % Copy’ego-Pejsta.

  % Oczywiście, nikt nie straci punktów za oddanie skomplikowanego programu,
  % ale jest wiele powodów, by poprzestać na pisaniu prostego kodu.
  % Po~pierwsze, to zwykle oszczędza czas spędzony na~rozwiązywaniu zadań
  % na~ten przedmiot. Po~drugie, prostota i~czytelność kodu są naprawdę ważne
  % przy tworzeniu programów na~każdym poziomie.

  % Jeżeli ktoś oddaje trudne rozwiązywania zadań, bo chce spróbować swoich
  % sił w~programowaniu, to oczywiście, naszym obowiązkiem jest poprawić
  % ten zestaw bez marudzenia. Wszystkie te uwagi nie są po to byśmy
  % my mieli prostszą pracę przy poprawianiu zdań, tylko by Państwo nie
  % trudzili~się niepotrzebnie.

  % Proszę też pamiętać, że~obowiązuje prawo Kernighana. \\
  % \textit{Debugowanie programu jest dwa razy trudniejsze, niż jego
  %   napisanie. Jeśli więc napiszesz program w~najbardziej pomysłowy sposób
  %   jaki możesz, to nie jesteś wystarczająco inteligentny by go zdebugować.}

  % Ostrzegam, ja bardzo \alert{nie lubię} ściągania. Student który
  % oblał~$20$ poprawkę nie będzie miał u~mnie takich problemów, jak osoba
  % którą złapię na ściąganiu. Moja zasada jest taka, że~jak kogoś na tym
  % nakryję na tego typu oszustwach, to więcej niż $3.0$ u~mnie
  % \alert{nie dostanie}.

  % Ściągania nie da~się całkiem wyeliminować, ale można je ograniczyć.
  % A~jak ktoś ściąga tak, żeby ślepy zauważył, to może mieć pretensje tylko
  % i~wyłącznie do siebie. ;)

  % Gdy chodzi o~zadania domowe i~projekt, to należy próbować rozwiązać je
  % możliwie samodzielnie. W~razie napotkania problemów nie tylko można,
  % ale i~należy prosić o~pomoc kolegów, korzystać z~materiałów w~internecie
  % i~używać programów takich jak Chat\textsc{gpt}.

  % Na zajęciach, poza testami, też można z~tych pomocy korzystać, ale nie do
  % tego stopnia, by czyjś wkład własny został zredukowany do metody
  % Copy’ego-Pejsta.

\end{frame}
% ##################







% ######################################
\section{Fazy uruchamiania programu w~języku~C}
% ######################################



% ##################
\begin{frame}
  \frametitle{Trzy fazy tworzenie programu w~języku~C}


  By stworzyć program w~języku~C musimy przejść trzy fazy.

  \begin{enumerate}

  \item Faza pisania programu.

  \item Faza kompilacji.

  \item Faza wykonywania.

  \end{enumerate}



  \alert{Faza pisania programu.} Jest dość oczywista. Otwieramy plik
  z~kodem źródłowym poleceniem: \\
  \texttt{\$ gedit helloWorld.c \&} \\
  wpisujemy w~nim \\
  \texttt{\#include <stdio.h>} \\
  \texttt{int main() \{ } \\
  \hphantom{aaaa} \texttt{printf("Hello, World!\textbackslash n");} \\
  \hphantom{aaaa} \texttt{return 0;} \\
  \texttt{ \} } \\
  Następnie zapisujemy plik na~dysku (\texttt{Ctrl-s}). Koniec.



  % \begin{figure}

  %   \label{fig:Translator-04}

  %   \begin{tikzpicture}

  %     \node[diagram rectangle block] (Source program) at (0,0)
  %     {Program źródłowy (język~C)};

  %     \node[diagram block,right=of Source program] (Compiler)
  %     {Kompilator};

  %     \draw[thick diagram arrow] (Source program) -- (Compiler);



  %     \node[diagram rectangle block,right=of Compiler]
  %     (Target program)
  %     {Program wynikowy (dia. asemblera)};

  %     \draw[thick diagram arrow] (Translator) -- (Target program);

  %   \end{tikzpicture}

  %   \caption{Translator jako czarna skrzynka}


  % \end{figure}

\end{frame}
% ##################





% ##################
\begin{frame}
  \frametitle{Faza kompilacji i~wykonania}


  \alert{Faza kompilacji.} Kompilator analizuj niezrozumiały dla komputera
  kod w~języku~C, następnie tłumaczy go na zrozumiał dla danej maszyny kod
  w~odpowiedni dialekcie asemblera. Wynik tego tłumaczenia jest zapisywany
  w~osobnym pliku.

  Kompilator \alert{tylko} przekłada kod z~jednego języka na drugi
  i~zapisuje wynik w~odpowiednim pliku, ale \alert{nie} uruchamia
  otrzymanego programu. Używając prostego przykładu, kompilator
  przetłumaczy „Eat the soup.” na „Zjedz zupę.”, ale nie powie nikomu,
  że~ma zjeść zupę.%  Kompilator dokonuje \alert{tylko} przekładu z~jednego
  % języka na drugi i~jego zapisu do pliku, \alert{nie} uruchamia jednak tego
  % programu. Używając prostego przykładu, kompilator przetłumaczy „Eat the
  % soup.” na „Zjedz zupę.”, ale nie powie nikomu, że~ma zjeść zupę.

  % Przez wywołanie polecenia \\
  % \texttt{\$ gcc --std=c99 -pedantic helloWorld.c -o outProg} \\
  % wykonujemy fazę kompilacji kodu źródłowego \texttt{helloWorld.c}
  % dostając w~rezultacie program wynikowy \texttt{outProg}.

\end{frame}
% ##################






% ##################
\begin{frame}
  \frametitle{Jak to wygląda w~praktyce?}


  Przez wywołanie polecenia \\
  \texttt{\$ gcc --std=c99 -pedantic helloWorld.c -o outProg} \\
  wykonujemy fazę kompilacji kodu źródłowego \texttt{helloWorld.c}
  dostając w~rezultacie program wynikowy \texttt{outProg}.

  By bardziej to zobrazować wywołajmy teraz polecenie \\
  \texttt{\$ file helloWorld.c} \\
  Zakładając, że~na dysku istnieje już plik \texttt{outProg}, wprowadźmy
  podobne do poprzedniego polecenie: \\
  \texttt{\$ file outProg} \\
  Można się czuć zagubionym wśród \textsc{elf}ów, ciast (\textsc{pie})
  i~tym podobnych tworów, ale da~się z~tym żyć.

  % \alert{Faza wykonania.} Otrzymany program uruchamiamy wywołując
  % polecenie \\
  % \texttt{\$ ./outProg}

\end{frame}
% ##################





% ##################
\begin{frame}
  \frametitle{Jak to wygląda w~praktyce?}


  % Przez wywołanie polecenia \\
  % \texttt{\$ gcc --std=c99 -pedantic helloWorld.c -o outProg} \\
  % wykonujemy fazę kompilacji kodu źródłowego \texttt{helloWorld.c}
  % dostając w~rezultacie program wynikowy \texttt{outProg}.

  % By bardziej to zobrazować wywołajmy teraz polecenie \\
  % \texttt{\$ file helloWorld.c} \\
  % Zakładając, że~na dysku istnieje już plik \texttt{outProg}, wprowadźmy
  % podobne do poprzedniego polecenie: \\
  % \texttt{\$ file outProg} \\
  % Można się czuć zagubionym wśród \textsc{elf}ów, ciast (\textsc{pie})
  % i~tym podobnych tworów, ale da~się z~tym żyć.

  \alert{Faza wykonania.} Wkraczamy w~nią uruchamiając otrzymany w~fazie
  kompilacji program \texttt{outProg} za pomocą polecenia \\
  \texttt{\$ ./outProg} \\
  W~jego rezultacie system operacyjny przekierowuje kod asemblera
  zawarty w~pliku \texttt{outProg} na procesor, który następnie go wykonuje
  i~w~rezultacie na ekranie widzimy napis „Hello, World!”.

  Jeśli komuś brak jeszcze mocnych wrażeń albo chce mocno odlecieć, niech
  wywoła polecenie \\
  \texttt{\$ xxd outProg outProg\_hex.txt} \\
  a~następnie otworzy plik \texttt{outProg\_hex.txt} wywołując \\
  \texttt{\$ gedit outProg\_hex.txt}

\end{frame}
% ##################





% ##################
\begin{frame}
  \frametitle{Faza kompilacji i~faza wykonania}


  \begin{figure}

    \begin{tikzpicture}[node distance=0.5]

      \node[diagram rectangle block] (Source code) at (0,0) {Kod źródłowy};

      \node[diagram block,below=of Source code] (Compiler) {Kompilator};

      \draw[thick diagram arrow] (Source code) -- (Compiler);



      \node[diagram rectangle block,below=of Compiler] (Target program)
      {Program wynikowy};

      \draw[thick diagram arrow] (Compiler) -- (Target program);



      \node[below=0.1em of Target program] {a) Faza kompilacji};






      \begin{scope}[xshift=5cm]


        \node[diagram rectangle block] (Input) at (0,0) {Wejście};

        \node[diagram block,below=of Input] (Running target program)
        {Uruchomiony program wynikowy};

        \draw[thick diagram arrow] (Input) -- (Running target program);



        \node[diagram rectangle block,below=of Running target program]
        (Results) {Wyniki działania programu};

        \draw[thick diagram arrow] (Running target program) --
        (Results);



        \node[below=0.1em of Results] {b) Faza wykonania};

      \end{scope}

    \end{tikzpicture}

    \caption{Ilustracja podziału na fazę kompilacji i~wykonania programu}


  \end{figure}

\end{frame}
% ##################





% ##################
\begin{frame}
  \frametitle{W~naszym wypadku}


  \begin{figure}

    \begin{tikzpicture}[node distance=0.5]

      \node[diagram rectangle block] (Source code) at (0,0)
      {\texttt{helloWorld.c}};

      \node[diagram block,below=of Source code] (Compiler)
      {Kompilator \texttt{gcc}};

      \draw[thick diagram arrow] (Source code) -- (Compiler);



      \node[diagram rectangle block,below=of Compiler] (Target program)
      {\texttt{outProg}};

      \draw[thick diagram arrow] (Compiler) -- (Target program);



      \node[below=0.1em of Target program] {a) Faza kompilacji};






      \begin{scope}[xshift=5cm]


        \node[diagram rectangle block] (Input) at (0,0)
        {\texttt{outProg}};

        \node[diagram block,below=of Input] (Running target program)
        {\texttt{\$ ./outProg}};

        \draw[thick diagram arrow] (Input) -- (Running target program);



        \node[diagram rectangle block,below=of Running target program]
        (Results) {Hello, World!};

        \draw[thick diagram arrow] (Running target program) --
        (Results);



        \node[below=0.1em of Results] {b) Faza wykonania};

      \end{scope}

    \end{tikzpicture}

    \caption{Kompilacja i~wykonanie programu „Hello, World!”.}


  \end{figure}

\end{frame}
% ##################










% ######################################
\section{Jakie są inne języki programowania?}
% ######################################


% ##################
\begin{frame}
  \frametitle{Jakie są inne języki programowania?}


  Wymieńmy 20 najpopularniejszych języków programowania według indeksu
  \colorhref{https://www.tiobe.com/tiobe-index/}{\textsc{tiobe}}. Poniżej
  prezentowany jest stan tego indeksu na lipiec 2024. Podane wielkości
  procentowe przedstawiają wyliczoną przez twórców indeksu popularność
  języka.



  \begin{itemize}

  \item[1)] \colorhref{https://www.python.org/}{Python}, 16.12\%.

  \item[2)] C++, 10.34\%.

  \item[3)] C, 9.48\%.

  \item[4)] \colorhref{https://www.java.com/pl/}{Java}, 8.59\%.

  \item[5)]
    \colorhref{https://learn.microsoft.com/en-us/dotnet/csharp/}{C\#},
    6.72\%.

  \item[6)] JavaScript, 3.79\%.

  \item[7)] \colorhref{https://go.dev/}{Go}, 2.19\%.

  \item[8)] Visual Basic, 2.08\%.

  \end{itemize}

\end{frame}
% ##################





% ##################
\begin{frame}
  \frametitle{Jakie są inne języki programowania?}


  \begin{itemize}

  \item[9)] Fortran, 2.05\%.

  \item[10)] \textsc{sql}, 2.04\%.

  \item[11)] Delphi/Object Pascal, 1.89\%.

  \item[12)] \textsc{matlab}, 1.34\%.

  \item[13)] \colorhref{https://www.rust-lang.org/}{Rust}, 1.18\%.

  \item[14)] \colorhref{https://www.ruby-lang.org/en/}{Ruby}, 1.16\%.

  \item[15)] Scratch, 1.15\%.

  \item[16)] \textsc{php}, 1.15\%.

  \item[17)] \colorhref{https://www.swift.org/}{Swift}, 1.13\%.

  \item[18)] Assembly language, 1.11\%.

  \item[19)] \textsc{cobol}, 1.08\%.

  \item[20)] \colorhref{https://kotlinlang.org/}{Kotlin}, 1.05\%.

  \end{itemize}

\end{frame}
% ##################







% % ############################
% \jagiellonianendslide{Czy są jakieś pytania do tej części?}
% % ############################































% ####################################################################
% ####################################################################
% Bibliography

\printbibliography





% ############################
% End of the document

\end{document}
