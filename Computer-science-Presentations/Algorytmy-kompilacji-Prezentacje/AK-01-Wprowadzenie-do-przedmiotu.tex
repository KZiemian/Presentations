% ------------------------------------------------------------------------------------------------------------------
% Basic configuration of Beamera class and Jagiellonian theme
% ------------------------------------------------------------------------------------------------------------------
\RequirePackage[l2tabu, orthodox]{nag}



\ifx\PresentationStyle\notset
  \def\PresentationStyle{dark}
\fi



% Options: t -- align text to the top of the frame
\documentclass[10pt,t]{beamer}
\mode<presentation>
\usetheme[style=\PresentationStyle]{jagiellonian}





% ------------------------------------------------------------------------------------
% Procesing configuration files of Jagiellonian theme located in directory
% "preambule"
% ------------------------------------------------------------------------------------
% Configuration for polish language
% Need description
\usepackage[polish]{babel}
% Need description
\usepackage[MeX]{polski}



% ------------------------------
% Better support of polish chars in technical parts of PDF
% ------------------------------
\hypersetup{pdfencoding=auto,psdextra}

% Package "textpos" give as enviroment "textblock" which is very usefull in
% arranging text on slides.

% This is standard configuration of "textpos"
\usepackage[overlay,absolute]{textpos}

% If you need to see bounds of "textblock's" comment line above and uncomment
% one below.

% Caution! When showboxes option is on significant ammunt of space is add
% to the top of textblock and as such, everyting put in them gone down.
% We need to check how to remove this bug.

% \usepackage[showboxes,overlay,absolute]{textpos}



% Setting scale length for package "textpos"
\setlength{\TPHorizModule}{10mm}
\setlength{\TPVertModule}{\TPHorizModule}


% ---------------------------------------
% TikZ
% ---------------------------------------
% Importing TikZ libraries
\usetikzlibrary{arrows.meta}
\usetikzlibrary{positioning}





% % Configuration package "bm" that need for making bold symbols
% \newcommand{\bmmax}{0}
% \newcommand{\hmmax}{0}
% \usepackage{bm}




% ---------------------------------------
% Packages for scientific texts
% ---------------------------------------
% \let\lll\undefined  % Sometimes you must use this line to allow
% "amsmath" package to works with packages with packages for polish
% languge imported
% /preambul/LanguageSettings/JagiellonianPolishLanguageSettings.tex.
% This comments (probably) removes polish letter Ł.
\usepackage{amsmath}  % Packages from American Mathematical Society (AMS)
\usepackage{amssymb}
\usepackage{amscd}
\usepackage{amsthm}
\usepackage{siunitx}  % Package for typsetting SI units.
\usepackage{upgreek}  % Better looking greek letters.
% Example of using upgreek: pi = \uppi


\usepackage{calrsfs}  % Zmienia czcionkę kaligraficzną w \mathcal
% na ładniejszą. Może w innych miejscach robi to samo, ale o tym nic
% nie wiem.










% ---------------------------------------
% Packages written for lectures "Geometria 3D dla twórców gier wideo"
% ---------------------------------------
% \usepackage{./ProgramowanieSymulacjiFizykiPaczki/ProgramowanieSymulacjiFizyki}
% \usepackage{./ProgramowanieSymulacjiFizykiPaczki/ProgramowanieSymulacjiFizykiIndeksy}
% \usepackage{./ProgramowanieSymulacjiFizykiPaczki/ProgramowanieSymulacjiFizykiTikZStyle}





% !!!!!!!!!!!!!!!!!!!!!!!!!!!!!!
% !!!!!!!!!!!!!!!!!!!!!!!!!!!!!!
% EVIL STUFF
\if\JUlogotitle1
\edef\LogoJUPath{LogoJU_\JUlogoLang/LogoJU_\JUlogoShape_\JUlogoColor.pdf}
\titlegraphic{\hfill\includegraphics[scale=0.22]
{./JagiellonianPictures/\LogoJUPath}}
\fi
% ---------------------------------------
% Commands for handling colors
% ---------------------------------------


% Command for setting normal text color for some text in math modestyle
% Text color depend on used style of Jagiellonian

% Beamer version of command
\newcommand{\TextWithNormalTextColor}[1]{%
  {\color{jNormalTextFGColor}
    \setbeamercolor{math text}{fg=jNormalTextFGColor} {#1}}
}

% Article and similar classes version of command
% \newcommand{\TextWithNormalTextColor}[1]{%
%   {\color{jNormalTextsFGColor} {#1}}
% }



% Beamer version of command
\newcommand{\NormalTextInMathMode}[1]{%
  {\color{jNormalTextFGColor}
    \setbeamercolor{math text}{fg=jNormalTextFGColor} \text{#1}}
}


% Article and similar classes version of command
% \newcommand{\NormalTextInMathMode}[1]{%
%   {\color{jNormalTextsFGColor} \text{#1}}
% }




% Command that sets color of some mathematical text to the same color
% that has normal text in header (?)

% Beamer version of the command
\newcommand{\MathTextFrametitleFGColor}[1]{%
  {\color{jFrametitleFGColor}
    \setbeamercolor{math text}{fg=jFrametitleFGColor} #1}
}

% Article and similar classes version of the command
% \newcommand{\MathTextWhiteColor}[1]{{\color{jFrametitleFGColor} #1}}





% Command for setting color of alert text for some text in math modestyle

% Beamer version of the command
\newcommand{\MathTextAlertColor}[1]{%
  {\color{jOrange} \setbeamercolor{math text}{fg=jOrange} #1}
}

% Article and similar classes version of the command
% \newcommand{\MathTextAlertColor}[1]{{\color{jOrange} #1}}





% Command that allow you to sets chosen color as the color of some text into
% math mode. Due to some nuances in the way that Beamer handle colors
% it not work in all cases. We hope that in the future we will improve it.

% Beamer version of the command
\newcommand{\SetMathTextColor}[2]{%
  {\color{#1} \setbeamercolor{math text}{fg=#1} #2}
}


% Article and similar classes version of the command
% \newcommand{\SetMathTextColor}[2]{{\color{#1} #2}}










% ---------------------------------------
% Commands for few special slides
% ---------------------------------------
\newcommand{\EndingSlide}[1]{%
  \begin{frame}[standout]

    \begingroup

    \color{jFrametitleFGColor}

    #1

    \endgroup

  \end{frame}
}










% ---------------------------------------
% Commands for setting background pictures for some slides
% ---------------------------------------
\newcommand{\TitleBackgroundPicture}
{./JagiellonianPictures/Backgrounds/LajkonikDark.png}
\newcommand{\SectionBackgroundPicture}
{./JagiellonianPictures/Backgrounds/LajkonikLight.png}



\newcommand{\TitleSlideWithPicture}{%
  \begingroup

  \usebackgroundtemplate{%
    \includegraphics[height=\paperheight]{\TitleBackgroundPicture}}

  \maketitle

  \endgroup
}





\newcommand{\SectionSlideWithPicture}[1]{%
  \begingroup

  \usebackgroundtemplate{%
    \includegraphics[height=\paperheight]{\SectionBackgroundPicture}}

  \setbeamercolor{titlelike}{fg=normal text.fg}

  \section{#1}

  \endgroup
}










% ---------------------------------------
% Commands for lectures "Geometria 3D dla twórców gier wideo"
% Polish version
% ---------------------------------------
% Komendy teraz wykomentowane były potrzebne, gdy loga były na niebieskim
% tle, nie na białym. A są na białym bo tego chcieli w biurze projektu.
% \newcommand{\FundingLogoWhitePicturePL}
% {./PresentationPictures/CommonPictures/logotypFundusze_biale_bez_tla2.pdf}
\newcommand{\FundingLogoColorPicturePL}
{./PresentationPictures/CommonPictures/European_Funds_color_PL.pdf}
% \newcommand{\EULogoWhitePicturePL}
% {./PresentationPictures/CommonPictures/logotypUE_biale_bez_tla2.pdf}
\newcommand{\EUSocialFundLogoColorPicturePL}
{./PresentationPictures/CommonPictures/EU_Social_Fund_color_PL.pdf}
% \newcommand{\ZintegrUJLogoWhitePicturePL}
% {./PresentationPictures/CommonPictures/zintegruj-logo-white.pdf}
\newcommand{\ZintegrUJLogoColorPicturePL}
{./PresentationPictures/CommonPictures/ZintegrUJ_color.pdf}
\newcommand{\JULogoColorPicturePL}
{./JagiellonianPictures/LogoJU_PL/LogoJU_A_color.pdf}





\newcommand{\GeometryThreeDSpecialBeginningSlidePL}{%
  \begin{frame}[standout]

    \begin{textblock}{11}(1,0.7)

      \begin{flushleft}

        \mdseries

        \footnotesize

        \color{jFrametitleFGColor}

        Materiał powstał w ramach projektu współfinansowanego ze środków
        Unii Europejskiej w ramach Europejskiego Funduszu Społecznego
        POWR.03.05.00-00-Z309/17-00.

      \end{flushleft}

    \end{textblock}





    \begin{textblock}{10}(0,2.2)

      \tikz \fill[color=jBackgroundStyleLight] (0,0) rectangle (12.8,-1.5);

    \end{textblock}


    \begin{textblock}{3.2}(1,2.45)

      \includegraphics[scale=0.3]{\FundingLogoColorPicturePL}

    \end{textblock}


    \begin{textblock}{2.5}(3.7,2.5)

      \includegraphics[scale=0.2]{\JULogoColorPicturePL}

    \end{textblock}


    \begin{textblock}{2.5}(6,2.4)

      \includegraphics[scale=0.1]{\ZintegrUJLogoColorPicturePL}

    \end{textblock}


    \begin{textblock}{4.2}(8.4,2.6)

      \includegraphics[scale=0.3]{\EUSocialFundLogoColorPicturePL}

    \end{textblock}

  \end{frame}
}



\newcommand{\GeometryThreeDTwoSpecialBeginningSlidesPL}{%
  \begin{frame}[standout]

    \begin{textblock}{11}(1,0.7)

      \begin{flushleft}

        \mdseries

        \footnotesize

        \color{jFrametitleFGColor}

        Materiał powstał w ramach projektu współfinansowanego ze środków
        Unii Europejskiej w ramach Europejskiego Funduszu Społecznego
        POWR.03.05.00-00-Z309/17-00.

      \end{flushleft}

    \end{textblock}





    \begin{textblock}{10}(0,2.2)

      \tikz \fill[color=jBackgroundStyleLight] (0,0) rectangle (12.8,-1.5);

    \end{textblock}


    \begin{textblock}{3.2}(1,2.45)

      \includegraphics[scale=0.3]{\FundingLogoColorPicturePL}

    \end{textblock}


    \begin{textblock}{2.5}(3.7,2.5)

      \includegraphics[scale=0.2]{\JULogoColorPicturePL}

    \end{textblock}


    \begin{textblock}{2.5}(6,2.4)

      \includegraphics[scale=0.1]{\ZintegrUJLogoColorPicturePL}

    \end{textblock}


    \begin{textblock}{4.2}(8.4,2.6)

      \includegraphics[scale=0.3]{\EUSocialFundLogoColorPicturePL}

    \end{textblock}

  \end{frame}





  \TitleSlideWithPicture
}



\newcommand{\GeometryThreeDSpecialEndingSlidePL}{%
  \begin{frame}[standout]

    \begin{textblock}{11}(1,0.7)

      \begin{flushleft}

        \mdseries

        \footnotesize

        \color{jFrametitleFGColor}

        Materiał powstał w ramach projektu współfinansowanego ze środków
        Unii Europejskiej w~ramach Europejskiego Funduszu Społecznego
        POWR.03.05.00-00-Z309/17-00.

      \end{flushleft}

    \end{textblock}





    \begin{textblock}{10}(0,2.2)

      \tikz \fill[color=jBackgroundStyleLight] (0,0) rectangle (12.8,-1.5);

    \end{textblock}


    \begin{textblock}{3.2}(1,2.45)

      \includegraphics[scale=0.3]{\FundingLogoColorPicturePL}

    \end{textblock}


    \begin{textblock}{2.5}(3.7,2.5)

      \includegraphics[scale=0.2]{\JULogoColorPicturePL}

    \end{textblock}


    \begin{textblock}{2.5}(6,2.4)

      \includegraphics[scale=0.1]{\ZintegrUJLogoColorPicturePL}

    \end{textblock}


    \begin{textblock}{4.2}(8.4,2.6)

      \includegraphics[scale=0.3]{\EUSocialFundLogoColorPicturePL}

    \end{textblock}





    \begin{textblock}{11}(1,4)

      \begin{flushleft}

        \mdseries

        \footnotesize

        \RaggedRight

        \color{jFrametitleFGColor}

        Treść niniejszego wykładu jest udostępniona na~licencji
        Creative Commons (\textsc{cc}), z~uzna\-niem autorstwa
        (\textsc{by}) oraz udostępnianiem na tych samych warunkach
        (\textsc{sa}). Rysunki i~wy\-kresy zawarte w~wykładzie są
        autorstwa dr.~hab.~Pawła Węgrzyna et~al. i~są dostępne
        na tej samej licencji, o~ile nie wskazano inaczej.
        W~prezentacji wykorzystano temat Beamera Jagiellonian,
        oparty na~temacie Metropolis Matthiasa Vogelgesanga,
        dostępnym na licencji \LaTeX{} Project Public License~1.3c
        pod adresem: \colorhref{https://github.com/matze/mtheme}
        {https://github.com/matze/mtheme}.

        Projekt typograficzny: Iwona Grabska-Gradzińska \\
        Skład: Kamil Ziemian;
        Korekta: Wojciech Palacz \\
        Modele: Dariusz Frymus, Kamil Nowakowski \\
        Rysunki i~wykresy: Kamil Ziemian, Paweł Węgrzyn, Wojciech Palacz

      \end{flushleft}

    \end{textblock}

  \end{frame}
}



\newcommand{\GeometryThreeDTwoSpecialEndingSlidesPL}[1]{%
  \begin{frame}[standout]


    \begin{textblock}{11}(1,0.7)

      \begin{flushleft}

        \mdseries

        \footnotesize

        \color{jFrametitleFGColor}

        Materiał powstał w ramach projektu współfinansowanego ze środków
        Unii Europejskiej w~ramach Europejskiego Funduszu Społecznego
        POWR.03.05.00-00-Z309/17-00.

      \end{flushleft}

    \end{textblock}





    \begin{textblock}{10}(0,2.2)

      \tikz \fill[color=jBackgroundStyleLight] (0,0) rectangle (12.8,-1.5);

    \end{textblock}


    \begin{textblock}{3.2}(1,2.45)

      \includegraphics[scale=0.3]{\FundingLogoColorPicturePL}

    \end{textblock}


    \begin{textblock}{2.5}(3.7,2.5)

      \includegraphics[scale=0.2]{\JULogoColorPicturePL}

    \end{textblock}


    \begin{textblock}{2.5}(6,2.4)

      \includegraphics[scale=0.1]{\ZintegrUJLogoColorPicturePL}

    \end{textblock}


    \begin{textblock}{4.2}(8.4,2.6)

      \includegraphics[scale=0.3]{\EUSocialFundLogoColorPicturePL}

    \end{textblock}





    \begin{textblock}{11}(1,4)

      \begin{flushleft}

        \mdseries

        \footnotesize

        \RaggedRight

        \color{jFrametitleFGColor}

        Treść niniejszego wykładu jest udostępniona na~licencji
        Creative Commons (\textsc{cc}), z~uzna\-niem autorstwa
        (\textsc{by}) oraz udostępnianiem na tych samych warunkach
        (\textsc{sa}). Rysunki i~wy\-kresy zawarte w~wykładzie są
        autorstwa dr.~hab.~Pawła Węgrzyna et~al. i~są dostępne
        na tej samej licencji, o~ile nie wskazano inaczej.
        W~prezentacji wykorzystano temat Beamera Jagiellonian,
        oparty na~temacie Metropolis Matthiasa Vogelgesanga,
        dostępnym na licencji \LaTeX{} Project Public License~1.3c
        pod adresem: \colorhref{https://github.com/matze/mtheme}
        {https://github.com/matze/mtheme}.

        Projekt typograficzny: Iwona Grabska-Gradzińska \\
        Skład: Kamil Ziemian;
        Korekta: Wojciech Palacz \\
        Modele: Dariusz Frymus, Kamil Nowakowski \\
        Rysunki i~wykresy: Kamil Ziemian, Paweł Węgrzyn, Wojciech Palacz

      \end{flushleft}

    \end{textblock}

  \end{frame}





  \begin{frame}[standout]

    \begingroup

    \color{jFrametitleFGColor}

    #1

    \endgroup

  \end{frame}
}



\newcommand{\GeometryThreeDSpecialEndingSlideVideoPL}{%
  \begin{frame}[standout]

    \begin{textblock}{11}(1,0.7)

      \begin{flushleft}

        \mdseries

        \footnotesize

        \color{jFrametitleFGColor}

        Materiał powstał w ramach projektu współfinansowanego ze środków
        Unii Europejskiej w~ramach Europejskiego Funduszu Społecznego
        POWR.03.05.00-00-Z309/17-00.

      \end{flushleft}

    \end{textblock}





    \begin{textblock}{10}(0,2.2)

      \tikz \fill[color=jBackgroundStyleLight] (0,0) rectangle (12.8,-1.5);

    \end{textblock}


    \begin{textblock}{3.2}(1,2.45)

      \includegraphics[scale=0.3]{\FundingLogoColorPicturePL}

    \end{textblock}


    \begin{textblock}{2.5}(3.7,2.5)

      \includegraphics[scale=0.2]{\JULogoColorPicturePL}

    \end{textblock}


    \begin{textblock}{2.5}(6,2.4)

      \includegraphics[scale=0.1]{\ZintegrUJLogoColorPicturePL}

    \end{textblock}


    \begin{textblock}{4.2}(8.4,2.6)

      \includegraphics[scale=0.3]{\EUSocialFundLogoColorPicturePL}

    \end{textblock}





    \begin{textblock}{11}(1,4)

      \begin{flushleft}

        \mdseries

        \footnotesize

        \RaggedRight

        \color{jFrametitleFGColor}

        Treść niniejszego wykładu jest udostępniona na~licencji
        Creative Commons (\textsc{cc}), z~uzna\-niem autorstwa
        (\textsc{by}) oraz udostępnianiem na tych samych warunkach
        (\textsc{sa}). Rysunki i~wy\-kresy zawarte w~wykładzie są
        autorstwa dr.~hab.~Pawła Węgrzyna et~al. i~są dostępne
        na tej samej licencji, o~ile nie wskazano inaczej.
        W~prezentacji wykorzystano temat Beamera Jagiellonian,
        oparty na~temacie Metropolis Matthiasa Vogelgesanga,
        dostępnym na licencji \LaTeX{} Project Public License~1.3c
        pod adresem: \colorhref{https://github.com/matze/mtheme}
        {https://github.com/matze/mtheme}.

        Projekt typograficzny: Iwona Grabska-Gradzińska;
        Skład: Kamil Ziemian \\
        Korekta: Wojciech Palacz;
        Modele: Dariusz Frymus, Kamil Nowakowski \\
        Rysunki i~wykresy: Kamil Ziemian, Paweł Węgrzyn, Wojciech Palacz \\
        Montaż: Agencja Filmowa Film \& Television Production~-- Zbigniew
        Masklak

      \end{flushleft}

    \end{textblock}

  \end{frame}
}





\newcommand{\GeometryThreeDTwoSpecialEndingSlidesVideoPL}[1]{%
  \begin{frame}[standout]

    \begin{textblock}{11}(1,0.7)

      \begin{flushleft}

        \mdseries

        \footnotesize

        \color{jFrametitleFGColor}

        Materiał powstał w ramach projektu współfinansowanego ze środków
        Unii Europejskiej w~ramach Europejskiego Funduszu Społecznego
        POWR.03.05.00-00-Z309/17-00.

      \end{flushleft}

    \end{textblock}





    \begin{textblock}{10}(0,2.2)

      \tikz \fill[color=jBackgroundStyleLight] (0,0) rectangle (12.8,-1.5);

    \end{textblock}


    \begin{textblock}{3.2}(1,2.45)

      \includegraphics[scale=0.3]{\FundingLogoColorPicturePL}

    \end{textblock}


    \begin{textblock}{2.5}(3.7,2.5)

      \includegraphics[scale=0.2]{\JULogoColorPicturePL}

    \end{textblock}


    \begin{textblock}{2.5}(6,2.4)

      \includegraphics[scale=0.1]{\ZintegrUJLogoColorPicturePL}

    \end{textblock}


    \begin{textblock}{4.2}(8.4,2.6)

      \includegraphics[scale=0.3]{\EUSocialFundLogoColorPicturePL}

    \end{textblock}





    \begin{textblock}{11}(1,4)

      \begin{flushleft}

        \mdseries

        \footnotesize

        \RaggedRight

        \color{jFrametitleFGColor}

        Treść niniejszego wykładu jest udostępniona na~licencji
        Creative Commons (\textsc{cc}), z~uzna\-niem autorstwa
        (\textsc{by}) oraz udostępnianiem na tych samych warunkach
        (\textsc{sa}). Rysunki i~wy\-kresy zawarte w~wykładzie są
        autorstwa dr.~hab.~Pawła Węgrzyna et~al. i~są dostępne
        na tej samej licencji, o~ile nie wskazano inaczej.
        W~prezentacji wykorzystano temat Beamera Jagiellonian,
        oparty na~temacie Metropolis Matthiasa Vogelgesanga,
        dostępnym na licencji \LaTeX{} Project Public License~1.3c
        pod adresem: \colorhref{https://github.com/matze/mtheme}
        {https://github.com/matze/mtheme}.

        Projekt typograficzny: Iwona Grabska-Gradzińska;
        Skład: Kamil Ziemian \\
        Korekta: Wojciech Palacz;
        Modele: Dariusz Frymus, Kamil Nowakowski \\
        Rysunki i~wykresy: Kamil Ziemian, Paweł Węgrzyn, Wojciech Palacz \\
        Montaż: Agencja Filmowa Film \& Television Production~-- Zbigniew
        Masklak

      \end{flushleft}

    \end{textblock}

  \end{frame}





  \begin{frame}[standout]


    \begingroup

    \color{jFrametitleFGColor}

    #1

    \endgroup

  \end{frame}
}










% ---------------------------------------
% Commands for lectures "Geometria 3D dla twórców gier wideo"
% English version
% ---------------------------------------
% \newcommand{\FundingLogoWhitePictureEN}
% {./PresentationPictures/CommonPictures/logotypFundusze_biale_bez_tla2.pdf}
\newcommand{\FundingLogoColorPictureEN}
{./PresentationPictures/CommonPictures/European_Funds_color_EN.pdf}
% \newcommand{\EULogoWhitePictureEN}
% {./PresentationPictures/CommonPictures/logotypUE_biale_bez_tla2.pdf}
\newcommand{\EUSocialFundLogoColorPictureEN}
{./PresentationPictures/CommonPictures/EU_Social_Fund_color_EN.pdf}
% \newcommand{\ZintegrUJLogoWhitePictureEN}
% {./PresentationPictures/CommonPictures/zintegruj-logo-white.pdf}
\newcommand{\ZintegrUJLogoColorPictureEN}
{./PresentationPictures/CommonPictures/ZintegrUJ_color.pdf}
\newcommand{\JULogoColorPictureEN}
{./JagiellonianPictures/LogoJU_EN/LogoJU_A_color.pdf}



\newcommand{\GeometryThreeDSpecialBeginningSlideEN}{%
  \begin{frame}[standout]

    \begin{textblock}{11}(1,0.7)

      \begin{flushleft}

        \mdseries

        \footnotesize

        \color{jFrametitleFGColor}

        This content was created as part of a project co-financed by the
        European Union within the framework of the European Social Fund
        POWR.03.05.00-00-Z309/17-00.

      \end{flushleft}

    \end{textblock}





    \begin{textblock}{10}(0,2.2)

      \tikz \fill[color=jBackgroundStyleLight] (0,0) rectangle (12.8,-1.5);

    \end{textblock}


    \begin{textblock}{3.2}(0.7,2.45)

      \includegraphics[scale=0.3]{\FundingLogoColorPictureEN}

    \end{textblock}


    \begin{textblock}{2.5}(4.15,2.5)

      \includegraphics[scale=0.2]{\JULogoColorPictureEN}

    \end{textblock}


    \begin{textblock}{2.5}(6.35,2.4)

      \includegraphics[scale=0.1]{\ZintegrUJLogoColorPictureEN}

    \end{textblock}


    \begin{textblock}{4.2}(8.4,2.6)

      \includegraphics[scale=0.3]{\EUSocialFundLogoColorPictureEN}

    \end{textblock}

  \end{frame}
}



\newcommand{\GeometryThreeDTwoSpecialBeginningSlidesEN}{%
  \begin{frame}[standout]

    \begin{textblock}{11}(1,0.7)

      \begin{flushleft}

        \mdseries

        \footnotesize

        \color{jFrametitleFGColor}

        This content was created as part of a project co-financed by the
        European Union within the framework of the European Social Fund
        POWR.03.05.00-00-Z309/17-00.

      \end{flushleft}

    \end{textblock}





    \begin{textblock}{10}(0,2.2)

      \tikz \fill[color=jBackgroundStyleLight] (0,0) rectangle (12.8,-1.5);

    \end{textblock}


    \begin{textblock}{3.2}(0.7,2.45)

      \includegraphics[scale=0.3]{\FundingLogoColorPictureEN}

    \end{textblock}


    \begin{textblock}{2.5}(4.15,2.5)

      \includegraphics[scale=0.2]{\JULogoColorPictureEN}

    \end{textblock}


    \begin{textblock}{2.5}(6.35,2.4)

      \includegraphics[scale=0.1]{\ZintegrUJLogoColorPictureEN}

    \end{textblock}


    \begin{textblock}{4.2}(8.4,2.6)

      \includegraphics[scale=0.3]{\EUSocialFundLogoColorPictureEN}

    \end{textblock}

  \end{frame}





  \TitleSlideWithPicture
}



\newcommand{\GeometryThreeDSpecialEndingSlideEN}{%
  \begin{frame}[standout]

    \begin{textblock}{11}(1,0.7)

      \begin{flushleft}

        \mdseries

        \footnotesize

        \color{jFrametitleFGColor}

        This content was created as part of a project co-financed by the
        European Union within the framework of the European Social Fund
        POWR.03.05.00-00-Z309/17-00.

      \end{flushleft}

    \end{textblock}





    \begin{textblock}{10}(0,2.2)

      \tikz \fill[color=jBackgroundStyleLight] (0,0) rectangle (12.8,-1.5);

    \end{textblock}


    \begin{textblock}{3.2}(0.7,2.45)

      \includegraphics[scale=0.3]{\FundingLogoColorPictureEN}

    \end{textblock}


    \begin{textblock}{2.5}(4.15,2.5)

      \includegraphics[scale=0.2]{\JULogoColorPictureEN}

    \end{textblock}


    \begin{textblock}{2.5}(6.35,2.4)

      \includegraphics[scale=0.1]{\ZintegrUJLogoColorPictureEN}

    \end{textblock}


    \begin{textblock}{4.2}(8.4,2.6)

      \includegraphics[scale=0.3]{\EUSocialFundLogoColorPictureEN}

    \end{textblock}





    \begin{textblock}{11}(1,4)

      \begin{flushleft}

        \mdseries

        \footnotesize

        \RaggedRight

        \color{jFrametitleFGColor}

        The content of this lecture is made available under a~Creative
        Commons licence (\textsc{cc}), giving the author the credits
        (\textsc{by}) and putting an obligation to share on the same terms
        (\textsc{sa}). Figures and diagrams included in the lecture are
        authored by Paweł Węgrzyn et~al., and are available under the same
        license unless indicated otherwise.\\ The presentation uses the
        Beamer Jagiellonian theme based on Matthias Vogelgesang’s
        Metropolis theme, available under license \LaTeX{} Project
        Public License~1.3c at: \colorhref{https://github.com/matze/mtheme}
        {https://github.com/matze/mtheme}.

        Typographic design: Iwona Grabska-Gradzińska \\
        \LaTeX{} Typesetting: Kamil Ziemian \\
        Proofreading: Wojciech Palacz,
        Monika Stawicka \\
        3D Models: Dariusz Frymus, Kamil Nowakowski \\
        Figures and charts: Kamil Ziemian, Paweł Węgrzyn, Wojciech Palacz

      \end{flushleft}

    \end{textblock}

  \end{frame}
}



\newcommand{\GeometryThreeDTwoSpecialEndingSlidesEN}[1]{%
  \begin{frame}[standout]


    \begin{textblock}{11}(1,0.7)

      \begin{flushleft}

        \mdseries

        \footnotesize

        \color{jFrametitleFGColor}

        This content was created as part of a project co-financed by the
        European Union within the framework of the European Social Fund
        POWR.03.05.00-00-Z309/17-00.

      \end{flushleft}

    \end{textblock}





    \begin{textblock}{10}(0,2.2)

      \tikz \fill[color=jBackgroundStyleLight] (0,0) rectangle (12.8,-1.5);

    \end{textblock}


    \begin{textblock}{3.2}(0.7,2.45)

      \includegraphics[scale=0.3]{\FundingLogoColorPictureEN}

    \end{textblock}


    \begin{textblock}{2.5}(4.15,2.5)

      \includegraphics[scale=0.2]{\JULogoColorPictureEN}

    \end{textblock}


    \begin{textblock}{2.5}(6.35,2.4)

      \includegraphics[scale=0.1]{\ZintegrUJLogoColorPictureEN}

    \end{textblock}


    \begin{textblock}{4.2}(8.4,2.6)

      \includegraphics[scale=0.3]{\EUSocialFundLogoColorPictureEN}

    \end{textblock}





    \begin{textblock}{11}(1,4)

      \begin{flushleft}

        \mdseries

        \footnotesize

        \RaggedRight

        \color{jFrametitleFGColor}

        The content of this lecture is made available under a~Creative
        Commons licence (\textsc{cc}), giving the author the credits
        (\textsc{by}) and putting an obligation to share on the same terms
        (\textsc{sa}). Figures and diagrams included in the lecture are
        authored by Paweł Węgrzyn et~al., and are available under the same
        license unless indicated otherwise.\\ The presentation uses the
        Beamer Jagiellonian theme based on Matthias Vogelgesang’s
        Metropolis theme, available under license \LaTeX{} Project
        Public License~1.3c at: \colorhref{https://github.com/matze/mtheme}
        {https://github.com/matze/mtheme}.

        Typographic design: Iwona Grabska-Gradzińska \\
        \LaTeX{} Typesetting: Kamil Ziemian \\
        Proofreading: Wojciech Palacz,
        Monika Stawicka \\
        3D Models: Dariusz Frymus, Kamil Nowakowski \\
        Figures and charts: Kamil Ziemian, Paweł Węgrzyn, Wojciech Palacz

      \end{flushleft}

    \end{textblock}

  \end{frame}





  \begin{frame}[standout]

    \begingroup

    \color{jFrametitleFGColor}

    #1

    \endgroup

  \end{frame}
}



\newcommand{\GeometryThreeDSpecialEndingSlideVideoVerOneEN}{%
  \begin{frame}[standout]

    \begin{textblock}{11}(1,0.7)

      \begin{flushleft}

        \mdseries

        \footnotesize

        \color{jFrametitleFGColor}

        This content was created as part of a project co-financed by the
        European Union within the framework of the European Social Fund
        POWR.03.05.00-00-Z309/17-00.

      \end{flushleft}

    \end{textblock}





    \begin{textblock}{10}(0,2.2)

      \tikz \fill[color=jBackgroundStyleLight] (0,0) rectangle (12.8,-1.5);

    \end{textblock}


    \begin{textblock}{3.2}(0.7,2.45)

      \includegraphics[scale=0.3]{\FundingLogoColorPictureEN}

    \end{textblock}


    \begin{textblock}{2.5}(4.15,2.5)

      \includegraphics[scale=0.2]{\JULogoColorPictureEN}

    \end{textblock}


    \begin{textblock}{2.5}(6.35,2.4)

      \includegraphics[scale=0.1]{\ZintegrUJLogoColorPictureEN}

    \end{textblock}


    \begin{textblock}{4.2}(8.4,2.6)

      \includegraphics[scale=0.3]{\EUSocialFundLogoColorPictureEN}

    \end{textblock}





    \begin{textblock}{11}(1,4)

      \begin{flushleft}

        \mdseries

        \footnotesize

        \RaggedRight

        \color{jFrametitleFGColor}

        The content of this lecture is made available under a Creative
        Commons licence (\textsc{cc}), giving the author the credits
        (\textsc{by}) and putting an obligation to share on the same terms
        (\textsc{sa}). Figures and diagrams included in the lecture are
        authored by Paweł Węgrzyn et~al., and are available under the same
        license unless indicated otherwise.\\ The presentation uses the
        Beamer Jagiellonian theme based on Matthias Vogelgesang’s
        Metropolis theme, available under license \LaTeX{} Project
        Public License~1.3c at: \colorhref{https://github.com/matze/mtheme}
        {https://github.com/matze/mtheme}.

        Typographic design: Iwona Grabska-Gradzińska;
        \LaTeX{} Typesetting: Kamil Ziemian \\
        Proofreading: Wojciech Palacz,
        Monika Stawicka \\
        3D Models: Dariusz Frymus, Kamil Nowakowski \\
        Figures and charts: Kamil Ziemian, Paweł Węgrzyn, Wojciech
        Palacz \\
        Film editing: Agencja Filmowa Film \& Television Production~--
        Zbigniew Masklak

      \end{flushleft}

    \end{textblock}

  \end{frame}
}



\newcommand{\GeometryThreeDSpecialEndingSlideVideoVerTwoEN}{%
  \begin{frame}[standout]

    \begin{textblock}{11}(1,0.7)

      \begin{flushleft}

        \mdseries

        \footnotesize

        \color{jFrametitleFGColor}

        This content was created as part of a project co-financed by the
        European Union within the framework of the European Social Fund
        POWR.03.05.00-00-Z309/17-00.

      \end{flushleft}

    \end{textblock}





    \begin{textblock}{10}(0,2.2)

      \tikz \fill[color=jBackgroundStyleLight] (0,0) rectangle (12.8,-1.5);

    \end{textblock}


    \begin{textblock}{3.2}(0.7,2.45)

      \includegraphics[scale=0.3]{\FundingLogoColorPictureEN}

    \end{textblock}


    \begin{textblock}{2.5}(4.15,2.5)

      \includegraphics[scale=0.2]{\JULogoColorPictureEN}

    \end{textblock}


    \begin{textblock}{2.5}(6.35,2.4)

      \includegraphics[scale=0.1]{\ZintegrUJLogoColorPictureEN}

    \end{textblock}


    \begin{textblock}{4.2}(8.4,2.6)

      \includegraphics[scale=0.3]{\EUSocialFundLogoColorPictureEN}

    \end{textblock}





    \begin{textblock}{11}(1,4)

      \begin{flushleft}

        \mdseries

        \footnotesize

        \RaggedRight

        \color{jFrametitleFGColor}

        The content of this lecture is made available under a Creative
        Commons licence (\textsc{cc}), giving the author the credits
        (\textsc{by}) and putting an obligation to share on the same terms
        (\textsc{sa}). Figures and diagrams included in the lecture are
        authored by Paweł Węgrzyn et~al., and are available under the same
        license unless indicated otherwise.\\ The presentation uses the
        Beamer Jagiellonian theme based on Matthias Vogelgesang’s
        Metropolis theme, available under license \LaTeX{} Project
        Public License~1.3c at: \colorhref{https://github.com/matze/mtheme}
        {https://github.com/matze/mtheme}.

        Typographic design: Iwona Grabska-Gradzińska;
        \LaTeX{} Typesetting: Kamil Ziemian \\
        Proofreading: Wojciech Palacz,
        Monika Stawicka \\
        3D Models: Dariusz Frymus, Kamil Nowakowski \\
        Figures and charts: Kamil Ziemian, Paweł Węgrzyn, Wojciech
        Palacz \\
        Film editing: IMAVI -- Joanna Kozakiewicz, Krzysztof Magda, Nikodem
        Frodyma

      \end{flushleft}

    \end{textblock}

  \end{frame}
}



\newcommand{\GeometryThreeDSpecialEndingSlideVideoVerThreeEN}{%
  \begin{frame}[standout]

    \begin{textblock}{11}(1,0.7)

      \begin{flushleft}

        \mdseries

        \footnotesize

        \color{jFrametitleFGColor}

        This content was created as part of a project co-financed by the
        European Union within the framework of the European Social Fund
        POWR.03.05.00-00-Z309/17-00.

      \end{flushleft}

    \end{textblock}





    \begin{textblock}{10}(0,2.2)

      \tikz \fill[color=jBackgroundStyleLight] (0,0) rectangle (12.8,-1.5);

    \end{textblock}


    \begin{textblock}{3.2}(0.7,2.45)

      \includegraphics[scale=0.3]{\FundingLogoColorPictureEN}

    \end{textblock}


    \begin{textblock}{2.5}(4.15,2.5)

      \includegraphics[scale=0.2]{\JULogoColorPictureEN}

    \end{textblock}


    \begin{textblock}{2.5}(6.35,2.4)

      \includegraphics[scale=0.1]{\ZintegrUJLogoColorPictureEN}

    \end{textblock}


    \begin{textblock}{4.2}(8.4,2.6)

      \includegraphics[scale=0.3]{\EUSocialFundLogoColorPictureEN}

    \end{textblock}





    \begin{textblock}{11}(1,4)

      \begin{flushleft}

        \mdseries

        \footnotesize

        \RaggedRight

        \color{jFrametitleFGColor}

        The content of this lecture is made available under a Creative
        Commons licence (\textsc{cc}), giving the author the credits
        (\textsc{by}) and putting an obligation to share on the same terms
        (\textsc{sa}). Figures and diagrams included in the lecture are
        authored by Paweł Węgrzyn et~al., and are available under the same
        license unless indicated otherwise.\\ The presentation uses the
        Beamer Jagiellonian theme based on Matthias Vogelgesang’s
        Metropolis theme, available under license \LaTeX{} Project
        Public License~1.3c at: \colorhref{https://github.com/matze/mtheme}
        {https://github.com/matze/mtheme}.

        Typographic design: Iwona Grabska-Gradzińska;
        \LaTeX{} Typesetting: Kamil Ziemian \\
        Proofreading: Wojciech Palacz,
        Monika Stawicka \\
        3D Models: Dariusz Frymus, Kamil Nowakowski \\
        Figures and charts: Kamil Ziemian, Paweł Węgrzyn, Wojciech
        Palacz \\
        Film editing: Agencja Filmowa Film \& Television Production~--
        Zbigniew Masklak \\
        Film editing: IMAVI -- Joanna Kozakiewicz, Krzysztof Magda, Nikodem
        Frodyma

      \end{flushleft}

    \end{textblock}

  \end{frame}
}



\newcommand{\GeometryThreeDTwoSpecialEndingSlidesVideoVerOneEN}[1]{%
  \begin{frame}[standout]

    \begin{textblock}{11}(1,0.7)

      \begin{flushleft}

        \mdseries

        \footnotesize

        \color{jFrametitleFGColor}

        This content was created as part of a project co-financed by the
        European Union within the framework of the European Social Fund
        POWR.03.05.00-00-Z309/17-00.

      \end{flushleft}

    \end{textblock}





    \begin{textblock}{10}(0,2.2)

      \tikz \fill[color=jBackgroundStyleLight] (0,0) rectangle (12.8,-1.5);

    \end{textblock}


    \begin{textblock}{3.2}(0.7,2.45)

      \includegraphics[scale=0.3]{\FundingLogoColorPictureEN}

    \end{textblock}


    \begin{textblock}{2.5}(4.15,2.5)

      \includegraphics[scale=0.2]{\JULogoColorPictureEN}

    \end{textblock}


    \begin{textblock}{2.5}(6.35,2.4)

      \includegraphics[scale=0.1]{\ZintegrUJLogoColorPictureEN}

    \end{textblock}


    \begin{textblock}{4.2}(8.4,2.6)

      \includegraphics[scale=0.3]{\EUSocialFundLogoColorPictureEN}

    \end{textblock}





    \begin{textblock}{11}(1,4)

      \begin{flushleft}

        \mdseries

        \footnotesize

        \RaggedRight

        \color{jFrametitleFGColor}

        The content of this lecture is made available under a Creative
        Commons licence (\textsc{cc}), giving the author the credits
        (\textsc{by}) and putting an obligation to share on the same terms
        (\textsc{sa}). Figures and diagrams included in the lecture are
        authored by Paweł Węgrzyn et~al., and are available under the same
        license unless indicated otherwise.\\ The presentation uses the
        Beamer Jagiellonian theme based on Matthias Vogelgesang’s
        Metropolis theme, available under license \LaTeX{} Project
        Public License~1.3c at: \colorhref{https://github.com/matze/mtheme}
        {https://github.com/matze/mtheme}.

        Typographic design: Iwona Grabska-Gradzińska;
        \LaTeX{} Typesetting: Kamil Ziemian \\
        Proofreading: Wojciech Palacz,
        Monika Stawicka \\
        3D Models: Dariusz Frymus, Kamil Nowakowski \\
        Figures and charts: Kamil Ziemian, Paweł Węgrzyn,
        Wojciech Palacz \\
        Film editing: Agencja Filmowa Film \& Television Production~--
        Zbigniew Masklak

      \end{flushleft}

    \end{textblock}

  \end{frame}





  \begin{frame}[standout]


    \begingroup

    \color{jFrametitleFGColor}

    #1

    \endgroup

  \end{frame}
}



\newcommand{\GeometryThreeDTwoSpecialEndingSlidesVideoVerTwoEN}[1]{%
  \begin{frame}[standout]

    \begin{textblock}{11}(1,0.7)

      \begin{flushleft}

        \mdseries

        \footnotesize

        \color{jFrametitleFGColor}

        This content was created as part of a project co-financed by the
        European Union within the framework of the European Social Fund
        POWR.03.05.00-00-Z309/17-00.

      \end{flushleft}

    \end{textblock}





    \begin{textblock}{10}(0,2.2)

      \tikz \fill[color=jBackgroundStyleLight] (0,0) rectangle (12.8,-1.5);

    \end{textblock}


    \begin{textblock}{3.2}(0.7,2.45)

      \includegraphics[scale=0.3]{\FundingLogoColorPictureEN}

    \end{textblock}


    \begin{textblock}{2.5}(4.15,2.5)

      \includegraphics[scale=0.2]{\JULogoColorPictureEN}

    \end{textblock}


    \begin{textblock}{2.5}(6.35,2.4)

      \includegraphics[scale=0.1]{\ZintegrUJLogoColorPictureEN}

    \end{textblock}


    \begin{textblock}{4.2}(8.4,2.6)

      \includegraphics[scale=0.3]{\EUSocialFundLogoColorPictureEN}

    \end{textblock}





    \begin{textblock}{11}(1,4)

      \begin{flushleft}

        \mdseries

        \footnotesize

        \RaggedRight

        \color{jFrametitleFGColor}

        The content of this lecture is made available under a Creative
        Commons licence (\textsc{cc}), giving the author the credits
        (\textsc{by}) and putting an obligation to share on the same terms
        (\textsc{sa}). Figures and diagrams included in the lecture are
        authored by Paweł Węgrzyn et~al., and are available under the same
        license unless indicated otherwise.\\ The presentation uses the
        Beamer Jagiellonian theme based on Matthias Vogelgesang’s
        Metropolis theme, available under license \LaTeX{} Project
        Public License~1.3c at: \colorhref{https://github.com/matze/mtheme}
        {https://github.com/matze/mtheme}.

        Typographic design: Iwona Grabska-Gradzińska;
        \LaTeX{} Typesetting: Kamil Ziemian \\
        Proofreading: Wojciech Palacz,
        Monika Stawicka \\
        3D Models: Dariusz Frymus, Kamil Nowakowski \\
        Figures and charts: Kamil Ziemian, Paweł Węgrzyn,
        Wojciech Palacz \\
        Film editing: IMAVI -- Joanna Kozakiewicz, Krzysztof Magda, Nikodem
        Frodyma

      \end{flushleft}

    \end{textblock}

  \end{frame}





  \begin{frame}[standout]


    \begingroup

    \color{jFrametitleFGColor}

    #1

    \endgroup

  \end{frame}
}



\newcommand{\GeometryThreeDTwoSpecialEndingSlidesVideoVerThreeEN}[1]{%
  \begin{frame}[standout]

    \begin{textblock}{11}(1,0.7)

      \begin{flushleft}

        \mdseries

        \footnotesize

        \color{jFrametitleFGColor}

        This content was created as part of a project co-financed by the
        European Union within the framework of the European Social Fund
        POWR.03.05.00-00-Z309/17-00.

      \end{flushleft}

    \end{textblock}





    \begin{textblock}{10}(0,2.2)

      \tikz \fill[color=jBackgroundStyleLight] (0,0) rectangle (12.8,-1.5);

    \end{textblock}


    \begin{textblock}{3.2}(0.7,2.45)

      \includegraphics[scale=0.3]{\FundingLogoColorPictureEN}

    \end{textblock}


    \begin{textblock}{2.5}(4.15,2.5)

      \includegraphics[scale=0.2]{\JULogoColorPictureEN}

    \end{textblock}


    \begin{textblock}{2.5}(6.35,2.4)

      \includegraphics[scale=0.1]{\ZintegrUJLogoColorPictureEN}

    \end{textblock}


    \begin{textblock}{4.2}(8.4,2.6)

      \includegraphics[scale=0.3]{\EUSocialFundLogoColorPictureEN}

    \end{textblock}





    \begin{textblock}{11}(1,4)

      \begin{flushleft}

        \mdseries

        \footnotesize

        \RaggedRight

        \color{jFrametitleFGColor}

        The content of this lecture is made available under a Creative
        Commons licence (\textsc{cc}), giving the author the credits
        (\textsc{by}) and putting an obligation to share on the same terms
        (\textsc{sa}). Figures and diagrams included in the lecture are
        authored by Paweł Węgrzyn et~al., and are available under the same
        license unless indicated otherwise. \\ The presentation uses the
        Beamer Jagiellonian theme based on Matthias Vogelgesang’s
        Metropolis theme, available under license \LaTeX{} Project
        Public License~1.3c at: \colorhref{https://github.com/matze/mtheme}
        {https://github.com/matze/mtheme}.

        Typographic design: Iwona Grabska-Gradzińska;
        \LaTeX{} Typesetting: Kamil Ziemian \\
        Proofreading: Leszek Hadasz, Wojciech Palacz,
        Monika Stawicka \\
        3D Models: Dariusz Frymus, Kamil Nowakowski \\
        Figures and charts: Kamil Ziemian, Paweł Węgrzyn,
        Wojciech Palacz \\
        Film editing: Agencja Filmowa Film \& Television Production~--
        Zbigniew Masklak \\
        Film editing: IMAVI -- Joanna Kozakiewicz, Krzysztof Magda, Nikodem
        Frodyma


      \end{flushleft}

    \end{textblock}

  \end{frame}





  \begin{frame}[standout]


    \begingroup

    \color{jFrametitleFGColor}

    #1

    \endgroup

  \end{frame}
}











% ------------------------------------------------------
% Packages, libraries and their settings
% ------------------------------------------------------
% Library improving positioning of nodes in graphs
\usetikzlibrary{positioning}





% ------------------------------------------------------
% Local packages
% ------------------------------------------------------
% Local configuration of this particular presentation
\usepackage{./Local-packages/local-settings}

% Stylef for drawing diagrams
\usepackage{./Local-packages/PGF-TikZ-Diagram-styles}










% ------------------------------------------------------------------------------------------------------------------
\title{Algorytmy kompilacji}
\subtitle{Wprowadzenie do~przedmiotu}

\author{Kamil Ziemian \\
  \email}


% \date{}
% ------------------------------------------------------------------------------------------------------------------










% ####################################################################
% Beginning of the document
\begin{document}
% ####################################################################





% ######################################
% Text is adjusted to the left and words are broken at the end of the line.
\RaggedRight
% Number of chars: 62k+, 80k+, 41k+, 36k+,
% ######################################





% ######################################
\maketitle
% ######################################





% ##################
\begin{frame}
  \frametitle{Spis treści}


  \tableofcontents

\end{frame}
% ##################





% ######################################
\section{Informacje ogólne}
% ######################################





% ##################
\begin{frame}
  \frametitle{Informacje wstępne}


  Obawiam~się, że na tych konkretnych zajęciach będzie sporo przynudzania,
  ale nie widzę sposobu, by~tego uniknąć.

  Według mnie to zajęcia są dla studentów, nie studenci dla zajęć. Tak samo
  ja jestem tu dla Państwa, a~nie Państwo dla mnie. W~związku z~tym, ja
  będę Państwa rozliczał tylko i~wyłącznie z~umiejętności i~wiedzy,
  z~niczego innego. Wychodzę bowiem z~założenia, że~Państwo sami najlepiej
  wiedzą, czemu warto poświęcić swój czas. (Choć jak dobrze wiemy, często
  po jakimś czasie stwierdzamy, że~może trzeba było wybrać coś innego.)

  Na zajęciach nie tylko można, ale \alert{należy} zadawać pytania
  na dowolne związane z~zajęciami zagadnienia. W~szczególności
  \alert{należy} zadawać pytania, jeśli~się czegoś nie rozumie, lub coś
  jest niejasne. Kompilatory to zagadnienie na którym dobrze zna się jakiś
  0.1\% informatyków, \alert{nie} zakładamy, że~Państwo należą do tej
  wąskiej grupy.

\end{frame}
% ##################





% ##################
\begin{frame}
  \frametitle{Informacje wstępne}


  Proszę pamiętać, gdy chodzi o~tematy związane z~zajęciami
  \alert{nie} ma pytań zbyt elementarnych lub zbyt głupich. Są~tylko
  niezadowalające odpowiedzi.

  Na zajęciach nie tylko można, ale \alert{należy} zadawać pytania
  na dowolne związane z~zajęciami zagadnienia. Nie ma tu pytań zbyt
  elementarnych lub zbyt głupich, są~tylko niezadowalające odpowiedzi.

  Pytania typu „Jaki jest najfajniejszy boss w~grze \textit{Hollow
    Knight}?” musimy jednak zostawić na czas po zajęciach.

\end{frame}
% ##################





% ##################
\begin{frame}
  \frametitle{Uwagi odnośnie treści zajęć}


  Współczesna informatyka jest dziedziną bardzo młodą. Ma sens datowanie
  jej powstania na rok 1945, gdy technologie komputerowe rozwinięte na
  potrzeby wojskowe w~czasie II~Wojny Światowej wchodzą do użytku
  powszechnego, choć z~przyczyn technicznych dostępne są bardzo wąskiej
  grupie ludzi. To czym się zajmujemy nie ma jeszcze 100~lat. Nauka ma to
  do siebie, że~potrafi naprawdę długo~się rozwijać, potrzebując niekiedy
  dekad jeśli nie wieków, by dokonać realnego postępu. W~informatyce zaś
  cały czas odkrywamy nowe i~kluczowe dla tej dziedziny rzeczy.

  W~szczególności, żyjemy teraz w~epoce intensywnego rozwoju kompilatorów,
  która raczej~się nie skończy już jutro. W~skutek tego cała masa rzeczy~się
  ciągle zmienia i~to co wczoraj uchodziło za czystą fantazję, dziś jest
  rzeczywistością wdrażaną do produkcji przemysłowej. Proszę mieć to na
  uwadze biorąc udział w~zajęciach i~ucząc~się do tego kursu.

\end{frame}
% ##################





% ##################
\begin{frame}
  \frametitle{Bardzo ważne}


  Ze względu na ramy czasowe tego wiele rzeczy będę \alert{upraszczał} lub
  \alert{pomijał}. Proszę o~tym pamiętać podczas uczęszczania na zajęcia
  i~czytania materiałów z~nich.

  Jeśli jednak ktoś chce~się zagłębić w~nie bardziej, po zajęciach służę
  całą moją wiedzą.

  Z~mojego doświadczenia wynika, że~ustalanie jednego terminu na konsultacje
  to nie jest dobry pomysł. W~zasadzie nikt wtedy nie przychodzi, a~ja
  wyznaję zasadę, że~konsultacje są dla Państwa, nie dla mnie. Jeśli
  Państwo chcą bym ustalił konkretne terminy na konsultacje, to proszę jako
  grupa wybrać jeden taki i~poinformować mnie o~tym mailowo, pisząc na
  adres \email.

\end{frame}
% ##################





% ##################
\begin{frame}
  \frametitle{Bardzo ważne}


  W~przeciwnym razie, jeśli ktoś z~Państwa ma problem i~chce zasięgnąć
  mojej pomocy, proszę do mnie napisać, na wspomniany już adres \email,
  kiedy, gdzie i~w~jakiej formie chcą Państwo uczestniczyć w konsultacjach.

  Mogą one być zarówno w~świecie rzeczywistym (niekoniecznie w~budynku
  \textsc{WSZiBu}), online lub telefonicznie.

  Będę wdzięczny za napisanie w~mailu z~czym konkretnie mają Państwo
  problem, rozumiem jedna, że~często wskazanie czy nazwanie tego jest
  trudne. Sam przez to przechodziłem.

\end{frame}
% ##################





% ##################
\begin{frame}
  \frametitle{Materiały do nauki}


  Prezentacje te są dostępne w~formie plików \LaTeX a (kodu źródłowego)
  na serwisie GitHub. Każdy kto ma na komputerze program Git i~dostęp
  do internetu może jest zdobyć wpisując \\
  \texttt{\$ git clone https://github.com/KZiemian/Presentation} \\
  Znajdują~się one w~katalogu „Podstawy-informatyki-ETC-Prezentacje”.

  Będą też dostępne w~formie PDFów na~Sake, wraz z~innymi materiałami
  do nauki.

\end{frame}
% ##################





% ##################
\begin{frame}
  \frametitle{Zgłaszanie błędu i~uwag}


  W~razie znalezienia jakiegokolwiek błędu lub jakichkolwiek uwag
  merytorycznych do zajęć lub dostępnych materiałów proszę pisać pod adres
  \email. Chcemy by te zajęcia i~towarzyszące im materiały były możliwie
  proste, łatwe w~zrozumieniu i~pozbawione błędów. Proszę jednak uwierzyć,
  że~osiągnięcie tego jest naprawdę trudne.

\end{frame}
% ##################










% ######################################
\section{Dygresja o~diagramach występujących w~tych
  prezentacjach}
% ######################################



% ##################
\begin{frame}
  \frametitle{Konwencja}


  \begin{textblock}{2.8}(2,1.5)

    \begin{tikzpicture}

      \node[diagram block] at (0,0) {Coś robi};

    \end{tikzpicture}

  \end{textblock}



  \begin{textblock}{2.8}(8,1.5)

    \begin{tikzpicture}

      \node[diagram rectangle block] at (0,0) {Czymś jest};

    \end{tikzpicture}

  \end{textblock}


  \vspace{6em}





  Bloki diagramu które mają kształt prostokąta z~zaokrąglonymi rogami
  (zwykle~są koloru niebieskiego) oznaczają \textbf{aktorów}, czyli
  taki obiekt który wykonuje jakąś czynność na zadanych obiektach
  wejściowych. Takim obiektem może być człowiek, komputer, program
  komputerowy, fragment programu komputerowego, etc.

  Bloki diagramu które mają kształt prostokąta z~ostrymi rogami (zwykle~są
  koloru karmelowego), przedstawiają rzeczy które są pobierane na~wejście
  przez aktorów lub przez nich wytwarzane.

\end{frame}
% ##################





% ##################
\begin{frame}
  \frametitle{Przykładowy diagram}


  Konwencję tą ilustruje poniższy, zrobionym z~przymrużeniem oka, diagram.
  Przedstawia on proces tworzenia przez człowieka za pomocą klawiatury
  i~komputera programu w~języku programowania~C.





  \begin{figure}

    \label{fig:Creating-code-in-C}

    \begin{tikzpicture}

      \node[diagram block] (Man) at (0,0) {Człowiek};

      \node[diagram block] (Keyboard) at (3.8,0) {Klawiatura};

      \draw[thick diagram arrow] (Man) -- (Keyboard);



      \node[diagram block] (Computer) at (7.6,0) {Komputer};

      \draw[thick diagram arrow] (Keyboard) -- (Computer);



      \node[diagram rectangle block] (Source code) at (7.6,-2.5)
      {Kod źródłowy w~języku~C};

      \draw[thick diagram arrow] (Computer) -- (Source code);

    \end{tikzpicture}

    \caption{Diagram ilustrujący tworzenie kodu w~języku~C.}


  \end{figure}

\end{frame}
% ##################





% ##################
\begin{frame}
  \frametitle{Inny diagram}


  Czy poniższy diagram jest lepszy czy gorszy od poprzedniego? Zależy
  od~tego jakie informacje ma dany diagram przekazywać. Proszę~się więc nie
  dziwić, jeśli na danym diagramie czegoś nie ma, a~nawet, że~nie ma czegoś
  co było na poprzedniej wersji diagramu. Zawsze trzeba dokonać selekcji co
  na diagramie zostanie umieszczone, a~co nie i~ten wybór często zależy
  od kontekstu.

  Jeśli uważają Państwo, że~na diagramie nie ma czegoś, co być powinno,
  albo coś innego jest nie w~porządku, \alert{proszę} to mi powiedzieć.
  Całkiem możliwe, że~popełniłem błąd rysując dany diagram.






  \begin{figure}

    \label{fig:Also-creating-code-in-C}

    \begin{tikzpicture}

      \node[diagram block] (Man) at (0,0) {Człowiek};

      \node[diagram rectangle block] (Source code) at (3.8,0)
      {Kod źródłowy w~języku~C};

      \draw[thick diagram arrow] (Man) -- (Source code);

    \end{tikzpicture}

    \caption{Inna wersja diagramu ilustrującego tworzenie kodu w~języku~C.}


  \end{figure}

\end{frame}
% ##################










% ######################################
\section{Spojrzenie na~przedmiot z~lotu ptaka}
% ######################################



% ##################
\begin{frame}
  \frametitle{Czy ten kurs jest praktyczny?}


  Jeśli kiedyś napisaliście Państwo i~uruchomili program w~języku C, C++,
  Fortran, Go, Java, Rust, etc., to na 99\% korzystali Państwo
  z~kompilatora. W~2024 cała infrastruktura informatyczna stoi
  na~kodzie napisany w~języku~C, który odpowiednie kompilatory przetworzyły
  w~programy komputerowe. W~tym sensie kurs ten jest superpraktyczny, bo
  ktoś te kompilatory musi pisać.

  Jeśli jednak nie mają Państwo zamiaru zajmować~się zawodowo pracą nad
  kompilatorami, jak przytłaczająca większość informatyków, to ten kurs nie
  będzie miał jakiejś szczególnie wielkiej wartości praktycznej dla Państw.
  Niemniej nawet w~takiej sytuacji mogą Państwo wynieść z~niego jakąś
  wiedzę i~umiejętności, które potencjalnie pozwoli wam później tworzyć
  bardziej wydajne programy. Jeśli mamy jakieś wyobrażenie na temat tego
  jak kompilator działa, to możemy mu ułatwić tworzenie wydajny programów.

\end{frame}
% ##################





% ##################
\begin{frame}
  \frametitle{Dlaczego potrzebujemy kompilatorów?}

  \pause


  Żeby komputer coś zrobił musimy więc mu przekazać polecenie w~języku,
  który rozumie. Dla mnie pierwotny językiem jest język polski, co zaś
  jest pierwotnym językiem komputera? Na potrzeby tego kursu przyjmiemy,
  że~pierwotnym językiem komputera jest \textbf{język asembler}
  (ang. \textit{assembly language}).

  Tak jak niektórzy ludzie mają jako swój pierwotny język angielski,
  hiszpański, japoński, niemiecki, polski, etc., tak komputer również
  posługują~się różnymi rodzajami języka asembler, takimi jak \textsc{arm}
  czy x86/Intel. Można też wymienić dialekty asemblera \textsc{gas},
  \textsc{fasm}, \textsc{masm}, \textsc{nasm}, \textsc{yasm} (Kto wymyśla
  te nazwy?).

  Różnica między komputerem, a~człowiekiem jest taka, że~pojedynczy człowiek
  poza swoim pierwotnym językiem, może znać kilka innych (angielski,
  arabski, chiński, farsi, francuski, japoński, niemiecki, polski, etc.).
  Komputery zwykle rozumieją tak naprawdę tylko \alert{jeden jedyny}
  język: właściwy mu dialekt asemblera.

\end{frame}
% ##################





% ##################
\begin{frame}
  \frametitle{Dlaczego potrzebujemy kompilatorów?}


  To który dialekt asemblera rozumiem komputer jest ustalony przez firmę,
  która wyprodukowała jego procesor. Asembler jest bowiem zakodowany w~tym
  jak są „podpięte kable” w~danym procesorze.

  By lepiej zrozumieć dlaczego potrzebujemy kompilatorów, przedstawimy
  teraz jeden program, który wypisuje „Hello, World!” na ekranie, napisany
  w~dialektach asemblera ??? oraz językach C i~Python.

  Wersje programu napisane w~\textsc{arm} 32,????, językach C i~Pythonie
  zostały przetestowane przeze mnie i~jeśli są chętni, to możemy je głębiej
  omówić.

\end{frame}
% ##################





% ##################
\begin{frame}
  \frametitle{„Hello, World!” w~asemblerze ARM~32,
    \parencite{Low-Level-Learning-You-Can-Learn-ARM-ETC-Ver-2020}}


  \texttt{.global \_start} \\
  \texttt{.section .text} \\



  \texttt{\_start:} \\[-0.2em]
  \hphantom{aaaaaaaa} \texttt{mov r7, \#0x4} \\
  \hphantom{aaaaaaaa} \texttt{mov r0, \#1} \\
  \hphantom{aaaaaaaa} \texttt{ldr r1, =message} \\
  \hphantom{aaaaaaaa} \texttt{mov r2, \#14} \\

  \hphantom{aaaaaaaa} \texttt{swi 0} \\

  \hphantom{aaaaaaaa} \texttt{mov r7, \#0x1} \\
  \hphantom{aaaaaaaa} \texttt{mov r0, \#65} \\

  \hphantom{aaaaaaaa} \texttt{swi 0} \\



  \texttt{.section .data} \\
  \texttt{message:} \\
  \hphantom{aaaaaaaa} \texttt{.ascii "Hello, World!\textbackslash n"}

\end{frame}
% ##################





% ##################
\begin{frame}
  \frametitle{„Hello, World!” w~asemblerze AArch64,
    \parencite{Low-Level-Learning-You-Can-Learn-AArch64-ETC-Ver-2020}}


  \texttt{.global \_start} \\
  \texttt{.section .text} \\



  \texttt{\_start:} \\[-0.2em]
  \hphantom{aaaaaaaa} \texttt{mov x8, \#64} \\
  \hphantom{aaaaaaaa} \texttt{mov x0, \#1} \\
  \hphantom{aaaaaaaa} \texttt{ldr x1, =message} \\
  \hphantom{aaaaaaaa} \texttt{mov x2, \#14} \\

  \hphantom{aaaaaaaa} \texttt{swi 0} \\

  \hphantom{aaaaaaaa} \texttt{mov r7, \#0x1} \\
  \hphantom{aaaaaaaa} \texttt{mov r0, \#65} \\

  \hphantom{aaaaaaaa} \texttt{swi 0} \\



  \texttt{.section .data} \\
  \texttt{message:} \\
  \hphantom{aaaaaaaa} \texttt{.ascii "Hello, World!\textbackslash n"}

\end{frame}
% ##################





% ##################
\begin{frame}
  \frametitle{„Hello, World!” w~asemblerze x86/Intel
    \parencite{Anonymous-Hello-World-in-x86-Assembly-Language}}


  \texttt{org 0x100} \\
  \vspace{0.8em}

  \texttt{mov dx, msg} \\
  \texttt{mov ah, 9} \\
  \texttt{int 0x21} \\
  \vspace{0.8em}

  \texttt{mov ah, 0x4c} \\
  \texttt{int 0x21} \\
  \vspace{0.8em}

  \texttt{msg db 'Hello, World!', 0x0d, 0x0a, '\$'}

\end{frame}
% ##################





% ##################
\begin{frame}
  \frametitle{„Hello, World!” w~języku~C}


  \texttt{\#include <stdio.h>} \\
  \vspace{0.8em}
  \texttt{int main() \{ } \\
  \hphantom{aaaa} \texttt{printf("Hello, World!\textbackslash n");} \\
  \vspace{0.8em}
  \vspace{0.8em}
  \vspace{0.8em}
  \vspace{0.8em}
  \hphantom{aaaa} \texttt{return 0;} \\
  \texttt{ \} }

\end{frame}
% ##################





% ##################
\begin{frame}
  \frametitle{„Hello, World!” w~języku~Python}


  \texttt{print("Hello, World!")}

\end{frame}
% ##################




% ##################
\begin{frame}
  \frametitle{Dlaczego potrzebujemy kompilatorów?}


  Na pierwszy rzut oka powinno być jasne, że~C jest prostszy od~asemblera,
  a~Python prostszy od~C. Dlaczego tak jest, to niestety temat na inny
  przedmiot, acz o~kilka elementów tego zagadnienia uda nam~się zahaczyć.

  Większość ludzi zrobi wszystko, by tylko uniknąć pracy w~asemblerze.
  Mam nadzieję, że~nie muszę tłumaczyć dlaczego.  A~nawet jeśli
  znajdzie~się człowieka, który lubi w~nim pisać, to unika~się jak tylko
  można tego, by kod napisany przez niego w~asemblerze wszedł w~skład
  danego programu.

  Jest tak dlatego, że~nawet najlepsi programiści zbyt łatwo mylą~się
  pisząc w~asemblerze, a~pomyłki na jego poziomie są szczególnie
  niebezpieczne. Jeśli są zainteresowani, to możemy omówić ten temat
  szerzej, ale tylko po zajęciach. Wykracza on bowiem mocno poza zakres tego
  przedmiotu.

\end{frame}
% ##################





% ##################
\begin{frame}
  \frametitle{Dlaczego potrzebujemy kompilatorów?}


  Co więc należy zrobić by uniknąć pisania kodu w~języku asemblera? Stwórzmy
  język programowania taki jak~C czy Python, który jest dla nas prostszy
  w~użyciu niż asembler i~napiszmy program w~nim. Proszę zwrócić uwagę,
  że~„prostszy” wcale nie oznacza \alert{„prosty”}!

  Dobrze, ale komputer dalej rozumie tylko język asembler, a~nie~C czy
  Pytona, co więc musimy zrobić, by uruchomić programy napisany w~jednym
  z~tych języków? Potrzebny jest nam program komputerowy zwany
  \textbf{translatorem}, który \alert{przetłumaczy} kod napisany,
  przykładowo z~języka~C na asemblera.





  \begin{figure}

    \label{fig:Translator-01}

    \begin{tikzpicture}

      \node[diagram rectangle block] (Source code in C) at (0,0)
      {Kod programu (język C)};

      \node[diagram block] (Translator) at (3.8,0) {Translator};

      \draw[thick diagram arrow] (Source code in C) -- (Translator);


      \node[diagram rectangle block] (Code in assembly) at (7.6,0)
      {Kod programu (język asemblera)};

      \draw[thick diagram arrow] (Translator) -- (Code in assembly);

    \end{tikzpicture}

    \caption{Ilustracja działania translatora}


  \end{figure}


  % Różnica między komputerem, a~człowiekiem jest taka, że~pojedynczy człowiek
  % potrafi znać kilka różnych języków (polski, angielski, francuski,
  % niemiecki, etc.), ale komputery zwykle rozumieją tak naprawdę tylko
  % \alert{jeden jedyny} język: właściwy mu (natywny) dialekt asemblera.
  % Wszystkie inne języki (C, C++, Java, Python, etc.), są dla niego
  % niezrozumiałe. Z~tego powodu program napisany w~tych językach muszą
  % zostać najpierw \alert{przetłumaczone} do jego natywnego asemblera.

  % Czy ktoś z~Państwa korzystał z~emulatora do gier? Choćby po to by
  % uruchomić grę \textit{Pokemon Yellow/Red/Blue} przeznaczoną na GameBoya
  % na swoim PCecie? (To nie jedyny emulator do gier z~którego korzystałem.)
  % Na tej samej zasadzie działają emulatory asemblerów, pozwalające
  % uruchomić asemblera x86 na procesorze w~architekturze \textsc{arm}.

  % Tutaj pojawia~się pewien problem. Przyjrzyjmy~się teraz jednemu programowi
  % napisanemu w~dwóch??? rodzajach asemblera, językach C i~Python.

\end{frame}
% ##################





% ##################
\begin{frame}
  \frametitle{Kilka uwag uzupełniających}


  \alert{Dygresja.} Czy ktoś z~Państwa korzystał z~emulatora do gier?
  Choćby po to by uruchomić grę \textit{Pokemon Yellow/Red/Blue}
  przeznaczoną na GameBoya na swoim PCecie? (To nie jedyny emulator do gier
  z~którego korzystałem.) Na tej samej zasadzie działają emulatory
  asemblerów, pozwalające uruchomić asemblera x86 na procesorze
  zbudowanym w~architekturze \textsc{arm}.



\end{frame}
% ##################





% ##################
% !!!!!!!!!!!!!!!!!!!!!!!!!!!!!!!!!!!!!!!!
% \jagiellonianendslide{Czy są jakieś pytania do tej części?}
% ##################




















% ######################################
\section{Praktyczny problem}
% ######################################



% ##################
\begin{frame}
  \frametitle{Kilka słów przypomnienia}


  Jak mówiliśmy wcześniej, na potrzeby tego kursu będziemy przyjmować,
  że~naturalnym językiem komputera jest język asembler i~nie będziemy
  wnikać głębiej w~to co~się tam dzieje.

  Jeśli jednak ktoś chce wniknąć głębiej w~to zagadnienie, to po zajęciach
  służę całą moją wiedzą w~tym temacie.

  \alert{Uwaga terminologiczna.} W~języku angielskim rozróżnia~się
  \textbf{język asemblera} (ang. \textit{assembly langauge})
  i~\textbf{program asembler} (ang. \textit{assembler}), który operuje
  na~programie napisany w~języku asembler. W~języku polskim „asembler”
  oznacza często język asembler, co może powodować pewne nieporozumienia.

\end{frame}
% ##################





% ##################
\begin{frame}
  \frametitle{Typowy sposób wykonywania programu}


  Aby opisać jak z~grubsza wygląda wykonanie dowolnego programu muszę
  wprowadzić pewną dozę uproszczeń. Przyjmijmy po pierwsze, że~gra
  \textit{Hollow Knight} (nie gorszy przykład programu niż dowolny inny),
  zawiera~się w~jednym pliku napisany w~języku asembler. Proces
  uruchamiania tej gry pokazuje poniższy diagram.





  \begin{figure}

    \label{fig:Running-Hollow-Knight-bad}

    \begin{tikzpicture}

      \node[diagram rectangle block] (Hollow Knight) at (0,0)
      {\textit{Hollow Knight} (język asemblera)};

      \node[diagram block] (Computer) at (3.8,0) {Komputer};

      \draw[thick diagram arrow] (Hollow Knight) -- (Computer);


      \node[diagram block] (Playable computer) at (7.6,0)
      {Komputer z~włączoną grą \textit{HK}};

      \draw[thick diagram arrow] (Computer) -- (Playable computer);

    \end{tikzpicture}

    \caption{Typowy sposób uruchamiania programu na naszym komputerze}


  \end{figure}

\end{frame}
% ##################





% ##################
\begin{frame}
  \frametitle{Pisanie gier w~języku asembler?}


  Jak już wspomnieliśmy, pisanie czegokolwiek, w~tym gier, w~języku
  asembler to zwykle nie jest dobry pomysł. Znanym przykładem tego, że~jest
  to możliwe, jest gra \textit{???} z~1?? roku, którą ??? napisał
  \textit{w~pojedynkę}. Dzięki czemu stał~się bohaterem popularnego
  w~środowisku informatyków mema.

  Wklej tu tego mema.

\end{frame}
% ##################






% ##################
\begin{frame}
  \frametitle{?????}


  Dzisiaj gry często tworzy~się środowiskach takich jak Unreal Engine,
  Unity czy Godot, nie zaś pisze kod źródłowy od zera. Unreal Engine
  i~Godot są napisane w~języku C++, Unity w~języku C\#. W~tym momencie nie
  mam jak sprawdzić, w~czym został napisany „Hollow Knight” więc przyjmijmy,
  że~języku~C++. Dlaczego jednak poniższy diagram jest błędny?





  \begin{figure}

    \label{fig:Running-Hollow-Knight-good}

    \begin{tikzpicture}

      \node[diagram rectangle block] (Hollow Knight) at (0,0)
      {\textit{Hollow Knight} (język C++)};

      \node[diagram block] (Computer) at (3.8,0) {Komputer};

      \draw[thick diagram arrow] (Hollow Knight) -- (Computer);


      \node[diagram block] (Playable computer) at (7.6,0)
      {Komputer z~włączoną grą \textit{HK}};

      \draw[thick diagram arrow] (Computer) -- (Playable computer);

    \end{tikzpicture}

    \caption{Błędny sposób uruchamiania gry „Hollow Knight”}


  \end{figure}

\end{frame}
% ##################





% ##################
\begin{frame}
  \frametitle{??? }


  Komputer jako taki nie rozumie języka~C++, więc nie jest w~stanie
  uruchomić gry napisanej w~takim języku. Aby ją uruchomić, translator
  musi najpierw przetłumaczyć ją na język asemblera.





  \begin{figure}

    \begin{tikzpicture}

      \node[diagram rectangle block] (Hollow Knight in C++) at (0,0)
      {\textit{Hollow Knight} (język C++)};

      \node[diagram block] (Translator) at (3.8,0) {Translator};

      \draw[thick diagram arrow]
      (Hollow Knight in C++) -- (Translator);


      \node[diagram rectangle block] (Hollow Knight in assembly) at (7.6,0)
      {\textit{Hollow Knight} (język asemblera)};

      \draw[thick diagram arrow] (Translator) -- (Hollow Knight in assembly);


      \node[diagram block] (Computer) at (7.6,-2.5) {Komputer};

      \draw[thick diagram arrow] (Hollow Knight in assembly) -- (Computer);


      \node[diagram block] (Computer running Hollow Knight) at (3.8,-2.5)
      {Komputer z~włączoną grą \textit{HK}};

      \draw[thick diagram arrow]
      (Computer) -- (Computer running Hollow Knight);

    \end{tikzpicture}

    \caption{Poprawny sposób włączania gry \textit{Hollow Knight}}


  \end{figure}

\end{frame}
% ##################





% ##################
\begin{frame}
  \frametitle{Czy to znaczy\ldots}


  Mogą~się Państwu nasunąć taka myśl.

  „Ja coś słyszałem, że~jest oprogramowanie o~otwarty kodzie, jak system
  GNU/Linux i~o~zamkniętym kodzie, takie jak Windows. Każdy może przeczytać
  kod źródłowy otwartego oprogramowania, ale tego o~zamkniętym kodzie tylko
  ci którzy pracują dla takich firm jak Microsoft. A~Pan tu twierdzi,
  że~program który dostaję na moim komputerze jest napisany w~języku
  assemblera, więc jeśli tylko otworzę ten program i~zrozumiem asemblera
  wewnątrz niego, to tak jakby miał on otwarty kod. Po co więc to całe
  rozróżnienie na kod otwarty i~zamknięty?”

  Wszystko to prawda. Haczyk tkwi w~zdaniu „zrozumiem asemblera wewnątrz
  niego”. Zrozumienie działania kodu źródłowego w~języku~C, czy~Python jest
  często nieznośnie trudne, a~asembler czyni to zadanie jeszcze
  trudniejszym.

\end{frame}
% ##################





% ##################
\begin{frame}
  \frametitle{Proszę~się nie poddawać}


  Proszę~się jednak nie poddawać i~pamiętać, że~„każdy program ma otwarty
  kod, jeśli potrafisz wykonać jego inżynierię odwrotną” (org.
  \textit{everything is open source, if you can reverse enginerring it},
  parencite{???}) ;).

  \vspace{1.5em}





  \texttt{.global \_start} \\
  \texttt{.section .text} \\



  \texttt{\_start:} \\[-0.2em]
  \hphantom{aaaaaaaa} \texttt{mov r7, \#0x4} \\
  \hphantom{aaaaaaaa} \texttt{mov r0, \#1} \\
  \hphantom{aaaaaaaa} \texttt{ldr r1, =message} \\
  \hphantom{aaaaaaaa} \texttt{mov r2, \#14} \\

  \hphantom{aaaaaaaa} \texttt{swi 0} \\

  \hphantom{aaaaaaaa} \texttt{mov r7, \#0x1} \\
  \hspace{5em} \vdots
  % \hphantom{aaaaaaaa} \texttt{mov r0, \#65} \\

  % \hphantom{aaaaaaaa} \texttt{swi 0} \\



  % \texttt{.section .data} \\
  % \texttt{message:} \\
  % \hphantom{aaaaaaaa} \texttt{.ascii "Hello, World!\textbackslash n"}

\end{frame}
% ##################





% ##################
% !!!!!!!!!!!!!!!!!!!!!!!!!!!!!!!!!!!!!!!!
% \jagiellonianendslide{Czy są jakieś pytania do tej części?}
% ##################










% ######################################
\section{Translatory, kompilatory i~interpretery}
% ######################################



% ##################
\begin{frame}
  \frametitle{Translatory}


  W~tym momencie już mam nadzieję wszyscy rozumiemy, czemu potrzebujemy
  kompilatorów. Zanim jednak przejdziemy dalej, potrzebujemy, choć to może
  być dość nudne, wprowadzić trochę pojęć i~terminologi.

  \textbf{Translator} to dowolny obiekt który tłumaczy tekst z~języka~A,
  na odpowiedni tekst w~języku~B. Wedle tej definicji człowiek tłumaczący
  z~polskiego na angielski też jest translatorem. To nie jest bug definicji,
  to jest feature.



  \begin{figure}

    \label{fig:Translator-02}


    \begin{tikzpicture}

      \node[diagram rectangle block] (Text in language A) at (0,0)
      {Tekst w~języku~A};

      \node[diagram block] (Translator) at (3.8,0) {Translator};

      \draw[thick diagram arrow] (Text in language A) -- (Translator);


      \node[diagram rectangle block] (Text in language B) at (7.6,0)
      {Tekst w~języku~B};

      \draw[thick diagram arrow] (Translator) -- (Text in language B);

    \end{tikzpicture}

    \caption{Błędny sposób uruchamiania gry „Hollow Knight”}


  \end{figure}


\end{frame}
% ##################










% ##################
\begin{frame}
  \frametitle{Krótka i~niezbyt poprawna historia problemu}


  Na~przestrzeni tysięcy lat ludzie tworzyli urządzenia, które dziś
  rozpoznajemy jako prekursorów współczesnych komputerów, wypada wymienić
  kilku z~nich, którzy działali przed rokiem 1900-nym.

  \textbf{2003 r.} Chris Lattner i~Vikram Adve tworzą system \textsc{llvm}.
  Możemy przyjąć, że~w~tym momencie zaczyna~się okres rozwoju technik
  kompilacji w~którym jesteśmy dziś (2024 rok).

\end{frame}
% ##################





% % ##################
% \begin{frame}
%   \frametitle{}




% \end{frame}
% % ##################





% ##################
\begin{frame}
  \frametitle{?????}


  \textsc{cisc}, ang. \textit{Complex Instruction Set Computing}

  \textsc{risc}, ang. \textit{Reduced Instruction Set Computing}


\end{frame}
% ##################





% % ##################
% \begin{frame}
%   \frametitle{}



% \end{frame}
% % ##################





% % ##################
% \begin{frame}
%   \frametitle{}




% \end{frame}
% % ##################





% % ##################
% \begin{frame}
%   \frametitle{}



% \end{frame}
% % ##################





% % ##################
% \begin{frame}
%   \frametitle{}




% \end{frame}
% % ##################





% % ##################
% \begin{frame}
%   \frametitle{}




% \end{frame}
% % ##################





% % ##################
% \begin{frame}
%   \frametitle{}




% \end{frame}
% % ##################










% % ######################################
% \section{????}
% % ######################################
























% ######################################
\section{Informacje dodatkowe}
% ######################################









% ##################
\begin{frame}
  \frametitle{?????}




\end{frame}
% ##################











% % ##################
% \begin{frame}
%   \frametitle{Rendering without perspective}




% \end{frame}
% % ##################





% % ##################
% \begin{frame}
%   \frametitle{Rendering with perspective}




% \end{frame}
% % ##################





% % ##################
% \begin{frame}
%   \frametitle{Basic concepts of computer graphics}






% \end{frame}
% % ##################





% % ##################
% \begin{frame}
%   \frametitle{Data structures}




% \end{frame}
% % ##################





% % ##################
% \begin{frame}
%   \frametitle{From vertices to fragments}



% \end{frame}
% % ##################











% % ##################
% \begin{frame}
%   \frametitle{Vertex attributes}




% \end{frame}
% % ##################






% % ##################
% \begin{frame}
%   \frametitle{Basic concepts of computer graphics}




% \end{frame}
% % ##################





% % ##################
% \begin{frame}
%   \frametitle{Primitives in the OpenGL library}



% \end{frame}
% % ##################





% % ##################
% \begin{frame}
%   \frametitle{Primitives in the OpenGL library}




% \end{frame}
% % ##################





% % ##################
% \begin{frame}
%   \frametitle{Primitives in the OpenGL library}



% \end{frame}
% % ##################





% % ##################
% \begin{frame}
%   \frametitle{Primitives in the OpenGL library}




% \end{frame}
% % ##################





% % ##################
% \begin{frame}
%   \frametitle{Primitives in the OpenGL library (deprecated since ver.~3.1)}




% \end{frame}
% % ##################





% % ##################
% \begin{frame}
%   \frametitle{Texturing and blending}




% \end{frame}
% % ##################





% % ##################
% \begin{frame}
%   \frametitle{Texturing and blending}




% \end{frame}
% % ##################





% % ##################
% \begin{frame}
%   \frametitle{Communication between CPU and GPU}




% \end{frame}
% % ##################





% % ##################
% \begin{frame}
%   \frametitle{Data stored in the video memory}




% \end{frame}
% % ##################





% % ##################
% \begin{frame}
%   \frametitle{Data stored in the video memory}




% \end{frame}
% % ##################





% % ##################
% \begin{frame}
%   \frametitle{Graphics pipeline (rendering pipeline)}




% \end{frame}
% % ##################





% % ##################
% \begin{frame}
%   \frametitle{Basic concepts of computer graphics}




% \end{frame}
% % ##################





% % ##################
% \begin{frame}
%   \frametitle{Basic concepts of computer graphics}





% \end{frame}
% % ##################





% % ##################
% \begin{frame}
%   \frametitle{Basic concepts of computer graphics}





% \end{frame}
% % ##################





% % ##################
% \begin{frame}
%   \frametitle{Vertex transformations (OpenGL nomenclature)}




% \end{frame}
% % ##################





% % ##################
% \begin{frame}[label=Przestrzen-uklad-wspolrzednych-1]
%   \frametitle{Spaces --~coordinate systems}





% \end{frame}
% % ##################





% % ##################
% \begin{frame}
%   \frametitle{Spaces --~coordinate systems}





% \end{frame}
% % ##################





% % ##################
% \begin{frame}
%   \frametitle{Rasterization and fragment operations}





% \end{frame}
% % ##################





% % ##################
% \begin{frame}
%   \frametitle{}




% \end{frame}
% % ##################





% % ##################
% \begin{frame}
%   \frametitle{Fragment operations}






% \end{frame}
% % ##################





% % ##################
% \begin{frame}
%   \frametitle{Basic concepts of computer graphics}





% \end{frame}
% % ##################





% % ##################
% \begin{frame}
%   \frametitle{Links}


%   [1] \colorhref{www.blender.org}{www.blender.org}

% \end{frame}
% % ##################










% % ######################################
% \appendix
% % ######################################





% % ##################
% \GeometryThreeDTwoSpecialEndingSlidesEN{Questions? Thank you for your attention.}
% % ##################



% % % ##################
% % \jagiellonianendslide{Dziękuję za~uwagę.}
% % % ##################










% % ######################################
% \SectionSlideWithPicture{Terminological notes}
% % ######################################



% ##################
% \begin{frame}[label=]
%   \frametitle{}




% \end{frame}
% ##################





% ##################
% \begin{frame}[label=]
%   \frametitle{}




% \end{frame}
% ##################





% ##################
% \begin{frame}[label=Uwagi-pojecie-przestrzeni-Przestrzen-afiniczna-i-wektorowa-3]
%   \frametitle{The concept of space. Space and coordinate system}



% \end{frame}
% ##################





% ##################
% \begin{frame}[label=]
%   \frametitle{Coordinate systems in computer graphics}



% \end{frame}
% ##################





% ##################
% \begin{frame}
%   \frametitle{Coordinate systems in computer graphics}



% \end{frame}
% ##################





% ##################
% \begin{frame}
%   \frametitle{Coordinate systems in computer graphics}



% \end{frame}
% ##################










% ####################################################################

% End of the document
\end{document}
