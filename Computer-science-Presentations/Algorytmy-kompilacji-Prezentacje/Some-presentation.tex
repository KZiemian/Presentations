% ---------------------------------------------------------------------
% Basic configuration of Beamera class and Jagiellonian theme
% ---------------------------------------------------------------------
\RequirePackage[l2tabu, orthodox]{nag}



\ifx\PresentationStyle\notset
  \def\PresentationStyle{dark}
\fi



\documentclass[10pt,t]{beamer}
\mode<presentation>
\usetheme[style=\PresentationStyle,JUlogotitle=no]{jagiellonian}




% ---------------------------------------
% Procesing configuration files of Jagiellonian theme loceted in directory
% "preambule".
% ---------------------------------------
% Configuration for polish language
% Need description
\usepackage[polish]{babel}
% Need description
\usepackage[MeX]{polski}



% ------------------------------
% Better support of polish chars in technical parts of PDF
% ------------------------------
\hypersetup{pdfencoding=auto,psdextra}

% Package "textpos" give as enviroment "textblock" which is very usefull in
% arranging text on slides.

% This is standard configuration of "textpos"
\usepackage[overlay,absolute]{textpos}

% If you need to see bounds of "textblock's" comment line above and uncomment
% one below.

% Caution! When showboxes option is on significant ammunt of space is add
% to the top of textblock and as such, everyting put in them gone down.
% We need to check how to remove this bug.

% \usepackage[showboxes,overlay,absolute]{textpos}



% Setting scale length for package "textpos"
\setlength{\TPHorizModule}{10mm}
\setlength{\TPVertModule}{\TPHorizModule}


% ---------------------------------------
% TikZ
% ---------------------------------------
% Importing TikZ libraries
\usetikzlibrary{arrows.meta}
\usetikzlibrary{positioning}





% % Configuration package "bm" that need for making bold symbols
% \newcommand{\bmmax}{0}
% \newcommand{\hmmax}{0}
% \usepackage{bm}




% ---------------------------------------
% Packages for scientific texts
% ---------------------------------------
% \let\lll\undefined  % Sometimes you must use this line to allow
% "amsmath" package to works with packages with packages for polish
% languge imported
% /preambul/LanguageSettings/JagiellonianPolishLanguageSettings.tex.
% This comments (probably) removes polish letter Ł.
\usepackage{amsmath}  % Packages from American Mathematical Society (AMS)
\usepackage{amssymb}
\usepackage{amscd}
\usepackage{amsthm}
\usepackage{siunitx}  % Package for typsetting SI units.
\usepackage{upgreek}  % Better looking greek letters.
% Example of using upgreek: pi = \uppi


\usepackage{calrsfs}  % Zmienia czcionkę kaligraficzną w \mathcal
% na ładniejszą. Może w innych miejscach robi to samo, ale o tym nic
% nie wiem.










% ---------------------------------------
% Packages written for lectures "Geometria 3D dla twórców gier wideo"
% ---------------------------------------
% \usepackage{./ProgramowanieSymulacjiFizykiPaczki/ProgramowanieSymulacjiFizyki}
% \usepackage{./ProgramowanieSymulacjiFizykiPaczki/ProgramowanieSymulacjiFizykiIndeksy}
% \usepackage{./ProgramowanieSymulacjiFizykiPaczki/ProgramowanieSymulacjiFizykiTikZStyle}





% !!!!!!!!!!!!!!!!!!!!!!!!!!!!!!
% !!!!!!!!!!!!!!!!!!!!!!!!!!!!!!
% EVIL STUFF
\if\JUlogotitle1
\edef\LogoJUPath{LogoJU_\JUlogoLang/LogoJU_\JUlogoShape_\JUlogoColor.pdf}
\titlegraphic{\hfill\includegraphics[scale=0.22]
{./JagiellonianPictures/\LogoJUPath}}
\fi
% ---------------------------------------
% Commands for handling colors
% ---------------------------------------


% Command for setting normal text color for some text in math modestyle
% Text color depend on used style of Jagiellonian

% Beamer version of command
\newcommand{\TextWithNormalTextColor}[1]{%
  {\color{jNormalTextFGColor}
    \setbeamercolor{math text}{fg=jNormalTextFGColor} {#1}}
}

% Article and similar classes version of command
% \newcommand{\TextWithNormalTextColor}[1]{%
%   {\color{jNormalTextsFGColor} {#1}}
% }



% Beamer version of command
\newcommand{\NormalTextInMathMode}[1]{%
  {\color{jNormalTextFGColor}
    \setbeamercolor{math text}{fg=jNormalTextFGColor} \text{#1}}
}


% Article and similar classes version of command
% \newcommand{\NormalTextInMathMode}[1]{%
%   {\color{jNormalTextsFGColor} \text{#1}}
% }




% Command that sets color of some mathematical text to the same color
% that has normal text in header (?)

% Beamer version of the command
\newcommand{\MathTextFrametitleFGColor}[1]{%
  {\color{jFrametitleFGColor}
    \setbeamercolor{math text}{fg=jFrametitleFGColor} #1}
}

% Article and similar classes version of the command
% \newcommand{\MathTextWhiteColor}[1]{{\color{jFrametitleFGColor} #1}}





% Command for setting color of alert text for some text in math modestyle

% Beamer version of the command
\newcommand{\MathTextAlertColor}[1]{%
  {\color{jOrange} \setbeamercolor{math text}{fg=jOrange} #1}
}

% Article and similar classes version of the command
% \newcommand{\MathTextAlertColor}[1]{{\color{jOrange} #1}}





% Command that allow you to sets chosen color as the color of some text into
% math mode. Due to some nuances in the way that Beamer handle colors
% it not work in all cases. We hope that in the future we will improve it.

% Beamer version of the command
\newcommand{\SetMathTextsColor}[2]{%
  {\color{#1} \setbeamercolor{math text}{fg=#1} #2}
}


% Article and similar classes version of the command
% \newcommand{\SetMathTextColor}[2]{{\color{#1} #2}}










% ---------------------------------------
% Commands for setting background pictures for some slides
% ---------------------------------------
\newcommand{\TitleBackgroundPicture}
{./PresentationPictures/CommonPictures/Cute_dragon_BG_dark.png}
\newcommand{\SectionBackgroundPicture}
{./PresentationPictures/CommonPictures/Cute_dragon_small_BG_light.png}



\newcommand{\TitleSlideWithPicture}{
  \begingroup

  \usebackgroundtemplate{ % \hspace*{-11.5em}
    \includegraphics[height=\paperheight]{\TitleBackgroundPicture}}

  \maketitle

  \endgroup
}





\newcommand{\SectionSlideWithPicture}[1]{%
  \begingroup

  \usebackgroundtemplate{ % \hspace*{-11.5em}
    \includegraphics[height=\paperheight]{\SectionBackgroundPicture}}

  \setbeamercolor{titlelike}{fg=normal text.fg}

  \section{#1}

  \endgroup
}





\newcommand{\EndingSlide}[1]{%
  \begin{frame}[standout]

    \begingroup

    \color{jFrametitleFGColor}

    #1

    \endgroup

  \end{frame}
}










% ------------------------------
% Importing packages, libraries and setting their configuration.
% ------------------------------





% ------------------------------
% Special configuration for this particular presentation.
% ------------------------------










% ---------------------------------------------------------------------
\title{Geometry 3D}
\subtitle{for Designers of Video Games}

\author{Paweł Węgrzyn, Ph.D., Assoc. Prof. \\[1em]
  Institute of Applied Computer Science \\
  Faculty of Physics, Astronomy \\
  and Applied Computer Science \\
  Jagiellonian University, Cracow, Poland}


\date{}
% ---------------------------------------------------------------------










% ####################################################################
% Beginning of the document.
\begin{document}
% ####################################################################





% ######################################
% Text is adjusted to the left and words are broken at the end of the line.
% Number of chars: 62k+.
\RaggedRight
% ######################################





% ######################################
\maketitle
% ######################################





% % ##################
% \begin{frame}
%   \frametitle{Table of contents}


%   \tableofcontents

% \end{frame}
% % ##################










% % ######################################
% \section{Getting ECTS~points. Requirements}
% % ######################################



% ##################
\begin{frame}
  \frametitle{Getting ECTS~points for the Course}




\end{frame}
% ##################





% % ##################
% \begin{frame}
%   \frametitle{Literature}




% \end{frame}
% % ##################





% % ##################
% \begin{frame}
%   \frametitle{Literature}




% \end{frame}
% % ##################





% % ##################
% \begin{frame}
%   \frametitle{\alert{Important information}}




% \end{frame}
% % ##################





% % ##################
% \begin{frame}
%   \frametitle{Kind request to listeners}




% \end{frame}
% % ##################










% % ######################################
% \SectionSlideWithPicture{Graphics pipeline (OpenGL)}
% % ######################################



% % ##################
% \begin{frame}
%   \frametitle{Graphics pipeline (rendering pipeline)}



% \end{frame}
% % ##################





% % ##################
% \begin{frame}
%   \frametitle{About rendering pipeline on
%     \href{https://www.khronos.org/}{https://www.khronos.org/}}



% \end{frame}
% % ##################





% % ##################
% \begin{frame}
%   \frametitle{Graphics pipeline (rendering pipeline)}




% \end{frame}
% % ##################










% % ######################################
% \SectionSlideWithPicture{Basic concepts of~computer~graphics}
% % ######################################



% % ##################
% \begin{frame}
%   \frametitle{Basic concepts of computer graphics}




% \end{frame}
% % ##################





% % ##################
% \begin{frame}
%   \frametitle{A sample scene}



% \end{frame}
% % ##################





% % ##################
% \begin{frame}
%   \frametitle{Rendering without perspective}



% \end{frame}
% % ##################





% % ##################
% \begin{frame}
%   \frametitle{Rendering with perspective}



% \end{frame}
% % ##################





% % ##################
% \begin{frame}
%   \frametitle{Another sample scene}




% \end{frame}
% % ##################





% % ##################
% \begin{frame}
%   \frametitle{Rendering without perspective}




% \end{frame}
% % ##################





% % ##################
% \begin{frame}
%   \frametitle{Rendering with perspective}



% \end{frame}
% % ##################





% % ##################
% \begin{frame}
%   \frametitle{Basic concepts of computer graphics}




% \end{frame}
% % ##################





% % ##################
% \begin{frame}
%   \frametitle{Data structures}




% \end{frame}
% % ##################





% % ##################
% \begin{frame}
%   \frametitle{From vertices to fragments}



% \end{frame}
% % ##################






% % ##################
% \begin{frame}
%   \frametitle{Basic concepts of computer graphics}




% \end{frame}
% % ##################





% % ##################
% \begin{frame}
%   \frametitle{Each model is based on  a set of vertices}




% \end{frame}
% % ##################





% % ##################
% \begin{frame}
%   \frametitle{Each model is based on  a set of vertices}



% \end{frame}
% % ##################





% % ##################
% \begin{frame}
%   \frametitle{Vertex attributes}




% \end{frame}
% % ##################





% % ##################
% \begin{frame}
%   \frametitle{Vertex attributes}



% \end{frame}
% % ##################






% % ##################
% \begin{frame}
%   \frametitle{Basic concepts of computer graphics}



% \end{frame}
% % ##################





% % ##################
% \begin{frame}
%   \frametitle{Primitives in the OpenGL library}




% \end{frame}
% % ##################





% % ##################
% \begin{frame}
%   \frametitle{Primitives in the OpenGL library}



% \end{frame}
% % ##################





% % ##################
% \begin{frame}
%   \frametitle{Primitives in the OpenGL library}



% \end{frame}
% % ##################





% % ##################
% \begin{frame}
%   \frametitle{Primitives in the OpenGL library}



% \end{frame}
% % ##################





% % ##################
% \begin{frame}
%   \frametitle{Primitives in the OpenGL library (deprecated since ver.~3.1)}



% \end{frame}
% % ##################





% % ##################
% \begin{frame}
%   \frametitle{Texturing and blending}



% \end{frame}
% % ##################





% % ##################
% \begin{frame}
%   \frametitle{Texturing and blending}



% \end{frame}
% % ##################





% % ##################
% \begin{frame}
%   \frametitle{Communication between CPU and GPU}



% \end{frame}
% % ##################





% % ##################
% \begin{frame}
%   \frametitle{Data stored in the video memory}




% \end{frame}
% % ##################





% % ##################
% \begin{frame}
%   \frametitle{Data stored in the video memory}




% \end{frame}
% % ##################





% % ##################
% \begin{frame}
%   \frametitle{Graphics pipeline (rendering pipeline)}




% \end{frame}
% % ##################





% % ##################
% \begin{frame}
%   \frametitle{Basic concepts of computer graphics}




% \end{frame}
% % ##################





% % ##################
% \begin{frame}
%   \frametitle{Basic concepts of computer graphics}





% \end{frame}
% % ##################





% % ##################
% \begin{frame}
%   \frametitle{Basic concepts of computer graphics}





% \end{frame}
% % ##################





% % ##################
% \begin{frame}
%   \frametitle{Vertex transformations (OpenGL nomenclature)}




% \end{frame}
% % ##################





% % ##################
% \begin{frame}[label=Przestrzen-uklad-wspolrzednych-1]
%   \frametitle{Spaces --~coordinate systems}





% \end{frame}
% % ##################





% % ##################
% \begin{frame}
%   \frametitle{Spaces --~coordinate systems}





% \end{frame}
% % ##################





% % ##################
% \begin{frame}
%   \frametitle{Rasterization and fragment operations}





% \end{frame}
% % ##################





% % ##################
% \begin{frame}
%   \frametitle{Rasterization and fragment operations}




% \end{frame}
% % ##################





% % ##################
% \begin{frame}
%   \frametitle{Fragment operations}






% \end{frame}
% % ##################





% % ##################
% \begin{frame}
%   \frametitle{Basic concepts of computer graphics}





% \end{frame}
% % ##################





% % ##################
% \begin{frame}
%   \frametitle{Links}




% \end{frame}
% % ##################










% % ######################################
% \appendix
% % ######################################





% % ##################
% \GeometryThreeDTwoSpecialEndingSlidesEN{Questions? Thank you for your attention.}
% % ##################



% % % ##################
% % \jagiellonianendslide{Dziękuję za~uwagę.}
% % % ##################










% % ######################################
% \SectionSlideWithPicture{Terminological notes}
% % ######################################



% % ##################
% \begin{frame}[label=Uwagi-pojecie-przestrzeni-Przestrzen-afiniczna-i-wektorowa-1]
%   \frametitle{The concept of space. Affine and vector space}




% \end{frame}
% % ##################





% ##################
% \begin{frame}[label=Uwagi-pojecie-przestrzeni-Przestrzen-afiniczna-i-wektorowa-2]
%   \frametitle{The concept of space. Space and coordinate system}




% \end{frame}
% ##################





% ##################
% \begin{frame}[label=Uwagi-pojecie-przestrzeni-Przestrzen-afiniczna-i-wektorowa-3]
%   \frametitle{The concept of space. Space and coordinate system}




% \end{frame}
% ##################





% ##################
% \begin{frame}[label=Uwagi-uklady-wspolrzednych-w-przestrzeni-Rodzaje-ukladow-1]
%   \frametitle{Coordinate systems in computer graphics}




% \end{frame}
% ##################





% ##################
% \begin{frame}[label=Uwagi-uklady-wspolrzednych-w-przestrzeni-Rodzaje-ukladow-2]
%   \frametitle{Coordinate systems in computer graphics}



% \end{frame}
% ##################





% #################
% \begin{frame}[label=Uwagi-uklady-wspolrzednych-w-przestrzeni-Rodzaje-ukladow-3]
%   \frametitle{Coordinate systems in computer graphics}




% \end{frame}
% ##################










% ####################################################################

% End of the document
\end{document}
