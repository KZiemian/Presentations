% ---------------------------------------------------------------------
% Podstawowe ustawienia Beamera i używane pakiety
% ---------------------------------------------------------------------
\RequirePackage[l2tabu, orthodox]{nag}  % Wykrywa przestarzałe i niewłaściwe
% sposoby używania LaTeXa. Więcej jest w l2tabu English version.


\ifx\StylWykladu\nieokreslony
\def\StylWykladu{dark}
\fi


\documentclass[10pt,t]{beamer}  % Klasa dokumentu
\mode<presentation>  % Rodzaj tworzonych slajdów Beamera
\usetheme[style=\StylWykladu,titlefontcolor=light]{jagiellonian}  % Temat
% graficzny: „light” -- wersja jasna, „dark” -- wersja ciemna



% General configuration of jagiellonian
\setbeamertemplate{section in toc}[sections numbered]  % Numeracja
% pozycji w spisie treści
% Wybór fontu
\usefonttheme{serif}
\usepackage{mathpazo}  % Nowoczesna paczka do Palatino



\usepackage[polish]{babel}  % Tłumaczy na polski automatyczne teksty LaTeXa
% i pomaga z typografią.
\usepackage[MeX]{polski}  % Polonizacja LaTeXa, bez niej będzie pracował
% w języku angielskim.
\usepackage[utf8]{inputenc}  % Włączenie kodowania UTF-8, co daje dostęp
% do polskich znaków.
\usepackage[T1]{fontenc}  % Potrzebne do używania fontów Latin Modern.



% ------------------------------
% Podstawowe paczki (niezwiązane z ustawieniami języka)
% ------------------------------
% \usepackage{mathfix}   %
\usepackage{microtype}  % Twierdzi, że poprawi rozmiar odstępów w tekście.
\usepackage{graphicx}  % Wprowadza bardzo potrzebne komendy do wstawiania
% grafiki.



% ------------------------------
% Lepsze wsparcie polskich znaków w takich miejscach jak konspekty PDFa
% ------------------------------
\hypersetup{pdfencoding=auto,psdextra}

% ------------------------------
% Paczki potrzebne w danym projekcie
% ------------------------------
% \usepackage{tabu}  % Lepsze tabele
\newcommand{\bmmax}{0}  % Górne ograniczenie na ilość "alfabetów
% matematycznych", które może wykorzystać paczka "bm" do tworzenia
% "Bold symbols". Zapobiega przekroczenia standardowej maksymalnej
% ilości jakie dopuszcza TeX (i pdfLaTeX), wynoszącej 16.
\newcommand{\hmmax}{0}
\usepackage{bm}  % Bold Math symbols, obecnie preferowany sposób.




% ------------------------------
% Pakiety do tekstów z nauk przyrodniczych
% ------------------------------
% \let\lll\undefined  % Amsmath gryzie się z pakietami do języka
% % polskiego, bo oba definiują komendę \lll. Aby rozwiązać ten problem
% % oddefiniowuję tę komendę, ale może tym samym pozbywam się dużego Ł.
\usepackage{amsmath}  % Podstawowe wsparcie od American
% Mathematical Society (w skrócie AMS)
% \usepackage{amsfonts, amssymb, amscd, amsthm}  % Dalsze wsparcie od AMS
\usepackage{amssymb, amscd, amsthm}  % Dalsze wsparcie od AMS
\usepackage{siunitx}  % Do prostszego pisania jednostek fizycznych
\usepackage{upgreek}  % Ładniejsze greckie litery
% Przykładowa składnia: pi = \uppi
% \usepackage{icomma} % Przecinek w trybie matematyczny ma właściwy odstęp



% ------------------------------
% TikZ
% ------------------------------
% \usepackage{tikz}  % Potężny pakiet PGF/TikZ.
% Włączenie konkretnych bibliotek pakietu TikZ
% \usetikzlibrary{arrows.meta}
% \usetikzlibrary{angles}
% \usetikzlibrary{3d}
% \usetikzlibrary{positioning}
% \usetikzlibrary{intersections}
% \usetikzlibrary{decorations.markings}
% \usetikzlibrary{calc}





% ------------------------------
% Komendy i ustawienia do wykładów ,,''.
% ------------------------------
\newcommand{\titlebackground}{./pictures/Clerk_Maxwell.jpg}
% Dla trybu Dark
% \newcommand{\sectionbackground}{./pictures/SzablonTytBluedark.png}
% Dla trybu Light
\newcommand{\sectionbackground}{./pictures/Newton.jpg}



\newcommand{\jagielloniantitlepage}{
  \begingroup

  \setbeamercolor{background canvas}{bg=JagiellonianBlue}

  \maketitle

  \endgroup

  \addtocounter{framenumber}{1}
}

\newcommand{\jagielloniantitlepagewithpicture}{
  \addtocounter{framenumber}{1}

  \begingroup
  \usebackgroundtemplate{ % \hspace*{-11.5em}
    \includegraphics[height=\paperheight]{\titlebackground}}

  \maketitle

  \endgroup
}


\newcommand{\jagielloniansectionwithpicture}[1]{
  \begingroup
  \usebackgroundtemplate{ % \hspace*{-11.5em}
    \includegraphics[height=\paperheight]{\sectionbackground}}

  \setbeamercolor{titlelike}{fg=normal text.fg}

  \section{#1}

  \endgroup
}



\newcommand{\jagiellonianendslide}[1]{
  \begin{frame}[standout]


    {\color{jStrongWhite} #1}

    % \begin{textblock}{5.5}(0.65,3)
    %   \includegraphics[scale=0.45]{./pictures/Fundusze_Europejskie.png}
    % \end{textblock}

    % \begin{textblock}{3}(3.65,3.9)
    %   \includegraphics{./pictures/CC-by-sa.png}
    % \end{textblock}

  \end{frame}
}





% ############################
% Beamer i LaTeX

% Zmiana koloru fontu matematycznego
\newcommand{\beamermathcolor}[2]{{\color{#1}\setbeamercolor{math text}
  {fg=#1} #2}}

% Zmiana koloru fontu na foreground BEAMER-koloru ,,normal text''
\newcommand{\normaltextcolor}[1]{{\color{normal text.fg}{#1}}}

% Zmiana koloru fontu matematycznego na jMathTextForegorundWhite
\newcommand{\beamermathcolorwhite}[1]{{\color{jStrongWhite}
    \setbeamercolor{math text}{fg=jStrongWhite} #1}}

\newcommand{\alertmath}[1]{{\color{jLightBrown}
  \setbeamercolor{math text}{fg=jLightBrown} #1}}


\newcommand{\colorhref}[2]{\href{#1}{\color{jLightBrown} #2}}





% ------------------------------
% Pakiety napisane do wykładów z ,,???'''.
% ------------------------------
% \usepackage{./SymulacjeFizykiPaczki/}
% \usepackage{./SymulacjeFizykiPaczki/SymulacjeFizyki}






% ------------------------------
% Paczki, biblioteki i ich ustawienia dla tego pliku
% ------------------------------
% \usepackage{./Geometria3DPaczki/Geometria3DTikZPics}

\usepackage{tikz}
\usetikzlibrary{positioning}
\usetikzlibrary{trees}
\usetikzlibrary{graphs}
\usetikzlibrary{automata}
\usetikzlibrary{arrows.meta}
% \usetikzlibrary{decorations.markings,decorations.pathmorphing}
% \usetikzlibrary{decorations.markings,decorations.pathmorphing}







% ------------------------------
% Ustawienie dla tego konkretnego pliku
% ------------------------------










% ---------------------------------------------------------------------
\title{Rysowanie grafów za pomocą paczki Ti\emph{k}Z}

\author{Kamil Ziemian}

\date{}

% \titlegraphic{\hfill\includegraphics[height=2cm]
%   {./pictures/LogoUJ_monochromatyczne_biale.png}}
% ---------------------------------------------------------------------










% ####################################################################
% Początek dokumentu
\begin{document}
% ####################################################################



% ######################################
\maketitle
% ######################################



% % ##################
% \begin{frame}
%   \frametitle{Spis treści}


%   %   \Large
%   \tableofcontents

% \end{frame}
% % ##################










% % ######################################
% \jagielloniansectionwithpicture{Empiryczne modele oświetlenia}
% % ######################################



% ##################
\begin{frame}
  \frametitle{Rysowanie grafów w~Ti\emph{k}Zie}


  Elementy grafu
  \begin{itemize}
  \item wierzchołki $\to$ Ti\emph{k}Zowe węzły (ang. \emph{nodes})

  \item krawędzie $\to$ Ti\emph{k}Zowe linie

  \end{itemize}

\end{frame}
% ##################




% ##################
\begin{frame}
  \frametitle{Trochę perspektywy}


  Licząc oszczędnie, najnowszy manual Ti\emph{k}Za ma około 1 300
  stron, z~czego
  \begin{itemize}
  \item 38 stron omawia są węzły i~stowarzyszone z~nimi krawędzie \\
    (ang. \emph{edges}),

  \item 180 stron omawia jest rysowaniu grafów.

  \end{itemize}

  Jest to ogromna ilość materiału, wielokrotnie przekraczająca czas
  naszego spotkania. To co tutaj prezentuje jest wysoce subiektywny
  wyborem kilku zagadnień z~tego materiału.

  Nie jestem zadowolony z~obecnego stanu prelekcji, sposób rozłożenia
  akcentów mógłby być lepiej zbalansowany.

\end{frame}
% ##################





% ##################
\begin{frame}
  \frametitle{Styl pisania kodu}


  Ti\emph{k}Z daje nam dużą swobodę w sposobie zapisu kodu źródłowego.
  Prezentowana na tym spotkaniu wersja, nie jest w żadnym stopniu
  kanonicze i~nie należy~się do niego przesadnie przywiązywać.

  Jest to styl, który mojej osobie, przy obecnym stanie mojej wiedzy
  i~umiejętności wydaje się najbardziej przejrzysty i~wygodny. Jest to
  jednak odczucie wysoce subiektywne, które wciąż ewoluuje.

\end{frame}
% ##################





% ##################
\begin{frame}
  \frametitle{Węzeły, nieformalna definicja}


  \textbf{Węzeł} (ang. \emph{node}) to fragment rysunku, stworzony po
  to, by coś zawierać. Może jednak pozostać pusty.

  \begin{figure}

    \centering

    \begin{tikzpicture}
      \node at (0,0) {Tutaj};

      \node[rotate=10] at (1,2)
      {$\displaystyle m \frac{ d^{ 2 } x( t ) }{ d t^{ 2 } } = F$};

      \node[shape=rectangle,draw] at (2,0) {Tam};

      \node[shape=rectangle,draw] at (2,-1) {};
    \end{tikzpicture}

    \caption{Węzły}

  \end{figure}

\end{frame}
% ##################





% ##################
\begin{frame}
  \frametitle{Węzły a~picsy}

  \textbf{Pics} (od ang. \emph{small PICture}) to mały rysunek
  „wklejany” do większego rysunku (nieformalna definicja).

  Kiedy użyć węzła, a~kiedy picsa?

  Jeśli będziemy wielokrotnie kopiować i~wklejać w nowe miejsca jakiś
  pomniejszy rysunek niezawierający tekstu, wtedy powinniśmy użyć
  picsa. Przykładowo, możemy chcieć umieścić w nim rysunek owcy,
  następnie wkleić 10 owiec do rysunku łąki.

  Jeśli chcemy stworzyć fragment rysunku zawierający tekst, lub gdy
  ważne są wzajemne relacje między takimi fragmentami, to powinniśmy
  użyć węzłów.

  \textbf{Ważne.} Istnieją dobre powody, by złamać te zasady. Jeżeli
  nie jesteś pewien, czy w danym przypadku można je zignorować, zwykle
  lepiej jest się ich trzymać.

\end{frame}
% ##################





% ##################
\begin{frame}
  \frametitle{Tworzenie węzłów}


  \texttt{\textbackslash node at (0,0) \{A\}};

  \begin{figure}

    \centering

    % (0,0), (1,2), (-1,1)

    \begin{tikzpicture}
      \node[draw] at (0,0) {A};

      \node[shape=circle,draw=red,line width=1pt,fill=cyan] at (1,2) {B};

      \node[draw] at (-1,1) {C};
    \end{tikzpicture}

    \caption{Tworzenie węzłów}

  \end{figure}

\end{frame}
% ##################





% ##################
\begin{frame}
  \frametitle{Style}


  \textbf{Styl Ti\emph{k}Z} to ciąg komend wykonywany na danym
  obiekcie Ti\emph{k}Za.

  \begin{figure}

    \centering

    \begin{tikzpicture}[grube kolo/.style={shape=circle,very thick,
        draw=blue,fill=white}]


      \node[grube kolo] at (0,0) {A};

      \node[grube kolo] at (1,2) {B};

      \node[grube kolo] at (-1,1) {C};
    \end{tikzpicture}

    \caption{Style}

  \end{figure}

\end{frame}
% ##################





% ##################
\begin{frame}
  \frametitle{Łączenie węzłów}


  \begin{figure}

    \centering

    % (0,0), (1,2), (-1,1)

    \begin{tikzpicture}[every node/.style={shape=circle,draw}]
      \node (A) at (0,0) {A};

      \node (B) at (1,2) {B};

      \node (C) at (-1,1) {C};


      \draw (A) -- (B);

      \draw[dashed,color=red] (A) -- (C);



      \node[shape=rectangle,draw] (D) at (1.4,-0.5) {D};

      \draw[->] (A) -- (D);
    \end{tikzpicture}

    \caption{Łączenie węzłów}

  \end{figure}

\end{frame}
% ##################






% ##################
\begin{frame}
  \frametitle{Krawędzie}


  Można uważać, że \textbf{krawędź} (ang. \emph{edge}) to linia
  przypisana do węzła. Ale nie jest to całkiem poprawne stwierdzenie.

  \begin{figure}

    \centering

    \begin{tikzpicture}[every node/.style={shape=circle,draw},
      style 1/.style={shape=rectangle}]
      \node (A) at (0,0) {A};

      \node (B) at (1,2) {B};

      \node (C) at (-1,1) {C};

      \draw[-{Stealth[scale=2]}] (C) -- (B);


      \node[shape=rectangle,draw] (D) at (1.4,-0.5) {D}
      edge [{Straight Barb[scale=2,length=3pt,width=4pt]}-] (A)
      edge [<->] node[style 1,auto,swap,scale=0.8] {2} (B)
      edge [-{Straight Barb[scale=2]},bend right=25]
      node[style 1,auto,swap,scale=0.8] {$1$} (C);
    \end{tikzpicture}

    % \node[shape=rectangle,draw] (D) at (1.4,-0.5) {D}
    % edge [<-] (A)
    % edge [->,bend right=25] (C)
    % edge [<->] (B);

    \caption{Krawędzie}

  \end{figure}

\end{frame}
% ##################





% ##################
\begin{frame}
  \frametitle{Położenie względne}


  \begin{figure}

    \centering

    \begin{tikzpicture}[node distance=0.5cm,
      every node/.style={shape=circle,draw}]


      \node (A) {A};

      % =of
      \node[right=of A] (B) {B};

      \node[left=of A] (C) {C};

      \node[above=of A] (D) {D};

      \node[below=of A] (E) {E};

      \node[below right of= B] (F) {F};
    \end{tikzpicture}

    \caption{Położenie względne}

  \end{figure}

\end{frame}
% ##################





% ##################
\begin{frame}
  \frametitle{Grafy drzewowe}


  \texttt{\textbackslash node \{Zawartość węzła\} child \{<kod węzła>\};}

  Biblioteka \texttt{trees}.


  \begin{figure}

    \centering

    \begin{tikzpicture}[line width=1pt,
      every node/.style={shape=circle,draw,very thick},
      level 1/.style={sibling distance=2.5cm},
      level 2/.style={sibling distance=1.2cm}]


      \node {A}
      child { node {B1}
        child { node {C1} }
        child { node {C2} }
      }
      child { node {B2}
        child { node {C3} }
        child { node {C4} }
      };
    \end{tikzpicture}

    %   grow=right

    %   \usetikzlibrary{trees}

    %   level 1/.style={sibling distance=2cm},
    %   level 2/.style={sibling distance=1cm},

    %   grow cyclic,
    %   level 1/.style={sibling distance=2cm,sibling angle=180},
    %   level 2/.style={sibling distance=1cm,sibling angle=90},

    \caption{Grafy drzewowe}

  \end{figure}

\end{frame}
% ##################





% ##################
\begin{frame}
  \frametitle{\texttt{\textbackslash graph}}


  Musimy dodać do preambuły linię: \\
  \texttt{\textbackslash usetikzlibrary\{graphs\}}.

  Komenda \texttt{\textbackslash graph} bierze na siebie zarządzanie
  węzłami i~krawędziami, pozwalając nam krótko zapisać wiele typów
  grafów.

  \textbackslash usetikzlibrary\{quotes\}


  \begin{figure}

    \centering

    \begin{tikzpicture}[every node/.style={shape=circle,draw}]
      % \graph { a -> {b, c} -> d };

      % \graph {
      % a -> b -> c;
      % d -> e -> f;
      % g -> f;
      % e -> b;
      % };

      % \graph { a/A1 -> b/Tu };

      % \graph { a ->[red] b --[line width=1pt] c };



      \node (A) at (0,-1) {A};

      \node (B) at (2,1) {B};

      \graph { (A) -> (B) };

      \graph { A -> B };
    \end{tikzpicture}

    \caption{\texttt{\textbackslash graph}}

  \end{figure}

\end{frame}
% ##################





% ##################
\begin{frame}
  \frametitle{Węzły w~macierzy}


  Kiedy potrzeba rozmieścić wierzchołki grafu w skomplikowany sposób,
  przydatne mogą okazać się ustawienie ich za pomocą składni
  macierzowej \LaTeX a.





  \begin{figure}

    \centering

    \begin{tikzpicture}[node loc/.style={shape=circle,draw}]


      \matrix[row sep=0.1cm,column sep=0.5cm, ampersand
      replacement=\amp] {
        % Pierwszy wiersz
        \amp \amp \node[node loc] (A) {A}; \amp \amp \\
        % Drugi wiersz
        \node[node loc] (B) {B}; \amp \amp \amp \node[node loc] {C}; \amp
        \node {F}; \\
        % Trzeci wiersz
        \amp \node[node loc] (D) {D}; \amp \node[node loc] {E}; \amp \amp \\
      };


      \path (A) edge [->] (B);

      \path (B) edge [<-] (D);

      \path (D) edge [<-] (E);
    \end{tikzpicture}

    \caption{Węzły w~macierzy}

  \end{figure}

\end{frame}
% ##################





% ##################
\begin{frame}
  \frametitle{\texttt{\textbackslash automata}}


  Biblioteka do rysowania automatów skończonych.

  \begin{figure}

    \centering

    \begin{tikzpicture}[shorten >=1pt,node distance=2cm,on grid,auto]
      \node[state,initial] (q_0) {$q_0$};

      \node[state] (q_1) [above right=of q_0] {$q_1$};

      \node[state] (q_2) [below right=of q_0] {$q_2$};

      \node[state,accepting](q_3) [below right=of q_1] {$q_3$};



      \path[->] (q_0) edge node {0} (q_1) edge node [swap] {1} (q_2)
      (q_1) edge node {1} (q_3) edge [loop above] node {0} () (q_2)
      edge node [swap] {0} (q_3) edge [loop below] node {1} ();
    \end{tikzpicture}

    \caption{\texttt{\textbackslash automata}}

  \end{figure}

\end{frame}
% ##################










% ######################################
\appendix
% ######################################





% ##################
\begin{frame}[standout]

  \begingroup

  \color{jStrongWhite}

  Pytania? Dziękuję za uwagę.

  \endgroup

\end{frame}
% ##################



% % ##################
% \jagiellonianendslide{Dziękuję za~uwagę.}
% % ##################










% % ######################################
% \jagielloniansectionwithpicture{Objaśnienia symboli}
% % ######################################










% ############################

% Koniec dokumentu
\end{document}
