\setbeamertemplate{section in toc}[sections numbered]  % Numeracja
% pozycji w spisie treści
% Wybór fontu
\usefonttheme{serif}
\usepackage{mathpazo}  % Nowoczesna paczka do Palatino



\usepackage[polish]{babel}  % Tłumaczy na polski automatyczne teksty LaTeXa
% i pomaga z typografią.
\usepackage[MeX]{polski}  % Polonizacja LaTeXa, bez niej będzie pracował
% w języku angielskim.
\usepackage[utf8]{inputenc}  % Włączenie kodowania UTF-8, co daje dostęp
% do polskich znaków.
\usepackage[T1]{fontenc}  % Potrzebne do używania fontów Latin Modern.



% ------------------------------
% Podstawowe paczki (niezwiązane z ustawieniami języka)
% ------------------------------
% \usepackage{mathfix}   %
\usepackage{microtype}  % Twierdzi, że poprawi rozmiar odstępów w tekście.
\usepackage{graphicx}  % Wprowadza bardzo potrzebne komendy do wstawiania
% grafiki.



% ------------------------------
% Lepsze wsparcie polskich znaków w takich miejscach jak konspekty PDFa
% ------------------------------
\hypersetup{pdfencoding=auto,psdextra}
