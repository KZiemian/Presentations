% ---------------------------------------------------------------------
% Basic configuration of Beamera and Jagiellonian
% ---------------------------------------------------------------------
\RequirePackage[l2tabu, orthodox]{nag}



\ifx\PresentationStyle\notset
\def\PresentationStyle{dark}
\fi



\documentclass[10pt,t]{beamer}
\mode<presentation>
\usetheme[style=\PresentationStyle,logoLang=Latin,logoColor=monochromaticJUwhite,JUlogotitle=yes]{jagiellonian}



% ---------------------------------------
% Configuration files of Jagiellonian loceted in catalog preambule
% ---------------------------------------
% Configuration for polish language
% Need description
\usepackage[english]{babel}





% % ------------------------------
% % Better support of polish chars in technical parts of PDF
% % ------------------------------
% \hypersetup{pdfencoding=auto,psdextra}

% Package "textpos" give as enviroment "textblock" which is very usefull in
% arranging text on slides.

% This is standard configuration of "textpos"
\usepackage[overlay,absolute]{textpos}

% If you need to see bounds of "textblock's" comment line above and uncomment
% one below.

% Caution! When showboxes option is on significant ammunt of space is add
% to the top of textblock and as such, everyting put in them gone down.
% We need to check how to remove this bug.

% \usepackage[showboxes,overlay,absolute]{textpos}



% Setting scale length for package "textpos"
\setlength{\TPHorizModule}{10mm}
\setlength{\TPVertModule}{\TPHorizModule}


% ---------------------------------------
% TikZ
% ---------------------------------------
% Importing TikZ libraries
\usetikzlibrary{arrows.meta}
\usetikzlibrary{positioning}





% % Configuration package "bm" that need for making bold symbols
% \newcommand{\bmmax}{0}
% \newcommand{\hmmax}{0}
% \usepackage{bm}




% ---------------------------------------
% Packages for scientific texts
% ---------------------------------------
% \let\lll\undefined  % Sometimes you must use this line to allow
% "amsmath" package to works with packages with packages for polish
% languge imported
% /preambul/LanguageSettings/JagiellonianPolishLanguageSettings.tex.
% This comments (probably) removes polish letter Ł.
\usepackage{amsmath}  % Packages from American Mathematical Society (AMS)
\usepackage{amssymb}
\usepackage{amscd}
\usepackage{amsthm}
\usepackage{siunitx}  % Package for typsetting SI units.
\usepackage{upgreek}  % Better looking greek letters.
% Example of using upgreek: pi = \uppi


\usepackage{calrsfs}  % Zmienia czcionkę kaligraficzną w \mathcal
% na ładniejszą. Może w innych miejscach robi to samo, ale o tym nic
% nie wiem.










% ---------------------------------------
% Packages written for lectures "Geometria 3D dla twórców gier wideo"
% ---------------------------------------
% \usepackage{./ProgramowanieSymulacjiFizykiPaczki/ProgramowanieSymulacjiFizyki}
% \usepackage{./ProgramowanieSymulacjiFizykiPaczki/ProgramowanieSymulacjiFizykiIndeksy}
% \usepackage{./ProgramowanieSymulacjiFizykiPaczki/ProgramowanieSymulacjiFizykiTikZStyle}





% !!!!!!!!!!!!!!!!!!!!!!!!!!!!!!
% !!!!!!!!!!!!!!!!!!!!!!!!!!!!!!
% EVIL STUFF
\if\JUlogotitle1
\edef\LogoJUPath{LogoJU_\JUlogoLang/LogoJU_\JUlogoShape_\JUlogoColor.pdf}
\titlegraphic{\hfill\includegraphics[scale=0.22]
{./JagiellonianPictures/\LogoJUPath}}
\fi
% ---------------------------------------
% Commands for handling colors
% ---------------------------------------


% Command for setting normal text color for some text in math modestyle
% Text color depend on used style of Jagiellonian

% Beamer version of command
\newcommand{\TextWithNormalTextColor}[1]{%
  {\color{jNormalTextFGColor}
    \setbeamercolor{math text}{fg=jNormalTextFGColor} {#1}}
}

% Article and similar classes version of command
% \newcommand{\TextWithNormalTextColor}[1]{%
%   {\color{jNormalTextsFGColor} {#1}}
% }



% Beamer version of command
\newcommand{\NormalTextInMathMode}[1]{%
  {\color{jNormalTextFGColor}
    \setbeamercolor{math text}{fg=jNormalTextFGColor} \text{#1}}
}


% Article and similar classes version of command
% \newcommand{\NormalTextInMathMode}[1]{%
%   {\color{jNormalTextsFGColor} \text{#1}}
% }




% Command that sets color of some mathematical text to the same color
% that has normal text in header (?)

% Beamer version of the command
\newcommand{\MathTextFrametitleFGColor}[1]{%
  {\color{jFrametitleFGColor}
    \setbeamercolor{math text}{fg=jFrametitleFGColor} #1}
}

% Article and similar classes version of the command
% \newcommand{\MathTextWhiteColor}[1]{{\color{jFrametitleFGColor} #1}}





% Command for setting color of alert text for some text in math modestyle

% Beamer version of the command
\newcommand{\MathTextAlertColor}[1]{%
  {\color{jOrange} \setbeamercolor{math text}{fg=jOrange} #1}
}

% Article and similar classes version of the command
% \newcommand{\MathTextAlertColor}[1]{{\color{jOrange} #1}}





% Command that allow you to sets chosen color as the color of some text into
% math mode. Due to some nuances in the way that Beamer handle colors
% it not work in all cases. We hope that in the future we will improve it.

% Beamer version of the command
\newcommand{\SetMathTextColor}[2]{%
  {\color{#1} \setbeamercolor{math text}{fg=#1} #2}
}


% Article and similar classes version of the command
% \newcommand{\SetMathTextColor}[2]{{\color{#1} #2}}










% ---------------------------------------
% Commands for few special slides
% ---------------------------------------
\newcommand{\EndingSlide}[1]{%
  \begin{frame}[standout]

    \begingroup

    \color{jFrametitleFGColor}

    #1

    \endgroup

  \end{frame}
}










% ---------------------------------------
% Commands for setting background pictures for some slides
% ---------------------------------------
\newcommand{\TitleBackgroundPicture}
{./JagiellonianPictures/Backgrounds/LajkonikDark.png}
\newcommand{\SectionBackgroundPicture}
{./JagiellonianPictures/Backgrounds/LajkonikLight.png}



\newcommand{\TitleSlideWithPicture}{%
  \begingroup

  \usebackgroundtemplate{%
    \includegraphics[height=\paperheight]{\TitleBackgroundPicture}}

  \maketitle

  \endgroup
}





\newcommand{\SectionSlideWithPicture}[1]{%
  \begingroup

  \usebackgroundtemplate{%
    \includegraphics[height=\paperheight]{\SectionBackgroundPicture}}

  \setbeamercolor{titlelike}{fg=normal text.fg}

  \section{#1}

  \endgroup
}










% ---------------------------------------
% Commands for lectures "Geometria 3D dla twórców gier wideo"
% Polish version
% ---------------------------------------
% Komendy teraz wykomentowane były potrzebne, gdy loga były na niebieskim
% tle, nie na białym. A są na białym bo tego chcieli w biurze projektu.
% \newcommand{\FundingLogoWhitePicturePL}
% {./PresentationPictures/CommonPictures/logotypFundusze_biale_bez_tla2.pdf}
\newcommand{\FundingLogoColorPicturePL}
{./PresentationPictures/CommonPictures/European_Funds_color_PL.pdf}
% \newcommand{\EULogoWhitePicturePL}
% {./PresentationPictures/CommonPictures/logotypUE_biale_bez_tla2.pdf}
\newcommand{\EUSocialFundLogoColorPicturePL}
{./PresentationPictures/CommonPictures/EU_Social_Fund_color_PL.pdf}
% \newcommand{\ZintegrUJLogoWhitePicturePL}
% {./PresentationPictures/CommonPictures/zintegruj-logo-white.pdf}
\newcommand{\ZintegrUJLogoColorPicturePL}
{./PresentationPictures/CommonPictures/ZintegrUJ_color.pdf}
\newcommand{\JULogoColorPicturePL}
{./JagiellonianPictures/LogoJU_PL/LogoJU_A_color.pdf}





\newcommand{\GeometryThreeDSpecialBeginningSlidePL}{%
  \begin{frame}[standout]

    \begin{textblock}{11}(1,0.7)

      \begin{flushleft}

        \mdseries

        \footnotesize

        \color{jFrametitleFGColor}

        Materiał powstał w ramach projektu współfinansowanego ze środków
        Unii Europejskiej w ramach Europejskiego Funduszu Społecznego
        POWR.03.05.00-00-Z309/17-00.

      \end{flushleft}

    \end{textblock}





    \begin{textblock}{10}(0,2.2)

      \tikz \fill[color=jBackgroundStyleLight] (0,0) rectangle (12.8,-1.5);

    \end{textblock}


    \begin{textblock}{3.2}(1,2.45)

      \includegraphics[scale=0.3]{\FundingLogoColorPicturePL}

    \end{textblock}


    \begin{textblock}{2.5}(3.7,2.5)

      \includegraphics[scale=0.2]{\JULogoColorPicturePL}

    \end{textblock}


    \begin{textblock}{2.5}(6,2.4)

      \includegraphics[scale=0.1]{\ZintegrUJLogoColorPicturePL}

    \end{textblock}


    \begin{textblock}{4.2}(8.4,2.6)

      \includegraphics[scale=0.3]{\EUSocialFundLogoColorPicturePL}

    \end{textblock}

  \end{frame}
}



\newcommand{\GeometryThreeDTwoSpecialBeginningSlidesPL}{%
  \begin{frame}[standout]

    \begin{textblock}{11}(1,0.7)

      \begin{flushleft}

        \mdseries

        \footnotesize

        \color{jFrametitleFGColor}

        Materiał powstał w ramach projektu współfinansowanego ze środków
        Unii Europejskiej w ramach Europejskiego Funduszu Społecznego
        POWR.03.05.00-00-Z309/17-00.

      \end{flushleft}

    \end{textblock}





    \begin{textblock}{10}(0,2.2)

      \tikz \fill[color=jBackgroundStyleLight] (0,0) rectangle (12.8,-1.5);

    \end{textblock}


    \begin{textblock}{3.2}(1,2.45)

      \includegraphics[scale=0.3]{\FundingLogoColorPicturePL}

    \end{textblock}


    \begin{textblock}{2.5}(3.7,2.5)

      \includegraphics[scale=0.2]{\JULogoColorPicturePL}

    \end{textblock}


    \begin{textblock}{2.5}(6,2.4)

      \includegraphics[scale=0.1]{\ZintegrUJLogoColorPicturePL}

    \end{textblock}


    \begin{textblock}{4.2}(8.4,2.6)

      \includegraphics[scale=0.3]{\EUSocialFundLogoColorPicturePL}

    \end{textblock}

  \end{frame}





  \TitleSlideWithPicture
}



\newcommand{\GeometryThreeDSpecialEndingSlidePL}{%
  \begin{frame}[standout]

    \begin{textblock}{11}(1,0.7)

      \begin{flushleft}

        \mdseries

        \footnotesize

        \color{jFrametitleFGColor}

        Materiał powstał w ramach projektu współfinansowanego ze środków
        Unii Europejskiej w~ramach Europejskiego Funduszu Społecznego
        POWR.03.05.00-00-Z309/17-00.

      \end{flushleft}

    \end{textblock}





    \begin{textblock}{10}(0,2.2)

      \tikz \fill[color=jBackgroundStyleLight] (0,0) rectangle (12.8,-1.5);

    \end{textblock}


    \begin{textblock}{3.2}(1,2.45)

      \includegraphics[scale=0.3]{\FundingLogoColorPicturePL}

    \end{textblock}


    \begin{textblock}{2.5}(3.7,2.5)

      \includegraphics[scale=0.2]{\JULogoColorPicturePL}

    \end{textblock}


    \begin{textblock}{2.5}(6,2.4)

      \includegraphics[scale=0.1]{\ZintegrUJLogoColorPicturePL}

    \end{textblock}


    \begin{textblock}{4.2}(8.4,2.6)

      \includegraphics[scale=0.3]{\EUSocialFundLogoColorPicturePL}

    \end{textblock}





    \begin{textblock}{11}(1,4)

      \begin{flushleft}

        \mdseries

        \footnotesize

        \RaggedRight

        \color{jFrametitleFGColor}

        Treść niniejszego wykładu jest udostępniona na~licencji
        Creative Commons (\textsc{cc}), z~uzna\-niem autorstwa
        (\textsc{by}) oraz udostępnianiem na tych samych warunkach
        (\textsc{sa}). Rysunki i~wy\-kresy zawarte w~wykładzie są
        autorstwa dr.~hab.~Pawła Węgrzyna et~al. i~są dostępne
        na tej samej licencji, o~ile nie wskazano inaczej.
        W~prezentacji wykorzystano temat Beamera Jagiellonian,
        oparty na~temacie Metropolis Matthiasa Vogelgesanga,
        dostępnym na licencji \LaTeX{} Project Public License~1.3c
        pod adresem: \colorhref{https://github.com/matze/mtheme}
        {https://github.com/matze/mtheme}.

        Projekt typograficzny: Iwona Grabska-Gradzińska \\
        Skład: Kamil Ziemian;
        Korekta: Wojciech Palacz \\
        Modele: Dariusz Frymus, Kamil Nowakowski \\
        Rysunki i~wykresy: Kamil Ziemian, Paweł Węgrzyn, Wojciech Palacz

      \end{flushleft}

    \end{textblock}

  \end{frame}
}



\newcommand{\GeometryThreeDTwoSpecialEndingSlidesPL}[1]{%
  \begin{frame}[standout]


    \begin{textblock}{11}(1,0.7)

      \begin{flushleft}

        \mdseries

        \footnotesize

        \color{jFrametitleFGColor}

        Materiał powstał w ramach projektu współfinansowanego ze środków
        Unii Europejskiej w~ramach Europejskiego Funduszu Społecznego
        POWR.03.05.00-00-Z309/17-00.

      \end{flushleft}

    \end{textblock}





    \begin{textblock}{10}(0,2.2)

      \tikz \fill[color=jBackgroundStyleLight] (0,0) rectangle (12.8,-1.5);

    \end{textblock}


    \begin{textblock}{3.2}(1,2.45)

      \includegraphics[scale=0.3]{\FundingLogoColorPicturePL}

    \end{textblock}


    \begin{textblock}{2.5}(3.7,2.5)

      \includegraphics[scale=0.2]{\JULogoColorPicturePL}

    \end{textblock}


    \begin{textblock}{2.5}(6,2.4)

      \includegraphics[scale=0.1]{\ZintegrUJLogoColorPicturePL}

    \end{textblock}


    \begin{textblock}{4.2}(8.4,2.6)

      \includegraphics[scale=0.3]{\EUSocialFundLogoColorPicturePL}

    \end{textblock}





    \begin{textblock}{11}(1,4)

      \begin{flushleft}

        \mdseries

        \footnotesize

        \RaggedRight

        \color{jFrametitleFGColor}

        Treść niniejszego wykładu jest udostępniona na~licencji
        Creative Commons (\textsc{cc}), z~uzna\-niem autorstwa
        (\textsc{by}) oraz udostępnianiem na tych samych warunkach
        (\textsc{sa}). Rysunki i~wy\-kresy zawarte w~wykładzie są
        autorstwa dr.~hab.~Pawła Węgrzyna et~al. i~są dostępne
        na tej samej licencji, o~ile nie wskazano inaczej.
        W~prezentacji wykorzystano temat Beamera Jagiellonian,
        oparty na~temacie Metropolis Matthiasa Vogelgesanga,
        dostępnym na licencji \LaTeX{} Project Public License~1.3c
        pod adresem: \colorhref{https://github.com/matze/mtheme}
        {https://github.com/matze/mtheme}.

        Projekt typograficzny: Iwona Grabska-Gradzińska \\
        Skład: Kamil Ziemian;
        Korekta: Wojciech Palacz \\
        Modele: Dariusz Frymus, Kamil Nowakowski \\
        Rysunki i~wykresy: Kamil Ziemian, Paweł Węgrzyn, Wojciech Palacz

      \end{flushleft}

    \end{textblock}

  \end{frame}





  \begin{frame}[standout]

    \begingroup

    \color{jFrametitleFGColor}

    #1

    \endgroup

  \end{frame}
}



\newcommand{\GeometryThreeDSpecialEndingSlideVideoPL}{%
  \begin{frame}[standout]

    \begin{textblock}{11}(1,0.7)

      \begin{flushleft}

        \mdseries

        \footnotesize

        \color{jFrametitleFGColor}

        Materiał powstał w ramach projektu współfinansowanego ze środków
        Unii Europejskiej w~ramach Europejskiego Funduszu Społecznego
        POWR.03.05.00-00-Z309/17-00.

      \end{flushleft}

    \end{textblock}





    \begin{textblock}{10}(0,2.2)

      \tikz \fill[color=jBackgroundStyleLight] (0,0) rectangle (12.8,-1.5);

    \end{textblock}


    \begin{textblock}{3.2}(1,2.45)

      \includegraphics[scale=0.3]{\FundingLogoColorPicturePL}

    \end{textblock}


    \begin{textblock}{2.5}(3.7,2.5)

      \includegraphics[scale=0.2]{\JULogoColorPicturePL}

    \end{textblock}


    \begin{textblock}{2.5}(6,2.4)

      \includegraphics[scale=0.1]{\ZintegrUJLogoColorPicturePL}

    \end{textblock}


    \begin{textblock}{4.2}(8.4,2.6)

      \includegraphics[scale=0.3]{\EUSocialFundLogoColorPicturePL}

    \end{textblock}





    \begin{textblock}{11}(1,4)

      \begin{flushleft}

        \mdseries

        \footnotesize

        \RaggedRight

        \color{jFrametitleFGColor}

        Treść niniejszego wykładu jest udostępniona na~licencji
        Creative Commons (\textsc{cc}), z~uzna\-niem autorstwa
        (\textsc{by}) oraz udostępnianiem na tych samych warunkach
        (\textsc{sa}). Rysunki i~wy\-kresy zawarte w~wykładzie są
        autorstwa dr.~hab.~Pawła Węgrzyna et~al. i~są dostępne
        na tej samej licencji, o~ile nie wskazano inaczej.
        W~prezentacji wykorzystano temat Beamera Jagiellonian,
        oparty na~temacie Metropolis Matthiasa Vogelgesanga,
        dostępnym na licencji \LaTeX{} Project Public License~1.3c
        pod adresem: \colorhref{https://github.com/matze/mtheme}
        {https://github.com/matze/mtheme}.

        Projekt typograficzny: Iwona Grabska-Gradzińska;
        Skład: Kamil Ziemian \\
        Korekta: Wojciech Palacz;
        Modele: Dariusz Frymus, Kamil Nowakowski \\
        Rysunki i~wykresy: Kamil Ziemian, Paweł Węgrzyn, Wojciech Palacz \\
        Montaż: Agencja Filmowa Film \& Television Production~-- Zbigniew
        Masklak

      \end{flushleft}

    \end{textblock}

  \end{frame}
}





\newcommand{\GeometryThreeDTwoSpecialEndingSlidesVideoPL}[1]{%
  \begin{frame}[standout]

    \begin{textblock}{11}(1,0.7)

      \begin{flushleft}

        \mdseries

        \footnotesize

        \color{jFrametitleFGColor}

        Materiał powstał w ramach projektu współfinansowanego ze środków
        Unii Europejskiej w~ramach Europejskiego Funduszu Społecznego
        POWR.03.05.00-00-Z309/17-00.

      \end{flushleft}

    \end{textblock}





    \begin{textblock}{10}(0,2.2)

      \tikz \fill[color=jBackgroundStyleLight] (0,0) rectangle (12.8,-1.5);

    \end{textblock}


    \begin{textblock}{3.2}(1,2.45)

      \includegraphics[scale=0.3]{\FundingLogoColorPicturePL}

    \end{textblock}


    \begin{textblock}{2.5}(3.7,2.5)

      \includegraphics[scale=0.2]{\JULogoColorPicturePL}

    \end{textblock}


    \begin{textblock}{2.5}(6,2.4)

      \includegraphics[scale=0.1]{\ZintegrUJLogoColorPicturePL}

    \end{textblock}


    \begin{textblock}{4.2}(8.4,2.6)

      \includegraphics[scale=0.3]{\EUSocialFundLogoColorPicturePL}

    \end{textblock}





    \begin{textblock}{11}(1,4)

      \begin{flushleft}

        \mdseries

        \footnotesize

        \RaggedRight

        \color{jFrametitleFGColor}

        Treść niniejszego wykładu jest udostępniona na~licencji
        Creative Commons (\textsc{cc}), z~uzna\-niem autorstwa
        (\textsc{by}) oraz udostępnianiem na tych samych warunkach
        (\textsc{sa}). Rysunki i~wy\-kresy zawarte w~wykładzie są
        autorstwa dr.~hab.~Pawła Węgrzyna et~al. i~są dostępne
        na tej samej licencji, o~ile nie wskazano inaczej.
        W~prezentacji wykorzystano temat Beamera Jagiellonian,
        oparty na~temacie Metropolis Matthiasa Vogelgesanga,
        dostępnym na licencji \LaTeX{} Project Public License~1.3c
        pod adresem: \colorhref{https://github.com/matze/mtheme}
        {https://github.com/matze/mtheme}.

        Projekt typograficzny: Iwona Grabska-Gradzińska;
        Skład: Kamil Ziemian \\
        Korekta: Wojciech Palacz;
        Modele: Dariusz Frymus, Kamil Nowakowski \\
        Rysunki i~wykresy: Kamil Ziemian, Paweł Węgrzyn, Wojciech Palacz \\
        Montaż: Agencja Filmowa Film \& Television Production~-- Zbigniew
        Masklak

      \end{flushleft}

    \end{textblock}

  \end{frame}





  \begin{frame}[standout]


    \begingroup

    \color{jFrametitleFGColor}

    #1

    \endgroup

  \end{frame}
}










% ---------------------------------------
% Commands for lectures "Geometria 3D dla twórców gier wideo"
% English version
% ---------------------------------------
% \newcommand{\FundingLogoWhitePictureEN}
% {./PresentationPictures/CommonPictures/logotypFundusze_biale_bez_tla2.pdf}
\newcommand{\FundingLogoColorPictureEN}
{./PresentationPictures/CommonPictures/European_Funds_color_EN.pdf}
% \newcommand{\EULogoWhitePictureEN}
% {./PresentationPictures/CommonPictures/logotypUE_biale_bez_tla2.pdf}
\newcommand{\EUSocialFundLogoColorPictureEN}
{./PresentationPictures/CommonPictures/EU_Social_Fund_color_EN.pdf}
% \newcommand{\ZintegrUJLogoWhitePictureEN}
% {./PresentationPictures/CommonPictures/zintegruj-logo-white.pdf}
\newcommand{\ZintegrUJLogoColorPictureEN}
{./PresentationPictures/CommonPictures/ZintegrUJ_color.pdf}
\newcommand{\JULogoColorPictureEN}
{./JagiellonianPictures/LogoJU_EN/LogoJU_A_color.pdf}



\newcommand{\GeometryThreeDSpecialBeginningSlideEN}{%
  \begin{frame}[standout]

    \begin{textblock}{11}(1,0.7)

      \begin{flushleft}

        \mdseries

        \footnotesize

        \color{jFrametitleFGColor}

        This content was created as part of a project co-financed by the
        European Union within the framework of the European Social Fund
        POWR.03.05.00-00-Z309/17-00.

      \end{flushleft}

    \end{textblock}





    \begin{textblock}{10}(0,2.2)

      \tikz \fill[color=jBackgroundStyleLight] (0,0) rectangle (12.8,-1.5);

    \end{textblock}


    \begin{textblock}{3.2}(0.7,2.45)

      \includegraphics[scale=0.3]{\FundingLogoColorPictureEN}

    \end{textblock}


    \begin{textblock}{2.5}(4.15,2.5)

      \includegraphics[scale=0.2]{\JULogoColorPictureEN}

    \end{textblock}


    \begin{textblock}{2.5}(6.35,2.4)

      \includegraphics[scale=0.1]{\ZintegrUJLogoColorPictureEN}

    \end{textblock}


    \begin{textblock}{4.2}(8.4,2.6)

      \includegraphics[scale=0.3]{\EUSocialFundLogoColorPictureEN}

    \end{textblock}

  \end{frame}
}



\newcommand{\GeometryThreeDTwoSpecialBeginningSlidesEN}{%
  \begin{frame}[standout]

    \begin{textblock}{11}(1,0.7)

      \begin{flushleft}

        \mdseries

        \footnotesize

        \color{jFrametitleFGColor}

        This content was created as part of a project co-financed by the
        European Union within the framework of the European Social Fund
        POWR.03.05.00-00-Z309/17-00.

      \end{flushleft}

    \end{textblock}





    \begin{textblock}{10}(0,2.2)

      \tikz \fill[color=jBackgroundStyleLight] (0,0) rectangle (12.8,-1.5);

    \end{textblock}


    \begin{textblock}{3.2}(0.7,2.45)

      \includegraphics[scale=0.3]{\FundingLogoColorPictureEN}

    \end{textblock}


    \begin{textblock}{2.5}(4.15,2.5)

      \includegraphics[scale=0.2]{\JULogoColorPictureEN}

    \end{textblock}


    \begin{textblock}{2.5}(6.35,2.4)

      \includegraphics[scale=0.1]{\ZintegrUJLogoColorPictureEN}

    \end{textblock}


    \begin{textblock}{4.2}(8.4,2.6)

      \includegraphics[scale=0.3]{\EUSocialFundLogoColorPictureEN}

    \end{textblock}

  \end{frame}





  \TitleSlideWithPicture
}



\newcommand{\GeometryThreeDSpecialEndingSlideEN}{%
  \begin{frame}[standout]

    \begin{textblock}{11}(1,0.7)

      \begin{flushleft}

        \mdseries

        \footnotesize

        \color{jFrametitleFGColor}

        This content was created as part of a project co-financed by the
        European Union within the framework of the European Social Fund
        POWR.03.05.00-00-Z309/17-00.

      \end{flushleft}

    \end{textblock}





    \begin{textblock}{10}(0,2.2)

      \tikz \fill[color=jBackgroundStyleLight] (0,0) rectangle (12.8,-1.5);

    \end{textblock}


    \begin{textblock}{3.2}(0.7,2.45)

      \includegraphics[scale=0.3]{\FundingLogoColorPictureEN}

    \end{textblock}


    \begin{textblock}{2.5}(4.15,2.5)

      \includegraphics[scale=0.2]{\JULogoColorPictureEN}

    \end{textblock}


    \begin{textblock}{2.5}(6.35,2.4)

      \includegraphics[scale=0.1]{\ZintegrUJLogoColorPictureEN}

    \end{textblock}


    \begin{textblock}{4.2}(8.4,2.6)

      \includegraphics[scale=0.3]{\EUSocialFundLogoColorPictureEN}

    \end{textblock}





    \begin{textblock}{11}(1,4)

      \begin{flushleft}

        \mdseries

        \footnotesize

        \RaggedRight

        \color{jFrametitleFGColor}

        The content of this lecture is made available under a~Creative
        Commons licence (\textsc{cc}), giving the author the credits
        (\textsc{by}) and putting an obligation to share on the same terms
        (\textsc{sa}). Figures and diagrams included in the lecture are
        authored by Paweł Węgrzyn et~al., and are available under the same
        license unless indicated otherwise.\\ The presentation uses the
        Beamer Jagiellonian theme based on Matthias Vogelgesang’s
        Metropolis theme, available under license \LaTeX{} Project
        Public License~1.3c at: \colorhref{https://github.com/matze/mtheme}
        {https://github.com/matze/mtheme}.

        Typographic design: Iwona Grabska-Gradzińska \\
        \LaTeX{} Typesetting: Kamil Ziemian \\
        Proofreading: Wojciech Palacz,
        Monika Stawicka \\
        3D Models: Dariusz Frymus, Kamil Nowakowski \\
        Figures and charts: Kamil Ziemian, Paweł Węgrzyn, Wojciech Palacz

      \end{flushleft}

    \end{textblock}

  \end{frame}
}



\newcommand{\GeometryThreeDTwoSpecialEndingSlidesEN}[1]{%
  \begin{frame}[standout]


    \begin{textblock}{11}(1,0.7)

      \begin{flushleft}

        \mdseries

        \footnotesize

        \color{jFrametitleFGColor}

        This content was created as part of a project co-financed by the
        European Union within the framework of the European Social Fund
        POWR.03.05.00-00-Z309/17-00.

      \end{flushleft}

    \end{textblock}





    \begin{textblock}{10}(0,2.2)

      \tikz \fill[color=jBackgroundStyleLight] (0,0) rectangle (12.8,-1.5);

    \end{textblock}


    \begin{textblock}{3.2}(0.7,2.45)

      \includegraphics[scale=0.3]{\FundingLogoColorPictureEN}

    \end{textblock}


    \begin{textblock}{2.5}(4.15,2.5)

      \includegraphics[scale=0.2]{\JULogoColorPictureEN}

    \end{textblock}


    \begin{textblock}{2.5}(6.35,2.4)

      \includegraphics[scale=0.1]{\ZintegrUJLogoColorPictureEN}

    \end{textblock}


    \begin{textblock}{4.2}(8.4,2.6)

      \includegraphics[scale=0.3]{\EUSocialFundLogoColorPictureEN}

    \end{textblock}





    \begin{textblock}{11}(1,4)

      \begin{flushleft}

        \mdseries

        \footnotesize

        \RaggedRight

        \color{jFrametitleFGColor}

        The content of this lecture is made available under a~Creative
        Commons licence (\textsc{cc}), giving the author the credits
        (\textsc{by}) and putting an obligation to share on the same terms
        (\textsc{sa}). Figures and diagrams included in the lecture are
        authored by Paweł Węgrzyn et~al., and are available under the same
        license unless indicated otherwise.\\ The presentation uses the
        Beamer Jagiellonian theme based on Matthias Vogelgesang’s
        Metropolis theme, available under license \LaTeX{} Project
        Public License~1.3c at: \colorhref{https://github.com/matze/mtheme}
        {https://github.com/matze/mtheme}.

        Typographic design: Iwona Grabska-Gradzińska \\
        \LaTeX{} Typesetting: Kamil Ziemian \\
        Proofreading: Wojciech Palacz,
        Monika Stawicka \\
        3D Models: Dariusz Frymus, Kamil Nowakowski \\
        Figures and charts: Kamil Ziemian, Paweł Węgrzyn, Wojciech Palacz

      \end{flushleft}

    \end{textblock}

  \end{frame}





  \begin{frame}[standout]

    \begingroup

    \color{jFrametitleFGColor}

    #1

    \endgroup

  \end{frame}
}



\newcommand{\GeometryThreeDSpecialEndingSlideVideoVerOneEN}{%
  \begin{frame}[standout]

    \begin{textblock}{11}(1,0.7)

      \begin{flushleft}

        \mdseries

        \footnotesize

        \color{jFrametitleFGColor}

        This content was created as part of a project co-financed by the
        European Union within the framework of the European Social Fund
        POWR.03.05.00-00-Z309/17-00.

      \end{flushleft}

    \end{textblock}





    \begin{textblock}{10}(0,2.2)

      \tikz \fill[color=jBackgroundStyleLight] (0,0) rectangle (12.8,-1.5);

    \end{textblock}


    \begin{textblock}{3.2}(0.7,2.45)

      \includegraphics[scale=0.3]{\FundingLogoColorPictureEN}

    \end{textblock}


    \begin{textblock}{2.5}(4.15,2.5)

      \includegraphics[scale=0.2]{\JULogoColorPictureEN}

    \end{textblock}


    \begin{textblock}{2.5}(6.35,2.4)

      \includegraphics[scale=0.1]{\ZintegrUJLogoColorPictureEN}

    \end{textblock}


    \begin{textblock}{4.2}(8.4,2.6)

      \includegraphics[scale=0.3]{\EUSocialFundLogoColorPictureEN}

    \end{textblock}





    \begin{textblock}{11}(1,4)

      \begin{flushleft}

        \mdseries

        \footnotesize

        \RaggedRight

        \color{jFrametitleFGColor}

        The content of this lecture is made available under a Creative
        Commons licence (\textsc{cc}), giving the author the credits
        (\textsc{by}) and putting an obligation to share on the same terms
        (\textsc{sa}). Figures and diagrams included in the lecture are
        authored by Paweł Węgrzyn et~al., and are available under the same
        license unless indicated otherwise.\\ The presentation uses the
        Beamer Jagiellonian theme based on Matthias Vogelgesang’s
        Metropolis theme, available under license \LaTeX{} Project
        Public License~1.3c at: \colorhref{https://github.com/matze/mtheme}
        {https://github.com/matze/mtheme}.

        Typographic design: Iwona Grabska-Gradzińska;
        \LaTeX{} Typesetting: Kamil Ziemian \\
        Proofreading: Wojciech Palacz,
        Monika Stawicka \\
        3D Models: Dariusz Frymus, Kamil Nowakowski \\
        Figures and charts: Kamil Ziemian, Paweł Węgrzyn, Wojciech
        Palacz \\
        Film editing: Agencja Filmowa Film \& Television Production~--
        Zbigniew Masklak

      \end{flushleft}

    \end{textblock}

  \end{frame}
}



\newcommand{\GeometryThreeDSpecialEndingSlideVideoVerTwoEN}{%
  \begin{frame}[standout]

    \begin{textblock}{11}(1,0.7)

      \begin{flushleft}

        \mdseries

        \footnotesize

        \color{jFrametitleFGColor}

        This content was created as part of a project co-financed by the
        European Union within the framework of the European Social Fund
        POWR.03.05.00-00-Z309/17-00.

      \end{flushleft}

    \end{textblock}





    \begin{textblock}{10}(0,2.2)

      \tikz \fill[color=jBackgroundStyleLight] (0,0) rectangle (12.8,-1.5);

    \end{textblock}


    \begin{textblock}{3.2}(0.7,2.45)

      \includegraphics[scale=0.3]{\FundingLogoColorPictureEN}

    \end{textblock}


    \begin{textblock}{2.5}(4.15,2.5)

      \includegraphics[scale=0.2]{\JULogoColorPictureEN}

    \end{textblock}


    \begin{textblock}{2.5}(6.35,2.4)

      \includegraphics[scale=0.1]{\ZintegrUJLogoColorPictureEN}

    \end{textblock}


    \begin{textblock}{4.2}(8.4,2.6)

      \includegraphics[scale=0.3]{\EUSocialFundLogoColorPictureEN}

    \end{textblock}





    \begin{textblock}{11}(1,4)

      \begin{flushleft}

        \mdseries

        \footnotesize

        \RaggedRight

        \color{jFrametitleFGColor}

        The content of this lecture is made available under a Creative
        Commons licence (\textsc{cc}), giving the author the credits
        (\textsc{by}) and putting an obligation to share on the same terms
        (\textsc{sa}). Figures and diagrams included in the lecture are
        authored by Paweł Węgrzyn et~al., and are available under the same
        license unless indicated otherwise.\\ The presentation uses the
        Beamer Jagiellonian theme based on Matthias Vogelgesang’s
        Metropolis theme, available under license \LaTeX{} Project
        Public License~1.3c at: \colorhref{https://github.com/matze/mtheme}
        {https://github.com/matze/mtheme}.

        Typographic design: Iwona Grabska-Gradzińska;
        \LaTeX{} Typesetting: Kamil Ziemian \\
        Proofreading: Wojciech Palacz,
        Monika Stawicka \\
        3D Models: Dariusz Frymus, Kamil Nowakowski \\
        Figures and charts: Kamil Ziemian, Paweł Węgrzyn, Wojciech
        Palacz \\
        Film editing: IMAVI -- Joanna Kozakiewicz, Krzysztof Magda, Nikodem
        Frodyma

      \end{flushleft}

    \end{textblock}

  \end{frame}
}



\newcommand{\GeometryThreeDSpecialEndingSlideVideoVerThreeEN}{%
  \begin{frame}[standout]

    \begin{textblock}{11}(1,0.7)

      \begin{flushleft}

        \mdseries

        \footnotesize

        \color{jFrametitleFGColor}

        This content was created as part of a project co-financed by the
        European Union within the framework of the European Social Fund
        POWR.03.05.00-00-Z309/17-00.

      \end{flushleft}

    \end{textblock}





    \begin{textblock}{10}(0,2.2)

      \tikz \fill[color=jBackgroundStyleLight] (0,0) rectangle (12.8,-1.5);

    \end{textblock}


    \begin{textblock}{3.2}(0.7,2.45)

      \includegraphics[scale=0.3]{\FundingLogoColorPictureEN}

    \end{textblock}


    \begin{textblock}{2.5}(4.15,2.5)

      \includegraphics[scale=0.2]{\JULogoColorPictureEN}

    \end{textblock}


    \begin{textblock}{2.5}(6.35,2.4)

      \includegraphics[scale=0.1]{\ZintegrUJLogoColorPictureEN}

    \end{textblock}


    \begin{textblock}{4.2}(8.4,2.6)

      \includegraphics[scale=0.3]{\EUSocialFundLogoColorPictureEN}

    \end{textblock}





    \begin{textblock}{11}(1,4)

      \begin{flushleft}

        \mdseries

        \footnotesize

        \RaggedRight

        \color{jFrametitleFGColor}

        The content of this lecture is made available under a Creative
        Commons licence (\textsc{cc}), giving the author the credits
        (\textsc{by}) and putting an obligation to share on the same terms
        (\textsc{sa}). Figures and diagrams included in the lecture are
        authored by Paweł Węgrzyn et~al., and are available under the same
        license unless indicated otherwise.\\ The presentation uses the
        Beamer Jagiellonian theme based on Matthias Vogelgesang’s
        Metropolis theme, available under license \LaTeX{} Project
        Public License~1.3c at: \colorhref{https://github.com/matze/mtheme}
        {https://github.com/matze/mtheme}.

        Typographic design: Iwona Grabska-Gradzińska;
        \LaTeX{} Typesetting: Kamil Ziemian \\
        Proofreading: Wojciech Palacz,
        Monika Stawicka \\
        3D Models: Dariusz Frymus, Kamil Nowakowski \\
        Figures and charts: Kamil Ziemian, Paweł Węgrzyn, Wojciech
        Palacz \\
        Film editing: Agencja Filmowa Film \& Television Production~--
        Zbigniew Masklak \\
        Film editing: IMAVI -- Joanna Kozakiewicz, Krzysztof Magda, Nikodem
        Frodyma

      \end{flushleft}

    \end{textblock}

  \end{frame}
}



\newcommand{\GeometryThreeDTwoSpecialEndingSlidesVideoVerOneEN}[1]{%
  \begin{frame}[standout]

    \begin{textblock}{11}(1,0.7)

      \begin{flushleft}

        \mdseries

        \footnotesize

        \color{jFrametitleFGColor}

        This content was created as part of a project co-financed by the
        European Union within the framework of the European Social Fund
        POWR.03.05.00-00-Z309/17-00.

      \end{flushleft}

    \end{textblock}





    \begin{textblock}{10}(0,2.2)

      \tikz \fill[color=jBackgroundStyleLight] (0,0) rectangle (12.8,-1.5);

    \end{textblock}


    \begin{textblock}{3.2}(0.7,2.45)

      \includegraphics[scale=0.3]{\FundingLogoColorPictureEN}

    \end{textblock}


    \begin{textblock}{2.5}(4.15,2.5)

      \includegraphics[scale=0.2]{\JULogoColorPictureEN}

    \end{textblock}


    \begin{textblock}{2.5}(6.35,2.4)

      \includegraphics[scale=0.1]{\ZintegrUJLogoColorPictureEN}

    \end{textblock}


    \begin{textblock}{4.2}(8.4,2.6)

      \includegraphics[scale=0.3]{\EUSocialFundLogoColorPictureEN}

    \end{textblock}





    \begin{textblock}{11}(1,4)

      \begin{flushleft}

        \mdseries

        \footnotesize

        \RaggedRight

        \color{jFrametitleFGColor}

        The content of this lecture is made available under a Creative
        Commons licence (\textsc{cc}), giving the author the credits
        (\textsc{by}) and putting an obligation to share on the same terms
        (\textsc{sa}). Figures and diagrams included in the lecture are
        authored by Paweł Węgrzyn et~al., and are available under the same
        license unless indicated otherwise.\\ The presentation uses the
        Beamer Jagiellonian theme based on Matthias Vogelgesang’s
        Metropolis theme, available under license \LaTeX{} Project
        Public License~1.3c at: \colorhref{https://github.com/matze/mtheme}
        {https://github.com/matze/mtheme}.

        Typographic design: Iwona Grabska-Gradzińska;
        \LaTeX{} Typesetting: Kamil Ziemian \\
        Proofreading: Wojciech Palacz,
        Monika Stawicka \\
        3D Models: Dariusz Frymus, Kamil Nowakowski \\
        Figures and charts: Kamil Ziemian, Paweł Węgrzyn,
        Wojciech Palacz \\
        Film editing: Agencja Filmowa Film \& Television Production~--
        Zbigniew Masklak

      \end{flushleft}

    \end{textblock}

  \end{frame}





  \begin{frame}[standout]


    \begingroup

    \color{jFrametitleFGColor}

    #1

    \endgroup

  \end{frame}
}



\newcommand{\GeometryThreeDTwoSpecialEndingSlidesVideoVerTwoEN}[1]{%
  \begin{frame}[standout]

    \begin{textblock}{11}(1,0.7)

      \begin{flushleft}

        \mdseries

        \footnotesize

        \color{jFrametitleFGColor}

        This content was created as part of a project co-financed by the
        European Union within the framework of the European Social Fund
        POWR.03.05.00-00-Z309/17-00.

      \end{flushleft}

    \end{textblock}





    \begin{textblock}{10}(0,2.2)

      \tikz \fill[color=jBackgroundStyleLight] (0,0) rectangle (12.8,-1.5);

    \end{textblock}


    \begin{textblock}{3.2}(0.7,2.45)

      \includegraphics[scale=0.3]{\FundingLogoColorPictureEN}

    \end{textblock}


    \begin{textblock}{2.5}(4.15,2.5)

      \includegraphics[scale=0.2]{\JULogoColorPictureEN}

    \end{textblock}


    \begin{textblock}{2.5}(6.35,2.4)

      \includegraphics[scale=0.1]{\ZintegrUJLogoColorPictureEN}

    \end{textblock}


    \begin{textblock}{4.2}(8.4,2.6)

      \includegraphics[scale=0.3]{\EUSocialFundLogoColorPictureEN}

    \end{textblock}





    \begin{textblock}{11}(1,4)

      \begin{flushleft}

        \mdseries

        \footnotesize

        \RaggedRight

        \color{jFrametitleFGColor}

        The content of this lecture is made available under a Creative
        Commons licence (\textsc{cc}), giving the author the credits
        (\textsc{by}) and putting an obligation to share on the same terms
        (\textsc{sa}). Figures and diagrams included in the lecture are
        authored by Paweł Węgrzyn et~al., and are available under the same
        license unless indicated otherwise.\\ The presentation uses the
        Beamer Jagiellonian theme based on Matthias Vogelgesang’s
        Metropolis theme, available under license \LaTeX{} Project
        Public License~1.3c at: \colorhref{https://github.com/matze/mtheme}
        {https://github.com/matze/mtheme}.

        Typographic design: Iwona Grabska-Gradzińska;
        \LaTeX{} Typesetting: Kamil Ziemian \\
        Proofreading: Wojciech Palacz,
        Monika Stawicka \\
        3D Models: Dariusz Frymus, Kamil Nowakowski \\
        Figures and charts: Kamil Ziemian, Paweł Węgrzyn,
        Wojciech Palacz \\
        Film editing: IMAVI -- Joanna Kozakiewicz, Krzysztof Magda, Nikodem
        Frodyma

      \end{flushleft}

    \end{textblock}

  \end{frame}





  \begin{frame}[standout]


    \begingroup

    \color{jFrametitleFGColor}

    #1

    \endgroup

  \end{frame}
}



\newcommand{\GeometryThreeDTwoSpecialEndingSlidesVideoVerThreeEN}[1]{%
  \begin{frame}[standout]

    \begin{textblock}{11}(1,0.7)

      \begin{flushleft}

        \mdseries

        \footnotesize

        \color{jFrametitleFGColor}

        This content was created as part of a project co-financed by the
        European Union within the framework of the European Social Fund
        POWR.03.05.00-00-Z309/17-00.

      \end{flushleft}

    \end{textblock}





    \begin{textblock}{10}(0,2.2)

      \tikz \fill[color=jBackgroundStyleLight] (0,0) rectangle (12.8,-1.5);

    \end{textblock}


    \begin{textblock}{3.2}(0.7,2.45)

      \includegraphics[scale=0.3]{\FundingLogoColorPictureEN}

    \end{textblock}


    \begin{textblock}{2.5}(4.15,2.5)

      \includegraphics[scale=0.2]{\JULogoColorPictureEN}

    \end{textblock}


    \begin{textblock}{2.5}(6.35,2.4)

      \includegraphics[scale=0.1]{\ZintegrUJLogoColorPictureEN}

    \end{textblock}


    \begin{textblock}{4.2}(8.4,2.6)

      \includegraphics[scale=0.3]{\EUSocialFundLogoColorPictureEN}

    \end{textblock}





    \begin{textblock}{11}(1,4)

      \begin{flushleft}

        \mdseries

        \footnotesize

        \RaggedRight

        \color{jFrametitleFGColor}

        The content of this lecture is made available under a Creative
        Commons licence (\textsc{cc}), giving the author the credits
        (\textsc{by}) and putting an obligation to share on the same terms
        (\textsc{sa}). Figures and diagrams included in the lecture are
        authored by Paweł Węgrzyn et~al., and are available under the same
        license unless indicated otherwise. \\ The presentation uses the
        Beamer Jagiellonian theme based on Matthias Vogelgesang’s
        Metropolis theme, available under license \LaTeX{} Project
        Public License~1.3c at: \colorhref{https://github.com/matze/mtheme}
        {https://github.com/matze/mtheme}.

        Typographic design: Iwona Grabska-Gradzińska;
        \LaTeX{} Typesetting: Kamil Ziemian \\
        Proofreading: Leszek Hadasz, Wojciech Palacz,
        Monika Stawicka \\
        3D Models: Dariusz Frymus, Kamil Nowakowski \\
        Figures and charts: Kamil Ziemian, Paweł Węgrzyn,
        Wojciech Palacz \\
        Film editing: Agencja Filmowa Film \& Television Production~--
        Zbigniew Masklak \\
        Film editing: IMAVI -- Joanna Kozakiewicz, Krzysztof Magda, Nikodem
        Frodyma


      \end{flushleft}

    \end{textblock}

  \end{frame}





  \begin{frame}[standout]


    \begingroup

    \color{jFrametitleFGColor}

    #1

    \endgroup

  \end{frame}
}











% ---------------------------------------
% Packages, libraries and their configuration
% ---------------------------------------





% ---------------------------------------
% Configuration for this particular presentation
% ---------------------------------------










% ---------------------------------------------------------------------
\title{Computational condensed matter physics with Julia}

\author{Kamil Ziemian \\
  \texttt{kziemianfvt@gmail.com}}


\institute{Uniwersytet Jagielloński w~Krakowie}

\date{16 March 2020}
% ---------------------------------------------------------------------










% ####################################################################
% Początek dokumentu
\begin{document}
% ####################################################################





% Ragged right text with breaking of words

\RaggedRight





% ######################################
\maketitle % Tytuł całego tekstu
% ######################################





% ######################################
\begin{frame}
  \frametitle{Table of contents}


  \tableofcontents % Spis treści

\end{frame}
% ######################################





% ##################
\begin{frame}
  \frametitle{Why this seminar topic was chosen?}


  \begingroup

  \large

  To make since at highest available level, we don't need use cutting edge
  technology. But, we must be aware of its current state.

  \endgroup

\end{frame}
% ##################










% ######################################
\section{Basic information}
% ######################################



% ##################
\begin{frame}
  \frametitle{Julia background information}


  \begin{itemize}
    \RaggedRight

  \item Julia project had started in 2009, first release 0.1 in 2012,
    version 1.0 on 8~August~2018.

  \item Current (10~December~2019) stable version is~1.3.0 (release at
    25 November 2019).

  \item It is \alert{free} and \alert{open software} (GitHub
    \colorhref{https://github.com/JuliaLang/julia}
    {https://github.com/JuliaLang/julia}).

  \item Most of ecosystem if open and free, with \alert{some}
    exception. \\
    Eg.~you must pay for Julia backend to Microsoft Exel.

  \item Created by Jeff Bezanson, Alan Edelman, Stefan Karpinski,
    and~Viral B.~Shah at MIT.

  \item Today one~of main center~of development~of the~language
    is~Julia Lab at~MIT, with Alan Edelman as it current head.

  \item They co-founded Julia Computing \textit{to develop products that
      make Julia easy to~use, easy to~deploy and easy to~scale}.

  \end{itemize}

\end{frame}
% ##################





% ##################
\begin{frame}
  \frametitle{Julia creators}


  \begin{figure}

    \centering

    \includegraphics[scale=0.30]{./PresentationPictures/Julia_creators.png}

  \end{figure}


  \begin{center}

    Stefan Karpinski, Viral B.~Shah, Jeff Bezanson \\
    \hspace{-12em} and Alan Edelman

  \end{center}

\end{frame}
% ##################





% ##################
\begin{frame}
  \frametitle{LLVM and all of that}


  \begin{figure}

    \includegraphics[scale=0.16]{./PresentationPictures/LLVM_logo.png}


    \caption{LLVM logo}

  \end{figure}


  By Source, Fair use, \\
  \colorhref{https://en.wikipedia.org/w/index.php?curid=27377220}
  {https://en.wikipedia.org/w/index.php?curid=27377220}.


  Julia as Apple's Swift (version 1.0 from 2014) and Mozilla Rust
  (version 1.0 from 2015) is build on the top of LLVM, which is a set
  of compilers and toolchains technologies.

\end{frame}
% ##################




% ##################
\begin{frame}
  \frametitle{Crude, oversimplified description of Julia}


  \begingroup

  \large

  90\% of Python look, with 90\% of C/Fortan speed.

  \vspace{1em}


  Julia is as compiled language as~C is, but~it is also dynamic
  language (not interpreted!) in spirit of Python.

  \endgroup

\end{frame}
% ##################





% ##################
\begin{frame}
  \frametitle{Aims of Julia (partialy by Alan Edelman)}


  Language that is
  \begin{itemize}
    \RaggedRight

  \item easy to~use (write and~read);

  \item fast to~write;

  \item open source (science need to be open source!);

  \item high performance (who says dynamic must mean low
    performance?);

  \item easy to parallelize;

  \item both compiled and dynamical (not interpreted out of box)
    in~use, by~REPL (shell) or~notebook;

  \item in most cases almost seamless incorporation code~in FORTRAN,
    C, C++, Python, R, etc.;

  \item give native support for~GPU;

  \item and~easy access to~documentation.

  \end{itemize}

\end{frame}
% ##################





% ##################
\begin{frame}
  \frametitle{There are lies, big lies and benchmarks}

  \vspace{-1em}


  \begin{figure}

    \centering

    \includegraphics[scale=0.29]
    {./PresentationPictures/Julia_micro_benchmarks.png}


    \caption{\textbf{Warning!} Python code use numpy libraries that
      is~written 52.8\% in C (state of GitHub repository at~4
      January~2019)}

  \end{figure}

\end{frame}
% ##################





% ##################
\begin{frame}
  \frametitle{Who use Julia?}


  \begin{figure}

    \centering

    \includegraphics[scale=0.27]
    {./PresentationPictures/Big_players_using_Julia.png}


    \caption{From \colorhref{https://juliacomputing.com/}{Julia
        Computing} page}

  \end{figure}

\end{frame}
% ##################





% ##################
\begin{frame}
  \frametitle{Bad code is~\alert{slow}, good code is~\alert{mostly}
    fast}


  I can't stress this point enough. Difference in~speed~of good
  and~bad Julia code can be~in~\alert{orders~of magnitude}. And~you
  \alert{must} learn Julia way of writing good code.

  So~while you testing how fast is~Julia, make sure that you check
  tips for~high performance.

  I can give example of two code that different by one letter in
  unfortunately place and by factor 40 in speed.

\end{frame}
% ##################





% ##################
\begin{frame}
  \frametitle{What make Julia code good?}


  \begin{itemize}
    \RaggedRight

  \item Write ``\alert{type-stable}'' functions.

  \item Avoid global variables. There are exception to this.

  \item Avoid changing the type of a variable.

  \item Avoid containers with abstract type parameters.

  \item Avoid fields with abstract type.

  \item Access arrays in memory order, along columns.

  \item More dots: Fuse vectorized operations.

  \item \ldots

  \end{itemize}


  More on
  \colorhref{https://docs.julialang.org/en/v1.2/manual/performance-tips/}
  {Julia Documentation: Performance Tips}.

\end{frame}
% ##################





% ##################
\begin{frame}
  \frametitle{Celeste.jl}


  Project written at~National Energy Research Scientific Computing
  Center (NERSC) at Lawrence Berkeley National Laboratory (Berkeley
  Lab). Aim: analyzing data from 35\% of visible sky gathered by~Sloan
  Digital Sky Survey.

  Peak performance of 1.54 petaflops ($10^{ 15 }$ flops) using 1.3
  million threads on 9,300 Knights Landing nodes~of the Cori
  supercomputer at~NERSC. They say that only assembler, C, C++ and
  FORTRAN achieved previously over 1~petaflops performance.

  Loaded approximately 178 terabytes of image data and give
  parameters estimates for 188 millions stars and galaxies
  in~14.6 minutes.

\end{frame}
% ##################





% ##################
\begin{frame}
  \frametitle{Celeste.jl}


  It is worth noting that this computation was performed before 2018,
  so~it used 0.x version of the language (probably 0.4 or~0.5), which
  had unstable syntax.

  For more information see~article: \\
  Jeffrey Regier, et~al., \textit{Cataloging the~Visible Universe
    through Bayesian Inference at~Petascale},
  \colorhref{https://arxiv.org/abs/1801.10277}{arXiv: 1801.10277 [cs.DC]}.

  Celeste.jl GitHub repositor
  \colorhref{https://github.com/jeff-regier/Celeste.jl}
  {https://github.com/jeff-regier/Celeste.jl}.

\end{frame}
% ##################










% ######################################
\section{Live coding in Julia}
% ######################################










% ######################################
\section{DFTK.jl}
% ######################################











% ######################################
\section{Closing remarks}
% ######################################



% ##################
\begin{frame}
  \frametitle{Is someone doing\ldots}


  \begin{itemize}
    \RaggedRight

  \item machine learning:
    \colorhref{https://github.com/FluxML/Flux.jl}{Flux.jl};

  \item deep learning:
    \colorhref{https://github.com/denizyuret/Knet.jl}{Knet.jl};

  \item dynamical systems:
    \colorhref{https://github.com/JuliaDynamics/DynamicalSystems.jl}
    {DynamicalSystems.jl};

  \item optimization:
    \colorhref{https://github.com/JuliaOpt/JuMP.jl}{JuMP.jl};

  \item quantum algorithm design:
    \colorhref{https://github.com/QuantumBFS/Yao.jl}{Yoa.jl};

  \item quantum information:
    \colorhref{https://github.com/iitis/QuantumInformation.jl}
    {QuantumInformation.jl};

  \item biology: \colorhref{https://github.com/BioJulia}{BioJulia};

  \item milk output optimization:
    \colorhref{https://github.com/odow/MOO}{MOO: the Milk Output Optimizer}.

\end{itemize}

  Basically, google what you are interested and check if you find some
  project.

\end{frame}
% ##################





% ##################
\begin{frame}
  \frametitle{Reflections}


  \textit{Julia is still not great when you want import some libraries
    and use them. It~is great when you must write new library.}

  \flushright{Dr Chris Rackauckas, Applied Mathematics Instructor
    at~MIT. \\
    From~Julia Discourse}


  \flushleft{\textbf{My reflections}}

  \begin{itemize}
    \RaggedRight

  \item Julia proved that it~can be~used to~make serious scientific
    work in~various fields, including numerical computations relevant
    to~field theory.

  \item Pedagogical perspective. Learning and~using it~is comparable
    to~Python (in my~opinion), which is~starting language for computer
    scientists on~MIT and~at~the same time is~more deep as~computer
    language and~have more computational power.

  \item Have potential to~make money outside academic world.

  \end{itemize}

\end{frame}
% ##################





% ##################
\begin{frame}
  \frametitle{Remember}


  \alert{If you use it in your research} creators of it ask you for
  citing paper \textbf{Julia: A~Fresh Approach to Numerical
    Computing}, Jeff Bezanson, Alan Edelman, Stefan Karpinski
  and~Viral B.~Shah~(2017) SIAM Review, 59:~65--98 and to add your
  paper to the list
  \colorhref{https://julialang.org/publications/}
  {https://julialang.org/publications/}.

  \vspace{1em}



  \textbf{I want to thank these people for help and discusions}
  \begin{itemize}
    \RaggedRight

  \item Antoine Levitt;

  \item Tamas Papp;

  \item John F. Gibson;

  \item Yakir Luc Gagnon;

  \item Krzysztof Musiał.

  \end{itemize}

  With most of them I converse on Julia Discourse,
  \colorhref{https://discourse.julialang.org/}
  {https://discourse.julialang.org/}.

\end{frame}
% ##################










% ######################################
\section{Additional topics if you are interested}
% ######################################





% ######################################
\appendix
% ######################################





% ##################
\EndingSlide{Thank you. Questions?}
% ##################











% ######################################
\section{Bibliography and~resources}
% ######################################



% ##################
\begin{frame}
  \frametitle{Relevant articles}


  Jeff Bezanson, Alan Edelman, Stefan Karpinski and~Viral B.~Shah,
  \textit{Julia: A~Fresh Approach to Numerical Computing}, (2017) SIAM
  Review, 59:~65--98. doi:~10.1137/141000671,
  url:~http://julialang.org/publications/julia-fresh-approach
  -BEKS.pdf,
  \colorhref{https://julialang.org/publications/julia-fresh-approach-BEKS.pdf}
  {https://julialang.org/publications/julia-fresh-approach-BEKS.pdf}.

  \vspace{0.3em}



  Chistopher Rackauckas and Qing~Nie, \textit{DifferentialEquations.jl
    --~A~Performant and~Feature-Rich Ecosystem for Solving Differential
    Equations in~Julia}. Journal~of Open Research Software (2017),
  5(1), p.~15. Doi: http://doi.org/10.5334/jors.151,
  \colorhref{https://openresearchsoftware.metajnl.com/articles/10.5334/jors.151/}
  {https://openresearchsoftware.metajnl.com/articles/10.5334/jors.151/}.

  \vspace{0.3em}



  Keno Fischer, Elliot Saba, \textit{Automatic Full Compilation~of
    Julia Programs and~ML Models to~Cloud TPUs},
  \colorhref{https://arxiv.org/abs/1810.09868}{arXiv: 1810.09868}.

\end{frame}
% ##################





% ##################
\begin{frame}
  \frametitle{Netography}


  JuliaLang/julia -- Julia GitHub repository,
  \colorhref{https://github.com/JuliaLang/julia}
  {https://github.com/JuliaLang/julia}.

  \vspace{0.3em}



  All code in all languages used for creating benchmarks for
  Kuramoto-Sivashinsky PDE is~available on~GitHub
  \colorhref{https://github.com/johnfgibson/julia-pde-benchmark}
  {johnfgibson/julia-pde-benchmark}. Description of used numerical
  algorithm can be~founded
  \colorhref{https://github.com/johnfgibson/julia-pde-benchmark/blob/master/1-Kuramoto-Sivashinksy-benchmark.ipynb}{here}.

  \vspace{0.3em}



  John F.~Gibson, \textit{Why~Julia}, GitHub
  \colorhref{https://github.com/johnfgibson/whyjulia/blob/master/1-whyjulia.ipynb}{johnfgibson/whyjulia}.

  \vspace{0.3em}



  Alan Edelman homepage \colorhref{http://math.mit.edu/~edelman/}
  {http://math.mit.edu/$\sim$ edelman/}.

\end{frame}
% ##################





% ##################
\begin{frame}
  \frametitle{Netography}


  Picture on slide \textit{Lies, big lies and benchmarks} taken from
  page \textit{Julia Micro-Benchmarks},
  \colorhref{https://julialang.org/benchmarks/}
  {https://julialang.org/benchmarks/}.

  \vspace{0.3em}



  Picture of Julia creators taken from MIT News page: \\
  \colorhref{http://news.mit.edu/2018/julia-language-co-creators-win-james-wilkinson-prize-numerical-software-1226}{http://news.mit.edu/2018/julia-language-co-creators-win-james-wilkinson-prize-numerical-software-1226}.

  \vspace{0.3em}



  Picture of companies using Julia taken from Julia Computing page
  \colorhref{https://juliacomputing.com/}{https://juliacomputing.com/}.

\end{frame}
% ##################





% ##################
\begin{frame}
  \frametitle{Mentioned projects and articles}


  Channelflow (written in~C++),
  \colorhref{http://channelflow.org/}{http://channelflow.org/}.

  \vspace{0.3em}



  Game and educational tool \textit{Paddle Battle}, GitHub
  \colorhref{https://github.com/NHDaly/PaddleBattleJL}
  {NHDaly/PaddleBattleJL}.

\end{frame}
% ##################










% ######################################
\section{Additional information}
% ######################################



% ##################
\begin{frame}
  \frametitle{Some statistics}


  \begin{itemize}
    \RaggedRight

  \item Google Scholar: 1043 papers about Julia or using it in research
    (state at~October~2019). See also topic \emph{Research}
    at~Julia Language page https://julialang.org/publications/,
    \colorhref{https://julialang.org/publications/}
    {https://julialang.org/publications/}.

  \item TIOBE Index for December 2019: 43.

  \item Used in some way in over 1,000 universities.

  \item Ecosystem of over 3,000 packages for wide are of topics.

  \item Over 10~millions downloads.

  \item Over 80,000 GitHub stars for language and packages.

  \item 130\% annual growth (based on~downloads).

  \end{itemize}
  Outside two first positions rest was taken from Julia
  Computing and~dated from September~2019.

\end{frame}
% ##################





% ##################
\begin{frame}
  \frametitle{Julia source code}


  \begin{table}
    \centering

      \begin{tabular}{|l|r|r|}
        \hline
        Language & Percent of code \\
        \hline
        Julia & 68.6\% \\
        \hline
        C & 16.3\% \\
        \hline
        C++ & 9.9\% \\
        \hline
        Scheme & 3.2\% \\
        \hline
        Makefile & 0.7\% \\
        \hline
        LLVM & 0.4\% \\
        \hline
        Other & 0.9\% \\
        \hline
      \end{tabular}

      \caption{Numbers from \colorhref{GitHub
          JuliaLang/julia}{https://github.com/JuliaLang/julia},
        3~October~2019.}

    \end{table}

\end{frame}
% ##################





% ##################
\begin{frame}
  \frametitle{Gamedev in Julia is almost not existing}


  \begin{figure}

    \centering

    \includegraphics[scale=0.17]{./PresentationPictures/Paddle_Battle.png}

  \end{figure}

  Nathan Daly write simple pong game using 96.6\% Julia and 3.5\%
  Shell code and SDL2 library (Simple DirectMedia Layer) to~show that
  if~you want, you~can do it quite easily. After that he put it on~Mac
  App Store with. Code of game is available
  \colorhref{https://github.com/NHDaly/PaddleBattleJL}{here}.

\end{frame}
% ##################










% ######################################
\section{Lies, big lies and~PDE benchmarks}
% ######################################



% ##################
\begin{frame}
  \frametitle{Solving Kuramoto-Sivashinsky PDE
    in~$\MathTextFrametitleFGColor{1 + 1}$ dimensions}


  Following benchmarks were crated by~John F.~Gibson, Dept.
  Mathematics and~Statistics, University~of New Hampshire, main
  author~of \textbf{Channelflow}: \textit{a~set of high-level software
    tools and~libraries for research in~turbulence in~channel
    geometries} written in~C++,
  \colorhref{http://channelflow.org/}{http://channelflow.org/}.
  Full list~of people that contribute to~benchmarks is
  \colorhref{https://github.com/johnfgibson/julia-pde-benchmark/blob/master/1-Kuramoto-Sivashinksy-benchmark.ipynb}{here}.

  They were created in November 2019 using Julia 1.2.

\end{frame}
% ##################





% ##################
\begin{frame}
  \frametitle{Solving Kuramoto-Sivashinsky PDE
    in~$\MathTextFrametitleFGColor{1 + 1}$ dimensions}


  Kuramoto-Sivashinsky equation is nonlinear (more precisely:
  semilinear) partial differential equation, which in $1 + 1$
  dimension takes form

  \begin{equation}
    \label{eq:Julia-Proposition-01}
    \frac{ \partial u( t, x ) }{ \partial t }
    + \frac{ \partial^{ 4 } u( t, x ) }{ \partial x^{ 4 } }
    + \frac{ \partial^{ 2 } u( t, x ) }{ \partial x^{ 2 } }
    + u( t, x ) \frac{ \partial u( t, x ) }{ \partial x }
    = 0.
  \end{equation}


  Full $1 + 3$ dimensional version~of this equation was proposed
  to~\textit{model the~diffusion instabilities in~a~laminar flame
    front}
  (from
  \colorhref{https://en.wikipedia.org/wiki/Kuramoto\%E2\%80\%93Sivashinsky_equation}{Wikipedia}).

\end{frame}
% ##################





% ##################
\begin{frame}
  \frametitle{Solving Kuramoto-Sivashinsky PDE
    in~$\MathTextFrametitleFGColor{1 + 1}$ dimensions}


  \textit{The KS-CNAB2 benchmark algorithm is~a~simple numerical
    integration scheme for~the~KS [Kuramoto-Sivashinsky] equation that
    uses Fourier expansion in~space [All codes call this same Fourier
    Transform external library.], collocation calculation~of
    the~nonlinear term $u( t, x ) u_{ x }( t, x )$,
    and~finite-differencing in~time, specifically 2nd-order
    Crank-Nicolson Adams-Bashforth (CNAB2) timestepping.}

  \flushright{John F. Gibson}

\end{frame}
% ##################





% ##################
\begin{frame}
  \frametitle{Solving Kuramoto-Sivashinsky PDE
    in~$\MathTextFrametitleFGColor{1 + 1}$ dimensions}


  \begin{figure}

    \centering

    \includegraphics[scale=0.22]{./PresentationPictures/KS_result.png}


    \caption{Kuramoto-Sivashinsky heat evolution in~1~dimension}

  \end{figure}

\end{frame}
% ##################





% ##################
\begin{frame}
  \frametitle{Chart with linear scale}


  \begin{figure}

    \centering

    \includegraphics[scale=0.22]
    {./PresentationPictures/JFG_benchmarks_linear_scale.png}


    \caption{Results for $N_{ x }$ points on $x$ axis, CPU time in
      seconds}

  \end{figure}


\end{frame}
% ##################





% ##################
\begin{frame}
  \frametitle{Chart with logarithmic scale}


  \begin{figure}

    \centering

    \includegraphics[scale=0.22]
    {./PresentationPictures/JFG_benchmarks_logarithm_scale_01.png}


    \caption{Results for $N_{ x }$ points on $x$ axis, CPU time in
      seconds}

  \end{figure}

\end{frame}
% ##################





% ##################
\begin{frame}
  \frametitle{Time for maximal
    $\MathTextFrametitleFGColor{N_{ x } = 2^{ 17 } = 131072}$ used
    in~computation}


  \begin{table}

    \centering

    \begin{tabular}{|l|r|r|}
      \hline
      Language & CPU time [s] & Ratio to~C \\
      \hline
      Fortran & 13.0 & 0.90 \\
      \hline
      C++ & 14.4 & 0.99 \\
      \hline
      C & 14.6 & 1.00 \\
      \hline
      Julia unrolled & 15.2 & 1.04 \\
      \hline
      Julia in-place & 15.5 & 1.06 \\
      \hline
      Julia naive & 25.9 & 1.77 \\
      \hline
      MATLAB & 26.8 & 1.83 \\
      \hline
      Python & 35.7 & 2.45 \\
      \hline
    \end{tabular}

    \caption{From the~fastest to~the~slowest code}

  \end{table}

\end{frame}
% ##################





% ##################
\begin{frame}
  \frametitle{Time over~line~of code}


  \begin{figure}

    \centering

    \includegraphics[scale=0.22]
    {./PresentationPictures/JFG_time_over_code.png}


    \caption{Compression~of time and~lines~of code}

  \end{figure}

\end{frame}
% ##################





% ##################
\begin{frame}
  \frametitle{Speed/lines~of code}


  \begin{figure}

    \centering

    \includegraphics[scale=0.22]
    {./PresentationPictures/JFG_speed_over_code_01.png}


    \caption{Compression~of ratio~of speed to the lines~of code}

  \end{figure}

\end{frame}
% ##################










% ######################################
\appendix
% ######################################





% ######################################
\EndingSlide{Thank you! Questions?}
% ######################################










% ####################################################################
% ####################################################################
% Bibliografia
% \bibliographystyle{plalpha}

% \bibliography{}{}





% ############################

% Koniec dokumentu
\end{document}
