\RequirePackage[l2tabu, orthodox]{nag}
\documentclass{beamer} \mode<presentation>
\usepackage[utf8]{inputenc}% Pozwala pisać polskie znaki bezpośrednio.
\usepackage[polish]{babel}
\usepackage[MeX]{polski}
\usepackage{amsfonts, amsmath, amscd}% Czcionki matematyczne od American Mathematic Society.
\usepackage{latexsym}% Więcej symboli.
\usepackage{epsfig}
\usepackage{subfigure}
\usepackage{xcolor}
\usepackage{graphicx}% Pozwala wstawiać grafikę.
\usepackage{hyperref}
\usepackage{verbatim}
\usepackage{xy}% Pozwala rysować grafy.
\usepackage{tikz-feynman}
\usepackage{chemfig}
\usepackage{skak}


\setbeamertemplate{navigation symbols}{}
\setbeamertemplate{headline}{}
\setbeamersize{text margin left=3mm}
\setbeamersize{text margin right=3mm}


\newcommand{\tb}{\textbf}
\newcommand{\tbs}{\textbackslash}


\newcommand{\eps}{\epsilon}
\newcommand{\veps}{\varepsilon}
\newcommand{\si}{\sigma}

\newcommand{\de}{\mathrm{d}}
\newcommand{\D}{\mathcal{D}}
\newcommand{\RdO}{\mathbb{R}^{ d } \setminus \{ 0 \}}
\newcommand{\Rd}{\mathbb{R}^{ d }}
\newcommand{\R}{\mathbb{R}}
\newcommand{\A}{\mathcal{A}}
\newcommand{\dd}[3]{\frac{ \de^{ #1 } #2 }{ \de #3^{ #1 } } }
\newcommand{\sd}{\textrm{sd}}
\newcommand{\Div}{\textrm{div}}



\title[]{Wstęp do \LaTeX} %Spotkanie pierwsze oby nie ostatnie.
\subtitle{O~co w~tym wszystkim chodzi?}
\author{Kamil Ziemian} 
%\\ \texttt{ziemniakzkosmosu@gmail.com}}
%\institute{Uniwersytet Jagielloński w Krakowie}
\date[10 XI 2016]{10 grudnia 2016}



\begin{document}



\begin{frame}
  \frametitle{Kilka rad na~dobry początek}

  \begin{block}{Mądrości programistów}
    Tworzenie tekstu w~\LaTeX u to trochę jak programowanie, więc
    warto pamiętać o~kilku mądrościach programistów.
    \begin{itemize}
    \item[--] Najtrudniejsza rzecz to skompilować pierwszy tekst do
      PDFa. Potem jest już z~górki.
    \item[--] Aby nauczyć~się pisać w~\LaTeX u, trzeba pisać w~\LaTeX
      u.
    \item[--] Twój główny wróg na samym początku to literówki, \\
      np.~\tb{\tbs beign} zamiast \tb{\tbs begin}. Nie zniechęcaj~się
      tym.
    \item[--] To jak ,,kody'' \LaTeX a wygląda ma znaczenie. Pisz go tak, by
      w~jakimś stopniu oddawał logikę tekstu.
    \end{itemize}
  \end{block}

  \begin{block}{Przykład}       % C-x C-+
    Niech będzie nim ten slajd.
  \end{block}
  
\end{frame}



\begin{frame}\frametitle{Kilka rad na~dobry początek}\begin{block}{Mądrości programistów}Tworzenie tekstu w~\LaTeX u to trochę jak programowanie, więc warto pamiętać o~kilku mądrościach programistów.\begin{itemize}\item[--] Najtrudniejsza rzecz to skompilować pierwszy tekst do PDFa. Potem jest już z~górki.\item[--] Aby nauczyć~się pisać w~\LaTeX u, trzeba pisać w~\LaTeX u.\item[--] Twój główny wróg na samym początku to literówki, \\ np.~\tb{\tbs beign} zamiast \tb{\tbs begin}. Nie zniechęcaj~się tym.\item[--] To jak ,,kody'' \LaTeX a wygląda ma znaczenie. Pisz go tak, by w~jakimś stopniu oddawał logikę tekstu. \end{itemize}\end{block}\begin{block}{Przykład}       % C-x C-+
    Niech będzie nim ten slajd.\end{block}\end{frame}



\end{document}