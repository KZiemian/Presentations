% ---------------------------------------------------------------------
% Basic configuration of Beamera and Jagiellonian
% ---------------------------------------------------------------------
\RequirePackage[l2tabu, orthodox]{nag}



\ifx\PresentationStyle\notset
\def\PresentationStyle{dark}
\fi



\documentclass[10pt,t]{beamer}
\mode<presentation>
\usetheme[style=\PresentationStyle,logoColor=monochromaticJUwhite,JUlogotitle=yes]{jagiellonian}



% ---------------------------------------
% Configuration files of Jagiellonian loceted in catalog preambule
% ---------------------------------------
% Configuration for polish language
% Need description
\usepackage[polish]{babel}
% Need description
\usepackage[MeX]{polski}



% ------------------------------
% Better support of polish chars in technical parts of PDF
% ------------------------------
\hypersetup{pdfencoding=auto,psdextra}

% Package "textpos" give as enviroment "textblock" which is very usefull in
% arranging text on slides.

% This is standard configuration of "textpos"
\usepackage[overlay,absolute]{textpos}

% If you need to see bounds of "textblock's" comment line above and uncomment
% one below.

% Caution! When showboxes option is on significant ammunt of space is add
% to the top of textblock and as such, everyting put in them gone down.
% We need to check how to remove this bug.

% \usepackage[showboxes,overlay,absolute]{textpos}



% Setting scale length for package "textpos"
\setlength{\TPHorizModule}{10mm}
\setlength{\TPVertModule}{\TPHorizModule}


% ---------------------------------------
% TikZ
% ---------------------------------------
% Importing TikZ libraries
\usetikzlibrary{arrows.meta}
\usetikzlibrary{positioning}





% % Configuration package "bm" that need for making bold symbols
% \newcommand{\bmmax}{0}
% \newcommand{\hmmax}{0}
% \usepackage{bm}




% ---------------------------------------
% Packages for scientific texts
% ---------------------------------------
% \let\lll\undefined  % Sometimes you must use this line to allow
% "amsmath" package to works with packages with packages for polish
% languge imported
% /preambul/LanguageSettings/JagiellonianPolishLanguageSettings.tex.
% This comments (probably) removes polish letter Ł.
\usepackage{amsmath}  % Packages from American Mathematical Society (AMS)
\usepackage{amssymb}
\usepackage{amscd}
\usepackage{amsthm}
\usepackage{siunitx}  % Package for typsetting SI units.
\usepackage{upgreek}  % Better looking greek letters.
% Example of using upgreek: pi = \uppi


\usepackage{calrsfs}  % Zmienia czcionkę kaligraficzną w \mathcal
% na ładniejszą. Może w innych miejscach robi to samo, ale o tym nic
% nie wiem.










% ---------------------------------------
% Packages written for lectures "Geometria 3D dla twórców gier wideo"
% ---------------------------------------
% \usepackage{./ProgramowanieSymulacjiFizykiPaczki/ProgramowanieSymulacjiFizyki}
% \usepackage{./ProgramowanieSymulacjiFizykiPaczki/ProgramowanieSymulacjiFizykiIndeksy}
% \usepackage{./ProgramowanieSymulacjiFizykiPaczki/ProgramowanieSymulacjiFizykiTikZStyle}





% !!!!!!!!!!!!!!!!!!!!!!!!!!!!!!
% !!!!!!!!!!!!!!!!!!!!!!!!!!!!!!
% EVIL STUFF
\if\JUlogotitle1
\edef\LogoJUPath{LogoJU_\JUlogoLang/LogoJU_\JUlogoShape_\JUlogoColor.pdf}
\titlegraphic{\hfill\includegraphics[scale=0.22]
{./JagiellonianPictures/\LogoJUPath}}
\fi
% ---------------------------------------
% Commands for handling colors
% ---------------------------------------


% Command for setting normal text color for some text in math modestyle
% Text color depend on used style of Jagiellonian

% Beamer version of command
\newcommand{\TextWithNormalTextColor}[1]{%
  {\color{jNormalTextFGColor}
    \setbeamercolor{math text}{fg=jNormalTextFGColor} {#1}}
}

% Article and similar classes version of command
% \newcommand{\TextWithNormalTextColor}[1]{%
%   {\color{jNormalTextsFGColor} {#1}}
% }



% Beamer version of command
\newcommand{\NormalTextInMathMode}[1]{%
  {\color{jNormalTextFGColor}
    \setbeamercolor{math text}{fg=jNormalTextFGColor} \text{#1}}
}


% Article and similar classes version of command
% \newcommand{\NormalTextInMathMode}[1]{%
%   {\color{jNormalTextsFGColor} \text{#1}}
% }




% Command that sets color of some mathematical text to the same color
% that has normal text in header (?)

% Beamer version of the command
\newcommand{\MathTextFrametitleFGColor}[1]{%
  {\color{jFrametitleFGColor}
    \setbeamercolor{math text}{fg=jFrametitleFGColor} #1}
}

% Article and similar classes version of the command
% \newcommand{\MathTextWhiteColor}[1]{{\color{jFrametitleFGColor} #1}}





% Command for setting color of alert text for some text in math modestyle

% Beamer version of the command
\newcommand{\MathTextAlertColor}[1]{%
  {\color{jOrange} \setbeamercolor{math text}{fg=jOrange} #1}
}

% Article and similar classes version of the command
% \newcommand{\MathTextAlertColor}[1]{{\color{jOrange} #1}}





% Command that allow you to sets chosen color as the color of some text into
% math mode. Due to some nuances in the way that Beamer handle colors
% it not work in all cases. We hope that in the future we will improve it.

% Beamer version of the command
\newcommand{\SetMathTextsColor}[2]{%
  {\color{#1} \setbeamercolor{math text}{fg=#1} #2}
}


% Article and similar classes version of the command
% \newcommand{\SetMathTextColor}[2]{{\color{#1} #2}}










% ---------------------------------------
% Commands for setting background pictures for some slides
% ---------------------------------------
\newcommand{\TitleBackgroundPicture}
{./PresentationPictures/CommonPictures/Cute_dragon_BG_dark.png}
\newcommand{\SectionBackgroundPicture}
{./PresentationPictures/CommonPictures/Cute_dragon_small_BG_light.png}



\newcommand{\TitleSlideWithPicture}{
  \begingroup

  \usebackgroundtemplate{ % \hspace*{-11.5em}
    \includegraphics[height=\paperheight]{\TitleBackgroundPicture}}

  \maketitle

  \endgroup
}





\newcommand{\SectionSlideWithPicture}[1]{%
  \begingroup

  \usebackgroundtemplate{ % \hspace*{-11.5em}
    \includegraphics[height=\paperheight]{\SectionBackgroundPicture}}

  \setbeamercolor{titlelike}{fg=normal text.fg}

  \section{#1}

  \endgroup
}





\newcommand{\EndingSlide}[1]{%
  \begin{frame}[standout]

    \begingroup

    \color{jFrametitleFGColor}

    #1

    \endgroup

  \end{frame}
}










% ---------------------------------------
% Packages, libraries and their configuration
% ---------------------------------------





% ---------------------------------------
% Configuration for this particular presentation
% ---------------------------------------
\usetikzlibrary{decorations.markings,decorations.pathmorphing}

\tikzstyle{spring style} = [decoration={aspect=0.7, segment length=1.7mm,
  amplitude=1.3mm,coil},decorate]

\tikzset{
  nasz kwadrat/.pic={
    \draw (-0.5,-0.5) rectangle (0.5,0.5);
  },
}





% ------------------------------
% Ustawienie dla tego konkretnego pliku
% ------------------------------
% \newcounter{}










% ---------------------------------------------------------------------
\title{Wprowadzenie do PGFa i~Ti\textit{k}Za}
\subtitle{Wprowadzenie do używania Ti\textit{k}Za}

\author{Kamil Ziemian}


\date{}
% ---------------------------------------------------------------------










% ####################################################################
% Początek dokumentu
\begin{document}
% ####################################################################





% Wyrównanie do lewej z łamaniem wyrazów

\RaggedRight





% ######################################
\maketitle
% ######################################










% ##################
\begin{frame}
  \frametitle{\textbackslash path \textbackslash draw}


  % #############
  \begin{figure}

    \centering

    \begin{tikzpicture}

      % \path[draw] (0,0) -- (1,0) -- (1,2) -- (3,4);

      % \draw (0,0) -- (0,2) -- (3,0) -- cycle;

    \end{tikzpicture}


    \caption{\textbackslash path, \textbackslash draw}

  \end{figure}
  % #############


  Pamiętajcie o średniku.

\end{frame}
% ##################





% ##################
\begin{frame}
  \frametitle{Strzałki}


  % \tikzstyle{strzalka 1} = [-{Straight Barb[scale=1.2]},line width=1]


  % #############
  \begin{figure}

    \centering

    \begin{tikzpicture}

      % \path (0,0) -- (0,2);



      % \draw[-{Straight Barb[scale=1.2]},line width=1] (0,0) -- (3,1);

      % \draw[strzalka 1] (0,1) -- (2,-1);

      % \draw[strzalka 1] (3,3) -- (0,2);

    \end{tikzpicture}


    \caption{Strzałki}

  \end{figure}
  % #############

\end{frame}
% ##################





% ##################
\begin{frame}
  \frametitle{Dekoracje}


  % #############
  \begin{figure}

    \centering

    \begin{tikzpicture}

      % \path (0,0) -- (0,3);



      % \draw[decoration={coil},decorate] (0,0) -- (3,2);

      % \draw[spring style] (1,0) -- (4,1);

    \end{tikzpicture}


    \caption{Dekoracje}

  \end{figure}
  % #############

\end{frame}
% ##################





% ##################
\begin{frame}
  \frametitle{\textbackslash fill}


  % #############
  \begin{figure}

    \centering

    \begin{tikzpicture}

      % \filldraw[fill=green,line width=1] (0,0) -- (2,0) -- (3,2) -- cycle;
      % \draw[->]
      % \path[fill]
      % \path[fill]

    \end{tikzpicture}

    \caption{\textbackslash fill}

  \end{figure}
  % #############

\end{frame}
% ##################





% ##################
\begin{frame}
  \frametitle{Figury}


  % #############
  \begin{figure}

    \centering

    \begin{tikzpicture}

      % \draw[rotate=30] (0,0) rectangle (3,2);

      % \draw (0,0) circle [radius=1];

      % \draw[rotate=45] (0,0) ellipse (2 and 0.75);

      % \fill[color=red] (0,0) rectangle (3,2);

      % \fill[color=green] (0,0) circle [radius=1];

      % \fill[color=blue] (0,0) ellipse (2 and 0.5);

    \end{tikzpicture}

    \caption{Figury}

  \end{figure}
  % #############

\end{frame}
% ##################





% ##################
\begin{frame}
  \frametitle{Kolory}


  % \definecolor{orange}{rgb}{1,0.5,0}
  % % RGB
  % \definecolor{somecolor}{HTML}{445E8D}
  % \colorlet{lightgray}{black!30}


  % #############
  \begin{figure}

    \centering

    \begin{tikzpicture}

      % \fill[color=red] (0,0) rectangle (3,2);

      % \fill[blue] (0,0) rectangle (3,2);

      % \fill[color=white] (0,0) rectangle (3,2);

      % \fill[color=purple] (0,0) rectangle (3,2);

      % \fill[color=gray] (0,0) rectangle (3,2);

      % \fill[somecolor] (0,0) rectangle (3,2);

      % \fill[color=red!50!blue!50!green] (0,0) rectangle (3,2);

      % \fill[color=lightgray] (0,0) rectangle (3,2);

    \end{tikzpicture}


    \caption{Kolory}

  \end{figure}
  % #############


  Z paczki xcolor. \\
  \textbackslash definecolor\{orange\}\{rgb\}\{1,0.5,0\} \\
  \textbackslash colorlet\{lightgray\}\{black!20\}

\end{frame}
% ##################





% ##################
\begin{frame}
  \frametitle{Zakres (ang. \textit{scope})}


  % #############
  \begin{figure}

    \centering

    \begin{tikzpicture}

      \fill[color=red] (0,0) rectangle (3,2);


      \begin{scope}[xshift=4cm,yshift=-2cm,xscale=0.5]

        \fill[color=blue] (0,0) rectangle (3,2);

      \end{scope}

    \end{tikzpicture}


    \caption{Figury}

  \end{figure}
  % #############

\end{frame}
% ##################





% ##################
\begin{frame}
  \frametitle{Mieszanie kolorów}


  \begin{figure}

    \centering

    \begin{tikzpicture}

      \begin{scope}[blend mode=lighten]
        % \fill[color=black] (-2,-2) rectangle (2,2);

        \fill[red!90!black] ( 90:.6) circle (1);

        \fill[green!80!black] (210:.6) circle (1);

        \fill[blue!90!black] (330:.6) circle (1);
      \end{scope}

    \end{tikzpicture}

    \caption{Mieszanie kolorów}

  \end{figure}

\end{frame}
% ##################






% ##################
\begin{frame}
  \frametitle{Węzły}


  \begin{figure}

    \centering

    \begin{tikzpicture}
      \draw[->,line width=1] (0,0) -- (2,1);

      \node[circle,draw,scale=0.8,text width=1.5cm] at (2.5,0.6) {droga donikąd};

      \node[rectangle,draw=red,scale=0.7] at (0,1) {$\int f( x ) \, dx$};
    \end{tikzpicture}

    \caption{Węzły}

  \end{figure}

\end{frame}
% ##################





% ##################
\begin{frame}
  \frametitle{Prosty diagram}


  \begin{figure}

    \centering

    \begin{tikzpicture}[node distance=0.5cm]
      \node[rectangle,draw] (kod zrodlowy) {Pisze kod źródłowy};

      \node[rectangle,draw,below =of kod zrodlowy] (kompilacja) {Kompiluje};

      \draw[-{Stealth}] (kod zrodlowy) -- (kompilacja);


      \node[rectangle,draw,below right=of kompilacja] (kompilacja udana)
      {Kompilacja udana};

      \draw[-{Stealth}] (kompilacja) -- (kompilacja udana);


      \node[rectangle,draw,below left=of kompilacja] (blad kompilator)
      {Syntax error};

      \draw[-{Stealth}] (kompilacja) -- (blad kompilator);


      \draw[-{Stealth}] (blad kompilator) to [bend left=45] (kod zrodlowy);
    \end{tikzpicture}

    \caption{Prosty diagram}

  \end{figure}

\end{frame}
% ##################





% ##################
\begin{frame}
  \frametitle{Krzywe}


  \begin{figure}

    \centering

    \begin{tikzpicture}
      \draw (0,0) .. controls (0.5,1.5) .. (2,0);

      \fill (0,0) circle [radius=0.1];

      \fill (0.5,1.5) circle [radius=0.1];

      \fill (2,0) circle [radius=0.1];


      \begin{scope}[yshift=-2cm]
        \draw (0,0) .. controls (0.8,1.5) and (2,-1) .. (3,0);

        \fill (0,0) circle [radius=0.1];

        \fill (0.8,1.5) circle [radius=0.1];

        \fill (2,-1) circle [radius=0.1];

        \fill (3,0) circle [radius=0.1];
      \end{scope}
    \end{tikzpicture}

    \caption{Krzywe}

  \end{figure}

\end{frame}
% ##################





% ##################
\begin{frame}
  \frametitle{Pics (od ang. \textit{small PICture})}


  % #############
  \begin{figure}

    \centering

    \begin{tikzpicture}[scale=2]

      % \path (0,0) -- (0,3);



      % \draw[dashed] (0,1) -- (4,1);

      % \draw[rotate=45] (0,0) rectangle (1,1);


      % \pic[rotate=45] at (4,1) {nasz kwadrat};

      % \pic[scale=0.5] at (2,2) {nasz kwadrat};

      % \pic[scale=0.5] at (0,2) {nasz kwadrat};

    \end{tikzpicture}


    \caption{Krzywe}

  \end{figure}
  % #############

\end{frame}
% ##################





% ##################
\begin{frame}
  \frametitle{Pics (od ang. \emph{small PICture})}


  \begin{figure}

    \centering

    \begin{tikzpicture}
      \coordinate (pierwszy punkt kontrolny) at (1.5,1);



      \draw (0,0) .. controls (pierwszy punkt kontrolny) and (2,-1) ..
      (3,0);


      \fill (0,0) circle [radius=0.1];

      \fill (pierwszy punkt kontrolny) circle [radius=0.1];

      \node[above,scale=0.7,text width=3.5cm] at (pierwszy punkt kontrolny)
      {pierwszy punkt kontrolny};


      \fill (2,-1) circle [radius=0.1];

      \fill (3,0) circle [radius=0.1];
    \end{tikzpicture}

    \caption{Krzywe}

  \end{figure}

\end{frame}
% ##################


% % ##################
% \begin{frame}
%   \frametitle{Model światła otoczenia}


%   Kolor światła otoczenia \vspace{-0.8em}
%   \begin{align*}
%     \Ctextscript{amb}
%     =
%     \begin{bmatrix}
%       \Ctextscript[r]{amb} & \Ctextscript[g]{amb} &
%       \Ctextscript[b]{amb}
%     \end{bmatrix} \hspace{30em}
%   \end{align*}

%   \vspace{-1em}


%   Kolor powierzchni w~białym świetle otoczenia \vspace{-0.8em}
%   \begin{align*}
%     \Dtextscript{amb}
%     =
%     \begin{bmatrix}
%       \Dtextscript[r]{amb} & \Dtextscript[g]{amb} &
%       \Dtextscript[b]{amb}
%     \end{bmatrix} \hspace{30em}
%   \end{align*}

%   \vspace{-1em}


%   Obserwowany kolor powierzchni (czyli światło docierające
%   do~obserwatora) \vspace{-0.8em}
%   \begin{align*}
%     \Ktextscript{ambient}
%     =
%     \Dtextscript{amb} \, \Ctextscript{amb}
%     =
%     \begin{bmatrix}
%       \Dtextscript[r]{amb} \, \Ctextscript[r]{amb} &
%       \Dtextscript[g]{amb} \, \Ctextscript[g]{amb} &
%       \Dtextscript[b]{amb} \, \Ctextscript[b]{amb}
%     \end{bmatrix} \hspace{30em}
%   \end{align*}

%   \vspace{-0.5em}


%   Proszę zauważyć, że~w~tym modelu obserwowany kolor powierzchni nie
%   zależy ani od odległości obserwatora od powierzchni, ani~od kąta
%   \red{pod jaki obserwator patrzy na powierzchnię}.

% \end{frame}
% % ##################





% % ##################
% \begin{frame}
%   \frametitle{Odbicie rozproszone (ang.~\emph{diffuse reflection})}


%   \mbox{Odbicie rozproszone jest charakterystyczne dla powierzchni
%     matowych} (zwanych również lambertowskimi), dla których
%   mikronierówności~są większe od~długości padających fal. Światło jest
%   wtedy odbijane losowo we~wszystkich możliwych kierunkach. Takie
%   odbicie nazywamy odbiciem dyfuzyjnym.

%   W~takim modelu \red{kolor światła (było „oglądania”)} odbicia rozproszonego nie zależy
%   od~kąta obserwacji.

%   Albedo $\Dtextscript{diff}$ powierzchni w~odbiciu dyfuzyjnym, to
%   kolor powierzchni oglądany z~bliskiej odległości w~białym świetle
%   padającym prostopadle na~powierzchnię i~odbijającym~się od~niej
%   dyfuzyjnie (kolor dyfuzyjny, ang. \emph{diffuse color}). Jeżeli
%   światło nie pada prostopadle, to obserwator będzie postrzegał inną
%   barwę niż ta, którą ma światło padające. Natomiast postrzegana barwa
%   \red{nigdy} nie zależy od~kierunku obserwacji. Kolor dyfuzyjny
%   materiału jest najbliższy naszej intuicji koloru materiału.

% \end{frame}
% % ##################





% % ##################
% \begin{frame}
%   \frametitle{Prawo Lamberta (prawo kosinusów)}


%   Kolor światła odbitego \vspace{-3em}

%   \begingroup

%   \large

%   \begin{align*}
%     &\Ktextscript{diffuse}
%       = \Dtextscript{diff} \,
%       C \, \max\{ \vecn \cdot \vecl, 0 \}\normaltextcolor{,} \hspace{30em}
%   \end{align*}

%   \endgroup

%   \vspace{-2em}


%   gdzie: \\[0.5em]
%   $\Dtextscript{diff}$~to~dyfuzyjny kolor powierzchni, \\% [1em]
%   $C$~to~kolor światła w~punkcie $\pointP$ powierzchni, \\% [1em]
%   $\vecn$~to~jednostkowy wektor normalny w punkcie $\pointP$ powierzchni,
%   \\% [1em]
%   $\vecl$~to~kierunek do źródła światła.





%   \begin{textblock}{5}(6,6.1)

%     \begin{tikzpicture}
%       \coordinate (P) at (1.2,1.2);

%       \coordinate (A) at (0.3,2.1);

%       \coordinate (B) at (-2,1.2);



%       \fill[color=LightYellow,opacity=0.5] (-2,0.5) -- (0.53,0.5) --
%       (2.03,2) -- (-2,2) -- cycle;

%       \draw[color=yellow,line width=1] (-2,1.2) -- (P);

%       \path (A) -- (P) -- (B) pic [draw,angle radius=33,line width=1,
%       color=jDarkOrange] {angle = A--P--B};

%       \node[scale=1.5] at (0.4,1.55) {$\theta$};


%       \draw[vector,color=jAxisBlue] (P) -- +(-1.75,0);

%       \node[symbol] at (-0.3,1.6) {$\vecl$};


%       \draw[vector] (P) -- +(135:1.75);

%       \node[symbol] at (0.5,2.45) {$\vecn$};


%       \fill[rotate=45] (0,-0.3) rectangle (3.5,0);


%       \pic at (1.2,1.2) {point};

%       \node[symbol] at (1.5,1) {$\Pp$};
%     \end{tikzpicture}

%   \end{textblock}

% \end{frame}
% % ##################





% % ##################
% \begin{frame}[label=Prawo-Lamberta-2]
%   \frametitle{Prawo Lamberta (prawo kosinusów)}


%   \begin{textblock}{2.1}(10,2.2)

%     \hyperlink{Objasnienie-prawo-Lamberta-2}
%     {\beamergotobutton{Objaśnienie symboli}}

%   \end{textblock}


%   \vspace{0.6em}


%   Prawo sformułowane przez Johanna Heinricha Lamberta w~1760~r.

%   \vspace{0.2em}

%   Kolor światła odbitego \vspace{-3em}

%   \begingroup

%   \large

%   \begin{align*}
%     &\Ktextscript{diffuse}
%       =
%       \Dtextscript{diff} \,\, C \, \max\{ \vecn \cdot \vecl, 0 \} =
%       \hspace{30em} \\
%     &\phantom{\Ktextscript{diffuse}}
%       =
%       \Dtextscript{diff} \,\, C \, \max\{ \cos \theta, 0 \}\normaltextcolor{,}
%       \quad
%       \normaltextcolor{\text{{\scalebox{0.86}{ponieważ}}}}
%       \quad
%       \cos \theta = \vecn \cdot \vecl\normaltextcolor{.}
%   \end{align*}

%   \endgroup

%   \vspace{-2.2em}


%   Pole oświetlanej powierzchni jest mnożone przez % czynnik
%   $\frac{ 1 }{ \cos \theta }$. Oznacza to, że~kolor światła jest
%   mnożony przez % czynnik
%   $\cos \theta$. Czynnik $\max\{ \cos \theta, 0 \}$ zeruje przypadek
%   źródła światła znajdującego~się z~tyłu powierzchni.





%   \begin{textblock}{5}(6,6.1)

%     \begin{tikzpicture}
%       \coordinate (P) at (1.2,1.2);

%       \coordinate (A) at (0.3,2.1);

%       \coordinate (B) at (-2,1.2);



%       \fill[color=LightYellow,opacity=0.5] (-2,0.5) -- (0.53,0.5) --
%       (2.03,2) -- (-2,2) -- cycle;

%       \draw[color=yellow,line width=1] (-2,1.2) -- (P);

%       \path (A) -- (P) -- (B) pic [draw,angle radius=33,line width=1]
%       {angle = A--P--B};

%       \node[scale=1.5] at (0.4,1.55) {$\theta$};


%       \draw[vector] (P) -- +(-1.75,0);

%       \node[symbol] at (-0.3,1.6) {$\vecl$};


%       \draw[vector] (P) -- +(135:1.75);

%       \node[symbol] at (0.5,2.45) {$\vecn$};


%       \fill[rotate=45] (0,-0.3) rectangle (3.5,0);


%       \pic at (1.2,1.2) {point};

%       \node[symbol] at (1.5,1) {$\Pp$};
%     \end{tikzpicture}

%   \end{textblock}

% \end{frame}
% % ##################





% % ##################
% \begin{frame}
%   \frametitle{Model światła rozproszonego (dyfuzyjnego)}


%   Kolor światła odbitego w punkcie $\Pp$ powierzchni

%   \vspace{-2em}


%   \begingroup

%   \large

%   \begin{align*}
%     \Ktextscript{diffuse}
%     =
%     \Dtextscript{diff} \sum_{ i = 1 }^{ M } \Cscript{i}
%     \max\{ \vecn \cdot  \veclscript{i}, 0 \}\normaltextcolor{,} \hspace{30em}
%   \end{align*}

%   \endgroup

%   \vspace{-1em}


%   gdzie: \\[0.4em]
%   $\Dtextscript{diff}$~to dyfuzyjny kolor w punkcie $\Pp$ powierzchni,
%   \\[0.3em]

%   $\Cscript{i}$~to kolor światła docierający ze źródła o numerze $i$
%   do punktu $\Pp$, \\[0.3em]

%   $\vecn$~to wektor normalny w punkcie $\Pp$ powierzchni, \\[0.3em]

%   $\displaystyle \veclscript{i} = \frac{ \pointQscript{i} - \Pp }{
%     \norm{ \pointQscript{i} - \Pp } }$
%   to kierunek, w~którym jest punktowe źródło światła \\[0.4em]
%   lub reflektor znajdujący~się w~punkcie $\pointQscript{i}$. Dla
%   źródła światła umieszczonego „w~nieskończoności”
%   $\displaystyle \veclscript{i}$ będzie oznaczać kierunek równoległej
%   wiązki światła.

% \end{frame}
% % ##################





% % ##################
% \begin{frame}
%   \frametitle{Odbicie lustrzane (ang.~\emph{specular reflection})}


%   Jest charakterystyczne dla powierzchni gładkich (lustrzanych),
%   dla~których mikronierówności~są mniejsze od~długości padających fal.
%   Światło jest wtedy odbijane zgodnie z~prawem odbicia. Promień
%   światła odbitego (patrz Wykład~8) wyliczamy ze wzoru:

%   \vspace{-3em}


%   \begingroup

%   \large

%   \begin{align*}
%     &\vecr = 2 ( \vecn \cdot \vecl ) \vecn - \vecl\normaltextcolor{.}
%       \hspace{30em}
%   \end{align*}

%   \endgroup

%   \vspace{-1em}


%   Wiązka światła odbitego skupiona jest wokół kierunku $\vecr$.

% \end{frame}
% % ##################





% % ##################
% \begin{frame}[label=Odbicie-lustrzane]
%   \frametitle{Odbicie lustrzane (ang.~\emph{specular reflection})}


%   \begin{textblock}{2.1}(10,2.9)

%     \hyperlink{Objasnienie-odbicie-lustrzane}
%     {\beamergotobutton{Objaśnienie symboli}}

%   \end{textblock}


%   \vspace{0.6em}


%   Obserwowany kolor światła odbitego tuż przy powierzchni dany jest
%   jako:

%   \vspace{-2.5em}

%   \begingroup

%   \large

%   \begin{align*}
%     &\Ktextscript{specular}
%       =
%       \Dtextscript{spec} \, C \, (\max\{ \vecr \cdot \vecv, 0 \} )^{ m }
%       \theta( \vecn \cdot \vecl )\normaltextcolor{,} \hspace{30em}
%   \end{align*}

%   \endgroup

%   \vspace{-1.75em}

%   gdzie: \\[0.2em]
%   $\Dtextscript{spec}$ to~lustrzany kolor powierzchni
%   (ang.~\emph{specular color}), \\
%   $m$ to~tak zwany wykładnik lustrzany (ang.~\textit{shininess}), \\
%   $\theta( x )$ jest funkcją Heaviside'a (inaczej schodkową, równą $1$
%   dla $x \geq 0$, a~$0$ dla $x < 0$).

%   \vspace{0.3em}


%   Wykładnik lustrzany jest zwykle duży, co~odpowiada temu, że~wiązka
%   odbita jest skupiona wokół kierunku $\vecr$.

%   Odbite światło znika, gdy kierunek obserwacji względem kierunku
%   odbicia tworzy kąt większy niż $90^{ \circ }$.

% \end{frame}
% % ##################





% % ##################
% \begin{frame}
%   \frametitle{Model funkcji rozbłysku Phonga}


%   Słup światła odbitego skupiony jest wokół kierunku $\vecr$, tym
%   bardziej im większe $m$. Kolor światła odbitego od powierzchni
%   obliczamy jako:

%   \vspace{-2.3em}


%   \begingroup

%   \large

%   \begin{align*}
%     &\Ktextscript{specular}
%       =
%       \Dtextscript{spec} \, C \, ( \max\{ \vecr \cdot \vecv, 0 \} )^{ m }
%       \theta( \vecn \cdot \vecl )\normaltextcolor{,} \hspace{30em}
%   \end{align*}

%   \endgroup

%   \vspace{-2.2em}


%   gdzie: \\[0.4em]
%   $\Dtextscript{spec}$ to lustrzany kolor \\
%   powierzchni w~punkcie $\Pp$, \\% [0.3em]
%   $C$ to kolor światła padającego \\
%   w~punkcie $\Pp$, \\% [0.3em]
%   $\vecl$ to kierunek promienia \\
%   padającego w punkcie $\Pp$, \\% [0.3em]
%   $\vecr$ to kierunek promienia \\
%   odbitego w punkcie $\Pp$, \\% [0.3em]
%   $\vecv$ to kierunek obserwacji, \\% [0.3em]
%   $\vecn$ to wektor normalny w~$\Pp$.





%   \begin{textblock}{5}(6.4,5)

%     \begin{tikzpicture}
%       \coordinate (P) at (1.2,1.2);

%       \coordinate (A) at (0.3,2.1);

%       \coordinate (B) at (-2,1.2);

%       \coordinate (C) at (1.2,2.9);



%       \fill[color=LightYellow,opacity=0.5] (-2,0.5) -- (0.53,0.5) --
%       (2.03,2) -- (-2,2) -- cycle;

%       \draw[color=yellow,line width=1] (-2,1.2) -- (P);


%       \path (A) -- (P) -- (B) pic [draw,angle radius=33,line width=1,
%       color=jDarkOrange] {angle = A--P--B};

%       \node[scale=1.5] at (0.4,1.55) {$\theta$};


%       \path (C) -- (P) -- (A) pic [draw,angle radius=33,line width=1,
%       color=jDarkOrange]{angle = C--P--A};

%       \node[scale=1.5] at (0.9,2) {$\theta$};


%       \draw[vector,color=jAxisBlue] (P) -- +(-1.75,0);

%       \node[symbol] at (-0.3,1.6) {$\vecl$};


%       \draw[vector] (P) -- +(135:1.75);

%       \node[symbol] at (0,2.6) {$\vecn$};

%       \draw (P) -- (A);


%       \draw[vector,color=jAxisBlue] (P) -- +(0,1.75);

%       \node[symbol] at (1.5,2.55) {$\vecr$};


%       \draw[line width=0.8] (P) -- +(63:2.7);


%       \draw[vector] (P) -- +(63:1.75);

%       \node[symbol] at (2.15,2.4) {$\vecv$};


%       \fill[rotate=45] (0,-0.3) rectangle (3.5,0);


%       \pic at (1.2,1.2) {point};

%       \node[symbol] at (1.5,1) {$\Pp$};
%     \end{tikzpicture}

%   \end{textblock}

% \end{frame}
% % ##################





% % ##################
% \begin{frame}
%   \frametitle{Model funkcji rozbłysku Blinna}


%   Słup światła odbitego skupiony najsilniej wokół kierunku $\vecr$, co
%   równoważne jest warunkowi, że wektor $\vech$ prawie pokrywa się
%   z~wektorem $\vecn$. Kolor światła odbitego to:

%   \vspace{-2.5em}


%   \begingroup

%   \large

%   \begin{align*}
%     &\Ktextscript{specular}
%       =
%       \Dtextscript{spec} \, C \, ( \max\{ \vecn \cdot \vech, 0 \} )^{ m }
%       \theta( \vecn \cdot \vecl )\normaltextcolor{,} \hspace{30em}
%   \end{align*}

%   \endgroup

%   \vspace{-2em}


%   gdzie: \\% [0.4em]
%   $\Dtextscript{spec}$ to lustrzany kolor   \\
%   powierzchni w~punkcie $\Pp$, \\% [0.3em]
%   $C$ to kolor światła padającego \\
%   w~punkcie $\Pp$, \\% [0.3em]
%   $\vecn$ to wektor normalny, \\% [0.3em]
%   $\displaystyle \vech = \frac{ \vecl + \vecv }{ \normfrac{ \vecl +
%       \vecv } }$ to wektor \\[0.3em]
%   połówkowy (ang. \emph{halfway vector}).

%   \textbf{Uwaga.} Nie jest to model \\
%   równoważny poprzedniemu.





%   \begin{textblock}{5}(6.4,5)

%     \begin{tikzpicture}
%       \coordinate (P) at (1.2,1.2);

%       \coordinate (A) at (0.3,2.1);

%       \coordinate (B) at (-2,1.2);

%       \coordinate (C) at (1.2,2.9);



%       \fill[color=LightYellow,opacity=0.5] (-2,0.5) -- (0.53,0.5) --
%       (2.03,2) -- (-2,2) -- cycle;

%       \draw[color=yellow,line width=1] (-2,1.2) -- (P);


%       \path (A) -- (P) -- (B) pic [draw,angle radius=33,line width=1]
%       {angle = A--P--B};

%       \node[scale=1.5] at (0.4,1.55) {$\theta$};


%       \path (C) -- (P) -- (A) pic [draw,angle radius=33,line width=1]
%       {angle = C--P--A};

%       \node[scale=1.5] at (0.9,2) {$\theta$};




%       \draw[vector,color=jAxisBlue] (P) -- +(-1.75,0);

%       \node[symbol] at (-0.3,1.6) {$\vecl$};


%       \draw[vector] (P) -- +(135:1.75);

%       \node[symbol] at (0,2.6) {$\vecn$};

%       \draw (P) -- (A);


%       \draw[vector,color=jAxisBlue] (P) -- +(0,1.75);

%       \node[symbol] at (1.5,2.55) {$\vecr$};


%       \draw[line width=0.8] (P) -- +(63:2.7);


%       \draw[vector] (P) -- +(63:1.75);

%       \node[symbol] at (2.15,2.4) {$\vecv$};


%       \draw[vector,color=jDarkOrange] (P) -- +(121:1.75);

%       \node[symbol] at (0.7,2.7) {$\vech$};


%       % Dzięki poniższej linii theta jest na wektorze h, a nie pod
%       % nim.
%       \node[scale=1.5] at (0.9,2) {$\theta$};


%       \fill[rotate=45] (0,-0.3) rectangle (3.5,0);


%       \pic at (1.2,1.2) {point};

%       \node[symbol] at (1.5,1) {$\Pp$};
%     \end{tikzpicture}

%   \end{textblock}

% \end{frame}
% % ##################





% % ##################
% \begin{frame}[label=Modele-funkcji-rozblysku-Phonga-i-Blinna]
%   \frametitle{Modele funkcji rozbłysku Phonga i~Blinna}


%   Dla funkcji rozbłysku Phonga liczymy $\vecl$, $\vecr$ i~$\vecv$,
%   przy czym w każdym punkcie powierzchni trzeba te wartości liczyć
%   osobno, $\vecr$ zależy od $\vecl$ i~$\vecn$. Dla funkcji rozbłysku
%   Blinna liczymy $\vecl$ i~$\vecv$, a potem $\vech$. Wyliczenie
%   $\vech$ wymaga normalizacji (czyli m.in. pierwiastkowania), czego
%   nie ma w~rachunku funkcji Phonga, dlatego funkcja Phonga powinna
%   liczyć się szybciej (przy optymalnej implementacji).

%   Ale jeżeli źródło światła i~obserwator są daleko od powierzchni,
%   wtedy wektory $\vecl$ i~$\vecv$ możemy uważać za stałe, czyli wektor
%   $\vech$ też jest stały. Można więc wektor $\vech$ policzyć raz (dla
%   różnych punktów powierzchni lub dla tego samego punktu ale w~różnych
%   klatkach czasowych), wtedy obliczenia z~funkcją Blinna będą szybsze
%   niż z~funkcją Phonga.





%   \begin{textblock}{2.1}(10,7.5)

%     \hyperlink{Objasnienie-modele-funkcji-rozblysku-Phonga-i-Blinna}
%     {\beamergotobutton{Objaśnienie symboli}}

%   \end{textblock}

% \end{frame}
% % ##################





% % ##################
% \begin{frame}
%   \frametitle{Model oświetlenia Blinna-Phonga}


%   % \red{Jedno zdanie wyjaśnienia.}
%   Model oświetlenia Blinna-Phonga zakłada, że odbicie światła od
%   powierzchni modelujemy jako sumę odbicia dyfuzyjnego i lustrzanego
%   (z funkcją rozbłysku Blinna) oraz odbitego światła otoczenia.
%   \begin{align*}
%     K
%     &=
%       \Ktextscript{ambient} + \Ktextscript{diffuse}
%       + \Ktextscript{specular} = \hspace{30em} \\
%     &=
%       \Dtextscript{amb} \, \Ctextscript{amb}
%       + \Dtextscript{diff} \sum_{ i = 1 }^{ M } \Cscript{i}
%       \max\{ \vecn \cdot \veclscript{i}, 0 \} \, + \\[-0.5em]
%     &\,\, + \Dtextscript{spec} \sum_{ i = 1 }^{ M } \Cscript{i} \,
%       ( \max\{ \vecn \cdot \vechscript{i}, 0\} )^{ m }
%       \theta( \vecn \cdot \veclscript{i} ) = \\[-0.5em]
%     &=
%       \Dtextscript{amb} \, \Ctextscript{amb}
%       + \sum_{ i = 1 }^{ M } \Cscript{i} \left( \Dtextscript{diff} \,
%       \vecn \cdot \veclscript{i}
%       + \Dtextscript{spec} \, ( \max\{ \vecn \cdot \vechscript{i} \} )^{ m }
%       \right) \theta( \vecn \cdot \veclscript{i} )
%   \end{align*}

%   Często przyjmuje się $\Dtextscript{amb} = \Dtextscript{diff}$.

% \end{frame}
% % ##################





% % ##################
% \begin{frame}
%   \frametitle{Model oświetlenia Phonga}


%   \begin{textblock}{2.6}(1,1.9)

%     \includegraphics[scale=0.07]{./LecturePictures/wyklad_10/Ambient.png}

%   \end{textblock}


%   \begin{textblock}{2.6}(1,4.8)

%     \centering

%     światło otoczenia

%   \end{textblock}


%   \begin{textblock}{0.4}(3.5,5)

%     \centering

%     \LARGE

%     +

%   \end{textblock}



%   \begin{textblock}{2.6}(3.9,1.9)

%     \includegraphics[scale=0.07]{./LecturePictures/wyklad_10/Diffuse.png}

%   \end{textblock}


%   \begin{textblock}{2.6}(3.9,4.8)

%     \centering

%     odbicie rozproszone

%   \end{textblock}


%   \begin{textblock}{0.4}(6.4,5)

%     \centering

%     \LARGE

%     +

%   \end{textblock}



%   \begin{textblock}{2.6}(6.8,1.9)

%     \includegraphics[scale=0.07]{./LecturePictures/wyklad_10/Specular.png}

%   \end{textblock}


%   \begin{textblock}{2.6}(6.8,4.8)

%     \centering

%     odbicie \\
%     lustrzane

%   \end{textblock}



%   \begin{textblock}{2.6}(9.7,1.9)

%     \includegraphics[scale=0.07]
%     {./LecturePictures/wyklad_10/Phong_calosc.png}

%   \end{textblock}


%   \begin{textblock}{2.6}(9.7,4.8)

%     \centering

%     model \\
%     Phonga

%   \end{textblock}


%   \begin{textblock}{0.4}(9.3,5)

%     \centering

%     \LARGE

%     =

%   \end{textblock}

% \end{frame}
% % ##################





% % ##################
% \begin{frame}
%   \frametitle{Porównanie efektu użycia różnych funkcji rozbłysku}


%   \begin{textblock}{2.6}(1,1.9)

%     \includegraphics[scale=0.07]
%     {./LecturePictures/wyklad_10/Blinn_Phong_exp_64.png}

%   \end{textblock}


%   \begin{textblock}{2.6}(1,4.8)

%     \centering

%     model \\
%     Blinna-Phonga \\
%     $m = 64$

%   \end{textblock}



%   \begin{textblock}{2.6}(3.9,1.9)

%     \includegraphics[scale=0.07]
%     {./LecturePictures/wyklad_10/Phong_exp_64.png}

%   \end{textblock}


%   \begin{textblock}{2.6}(3.9,4.8)

%     \centering

%     model \\
%     Phonga \\
%     $m = 64$

%   \end{textblock}



%   \begin{textblock}{2.6}(6.8,1.9)

%     \includegraphics[scale=0.07]
%     {./LecturePictures/wyklad_10/Blinn_Phong_exp_128.png}

%   \end{textblock}


%   \begin{textblock}{2.6}(6.8,4.8)

%     \centering

%     model \\
%     Blinna-Phonga \\
%     z~wyższym \\
%     wykładnikiem \\
%     $m = 128$

%   \end{textblock}



%   \begin{textblock}{2.6}(9.7,1.9)

%     \includegraphics[scale=0.07]
%     {./LecturePictures/wyklad_10/Blinn_Phong_exp_256.png}

%   \end{textblock}


%   \begin{textblock}{2.6}(9.7,4.8)

%     \centering

%     model \\
%     Blinna-Phonga \\
%     z~wyższym \\
%     wykładnikiem \\
%     $m = 256$

%   \end{textblock}

% \end{frame}
% % ##################










% % ######################################
% \jagielloniansectionwithpicture{Teksturowanie}
% % ######################################



% % ##################
% \begin{frame}
%   \frametitle{Teksturowanie}


%   Jest to technika przedstawienia szczegółów powierzchni obiektów
%   przestrzennych za pomocą obrazów bitmapowych (tekstur) lub funkcji
%   matematycznych (tekstur proceduralnych).

%   Odwzorowanie tekstury (mapa tekstury) określa, w~jaki sposób
%   powiązać piksele (nazywane w tym kontekście tekselami) lub wartości
%   funkcji \red{Jakie funkcji?} z~powierzchnią obiektu. Wartość teksela
%   moduluje (mnoży) kolor powierzchni.

%   Możemy modulować zarówno dyfuzyjny kolor powierzchni (wtedy mówimy
%   o~teksturze koloru $\Ttextscript{color}$), jak i~lustrzany kolor
%   powierzchni (wtedy mówimy o teksturze połysku
%   $\Ttextscript{gloss}$).

% \end{frame}
% % ##################





% % ##################
% \begin{frame}[label=Standardowe-mapy-tekstur]
%   \frametitle{Standardowe mapy tekstur}


%   \begin{textblock}{2.1}(10,1.4)

%     \hyperlink{Objasnienie-standardowe-mapy-tekstur}
%     {\beamergotobutton{Objaśnienie symboli}}

%   \end{textblock}


%   \vspace{0.5em}


%   Jednowymiarowe \\% [0.2em]
%   $s \in [ 0, 1 ]$

%   \vspace{0.5em}


%   Dwuwymiarowe \\% [0.2em]
%   $s, t \in [ 0, 1 ]$

%   \vspace{0.5em}


%   Trójwymiarowe \\% [0.2em]
%   $s, t, p \in [ 0, 1 ]$

%   \vspace{1em}


%   Jeżeli mapy nie są trójwymiarowe, to współrzędna $p$ oraz
%   ewentualnie współrzędna $t$ są równe zero.

% \end{frame}
% % ##################





% % ##################
% \begin{frame}[label=Rzutowe-mapy-tekstur]
%   \frametitle{Rzutowe mapy tekstur}


%   \begin{textblock}{2.1}(10,1.4)

%     \hyperlink{Objasnienie-rzutowe-mapy-tekstur}
%     {\beamergotobutton{Objaśnienie symboli}}

%   \end{textblock}


%   \vspace{0.5em}


%   Przechodzimy do współrzędnych jednorodnych. \red{Do, czy Z
%     współrzędnych jednorodnych do trójwymiarowych?}
%   \begin{align*}
%     \begin{bmatrix}
%       s \\
%       t \\
%       p \\
%       q
%     \end{bmatrix}
%     \longrightarrow
%     \begin{bmatrix}
%       \frac{ s }{ q } \\[0.7em]
%       \frac{ t }{ q } \\[0.7em]
%       \frac{ p }{ q }
%     \end{bmatrix} \hspace{30em}
%   \end{align*}

% \end{frame}
% % ##################





% % ##################
% \begin{frame}
%   \frametitle{Filtrowanie i~mipmapy}


%   Kiedy model obiektu jest renderowany razem z teksturą, to zwykle
%   rozdzielczość mapy tekstury jest inna niż rozdzielczość okna widoku.

%   Jeżeli kamera zbliża~się do~obiektu, to~zwykle dokonujemy
%   \textbf{biliniowego filtrowania}.

%   Jeżeli kamera oddala~się od~obiektu, to~zwykle posługujemy~się
%   \textbf{mipmapami}.

% \end{frame}
% % ##################





% % ##################
% \begin{frame}[label=Biliniowe-filtrowanie]
%   \frametitle{Biliniowe filtrowanie}


%   \begin{textblock}{2.1}(10,2.8)

%     \hyperlink{Objasnienie-biliniowe-filtrowanie}
%     {\beamergotobutton{Objaśnienie symboli}}

%   \end{textblock}


%   \vspace{0.65em}


%   Jeżeli dwuwymiarowa mapa tekstury ma długość $w$ i~wysokość $h$, to
%   dla próbkowanego punktu $( s, t )$ obliczamy:
%   \begin{align*}
%     &w s - \frac{ 1 }{ 2 } = \Big\lfloor ws - \frac{ 1 }{ 2 } \Big\rfloor
%       + \alpha\normaltextcolor{,}
%       \hspace{30em} \\[0.5em]
%     &h t - \frac{ 1 }{ 2 } = \Big\lfloor ht - \frac{ 1 }{ 2 } \Big\rfloor
%       + \beta\normaltextcolor{.}
%   \end{align*}

%   \red{Nie rozumiem tego $T$. KZ}
%   \begin{align*}
%     T
%     =
%     ( 1 - \alpha ) ( 1 - \beta ) \Tscript{i,\,j} + \alpha ( 1 - \beta ) \Tscript{i + 1,\, j}
%     + ( 1 - \alpha ) \beta \Tscript{i,\, j + 1} + \alpha \beta \Tscript{i + 1,\, j + 1}
%     \hspace{30em}
%   \end{align*}

% \end{frame}
% % ##################





% % ##################
% \begin{frame}
%   \frametitle{Mipmapa}


%   Nazwa pochodzi od zwrotu \emph{multum in parvo} (z~łac.~„wiele
%   w~małym”).

%   \mbox{Dla wyjściowej tekstury o~rozmiarze będący potęgą dwójki,
%     czyli $2^{ n } \times 2^{ n }$,} zostanie wygenerowanych $n$
%   mipmap o~rozmiarach $2^{ n - 1 } \times 2^{ n - 1 }$,
%   $2^{ n - 2 } \times 2^{ n - 2 }$, etc., aż do tekstury o rozmiarach
%   $1 \times 1$ pikseli.

%   Zwiększa to zajmowany obszar pamięci tylko o~jedną trzecią, za to
%   znacznie przyspiesza wydajność i~jakość renderowania.

% \end{frame}
% % ##################










% % ######################################
% \jagielloniansectionwithpicture{Model oświetlenia z~teksturami}
% % ######################################



% % ##################
% \begin{frame}
%   \frametitle{Model oświetlenia z~teksturami}


%   Zaczniemy od prostego modulowania kolorów.

%   Rozważamy tekstury koloru $\Ttextscript{color}$ oraz~tekstury
%   połysku $\Ttextscript{gloss}$, które mogą być odpowiednio filtrowane
%   lub mipmapowane.

%   Zakładamy, że mamy zadane pewne odwzorowanie przyporządkowu- \\
%   jące każdemu punktowi powierzchni teksel tekstury z odpowiednim
%   kolorem lub połyskiem.

% \end{frame}
% % ##################





% % ##################
% \begin{frame}[label=Oswietlenie-powierzchni-modyfikowane-teksturami]
%   \frametitle{Oświetlenie powierzchni modyfikowane teksturami}


%   \begin{textblock}{2.1}(10,2.8)

%     \hyperlink{Objasnienie-oswietlenie-powierzchni-modyfikowane-teksturami}
%     {\beamergotobutton{Objaśnienie symboli}}

%   \end{textblock}


%   \vspace{0.6em}


%   Dokonujemy modulacji kolorów używając tekstury koloru i~tekstury
%   połysku.

%   \vspace{-2em}


%   \begingroup

%   \large

%   \begin{align*}
%     &\Ktextscript{ambient}
%       =
%       \Dtextscript{amb} \, \Ttextscript{color} \, \Ctextscript{amb}
%       \hspace{30em} \\
%     &\Ktextscript{diffuse}
%       =
%       \Dtextscript{diff} \, \Ttextscript{color}
%       \sum_{ i = 1 }^{ M } \Cscript{i} \, \max\{ \vecn \cdot \veclscript{i}, 0 \} \\
%     &\Ktextscript{specular}
%       =
%       \Dtextscript{spec} \, \Ttextscript{gloss} \sum_{ i = 1 }^{ M }
%       \Cscript{i} \, ( \max\{ \vecn \cdot \vechscript{i}, 0 \} )^{ m }
%       \theta( \vecn \cdot \veclscript{i} )
%   \end{align*}

%   \endgroup

% \end{frame}
% % ##################





% % ##################
% \begin{frame}
%   \frametitle{Mapy emisyjne}


%   Niektóre obiekty mogą emitować własne słabe światło (np.~świecące
%   okna \red{Nie jestem pewien czy rozumiem. Okno świeci światłem z
%     pomieszczenia?}, dysze silników rakiet, diody \textsc{led} na~stopniach
%   schodów).

%   \vspace{-1em}


%   \begingroup

%   \large

%   \begin{align*}
%     \Ktextscript{emission}
%     =
%     \Ttextscript{em} \, \Ctextscript{em}\normaltextcolor{,} \hspace{30em}
%   \end{align*}

%   \endgroup

%   \vspace{-2.5em}

%   gdzie: \\[0.4em]
%   $\Ttextscript{em}$ to mapa emisji (ang. \emph{emission map}), czyli
%   lokalizacje
%   świecących miejsc na powierzchni, \\[0.3em]

%   $\Ctextscript{em}$ to kolor światła emitowanego przez świecące
%   miejsca powierzchni.

% \end{frame}
% % ##################












% % ######################################
% \jagielloniansectionwithpicture{Modele cieniowania}
% % ######################################



% % ##################
% \begin{frame}
%   \frametitle{Modele cieniowania}


%   Informacja o~powierzchni modelu (współrzędne punktów powierzchni,
%   wektory normalne) przechowywana jest zwykle w~wierzchołkach siatki
%   trójkątów.

%   Podczas renderowania pojedynczego trójkąta informacje dla punktów
%   wnętrza trójkąta są interpolowane z~danych z~wierzchołków trójkąta.

%   Metoda wyznaczania oświetlenia i~koloru światła odbitego dla wnętrza
%   trójkąta połączona z~odwzorowaniami zbioru tekstur nazywa~się
%   \alert{cieniowaniem} (ang.~\emph{shading}).

% \end{frame}
% % ##################





% % ##################
% \begin{frame}
%   \frametitle{Wyliczanie wektorów normalnych}


%   Zastosowanie wybranego modelu oświetlenia do~siatki trójkątów wymaga
%   reprezentacji wektorów normalnych do~powierzchni w~każdym
%   wierzchołku.

%   Numerujemy wierzchołki trójkąta w kolejności przeciwnej do~wskazówek
%   zegara oraz wyliczamy wektor normalny do płaszczyzny trójkąta.

%   \vspace{-1em}


%   \begingroup

%   \large

%   \begin{align*}
%     \vecn
%     =
%     \frac{ ( \pointPscript{2} - \pointPscript{1} )
%     \times ( \pointPscript{3} - \pointPscript{1} ) }
%     { \normfrac{ ( \pointPscript{2} - \pointPscript{1} )
%     \times ( \pointPscript{3} - \pointPscript{1} ) } }
%     \hspace{30em}
%   \end{align*}

%   \endgroup





%   \begin{textblock}{3.1}(7.5,4.5)

%     \begin{tikzpicture}
%       \coordinate (P 1) at (0,0,0);

%       \coordinate (P 2) at (1,0,-3);

%       \coordinate (P 3) at (0,3,0);



%       \fill[color=gray] (P 1) -- (P 2) -- (P 3) -- cycle;


%       \pic at (P 1) {point};

%       \node[symbol,below] at (P 1) {$\pointPscript{1}$};


%       \pic at (P 2) {point};

%       \node[symbol,right] at (P 2) {$\pointPscript{2}$};


%       \pic at (P 3) {point};

%       \node[symbol,above] at (P 3) {$\pointPscript{3}$};



%       \draw[vector,color=jDarkOrange] (0.5,2) -- +(40:1.4);

%       \node[symbol] at (1.2,3.05) {$\vecn$};
%     \end{tikzpicture}

%   \end{textblock}

% \end{frame}
% % ##################





% % ##################
% \begin{frame}
%   \frametitle{Wyliczanie wektorów normalnych}


%   Wektor normalny w~konkretnym wierzchołku siatki trójkątów możemy
%   wyliczyć uśredniając wektory normalne wszystkich trójkątów siatki
%   zawierających ten wierzchołek.

%   \begingroup

%   \Large

%   \begin{align*}
%     \vecn
%     =
%     \frac{ \sum_{ i = 1 }^{ M } \vecnscript{i} }
%     { \normfrac{ \sum_{ i = 1 }^{ M } \vecnscript{i} } } \hspace{30em}
%   \end{align*}

%   \endgroup





%   \begin{textblock}{4.5}(6.2,4)

%     \begin{tikzpicture}
%       \draw[line width=1] (-1.6,1.8) -- (0,0);

%       \draw[line width=1] (2.3,2.2) -- (0,0);

%       \draw[line width=1] (1.9,-2.5) -- (0,0);

%       \draw[line width=1] (-1.4,-2) -- (0,0);

%       \draw[line width=1] (-1.6,1.8) -- (2.3,2.2) -- (1.9,-2.5) --
%       (-1.4,-2) -- cycle;


%       \pic at (0,0) {point};


%       \node[symbol] at (-1,0) {$\vecnscript{1}$};

%       \node[symbol] at (0,1) {$\vecnscript{2}$};

%       \node[symbol] at (1,0) {$\vecnscript{3}$};

%       \node[symbol] at (0,-1) {$\vecnscript{4}$};
%     \end{tikzpicture}

%   \end{textblock}

% \end{frame}
% % ##################





% % ##################
% \begin{frame}
%   \frametitle{Wyliczanie wektorów normalnych --~inna metoda}


%   Inna metoda wyliczania wektora normalnego to uśrednianie
%   nieznormalizowanych wektorów normalnych do~trójkątów przyległych do
%   wierzchołka. Wtedy wyższą wagę w~uśrednianiu mają wektory normalne
%   trójkątów o~większym polu.

%   \vspace{-1.4em}


%   \begingroup

%   \Large

%   \begin{align*}
%     \vecn
%     =
%     \frac{ \sum_{ i = 1 }^{ M } \vecNscript{i} }
%     { \normfrac{ \sum_{ i = 1 }^{ M } \vecNscript{i}} } \hspace{30em}
%   \end{align*}

%   \endgroup





%   \begin{textblock}{4.5}(6.2,4)

%     \begin{tikzpicture}
%       \draw[line width=1] (-1.6,1.8) -- (0,0);

%       \draw[line width=1] (2.3,2.2) -- (0,0);

%       \draw[line width=1] (1.9,-2.5) -- (0,0);

%       \draw[line width=1] (-1.4,-2) -- (0,0);

%       \draw[line width=1] (-1.6,1.8) -- (2.3,2.2) -- (1.9,-2.5) --
%       (-1.4,-2) -- cycle;


%       \pic at (0,0) {point};


%       \node[symbol] at (-0.9,0) {$\vecNscript{1}$};

%       \node[symbol] at (0,1) {$\vecNscript{2}$};

%       \node[symbol] at (1,0) {$\vecNscript{3}$};

%       \node[symbol] at (0,-1) {$\vecNscript{4}$};
%     \end{tikzpicture}

%   \end{textblock}

% \end{frame}
% % ##################





% % ##################
% \begin{frame}[label=Cieniowanie-Gourauda]
%   \frametitle{Cieniowanie Gourauda}


%   \begin{textblock}{2.1}(10,4.2)

%     \hyperlink{Objasnienie-cieniowanie-Gourauda}
%     {\beamergotobutton{Objaśnienie symboli}}

%   \end{textblock}


%   \vspace{0.65em}


%   \red{Równoważnik zdania/zadania?} Interpolacja wartości oświetlenia
%   wewnątrz trójkąta na~podstawie danych z~wierzchołków trójkąta, czyli
%   inaczej cieniowanie na zasadzie interpolowania jasności. W
%   literaturze używanych jest wiele równoważnych terminów:
%   \emph{Gouraud shading, per-vertex shading, vertex shading, vertex
%     lighting, interpolated shading.}

%   \vspace{0.8em}


%   Podstawowy kolor wierzchołka \vspace{-2em} \begingroup \large

%   \begin{align*}
%     &\Ktextscript{primary}
%       = \Ctextscript{em} + \Dtextscript{amb} \, \Ctextscript{amb}
%       + \Dtextscript{diff} \sum_{ i = 1 }^{ M } \Cscript{i} \,
%       \max\{ \vecn \cdot \veclscript{i}, 0 \} \hspace{30em}
%   \end{align*}

%   \endgroup


%   Poboczny kolor wierzchołka

%   \vspace{-2em}

%   \begingroup

%   \large

%   \begin{align*}
%     &\Ktextscript{secondary}
%       =
%       \Dtextscript{spec}
%       \sum_{ i = 1 }^{ M } \Cscript{i} \,
%       ( \max\{ \vecn \cdot \vechscript{ i }, 0 \} )^{ m }
%       \theta( \vecn \cdot \veclscript{i} ) \hspace{30em}
%   \end{align*}

%   \endgroup

% \end{frame}
% % ##################





% % ##################
% \begin{frame}[label=Przyklad-Szader-wierzcholkow-dla-cieniowania-Gourauda-1]
%   \frametitle{Przykład. Szader wierzchołków dla cieniowania Gourauda}


%   \textbf{Dane wejściowe} -- współrzędne wierzchołka $\Pp$
%   w~przestrzeni obiektu, wektor normalny $\vecn$ do powierzchni
%   w~punkcie $\Pp$ w~przestrzeni obiektu, współrzędne tekstur koloru
%   i~połysku $( u, v )$.

%   \vspace{0.6em}


%   \textbf{Dane wyjściowe} -- współrzędne wierzchołka $\Phom$
%   w~przestrzeni jednorodnej przyciętej, współrzędne tekstur koloru
%   i~połysku $( u, v )$, podstawowy kolor wierzchołka
%   $\Ktextscript{primary}$, poboczny kolor
%   wierzchołka~$\Ktextscript{secondary}$.

%   \vspace{0.6em}


%   \textbf{Zmienne globalne (ang. \emph{uniforms}):} macierz
%   transformacji z~przestrzeni obiektu do jednorodnej przestrzeni
%   przyciętej $\homM_{MVP}$ \red{Czy indeksy mają być pomidorowe?}
%   (ang. \emph{model-view-projection}), \red{Wojtek Palacz się pytał,
%     czy M to trójka liczb? Tak, to trójka liczb.}kolor światła
%   otoczenia $\Ctextscript{amb}$, kolor światła punktowego
%   $\Cscript{0}$, pozycja światła punktowego $\Qp$ w~przestrzeni
%   obiektu, współczynniki atenuacji (osłabienia) światła punktowego
%   $\kscript{c}$, $\kscript{l}$, $\kscript{q}$, wykładnik lustrzany
%   $m$, pozycja obserwatora $\pointPscript{O}$ w~przestrzeni obiektu.

% \end{frame}
% % ##################





% % ##################
% \begin{frame}
%   \frametitle{Przykład. Szader wierzchołków dla cieniowania Gourauda}


%   Kolejne kroki: \vspace{-1em}
%   \begin{itemize}
%   \item transformacja wierzchołka: $\Phom = \homM_{MVP}
%     \begin{pmatrix}
%       \Pp \\
%       1
%     \end{pmatrix}$,

%   \item obliczenie odległości: $d = \norm{ \Qp - \Pp }$ i~kierunku:
%     $\vecl = { \displaystyle \frac{ \Qp - \Pp }{ d } }$ do~punktowego
%     źródła światła,

%   \item obliczenie kierunku do obserwatora:
%     $\vecv = { \displaystyle \frac{ \pointPscript{O} - \Pp }{
%         \normfrac{ \pointPscript{O} - \Pp } } }$,

%     \vspace{0.6em}


%   \item obliczenie wektora połówkowego:
%     $\vech = { \displaystyle \frac{ \vecl + \vecv }{ \normfrac{ \vecl
%           + \vecv } } }$,

%     \vspace{0.5em}


%   \item obliczenie osłabienia światła punktowego:
%     $C = { \displaystyle \frac{ \Cscript{0} }{ \kscript{c} +
%         \kscript{l} d + \kscript{q} d^{ 2 } } }$,

%     \vspace{0.5em}


%   \item obliczenie podstawowego i~pochodnego koloru wierzchołka:
%     $\Ktextscript{primary} = \Ctextscript{amb} + C \, \max\{ \vecn
%     \cdot \vecl, 0 \}$, \\[0.5em]
%     $\Ktextscript{secondary} = C \, ( \max\{ \vecn \cdot \vech, 0 \} )^{ m
%     } \theta( \vecn \cdot \vecl )$.
%   \end{itemize}

% \end{frame}
% % ##################





% % ##################
% \begin{frame}
%   \frametitle{Przykład. Szader fragmentów dla cieniowania Gourauda}


%   \textbf{Interpolowane dane wejściowe} -- współrzędne tekstur koloru
%   i~połysku $(u,v)$, podstawowy kolor piksela:
%   $\Ktextscript{primary}$, poboczny~kolor piksela:
%   $\Ktextscript{secondary}$.

%   \vspace{1em}


%   \textbf{Dane wyjściowe} -- kolor piksela $K$.

%   \vspace{1em}


%   \textbf{Dwuwymiarowe mapy tekstur (ang. \emph{2D map samplers})} --
%   tekstura koloru $\Ttextscript{color}( u, v )$, tekstura połysku
%   $\Ttextscript{gloss}( u, v )$.

% \end{frame}
% % ##################





% % ##################
% \begin{frame}
%   \frametitle{Przykład. Szader fragmentów dla cieniowania Gourauda}


%   Kolejne kroki: \vspace{-0.5em}
%   \begin{itemize}
%     \setlength{\itemsep}{0.7em}

%   \item pobranie teksela koloru:
%     $\Ttextscript{color} = \Ttextscript{color}( u, v )$,

%   \item pobranie teksela połysku:
%     $\Ttextscript{gloss} = \Ttextscript{gloss}( u, v )$,

%   \item obliczenie koloru piksela:
%     $K = \Ttextscript{color} \, \Ktextscript{primary} +
%     \Ttextscript{gloss} \, \Ktextscript{secondary}$.
%   \end{itemize}

% \end{frame}
% % ##################





% % ##################
% \begin{frame}
%   \frametitle{Cieniowanie Phonga}


%   Zamiast interpolacji wartości oświetlenia wewnątrz trójkąta,
%   wykonujemy interpolowanie wektorów normalnych, kierunków źródeł
%   światła i~kierunków obserwacji. W~literaturze używane są równoważne
%   terminy: \emph{per-pixel shading}, \emph{pixel shading},
%   \emph{fragment shading}.

%   Cieniowanie Phonga daje lepsze rezultaty niż cieniowanie Gourauda
%   z~powodu lepszego modelowania odbicia lustrzanego. Przykładowo,
%   jeżeli światło pada na środek trójkąta, cieniowanie Gourauda daje
%   gorsze rezultaty, gdyż interpolowane wartości funkcji rozbłysków
%   z~wierzchołków trójkąta nie są reprezentatywne dla środka trójkąta.
%   Natomiast w~cieniowaniu Phonga funkcja rozbłysku wyliczana jest
%   osobno dla każdego piksela wewnątrz trójkąta.

% \end{frame}
% % ##################





% % ##################
% \begin{frame}[label=Standardowe-rownanie-cieniowania-1]
%   \frametitle{Standardowe równanie cieniowania}


%   \begin{textblock}{2.1}(10,1.4)

%     \hyperlink{Objasnienie-standardowe-rownanie-cieniowania-1}
%     {\beamergotobutton{Objaśnienie symboli}}

%   \end{textblock}


%   % \vspace{-2em}


%   \begingroup

%   \large

%   \begin{align*}
%     &K = \Ktextscript{emission} + \Ktextscript{ambient}
%       + \Ktextscript{diffuse}
%       + \Ktextscript{specular} = \hspace{30em} \\
%     &= \Ttextscript{em} \, \Ctextscript{em}
%       + \Dtextscript{amb} \, \Ttextscript{color} \, \Ctextscript{amb} + \\
%     &+ \sum_{ i = 1 }^{ M } \Cscript{i} \, \big[ \Dtextscript{diff} \,
%       \Ttextscript{color} \, \vecn \cdot \veclscript{i}
%       + \Dtextscript{spec} \, \Ttextscript{spec}
%       ( \max\{ \vecn \cdot \vechscript{i}, 0 \} )^{ m } \big]
%       \theta( \vecn \cdot \veclscript{i} )
%   \end{align*}

%   \endgroup

%   \vspace{-1em}


%   gdzie: \\[0.4em]
%   $\Ctextscript{em}$ i~$\Ctextscript{amb}$ to~kolory światła
%   emitowanego i~światła otoczenia, \\[0.3em]

%   $\Ttextscript{em}$, $\Ttextscript{color}$ i~$\Ttextscript{gloss}$
%   to~mapy światła emitowanego, koloru i~połysku, \\[0.3em]

%   $\Dtextscript{diff}$ i~$\Dtextscript{spec}$ to~kolory dyfuzyjny
%   i~lustrzany punktu powierzchni, \\[0.3em]

%   $\Cscript{i}$ to~kolor światła z~$i$-tego źródła padającego na~punkt
%   powierzchni, \\[0.3em]

%   $\vecn$ to wektor prostopadły do powierzchni w~danym punkcie.

% \end{frame}
% % ##################





% % ##################
% \begin{frame}[label=Standardowe-rownanie-cieniowania-2]
%   \frametitle{Standardowe równanie cieniowania}


%   \begin{textblock}{2.1}(10,1.4)

%     \hyperlink{Objasnienie-standardowe-rownanie-cieniowania-2}
%     {\beamergotobutton{Objaśnienie symboli}}

%   \end{textblock}

%   % \vspace{-2em}


%   \begingroup

%   \large

%   \begin{align*}
%     &K = \Ktextscript{emission} + \Ktextscript{ambient}
%       + \Ktextscript{diffuse} + \Ktextscript{specular} = \hspace{30em} \\
%     &=
%       \Ttextscript{em} \, \Ctextscript{em} + \Dtextscript{amb} \,
%       \Ttextscript{color} \, \Ctextscript{amb} + \\
%     &+ \sum_{ i = 1 }^{ M } \Cscript{i} \, \big[ \Dtextscript{diff} \,
%       \Ttextscript{color} \, \vecn \cdot \veclscript{i}
%       + \Dtextscript{spec} \, \Ttextscript{spec}
%       ( \max\{ \vecn \cdot \vechscript{i}, 0 \} )^{ m } \big]
%       \theta( \vecn \cdot \veclscript{i} )
%   \end{align*}

%   \endgroup

%   \vspace{-1em}


%   gdzie: \\[0.4em]
%   $\veclscript{i}$ to wektor w kierunku $i$-tego źródła światła, \\[0.3em]

%   $\vecv$ to wektor w kierunku punktu obserwatora, \\[0.3em]

%   $\vechscript{i} = { \displaystyle \frac{ \veclscript{i} + \vecv }{ \norm{ \veclscript{i} + \vecv } } }$ to wektor połówkowy, \\[0.5em]

%   $m$ to~wykładnik lustrzany.

% \end{frame}
% % ##################





% % ##################
% \begin{frame}
%   \frametitle{Przykład. Szader wierzchołków dla cieniowania Phonga}


%   \textbf{Dane wejściowe} -- współrzędne wierzchołka $\Pp$ w przestrzeni
%   obiektu, wektor normalny $\vecn$ do powierzchni w punkcie $\Pp$
%   w~przestrzeni obiektu, współrzędne tekstur koloru i połysku
%   $( u, v )$.

%   \vspace{1em}


%   \textbf{Dane wyjściowe} -- współrzędne wierzchołka $\Pp$ w przestrzeni
%   obiektu, współrzędne wierzchołka $\Phom$ w przestrzeni jednorodnej
%   przyciętej, wektor normalny $\vecn$ w przestrzeni obiektu,
%   współrzędne tekstur koloru i połysku $( u, v )$.

%   \vspace{1em}


%   \textbf{Zmienne globalne (ang. \emph{uniforms})} -- macierz
%   transformacji z~przestrzeni obiektu do jednorodnej przestrzeni
%   przyciętej $\homM_{MVP}$ (ang.
%   \emph{model-view-projection}). % pomidor

% \end{frame}
% % ##################





% % ##################
% \begin{frame}
%   \frametitle{Przykład. Szader wierzchołków dla cieniowania Phonga}


%   Pierwszy krok: \vspace{-1.2em}
%   \begin{itemize}
%   \item transformacja wierzchołka: $\pointPhom = \homM_{MVP} % pomidor
%     \begin{pmatrix}
%       \pointP \\
%       1
%     \end{pmatrix}$.

%   \end{itemize}

% \end{frame}
% % ##################





% % ##################
% \begin{frame}
%   \frametitle{Przykład. Szader fragmentów dla cieniowania Phonga}


%   \textbf{Interpolowane dane wejściowe} -- współrzędne punktu $\Pp$
%   w~przestrzeni obiektu, interpolowany (nieznormalizowany) wektor
%   normalny $\vecN$ w~przestrzeni obiektu, współrzędne tekstur koloru
%   i~połysku $( u, v )$.

%   \vspace{1em}


%   \textbf{Dane wyjściowe} -- kolor piksela $K$.

%   \vspace{1em}


%   \textbf{Dwuwymiarowe mapy tekstur (ang. \emph{2D map samplers})} --
%   tekstura~koloru $\Ttextscript{color}( u, v )$, tekstura połysku
%   $\Ttextscript{gloss}( u, v )$.

%   \vspace{1em}


%   \textbf{Zmienne globalne (ang. \emph{uniforms})} -- kolor światła
%   otoczenia $\Ctextscript{amb}$, kolor światła punktowego
%   $\Cscript{0}$, pozycja światła punktowego $\Qp$ w~przestrzeni
%   obiektu, współczynniki atenuacji (osłabienia) światła punktowego
%   $\kscript{c}$, $\kscript{l}$, $\kscript{q}$, wykładnik lustrzany
%   $m$, pozycja obserwatora $\pointPscript{O}$ w~przestrzeni obiektu.

% \end{frame}
% % ##################





% % ##################
% \begin{frame}[label=Przyklad-Szader-fragmentow-dla-cieniowania-Phonga]
%   \frametitle{Przykład. Szader fragmentów dla cieniowania Phonga}


%   \begin{textblock}{2.1}(10,1.4)

%     \hyperlink{Objasnienie-przyklad-Szader-fragmentow-dla-cieniowania-Phonga}
%     {\beamergotobutton{Objaśnienie symboli}}

%   \end{textblock}


%   \vspace{0.65em}


%   Kolejne kroki: \vspace{-0.4em}
%   \begin{itemize}
%   \item pobranie teksela koloru:
%     $\Ttextscript{color} = \Ttextscript{color}( u, v )$,

%   \item pobranie teksela połysku:
%     $\Ttextscript{gloss} = \Ttextscript{gloss}( u, v )$,

%   \item normalizacja interpolowanego wektora normalnego:
%     $\vecn = { \displaystyle \frac{ \vecN }{ \norm{ \vecN } } }$,

%   \item obliczenie odległości: $d = \norm{ \Qp - \Pp }$, i~kierunku
%     do~punktowego źródła światła
%     $\vecl = { \displaystyle \frac{ \Qp - \Pp }{ d } }$,

%   \item obliczenie kierunku do obserwatora:
%     $\vecv = { \displaystyle \frac{ \pointPscript{O} - \Pp } { \norm{
%           \pointPscript{O} - \Pp } } }$,

%     \vspace{0.8em}


%   \item obliczenie wektora połówkowego:
%     $\vech = { \displaystyle \frac{ \vecl + \vecv }{ \norm{ \vecl +
%           \vecv } } }$,

%     \vspace{0.5em}


%   \item obliczenie osłabienia światła punktowego:
%     $C = { \displaystyle \frac{ \Cscript{0} }{ \kscript{c} +
%         \kscript{l} d + \kscript{q} d^{ 2 } } }$.

%   \end{itemize}

% \end{frame}
% % ##################






% % ##################
% \begin{frame}
%   \frametitle{Przykład. Szader fragmentów dla cieniowania Phonga}


%   Kolejne kroki: \vspace{-0.45em}
%   \begin{itemize}
%   \item obliczenie koloru wierzchołka:
%     $K = \Ttextscript{color} \, \Ctextscript{amb} + C \, \big[
%     \Ttextscript{color} \, \vecn \cdot \veclscript{i} +
%     \Ttextscript{spec} \, ( \max\{ \vecn \cdot \vechscript{i}, 0 \}
%     )^{ m } \big] \theta( \vecn \cdot \veclscript{i} )$.
%   \end{itemize}

% \end{frame}
% % ##################










% % ######################################
% \jagielloniansectionwithpicture{Mapowanie wypukłości}
% % ######################################



% % ##################
% \begin{frame}[label=Mapowanie-wypuklosci]
%   \frametitle{Mapowanie wypukłości (ang. \emph{bump mapping})}


%   \begin{textblock}{2.1}(10,8.05)

%     \hyperlink{Objasnienie-mapowanie-wypuklosci}
%     {\beamergotobutton{Objaśnienie symboli}}

%   \end{textblock}


%   \vspace{0.6em}


%   Jest to technika bazująca na iluzji większej szczegółowości mapy
%   tekstury wywoływanej perturbacjami wektora normalnego w~każdym
%   pikselu.

%   Informacja dotycząca zaburzenia wektora normalnego przechowywana
%   jest w dwuwymiarowej macierzy trójwymiarowych wektorów, „teksturze
%   wektorów normalnych”, nazywanej mapą wypukłości (ang. \emph{bump
%     map}, \emph{normal map}). Informacja dotycząca zaburzenia wektora
%   normalnego reprezentowana jest jako „kolor”, czyli np.~trójka liczb
%   z przedziału $[ 0, 1 ]$.

%   Mapę (teksturę) wypukłości będziemy oznaczać
%   $\Ttextscript{bump}( u, v )$.

%   $[ 0 \quad 0 \quad 1 ]^{ \, T }$ reprezentuje
%   niezaburzony wektor normalny. Wektory normalne możemy odczytywać z
%   mapy wypukłości jako:
%   \begin{equation*}
%     \vecn\hspace{0.07em}'^{ \, T }
%     =
%     2 \Ttextscript{bump}( u, v )
%     -
%     \begin{bmatrix}
%       1 & 1 & 1
%     \end{bmatrix}\normaltextcolor{.} \hspace{30em}
%   \end{equation*}

% \end{frame}
% % ##################





% % ##################
% \begin{frame}
%   \frametitle{Mapy wysokości (ang. \emph{height map})}


%   \mbox{Do konstrukcji mapy wypukłości można użyć tzw.~„mapy
%     wysokości” –} informacji o~„wysokościach” pikseli powierzchni:

%   \vspace{-3em}

%   \begingroup

%   \large

%   \begin{align*}
%     H( i, j )\normaltextcolor{.} \hspace{30em}
%   \end{align*}

%   \endgroup

%   \vspace{-1em}


%   $H$ reprezentuje wysokość punktu powierzchni dla piksela $( i, j )$
%   -- „pikselowe” funkcyjne równanie powierzchni.

%   Wektory styczne do~powierzchni i~wektory normalne mogą być wyliczone
%   z~mapy wysokości.

% \end{frame}
% % ##################





% % ##################
% \begin{frame}[label=Wektory-styczne]
%   \frametitle{Wektory styczne}


%   \begin{textblock}{2.1}(10,1.4)

%     \hyperlink{Objasnienie-wektory-styczne}
%     {\beamergotobutton{Objaśnienie symboli}}

%   \end{textblock}


%   \vspace{0.3em}


%   \begin{align*}
%     &\vecS( i, j )
%       =
%       \begin{bmatrix}
%         1 \\
%         0 \\
%         a H( i + 1, j ) - a H( i - 1, j )
%       \end{bmatrix} \\[0.5em]
%     &\vecT( i, j ) =
%       \begin{bmatrix}
%         0 \\
%         1 \\
%         a H( i, j + 1 ) - a H( i, j - 1 )
%       \end{bmatrix}
%   \end{align*}
%   Symbol $a$ oznacza parametr skali (kontroluje stopień wyrazistości
%   oglądanej zmiany wyglądu tekstury, czyli końcowego efektu wywołanego
%   perturbacją wektorów normalnych).

% \end{frame}
% % ##################





% % ##################
% \begin{frame}[label=Wektory-normalne]
%   \frametitle{Wektory normalne}


%   \begin{textblock}{2.1}(10,1.4)

%     \hyperlink{Objasnienie-wektory-normalne}
%     {\beamergotobutton{Objaśnienie symboli}}

%   \end{textblock}


%   \vspace{0.5em}


%   % Wektory normalne:

%   \begin{align*}
%     \vecn( i, j )
%     =
%     \frac{ \vecS( i, j ) \times \vecT( i, j ) }
%     { \normfrac{ \vecS( i, j ) \times \vecT( i, j ) } }
%     =
%     \frac{ 1 }{ \sqrt{ { \Sscript{z} }{}^{ 2 }
%     + { \Tscript{z} }{}^{ 2 } + 1 } }
%     \begin{bmatrix}
%       -\Sscript{z} \\
%       -\Tscript{z} \\
%       \hphantom{-} 1
%     \end{bmatrix} \hspace{30em}
%   \end{align*}

% \end{frame}
% % ##################





% % ##################
% \begin{frame}
%   \frametitle{Przestrzeń styczna (ang. \emph{tangent space},
%     \emph{vertex space})}


%   Dla każdego wierzchołka obiektu definiujemy układ współrzędnych
%   (przestrzeń styczna), którego środkiem jest dany wierzchołek, a~bazę
%   przestrzeni wektorowej stanowią dwa wektory styczne do powierzchni
%   obiektu oraz wektor normalny, prostopadły do powierzchni obiektu.

% \end{frame}
% % ##################





% % ##################
% \begin{frame}
%   \frametitle{Obliczenia związane z~cieniowaniem}


%   Obliczenia związane z~cieniowaniem rozdzielamy na obliczenia dla
%   każdego wierzchołka oraz obliczenia dla każdego piksela.

%   W~każdym wierzchołku obliczamy kierunki źródeł światła i~kierunki
%   obserwacji (lub wektory połówkowe), po czym transformujemy je do
%   przestrzeni stycznej.

%   W~przestrzeni stycznej rachunki dla każdego piksela (związane
%   z~interpolacją danych oświetlenia, odwzorowaniem tekstur
%   i~wypukłości) są szybsze niż w przestrzeni obiektu.

% \end{frame}
% % ##################





% % ##################
% \begin{frame}
%   \frametitle{Wyliczenie wektorów bazy przestrzeni stycznej}


%   Współrzędne tekstur oznaczamy przez $( u, v )$.

%   Dla punktów $\Pp$ wewnątrz trójkąta skojarzonymi ze współrzędnymi
%   $( u, v )$ tekstury chcemy mieć:

%   \vspace{-2em}

%   \begin{align*}
%     \Pp - \pointPscript{1} = ( u - \uscript{1} ) \vecT
%     + ( v - \vscript{1} ) \vecB\normaltextcolor{,}
%     \hspace{30em}
%   \end{align*}
%   gdzie: \\
%   $\vecT$, $\bm{B}$ to~wektory styczne \\
%   do powierzchni według mapy \\
%   tekstury.





%   \begin{textblock}{5}(7,3)

%     \begin{tikzpicture}
%       \coordinate (P 1) at (0,0,0);

%       \coordinate (P 2) at (1,0,-3);

%       \coordinate (P 3) at (0,3,0);



%       \fill[color=gray] (P 1) -- (P 2) -- (P 3) -- cycle;


%       \pic at (P 1) {point};

%       \node[left,align=left,scale=1.1] at (P 1)
%       {\phantom{A} $\pointPscript{1}$ \\
%         $( \uscript{1}, \vscript{1} )$};


%       \pic at (P 2) {point};

%       \node[right,align=left,scale=1.1] at (P 2)
%       {\phantom{A} $\pointPscript{2}$ \\
%         $( \uscript{2}, \vscript{2} )$};


%       \pic at (P 3) {point};

%       \node[above=0.3em,align=left,scale=1.1] at (P 3)
%       {\phantom{A} $\pointPscript{3}$ \\
%         $( \uscript{3}, \vscript{3} )$};
%     \end{tikzpicture}

%   \end{textblock}

% \end{frame}
% % ##################





% % ##################
% \begin{frame}[label=Wyliczanie-wektorow-bazy-przestrzeni-stycznej-2]
%   \frametitle{Wyliczenie wektorów bazy przestrzeni stycznej}


%   \begin{textblock}{2.1}(10,1.4)

%     \hyperlink{Objasnienie-wyliczanie-wektorow-bazy-przestrzeni-stycznej-2}
%     {\beamergotobutton{Objaśnienie symboli}}

%   \end{textblock}


%   \vspace{0.65em}


%   \red{Coś nie tak z oznaczeniami.} $u_{ 3 }$ a może powinno być
%   $u_{ 0 }$?

%   \begin{align*}
%     &\pointPscript{2} - \pointPscript{1}
%       = ( \uscript{2} - \uscript{1} ) \vecT
%       + ( \vscript{2} - \vscript{1} ) \vecB \hspace{30em} \\
%     &\pointPscript{3} - \pointPscript{2}
%       = ( \uscript{2} - \uscript{1} ) \vecT
%       + ( \vscript{2} - \vscript{1} ) \vecB
%   \end{align*}




%   \begin{textblock}{5}(7,3)

%     \begin{tikzpicture}
%       \coordinate (P 1) at (0,0,0);

%       \coordinate (P 2) at (1,0,-3);

%       \coordinate (P 3) at (0,3,0);



%       \fill[color=gray] (P 1) -- (P 2) -- (P 3) -- cycle;


%       \pic at (P 1) {point};

%       \node[left,align=left,scale=1.1] at (P 1)
%       {\phantom{A} $\pointPscript{1}$ \\
%         $( \uscript{1}, \vscript{1} )$};


%       \pic at (P 2) {point};

%       \node[right,align=left,scale=1.1] at (P 2)
%       {\phantom{A} $\pointPscript{2}$ \\
%         $( \uscript{2}, \vscript{2} )$};


%       \pic at (P 3) {point};

%       \node[above=0.3em,align=left,scale=1.1] at (P 3)
%       {\phantom{A} $\pointPscript{3}$ \\
%         $( \uscript{3}, \vscript{3} )$};
%     \end{tikzpicture}

%   \end{textblock}

% \end{frame}
% % ##################





% % ##################
% \begin{frame}[label=Wyliczanie-wektorow-bazy-przestrzeni-stycznej-3]
%   \frametitle{Wyliczenie wektorów bazy przestrzeni stycznej}


%   \begin{textblock}{2.1}(10,1.4)

%     \hyperlink{Objasnienie-wyliczanie-wektorow-bazy-przestrzeni-stycznej-3}
%     {\beamergotobutton{Objaśnienie symboli}}

%   \end{textblock}


%   % \vspace{-2em}


%   \begin{align*}
%     \begin{pmatrix}
%       \pointPscript{2} - \pointPscript{1} & \pointPscript{3} -
%       \pointPscript{2}
%     \end{pmatrix}^{T} =
%                                             \begin{pmatrix}
%                                               \uscript{2} -
%                                               \uscript{1}
%                                               & \vscript{2} - \vscript{1} \\
%                                               \uscript{3} -
%                                               \uscript{2} &
%                                               \vscript{3} -
%                                               \vscript{2}
%                                             \end{pmatrix}
%                                                             \begin{pmatrix}
%                                                               \vecT &
%                                                               \vecB
%                                                             \end{pmatrix}^{T}
%                                                             \hspace{30em}
%   \end{align*}

%   Możemy stąd wyliczyć wektory styczne.
%   \begin{align*}
%     \begin{pmatrix}
%       \vecT & \vecB
%     \end{pmatrix}^{T} =
%               \begin{pmatrix}
%                 \uscript{2} - \uscript{1} & \vscript{2} - \vscript{1} \\
%                 \uscript{3} - \uscript{2} & \vscript{3} - \vscript{2}
%               \end{pmatrix}^{ -1 }
%                                             \begin{pmatrix}
%                                               \pointPscript{2} -
%                                               \pointPscript{1} &
%                                               \pointPscript{3} -
%                                               \pointPscript{1}
%                                             \end{pmatrix}^{T}
%                                             \hspace{30em}
%   \end{align*}

%   Powyższe wektory razem z~wektorem normalnym $\vecn$ tworzą bazę
%   przestrzeni stycznej. Żeby to była baza ortonormalna, trzeba
%   przeprowadzić ortogonalizację Grama-Schmidta.
%   \begin{align*}
%     &\vect
%       =
%       \frac{ \vecT - ( \vecn \cdot \vect ) \vecn }
%       { \normfrac{ \vecT - ( \vecn \cdot \vecT ) \vecn } } \hspace{30em}
%     \\[0.5em]
%     &\vecb
%       =
%       \frac{ \vecB - ( \vecn \cdot \vecB ) \vecn - ( \vect \cdot \vecB ) \vect }
%       { \normfrac{ \vecB - ( \vecn \cdot \vecB ) \vecn
%       - ( \vect \cdot \vecB ) \vect } }
%   \end{align*}

% \end{frame}
% % ##################





% % ##################
% \begin{frame}
%   \frametitle{Obliczenia związane z~cieniowaniem}


%   Macierz transformacji z~przestrzeni stycznej do~przestrzeni obiektu
%   \begin{align*}
%     \matM
%     =
%     \begin{pmatrix}
%       \vect & \vecb & \vecn
%     \end{pmatrix}
%                       =
%                       \begin{pmatrix}
%                         \tscript{x} & \bscript{x} & \nscript{x} \\
%                         \tscript{y} & \bscript{y} & \nscript{y} \\
%                         \tscript{z} & \bscript{z} & \nscript{z}
%                       \end{pmatrix} \hspace{30em}
%   \end{align*}

%   Macierz transformacji z przestrzeni obiektu do przestrzeni stycznej
%   \begin{align*}
%     \matM^{ -1 }
%     =
%     \begin{pmatrix}
%       \vect & \vecb & \vecn
%     \end{pmatrix}^{T} =
%                       \begin{pmatrix}
%                         \tscript{x} & \tscript{y} & \tscript{z} \\
%                         \bscript{x} & \bscript{y} & \bscript{z} \\
%                         \nscript{x} & \nscript{y} & \nscript{z}
%                       \end{pmatrix} \hspace{30em}
%   \end{align*}

% \end{frame}
% % ##################





% % ##################
% \begin{frame}
%   \frametitle{Uwaga techniczna}


%   Często (tak jak w kontekście poprzedniego slajdu) w praktycznych
%   obliczeniach ograniczamy się do wyliczenia dwóch wektorów bazy
%   ortonormalnej, natomiast trzeci wyliczamy z~następującego wzoru
%   (znak dobieramy zgodnie z pożądaną orientacją):

%   \vspace{-3em}

%   \begingroup

%   \large

%   \begin{align*}
%     \vecb = \pm \vecn \times \vect\normaltextcolor{.} \hspace{30em}
%   \end{align*}

%   \endgroup

%   W~praktycznych zastosowaniach często wektor styczny $\vect$
%   przechowuje się w czterokomponentowej strukturze danych, gdzie
%   czwarta komponenta jest równa $+1$ lub $-1$ w zależności od wyboru
%   znaku powyżej zgodnie z pożądaną orientacją.

% \end{frame}
% % ##################





% % ##################
% \begin{frame}
%   \frametitle{Przykład. Szader wierzchołków dla mapowania wypukłości}


%   \textbf{Dane wejściowe} -- współrzędne wierzchołka $\Pp$ w przestrzeni
%   obiektu, wektor normalny $\vecn$ w~przestrzeni obiektu, współrzędne
%   tekstur koloru i połysku $( u, v )$, wektor styczny $\vect$
%   w~przestrzeni obiektu.

%   \vspace{1em}


%   \textbf{Dane wyjściowe} -- współrzędne wierzchołka $\Phom$
%   w~przestrzeni jednorodnej przyciętej, współrzędne tekstur koloru
%   i~połysku $( u, v )$, wektor kierunkowy do~źródła światła
%   $\textbf{\textit{L}}'$ w przestrzeni stycznej, wektor połówkowy
%   $\textbf{\textit{H}}'$ w przestrzeni stycznej, kolor światła $C$.

%   \vspace{1em}


%   \textbf{Zmienne globalne (ang. \emph{uniforms})} -- macierz
%   transformacji z~przestrzeni obiektu do jednorodnej przestrzeni
%   przyciętej $\homM_{MVP}$ (ang.
%   \emph{model-view-projection}), % pomidor
%   kolor światła otoczenia $\Ctextscript{amb}$, kolor światła
%   punktowego $\Cscript{0}$, pozycja światła punktowego $\Qp$ w
%   przestrzeni obiektu, współczynniki atenuacji (osłabienia) światła
%   punktowego $\kscript{c}$, $\kscript{l}$, $\kscript{q}$,
%   pozycja~obserwatora $\pointPscript{O}$ w przestrzeni obiektu.

% \end{frame}
% % ##################





% % ##################
% \begin{frame}
%   \frametitle{Przykład. Szader wierzchołków dla mapowania wypukłości}


%   Kolejne kroki: \vspace{-1.2em}
%   \begin{itemize}
%     \setlength{\itemsep}{0.7em}
%   \item transformacja wierzchołka: $\Phom = \homM_{MVP}
%     \begin{pmatrix}
%       \pointP \\
%       1
%     \end{pmatrix}$,

%   \item obliczenie wektora przesunięcia od wierzchołka do~źródła
%     światła punktowego: $\textbf{ \textit{L} } = \Qp - \pointP$
%     w~przestrzeni obiektu,

%   \item obliczenie odległości: $d = \norm{ \textbf{\textit{L}} }$,
%     i~kierunku:
%     $\vecl = { \displaystyle \frac{ \textbf{\textit{L}} }{ d } }$
%     do~źródła światła punktowego w przestrzeni obiektu,

%   \item obliczenie wektora obserwacji w przestrzeni obiektu: \\[0.2em]
%     $\vecv = { \displaystyle \frac{ \pointPscript{O} - \Pp } {
%         \norm{\pointPscript{O} - \Pp} } }$,

%   \item obliczenie wektora połówkowego w przestrzeni obiektu: \\[0.2em]
%     $\vech = { \displaystyle \frac{ \vecl + \vecv }{ \norm{ \vecl +
%           \vecv } } }$.
%   \end{itemize}

% \end{frame}
% % ##################




% % ##################
% \begin{frame}
%   \frametitle{Przykład. Szader wierzchołków dla mapowania wypukłości}


%   Kolejne kroki: \vspace{-0.45em}
%   \begin{itemize}
%     \setlength{\itemsep}{0.7em}

%   \item obliczenie wektora binormalnego w przestrzeni obiektu:
%     $\vecb = \vecn \times \vect$,

%   \item transformacja wektorów do przestrzenie stycznej: \\
%     $\textbf{\textit{L}}' =
%     \begin{pmatrix}
%       \vect & \vecb & \vecn
%     \end{pmatrix}^{T} \vecl$, \\[0.3em]
%     $\textbf{\textit{H}}' =
%     \begin{pmatrix}
%       \vect & \vecb & \vecn
%     \end{pmatrix}^{T} \vech$,

%   \item obliczenie koloru światła w punkcie $\Pp$: \\
%     $C = { \displaystyle \frac{ \Cscript{0} }{ \kscript{c} +
%         \kscript{l} d + \kscript{ q } d^{ 2 } } }$.
%   \end{itemize}

% \end{frame}
% % ##################





% % ##################
% \begin{frame}
%   \frametitle{Przykład. Szader fragmentów dla mapowania wypukłości}


%   \textbf{Interpolowane dane wejściowe} -- współrzędne tekstur koloru
%   i~połysku $( u, v )$, wektor kierunkowy do źródła światła
%   $\textbf{\textit{L}}'$ w przestrzeni stycznej, wektor połówkowy
%   $\textbf{\textit{H}}'$ w przestrzeni stycznej, kolor światła~$C$.

%   \vspace{1em}


%   \textbf{Dane wyjściowe} -- kolor piksela $K$.

%   \vspace{1em}


%   \textbf{Dwuwymiarowe mapy tekstur (ang. \emph{2D map samplers})} --
%   tekstura~koloru $\Ttextscript{color}( u, v )$, tekstura połysku
%   $\Ttextscript{gloss}( u, v )$, \\
%   tekstura wypukło\'sci $\Ttextscript{bump}( u, v )$.

%   \vspace{1em}


%   \textbf{Zmienne globalne (ang. \emph{uniforms})} -- kolor światła
%   otoczenia $\Ctextscript{amb}$, wykładnik lustrzany $m$.

% \end{frame}
% % ##################





% % ##################
% \begin{frame}[label=Przyklad-Szader-fragmentow-dla-mapowania-wypuklosci-5]
%   \frametitle{Przykład. Szader fragmentów dla mapowania wypukłości}


%   \begin{textblock}{2.1}(10,1.4)

%     \hyperlink{Objasnienie-przyklad-Szader-fragmentow-dla-mapowania-wypuklosci-5}
%     {\beamergotobutton{Objaśnienie symboli}}

%   \end{textblock}


%   \vspace{0.65em}


%   Kolejne kroki: \vspace{-0.5em}
%   \begin{itemize}
%     \setlength{\itemsep}{0.6em}

%   \item pobranie teksela koloru:
%     $\Ttextscript{color} = \Ttextscript{color}( u, v )$,

%   \item pobranie teksela połysku:
%     $\Ttextscript{gloss} = \Ttextscript{gloss}( u, v )$,

%   \item pobranie wektora normalnego w przestrzeni stycznej:
%     $\vecnscript[']{nil} = 2 \Ttextscript{bump}( u, v ) - 1$,

%   \item normalizacja interpolowanych wektorów: \\
%     $\displaystyle \veclscript[']{nil} = \frac{ \textbf{\textit{L}}'
%     }{
%       \norm{ \textbf{\textit{L}}' } }$,
%     \quad
%     $\displaystyle \vechscript[']{nil} = \frac{ \textbf{\textit{H}}' }
%     { \norm{ \textbf{\textit{H}}' } }$,

%   \item obliczenie podstawowego koloru wierzchołka:
%     $\Ktextscript{primary} = \Ctextscript{amb} + C \, \max\{
%     \vecnscript[']{nil} \cdot \vecl\hspace{0.07em}', 0 \}$,

%   \item obliczenie pochodnego koloru wierzchołka:
%     $\Ktextscript{secondary} = C \, ( \max\{ \vecn\hspace{0.07em}'
%     \cdot \vech\hspace{0.07em}', 0 \} )^{ m } \theta(
%     \vecn\hspace{0.07em}' \cdot \vecl\hspace{0.07em}' )$,

%   \item obliczenie koloru wierzchołka:
%     $K = \Ttextscript{color} \, \Ktextscript{primary} +
%     \Ttextscript{gloss} \, \Ktextscript{secondary}$.

%   \end{itemize}

% \end{frame}
% % ##################










% ######################################
\appendix
% ######################################





% ##################
\begin{frame}[standout]

  \begingroup

  \color{jStrongWhite}

  Pytania? Dziękuję za uwagę.

  \endgroup

\end{frame}
% ##################



% % ##################
% \jagiellonianendslide{Dziękuję za~uwagę.}
% % ##################










% % ######################################
% \jagielloniansectionwithpicture{Objaśnienia symboli}
% % ######################################



% % ##################
% \begin{frame}[label=Objasnienie-prawo-Lamberta-2]
%   \frametitle{Objaśnienie symboli}

%   % \vspace{-3em}

%   \begin{align*}
%     &\Ktextscript{diffuse}
%       =
%       \Dtextscript{diff} \,\, C \, \max\{ \vecn \cdot \vecl, 0 \}
%       =
%       \Dtextscript{diff} \,\, C \, \max\{ \cos \theta, 0 \} \hspace{30em}
%   \end{align*}

%   % \vspace{-1.5em}


%   $\Ktextscript{diffuse}$ to \red{?????}, \\
%   $\Dtextscript{diff}$~to~dyfuzyjny kolor powierzchni, \\% [1em]
%   $C$~to~kolor światła w~punkcie $\pointP$ powierzchni, \\% [1em]
%   $\vecn$~to~jednostkowy wektor normalny w punkcie $\pointP$ powierzchni,
%   \\% [1em]
%   $\vecl$~to~kierunek do źródła światła.
%   $\theta$ to kąt utworzony przez wektory $\vecn$ i~$\vecl$.





%   \begin{textblock}{2.1}(1,8.7)

%     \hyperlink{Prawo-Lamberta-2}{\beamerreturnbutton{Powrót do
%         wykładu}}

%   \end{textblock}

% \end{frame}
% % ##################





% % ##################
% \begin{frame}[label=Objasnienie-odbicie-lustrzane]
%   \frametitle{Objaśnienie symboli}

%   % \vspace{-3em}

%   \begin{align*}
%     &\Ktextscript{specular}
%       =
%       \Dtextscript{spec} \, C \, ( \max\{ \vecr \cdot \vecv, 0 \} )^{ m }
%       \theta( \vecn \cdot \vecl ) \hspace{30em}
%   \end{align*}

%   % \vspace{-1.5em}


%   $\Ktextscript{specular}$ to \red{?????}, \\
%   $\Dtextscript{spec}$ to~lustrzany kolor powierzchni
%   (ang.~\emph{specular color}), \\
%   $C$~to~kolor światła w~punkcie $\pointP$ powierzchni, \\% [1em]\red{?????}, \\
%   $\vecr$ to \red{?????}, \\
%   $\vecv$ to wersor pokazujący kierunek w~którym znajduje~się obserwator, \\
%   $\vecl$~to~kierunek do źródła światła, \\
%   $\vecn$~to~jednostkowy wektor normalny w punkcie $\pointP$ powierzchni, \\
%   $m$ to~tak zwany wykładnik lustrzany (ang.~\textit{shininess}), \\
%   $\theta( x )$ jest funkcją Heaviside'a (inaczej schodkową, równą $1$
%   dla $x \geq 0$, a~$0$ dla $x < 0$),





%   \begin{textblock}{2.1}(1,8.7)

%     \hyperlink{Odbicie-lustrzane}{\beamerreturnbutton{Powrót do
%         wykładu}}

%   \end{textblock}

% \end{frame}
% % ##################





% % ##################
% \begin{frame}[label=Objasnienie-standardowe-mapy-tekstur]
%   \frametitle{Objaśnienie symboli}

%   % \vspace{-3em}

%   \begin{align*}
%     s, t, p \in [ 0, 1 ] \hspace{30em}
%   \end{align*}


%   % \vspace{-1.5em}


%   $s$ to \red{?????}, \\
%   $t$ to \red{?????}, \\
%   $p$ to \red{?????}.





%   \begin{textblock}{2.1}(1,8.7)

%     \hyperlink{Standardowe-mapy-tekstur} {\beamerreturnbutton{Powrót
%         do wykładu}}

%   \end{textblock}

% \end{frame}
% % ##################





% % ##################
% \begin{frame}[label=Objasnienie-rzutowe-mapy-tekstur]
%   \frametitle{Objaśnienie symboli}

%   % \vspace{-3em}

%   \begin{align*}
%     \begin{bmatrix}
%       s \\
%       t \\
%       p \\
%       q
%     \end{bmatrix}
%     \longrightarrow
%     \begin{bmatrix}
%       \frac{ s }{ q } \\[0.7em]
%       \frac{ t }{ q } \\[0.7em]
%       \frac{ p }{ q }
%     \end{bmatrix} \hspace{30em}
%   \end{align*}


%   % \vspace{-1.5em}


%   $s$ to \red{?????}, \\
%   $t$ to \red{?????}, \\
%   $p$ to \red{?????}, \\
%   $q$ to \red{?????}.





%   \begin{textblock}{2.1}(1,8.7)

%     \hyperlink{Rzutowe-mapy-tekstur} {\beamerreturnbutton{Powrót do
%         wykładu}}

%   \end{textblock}

% \end{frame}
% % ##################





% % ##################
% \begin{frame}[label=Objasnienie-biliniowe-filtrowanie]
%   \frametitle{Objaśnienie symboli}

%   % \vspace{-3em}

%   \begin{align*}
%     &w s - \frac{ 1 }{ 2 } = \Big\lfloor ws - \frac{ 1 }{ 2 } \Big\rfloor
%       + \alpha\normaltextcolor{,}
%       \quad
%       h t - \frac{ 1 }{ 2 } = \Big\lfloor ht - \frac{ 1 }{ 2 } \Big\rfloor
%       + \beta\normaltextcolor{,} \hspace{30em} \\
%     &T
%       =
%       ( 1 - \alpha ) ( 1 - \beta ) \Tscript{i,\,j} + \alpha ( 1 - \beta ) \Tscript{i + 1,\, j}
%       + ( 1 - \alpha ) \beta \Tscript{i,\, j + 1} + \alpha \beta \Tscript{i + 1,\, j + 1}
%   \end{align*}


%   % \vspace{-1.5em}


%   $w$, $h$ to szerokość i~wysokość mapy tekstury, \\
%   $s$, $t$ to współrzędne punktu w mapie tekstury\red{?????}, \\
%   $\alpha$ to \red{?????}, \\
%   $\beta$ to \red{?????}, \\
%   $T$ to \red{?????}, \\
%   $\Tscript{i,\, j}$ to \red{?????}.





%   \begin{textblock}{2.1}(1,8.7)

%     \hyperlink{Biliniowe-filtrowanie}
%     {\beamerreturnbutton{Powrót do wykładu}}

%   \end{textblock}

% \end{frame}
% % ##################





% % ##################
% \begin{frame}[label=Objasnienie-oswietlenie-powierzchni-modyfikowane-teksturami]
%   \frametitle{Objaśnienie symboli}

%   % \vspace{-3em}

%   \begin{align*}
%     &\Ktextscript{ambient}
%       =
%       \Dtextscript{amb} \, \Ttextscript{color} \,
%       \Ctextscript{amb}\normaltextcolor{,}
%       \quad
%       \Ktextscript{diffuse}
%       =
%       \Dtextscript{diff} \, \Ttextscript{color}
%       \sum_{ i = 1 }^{ M } \Cscript{i} \, \max\{ \vecn \cdot \veclscript{i}, 0 \}
%       \hspace{30em} \\
%     &\Ktextscript{specular}
%       =
%       \Dtextscript{spec} \, \Ttextscript{gloss} \sum_{ i = 1 }^{ M }
%       \Cscript{i} \, ( \max\{ \vecn \cdot \vechscript{i}, 0 \} )^{ m }
%       \theta( \vecn \cdot \veclscript{i} )
%   \end{align*}

%   % \vspace{-1.5em}


%   $\Ktextscript{ambient}$ to \red{?????}, \\
%   $\Dtextscript{amb}$ to~kolor materiału w~\red{białym} świetle
%   otoczenia, \\
%   $\Ttextscript{color}$ to tekstura koloru, \\
%   $\Ctextscript{amb}$ to~kolor światła otoczenia, \\
%   $\Ktextscript{diffuse}$ to \red{?????}, \\
%   $\Dtextscript{diff}$~to~dyfuzyjny kolor powierzchni, \\% [1em]
%   $\Cscript{i}$ to kolor $i$-tego źródła światła, \\
%   $\vecn$ to wersor normalny do powierzchni w~punkcie $\pointP$, \\
%   $\veclscript{i}$ to \red{?????}, \\
%   $\Ktextscript{specular}$ to \red{?????}, \\
%   $\Dtextscript{spec}$ to \red{?????}, \\
%   $\Ttextscript{gloss}$ to tekstura połysku, \\
%   $\vechscript{i}$ to \red{?????}.





%   \begin{textblock}{2.1}(1,8.7)

%     \hyperlink{Oswietlenie-powierzchni-modyfikowane-teksturami}
%     {\beamerreturnbutton{Powrót do wykładu}}

%   \end{textblock}

% \end{frame}
% % ##################





% % ##################
% \begin{frame}[label=Objasnienie-cieniowanie-Gourauda]
%   \frametitle{Objaśnienie symboli}

%   % \vspace{-3em}

%   \begin{align*}
%     &\Ktextscript{primary}
%       = \Ctextscript{em} + \Dtextscript{amb} \, \Ctextscript{amb}
%       + \Dtextscript{diff} \sum_{ i = 1 }^{ M } \Cscript{i} \,
%       \max\{ \vecn \cdot \veclscript{i}, 0 \} \hspace{30em} \\
%     &\Ktextscript{secondary}
%       =
%       \Dtextscript{spec}
%       \sum_{ i = 1 }^{ M } \Cscript{i} \,
%       ( \max\{ \vecn \cdot \vechscript{i}, 0 \} )^{ m }
%       \theta( \vecn \cdot \veclscript{i} ) \hspace{30em}
%   \end{align*}


%   % \vspace{-1.5em}


%   $\Ktextscript{primary}$ to \red{?????}, \\
%   $\Ctextscript{em}$ to \red{?????}, \\
%   $\Dtextscript{amb}$ to~kolor materiału w~\red{białym} świetle
%   otoczenia, \\
%   $\Ctextscript{amb}$ to~kolor światła otoczenia, \\
%   $\Dtextscript{diff}$~to~dyfuzyjny kolor powierzchni, \\% [1em]
%   $\Cscript{i}$ to \red{?????}, \\
%   $\vecn$ to wektor normalny do powierzchni w~punkcie $\pointP$, \\
%   $\veclscript{i}$ to \red{?????}, \\
%   $\Ktextscript{secondary}$ to \red{?????}, \\
%   $\Dtextscript{spec}$ to \red{?????}, \\
%   $\vechscript{i}$ to \red{?????}.





%   \begin{textblock}{2.1}(1,8.7)

%     \hyperlink{Cieniowanie-Gourauda} {\beamerreturnbutton{Powrót do
%         wykładu}}

%   \end{textblock}

% \end{frame}
% % ##################





% % ##################
% \begin{frame}[label=Objasnienie-standardowe-rownanie-cieniowania-1]
%   \frametitle{Objaśnienie symboli}

%   % \vspace{-3em}

%   \begin{align*}
%     K &= \Ktextscript{emission} + \Ktextscript{ambient}
%         + \Ktextscript{diffuse}
%         + \Ktextscript{specular} = \hspace{30em} \\
%       &= \Ttextscript{em} \, \Ctextscript{em}
%         + \Dtextscript{amb} \, \Ttextscript{color} \, \Ctextscript{amb} + \\
%       &+ \sum_{ i = 1 }^{ M } \Cscript{i} \, \big[ \Dtextscript{diff} \,
%         \Ttextscript{color} \, \vecn \cdot \veclscript{i}
%         + \Dtextscript{spec} \, \Ttextscript{spec}
%         ( \max\{ \vecn \cdot \vechscript{i}, 0 \} )^{ m } \big]
%         \theta( \vecn \cdot \veclscript{i} )
%   \end{align*}

%   % \vspace{-1.5em}


%   $K$ to \red{?????}, \\
%   $\Ktextscript{emission}$ to \red{?????}, \\
%   $\Ktextscript{ambient}$ to \red{?????}, \\
%   $\Ktextscript{diffuse}$ to \red{?????}, \\
%   $\Ktextscript{specular}$ to \red{?????}, \\
%   $\Dtextscript{amb}$ to~kolor materiału w~\red{białym} świetle
%   otoczenia, \\
%   $\Ttextscript{spec}$ to \red{?????}, \\
%   $m$ to \red{?????}, \\
%   $\vechscript{i}$ to \red{?????}, \\
%   $\veclscript{i}$ to \red{?????}.





%   \begin{textblock}{2.1}(1,8.7)

%     \hyperlink{Standardowe-rownanie-cieniowania-1}
%     {\beamerreturnbutton{Powrót do wykładu}}

%   \end{textblock}

% \end{frame}
% % ##################





% % ##################
% \begin{frame}[label=Objasnienie-standardowe-rownanie-cieniowania-2]
%   \frametitle{Objaśnienie symboli}

%   % \vspace{-3em}

%   \begin{align*}
%     K &= \Ktextscript{emission} + \Ktextscript{ambient}
%         + \Ktextscript{diffuse} + \Ktextscript{specular} = \hspace{30em} \\
%       &=
%         \Ttextscript{em} \, \Ctextscript{em} + \Dtextscript{amb} \,
%         \Ttextscript{color} \, \Ctextscript{amb} + \\
%       &+ \sum_{ i = 1 }^{ M } \Cscript{i} \, \big[ \Dtextscript{diff} \,
%         \Ttextscript{color} \, \vecn \cdot \veclscript{i}
%         + \Dtextscript{spec} \, \Ttextscript{spec}
%         ( \max\{ \vecn \cdot \vechscript{i}, 0 \} )^{ m } \big]
%         \theta( \vecn \cdot \veclscript{i} )
%   \end{align*}

%   % \vspace{-1.5em}


%   $K$ to \red{????????}, \\
%   $\Ktextscript{emission}$ to \red{????????}, \\
%   $\Ktextscript{diffuse}$ to \red{????????}, \\
%   $\Ktextscript{specular}$ to \red{????????}, \\
%   $\Ttextscript{em}$ to \red{????????}, \\
%   $\Ctextscript{em}$ to \red{????????}, \\
%   $\Dtextscript{amb}$ to~kolor materiału w~\red{białym} świetle
%   otoczenia, \\
%   $\Ttextscript{color}$ to tekstura koloru, \\
%   $\Ctextscript{amb}$ to~kolor światła otoczenia, \\
%   $\Cscript{i}$ to \red{????????}, \\
%   $\Dtextscript{diff}$ to \red{????????}, \\
%   $\vecn$ to wersor normalny do powierzchni w~punkcie $\pointP$, \\
%   $\veclscript{i}$ to \red{????????}, \\
%   $\Dtextscript{spec}$ to \red{????????}, \\
%   $\Ttextscript{spec}$ to \red{????????}, \\
%   $\vechscript{i}$ to \red{????????}, \\
%   % to , \\
%   % to , \\
%   % to , \\
%   % to , \\
%   $m$ to \red{????????}.





%   \begin{textblock}{2.1}(1,8.7)

%     \hyperlink{Standardowe-rownanie-cieniowania-2}
%     {\beamerreturnbutton{Powrót do wykładu}}

%   \end{textblock}

% \end{frame}
% % ##################





% % ##################
% \begin{frame}[label=Objasnienie-przyklad-Szader-fragmentow-dla-cieniowania-Phonga]
%   \frametitle{Objaśnienie symboli}

%   % \vspace{-3em}


%   \begin{align*}
%     &\Ttextscript{color} = \Ttextscript{color}( u, v )\normaltextcolor{,}
%       \qquad
%       \Ttextscript{gloss} = \Ttextscript{gloss}( u, v ) \hspace{30em} \\
%     &\vecn = \frac{ \vecN }{ \norm{ \vecN } }\normaltextcolor{,}
%       \qquad
%       \vecl = \frac{ \pointQ - \pointP }{ d }\normaltextcolor{,}
%       \qquad
%       d = \norm{ \pointQ - \pointP } \\
%     &\vecv
%       =
%       \frac{ \pointPscript{O} - \pointP }
%       { \norm{ \pointPscript{O} - \pointP } }\normaltextcolor{,}
%       \qquad
%       \vech = \frac{ \vecl + \vecv }{ \norm{ \vecl + \vecv } } \\
%     &C
%       =
%       \frac{ \Cscript{0} }
%       { \kscript{c} + \kscript{l} d + \kscript{q} d^{ 2 } }
%   \end{align*}

%   % \vspace{-1.5em}


%   $\Ttextscript{color}( u, v )$ to \red{????????}, \\
%   $\Ttextscript{gloss}( u, v )$ to \red{????????}, \\
%   $\vecN$ to \red{????????}, \\
%   $\vecn$ to wersor normalny do powierzchni w~punkcie $\pointP$, \\
%   $\vecl$ to \red{????????}, \\
%   $\pointQ$ to \red{????????}, \\
%   $\pointP$ to \red{????????}, \\
%   $\vecv$ to \red{????????}, \\
%   $\pointPscript{O}$ to \red{????????}, \\
%   $\vech$ to \red{????????}, \\
%   $C$ to \red{????????}, \\
%   $\Cscript{0}$ to \red{????????}, \\
%   $d$ to \red{????????}, \\
%   $\kscript{c}$ to \red{????????}, \\
%   $\kscript{l}$ to \red{????????}, \\
%   $\kscript{q}$ to \red{????????}.





%   \begin{textblock}{2.1}(1,8.7)

%     \hyperlink{Przyklad-Szader-fragmentow-dla-cieniowania-Phonga}
%     {\beamerreturnbutton{Powrót do wykładu}}

%   \end{textblock}

% \end{frame}
% % ##################





% % ##################
% \begin{frame}[label=Objasnienie-mapowanie-wypuklosci]
%   \frametitle{Objaśnienie symboli}

%   % \vspace{-3em}

%   \begin{equation*}
%     { \vecnscript[']{nil} }^{ \, T }
%     =
%     2 \Ttextscript{bump}( u, v )
%     -
%     \begin{bmatrix}
%       1 & 1 & 1
%     \end{bmatrix} \hspace{30em}
%   \end{equation*}

%   % \vspace{-1.5em}


%   $\vecnscript[']{nil}$ to \red{?????}, \\
%   $\Ttextscript{bump}( u, v )$ to \red{?????}, \\
%   $\begin{bmatrix} 1 & 1 & 1 \end{bmatrix}$ to \red{?????}, \\
%   $u$, $v$ to \red{?????}.




%   \begin{textblock}{2.1}(1,8.7)

%     \hyperlink{Mapowanie-wypuklosci}{\beamerreturnbutton{Powrót do
%         wykładu}}

%   \end{textblock}

% \end{frame}
% % ##################





% % ##################
% \begin{frame}[label=Objasnienie-wektory-styczne]
%   \frametitle{Objaśnienie symboli}

%   % \vspace{-3em}

%   \begin{align*}
%     &\vecS( i, j )
%       =
%       \begin{bmatrix}
%         1 \\
%         0 \\
%         a H( i + 1, j ) - a H( i - 1, j )
%       \end{bmatrix} \\[0.5em]
%     &\vecT( i, j ) =
%       \begin{bmatrix}
%         0 \\
%         1 \\
%         a H( i, j + 1 ) - a H( i, j - 1 )
%       \end{bmatrix} \hspace{30em}
%   \end{align*}

%   % \vspace{-1.5em}


%   $\vecS( i,j )$ to \red{????????}, \\
%   $a$ to \red{????????}, \\
%   $H( i, j )$ to \red{????????}, \\
%   $\vecT( i, j )$ to \red{????????}.





%   \begin{textblock}{2.1}(1,8.7)

%     \hyperlink{Wektory-styczne}{\beamerreturnbutton{Powrót do
%         wykładu}}

%   \end{textblock}

% \end{frame}
% % ##################





% % ##################
% \begin{frame}[label=Objasnienie-wektory-normalne]
%   \frametitle{Objaśnienie symboli}

%   % \vspace{-3em}

%   \begin{align*}
%     \vecn( i, j )
%     =
%     \frac{ \vecS( i, j ) \times \vecT( i, j ) }
%     { \normfrac{ \vecS( i, j ) \times \vecT( i, j ) } }
%     =
%     \frac{ 1 }{ \sqrt{ { \Sscript{z} }{}^{ 2 }
%     + { \Tscript{z} }{}^{ 2 } + 1 } }
%     \begin{bmatrix}
%       -\Sscript{z} \\
%       -\Tscript{z} \\
%       \hphantom{-} 1
%     \end{bmatrix} \hspace{30em}
%   \end{align*}

%   % \vspace{-1.5em}


%   $\vecn( i, j )$ to \red{????????}, \\
%   $\vecS( i, j )$ to \red{????????}, \\
%   $\vecT( i, j )$ to \red{????????}, \\
%   $\Sscript{z}$ to \red{????????}, \\
%   $\Tscript{z}$ to \red{????????}.





%   \begin{textblock}{2.1}(1,8.7)

%     \hyperlink{Wektory-normalne} {\beamerreturnbutton{Powrót do
%         wykładu}}

%   \end{textblock}

% \end{frame}
% % ##################





% % ##################
% \begin{frame}[label=Objasnienie-wyliczanie-wektorow-bazy-przestrzeni-stycznej-2]
%   \frametitle{Objaśnienie symboli}

%   % \vspace{-3em}

%   \begin{align*}
%     &\pointPscript{2} - \pointPscript{1}
%       = ( \uscript{2} - \uscript{1} ) \vecT
%       + ( \vscript{2} - \vscript{1} ) \vecB \hspace{30em} \\
%     &\pointPscript{3} - \pointPscript{2}
%       = ( \uscript{2} - \uscript{1} ) \vecT
%       + ( \vscript{2} - \vscript{1} ) \vecB \ldots
%   \end{align*}

%   % \vspace{-1.5em}


%   $\pointPscript{1}$, $\pointPscript{2}$, $\pointPscript{3}$ to wierzchołki
%   trójkąta, \\
%   $\vecT$ to \red{???????}, \\
%   $\vecB$ to \red{???????}, \\
%   $\uscript{1}$, $\uscript{2}$, $\uscript{3}$ to \red{???????}, \\
%   $\vscript{1}$, $\vscript{2}$, $\vscript{3}$ to \red{???????}.





%   \begin{textblock}{2.1}(1,8.7)

%     \hyperlink{Wyliczanie-wektorow-bazy-przestrzeni-stycznej-2}
%     {\beamerreturnbutton{Powrót do wykładu}}

%   \end{textblock}

% \end{frame}
% % ##################





% % ##################
% \begin{frame}[label=Objasnienie-wyliczanie-wektorow-bazy-przestrzeni-stycznej-3]
%   \frametitle{Objaśnienie symboli}

%   % \vspace{-3em}


%   \begin{align*}
%     \begin{pmatrix}
%       \pointPscript{2} - \pointPscript{1} & \pointPscript{3} -
%       \pointPscript{2}
%     \end{pmatrix}^{ T }
%                                             =
%                                             \begin{pmatrix}
%                                               \uscript{2} -
%                                               \uscript{1}
%                                               & \vscript{2} - \vscript{1} \\
%                                               \uscript{3} -
%                                               \uscript{2} &
%                                               \vscript{3} -
%                                               \vscript{2}
%                                             \end{pmatrix}
%                                                             \begin{pmatrix}
%                                                               \vecT &
%                                                               \vecB
%                                                             \end{pmatrix}^{T} \hspace{30em} \\
% &\vect = \frac{ \vecT - ( \vecn \cdot \vect ) \vecn }
%                                                                                               { \normfrac{ \vecT - ( \vecn \cdot \vecT ) \vecn }
%                                                                                               }\normaltextcolor{,}
%                                                                                               \qquad \vecb = \frac{
%                                                                                               \vecB - ( \vecn \cdot
%                                                                                               \vecB ) \vecn - ( \vect
%                                                                                               \cdot \vecB ) \vect } {
%                                                                                               \normfrac{ \vecB - ( \vecn
%                                                                                               \cdot \vecB ) \vecn - (
%                                                                                               \vect \cdot \vecB ) \vect } }
%   \end{align*}

%   % \vspace{-1.5em}


%   $\pointPscript{1}$, $\pointPscript{2}$, $\pointPscript{3}$ to wierzchołki trójkąta, \\
%   $\vecT$ to \red{???????}, \\
%   $\vecB$ to \red{???????}, \\
%   $\uscript{1}$, $\uscript{2}$, $\uscript{3}$ to \red{???????}, \\
%   $\vscript{1}$, $\vscript{2}$, $\vscript{3}$ to \red{???????}, \\
%   $\vect$ to \red{???????}, \\
%   $\vecn$ to wersor normalny do powierzchni w~punkcie $\pointP$, \\
%   $\vecb$ to \red{???????}.





%   \begin{textblock}{2.1}(1,8.7)

%     \hyperlink{Wyliczanie-wektorow-bazy-przestrzeni-stycznej-3}
%     {\beamerreturnbutton{Powrót do wykładu}}

%   \end{textblock}

% \end{frame}
% % ##################





% % ##################
% \begin{frame}[label=Objasnienie-przyklad-Szader-fragmentow-dla-mapowania-wypuklosci-5]
%   \frametitle{Objaśnienie symboli}

%   % \vspace{-3em}

%   \begin{align*}
%     &\Ttextscript{color} = \Ttextscript{color}( u, v )\normaltextcolor{,}
%       \qquad
%       \Ttextscript{gloss} = \Ttextscript{gloss}( u, v ) \hspace{30em} \\
%     &\vecn\hspace{0.07em}'
%       =
%       2 \Ttextscript{bump}( u, v ) - 1\normaltextcolor{,}
%       \qquad
%       \vecl\hspace{0.07em}'
%       =
%       \frac{ \textbf{\textit{L}}' }
%       { \norm{ \textbf{\textit{L}}' } }\normaltextcolor{,}
%       \qquad
%       \vech\hspace{0.07em}'
%       =
%       \frac{ \textbf{\textit{H}}' }{ \norm{ \textbf{\textit{H}}' } } \\
%     &\Ktextscript{primary} = \Ctextscript{amb} + C \, \max\{
%       \vecnscript[']{nil} \cdot \vecl\hspace{0.07em}', 0 \} \\
%     &\Ktextscript{secondary} = C \, ( \max\{ \vecn\hspace{0.07em}'
%       \cdot \vech\hspace{0.07em}', 0 \} )^{ m } \theta(
%       \vecn\hspace{0.07em}' \cdot \vecl\hspace{0.07em}' ) \\
%     &K = \ldots
%   \end{align*}

%   % \vspace{-1.5em}


%   $\Ttextscript{color}( u, v )$ to \red{???????}, \\
%   $\Ttextscript{gloss}( u, v )$ to \red{???????}, \\
%   $\vecn\hspace{0.07em}'$ to \red{???????}, \\
%   $\Ttextscript{bump}( u, v )$ to \red{???????}, \\
%   $\vecl\hspace{0.07em}'$ to \red{???????}, \\
%   $\textbf{\textit{L}}'$ to \red{???????}, \\
%   $\vech\hspace{0.07em}'$ to \red{???????}, \\
%   $\textbf{\textit{H}}'$ to \red{????????}, \\
%   $\Ktextscript{primary}$ to \red{???????}, \\
%   $\Ctextscript{amb}$ to \red{???????}, \\
%   $C$ to \red{???????}, \\
%   $\Ktextscript{secondary}$ to \red{???????}, \\
%   $K$ to \red{???????}.





%   \begin{textblock}{2.1}(1,8.7)

%     \hyperlink{Przyklad-Szader-fragmentow-dla-mapowania-wypuklosci-5}
%     {\beamerreturnbutton{Powrót do wykładu}}

%   \end{textblock}

% \end{frame}
% % ##################





% % ##################
% \begin{frame}[label=Objasnienie-modele-funkcji-rozblysku-Phonga-i-Blinna]
%   \frametitle{Objaśnienie symboli}


%   $\vecl$ to wersor wskazujący kierunek gdzie znajduje~się źródła światła, \\
%   $\vecn$ to wersor normalny do płaszczyzny na~którą pada światło, \\
%   $\vecv$ to wersor pokazujący kierunek w~którym znajduje~się obserwator, \\
%   $\vech$ to wersor połówkowy wersorów $\vecl$ i~$\vecr$.





%   \begin{textblock}{2.1}(1,8.7)

%     \hyperlink{Modele-funkcji-rozblysku-Phonga-i-Blinna}
%     {\beamerreturnbutton{Powrót do wykładu}}

%   \end{textblock}

% \end{frame}
% ##################










% ############################

% Koniec dokumentu
\end{document}
