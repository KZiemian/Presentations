% ---------------------------------------------------------------------
% Basic configuration of Beamera and Jagiellonian
% ---------------------------------------------------------------------
\RequirePackage[l2tabu, orthodox]{nag}



\ifx\PresentationStyle\notset
\def\PresentationStyle{dark}
\fi



\documentclass[10pt,t]{beamer}
\mode<presentation>
\usetheme[style=\PresentationStyle,logoColor=monochromaticJUwhite,JUlogotitle=yes]{jagiellonian}



% ---------------------------------------
% Configuration files of Jagiellonian loceted in catalog preambule
% ---------------------------------------
% Configuration for polish language
% Need description
\usepackage[polish]{babel}
% Need description
\usepackage[MeX]{polski}



% ------------------------------
% Better support of polish chars in technical parts of PDF
% ------------------------------
\hypersetup{pdfencoding=auto,psdextra}

% Package "textpos" give as enviroment "textblock" which is very usefull in
% arranging text on slides.

% This is standard configuration of "textpos"
\usepackage[overlay,absolute]{textpos}

% If you need to see bounds of "textblock's" comment line above and uncomment
% one below.

% Caution! When showboxes option is on significant ammunt of space is add
% to the top of textblock and as such, everyting put in them gone down.
% We need to check how to remove this bug.

% \usepackage[showboxes,overlay,absolute]{textpos}



% Setting scale length for package "textpos"
\setlength{\TPHorizModule}{10mm}
\setlength{\TPVertModule}{\TPHorizModule}


% ---------------------------------------
% TikZ
% ---------------------------------------
% Importing TikZ libraries
\usetikzlibrary{arrows.meta}
\usetikzlibrary{positioning}





% % Configuration package "bm" that need for making bold symbols
% \newcommand{\bmmax}{0}
% \newcommand{\hmmax}{0}
% \usepackage{bm}




% ---------------------------------------
% Packages for scientific texts
% ---------------------------------------
% \let\lll\undefined  % Sometimes you must use this line to allow
% "amsmath" package to works with packages with packages for polish
% languge imported
% /preambul/LanguageSettings/JagiellonianPolishLanguageSettings.tex.
% This comments (probably) removes polish letter Ł.
\usepackage{amsmath}  % Packages from American Mathematical Society (AMS)
\usepackage{amssymb}
\usepackage{amscd}
\usepackage{amsthm}
\usepackage{siunitx}  % Package for typsetting SI units.
\usepackage{upgreek}  % Better looking greek letters.
% Example of using upgreek: pi = \uppi


\usepackage{calrsfs}  % Zmienia czcionkę kaligraficzną w \mathcal
% na ładniejszą. Może w innych miejscach robi to samo, ale o tym nic
% nie wiem.










% ---------------------------------------
% Packages written for lectures "Geometria 3D dla twórców gier wideo"
% ---------------------------------------
% \usepackage{./ProgramowanieSymulacjiFizykiPaczki/ProgramowanieSymulacjiFizyki}
% \usepackage{./ProgramowanieSymulacjiFizykiPaczki/ProgramowanieSymulacjiFizykiIndeksy}
% \usepackage{./ProgramowanieSymulacjiFizykiPaczki/ProgramowanieSymulacjiFizykiTikZStyle}





% !!!!!!!!!!!!!!!!!!!!!!!!!!!!!!
% !!!!!!!!!!!!!!!!!!!!!!!!!!!!!!
% EVIL STUFF
\if\JUlogotitle1
\edef\LogoJUPath{LogoJU_\JUlogoLang/LogoJU_\JUlogoShape_\JUlogoColor.pdf}
\titlegraphic{\hfill\includegraphics[scale=0.22]
{./JagiellonianPictures/\LogoJUPath}}
\fi
% ---------------------------------------
% Commands for handling colors
% ---------------------------------------


% Command for setting normal text color for some text in math modestyle
% Text color depend on used style of Jagiellonian

% Beamer version of command
\newcommand{\TextWithNormalTextColor}[1]{%
  {\color{jNormalTextFGColor}
    \setbeamercolor{math text}{fg=jNormalTextFGColor} {#1}}
}

% Article and similar classes version of command
% \newcommand{\TextWithNormalTextColor}[1]{%
%   {\color{jNormalTextsFGColor} {#1}}
% }



% Beamer version of command
\newcommand{\NormalTextInMathMode}[1]{%
  {\color{jNormalTextFGColor}
    \setbeamercolor{math text}{fg=jNormalTextFGColor} \text{#1}}
}


% Article and similar classes version of command
% \newcommand{\NormalTextInMathMode}[1]{%
%   {\color{jNormalTextsFGColor} \text{#1}}
% }




% Command that sets color of some mathematical text to the same color
% that has normal text in header (?)

% Beamer version of the command
\newcommand{\MathTextFrametitleFGColor}[1]{%
  {\color{jFrametitleFGColor}
    \setbeamercolor{math text}{fg=jFrametitleFGColor} #1}
}

% Article and similar classes version of the command
% \newcommand{\MathTextWhiteColor}[1]{{\color{jFrametitleFGColor} #1}}





% Command for setting color of alert text for some text in math modestyle

% Beamer version of the command
\newcommand{\MathTextAlertColor}[1]{%
  {\color{jOrange} \setbeamercolor{math text}{fg=jOrange} #1}
}

% Article and similar classes version of the command
% \newcommand{\MathTextAlertColor}[1]{{\color{jOrange} #1}}





% Command that allow you to sets chosen color as the color of some text into
% math mode. Due to some nuances in the way that Beamer handle colors
% it not work in all cases. We hope that in the future we will improve it.

% Beamer version of the command
\newcommand{\SetMathTextsColor}[2]{%
  {\color{#1} \setbeamercolor{math text}{fg=#1} #2}
}


% Article and similar classes version of the command
% \newcommand{\SetMathTextColor}[2]{{\color{#1} #2}}










% ---------------------------------------
% Commands for setting background pictures for some slides
% ---------------------------------------
\newcommand{\TitleBackgroundPicture}
{./PresentationPictures/CommonPictures/Cute_dragon_BG_dark.png}
\newcommand{\SectionBackgroundPicture}
{./PresentationPictures/CommonPictures/Cute_dragon_small_BG_light.png}



\newcommand{\TitleSlideWithPicture}{
  \begingroup

  \usebackgroundtemplate{ % \hspace*{-11.5em}
    \includegraphics[height=\paperheight]{\TitleBackgroundPicture}}

  \maketitle

  \endgroup
}





\newcommand{\SectionSlideWithPicture}[1]{%
  \begingroup

  \usebackgroundtemplate{ % \hspace*{-11.5em}
    \includegraphics[height=\paperheight]{\SectionBackgroundPicture}}

  \setbeamercolor{titlelike}{fg=normal text.fg}

  \section{#1}

  \endgroup
}





\newcommand{\EndingSlide}[1]{%
  \begin{frame}[standout]

    \begingroup

    \color{jFrametitleFGColor}

    #1

    \endgroup

  \end{frame}
}










% ---------------------------------------
% Packages, libraries and their configuration
% ---------------------------------------





% ---------------------------------------
% Configuration for this particular presentation
% ---------------------------------------
\usetikzlibrary{decorations.markings,decorations.pathmorphing}

\tikzstyle{spring style} = [decoration={aspect=0.7, segment length=1.7mm,
  amplitude=1.3mm,coil},decorate]

\tikzset{
  nasz kwadrat/.pic={
    \draw (-0.5,-0.5) rectangle (0.5,0.5);
  },
}





% ------------------------------
% Ustawienie dla tego konkretnego pliku
% ------------------------------
% \newcounter{}










% ---------------------------------------------------------------------
\title{Wprowadzenie do PGFa i~Ti\textit{k}Za}
\subtitle{Wprowadzenie do używania Ti\textit{k}Za}

\author{Kamil Ziemian}


\date{}
% ---------------------------------------------------------------------










% ####################################################################
% Początek dokumentu
\begin{document}
% ####################################################################





% Wyrównanie do lewej z łamaniem wyrazów

\RaggedRight





% ######################################
\maketitle
% ######################################










% ##################
\begin{frame}
  \frametitle{\textbackslash path \textbackslash draw}


  % #############
  \begin{figure}

    \centering

    \begin{tikzpicture}

      % \path[draw] (0,0) -- (1,0) -- (1,2) -- (3,4);

      % \draw (0,0) -- (0,2) -- (3,0) -- cycle;

    \end{tikzpicture}


    \caption{\textbackslash path, \textbackslash draw}

  \end{figure}
  % #############


  Pamiętajcie o średniku.

\end{frame}
% ##################





% ##################
\begin{frame}
  \frametitle{Strzałki}


  % \tikzstyle{strzalka 1} = [-{Straight Barb[scale=1.2]},line width=1]


  % #############
  \begin{figure}

    \centering

    \begin{tikzpicture}

      % \path (0,0) -- (0,2);



      % \draw[-{Straight Barb[scale=1.2]},line width=1] (0,0) -- (3,1);

      % \draw[strzalka 1] (0,1) -- (2,-1);

      % \draw[strzalka 1] (3,3) -- (0,2);

    \end{tikzpicture}


    \caption{Strzałki}

  \end{figure}
  % #############

\end{frame}
% ##################





% ##################
\begin{frame}
  \frametitle{Dekoracje}


  % #############
  \begin{figure}

    \centering

    \begin{tikzpicture}

      % \path (0,0) -- (0,3);



      % \draw[decoration={coil},decorate] (0,0) -- (3,2);

      % \draw[spring style] (1,0) -- (4,1);

    \end{tikzpicture}


    \caption{Dekoracje}

  \end{figure}
  % #############

\end{frame}
% ##################





% ##################
\begin{frame}
  \frametitle{\textbackslash fill}


  % #############
  \begin{figure}

    \centering

    \begin{tikzpicture}

      % \filldraw[fill=green,line width=1] (0,0) -- (2,0) -- (3,2) -- cycle;

      % \draw[->]

      % \path[fill]

      % \path[fill]

    \end{tikzpicture}


    \caption{\textbackslash fill}

  \end{figure}
  % #############

\end{frame}
% ##################





% ##################
\begin{frame}
  \frametitle{Figury}


  % #############
  \begin{figure}

    \centering

    \begin{tikzpicture}

      % \draw[rotate=30] (0,0) rectangle (3,2);

      % \draw (0,0) circle [radius=1];

      % \draw[rotate=45] (0,0) ellipse [x radius=2,y radius=0.75];

      % \fill[color=red] (0,0) rectangle (3,2);

      % \fill[color=green] (0,0) circle [radius=1];

      % \fill[color=blue] (0,0) ellipse [x radius=2,y radius=0.5];

    \end{tikzpicture}


    \caption{Figury}

  \end{figure}
  % #############

\end{frame}
% ##################





% ##################
\begin{frame}
  \frametitle{Kolory}


  % \definecolor{orange}{rgb}{1,0.5,0}
  % % RGB
  % \definecolor{somecolor}{HTML}{445E8D}
  % \colorlet{lightgray}{black!30}


  % #############
  \begin{figure}

    \centering

    \begin{tikzpicture}

      % \fill[color=red] (0,0) rectangle (3,2);

      % \fill[blue] (0,0) rectangle (3,2);

      % \fill[color=white] (0,0) rectangle (3,2);

      % \fill[color=purple] (0,0) rectangle (3,2);

      % \fill[color=gray] (0,0) rectangle (3,2);

      % \fill[somecolor] (0,0) rectangle (3,2);

      % \fill[color=red!50!blue!50!green] (0,0) rectangle (3,2);

      % \fill[color=lightgray] (0,0) rectangle (3,2);

    \end{tikzpicture}


    \caption{Kolory}

  \end{figure}
  % #############


  Z paczki xcolor. \\
  \textbackslash definecolor\{orange\}\{rgb\}\{1,0.5,0\} \\
  \textbackslash colorlet\{lightgray\}\{black!20\}

\end{frame}
% ##################





% ##################
\begin{frame}
  \frametitle{Zakres (ang. \textit{scope})}


  % #############
  \begin{figure}

    \centering

    \begin{tikzpicture}

      % \fill[color=red] (0,0) rectangle (3,2);


      % \begin{scope}[xshift=4cm,yshift=-2cm,xscale=0.5]

      %   \fill[color=blue] (0,0) rectangle (3,2);

      % \end{scope}

    \end{tikzpicture}


    \caption{Figury}

  \end{figure}
  % #############

\end{frame}
% ##################





% ##################
\begin{frame}
  \frametitle{Mieszanie kolorów}


  % #############
  \begin{figure}

    \centering

    \begin{tikzpicture}

      \begin{scope}[blend mode=lighten]

        \fill[red!90!black]   ( 90:0.6) circle [radius=1];

        \fill[green!80!black] (210:0.6) circle [radius=1];

        \fill[blue!90!black]  (330:0.6) circle [radius=1];

      \end{scope}

    \end{tikzpicture}


    \caption{Mieszanie kolorów}

  \end{figure}
  % #############

\end{frame}
% ##################






% ##################
\begin{frame}
  \frametitle{Węzły}


  % #############
  \begin{figure}

    \centering

    \begin{tikzpicture}

      % \draw[->,line width=1] (0,0) -- (2,1);

      % \node[circle,draw,scale=0.8,text width=1.5cm] at (2.5,0.6)
      % {droga donikąd};

      % \node[rectangle,draw=red,scale=0.7] at (0,1) {$\int f( x ) \, dx$};

    \end{tikzpicture}


    \caption{Węzły}

  \end{figure}
  % #############

\end{frame}
% ##################





% ##################
\begin{frame}
  \frametitle{Prosty diagram}


  % #############
  \begin{figure}

    \centering

    \begin{tikzpicture}[node distance=0.5cm]

      % \node[rectangle,draw] (kod zrodlowy) {Pisze kod źródłowy};

      % \node[rectangle,draw,below =of kod zrodlowy] (kompilacja)
      % {Kompiluje};

      % \draw[-{Stealth}] (kod zrodlowy) -- (kompilacja);


      % \node[rectangle,draw,below right=of kompilacja] (kompilacja udana)
      % {Kompilacja udana};

      % \draw[-{Stealth}] (kompilacja) -- (kompilacja udana);


      % \node[rectangle,draw,below left=of kompilacja] (blad kompilator)
      % {Syntax error};

      % \draw[-{Stealth}] (kompilacja) -- (blad kompilator);


      % \draw[-{Stealth}] (blad kompilator) to [bend left=45]
      % (kod zrodlowy);

    \end{tikzpicture}


    \caption{Prosty diagram}

  \end{figure}
  % #############

\end{frame}
% ##################





% ##################
\begin{frame}
  \frametitle{Krzywe}


  % #############
  \begin{figure}

    \centering

    \begin{tikzpicture}

      % \draw (0,0) .. controls (0.5,1.5) .. (2,0);

      % \fill (0,0) circle [radius=0.09];

      % \fill (0.5,1.5) circle [radius=0.09];

      % \fill (2,0) circle [radius=0.09];


      \begin{scope}[yshift=-2cm]

        % \draw (0,0) .. controls (0.8,1.5) and (2,-1) .. (3,0);

        % \fill (0,0) circle [radius=0.09];

        % \fill (0.8,1.5) circle [radius=0.09];

        % \fill (2,-1) circle [radius=0.09];

        % \fill (3,0) circle [radius=0.09];

      \end{scope}

    \end{tikzpicture}


    \caption{Krzywe}

  \end{figure}
  % #############

\end{frame}
% ##################





% ##################
\begin{frame}
  \frametitle{Krzywe}


  % #############
  \begin{figure}

    \centering

    \begin{tikzpicture}

      % \coordinate (pierwszy punkt kontrolny) at (1.5,1);



      % \draw (0,0) .. controls (pierwszy punkt kontrolny) and (2,-1) ..
      % (3,0);


      % \fill (0,0) circle [radius=0.09];

      % \fill (pierwszy punkt kontrolny) circle [radius=0.09];

      % \node[above,scale=0.7,text width=3.5cm] at
      % (pierwszy punkt kontrolny) {pierwszy punkt kontrolny};


      % \fill (2,-1) circle [radius=0.09];

      % \fill (3,0) circle [radius=0.09];

    \end{tikzpicture}


    \caption{Krzywe}

  \end{figure}
  % #############

\end{frame}
% ##################





% ##################
\begin{frame}
  \frametitle{Pics (od ang. \textit{small PICture})}


  % #############
  \begin{figure}

    \centering

    \begin{tikzpicture}[scale=2]

      % \path (0,0) -- (0,3);



      % \draw[dashed] (0,1) -- (4,1);

      % \draw[rotate=45] (0,0) rectangle (1,1);


      % \pic[rotate=45] at (4,1) {nasz kwadrat};

      % \pic[scale=0.5] at (2,2) {nasz kwadrat};

      % \pic[scale=0.5] at (0,2) {nasz kwadrat};

    \end{tikzpicture}


    \caption{Krzywe}

  \end{figure}
  % #############

\end{frame}
% ##################










% ######################################
\appendix
% ######################################





% ##################
\EndingSlide{Dziękuję za uwagę. Pytania?}
% ##################










% ############################

% Koniec dokumentu
\end{document}
