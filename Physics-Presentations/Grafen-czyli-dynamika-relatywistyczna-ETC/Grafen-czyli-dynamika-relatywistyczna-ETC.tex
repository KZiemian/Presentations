% ---------------------------------------------------------------------
% Basic configuration of Beamera and Jagiellonian
% ---------------------------------------------------------------------
\RequirePackage[l2tabu, orthodox]{nag}



\ifx\PresentationStyle\notset
\def\PresentationStyle{dark}
\fi



\documentclass[10pt,t]{beamer}
\mode<presentation>
\usetheme[style=\PresentationStyle,logoLang=Latin,logoColor=monochromaticJUwhite,JUlogotitle=yes]{jagiellonian}



% ---------------------------------------
% Configuration files of Jagiellonian loceted in catalog preambule
% ---------------------------------------
% Configuration for polish language
% Need description
\usepackage[polish]{babel}
% Need description
\usepackage[MeX]{polski}



% ------------------------------
% Better support of polish chars in technical parts of PDF
% ------------------------------
\hypersetup{pdfencoding=auto,psdextra}

% Package "textpos" give as enviroment "textblock" which is very usefull in
% arranging text on slides.

% This is standard configuration of "textpos"
\usepackage[overlay,absolute]{textpos}

% If you need to see bounds of "textblock's" comment line above and uncomment
% one below.

% Caution! When showboxes option is on significant ammunt of space is add
% to the top of textblock and as such, everyting put in them gone down.
% We need to check how to remove this bug.

% \usepackage[showboxes,overlay,absolute]{textpos}



% Setting scale length for package "textpos"
\setlength{\TPHorizModule}{10mm}
\setlength{\TPVertModule}{\TPHorizModule}


% ---------------------------------------
% TikZ
% ---------------------------------------
% Importing TikZ libraries
\usetikzlibrary{arrows.meta}
\usetikzlibrary{positioning}





% % Configuration package "bm" that need for making bold symbols
% \newcommand{\bmmax}{0}
% \newcommand{\hmmax}{0}
% \usepackage{bm}




% ---------------------------------------
% Packages for scientific texts
% ---------------------------------------
% \let\lll\undefined  % Sometimes you must use this line to allow
% "amsmath" package to works with packages with packages for polish
% languge imported
% /preambul/LanguageSettings/JagiellonianPolishLanguageSettings.tex.
% This comments (probably) removes polish letter Ł.
\usepackage{amsmath}  % Packages from American Mathematical Society (AMS)
\usepackage{amssymb}
\usepackage{amscd}
\usepackage{amsthm}
\usepackage{siunitx}  % Package for typsetting SI units.
\usepackage{upgreek}  % Better looking greek letters.
% Example of using upgreek: pi = \uppi


\usepackage{calrsfs}  % Zmienia czcionkę kaligraficzną w \mathcal
% na ładniejszą. Może w innych miejscach robi to samo, ale o tym nic
% nie wiem.










% ---------------------------------------
% Packages written for lectures "Geometria 3D dla twórców gier wideo"
% ---------------------------------------
% \usepackage{./ProgramowanieSymulacjiFizykiPaczki/ProgramowanieSymulacjiFizyki}
% \usepackage{./ProgramowanieSymulacjiFizykiPaczki/ProgramowanieSymulacjiFizykiIndeksy}
% \usepackage{./ProgramowanieSymulacjiFizykiPaczki/ProgramowanieSymulacjiFizykiTikZStyle}





% !!!!!!!!!!!!!!!!!!!!!!!!!!!!!!
% !!!!!!!!!!!!!!!!!!!!!!!!!!!!!!
% EVIL STUFF
\if\JUlogotitle1
\edef\LogoJUPath{LogoJU_\JUlogoLang/LogoJU_\JUlogoShape_\JUlogoColor.pdf}
\titlegraphic{\hfill\includegraphics[scale=0.22]
{./JagiellonianPictures/\LogoJUPath}}
\fi
% ---------------------------------------
% Commands for handling colors
% ---------------------------------------


% Command for setting normal text color for some text in math modestyle
% Text color depend on used style of Jagiellonian

% Beamer version of command
\newcommand{\TextWithNormalTextColor}[1]{%
  {\color{jNormalTextFGColor}
    \setbeamercolor{math text}{fg=jNormalTextFGColor} {#1}}
}

% Article and similar classes version of command
% \newcommand{\TextWithNormalTextColor}[1]{%
%   {\color{jNormalTextsFGColor} {#1}}
% }



% Beamer version of command
\newcommand{\NormalTextInMathMode}[1]{%
  {\color{jNormalTextFGColor}
    \setbeamercolor{math text}{fg=jNormalTextFGColor} \text{#1}}
}


% Article and similar classes version of command
% \newcommand{\NormalTextInMathMode}[1]{%
%   {\color{jNormalTextsFGColor} \text{#1}}
% }




% Command that sets color of some mathematical text to the same color
% that has normal text in header (?)

% Beamer version of the command
\newcommand{\MathTextFrametitleFGColor}[1]{%
  {\color{jFrametitleFGColor}
    \setbeamercolor{math text}{fg=jFrametitleFGColor} #1}
}

% Article and similar classes version of the command
% \newcommand{\MathTextWhiteColor}[1]{{\color{jFrametitleFGColor} #1}}





% Command for setting color of alert text for some text in math modestyle

% Beamer version of the command
\newcommand{\MathTextAlertColor}[1]{%
  {\color{jOrange} \setbeamercolor{math text}{fg=jOrange} #1}
}

% Article and similar classes version of the command
% \newcommand{\MathTextAlertColor}[1]{{\color{jOrange} #1}}





% Command that allow you to sets chosen color as the color of some text into
% math mode. Due to some nuances in the way that Beamer handle colors
% it not work in all cases. We hope that in the future we will improve it.

% Beamer version of the command
\newcommand{\SetMathTextsColor}[2]{%
  {\color{#1} \setbeamercolor{math text}{fg=#1} #2}
}


% Article and similar classes version of the command
% \newcommand{\SetMathTextColor}[2]{{\color{#1} #2}}










% ---------------------------------------
% Commands for setting background pictures for some slides
% ---------------------------------------
\newcommand{\TitleBackgroundPicture}
{./PresentationPictures/CommonPictures/Cute_dragon_BG_dark.png}
\newcommand{\SectionBackgroundPicture}
{./PresentationPictures/CommonPictures/Cute_dragon_small_BG_light.png}



\newcommand{\TitleSlideWithPicture}{
  \begingroup

  \usebackgroundtemplate{ % \hspace*{-11.5em}
    \includegraphics[height=\paperheight]{\TitleBackgroundPicture}}

  \maketitle

  \endgroup
}





\newcommand{\SectionSlideWithPicture}[1]{%
  \begingroup

  \usebackgroundtemplate{ % \hspace*{-11.5em}
    \includegraphics[height=\paperheight]{\SectionBackgroundPicture}}

  \setbeamercolor{titlelike}{fg=normal text.fg}

  \section{#1}

  \endgroup
}





\newcommand{\EndingSlide}[1]{%
  \begin{frame}[standout]

    \begingroup

    \color{jFrametitleFGColor}

    #1

    \endgroup

  \end{frame}
}










% ---------------------------------------
% Packages, libraries and their configuration
% ---------------------------------------
\usepackage{mathcommands}





% ---------------------------------------
% Configuration for this particular presentation
% ---------------------------------------










% ---------------------------------------------------------------------
\title{Grafen, czyli dynamika relatywistyczna przy trochę mniejszych prędkościach}

\author{Kamil Ziemian \\
  \texttt{kziemianfvt@gmail.com}}


\institute{II rok, fizyka teoretyczna, studia magisterskie}

\date[16 V 2013]{16 maja 2013 r.}
% ---------------------------------------------------------------------










% ####################################################################
% Początek dokumentu
\begin{document}
% ####################################################################





% Wyrównanie do lewej z łamaniem wyrazów

\RaggedRight





% ######################################
\maketitle
% ######################################





% ##################
\begin{frame}
  \frametitle{Wiadomości wstępne}


  Grafen to pojedyncza warstwa atomów węgla ułożonych w kształt plastra
  miodu.

  Kilka podstawowych informacja o~jego strukturze fizycznej.
  \begin{itemize}
    \RaggedRight

  \item Sieć Bravais'a grafenu jest siecią trójkątną zawierającą w~każdej
    komórce dwa atomy węgla.

  \item Pierwsza strefa Brillouin'a, otrzymana metodą Wigner-Seitz'a jest
    heksagonalna. Nie ma to jednak nic wspólnego z rozkładem atomów
    w~przestrzeni.

  \item Spośród jej sześciu wierzchołków jedynie dwa dają fizycznie różne
    wyniki. Noszą one nazwę K-punktów Diraca.

  \end{itemize}

\end{frame}
% ##################





% ##################
\begin{frame}
  \frametitle{Model ciasnego wiązania}


  Podstawowe założenie modelu ciasnego wiązania. Ruch elektronu w grafenie
  jest opisywany przez nierelatywistyczne równanie Schr\"{o}dingera
  z~hamiltonianem:
  \begin{equation}
    \label{eq:Grafen-czyli-01}
    \widehat{ H } =
    -\frac{ \hbar^{ 2 } }{ 2 m } \triangle + V_{ \textrm{eff} }( \vecxbold ),
  \end{equation}
  gdzie $V_{ \textrm{eff} }( \vecxbold )$ jest potencjałem efektywnym
  pochodzącym od jonów sieci i~innych elektronów \cite{EWQTN}.

  Wyprowadzenie modelu. Ruch elektronu polega na przeskokach między stanami
  związanymi $2 p_{ z }$ poszczególnych atomów węgla. Przyjmujemy następujące
  oznaczenie:
  \begin{equation}
    \label{eq:Grafen-czyli-02}
    \psi_{ i, \alpha }( \vecxbold ) = | i, \alpha \rangle,
  \end{equation}
  na funkcję falową elektronu związanego na $i$-węźle, atomu $\alpha$.

\end{frame}
% ##################





% ##################
\begin{frame}
  \frametitle{Model ciasnego wiązania}


  Ze względu na naturę funkcji $2 p_{ z }$ dochodzi do ich przekrywania,
  niemniej przyjmiemy iż:
  \begin{equation}
    \label{eq:Grafen-czyli-03}
    \langle i, \alpha \, | \, j, \beta \rangle = \delta_{ i j } \, \delta_{ \alpha \beta }.
  \end{equation}

  Ze względu na standardowy rozkład hamiltonianu:
  \begin{equation}
    \label{eq:Grafen-czyli-04}
    \widehat{ H } =
    \sum_{ i, \alpha, \, j, \beta } \langle i, \alpha \, | \widehat{ H } | \, j, \beta \rangle \,
    | \, i, \alpha \rangle \langle \, j,\beta |,
  \end{equation}
  do pełnego zdefiniowania problemu potrzebujemy jedynie określić elementy
  macierzowe.

\end{frame}
% ##################





% ##################
\begin{frame}
  \frametitle{Model ciasnego wiązania}


  Elementy macierzowe
  \begin{equation}
    \label{eq:Grafen-czyli-05}
    \langle i, \alpha \, | \widehat{ H } | \, j, \beta \rangle =
    \begin{cases}
      V_{ 0 } + M, & i = j, \quad \alpha = \beta = 1, \\
      V_{ 0 } - M, & i = j, \quad \alpha = \beta = 2, \\
      -t,  & \textrm{dla najbliższych sąsiadów,} \\
      -t', & \textrm{dla prawie najbliższych sąsiadów,}
    \end{cases}
  \end{equation}
  gdzie $t = 2.7 \textrm{ eV}$, $t' = 0.1 \, t$.

  Ansatz. Stany własne są postaci:
  \begin{subequations}
    \begin{align}
      \label{eq:Grafen-czyli-06-A}
      &\varphi( \veckbold, \vecxbold ) =
        \varphi_{ 1 }( \veckbold ) \, \Phi_{ 1 }( \veckbold, \vecxbold )
        + \varphi_{ 2 }( \veckbold ) \, \Phi_{ 2 }( \veckbold, \vecxbold ), \\
      \label{eq:Grafen-czyli-06-B}
      &\Phi_{ \alpha }( \veckbold , \vecxbold ) =
        \frac{ 1 }{ \sqrt{ N } }
        \sum_{ i } e^{ i \veckbold ( \vecRbold_{ i } + \vecdbold_{ \alpha } ) } \,
        \psi_{ i, \alpha }( \vecxbold ).
    \end{align}
  \end{subequations}

\end{frame}
% ##################





% ##################
\begin{frame}
  \frametitle{Rozwiązanie analityczne}


  Pasma energetyczne
  \begin{equation}
    \begin{split}
      E_{ \pm }( \veckbold )
      &=
        V_{ 0 } - t' \left( 2 \cos( k_{ x } a )
        + 4 \, \cos\left( \frac{ k_{ x } a }{ 2 } \right) \,
        \cos\left( \frac{ \sqrt{ 3 } k_{ y } a }{ 2 } \right)
        \right) \pm \\[0.3em]
      &\pm \, \sqrt{ M^{ 2 } + t^{ 2 } \left( 3 + 2 \cos( k_{ x } a )
        + 4 \cos\left( \frac{ k_{ x } a }{ 2 } \right)
        \cos\left( \frac{ \sqrt{ 3 } k_{ y } a }{ 2 } \right) \right) }.
    \end{split}
  \end{equation}

  Rozwinięcie wokół K-punktów Diraca
  \begin{equation}
    E_{ \pm }( \vecKbold + \vecqbold ) \approx
    \begin{cases}
      \hbar v_{ \textrm{F} } | \vecqbold |, & M = 0, \\[0.3em]
      \pm\sqrt{ \left( \frac{ M }{ v_{ \textrm{F} }^{ 2 } } \right)^{ 2 }
      v_{ \textrm{F} }^{ 4 }
      + v_{ \textrm{F} }^{ 2 } \, ( \, \hbar | \vecqbold | )^{ 2 } }, & M \neq 0,
    \end{cases}
  \end{equation}
  gdzie $v_{ \textrm{F} } = ( \sqrt{ 2 } t a ) / ( 2 \; \hbar ) \approx 10^{ 6 } \,
  \frac{ \textrm{m} }{ \textrm{s} }$.

\end{frame}
% ##################










% ######################################
\appendix
% ######################################





% ######################################
\EndingSlide{Dziękuję! Pytania?}
% ######################################










% ##################
\begin{frame}
  \frametitle{Bibliografia}


  \bibliographystyle{plalpha}

  \bibliography{VariousFieldsBooks}{}

\end{frame}
% ##################





% ############################

% Koniec dokumentu
\end{document}
