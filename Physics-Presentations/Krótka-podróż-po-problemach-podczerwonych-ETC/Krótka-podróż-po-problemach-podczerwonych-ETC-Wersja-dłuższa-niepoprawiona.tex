% ------------------------------------------------------------------------------------------------------------------
% Basic configuration of Beamera class and Jagiellonian theme
% ------------------------------------------------------------------------------------------------------------------
\RequirePackage[l2tabu, orthodox]{nag}



\ifx\PresentationStyle\notset
  \def\PresentationStyle{dark}
\fi



% Options: t -- align frame text to the top
\documentclass[10pt,t]{beamer}
\mode<presentation>
\usetheme[style=\PresentationStyle]{jagiellonian}





% ------------------------------------------------------------------------------------
% Procesing configuration files of Jagiellonian theme located in
% the directory "preambule"
% ------------------------------------------------------------------------------------
% Configuration for polish language
% Need description
\usepackage[polish]{babel}
% Need description
\usepackage[MeX]{polski}



% ------------------------------
% Better support of polish chars in technical parts of PDF
% ------------------------------
\hypersetup{pdfencoding=auto,psdextra}

% Package "textpos" give as enviroment "textblock" which is very usefull in
% arranging text on slides.

% This is standard configuration of "textpos"
\usepackage[overlay,absolute]{textpos}

% If you need to see bounds of "textblock's" comment line above and uncomment
% one below.

% Caution! When showboxes option is on significant ammunt of space is add
% to the top of textblock and as such, everyting put in them gone down.
% We need to check how to remove this bug.

% \usepackage[showboxes,overlay,absolute]{textpos}



% Setting scale length for package "textpos"
\setlength{\TPHorizModule}{10mm}
\setlength{\TPVertModule}{\TPHorizModule}


% ---------------------------------------
% TikZ
% ---------------------------------------
% Importing TikZ libraries
\usetikzlibrary{arrows.meta}
\usetikzlibrary{positioning}





% % Configuration package "bm" that need for making bold symbols
% \newcommand{\bmmax}{0}
% \newcommand{\hmmax}{0}
% \usepackage{bm}




% ---------------------------------------
% Packages for scientific texts
% ---------------------------------------
% \let\lll\undefined  % Sometimes you must use this line to allow
% "amsmath" package to works with packages with packages for polish
% languge imported
% /preambul/LanguageSettings/JagiellonianPolishLanguageSettings.tex.
% This comments (probably) removes polish letter Ł.
\usepackage{amsmath}  % Packages from American Mathematical Society (AMS)
\usepackage{amssymb}
\usepackage{amscd}
\usepackage{amsthm}
\usepackage{siunitx}  % Package for typsetting SI units.
\usepackage{upgreek}  % Better looking greek letters.
% Example of using upgreek: pi = \uppi


\usepackage{calrsfs}  % Zmienia czcionkę kaligraficzną w \mathcal
% na ładniejszą. Może w innych miejscach robi to samo, ale o tym nic
% nie wiem.










% ---------------------------------------
% Packages written for lectures "Geometria 3D dla twórców gier wideo"
% ---------------------------------------
% \usepackage{./ProgramowanieSymulacjiFizykiPaczki/ProgramowanieSymulacjiFizyki}
% \usepackage{./ProgramowanieSymulacjiFizykiPaczki/ProgramowanieSymulacjiFizykiIndeksy}
% \usepackage{./ProgramowanieSymulacjiFizykiPaczki/ProgramowanieSymulacjiFizykiTikZStyle}





% !!!!!!!!!!!!!!!!!!!!!!!!!!!!!!
% !!!!!!!!!!!!!!!!!!!!!!!!!!!!!!
% EVIL STUFF
\if\JUlogotitle1
\edef\LogoJUPath{LogoJU_\JUlogoLang/LogoJU_\JUlogoShape_\JUlogoColor.pdf}
\titlegraphic{\hfill\includegraphics[scale=0.22]
{./JagiellonianPictures/\LogoJUPath}}
\fi
% ---------------------------------------
% Commands for handling colors
% ---------------------------------------


% Command for setting normal text color for some text in math modestyle
% Text color depend on used style of Jagiellonian

% Beamer version of command
\newcommand{\TextWithNormalTextColor}[1]{%
  {\color{jNormalTextFGColor}
    \setbeamercolor{math text}{fg=jNormalTextFGColor} {#1}}
}

% Article and similar classes version of command
% \newcommand{\TextWithNormalTextColor}[1]{%
%   {\color{jNormalTextsFGColor} {#1}}
% }



% Beamer version of command
\newcommand{\NormalTextInMathMode}[1]{%
  {\color{jNormalTextFGColor}
    \setbeamercolor{math text}{fg=jNormalTextFGColor} \text{#1}}
}


% Article and similar classes version of command
% \newcommand{\NormalTextInMathMode}[1]{%
%   {\color{jNormalTextsFGColor} \text{#1}}
% }




% Command that sets color of some mathematical text to the same color
% that has normal text in header (?)

% Beamer version of the command
\newcommand{\MathTextFrametitleFGColor}[1]{%
  {\color{jFrametitleFGColor}
    \setbeamercolor{math text}{fg=jFrametitleFGColor} #1}
}

% Article and similar classes version of the command
% \newcommand{\MathTextWhiteColor}[1]{{\color{jFrametitleFGColor} #1}}





% Command for setting color of alert text for some text in math modestyle

% Beamer version of the command
\newcommand{\MathTextAlertColor}[1]{%
  {\color{jOrange} \setbeamercolor{math text}{fg=jOrange} #1}
}

% Article and similar classes version of the command
% \newcommand{\MathTextAlertColor}[1]{{\color{jOrange} #1}}





% Command that allow you to sets chosen color as the color of some text into
% math mode. Due to some nuances in the way that Beamer handle colors
% it not work in all cases. We hope that in the future we will improve it.

% Beamer version of the command
\newcommand{\SetMathTextColor}[2]{%
  {\color{#1} \setbeamercolor{math text}{fg=#1} #2}
}


% Article and similar classes version of the command
% \newcommand{\SetMathTextColor}[2]{{\color{#1} #2}}










% ---------------------------------------
% Commands for few special slides
% ---------------------------------------
\newcommand{\EndingSlide}[1]{%
  \begin{frame}[standout]

    \begingroup

    \color{jFrametitleFGColor}

    #1

    \endgroup

  \end{frame}
}










% ---------------------------------------
% Commands for setting background pictures for some slides
% ---------------------------------------
\newcommand{\TitleBackgroundPicture}
{./JagiellonianPictures/Backgrounds/LajkonikDark.png}
\newcommand{\SectionBackgroundPicture}
{./JagiellonianPictures/Backgrounds/LajkonikLight.png}



\newcommand{\TitleSlideWithPicture}{%
  \begingroup

  \usebackgroundtemplate{%
    \includegraphics[height=\paperheight]{\TitleBackgroundPicture}}

  \maketitle

  \endgroup
}





\newcommand{\SectionSlideWithPicture}[1]{%
  \begingroup

  \usebackgroundtemplate{%
    \includegraphics[height=\paperheight]{\SectionBackgroundPicture}}

  \setbeamercolor{titlelike}{fg=normal text.fg}

  \section{#1}

  \endgroup
}










% ---------------------------------------
% Commands for lectures "Geometria 3D dla twórców gier wideo"
% Polish version
% ---------------------------------------
% Komendy teraz wykomentowane były potrzebne, gdy loga były na niebieskim
% tle, nie na białym. A są na białym bo tego chcieli w biurze projektu.
% \newcommand{\FundingLogoWhitePicturePL}
% {./PresentationPictures/CommonPictures/logotypFundusze_biale_bez_tla2.pdf}
\newcommand{\FundingLogoColorPicturePL}
{./PresentationPictures/CommonPictures/European_Funds_color_PL.pdf}
% \newcommand{\EULogoWhitePicturePL}
% {./PresentationPictures/CommonPictures/logotypUE_biale_bez_tla2.pdf}
\newcommand{\EUSocialFundLogoColorPicturePL}
{./PresentationPictures/CommonPictures/EU_Social_Fund_color_PL.pdf}
% \newcommand{\ZintegrUJLogoWhitePicturePL}
% {./PresentationPictures/CommonPictures/zintegruj-logo-white.pdf}
\newcommand{\ZintegrUJLogoColorPicturePL}
{./PresentationPictures/CommonPictures/ZintegrUJ_color.pdf}
\newcommand{\JULogoColorPicturePL}
{./JagiellonianPictures/LogoJU_PL/LogoJU_A_color.pdf}





\newcommand{\GeometryThreeDSpecialBeginningSlidePL}{%
  \begin{frame}[standout]

    \begin{textblock}{11}(1,0.7)

      \begin{flushleft}

        \mdseries

        \footnotesize

        \color{jFrametitleFGColor}

        Materiał powstał w ramach projektu współfinansowanego ze środków
        Unii Europejskiej w ramach Europejskiego Funduszu Społecznego
        POWR.03.05.00-00-Z309/17-00.

      \end{flushleft}

    \end{textblock}





    \begin{textblock}{10}(0,2.2)

      \tikz \fill[color=jBackgroundStyleLight] (0,0) rectangle (12.8,-1.5);

    \end{textblock}


    \begin{textblock}{3.2}(1,2.45)

      \includegraphics[scale=0.3]{\FundingLogoColorPicturePL}

    \end{textblock}


    \begin{textblock}{2.5}(3.7,2.5)

      \includegraphics[scale=0.2]{\JULogoColorPicturePL}

    \end{textblock}


    \begin{textblock}{2.5}(6,2.4)

      \includegraphics[scale=0.1]{\ZintegrUJLogoColorPicturePL}

    \end{textblock}


    \begin{textblock}{4.2}(8.4,2.6)

      \includegraphics[scale=0.3]{\EUSocialFundLogoColorPicturePL}

    \end{textblock}

  \end{frame}
}



\newcommand{\GeometryThreeDTwoSpecialBeginningSlidesPL}{%
  \begin{frame}[standout]

    \begin{textblock}{11}(1,0.7)

      \begin{flushleft}

        \mdseries

        \footnotesize

        \color{jFrametitleFGColor}

        Materiał powstał w ramach projektu współfinansowanego ze środków
        Unii Europejskiej w ramach Europejskiego Funduszu Społecznego
        POWR.03.05.00-00-Z309/17-00.

      \end{flushleft}

    \end{textblock}





    \begin{textblock}{10}(0,2.2)

      \tikz \fill[color=jBackgroundStyleLight] (0,0) rectangle (12.8,-1.5);

    \end{textblock}


    \begin{textblock}{3.2}(1,2.45)

      \includegraphics[scale=0.3]{\FundingLogoColorPicturePL}

    \end{textblock}


    \begin{textblock}{2.5}(3.7,2.5)

      \includegraphics[scale=0.2]{\JULogoColorPicturePL}

    \end{textblock}


    \begin{textblock}{2.5}(6,2.4)

      \includegraphics[scale=0.1]{\ZintegrUJLogoColorPicturePL}

    \end{textblock}


    \begin{textblock}{4.2}(8.4,2.6)

      \includegraphics[scale=0.3]{\EUSocialFundLogoColorPicturePL}

    \end{textblock}

  \end{frame}





  \TitleSlideWithPicture
}



\newcommand{\GeometryThreeDSpecialEndingSlidePL}{%
  \begin{frame}[standout]

    \begin{textblock}{11}(1,0.7)

      \begin{flushleft}

        \mdseries

        \footnotesize

        \color{jFrametitleFGColor}

        Materiał powstał w ramach projektu współfinansowanego ze środków
        Unii Europejskiej w~ramach Europejskiego Funduszu Społecznego
        POWR.03.05.00-00-Z309/17-00.

      \end{flushleft}

    \end{textblock}





    \begin{textblock}{10}(0,2.2)

      \tikz \fill[color=jBackgroundStyleLight] (0,0) rectangle (12.8,-1.5);

    \end{textblock}


    \begin{textblock}{3.2}(1,2.45)

      \includegraphics[scale=0.3]{\FundingLogoColorPicturePL}

    \end{textblock}


    \begin{textblock}{2.5}(3.7,2.5)

      \includegraphics[scale=0.2]{\JULogoColorPicturePL}

    \end{textblock}


    \begin{textblock}{2.5}(6,2.4)

      \includegraphics[scale=0.1]{\ZintegrUJLogoColorPicturePL}

    \end{textblock}


    \begin{textblock}{4.2}(8.4,2.6)

      \includegraphics[scale=0.3]{\EUSocialFundLogoColorPicturePL}

    \end{textblock}





    \begin{textblock}{11}(1,4)

      \begin{flushleft}

        \mdseries

        \footnotesize

        \RaggedRight

        \color{jFrametitleFGColor}

        Treść niniejszego wykładu jest udostępniona na~licencji
        Creative Commons (\textsc{cc}), z~uzna\-niem autorstwa
        (\textsc{by}) oraz udostępnianiem na tych samych warunkach
        (\textsc{sa}). Rysunki i~wy\-kresy zawarte w~wykładzie są
        autorstwa dr.~hab.~Pawła Węgrzyna et~al. i~są dostępne
        na tej samej licencji, o~ile nie wskazano inaczej.
        W~prezentacji wykorzystano temat Beamera Jagiellonian,
        oparty na~temacie Metropolis Matthiasa Vogelgesanga,
        dostępnym na licencji \LaTeX{} Project Public License~1.3c
        pod adresem: \colorhref{https://github.com/matze/mtheme}
        {https://github.com/matze/mtheme}.

        Projekt typograficzny: Iwona Grabska-Gradzińska \\
        Skład: Kamil Ziemian;
        Korekta: Wojciech Palacz \\
        Modele: Dariusz Frymus, Kamil Nowakowski \\
        Rysunki i~wykresy: Kamil Ziemian, Paweł Węgrzyn, Wojciech Palacz

      \end{flushleft}

    \end{textblock}

  \end{frame}
}



\newcommand{\GeometryThreeDTwoSpecialEndingSlidesPL}[1]{%
  \begin{frame}[standout]


    \begin{textblock}{11}(1,0.7)

      \begin{flushleft}

        \mdseries

        \footnotesize

        \color{jFrametitleFGColor}

        Materiał powstał w ramach projektu współfinansowanego ze środków
        Unii Europejskiej w~ramach Europejskiego Funduszu Społecznego
        POWR.03.05.00-00-Z309/17-00.

      \end{flushleft}

    \end{textblock}





    \begin{textblock}{10}(0,2.2)

      \tikz \fill[color=jBackgroundStyleLight] (0,0) rectangle (12.8,-1.5);

    \end{textblock}


    \begin{textblock}{3.2}(1,2.45)

      \includegraphics[scale=0.3]{\FundingLogoColorPicturePL}

    \end{textblock}


    \begin{textblock}{2.5}(3.7,2.5)

      \includegraphics[scale=0.2]{\JULogoColorPicturePL}

    \end{textblock}


    \begin{textblock}{2.5}(6,2.4)

      \includegraphics[scale=0.1]{\ZintegrUJLogoColorPicturePL}

    \end{textblock}


    \begin{textblock}{4.2}(8.4,2.6)

      \includegraphics[scale=0.3]{\EUSocialFundLogoColorPicturePL}

    \end{textblock}





    \begin{textblock}{11}(1,4)

      \begin{flushleft}

        \mdseries

        \footnotesize

        \RaggedRight

        \color{jFrametitleFGColor}

        Treść niniejszego wykładu jest udostępniona na~licencji
        Creative Commons (\textsc{cc}), z~uzna\-niem autorstwa
        (\textsc{by}) oraz udostępnianiem na tych samych warunkach
        (\textsc{sa}). Rysunki i~wy\-kresy zawarte w~wykładzie są
        autorstwa dr.~hab.~Pawła Węgrzyna et~al. i~są dostępne
        na tej samej licencji, o~ile nie wskazano inaczej.
        W~prezentacji wykorzystano temat Beamera Jagiellonian,
        oparty na~temacie Metropolis Matthiasa Vogelgesanga,
        dostępnym na licencji \LaTeX{} Project Public License~1.3c
        pod adresem: \colorhref{https://github.com/matze/mtheme}
        {https://github.com/matze/mtheme}.

        Projekt typograficzny: Iwona Grabska-Gradzińska \\
        Skład: Kamil Ziemian;
        Korekta: Wojciech Palacz \\
        Modele: Dariusz Frymus, Kamil Nowakowski \\
        Rysunki i~wykresy: Kamil Ziemian, Paweł Węgrzyn, Wojciech Palacz

      \end{flushleft}

    \end{textblock}

  \end{frame}





  \begin{frame}[standout]

    \begingroup

    \color{jFrametitleFGColor}

    #1

    \endgroup

  \end{frame}
}



\newcommand{\GeometryThreeDSpecialEndingSlideVideoPL}{%
  \begin{frame}[standout]

    \begin{textblock}{11}(1,0.7)

      \begin{flushleft}

        \mdseries

        \footnotesize

        \color{jFrametitleFGColor}

        Materiał powstał w ramach projektu współfinansowanego ze środków
        Unii Europejskiej w~ramach Europejskiego Funduszu Społecznego
        POWR.03.05.00-00-Z309/17-00.

      \end{flushleft}

    \end{textblock}





    \begin{textblock}{10}(0,2.2)

      \tikz \fill[color=jBackgroundStyleLight] (0,0) rectangle (12.8,-1.5);

    \end{textblock}


    \begin{textblock}{3.2}(1,2.45)

      \includegraphics[scale=0.3]{\FundingLogoColorPicturePL}

    \end{textblock}


    \begin{textblock}{2.5}(3.7,2.5)

      \includegraphics[scale=0.2]{\JULogoColorPicturePL}

    \end{textblock}


    \begin{textblock}{2.5}(6,2.4)

      \includegraphics[scale=0.1]{\ZintegrUJLogoColorPicturePL}

    \end{textblock}


    \begin{textblock}{4.2}(8.4,2.6)

      \includegraphics[scale=0.3]{\EUSocialFundLogoColorPicturePL}

    \end{textblock}





    \begin{textblock}{11}(1,4)

      \begin{flushleft}

        \mdseries

        \footnotesize

        \RaggedRight

        \color{jFrametitleFGColor}

        Treść niniejszego wykładu jest udostępniona na~licencji
        Creative Commons (\textsc{cc}), z~uzna\-niem autorstwa
        (\textsc{by}) oraz udostępnianiem na tych samych warunkach
        (\textsc{sa}). Rysunki i~wy\-kresy zawarte w~wykładzie są
        autorstwa dr.~hab.~Pawła Węgrzyna et~al. i~są dostępne
        na tej samej licencji, o~ile nie wskazano inaczej.
        W~prezentacji wykorzystano temat Beamera Jagiellonian,
        oparty na~temacie Metropolis Matthiasa Vogelgesanga,
        dostępnym na licencji \LaTeX{} Project Public License~1.3c
        pod adresem: \colorhref{https://github.com/matze/mtheme}
        {https://github.com/matze/mtheme}.

        Projekt typograficzny: Iwona Grabska-Gradzińska;
        Skład: Kamil Ziemian \\
        Korekta: Wojciech Palacz;
        Modele: Dariusz Frymus, Kamil Nowakowski \\
        Rysunki i~wykresy: Kamil Ziemian, Paweł Węgrzyn, Wojciech Palacz \\
        Montaż: Agencja Filmowa Film \& Television Production~-- Zbigniew
        Masklak

      \end{flushleft}

    \end{textblock}

  \end{frame}
}





\newcommand{\GeometryThreeDTwoSpecialEndingSlidesVideoPL}[1]{%
  \begin{frame}[standout]

    \begin{textblock}{11}(1,0.7)

      \begin{flushleft}

        \mdseries

        \footnotesize

        \color{jFrametitleFGColor}

        Materiał powstał w ramach projektu współfinansowanego ze środków
        Unii Europejskiej w~ramach Europejskiego Funduszu Społecznego
        POWR.03.05.00-00-Z309/17-00.

      \end{flushleft}

    \end{textblock}





    \begin{textblock}{10}(0,2.2)

      \tikz \fill[color=jBackgroundStyleLight] (0,0) rectangle (12.8,-1.5);

    \end{textblock}


    \begin{textblock}{3.2}(1,2.45)

      \includegraphics[scale=0.3]{\FundingLogoColorPicturePL}

    \end{textblock}


    \begin{textblock}{2.5}(3.7,2.5)

      \includegraphics[scale=0.2]{\JULogoColorPicturePL}

    \end{textblock}


    \begin{textblock}{2.5}(6,2.4)

      \includegraphics[scale=0.1]{\ZintegrUJLogoColorPicturePL}

    \end{textblock}


    \begin{textblock}{4.2}(8.4,2.6)

      \includegraphics[scale=0.3]{\EUSocialFundLogoColorPicturePL}

    \end{textblock}





    \begin{textblock}{11}(1,4)

      \begin{flushleft}

        \mdseries

        \footnotesize

        \RaggedRight

        \color{jFrametitleFGColor}

        Treść niniejszego wykładu jest udostępniona na~licencji
        Creative Commons (\textsc{cc}), z~uzna\-niem autorstwa
        (\textsc{by}) oraz udostępnianiem na tych samych warunkach
        (\textsc{sa}). Rysunki i~wy\-kresy zawarte w~wykładzie są
        autorstwa dr.~hab.~Pawła Węgrzyna et~al. i~są dostępne
        na tej samej licencji, o~ile nie wskazano inaczej.
        W~prezentacji wykorzystano temat Beamera Jagiellonian,
        oparty na~temacie Metropolis Matthiasa Vogelgesanga,
        dostępnym na licencji \LaTeX{} Project Public License~1.3c
        pod adresem: \colorhref{https://github.com/matze/mtheme}
        {https://github.com/matze/mtheme}.

        Projekt typograficzny: Iwona Grabska-Gradzińska;
        Skład: Kamil Ziemian \\
        Korekta: Wojciech Palacz;
        Modele: Dariusz Frymus, Kamil Nowakowski \\
        Rysunki i~wykresy: Kamil Ziemian, Paweł Węgrzyn, Wojciech Palacz \\
        Montaż: Agencja Filmowa Film \& Television Production~-- Zbigniew
        Masklak

      \end{flushleft}

    \end{textblock}

  \end{frame}





  \begin{frame}[standout]


    \begingroup

    \color{jFrametitleFGColor}

    #1

    \endgroup

  \end{frame}
}










% ---------------------------------------
% Commands for lectures "Geometria 3D dla twórców gier wideo"
% English version
% ---------------------------------------
% \newcommand{\FundingLogoWhitePictureEN}
% {./PresentationPictures/CommonPictures/logotypFundusze_biale_bez_tla2.pdf}
\newcommand{\FundingLogoColorPictureEN}
{./PresentationPictures/CommonPictures/European_Funds_color_EN.pdf}
% \newcommand{\EULogoWhitePictureEN}
% {./PresentationPictures/CommonPictures/logotypUE_biale_bez_tla2.pdf}
\newcommand{\EUSocialFundLogoColorPictureEN}
{./PresentationPictures/CommonPictures/EU_Social_Fund_color_EN.pdf}
% \newcommand{\ZintegrUJLogoWhitePictureEN}
% {./PresentationPictures/CommonPictures/zintegruj-logo-white.pdf}
\newcommand{\ZintegrUJLogoColorPictureEN}
{./PresentationPictures/CommonPictures/ZintegrUJ_color.pdf}
\newcommand{\JULogoColorPictureEN}
{./JagiellonianPictures/LogoJU_EN/LogoJU_A_color.pdf}



\newcommand{\GeometryThreeDSpecialBeginningSlideEN}{%
  \begin{frame}[standout]

    \begin{textblock}{11}(1,0.7)

      \begin{flushleft}

        \mdseries

        \footnotesize

        \color{jFrametitleFGColor}

        This content was created as part of a project co-financed by the
        European Union within the framework of the European Social Fund
        POWR.03.05.00-00-Z309/17-00.

      \end{flushleft}

    \end{textblock}





    \begin{textblock}{10}(0,2.2)

      \tikz \fill[color=jBackgroundStyleLight] (0,0) rectangle (12.8,-1.5);

    \end{textblock}


    \begin{textblock}{3.2}(0.7,2.45)

      \includegraphics[scale=0.3]{\FundingLogoColorPictureEN}

    \end{textblock}


    \begin{textblock}{2.5}(4.15,2.5)

      \includegraphics[scale=0.2]{\JULogoColorPictureEN}

    \end{textblock}


    \begin{textblock}{2.5}(6.35,2.4)

      \includegraphics[scale=0.1]{\ZintegrUJLogoColorPictureEN}

    \end{textblock}


    \begin{textblock}{4.2}(8.4,2.6)

      \includegraphics[scale=0.3]{\EUSocialFundLogoColorPictureEN}

    \end{textblock}

  \end{frame}
}



\newcommand{\GeometryThreeDTwoSpecialBeginningSlidesEN}{%
  \begin{frame}[standout]

    \begin{textblock}{11}(1,0.7)

      \begin{flushleft}

        \mdseries

        \footnotesize

        \color{jFrametitleFGColor}

        This content was created as part of a project co-financed by the
        European Union within the framework of the European Social Fund
        POWR.03.05.00-00-Z309/17-00.

      \end{flushleft}

    \end{textblock}





    \begin{textblock}{10}(0,2.2)

      \tikz \fill[color=jBackgroundStyleLight] (0,0) rectangle (12.8,-1.5);

    \end{textblock}


    \begin{textblock}{3.2}(0.7,2.45)

      \includegraphics[scale=0.3]{\FundingLogoColorPictureEN}

    \end{textblock}


    \begin{textblock}{2.5}(4.15,2.5)

      \includegraphics[scale=0.2]{\JULogoColorPictureEN}

    \end{textblock}


    \begin{textblock}{2.5}(6.35,2.4)

      \includegraphics[scale=0.1]{\ZintegrUJLogoColorPictureEN}

    \end{textblock}


    \begin{textblock}{4.2}(8.4,2.6)

      \includegraphics[scale=0.3]{\EUSocialFundLogoColorPictureEN}

    \end{textblock}

  \end{frame}





  \TitleSlideWithPicture
}



\newcommand{\GeometryThreeDSpecialEndingSlideEN}{%
  \begin{frame}[standout]

    \begin{textblock}{11}(1,0.7)

      \begin{flushleft}

        \mdseries

        \footnotesize

        \color{jFrametitleFGColor}

        This content was created as part of a project co-financed by the
        European Union within the framework of the European Social Fund
        POWR.03.05.00-00-Z309/17-00.

      \end{flushleft}

    \end{textblock}





    \begin{textblock}{10}(0,2.2)

      \tikz \fill[color=jBackgroundStyleLight] (0,0) rectangle (12.8,-1.5);

    \end{textblock}


    \begin{textblock}{3.2}(0.7,2.45)

      \includegraphics[scale=0.3]{\FundingLogoColorPictureEN}

    \end{textblock}


    \begin{textblock}{2.5}(4.15,2.5)

      \includegraphics[scale=0.2]{\JULogoColorPictureEN}

    \end{textblock}


    \begin{textblock}{2.5}(6.35,2.4)

      \includegraphics[scale=0.1]{\ZintegrUJLogoColorPictureEN}

    \end{textblock}


    \begin{textblock}{4.2}(8.4,2.6)

      \includegraphics[scale=0.3]{\EUSocialFundLogoColorPictureEN}

    \end{textblock}





    \begin{textblock}{11}(1,4)

      \begin{flushleft}

        \mdseries

        \footnotesize

        \RaggedRight

        \color{jFrametitleFGColor}

        The content of this lecture is made available under a~Creative
        Commons licence (\textsc{cc}), giving the author the credits
        (\textsc{by}) and putting an obligation to share on the same terms
        (\textsc{sa}). Figures and diagrams included in the lecture are
        authored by Paweł Węgrzyn et~al., and are available under the same
        license unless indicated otherwise.\\ The presentation uses the
        Beamer Jagiellonian theme based on Matthias Vogelgesang’s
        Metropolis theme, available under license \LaTeX{} Project
        Public License~1.3c at: \colorhref{https://github.com/matze/mtheme}
        {https://github.com/matze/mtheme}.

        Typographic design: Iwona Grabska-Gradzińska \\
        \LaTeX{} Typesetting: Kamil Ziemian \\
        Proofreading: Wojciech Palacz,
        Monika Stawicka \\
        3D Models: Dariusz Frymus, Kamil Nowakowski \\
        Figures and charts: Kamil Ziemian, Paweł Węgrzyn, Wojciech Palacz

      \end{flushleft}

    \end{textblock}

  \end{frame}
}



\newcommand{\GeometryThreeDTwoSpecialEndingSlidesEN}[1]{%
  \begin{frame}[standout]


    \begin{textblock}{11}(1,0.7)

      \begin{flushleft}

        \mdseries

        \footnotesize

        \color{jFrametitleFGColor}

        This content was created as part of a project co-financed by the
        European Union within the framework of the European Social Fund
        POWR.03.05.00-00-Z309/17-00.

      \end{flushleft}

    \end{textblock}





    \begin{textblock}{10}(0,2.2)

      \tikz \fill[color=jBackgroundStyleLight] (0,0) rectangle (12.8,-1.5);

    \end{textblock}


    \begin{textblock}{3.2}(0.7,2.45)

      \includegraphics[scale=0.3]{\FundingLogoColorPictureEN}

    \end{textblock}


    \begin{textblock}{2.5}(4.15,2.5)

      \includegraphics[scale=0.2]{\JULogoColorPictureEN}

    \end{textblock}


    \begin{textblock}{2.5}(6.35,2.4)

      \includegraphics[scale=0.1]{\ZintegrUJLogoColorPictureEN}

    \end{textblock}


    \begin{textblock}{4.2}(8.4,2.6)

      \includegraphics[scale=0.3]{\EUSocialFundLogoColorPictureEN}

    \end{textblock}





    \begin{textblock}{11}(1,4)

      \begin{flushleft}

        \mdseries

        \footnotesize

        \RaggedRight

        \color{jFrametitleFGColor}

        The content of this lecture is made available under a~Creative
        Commons licence (\textsc{cc}), giving the author the credits
        (\textsc{by}) and putting an obligation to share on the same terms
        (\textsc{sa}). Figures and diagrams included in the lecture are
        authored by Paweł Węgrzyn et~al., and are available under the same
        license unless indicated otherwise.\\ The presentation uses the
        Beamer Jagiellonian theme based on Matthias Vogelgesang’s
        Metropolis theme, available under license \LaTeX{} Project
        Public License~1.3c at: \colorhref{https://github.com/matze/mtheme}
        {https://github.com/matze/mtheme}.

        Typographic design: Iwona Grabska-Gradzińska \\
        \LaTeX{} Typesetting: Kamil Ziemian \\
        Proofreading: Wojciech Palacz,
        Monika Stawicka \\
        3D Models: Dariusz Frymus, Kamil Nowakowski \\
        Figures and charts: Kamil Ziemian, Paweł Węgrzyn, Wojciech Palacz

      \end{flushleft}

    \end{textblock}

  \end{frame}





  \begin{frame}[standout]

    \begingroup

    \color{jFrametitleFGColor}

    #1

    \endgroup

  \end{frame}
}



\newcommand{\GeometryThreeDSpecialEndingSlideVideoVerOneEN}{%
  \begin{frame}[standout]

    \begin{textblock}{11}(1,0.7)

      \begin{flushleft}

        \mdseries

        \footnotesize

        \color{jFrametitleFGColor}

        This content was created as part of a project co-financed by the
        European Union within the framework of the European Social Fund
        POWR.03.05.00-00-Z309/17-00.

      \end{flushleft}

    \end{textblock}





    \begin{textblock}{10}(0,2.2)

      \tikz \fill[color=jBackgroundStyleLight] (0,0) rectangle (12.8,-1.5);

    \end{textblock}


    \begin{textblock}{3.2}(0.7,2.45)

      \includegraphics[scale=0.3]{\FundingLogoColorPictureEN}

    \end{textblock}


    \begin{textblock}{2.5}(4.15,2.5)

      \includegraphics[scale=0.2]{\JULogoColorPictureEN}

    \end{textblock}


    \begin{textblock}{2.5}(6.35,2.4)

      \includegraphics[scale=0.1]{\ZintegrUJLogoColorPictureEN}

    \end{textblock}


    \begin{textblock}{4.2}(8.4,2.6)

      \includegraphics[scale=0.3]{\EUSocialFundLogoColorPictureEN}

    \end{textblock}





    \begin{textblock}{11}(1,4)

      \begin{flushleft}

        \mdseries

        \footnotesize

        \RaggedRight

        \color{jFrametitleFGColor}

        The content of this lecture is made available under a Creative
        Commons licence (\textsc{cc}), giving the author the credits
        (\textsc{by}) and putting an obligation to share on the same terms
        (\textsc{sa}). Figures and diagrams included in the lecture are
        authored by Paweł Węgrzyn et~al., and are available under the same
        license unless indicated otherwise.\\ The presentation uses the
        Beamer Jagiellonian theme based on Matthias Vogelgesang’s
        Metropolis theme, available under license \LaTeX{} Project
        Public License~1.3c at: \colorhref{https://github.com/matze/mtheme}
        {https://github.com/matze/mtheme}.

        Typographic design: Iwona Grabska-Gradzińska;
        \LaTeX{} Typesetting: Kamil Ziemian \\
        Proofreading: Wojciech Palacz,
        Monika Stawicka \\
        3D Models: Dariusz Frymus, Kamil Nowakowski \\
        Figures and charts: Kamil Ziemian, Paweł Węgrzyn, Wojciech
        Palacz \\
        Film editing: Agencja Filmowa Film \& Television Production~--
        Zbigniew Masklak

      \end{flushleft}

    \end{textblock}

  \end{frame}
}



\newcommand{\GeometryThreeDSpecialEndingSlideVideoVerTwoEN}{%
  \begin{frame}[standout]

    \begin{textblock}{11}(1,0.7)

      \begin{flushleft}

        \mdseries

        \footnotesize

        \color{jFrametitleFGColor}

        This content was created as part of a project co-financed by the
        European Union within the framework of the European Social Fund
        POWR.03.05.00-00-Z309/17-00.

      \end{flushleft}

    \end{textblock}





    \begin{textblock}{10}(0,2.2)

      \tikz \fill[color=jBackgroundStyleLight] (0,0) rectangle (12.8,-1.5);

    \end{textblock}


    \begin{textblock}{3.2}(0.7,2.45)

      \includegraphics[scale=0.3]{\FundingLogoColorPictureEN}

    \end{textblock}


    \begin{textblock}{2.5}(4.15,2.5)

      \includegraphics[scale=0.2]{\JULogoColorPictureEN}

    \end{textblock}


    \begin{textblock}{2.5}(6.35,2.4)

      \includegraphics[scale=0.1]{\ZintegrUJLogoColorPictureEN}

    \end{textblock}


    \begin{textblock}{4.2}(8.4,2.6)

      \includegraphics[scale=0.3]{\EUSocialFundLogoColorPictureEN}

    \end{textblock}





    \begin{textblock}{11}(1,4)

      \begin{flushleft}

        \mdseries

        \footnotesize

        \RaggedRight

        \color{jFrametitleFGColor}

        The content of this lecture is made available under a Creative
        Commons licence (\textsc{cc}), giving the author the credits
        (\textsc{by}) and putting an obligation to share on the same terms
        (\textsc{sa}). Figures and diagrams included in the lecture are
        authored by Paweł Węgrzyn et~al., and are available under the same
        license unless indicated otherwise.\\ The presentation uses the
        Beamer Jagiellonian theme based on Matthias Vogelgesang’s
        Metropolis theme, available under license \LaTeX{} Project
        Public License~1.3c at: \colorhref{https://github.com/matze/mtheme}
        {https://github.com/matze/mtheme}.

        Typographic design: Iwona Grabska-Gradzińska;
        \LaTeX{} Typesetting: Kamil Ziemian \\
        Proofreading: Wojciech Palacz,
        Monika Stawicka \\
        3D Models: Dariusz Frymus, Kamil Nowakowski \\
        Figures and charts: Kamil Ziemian, Paweł Węgrzyn, Wojciech
        Palacz \\
        Film editing: IMAVI -- Joanna Kozakiewicz, Krzysztof Magda, Nikodem
        Frodyma

      \end{flushleft}

    \end{textblock}

  \end{frame}
}



\newcommand{\GeometryThreeDSpecialEndingSlideVideoVerThreeEN}{%
  \begin{frame}[standout]

    \begin{textblock}{11}(1,0.7)

      \begin{flushleft}

        \mdseries

        \footnotesize

        \color{jFrametitleFGColor}

        This content was created as part of a project co-financed by the
        European Union within the framework of the European Social Fund
        POWR.03.05.00-00-Z309/17-00.

      \end{flushleft}

    \end{textblock}





    \begin{textblock}{10}(0,2.2)

      \tikz \fill[color=jBackgroundStyleLight] (0,0) rectangle (12.8,-1.5);

    \end{textblock}


    \begin{textblock}{3.2}(0.7,2.45)

      \includegraphics[scale=0.3]{\FundingLogoColorPictureEN}

    \end{textblock}


    \begin{textblock}{2.5}(4.15,2.5)

      \includegraphics[scale=0.2]{\JULogoColorPictureEN}

    \end{textblock}


    \begin{textblock}{2.5}(6.35,2.4)

      \includegraphics[scale=0.1]{\ZintegrUJLogoColorPictureEN}

    \end{textblock}


    \begin{textblock}{4.2}(8.4,2.6)

      \includegraphics[scale=0.3]{\EUSocialFundLogoColorPictureEN}

    \end{textblock}





    \begin{textblock}{11}(1,4)

      \begin{flushleft}

        \mdseries

        \footnotesize

        \RaggedRight

        \color{jFrametitleFGColor}

        The content of this lecture is made available under a Creative
        Commons licence (\textsc{cc}), giving the author the credits
        (\textsc{by}) and putting an obligation to share on the same terms
        (\textsc{sa}). Figures and diagrams included in the lecture are
        authored by Paweł Węgrzyn et~al., and are available under the same
        license unless indicated otherwise.\\ The presentation uses the
        Beamer Jagiellonian theme based on Matthias Vogelgesang’s
        Metropolis theme, available under license \LaTeX{} Project
        Public License~1.3c at: \colorhref{https://github.com/matze/mtheme}
        {https://github.com/matze/mtheme}.

        Typographic design: Iwona Grabska-Gradzińska;
        \LaTeX{} Typesetting: Kamil Ziemian \\
        Proofreading: Wojciech Palacz,
        Monika Stawicka \\
        3D Models: Dariusz Frymus, Kamil Nowakowski \\
        Figures and charts: Kamil Ziemian, Paweł Węgrzyn, Wojciech
        Palacz \\
        Film editing: Agencja Filmowa Film \& Television Production~--
        Zbigniew Masklak \\
        Film editing: IMAVI -- Joanna Kozakiewicz, Krzysztof Magda, Nikodem
        Frodyma

      \end{flushleft}

    \end{textblock}

  \end{frame}
}



\newcommand{\GeometryThreeDTwoSpecialEndingSlidesVideoVerOneEN}[1]{%
  \begin{frame}[standout]

    \begin{textblock}{11}(1,0.7)

      \begin{flushleft}

        \mdseries

        \footnotesize

        \color{jFrametitleFGColor}

        This content was created as part of a project co-financed by the
        European Union within the framework of the European Social Fund
        POWR.03.05.00-00-Z309/17-00.

      \end{flushleft}

    \end{textblock}





    \begin{textblock}{10}(0,2.2)

      \tikz \fill[color=jBackgroundStyleLight] (0,0) rectangle (12.8,-1.5);

    \end{textblock}


    \begin{textblock}{3.2}(0.7,2.45)

      \includegraphics[scale=0.3]{\FundingLogoColorPictureEN}

    \end{textblock}


    \begin{textblock}{2.5}(4.15,2.5)

      \includegraphics[scale=0.2]{\JULogoColorPictureEN}

    \end{textblock}


    \begin{textblock}{2.5}(6.35,2.4)

      \includegraphics[scale=0.1]{\ZintegrUJLogoColorPictureEN}

    \end{textblock}


    \begin{textblock}{4.2}(8.4,2.6)

      \includegraphics[scale=0.3]{\EUSocialFundLogoColorPictureEN}

    \end{textblock}





    \begin{textblock}{11}(1,4)

      \begin{flushleft}

        \mdseries

        \footnotesize

        \RaggedRight

        \color{jFrametitleFGColor}

        The content of this lecture is made available under a Creative
        Commons licence (\textsc{cc}), giving the author the credits
        (\textsc{by}) and putting an obligation to share on the same terms
        (\textsc{sa}). Figures and diagrams included in the lecture are
        authored by Paweł Węgrzyn et~al., and are available under the same
        license unless indicated otherwise.\\ The presentation uses the
        Beamer Jagiellonian theme based on Matthias Vogelgesang’s
        Metropolis theme, available under license \LaTeX{} Project
        Public License~1.3c at: \colorhref{https://github.com/matze/mtheme}
        {https://github.com/matze/mtheme}.

        Typographic design: Iwona Grabska-Gradzińska;
        \LaTeX{} Typesetting: Kamil Ziemian \\
        Proofreading: Wojciech Palacz,
        Monika Stawicka \\
        3D Models: Dariusz Frymus, Kamil Nowakowski \\
        Figures and charts: Kamil Ziemian, Paweł Węgrzyn,
        Wojciech Palacz \\
        Film editing: Agencja Filmowa Film \& Television Production~--
        Zbigniew Masklak

      \end{flushleft}

    \end{textblock}

  \end{frame}





  \begin{frame}[standout]


    \begingroup

    \color{jFrametitleFGColor}

    #1

    \endgroup

  \end{frame}
}



\newcommand{\GeometryThreeDTwoSpecialEndingSlidesVideoVerTwoEN}[1]{%
  \begin{frame}[standout]

    \begin{textblock}{11}(1,0.7)

      \begin{flushleft}

        \mdseries

        \footnotesize

        \color{jFrametitleFGColor}

        This content was created as part of a project co-financed by the
        European Union within the framework of the European Social Fund
        POWR.03.05.00-00-Z309/17-00.

      \end{flushleft}

    \end{textblock}





    \begin{textblock}{10}(0,2.2)

      \tikz \fill[color=jBackgroundStyleLight] (0,0) rectangle (12.8,-1.5);

    \end{textblock}


    \begin{textblock}{3.2}(0.7,2.45)

      \includegraphics[scale=0.3]{\FundingLogoColorPictureEN}

    \end{textblock}


    \begin{textblock}{2.5}(4.15,2.5)

      \includegraphics[scale=0.2]{\JULogoColorPictureEN}

    \end{textblock}


    \begin{textblock}{2.5}(6.35,2.4)

      \includegraphics[scale=0.1]{\ZintegrUJLogoColorPictureEN}

    \end{textblock}


    \begin{textblock}{4.2}(8.4,2.6)

      \includegraphics[scale=0.3]{\EUSocialFundLogoColorPictureEN}

    \end{textblock}





    \begin{textblock}{11}(1,4)

      \begin{flushleft}

        \mdseries

        \footnotesize

        \RaggedRight

        \color{jFrametitleFGColor}

        The content of this lecture is made available under a Creative
        Commons licence (\textsc{cc}), giving the author the credits
        (\textsc{by}) and putting an obligation to share on the same terms
        (\textsc{sa}). Figures and diagrams included in the lecture are
        authored by Paweł Węgrzyn et~al., and are available under the same
        license unless indicated otherwise.\\ The presentation uses the
        Beamer Jagiellonian theme based on Matthias Vogelgesang’s
        Metropolis theme, available under license \LaTeX{} Project
        Public License~1.3c at: \colorhref{https://github.com/matze/mtheme}
        {https://github.com/matze/mtheme}.

        Typographic design: Iwona Grabska-Gradzińska;
        \LaTeX{} Typesetting: Kamil Ziemian \\
        Proofreading: Wojciech Palacz,
        Monika Stawicka \\
        3D Models: Dariusz Frymus, Kamil Nowakowski \\
        Figures and charts: Kamil Ziemian, Paweł Węgrzyn,
        Wojciech Palacz \\
        Film editing: IMAVI -- Joanna Kozakiewicz, Krzysztof Magda, Nikodem
        Frodyma

      \end{flushleft}

    \end{textblock}

  \end{frame}





  \begin{frame}[standout]


    \begingroup

    \color{jFrametitleFGColor}

    #1

    \endgroup

  \end{frame}
}



\newcommand{\GeometryThreeDTwoSpecialEndingSlidesVideoVerThreeEN}[1]{%
  \begin{frame}[standout]

    \begin{textblock}{11}(1,0.7)

      \begin{flushleft}

        \mdseries

        \footnotesize

        \color{jFrametitleFGColor}

        This content was created as part of a project co-financed by the
        European Union within the framework of the European Social Fund
        POWR.03.05.00-00-Z309/17-00.

      \end{flushleft}

    \end{textblock}





    \begin{textblock}{10}(0,2.2)

      \tikz \fill[color=jBackgroundStyleLight] (0,0) rectangle (12.8,-1.5);

    \end{textblock}


    \begin{textblock}{3.2}(0.7,2.45)

      \includegraphics[scale=0.3]{\FundingLogoColorPictureEN}

    \end{textblock}


    \begin{textblock}{2.5}(4.15,2.5)

      \includegraphics[scale=0.2]{\JULogoColorPictureEN}

    \end{textblock}


    \begin{textblock}{2.5}(6.35,2.4)

      \includegraphics[scale=0.1]{\ZintegrUJLogoColorPictureEN}

    \end{textblock}


    \begin{textblock}{4.2}(8.4,2.6)

      \includegraphics[scale=0.3]{\EUSocialFundLogoColorPictureEN}

    \end{textblock}





    \begin{textblock}{11}(1,4)

      \begin{flushleft}

        \mdseries

        \footnotesize

        \RaggedRight

        \color{jFrametitleFGColor}

        The content of this lecture is made available under a Creative
        Commons licence (\textsc{cc}), giving the author the credits
        (\textsc{by}) and putting an obligation to share on the same terms
        (\textsc{sa}). Figures and diagrams included in the lecture are
        authored by Paweł Węgrzyn et~al., and are available under the same
        license unless indicated otherwise. \\ The presentation uses the
        Beamer Jagiellonian theme based on Matthias Vogelgesang’s
        Metropolis theme, available under license \LaTeX{} Project
        Public License~1.3c at: \colorhref{https://github.com/matze/mtheme}
        {https://github.com/matze/mtheme}.

        Typographic design: Iwona Grabska-Gradzińska;
        \LaTeX{} Typesetting: Kamil Ziemian \\
        Proofreading: Leszek Hadasz, Wojciech Palacz,
        Monika Stawicka \\
        3D Models: Dariusz Frymus, Kamil Nowakowski \\
        Figures and charts: Kamil Ziemian, Paweł Węgrzyn,
        Wojciech Palacz \\
        Film editing: Agencja Filmowa Film \& Television Production~--
        Zbigniew Masklak \\
        Film editing: IMAVI -- Joanna Kozakiewicz, Krzysztof Magda, Nikodem
        Frodyma


      \end{flushleft}

    \end{textblock}

  \end{frame}





  \begin{frame}[standout]


    \begingroup

    \color{jFrametitleFGColor}

    #1

    \endgroup

  \end{frame}
}











% ------------------------------------------------------------------------------------
% Importing packages, libraries and setting their configuration
% ------------------------------------------------------------------------------------





% ------------------------------------------------------
% BibLaTeX
% ------------------------------------------------------
% Package biblatex, with biber as its backend, allow us to handle
% bibliography entries that use Unicode symbols outside ASCII.
\usepackage[
language=polish,
backend=biber,
style=alphabetic,
url=false,
eprint=true,
]{biblatex}

\addbibresource{Krótka-podróż-po-problemach-podczerwonych-ETC-Bibliography.bib}





% ------------------------------------------------------
% Wonderful package PGF/TikZ
% ------------------------------------------------------

% Node and pics for drawing charts
\usepackage{./Local-packages/PGF-TikZ-Chart-nodes-and-pics}

% Styles for arrows
\usepackage{./Local-packages/PGF-TikZ-Arrows-styles}





% ------------------------------------------------------
% Local packages
% ------------------------------------------------------
% Special configuration for this particular presentation
\usepackage{./Local-packages/local-settings}

% Package containing various command useful for working with a text
\usepackage{./Local-packages/general-commands}

% Package containing commands and other code useful for working with
% mathematical text
\usepackage{./Local-packages/math-commands}










% ------------------------------------------------------------------------------------------------------------------
\title{Krótki przewodni po problemach podczerwonych
  w~elektrodynamice kwantowej}

\author{Kamil Ziemian \\
  \email}


% \institute{Uniwersytet Jagielloński w~Krakowie}

\date[???]{???}
% ------------------------------------------------------------------------------------------------------------------









% ####################################################################
\begin{document}
% ####################################################################





% ######################################
% Number of chars: 47k+,
% Text is adjusted to the left and words are broken at the end of the line.
\RaggedRight
% ######################################





% ######################################
\maketitle
% ######################################





% ######################################
\begin{frame}
  \frametitle{Plan prezentacji}


  \tableofcontents

\end{frame}
% ######################################










% ######################################
\section{Problemy podczerwone i~nadfioletowe}
% ######################################



% ##################
\begin{frame}
  \frametitle{Informacji wstępne}


  Tematyka problemów podczerwonych (\textsc{ir}, ang. \textit{infrared
    problems}) dotyka wielu fizycznych zagadnień i~nie może być
  wyczerpująco przedstawiona w~ramach jednego wystąpienia. Naszym
  celem jest przedstawienie szerokiej panoramy tych zagadnień,
  gdzie~poglądowość, prostota i~zwięzłość przekazu jest dużo ważniejsza,
  niż pełna prezentacja każdego
  z~zagadnień. Będziemy też pomijać różnorakie, często bardzo ciekawe, ich
  aspekty techniczne. W~takiej sytuacji poważne uproszczenia są
  nieuniknione.

  To samo, choć w~dużo mniejszej skali, można powiedzieć o~problemach
  nadfioletowych (\textsc{uv}, ang. \textit{ultraviolet problems}).
  Musimy zaznaczyć, że~z~punktu widzenia tego wystąpienia problemy
  nadfioletowe, \alert{nie} są problemami fizycznymi, lecz kwestia
  techniczną wynikającą z~braku w~pełni matematycznie ścisłych metod
  rachunkowych w~kwantowej teorii pola. Możemy o~tym porozmawiać
  podczas~Q\&A.

  Standardowo, w~większości wzorów przyjęto „$c = 1$”
  i~„\HorSpaceFour$\hbar = 1$”.

\end{frame}
% ##################





% ##################
\begin{frame}[label=sld-Problemy-podczerwone-i-nadfioletowe]
  \frametitle{Problemy podczerowne i~nadfioletowe}


  Zaczniemy od~prostej ilustracji jak problemy podczerwone i~nadfioletowe
  przejawiają~się na poziomie reguł rachunkowych. Niech $k = 2 \pi / \lambda$
  będzie modułem wektora falowego. Załóżmy, że~metody rachunkowe danej
  teorii wymagają od nas obliczenia całki
  % Całka ma indeks „rul” od ang. „rules”, reguły rachunkowe.
  \begin{equation}
    \label{eq:Problemy-podczerwone-i-nadfioletowe-01}
    I_{ \HorSpaceOne \text{rul} } = \int\limits_{ 0 }^{ +\infty } \frac{ 1 }{ k } \, dk.
  \end{equation}
  Zastąpimy ją przez całkę po przedziale $[ \varepsilon, \Lambda ]$, $0 < \varepsilon < \Lambda < +\infty$:
  \begin{equation}
    \label{eq:Problemy-podczerwone-i-nadfioletowe-02}
    I_{ \HorSpaceOne \text{rul} }( \varepsilon, \Lambda ) =
    \int\limits_{ \varepsilon }^{ \Lambda } \frac{ 1 }{ k } \, dk.
  \end{equation}
  Ponieważ ta całka jest rozbieżna w~granicy $\varepsilon \searrow 0$ to mówimy,
  że~w~teorii występują \textbf{problemy podczerwone}. Ponieważ jest ona
  rozbieżna w~granicy $\Lambda \nearrow +\infty$, to mówimy, że~są obecne
  \textbf{problemy nadfioletowe}.

\end{frame}
% ##################










% ######################################
\section{Główne rodzaje problemów podczerwonych}
% ######################################


% ##################
\begin{frame}
  \frametitle{Reguły nadwyboru}


  Rozpatrzmy model w~którym przestrzeń Hilberta opisująca cząstki
  naładowane jest sumą prostą dwóch przestrzeni:
  \begin{equation}
    \label{eq:Glowne-rodzaje-problemow-podczerwonych-01}
    \Hcal_{ \HorSpaceTwo \text{phy} } =
    \Hcal_{ \HorSpaceTwo 0 } \oplus \Hcal_{ \HorSpaceTwo e }.
  \end{equation}
  Jeśli $\psi \in \Hcal_{ 0 }$ to stan ten opisuje system, którego całkowity
  ładunek elektryczny jest równy~$0$. Analogicznie, jeśli
  $\psi \in \Hcal_{ e }$ to system ten ma całkowity ładunek elektryczny
  równy~$e$.

  Czysto matematycznie ma sens następujące wyrażenie
  % nc od not conserved, brak zachowanego ładunku
  \begin{equation}
    \label{eq:Glowne-rodzaje-problemow-podczerwonych-02}
    \psi_{ \HorSpaceTwo \text{nc} } =
    \frac{ 1 }{ \sqrt{ 2 } } \psi_{ \HorSpaceTwo 0 } +
    \frac{ 1 }{ \sqrt{ 2 } } \psi_{ \HorSpaceTwo 1 }, \quad
    \psi_{ \HorSpaceTwo 0 } \in \Hcal_{ \HorSpaceTwo 0 }, \,
    \psi_{ \HorSpaceTwo 1 } \in \Hcal_{ \HorSpaceTwo e }.
  \end{equation}
  Czy ten stan ma sens fizyczny? W~zależności od wyniku pomiaru
  dostawalibyśmy stan zawierający całkowity ładunek elektryczny równy
  $0$ lub $e$, co jest sprzeczne z~zasadą zachowania ładunku elektrycznego.
  Wobec tego stwierdzamy, że~\alert{fizycznie} superpozycja stanów
  $\psi_{ \HorSpaceOne a }$, $\psi_{ \HorSpaceOne b }$ jest dopuszczona wtedy i~tylko
  wtedy, gdy $\psi_{ \HorSpaceOne a }, \psi_{ \HorSpaceOne b } \in \Hcal_{ \HorSpaceTwo 0 }$
  lub $\psi_{ \HorSpaceOne a }, \psi_{ \HorSpaceOne b } \in \Hcal_{ \HorSpaceTwo e }$.

\end{frame}
% ##################





% ##################
\begin{frame}
  \frametitle{Reguły nadwyboru}


  Ogólną \textbf{regułę nadwyboru}, na potrzeby tego wystąpienia, możemy
  sformułować w~następujący sposób. Przestrzeń Hilberta $\Hcal$ danej
  teorii fizycznej
  rozkłada~się na sumę prostą podprzestrzeni $\Hcal_{ \HorSpaceTwo i }$
  \begin{equation}
    \label{eq:Glowne-rodzaje-problemow-podczerwonych-03}
    \Hcal =
    \Hcal_{ \HorSpaceTwo 0 } \oplus \Hcal_{ \HorSpaceTwo 1 } \oplus
    \Hcal_{ \HorSpaceTwo 2 } \oplus \ldots,
  \end{equation}
  i~superpozycja stanów $\psi_{ a }$ i~$\psi_{ b }$ jest fizycznie dopuszczalna
  wtedy i~tylko wtedy, gdy $\psi_{ a }, \psi_{ b } \in \Hcal_{ \HorSpaceTwo i }$
  dla pewnego~$i$.

  Matematycznie, na mocy
  \colorhref{https://en.wikipedia.org/wiki/Schur\%27s\_lemma}{lematu
    Schura},
  oznacza to, że~istnieje operator $Q \neq C \id$, który jest przemienny
  z~wszystkimi fizycznymi obserwablami. Dla pola elektromagnetycznego tym
  operatorem jest to operator ładunku elektrycznego, którego wartości własne
  mają postać $n \HorSpaceOne e$, $n \in \Zbb$.

\end{frame}
% ##################




% ##################
\begin{frame}
  \frametitle{Krótko- i~długozasięgowe oddziaływania}


  Jak dobrze wiadomo fotonowi przypisujemy zerową masę spoczynkową.
  Teoria pola uczy nas, że~wobec tego \alert{klasyczny} potencjał
  naładowanej cząstki punktowej w~elektrodynamice będzie miał postać
  \begin{equation}
    \label{eq:Glowne-rodzaje-problemow-podczerwonych-04}
    V_{ \text{Cou} }( \vecx \HorSpaceOne ) =
    \frac{ \alpha_{ \HorSpaceThree \text{el} } \HorSpaceThree q }
    { \absOne{ \vecx } }.
  \end{equation}
  Z~drugiej strony jeśli oddziaływanie przenosi bozon o~masie $m$, wówczas
  należy spodziewać~się potencjału posiadającego formę
  \colorhref{https://en.wikipedia.org/wiki/Hideki\_Yukawa}{Yukały}:
  \begin{equation}
    \label{eq:Glowne-rodzaje-problemow-podczerwonych-05}
    V_{ \text{Yuk} }( \vecx \HorSpaceOne ) =
    -( g_{ \HorSpaceOne \text{Yuk} } )^{ \HorSpaceOne 2 }
    \frac{ e^{ -\alpha_{ \HorSpaceThree \text{Yuk} } \HorSpaceThree m
        \HorSpaceOne \absOne{ \vecx } } }
    { \absOne{ \vecx } },
  \end{equation}
  gdzie $g_{ \HorSpaceOne \text{Yuk} }$ jest odpowiednikiem, z~grubsza, ładunku
  elektrycznego. Obliczając gradient tych potencjałów dostajemy

  \vspace{-1.5em}


  \begin{subequations}

    \begin{align}
      \label{eq:Glowne-rodzaje-problemow-podczerwonych-06-A}
      \vecE( \vecx \HorSpaceOne )
      &= -\nabla V_{ \text{Cou} }( \vecx \HorSpaceOne ) =
        \frac{ \alpha_{ \HorSpaceThree \text{el} } \HorSpaceThree q }
        { \absOne{ \vecx }^{ 2 } } \,
        \widehat{ r }, \\
      \label{eq:Glowne-rodzaje-problemow-podczerwonych-06-B}
      \vec{ M }( \vecx \HorSpaceOne )
      &=
        -\nabla V_{ \text{Yuk} }( \vecx ) =
        -( g_{ \HorSpaceOne \text{Yuk} } )^{ \HorSpaceOne 2 } \,
        e^{ -\alpha_{ \HorSpaceThree \text{Yuk} } \HorSpaceThree m \HorSpaceOne
        \absOne{ \vecx } }
        \left( \frac{ 1 }{ \absOne{ \vecx }^{ 2 } } +
        \frac{ \alpha_{ \HorSpaceThree \text{Yuk} } \HorSpaceThree m }
        { \absOne{ \vecx } } \right)
        \widehat{ x }.
    \end{align}

  \end{subequations}

\end{frame}
% ##################





% ##################
\begin{frame}
  \frametitle{Krótko- i~długozasięgowe oddziaływania}


  Łatwo obliczamy całki z~tych pól wektorowych po sferze $S( R )$, gdzie
  $R$~to jej promień.

  \vspace{-1.5em}


  \begin{subequations}

    \begin{align}
      \label{eq:Glowne-rodzaje-problemow-podczerwonych-07-A}
      \int\limits_{ S( R ) } \vecE( \vecx \HorSpaceOne ) \cdot d\vec{ \sigma }
      &= 4 \pi \alpha_{ \HorSpaceThree \text{el} } \HorSpaceThree q, \\
      \label{eq:Glowne-rodzaje-problemow-podczerwonych-07-B}
      \int\limits_{ S( R ) } \vec{ M }( \vecx \HorSpaceOne ) \cdot d\vec{ \sigma }
      &=
        -4 \pi \HorSpaceTwo g_{ \text{Yuk} }^{ \, 2 } \,
        e^{ -\alpha_{ \HorSpaceThree \text{Yuk} } \HorSpaceThree m \HorSpaceOne R }
        \left( 1 + \alpha_{ \HorSpaceThree \text{Yuk} } \HorSpaceThree m R \right).
    \end{align}

  \end{subequations}

  \vspace{-1em}



  Widzimy więc, że~dla teorii z~bozonem o~masie spoczynkowej $m = 0$
  całka ta pozostaje stała w~granicy $R \nearrow +\infty$, lecz gdy masa bozonu
  $m > 0$, to całka ta w~tej granicy znika. Pierwszy wynik to dobrze nam
  znane prawo Gaussa, rachunek ten zaś miał służyć pokazaniu jego związek
  z~masą bozonu obecnego w~tej teorii.

\end{frame}
% ##################





% ##################
\begin{frame}
  \frametitle{Kilka nieoczywistych konsekwencji}


  Poprzednio rozważaliśmy pole od cząstki punktowej, ale jak wiemy
  wynikający z~prawa Gaussa wzór
  \begin{equation}
    \label{eq:Glowne-rodzaje-problemow-podczerwonych-08}
    \lim_{ R \nearrow +\infty } \int\limits_{ S( R ) } \vecE( \vecx, t ) \cdot d\vec{ \sigma } =
    4 \pi \alpha_{ \HorSpaceThree \text{el} } \HorSpaceThree q,
  \end{equation}
  jest zawsze spełniony w~teorii Clerka Maxwella, proszę przy prawa strona
  nie zależy od wyboru chwili $t$, bo ładunek jest zachowany. Niech $\vecn$
  będzie dowolnym unormowanym wektorem w~$\Rbb^{ 3 }$: $\Vert \vecn \Vert = 0$.
  Określamy asymptotyczne pole elektromagnetyczne:
  \begin{equation}
    \label{eq:Glowne-rodzaje-problemow-podczerwonych-09}
    \vecEasm :=
    \lim_{ R \nearrow +\infty } R^{ 2 } \vecE( R \vecn, t ), \quad
    \vecBasm :=
    \lim_{ R \nearrow +\infty } R^{ 2 } \vecB( R \vecn, t ).
  \end{equation}
  Jeśli mamy do czynienia ze standardowym problemem rozpraszania masywnych
  cząstek (elektronów), to można pokazać, że~pola asymptotyczne nie zależą
  od wyboru chwili~$t$. Fakt ten przez niektórych będzie uznany za
  oczywisty, przez innych za zastanawiający.

\end{frame}
% ##################





% ##################
\begin{frame}
  \frametitle{Kilka nieoczywistych konsekwencji}

  \vspace{-1.5em}


  \begin{equation}
    \label{eq:Glowne-rodzaje-problemow-podczerwonych-10}
    \vecEasm :=
    \lim_{ R \nearrow +\infty } R^{ 2 } \vecE( R \vecn, t ), \quad
    \vecBasm :=
    \lim_{ R \nearrow +\infty } R^{ 2 } \vecB( R \vecn, t ).
  \end{equation}
  Istnieje nieskończenie wiele różnych pól $\vecEasm$, $\vecBasm$
  czyniących zadość prawu Gaussa wyrażonemu przez wzór
  \eqref{eq:Glowne-rodzaje-problemow-podczerwonych-08}. Do pola Coulomba
  zadanego rozkładu ładunków zawsze można dodać dowolną kombinację
  swobodnych fal elektromagnetycznych, które uciekają „do nieskończoności”
  i~znikają jak $r^{ -2 }$ dla $r >> 1$.

  Jak powiedziano wcześniej, dla standardowego problemu rozpraszania,
  pola asymptotyczne nie zależą od~$t$. Prof. Herdegen stwierdził,
  że~wobec tego, ciężko uniknąć wniosku, że~w~tym problemie wszystkie pola
  są jakby podwładnymi pola Coulomba (str.~2,
  \parencite{Herdegen-Infrared-structure-beyond-locality-ETC-Ver-2024}).
  Pole coulombowskie~jest odpowiedzialne za „ustalenie” ładunku
  elektrycznego, wszystkie pozostałe służą tylko „poprawieniu” ich
  zachowania w~nieskończoności, tak by pola asymptotyczne $\vecEasm$,
  $\vecBasm$ miały zawsze tą samą, niezależną od czasu postać.

\end{frame}
% ##################





% ##################
\begin{frame}
  \frametitle{Kilka nieoczywistych konsekwencji}


  \begin{equation}
    \label{eq:Glowne-rodzaje-problemow-podczerwonych-11}
    \lim_{ R \nearrow +\infty } \int\limits_{ S( R ) } \vecE( \vecx, t ) \cdot d\vec{ \sigma } =
    4 \pi \alpha_{ \HorSpaceThree \text{el} } \HorSpaceThree q
  \end{equation}
  Inną konsekwencją tego wzoru jest to, że~w~elektrodynamice ładunek
  elektryczny nie jest wielkością lokalną. Jeśli ładunek jest skupiony
  w~ograniczonym obszarze $\Ocal$, to jego \alert{całkowitą} wartość można
  znaleźć przez całkowanie po sferze, które jest „bardzo daleko” od niego.
  W~szczególności całkowity ładunek pełnego układu możemy znaleźć poprzez
  całkowanie „po sferze w~nieskończoności”.

  Przyjrzyjmy~się teraz tej nielokalności z~punktu widzenia algebraicznej
  kwantowej teorii pola \textsc{aqft}.

\end{frame}
% ##################





% ##################
\begin{frame}
  \frametitle{Lokalizacja zjawisk w~czasoprzestrzeni}

  \vspace{-0.8em}


  \begin{figure}

    \centering


    \begin{tikzpicture}

      % x axis
      \draw[axis arrow] (-5,0) -- (5,0);

      \pic at (5,0) {x mark for horizontal axis 1};


      % t axis
      \draw[axis arrow] (0,-2.5) -- (0,4);

      \pic at (0,4) {t mark for vertical axis 1};



      % Light cone
      \fill[color=blue,opacity=0.4] (0,0) -- (3.5,3.5) -- (-3.5,3.5) -- cycle;

      \fill[color=blue,opacity=0.4] (0,0) -- (2.2,-2.2) -- (-2.2,-2.2) --
      cycle;

      \draw[dashed] (2.2,-2.2) -- (-3.5,3.5);

      \draw[dashed] (-2.2,-2.2) -- (3.5,3.5);



      % Region number 1
      \draw[dashed] plot[smooth cycle]
      coordinates { (1.2,0) (3,-1) (3.5,1.4) (2.5,1.2) };

      \node[scale=1.3] at (2.3,0.3) {$\Ocal_{ 1 }$};


      % Region number 2
      \draw[dashed] plot[smooth cycle]
      coordinates { (0.8,0) (2.9,-1.5) (4.2,1.8) (1,3) };

      \node[scale=1.3] at (2.7,1.8) {$\Ocal_{ 2 }$};


      % Region number 3
      \draw[dashed] plot[smooth cycle]
      coordinates { (-1,-0.2) (-2,1.2) (-3,0.3) (-2,-1) };

      \node[scale=1.3] at (-2,0.5) {$\Ocal_{ 3 }$};


      % Event reaching form region 3 to region 1
      \draw[dashed,thick,color=yellowLoc] (-1.3,0) -- (2,3.3);

      \fill[color=brown] (-1.3,0) circle [radius=0.09];

      \node[below] at (-1.3,-0.05) {$E$};

    \end{tikzpicture}

    \caption{Trzy ograniczone obszary $\Ocal_{ 1 }$, $\Ocal_{ 2 }$
      i~$\Ocal_{ 3 }$, oraz zjawisko $E$ wysyłające promień światła.}


  \end{figure}

\end{frame}
% ##################





% ##################
\begin{frame}
  \frametitle{Konsekwencje prawa Gaussa}


  Standardowo obserwable przypisujemy ograniczonym obszarom
  czasoprzestrzeni i~jeśli dwa z~nich są rozłączne przestrzenie to zawarte
  w~nich obserwable komutują $[ A, B ] = 0$. Pola $\vecEasm$,
  $\vecBasm$a są zlokalizowane na „sferze w~nieskończoności”, która jest
  rozłączna przestrzennie z~dowolnym ograniczonym obszarem
  czasoprzestrzeni, więc są przemienne z~wszystkimi lokalnymi obserwablami.

  Ten wynik jest niezwykle ważny dla prac prof. Staruszkiewicza i~Herdegena,
  proszę więc zwrócić na niego szczególną uwagę.

  Na mocy lematu Schura pola $\vecEasm$, $\vecBasm$ definiują nową regułę
  nadwyboru, gdzie każda konfiguracja tych pól, zadaje jeden sektor.
  Przestrzeń Hilberta rozkłada~się więc na \alert{nieprzeliczalną} sumę
  prostą odpowiednich sektorów, wyznaczonych przez asymptotyczny kształt
  pól. Nie będę ukrywał, że~właściwe zrozumienie sensu fizycznego tego
  faktu, nie jest dla mnie proste.

\end{frame}
% ##################





% ##################
\begin{frame}
  \frametitle{Co to znaczy?}


  Puszczając trochę wodze fizycznej fantazji, możemy sobie to wytłumaczyć
  w~następujący sposób. Sama elektrodynamika nie wyznacza nam ładunku
  całkowitego obecnego w~danych układzie fizycznym. Jeśli jednak w~pewnej
  chwili~$t$ przyjmiemy, że~ten ładunek wynosi $n e$ to na mocy zasady
  zachowania ładunku, ładunek elektryczny tego układu \alert{zawsze}
  wynosi~$n e$.

  Reguła nadwyboru mówi nam więc, że~zbiór wszystkich możliwych układów
  fizycznych rozpada~się na rozłączne klasy, ponumerowany całkowitą
  wielokrotnością ładunku elementarnego~$e$ i~\alert{żaden} proces fizyczny
  nie może przeprowadzić nam układu z~jednej klasy do~drugiej. To w~pewnym
  sensie jest oczywiste dla nas, bo jesteśmy przyzwyczajeni do własność
  ładunku elektrycznego.

  % Reguła nadwyboru pól asymptotycznych
  % $\vecE^{ \HorSpaceOne \text{asm} }( \vecn )$,
  % $\vecB^{ \HorSpaceOne \text{asm} }( \vecn )$, przynajmniej dla układu
  % opisującego proces rozpraszania, oznacza, że~gdy w~jednej chwili~$t$
  % ustalimy pola asymptotyczne, to już \textit{nigdy} nie możemy ich zmienić.

\end{frame}
% ##################





% ##################
\begin{frame}
  \frametitle{Mamy poważny problem}


  % Puszczając trochę wodze fizycznej fantazji, możemy sobie to wytłumaczyć
  % w~następujący sposób. Sama elektrodynamika nie wyznacza nam ładunku
  % całkowitego obecnego w~danych układzie fizycznym. Jeśli jednak w~pewnej
  % chwili~$t$ przyjmiemy, że~ten ładunek wynosi $n e$ to na mocy zasady
  % zachowania ładunku, ładunek elektryczny tego układu \alert{zawsze}
  % wynosi~$n e$.

  % Reguła nadwyboru mówi nam więc, że~zbiór wszystkich możliwych układów
  % fizycznych rozpada~się na rozłączne klasy, ponumerowany całkowitą
  % wielokrotnością ładunku elementarnego~$e$ i~\alert{żaden} proces fizyczny
  % nie może przeprowadzić nam układu z~jednej klasy do~drugiej. To w~pewnym
  % sensie jest oczywiste dla nas, bo jesteśmy przyzwyczajeni do własność
  % ładunku elektrycznego.

  Reguła nadwyboru pól asymptotycznych $\vecEasm$, $\vecBasm$, przynajmniej
  dla układu
  opisującego proces rozpraszania, oznacza, że~gdy w~jednej chwili~$t$
  ustalimy pola asymptotyczne, to już \alert{nigdy} nie możemy ich zmienić.
  Postać tych pól ustalamy, choć zwykle nie jesteśmy tego świadomi, podając
  dla wybranej chwili~$t$, gęstość ładunku~$\rho( \vecx, t )$, gęstość
  prądu~$\vecj( \vecx, t )$ oraz pola $\vecE( \vecx, t )$
  i~$\vecB( \vecx, t )$.

  Przez analogię możemy powiedzieć, że~zbiór układów fizycznych rozpada~się
  na~rozłączne klasy „ponumerowane” postacią pól asymptotyczny
  $\vecEasm$, $\vecBasm$ i~\alert{nie} jest możliwe, by proces
  fizyczny przeprowadził układ między dwoma klasami.

  Istnienie reguły nadwyboru związanej z~polami $\vecEasm$, $\vecBasm$ jest
  \alert{sprzeczne} z~niezmienniczością Lorentza. Już na poziomie klasycznym
  nie można znaleźć lorentzowsko niezmienniczych pól $\vecEasm$, $\vecBasm$
  i~ten sam problemy występuje na poziomie kwantowym.

\end{frame}
% ##################





% ##################
\begin{frame}
  \frametitle{Spontaniczne łamanie symetrii Lorentza}


  Transformacja Lorentza zmienia pola $\vecEasm$, $\vecBasm$ co określa~się
  literaturze jako spontanicznym złamanie symetrii Lorentza, analogiczną do
  tej dotyczącej bozonu Higgsa, czy ferromagnetyka. Określając konkretny
  układ fizyczny, musimy określić, tak jak przedtem określaliśmy ładunek,
  teraz określamy postać pól asymptotycznych, co uprzywilejowuje pewną klasę
  układów inercjalnych.

  Zwykle ten problem opisuje~się na poziomie algebr i~ich reprezentacji,
  omówienie tego zajęłoby jednak sporo czasu. Kontynuując nasze luźne
  rozważania, możemy powiedzieć coś takiego. Elektrodynamika jest
  „symetryczna” względem wartości ładunku elektrycznego, bo teoria dopuszcza
  wszystkie jego wartości. Jednak każdy konkretny \alert{układ} nie jest
  symetryczny w~tym sensie, bo~ma ustalony ładunek elektryczny.

  Tak samo elektrodynamika jako teoria jest symetryczna względem
  transformacji Lorentza, ale~każdy konkretny układ już tej symetrii jest
  pozbawiony. Jest to niewątpliwie ciekawy i~trudny temat.

\end{frame}
% ##################





% ##################
\begin{frame}
  \frametitle{Jaka jest masa elektronu?}


  Jakkolwiek może istnieć sposób obejścia nieprzyjemnego wyniku, jakim jest
  złamanie transformacji Lorentza, przez odrzucenie pewnych założeń
  o~lokalności pola, to istnieje też drugi
  problem, znacznie trudniejszy do obejścia. Jak wskazuje praca Detlev’a
  Buchholza istnienie reguł nadwyboru związanej z~polami asympotycznymi,
  sprawia, że~masa elektronu jest „rozmyta”
  \parencite{Buchholz-The-Physical-State-Space-Of-Quantum-ETC-Pub-1982}.

  Jak wiadomo, dla swobodnej cząstki w~\textsc{stw} musi obowiązywać
  dobrze znana relacja
  \begin{equation}
    \label{eq:Glowne-rodzaje-problemow-podczerwonych-12}
    E^{ \HorSpaceOne 2 } - p^{ \HorSpaceOne 2 } =
    p^{ \HorSpaceOne \mu } p_{ \mu }  = m^{ \HorSpaceOne 2 }.
  \end{equation}
  Gdy przechodzimy do teorii kwantowej pęd i~energia stają~się operatorami
  $P^{ \HorSpaceOne \mu }$. Jeśli zachodzi
  \begin{equation}
    \label{eq:Glowne-rodzaje-problemow-podczerwonych-13}
    P^{ \HorSpaceOne \mu } P_{ \mu } \psi  = m^{ \HorSpaceOne 2 } \psi, \quad
    \psi \in \Hcal,
  \end{equation}
  to $\psi$ reprezentuje swobodną cząstkę o~„ostrej” masie $m > 0$.%  Argument
  % Buchholza wskazuje, że~dla elektrodynamiki kwantowej taki stan $\psi$
  % istnieje tylko dla $m^{ 2 } = 0$, czyli jest pozbawiony elektronów!

  % Dlaczego tak jest? Intuicyjnie jego rozumowanie, po ogromny uproszczeniu,
  % przebiega mniej więcej tak. Jeśli $\vecEasm \neq 0$ lub~$\vecBasm \neq 0$, to
  % elektron musi cały czas emitować niskoenergetyczne fotony, aby podtrzymać
  % istnienie pól asymptotycznych.

\end{frame}
% ##################





% ##################
\begin{frame}
  \frametitle{Jaka jest masa elektronu?}

  \vspace{-1em}


  \begin{equation}
    \label{eq:Glowne-rodzaje-problemow-podczerwonych-13}
    P^{ \HorSpaceOne \mu } P_{ \mu } \psi  = m^{ \HorSpaceOne 2 } \psi, \quad
    \psi \in \Hcal,
  \end{equation}
  Argument Buchholza wskazuje, że~dla elektrodynamiki kwantowej taki
  stan~$\psi$ istnieje tylko dla $m^{ 2 } = 0$, czyli rozważony układ jest
  pozbawiony elektronów i~pozytronów!

  Dlaczego tak jest? Intuicyjnie jego rozumowanie, po ogromny uproszczeniu,
  przebiega mniej więcej tak. Jeśli $\vecEasm \neq 0$ lub~$\vecBasm \neq 0$, to
  elektron musi cały czas emitować niskoenergetyczne fotony, aby podtrzymać
  istnienie pól asymptotycznych.


  W~skutek emisji niskoenergetycznego fotonu elektron doznaje lekkiego
  odrzutu, który zmienia jego ruch. Wygląda to tak, jakby masa
  elektronu fluktuowała w~czasie jego ruchu, nie może więc on być stanem
  własnym operatora $P^{ \mu } P_{ \mu }$ do dyskretnej wartości
  własnej~$m^{ \HorSpaceOne 2 }$.

\end{frame}
% ##################





% ##################
\begin{frame}
  \frametitle{Infracząstki}


  Cząstki które nie mają ściśle określonej masy nazywamy
  \textbf{infracząstkami} (ang.~\textit{infraparticles})
  lub~\textbf{cząstkami podczerwonymi}. Poglądowy obraz elektronu jako
  infracząstki jest taki, że~jest on zawsze otoczony przez chmurę fotonów
  („ubrany w~chmurę fotonów”). Chmura ta jest odpowiedzialna za generowanie
  pól asymptotycznych i~przez to, iż elektron „ciągnie” tę chmurę jego masa
  nie jest „ostro” określona.

  Intuicyjne wyjaśnienia łamania symetrii Lorentzowskiej w~tym obrazie
  jest następujące. Chmury fotonów zmieniają kształt pod wpływem
  transformacji Lorentza, więc wyróżnione są te układy odniesienia,
  w~których chmury te są sferycznie symetryczne.

  Ma to znaczenie dla ważnego wyniku
  \colorhref{https://en.wikipedia.org/wiki/Eugene\_Wigner}{Eugena Wignera}
  Pokazał on, że~każdą nieredukowalną, skończenie wymiarową reprezentację
  grupy Lorentza określają dwie liczby: $m^{ \HorSpaceOne 2 }$ i~$s$. Jak
  wynika z~zapisu, każdą taką reprezentację można jednoznacznie skojarzyć
  z~cząstką swobodną, która ma zadany kwadrat masy i~spin (przez relację
  $S^{ 2 } = s ( s + 1 )$).

\end{frame}
% ##################





% ##################
\begin{frame}
  \frametitle{Infracząstki}


  Niestety klasyfikacja ta \alert{nie} może obejmować infracząstek, bo
  te z~definicji nie mają one „ostro” określonej masy. Ponieważ do tej
  kategorii należą elektron i~pozytron, jak i~większość fermionów modelu
  standardowego, stawia pod znakiem zapytania wyniki niemałej ilości prac
  obecnej w~literaturze.

  Temat infracząstek, choć dość niszowy w~środowisku, posiada literaturę
  zbyt dużą by móc ją choćby wymienić. Osobom zainteresowany polecamy
  sięgnąć po dalsze informacje i~odniesienia do literatury po pracę
  J.~Munda, K.H.~Rehrena i B.~Schroera
  \colorhref{https://arxiv.org/abs/2109.10342}{\textit{Infraparticle
      quantum fields and the~formation~of photon clouds}}
  \parencite{Mund-Rehren-Schroer-Infraparticle-quantum-fiels-ETC-Pub-2022}.
  W~pracy tej omówione są takie zagadnienia jak reguła nadwyboru pól
  asymptotycznych, „rozmycie” masy czy złamanie symetrii Lorentza.

\end{frame}
% ##################





% ##################
\begin{frame}
  \frametitle{Problemy podczerwone w~teorii rozpraszania}


  Rozważmy podstawowy problem z~teorii rozpraszania. Chcemy obliczyć
  przekrój czynny na proces rozpraszania w~którym przechodzimy ze~stanu
  dwóch elektronów~$\alpha$ do~stanu~$\beta$, oznaczamy go
  $\sigma_{ \HorSpaceOne \alpha, \HorSpaceOne \beta }$. Zachodzi,trochę symboliczna relacja

  \vspace{-2.1em}


  \begin{equation}
    \label{eq:Glowne-rodzaje-problemow-podczerwonych-14}
    \sigma_{ \HorSpaceOne \alpha, \HorSpaceTwo \beta } \sim
    \absOne{ S_{ \alpha, \HorSpaceOne \beta } }^{ 2 },
  \end{equation}

  \vspace{-1.8em}


  gdzie $S_{ \alpha, \HorSpaceOne \beta }$ to odpowiedni element macierzy~$S$. Elementu
  tego nie możemy obliczyć bezpośrednio, ze~względu na rozbieżności
  podczerwone, takie jak te zaznaczone na~slajdzie
  \eqref{sld-Problemy-podczerwone-i-nadfioletowe}, musimy więc wprowadzić
  parametr obcięcia $\varepsilon$. Poprzez wprowadzenie go blokujemy powstanie
  fotonów o~energii mniejszej niż~$\varepsilon$.

  Nasza wiedza fizyczna wymaga, by na końcu przejść z~$\varepsilon$ do~zera,
  co daje nam

  \vspace{-2.3em}


  \begin{equation}
    \label{eq:Glowne-rodzaje-problemow-podczerwonych-15}
    \sigma_{ \HorSpaceOne \alpha, \HorSpaceTwo \beta } \sim
    \lim_{ \varepsilon \searrow 0 }
    \absOne{ S_{ \alpha, \HorSpaceOne \beta }^{ \HorSpaceOne \varepsilon } }^{ 2 } = 0.
  \end{equation}

  \vspace{-1.7em}


  Mało który wynik fizyki teoretycznej jest tak sprzeczny z~rzeczywistością
  jak ten. Ale~się nie poddajemy.

\end{frame}
% ##################





% ##################
\begin{frame}
  \frametitle{Problemy podczerwone w~teorii rozpraszania}


  D.R. Yennie, S.C. Frautschi i~H.~Suura podali przepis rachunkowy na
  obliczenie, tzw. inkluzywnego przekroju czynnego, który na poziomie
  zgodności teorii z~eksperymentem rozwiązuje ten problem bardzo dobrze
  \parencite{Yennie-Frautschi-Suura-The-infrared-divergence-ETC-Pub-1961}.
  Idea ich podejścia jest następująca.

  Gdyby foton miał masę spoczynkową $m > 0$, to wówczas elektron
  potrzebowałby mieć energię kinetyczną $E_{ \text{kin} } \geq m$, aby móc
  wytworzyć pojedynczy foton. Ponieważ jednak foton jest bezmasowy, to
  elektron o~dowolnej energii $E_{ \text{kin} } > 0$ jest w~stanie
  wytworzyć \alert{dowolną} ilość fotonów, pod warunkiem, że~ich sumaryczna
  energia jest odpowiednio mała. Z~drugiej strony czułość każdego detektora
  jest skończona i~nie jest w~stanie on wykrywać fotonów o~energii niższej,
  niż pewna energia progowa $E_{ \HorSpaceOne \text{thre} } > 0$.
  % thre - threshold

\end{frame}
% ##################





% ##################
\begin{frame}
  \frametitle{Problemy podczerwone w~teorii rozpraszania}


  Wobec tego, należy przyjąć, że~każdy proces rozpraszania dwóch elektronów
  zawiera pewną liczbę miękkich fotonów (ang. \textit{soft photons}),
  których detektor był w~stanie zarejestrować. Nasza metoda powinna być
  inkluzywna ze względu na te niezarejestrowane fotony, co prowadzi nas do
  wzoru
  \begin{equation}
    \label{eq:Glowne-rodzaje-problemow-podczerwonych-16}
    \sigma_{ \HorSpaceOne \alpha, \HorSpaceTwo \beta }^{ \HorSpaceOne \text{inc} } \sim
    \lim_{ \varepsilon \searrow 0 } \sum_{ n = 0 }^{ \infty }
    \absOne{ S_{ \alpha, \HorSpaceOne \beta_{ \HorSpaceOne n } }^{ \HorSpaceOne \varepsilon } }^{ 2 },
  \end{equation}
  gdzie $\beta_{ \HorSpaceOne n }$ oznacza stan, który poza cząstkami obecnymi
  w~stanie $\beta$ zawiera $n$~miękkich fotonów o~energii poniżej progu
  czułości detektora.

  Choć ten przepis prowadzi do bardzo dobrych przewidywań, nie potrafimy
  go w~sposób satysfakcjonujący wyprowadzić z~kwantowej teorii pola.
  Osobom zainteresowany tym tematem polecam lekturę artykułu Pawła Ducha
  i~Wojciecha Dybalskiego \colorhref{https://arxiv.org/abs/2307.06114}{
    \textit{Infrared problem in quantum electrodynamics}}
  \parencite{Duch-Dybalski-Infrared-problem-in-quantum-electrodynamics-Pub-2023}
  i~\colorhref{https://arxiv.org/abs/1906.00940}{\textit{Infrared problem\ldots}}
  pierwszego z~tych autorów
  \parencite{Duch-Infrared-problem-in-perturbative-quantum-ETC-Pub-2021}.

\end{frame}
% ##################





% ##################
\begin{frame}
  \frametitle{O~wadze zagadnień podczerwonych}


  Na chwilę obecną nie istnieje matematycznie ścisły sposób konstrukcji
  kwantowej elektrodynamiki w~$1 + 3$ wymiarach, dlaczego więc, z~punktu
  widzenia dla której matematyczna spójność jest ważna, warto zajmować~się
  problemami podczerwonymi, zanim taki model kwantowej elektrodynamiki
  zostanie zbudowany? Tutaj odwołam~się do słów prof. Herdegena (zob.
  str.~2
  w~\parencite{Herdegen-Semidirect-product-of-CCR-and-CAR-algebras-ETC-Pub-1998}).

  Po pierwsze, w~granicy asymptotycznej większość elementów dynamiki pola
  powinna tracić znaczenie. Jest więc prawdopodobne, że~teoria problemów
  podczerwonych oparta na solidnych podstawach fizycznych, będzie zgodna
  z~tą wyprowadzoną z~pełnego, ścisłego modelu elektrodynamiki kwantowej
  (miejmy nadzieję, że~taki kiedyś zostanie zbudowany), pomimo że~on
  sam nie jest dziś znany.

\end{frame}
% ##################





% ##################
\begin{frame}
  \frametitle{O~wadze zagadnień podczerwonych}


  Po drugie, teoria asymptotyczna musi czynić zadość prawu Gaussa
  i~związanym z~nim długozasięgowym oddziaływaniom. Ponieważ to musi być
  prawdą również w, standardowo pojętej, kwantowej elektrodynamice, więc
  jest zasadne przyjąć, iż~odtworzenie teorii asymptotycznej jako pewnej
  granicy kwantowej elektrodynamiki. W~jednej ze swoich prac prof.~Herdegen
  badał jak w~\alert{klasycznej} elektrodynamice energia, pęd i~moment pędu
  związane są z~ich asymptotycznymi odpowiednikami, o~czym powiem
  trochę później
  (zob.~\parencite{Herdegen-Long-range-effects-in-asymptotic-ETC-Pub-1995}).

  W~takiej sytuacji badanie reżimu asymptotycznego może rzucić nowe światło
  na pełnej elektrodynamiki kwantowej, a~dziś każdy taka informacja jest
  niezwykle cenna.

\end{frame}
% ##################










% ######################################
\section{Pola asymptotyczne i~efekt pamięci}
% ######################################



% ##################
\begin{frame}
  \frametitle{Problem pól asymptotycznych}


  Wróćmy do problemu roli pól asymptotycznych $\vecEasm$, $\vecBasm$
  w~zagadnieniach rozpraszania i~ponownie odwołamy~się do
  prof.~Herdegena, który zauważył, że~z~przytoczonych wcześniej informacji
  wyciągane~są mocno przeciwstawne wnioski (str.~2,
  \parencite{Herdegen-Infrared-structure-beyond-locality-ETC-Ver-2024}).

  \begin{itemize}

  \item[1)] Konkretna postać pól asymptotycznych $\vecEasm$, $\vecBasm$
    jest w~dużej mierze kwestią konwencji. Ważne jest tylko by było
    dla nich spełnione prawo Gaussa.

  \item[2)] Podczerwone stopnie swobody swobodnego pola różnią~się
    fundamentalnie od jego lokalnych stopni swobody. Warto zaznaczyć,
    że~nadfioletowe stopnie swobody są lokalne w~podanym tu sensie.

  \end{itemize}

\end{frame}
% ##################





% ##################
\begin{frame}
  \frametitle{Problem pól asymptotycznych}


  Przeciwne spojrzenie na ten problemy prezentuje~się następująco.

  \begin{itemize}

  \item[1)] Eksperymentalne testy teorii odbywają~się na skali metrów,
    więc odległość rzędu kilku metrów jest wystarczając by mówić o~efekcie
    „w~nieskończoności”. Ponadto efekt pamięci, odkryty w~przypadku
    elektrodynamiki przez
    \colorhref{https://pl.wikipedia.org/wiki/Andrzej_Staruszkiewicz}{prof.~Staruszkiewicza}, pokazuje
    istnienie obserwowalne efekty pól asymptotycznych.

  \item[2)] W~klasycznym przypadku, w~teorii rozpraszania swobodne pola
    muszą mieć silne zachowanie podczerwone i~wszystkie ich stopnie swobody
    są niezależne. W~teorii kwantowej lokalne pole swobodne ma naturalne
    rozszerzenie, zawierające pola wykazujące podczerwone zachowanie.

  \end{itemize}

  Prof. Herdegen i~Staruszkiewicz w~swoich badaniach opowiedzieli~się
  za~drugim z~tych podejść i~pewne osiągnięte przez nich wyniki zreferujemy
  poniżej, zaczynając od~efektu pamięci (ang. \textit{memory effect}).

  \end{frame}
% ##################





% ##################
\begin{frame}
  \frametitle{Efekt pamięci}


  Efekt pamięci został znaleziony dla pola grawitacyjnego przez
  Zel'dovicha i~Polnareva w~$1974$ roku (data publikacji angielskiego
  tłumaczenia ich pracy
  \parencite{Zeldovich-Polnarev-Radiation-of-gravitational-ETC-Pub-1974}),
  a~dla pola elektromagnetycznego przez prof. Staruszkiewicza w~roku~$1981$
  \parencite{Staruszkiewicz-Gauge-invariant-surface-contribution-ETC-Pub-1981}.

  Zacznijmy od ustalenia postać pól asymptotycznych $\vecEasm$,
  $\vecBasm$. Okazuje~się, że~możemy wziąć pole elektromagnetyczne
  o~dowolnie małej energii, które posiada taką postać asymptotyczną,
  co~ponownie unaocznia, że~pola asymptotyczne są zjawiskiem z~reżimu
  podczerwonego. Wystarczy bowiem wziąć jedno pole które generuje
  tego typu pola asymptotyczne i~następnie przeskalować je w~prosty sposób.

\end{frame}
% ##################





% ##################
\begin{frame}
  \frametitle{Efekt pamięci}


  Niech teraz dane będzie padająca fala płaska
  $\exp\!\big( -i 2\pi ( E t + \vecp \cdot \vecx \HorSpaceOne ) \big)$. W~wielkim
  uproszczeniu prof. Staruszkiewicz wykazał, że~istnieje funkcja
  $f\big( \HorSpaceOne \vecp, \vecEasm, \vecBasm \big)$, taka
  że~przesunięcie fazowe $\delta( \HorSpaceOne \vecp \HorSpaceOne )$ jakiego
  dozna ta fala wyraża się wzorem

  \vspace{-1em}


  \begin{equation}
    \label{eq:Pola-asymptotyczne-i-efekt-pamieci-01}
    \delta( \HorSpaceOne \vecp \HorSpaceOne ) =
    -\frac{ e }{ 2\pi }
    \int \frac{ f\big( \HorSpaceOne \vecp, \vecEasm,
      \vecBasm \big) }{ u( \HorSpaceOne \vecp, l ) } \,
    d^{ \HorSpaceOne 2 } l.
  \end{equation}
  W~notacji relatywistycznej możemy zapisać
  $u( \HorSpaceOne \vecp, l ) = p \cdot l$.

  Wzór ten jest ciekawe zarówno z~punkt widzenia fizyki jak i~matematyki,
  jednak jego dyskusja przekracza ramy tego przeglądowego wystąpienia. To co
  jest ważne to fakt, że~przesunięcie fazowe zależy \alert{tylko
    i~wyłącznie} od pędu fali padającej i~postaci pól asymptotycznych.
  W~szczególności, ponieważ mogę dowolnie zamieszczać energię zawartą
  w~polu elektromagnetycznym $E_{ \HorSpaceOne \text{el} }$, nie zmieniają jego
  postaci asymptotycznej. Wynika stąd, że~przesunięcie fazowe
  $\delta( \HorSpaceOne \vecp \HorSpaceOne )$ również nie zależy od tego tej
  energii.

\end{frame}
% ##################





% ##################
\begin{frame}
  \frametitle{Efekt pamięci}


  Ale to oznacza, że~mogę wziąć granicę $E_{ \HorSpaceOne \text{el} } \searrow 0$,
  w~której \textit{niemożliwa} jest zmiana pędu $\vecp$ fali padającej,
  a~mimo tego $\delta( \HorSpaceOne \vecp \HorSpaceOne ) \neq 0$. Jak to zrozumieć?

  Po pierwsze zwróćmy uwagę, że~niezerowe przesunięcie fazowe nie łamie
  zasady zachowania energii. Po drugie, zadaną postać pól asymptotycznych
  $\vecEasm$ i~$\vecBasm$ mogę otrzymać dla pola o~dowolnej energii
  całkowitej $E_{ \HorSpaceOne \text{el} } > 0$ odpowiednie rozkładając
  niskoenergetyczne fotony po czasoprzestrzeni. Powyższy wynik stwierdza
  więc, że~niezależnie jak małą energią pola dysponujemy, by~odtworzyć
  zadane pola asymptotyczne, musimy tak rozłożyć niskoenergetyczne fotony,
  że~zawsze spowodują to samo przesunięcie fazowe fali płaskiej.

  % Chyba właśnie fakt otrzymania $\delta( \vecp ) \neq 0$ w~granicy
  % $E_{ \HorSpaceOne \text{el} } \searrow 0$ jest źródłem nazwy „efekt pamięci”.
  % Acz~to tylko moje przypuszczenie.

\end{frame}
% ##################





% ##################
\begin{frame}
  \frametitle{Nowe badania nad efektem pamięci}


  Na~tym musimy zakończyć to bardzo pobieżne omówienie efektu pamięci.
  Dokładniejsze jego omówienie można znaleźć w~rozdziale $2.6$ pracy
  prof.~Herdegena
  \colorhref{https://arxiv.org/abs/2403.09234}{\textit{Infrared structure
      beyond\ldots}}
  \parencite{Herdegen-Infrared-structure-beyond-locality-ETC-Ver-2024}. Tam
  też znajduje~się wyjaśnienie jak efekt pamięci przejawia~się w~przypadku
  bardziej fizycznych problemów, niż padająca fala płaska.

  Zanim przejdziemy dalej warto dodać, w~ostatnich
  kilkunastu latach, efekt pamięci zyskał stosunkowo dużą popularność, jak
  też stał~się przedmiotem paru kontrowersji. Polecam w~tym względzie pracę
  prof. Herdegena
  \colorhref{https://arxiv.org/abs/2311.06325}
  {\textit{There is no 'veolcity kick'\ldots}}, w~której stwierdza on,
  że~Lydia~Bieri i~David~Garfinkle błędnie przewidzieli dla efektu pamięci
  występowanie „kopnięcia prędkości”
  (por.~\parencite{Bieri-Garfinkle-An-electromagnetic-analogue-of-ETC-Pub-2013}),
  gdyż dokonali nieuprawnionej zamiany kolejności brania granic
  \parencite{Herdegen-There-Is-No-Velocity-Kick-Memory-ETC-Pub-2024}.

\end{frame}
% ##################










% ######################################
\section{Wybrane badania prof. Herdegena}
% ######################################


% ##################
\begin{frame}
  \frametitle{Prof. Staruszkiewicz i~prace prof.~Herdegena}


  Teraz chcemy pokrótce zarysować wyniki kilku prac prof.~Herdegena
  poświęconych problemom podczerwony. Wszystkie one mają na celu
  rozjaśnienie problemów wymienionych w~pierwszej części niniejszego
  wystąpienia, widać też w~nich wyraźny wpływ badań prof. Staruszkiewicza,
  dlatego należy wspomnieć od dwóch jego badań.

  Prof. Staruszkiewicz, i~za nim prof. Herdegen, kładą duży nacisk
  na~geometryczne własności czasoprzestrzeni Minkowskiego i~ich ścisły
  związek z~równaniami ruchu pola elektromagnetycznego (wariantami równań
  Clerka Maxwella). Pozostaje ubolewać, że~geometryczne własności z~których
  korzystają i~ich związek z~teorią reprezentacji grupy Lorentza nie są
  częściej przytaczane w~literaturze dotyczącej~\textsc{stw}.

  Obecnie jestem w~trakcie przygotowywania notatek na temat tych własności
  do cyklu seminariów \textsc{nkf}u, ciągle wynajdując dość proste
  geometryczne ich wyjaśnienia, dlatego wolę~się dzisiaj nie zagłębiać~się
  przesadnie w~ten temat.

\end{frame}
% ##################





% ##################
\begin{frame}
  \frametitle{Prof. Staruszkiewicz i~prace prof.~Herdegena}


  Jak już zostało to wspomniane w~pierwszej części, na gruncie algebraicznej
  kwantowej teorii pola, pola asymptotyczne są przemienne z~wszystkimi
  lokalnymi obserwablami. Wielkości wykazujące tę własność są
  właściwie zmiennymi klasycznymi, które przetrwały w~teorii kwantowej
  i~zwykle mają tylko charakter pomocniczy w~stosunku do prawdziwie
  „fizycznych” wielkości.

  Wydaje~się, że~pierwszą osobą, która postawiła pytanie o~uczynienie pól
  asymptotycznych zmiennymi prawdziwie kwantowymi, był właśnie
  prof.~Staruszkiewicz, inspiracja zaś wypłynęła z~jego pracy o~efekcie
  pamięci. Podał on teorię, która zasadniczo sprowadza~się do kwantowej
  teorii pola Coulomba w~nieskończoności przestrzennopodobnej
  \parencite{Staruszkiewicz-Quantum-Mechanics-of-Phase-and-Charge-ETC-Pub-1989}.
  Prof.~Herdegen rozwinął później jej algebraiczny formalizm
  \parencite{Herdegen-Asymptotic-algebra-of-quantum-electrodynamics-Pub-2005},
  \parencite{Herdegen-Remarks-on-mathematical-structure-of-ETC-Pub-2022} .

\end{frame}
% ##################





% ##################
\begin{frame}
  \frametitle{Prof. Staruszkiewicz i~prace prof.~Herdegena}


  Nie możemy zbyt dogłębnie wnikać w~tą teorię, warto jednak zaznaczyć,
  że~teoria ta wykazuje intrygujące własności na poziomie reprezentacji
  grupy Lorentza, w~sensie teorii reprezentacji grup. Mianowicie własności
  pewnych jej reprezentacji, zależą od~wartości stałej struktury subtelnej
  $\alpha = e^{ 2 } / ( \HorSpaceThree \hbar c ) =
  0.007 \, 297 \, 352 \, 56 \, 43(11)$. W~teorii Staruszkiewicza
  wyróżniona jest wartość $\alpha = \pi = 3.1415\ldots$, która nie ma większego związku
  z~rzeczywistym światem. Niemniej sam fakt, że~jakaś teoria może wyróżnić
  jakąkolwiek wartość stałej struktury subtelnej, miał na pewno spore
  znaczenie dla osób zajmujących~się tematyką stałej struktury subtelnej.

  Osobom bardziej zainteresowanym teorią Staruszkiewicza polecam lekturę
  rozdziału~$4$ w~cytowanej już wielokrotnie pracy
  \parencite{Herdegen-Infrared-structure-beyond-locality-ETC-Ver-2024}.

\end{frame}
% ##################





% ##################
\begin{frame}
  \frametitle{Struktura asymptotyczna klasycznej elektrodynamiki}


  Artykuł \colorhref{https://arxiv.org/abs/2411.13977}
  {\textit{Long-range effects in asymptotic fields and~angular momentu~of
      classical field electrodynamics}} z~$1995$ roku jest poświęcony
  badaniom asmyptotycznego zachowania klasycznego pola elektromagnetycznego
  i~pola Diraca
  \parencite{Herdegen-Long-range-effects-in-asymptotic-ETC-Pub-1995}.
  Badanie to jest podjęte z~jawną motywacją podstawienia procedury
  kwantyzacji na solidnym gruncie.

  Dla przypadku pola elektromagnetycznego badane są jego asymptotyki
  w~świetlnej nieskończoności, korzystając z~funkcji jednorodnych i~dobrych
  własności ich całkowania po kierunkach świetlnych. Analogiczne badanie
  jest przeprowadzone dla czasopodobnej przyszłości dla pola Diraca. Jak
  sam ten opis wskazuje, geometryczne własności czasoprzestrzeni
  Minkowskiego są w~niej mocno wykorzystywane.

  Pomimo braku dowodów dla pewnych jawnie zapostulowanych własności
  oddziałujące teorii pola elektromagnetycznego i~pola Diraca, poza tym jej
  fragmentem praca jest matematycznie ścisła i~zawiera wiele interesujących
  wyników.

\end{frame}
% ##################





% ##################
\begin{frame}
  \frametitle{Struktura asymptotyczna klasycznej elektrodynamiki}


  Wśród nich należy wymienić procedurę obliczania całkowitego momentu pędu
  pola elektromagnetycznego (rozbieżne całki i~jak ich uniknąć), rozkład
  energii i~pędu systemu na sumę energii oraz pędu asymptotycznych pól
  promieniowania i~materii oraz brak analogicznego wyniku dla momentu pędu.
  W~tym ostatnim przypadku pojawia~się człon zawierający wkład do momentu
  pędu zarówno od pól Coulomba jak i~od fal elektromagnetycznych.

  Należy dodać, że~tego członu łączącego wkład materii i~promieniowania
  można~się pozbyć wprowadzając nielokalną transformację pola Diraca. Jeśli
  zostanie to wykonane, to wówczas można otrzymać wyniki zgodne
  z~wcześniejszymi pracami prof. Staruszkiewicza.

  W~mojej ocenie ten artykuł jest bardzo wartościowy i~wart dokładnego
  studiowania w~roku $2025$, ze względu na duże zainteresowanie problemami
  podczerwonymi w~ostatnich latach.

\end{frame}
% ##################





% ##################
\begin{frame}
  \frametitle{Konstrukcja nielokalnej asymptotycznej
    elektrodynamiki}


  Ostatnią pracą prof.~Herdegena jaką czas pozawala nam pokrótce omówić
  jest \colorhref{https://arxiv.org/abs/hep-th/9711066}
  {\textit{Semidirect product~of CCR and~CAR algebras and~asymptotic states
      in~quantum electrodynamics}} z~$1998$ roku
  \parencite{Herdegen-Semidirect-product-of-CCR-and-CAR-algebras-ETC-Pub-1998}.
  Praca ta jest fascynująca matematycznie,
  jak i~pod względem pewnych wyników fizycznych, ale wymaga dokładnego
  studiowania, podbudowanego dużą wiedzą i~zrozumieniem matematyki.
  W~poniższej dyskusji musieliśmy bardzo wiele elementów uprościć.

  Główny wątek tej pracy można, znów w~dużym uproszczeniu, opisać
  w~następujący sposób. Z~omówionej poprzednio pracy prof.~Herdegena
  wiemy jak wyglądają warunki asymptotyczne w~„świetlnej nieskończoności”.
  Mechanika Newtona w~sformułowaniu hamiltonowskim i~nierelatywistyczna
  mechanika kwantowa uczą nas, jak wychodząc ze struktury symplektycznej
  klasycznej teorii sformułować warunku określające teorię kwantową.

\end{frame}
% ##################





% ##################
\begin{frame}
  \frametitle{Konstrukcja nielokalnej asymptotycznej
    elektrodynamiki}


  Rozważmy teraz funkcje gładkie o~zwartym nośniku (znikające poza kulą
  o~skończonym promieniu), który to nośnika zawarty jest  w~„świetlnej
  nieskończoności”. Dla tych funkcji możemy określić symplektyczny iloczyn
  skalarny.

  Następnie korzystamy z~twierdzenia
  \colorhref{https://en.wikipedia.org/wiki/Hermann_Weyl}{Weyl’a},
  które mówi jak dla danej przestrzeni symplektycznej stworzyć algebrę
  $C^{ \HorSpaceOne * }$
  (zob. tom drugi książki Brattelego i~Robinsona
  \parencite{Bratteli-Robinson-Operator-algebras-ETC-Vol-I-Pub-2002}).
  Osoby znające trochę algebraiczną kwantową teorią pola wiedzą, że~to
  zasadniczo pozwala nam mówić o~teorii kwantowej.

  Pojawia~się problem, że~startują z~przestrzeni symplektycznej funkcji
  o~zwartym nośniku możemy otrzymać tylko \alert{lokalne} obserwable,
  więc z~oczywistych względów nie obejmują one pól asymptotycznych.

\end{frame}
% ##################





% % ##################
% \begin{frame}
%   \frametitle{Konstrukcja nielokalnej asymptotycznej
%     elektrodynamiki}


%   Wyniki poprzednio omówionej pracy prof.~Herdegena pozawalają zrozumieć
%   jaką wagę mają warunki w~„świetlnej nieskończoności” i~określić
%   odpowiednią przestrzeń symplektyczną. Należy rozważyć funkcje gładkie
%   o~zwartym nośniku (znikające poza kulą o~skończonym promieniu) zawartym
%   w~„świetlnej nieskończoności”, na których określamy odpowiedni skalarny
%   iloczyn symplektyczny.

%   Następnie korzystamy z~twierdzenia
%   \colorhref{https://en.wikipedia.org/wiki/Hermann_Weyl}{Weyl’a},
%   które mówi jak dla danej przestrzeni symplektycznej stworzyć algebrę
%   $C^{ \HorSpaceOne * }$
%   (zob. tom drugi książki Brattelego i~Robinsona
%   \parencite{Bratteli-Robinson-Operator-algebras-ETC-Vol-I-Pub-2002}).
%   Osoby znające trochę algebraiczną kwantową teorią pola wiedzą, że~to
%   zasadniczo pozwala nam mówić o~teorii kwantowej.

%   Pojawia~się problem, że~startują z~przestrzeni symplektycznej funkcji
%   o~zwartym nośniku możemy otrzymać tylko \alert{lokalne} obserwable,
%   więc z~oczywistych względów nie obejmują one pól asymptotycznych.

% \end{frame}
% % ##################





% ##################
\begin{frame}
  \frametitle{Konstrukcja nielokalnej asymptotycznej
    elektrodynamiki}


  By temu zaradzić prof.~Herdegen rozszerza algebrę pól asymptotycznych.
  w~następujący sposób. Obok funkcji gładkich o~zwartym nośniku
  w~„świetlnej nieskończoności” do przestrzeni symplektycznej dodawane
  są pewne odpowiednio szybko gasnące funkcje. Procedura ta nie jest wcale
  tak prosta jakby~się zdawało, bo trzeba zagwarantować, że~rozszerzona
  przestrzeń też jest przestrzenią symplektyczną.

  Twierdzenie Weyl’a zastosowane do nowej przestrzeni symplektycznej daje
  nam algebrę $C^{ \HorSpaceOne * }$, z~tą różnicą, że~elementy
  odpowiadające gasnącym funkcją „generują” w~pewnym konkretnym sensie
  \alert{nielokalne} elementy tej algebry.

  Z~mojego punktu widzenia szczególnie trudną częścią pracy jest utworzenie
  reprezentacji tej algebry. Prof.~Herdegen jest w~stanie to zrobić
  korzystając z~odpowiednich
  \colorhref{https://en.wikipedia.org/wiki/Direct\_integral}{całek prostych}
  przestrzeni Hilberta, ale~to temat na zupełnie osobne spotkanie.

\end{frame}
% ##################





% ##################
\begin{frame}
  \frametitle{Konstrukcja nielokalnej asymptotycznej
    elektrodynamiki}


  Podsumowując w~tej pracy prof.~Herdegen zbudował algebrę z~reprezentacją
  na tyle dużą, by objęła operatory kreacji i~anihilacji bozonów
  i~fermionów, a~także odpowiednie pola asymptotyczne elektrodynamiki.
  Znalazł też warunek na to, by w~teorii opisywanej taką algebrą ładunek
  elektryczny był skwantowany, por. \textit{Colloraly $3.1$}
  i~poprzedzający ją dyskusja
  w~\parencite{Herdegen-Semidirect-product-of-CCR-and-CAR-algebras-ETC-Pub-1998}.

  Prace naukowe prof.~Herdegena nie ograniczają~się do problemów
  podczerownych, ale stanowią one główny temat jego zainteresowań. Choć nie
  możemy ich omówić tutaj, to chcielibyśmy wspomnieć o~kilku jego dalszych
  pracach ich dotyczących:
  \colorhref{https://arxiv.org/abs/1210.1731}{\textit{Infraparticle problem,
      asymptotic fields and Haag-Ruelle theory}} z~$2013$ roku
  \parencite{Herdegen-Infraparticle-Problem-Asymptotic-Fields-ETC-Pub-2013},
  napisaną razem z~Pawłem Duchem \colorhref{https://arxiv.org/abs/1410.7665}
  {\textit{Massless asymptotic fields and~Haag-Ruelle theory}} z~$2014$
  \parencite{Duch-Herdegen-Massless-asymptotic-fields-and-ETC-Pub-2014},
  \colorhref{https://arxiv.org/abs/2012.14170}
  {\textit{Infrared problem vs gague choice: scattering~of classical Dirac
      field}} z~$2021$
  \parencite{Herdegen-Infrared-Problem-vs-Gauge-Choice-ETC-Ver-2021},
  \colorhref{https://arxiv.org/abs/2107.02044}{\textit{Almost radial
      gauge}} z~$2022$
  \parencite{Herdegen-Almost-radial-gauge-ETC-Pub-2022}
  i~\colorhref{https://arxiv.org/abs/2310.20560}
  {\textit{Undressing the~electron}} z~$2024$
  \parencite{Herdegen-Undressing-the-electron-Pub-2024}.

\end{frame}
% ##################





% % ##################
% \begin{frame}
%   \frametitle{Pytania, problemy, wątpliwości}




% \end{frame}
% % ##################










% ######################################
\appendix
% ######################################





% ######################################
\EndingSlide{Dziękuję! Pytania?}
% ######################################





% ####################################################################
% ####################################################################
% Bibliography

\printbibliography










% ############################

% Koniec dokumentu
\end{document}
