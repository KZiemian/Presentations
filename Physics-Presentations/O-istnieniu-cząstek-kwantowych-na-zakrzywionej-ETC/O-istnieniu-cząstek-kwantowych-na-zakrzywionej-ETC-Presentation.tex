\RequirePackage[l2tabu, orthodox]{nag}



\ifx\PresentationStyle\notset
  \def\PresentationStyle{light}
\fi



\documentclass[10pt,t]{beamer}
\mode<presentation>
\usetheme[style=\PresentationStyle]{jagiellonian}



% ---------------------------------------
% Configuration files of Jagiellonian loceted in catalog preambule
% ---------------------------------------
% Configuration for polish language
% Need description
\usepackage[english]{babel}





% % ------------------------------
% % Better support of polish chars in technical parts of PDF
% % ------------------------------
% \hypersetup{pdfencoding=auto,psdextra}

% Package "textpos" give as enviroment "textblock" which is very usefull in
% arranging text on slides.

% This is standard configuration of "textpos"
\usepackage[overlay,absolute]{textpos}

% If you need to see bounds of "textblock's" comment line above and uncomment
% one below.

% Caution! When showboxes option is on significant ammunt of space is add
% to the top of textblock and as such, everyting put in them gone down.
% We need to check how to remove this bug.

% \usepackage[showboxes,overlay,absolute]{textpos}



% Setting scale length for package "textpos"
\setlength{\TPHorizModule}{10mm}
\setlength{\TPVertModule}{\TPHorizModule}


% ---------------------------------------
% TikZ
% ---------------------------------------
% Importing TikZ libraries
\usetikzlibrary{arrows.meta}
\usetikzlibrary{positioning}





% % Configuration package "bm" that need for making bold symbols
% \newcommand{\bmmax}{0}
% \newcommand{\hmmax}{0}
% \usepackage{bm}




% ---------------------------------------
% Packages for scientific texts
% ---------------------------------------
% \let\lll\undefined  % Sometimes you must use this line to allow
% "amsmath" package to works with packages with packages for polish
% languge imported
% /preambul/LanguageSettings/JagiellonianPolishLanguageSettings.tex.
% This comments (probably) removes polish letter Ł.
\usepackage{amsmath}  % Packages from American Mathematical Society (AMS)
\usepackage{amssymb}
\usepackage{amscd}
\usepackage{amsthm}
\usepackage{siunitx}  % Package for typsetting SI units.
\usepackage{upgreek}  % Better looking greek letters.
% Example of using upgreek: pi = \uppi


\usepackage{calrsfs}  % Zmienia czcionkę kaligraficzną w \mathcal
% na ładniejszą. Może w innych miejscach robi to samo, ale o tym nic
% nie wiem.










% ---------------------------------------
% Packages written for lectures "Geometria 3D dla twórców gier wideo"
% ---------------------------------------
% \usepackage{./ProgramowanieSymulacjiFizykiPaczki/ProgramowanieSymulacjiFizyki}
% \usepackage{./ProgramowanieSymulacjiFizykiPaczki/ProgramowanieSymulacjiFizykiIndeksy}
% \usepackage{./ProgramowanieSymulacjiFizykiPaczki/ProgramowanieSymulacjiFizykiTikZStyle}





% !!!!!!!!!!!!!!!!!!!!!!!!!!!!!!
% !!!!!!!!!!!!!!!!!!!!!!!!!!!!!!
% EVIL STUFF
\if\JUlogotitle1
\edef\LogoJUPath{LogoJU_\JUlogoLang/LogoJU_\JUlogoShape_\JUlogoColor.pdf}
\titlegraphic{\hfill\includegraphics[scale=0.22]
{./JagiellonianPictures/\LogoJUPath}}
\fi
% ---------------------------------------
% Commands for handling colors
% ---------------------------------------


% Command for setting normal text color for some text in math modestyle
% Text color depend on used style of Jagiellonian

% Beamer version of command
\newcommand{\TextWithNormalTextColor}[1]{%
  {\color{jNormalTextFGColor}
    \setbeamercolor{math text}{fg=jNormalTextFGColor} {#1}}
}

% Article and similar classes version of command
% \newcommand{\TextWithNormalTextColor}[1]{%
%   {\color{jNormalTextsFGColor} {#1}}
% }



% Beamer version of command
\newcommand{\NormalTextInMathMode}[1]{%
  {\color{jNormalTextFGColor}
    \setbeamercolor{math text}{fg=jNormalTextFGColor} \text{#1}}
}


% Article and similar classes version of command
% \newcommand{\NormalTextInMathMode}[1]{%
%   {\color{jNormalTextsFGColor} \text{#1}}
% }




% Command that sets color of some mathematical text to the same color
% that has normal text in header (?)

% Beamer version of the command
\newcommand{\MathTextFrametitleFGColor}[1]{%
  {\color{jFrametitleFGColor}
    \setbeamercolor{math text}{fg=jFrametitleFGColor} #1}
}

% Article and similar classes version of the command
% \newcommand{\MathTextWhiteColor}[1]{{\color{jFrametitleFGColor} #1}}





% Command for setting color of alert text for some text in math modestyle

% Beamer version of the command
\newcommand{\MathTextAlertColor}[1]{%
  {\color{jOrange} \setbeamercolor{math text}{fg=jOrange} #1}
}

% Article and similar classes version of the command
% \newcommand{\MathTextAlertColor}[1]{{\color{jOrange} #1}}





% Command that allow you to sets chosen color as the color of some text into
% math mode. Due to some nuances in the way that Beamer handle colors
% it not work in all cases. We hope that in the future we will improve it.

% Beamer version of the command
\newcommand{\SetMathTextsColor}[2]{%
  {\color{#1} \setbeamercolor{math text}{fg=#1} #2}
}


% Article and similar classes version of the command
% \newcommand{\SetMathTextColor}[2]{{\color{#1} #2}}










% ---------------------------------------
% Commands for setting background pictures for some slides
% ---------------------------------------
\newcommand{\TitleBackgroundPicture}
{./PresentationPictures/CommonPictures/Cute_dragon_BG_dark.png}
\newcommand{\SectionBackgroundPicture}
{./PresentationPictures/CommonPictures/Cute_dragon_small_BG_light.png}



\newcommand{\TitleSlideWithPicture}{
  \begingroup

  \usebackgroundtemplate{ % \hspace*{-11.5em}
    \includegraphics[height=\paperheight]{\TitleBackgroundPicture}}

  \maketitle

  \endgroup
}





\newcommand{\SectionSlideWithPicture}[1]{%
  \begingroup

  \usebackgroundtemplate{ % \hspace*{-11.5em}
    \includegraphics[height=\paperheight]{\SectionBackgroundPicture}}

  \setbeamercolor{titlelike}{fg=normal text.fg}

  \section{#1}

  \endgroup
}





\newcommand{\EndingSlide}[1]{%
  \begin{frame}[standout]

    \begingroup

    \color{jFrametitleFGColor}

    #1

    \endgroup

  \end{frame}
}










% ------------------------------------------------------
% BibLaTeX
% ------------------------------------------------------
% Package biblatex, with biber as its backend, allow us to handle
% bibliography entries that use Unicode symbols outside ASCII.
\usepackage[
language=polish,
backend=biber,
style=alphabetic,
url=false,
eprint=true,
]{biblatex}

\addbibresource{Stochastic-quantization-and-singular-stochastic-equations-Bibliography.bib}





% ------------------------------------------------------
% Local packages
% ------------------------------------------------------
% Special configuration for this particular presentation
\usepackage{./Local-packages/local-settings}

% Package containing various command useful for working with a text
\usepackage{./Local-packages/general-commands}

% Package containing commands and other code useful for working with
% mathematical text
\usepackage{./Local-packages/math-commands}





% ---------------------------------------
% Configuration for this particular presentation
% ---------------------------------------










% ---------------------------------------------------------------------
\title{O~istnieniu cząstek kwantowych na~zakrzywionej
  czasoprzestrzeni}

\author{Kamil Ziemian \\
  \texttt{kziemianfvt@gmail.com}}


\date{}
% ---------------------------------------------------------------------










% ####################################################################
% Początek dokumentu
\begin{document}
% ####################################################################





% Wyrównanie do lewej z łamaniem wyrazów

\RaggedRight





% ######################################
\maketitle
% ######################################





% ######################################
\begin{frame}
  \frametitle{Spis treści}


  \tableofcontents % Spis treści

\end{frame}
% ######################################





% ######################################
\section{Cząstki w~zakrzywionej czasoprzestrzeni}
% ######################################





% ##################
\begin{frame}
  \frametitle{Co to jest cząstka?}


  Zacznijmy od zadania sobie pytania, co to jest cząstka?
  Proste: nieskończenie mała kulka poruszając~się w~przestrzeni.

  Odkładając żarty na bok, każdy z~nas ma jakąś intuicję tego czym cząstka
  jest i~na tym intuicyjnym poziomie, trudno dostrzec czemu w~danej teorii
  takie obiekty miałyby nieistnień. Na tym seminarium będziemy starali~się
  wyjaśnić, czemu w~konkretnym formalizmie teoretycznym, to pojęcie staje~się
  problematyczne.

  Wiemy, że~w~płaskiej czasoprzestrzeni Minkowskiego cząstka kwantowa to
  dobrze określone pojęcie. Czasoprzestrzeń wokół nas jest prawie płaska,
  więc to, że~w~różnych eksperymentach obserwujemy cząstki, nie powinno nas
  dziwić. Pytanie, przed którym stajemy teraz jest to, czy jeśli krzywizna
  czasoprzestrzenie będzie rosnąć pojęcie cząstki kwantowej zachowa swój
  sens?

\end{frame}
% ##################





% ##################
\begin{frame}
  \frametitle{Niekwantowe cząstki w~OTW}


  Na tym seminarium nie będziemy~się zagłębiać w~zawiłości równań Einsteina,
  gdyż nie jest to potrzebne do zrozumienia istoty problemu. Potrzebujemy
  jednak powiedzieć kilka rzeczy o~czasoprzestrzeniach, które będziemy
  rozważać.

  Przyjmiemy, że~geometria czasoprzestrzeni jest zadana przez
  \alert{ustalony} rozkład materii i~od teraz będziemy rozważać tylko
  cząstki, które \alert{nie} dają wkładu do zastanego pola grawitacyjne.
  To oczywiście jest tylko pewnym przybliżeniem realnej sytuacji, pozwala
  jednak postawić kilka pytań w~bardzo precyzyjny sposób.

  W~takiej sytuacji rozważanie cząstek punktowych w~\textsc{otw} nie stwarza
  żadnych problemów. Przez „niekwantowy” foton rozumiem cząstkę punktową
  o~zerowej masie, której czteroprędkość jest zawsze wektorem świetlnym.
  Analogicznie będziemy rozumieli elektron, którego czteroprędkość
  jest zawsze wektorem czasopodobnym.

\end{frame}
% ##################















% ######################################
\section{Czy kwantowe fotony istnieją w~zakrzywionej
  czasoprzestrzeni?}
% ######################################


% ##################
\begin{frame}
  \frametitle{Problem przestrzeni Hilberta}


  Zanim przejdziemy do problemu cząstek kwantowych, musimy wspomnieć o~kilku
  kwestiach technicznych. Dla nierelatywistycznej mechaniki kwantowej
  istnieje tylko \alert{jedna} przestrzeń Hilberta $\Hcal$ . Dla kwantowej
  teorii pola możliwych przestrzeni Hilberta jest ich nieprzeliczalnie
  wiele, albo jeszcze więcej.

  Trochę bardziej ściśle. Twierdzenie Stone'a-von Neumanna głosi,
  że~mechanika kwantowa żyje w~ośrodkowej przestrzeni Hilberta, a~ta jest
  tylko jedna, z~dokładnością do transformacji unitarnej. Kwantowej teorii
  pola nie można sformułować w~ośrodkowej przestrzeni Hilberta,
  a~przestrzeni nieośrodkowych znamy już nieprzeliczalnie wiele.

  Co więcej konstrukcje takich modeli \textsc{qft} jak $\Phi_{ d }^{ \, 4 }$,
  dla $d = 2, 3$, dostarczają nam jawnego przykładów tego problemu.
  Jeśli człon oddziaływania oznaczymy jako $\lambda \Phi^{ 4 } / 4!$, to dla każdej
  wartości parametru $\lambda_{ 1 }$ potrafimy zbudować przestrzeń Hilberta tej
  teorii, która \alert{nie} jest unitarna równoważna tej dla
  $\lambda_{ 2 } \neq \lambda_{ 1 }$.

\end{frame}
% ##################





% ##################
\begin{frame}
  \frametitle{Problem przestrzeni Hilberta}


  Jednak zmiana przestrzeni Hilberta wpływa na widmo wszystkich operatorów,
  więc różne przestrzeni Hilberta „tej samej” teorii, odpowiadają fizycznie
  różnym zjawiskom. Dlatego problemu „W~której przestrzeni Hilberta
  jesteśmy?” nie możemy łatwo zignorować.

\end{frame}
% ##################





% ##################
\begin{frame}
  \frametitle{Elektrodynamika Clerka Maxwella}


  Klasyczna elektrodynamika Clerka Maxwella relatywnie łatwo~się uogólnia
  do przypadku zakrzywionej czasoprzestrzeni, dlatego zilustrujemy nasz
  problem na jej przykładzie.

  Pole zapisujemy za pomocą tensora Clerka Maxwell.
  \begin{equation}
    \label{eq:Czemu-rozwazanie-01}
    F_{ \mu \nu }( x )
    =
    \begin{pmatrix}
      \hphantom{-} 0
      & \frac{ 1 }{ c } E_{ x }( x ) & \frac{ 1 }{ c } E_{ y }( x )
      & \frac{ 1 }{ c } E_{ z }( x ) \\
      -\frac{ 1 }{ c } E_{ x }( x )
      & \hphantom{-} 0 & -B_{ z }( x ) & \hphantom{-} B_{ y }( x ) \\
      -\frac{ 1 }{ c } E_{ y }( x )
      & \hphantom{-} B_{ z }( x ) & \hphantom{-} 0 & -B_{ z }( x ) \\
      -\frac{ 1 }{ c } E_{ z }( x )
      & -B_{ y }( x ) & \hphantom{-} B_{ x }( x ) & \hphantom{-} 0
    \end{pmatrix}
  \end{equation}
  Jeśli wprowadzimy czteroformę pola
  $F( x ) = F_{ \mu \nu }( x ) \, dx^{ \mu } \wedge dx^{ \nu }$
  i~czteroformę potencjału $A( x ) = A_{ \mu }( x ) \, dx^{ \mu }$ to
  równania Clerka Maxwella przyjmują formę (trochę upraszczamy)
  \begin{equation}
    F( x ) = \widetilde{d}A( x ), \quad dF( x ) = 0.
  \end{equation}
  Równania te mają sens na dowolnej zakrzywionej czasoprzestrzeni.

\end{frame}
% ##################





% ##################
\begin{frame}
  \frametitle{Elektrodynamika Clerka Maxwella}


  \begin{subequations}
    \begin{align}
      \label{eq:Czemu-rozwazanie-02-A}
      &F( x ) = \widetilde{d}A( x ), \quad dF( x ) = 0, \quad
        F( x ) = F_{ \mu \nu }( x ) \, dx^{ \mu } \wedge dx^{ \nu }, \\[0.3em]
      \label{eq:Czemu-rozwazanie-02-B}
      &F_{ \mu \nu }( x )
        =
        \begin{pmatrix}
          \hphantom{-} 0
          & \frac{ 1 }{ c } E_{ x }( x ) & \frac{ 1 }{ c } E_{ y }( x )
          & \frac{ 1 }{ c } E_{ z }( x ) \\
          -\frac{ 1 }{ c } E_{ x }( x )
          & \hphantom{-} 0 & -B_{ z }( x ) & \hphantom{-} B_{ y }( x ) \\
          -\frac{ 1 }{ c } E_{ y }( x )
          & \hphantom{-} B_{ z }( x ) & \hphantom{-} 0 & -B_{ z }( x ) \\
          -\frac{ 1 }{ c } E_{ z }( x )
          & -B_{ y }( x ) & \hphantom{-} B_{ x }( x ) & \hphantom{-} 0
        \end{pmatrix}
    \end{align}
  \end{subequations}

  Te równania definiują nam elektrodynamikę na dowolnej gładkiej
  rozmaitości. Teraz potrzeba je skwantować.

  I tu pojawia się problemy. Będziemy argumentowali, że~standardowa
  kwantyzacja pola elektromagnetycznego silnie korzysta z~własności
  czasoprzestrzeni Minkowskiego i~dlatego nie da~się jej przenieść
  na ogólny przypadek zakrzywionej czasoprzestrzeni.

\end{frame}
% ##################





% ##################
\begin{frame}
  \frametitle{Uproszczona kwantyzacja w~czasoprzestrzeni
    Minkowskiego}


  1. Wybierz inercjalny układ współrzędnych.

  2. Wypisz w~nim równania Clerka Maxwella na czteropotencjał jako
  \begin{equation}
    \label{eq:Czemu-rozwazanie-03}
    \square \, A_{ \mu }( x ) = 0, \quad \partial^{ \nu } A_{ \nu }( x ) = 0
  \end{equation}

  \vspace{-1.5em}



  3. Zapisz rozwiązanie równań Clerka Maxwella jako
  \begin{equation}
    \label{eq:Czemu-rozwazanie-04}
    \begin{split}
      A_{ \mu }( x )
      &=
        \int d^{ 3 }k \sum_{ \mu = \pm 1 } \frac{ 1 }{ \omega( \vec{ k } ) }
        \left( \vec{ e }^{ \, ( \mu ) }\!( \vec{ k } ) \,
        a_{ \vec{ k } }^{ ( \mu ) }\!( t ) \,
        e^{ -i \omega( \vec{ k } ) t + i \vec{ k } \cdot \vec{ x } } \, + \right. \\
      &\hspace{1.5em}
        \left. + \, \overline{ \vece^{ \, ( \mu ) } }\!( \vec{ k } ) \,
        \overline{ a }_{ \vec{ k } }^{ \, ( \mu ) }\!( t ) \,
        e^{ \, i \omega( \vec{ k } ) t - i \vec{ k } \cdot \vec{ x } } \right).
    \end{split}
  \end{equation}

  \vspace{-1em}



  4. Podnieś $A_{ \mu }( x )$, $a_{ \vec{ k } }^{ ( \mu ) }\!( t )$ do rangi
  operatorów: $\widehat{A}_{ \mu }( x )$,
  $\widehat{a}_{ \vec{ k } }^{ \, ( \mu ) }\!( t )$.

\end{frame}
% ##################





% ##################
\begin{frame}
  \frametitle{Uproszczona kwantyzacja w~czasoprzestrzeni
    Minkowskiego}

  \vspace{-2em}


  \begin{equation}
    \label{eq:Czemu-rozwazanie-05}
    \begin{split}
      \widehat{A}_{ \mu }( x )
      &=
        \int d^{ 3 }k \sum_{ \mu = \pm 1 } \frac{ 1 }{ \omega( \vec{ k } ) }
        \left( \vec{ e }^{ \, ( \mu ) }\!( \vec{ k } ) \,
        \widehat{a}_{ \veck }^{ \, ( \mu ) }\!( t ) \,
        e^{ -i \omega( \vec{ k } ) t + i \veck \cdot \vec{ x } } \, + \right. \\
      &\hspace{1.5em}
        \left. + \, \overline{ \vece^{ \, ( \mu ) } }\!( \veck ) \,
        \big( \, \widehat{a}_{ \vec{ k } }^{ \, ( \mu ) } \!( t )
        \big)^{ \dagger } \,
        e^{ i \omega( \vec{ k } ) t - i \vec{ k } \cdot \vec{ x } } \right).
    \end{split}
  \end{equation}

  \vspace{-1em}



  5. Zinterpretuj operator przy
  $\exp( -i \omega( \vec{ k } ) t + i \vec{ k } \cdot \vec{ x } )$
  jako operator anihilacji fotonu. Poszukujemy więc stanu, który spełnia
  równanie
  \begin{equation}
    \label{eq:Czemu-rozwazanie-06}
    \widehat{a}_{ \vec{ k } }^{ \, ( \mu ) }\!( t ) | 0 \rangle = 0, \quad
    \forall \, \vec{ k }, \mu.
  \end{equation}

  \vspace{-1em}



  6. Zbuduj przestrzeń Hilberta kwantowej elektrodynamiki ze stanów
  zawierających $0, 1, 2, 3, \ldots,$ fotonów, czyli
  \begin{equation}
    \label{eq:Czemu-rozwazanie-07}
    \begin{split}
      &| 0 \rangle, \; \big( \widehat{a}^{ \, ( \mu ) }_{ \vec{ k } }\!( t )
        \big)^{ \dagger } | 0 \rangle, \;
        \big( \widehat{a}^{ \, ( \mu ) }_{ \vec{ k } }\!( t ) \big)^{ \dagger }
        \big( \widehat{a}^{ \, ( \nu ) }_{ \vec{ k }' }\!( t ) \big)^{ \dagger }
        | 0 \rangle, \\
      &\big( \widehat{a}^{ \, ( \mu ) }_{ \vec{ k } }\!( t ) \big)^{ \dagger }
        \big( \widehat{a}^{ \, ( \nu ) }_{ \vec{ k }' }\!( t ) \big)^{ \dagger }
        \big( \widehat{a}^{ \, ( \rho ) }_{ \vec{ k }'' }\!( t ) \big)^{ \dagger }
        | 0 \rangle, \\
      &\big( \widehat{a}^{ \, ( \mu ) }_{ \vec{ k } }\!( t ) \big)^{ \dagger }
        \big( \widehat{a}^{ \, ( \nu ) }_{ \vec{ k }' }\!( t ) \big)^{ \dagger }
        \big( \widehat{a}^{ \, ( \rho ) }_{ \vec{ k }'' }\!( t ) \big)^{ \dagger }
        \big( \widehat{a}^{ \, ( \sigma ) }_{ \vec{ k }''' }\!( t ) \big)^{ \dagger }
        | 0 \rangle, \; \ldots
    \end{split}
  \end{equation}

\end{frame}
% ##################





% ##################
\begin{frame}
  \frametitle{Musimy znaleźć właściwy stan próżni}


  \begin{equation}
    \label{eq:Czemu-rozwazanie-08}
    \widehat{a}_{ \vec{ k } }^{ \, ( \mu ) }( t ) | 0 \rangle = 0, \quad
    \forall \, \vec{ k }, \mu
  \end{equation}
  Jest zbyt wiele stanów spełniających tą relację. Wybór każdego z~nich daje
  inną przestrzeń Hilberta budowanej wedle przepisu
  \begin{equation}
    \label{eq:Czemu-rozwazanie-09}
    | 0 \rangle, \; \big( \widehat{a}^{ \, ( \mu ) }_{ \vec{ k } }\!( t )
    \big)^{ \dagger } | 0 \rangle, \;
    \big( \widehat{a}^{ \, ( \mu ) }_{ \vec{ k } }\!( t ) \big)^{ \dagger }
    \big( \widehat{a}^{ \, ( \nu ) }_{ \vec{ k }' }\!( t ) \big)^{ \dagger }
    | 0 \rangle, \ldots
  \end{equation}
  Każda taka przestrzeń to inna teoria fizyczna.

  Ale istnieje tylko \alert{jeden} stan $| 0 \rangle$ który jest taki sam
  w~każdym inercjalnym układzie współrzędnych. Inaczej mówiąc, spełnia
  on relację:
  \begin{equation}
    \label{eq:Czemu-rozwazanie-10}
    \widehat{U}( L, a ) | 0 \rangle = | 0 \rangle,
  \end{equation}
  gdzie $L$ to transformacja Lorentza, a~$a$ to czterowektor translacji, \\
  zaś $\widehat{U}( L, a )$ to operator unitarny odpowiedzialny
  za~implementację tej transformacji na przestrzeni Hilberta.

\end{frame}
% ##################





% ##################
\begin{frame}
  \frametitle{Jak duża jest nasza grupa symetrii?}


  Samo pole kwantowe $\widehat{A}_{ \mu }( x )$ wygląda tak samo w~każdym
  inercjalnym układzie współrzędnych, ale~istnieje tylko jeden stan
  $| 0 \rangle$, który wygląda tak samo w~każdym z~tych układów.

  Mówiąc inaczej, jest tylko jeden „stan próżni”, który jest pusty
  w~\alert{każdym} inercjalnym układzie odniesienia. Pozostałe są puste
  w~jednych układzie inercjalnym, a~zawierają cząstki w~innym. Możemy je
  nazwać „dziwnymi stanami próżni”.

  Tym samy otrzymaliśmy dobrze określoną elektrodynamikę kwantową
  w~czasoprzestrzeni Minkowskiego.

  Jeżeli weźmiemy jako przykład przestrzenie zakrzywionej widzialne
  gołym okiem powierzchnie w~trzech wymiarach, to łatwo zauważyć, że~ich
  grupa symetrii będzie zwykle mniejsza od~grupy symetrii całej przestrzeni,
  a~często będzie składała~się z~pojedynczego elementu: odwzorowania
  identyczności.

\end{frame}
% ##################





% ##################
\begin{frame}
  \frametitle{Czy to jest problem?}


  Analogicznie, jeśli rozważymy czasoprzestrzeń \textsc{otw}, to z~tych
  samych powodów co dla powierzchni, nie możemy liczyć na zbyt dużą grupę
  symetrii. Są ważne wyjątki od tej reguły, acz to nie zmienia tak bardzo
  całej sytuacji.

  Jeśli zaś nie mamy tak dużej grupy symetrii, jak grupa Poincar\'{e}’ego,
  to nie będziemy mogli znaleźć „poprawnego stanu próżni”, który jest
  pusty w~każdym układzie odniesienia. Przypomnijmy, że~zwykle uważa~się,
  iż~w~\textsc{otw} każdy układ odniesienia jest równie „(nie)inercjalny”.
  Wszystkie te „stany próżni” są bardzo dziwne.

  Wobec tych rozważań można wysunąć dwa poważne zastrzeżenia.

  \vspace{-0.7em}



  \begin{enumerate}
    \RaggedRight

  \item Przecież dobrze wiem, że~cząstki istnieją, więc to musi być błąd
    w~naszej teorii.

  \item To jest niuans matematyczny, bez znaczenia fizycznego.

  \end{enumerate}

\end{frame}
% ##################





% ##################
\begin{frame}
  \frametitle{Czy to jest problem?}


  Gdy chodzi o~pierwszą kwestię, to ze względu na charakter grawitacji
  większość obserwowanej czasoprzestrzeni jest praktycznie płaska. Więc
  nie powinno nas specjalnie dziwić, że~formalizm rozwinięty dla
  czasoprzestrzeni Minkowskiego działa bardzo dobrze.

  Z~punktu widzenia czysto teoretycznego, ważne jest to czy teorię
  cząstek będzie można odtworzyć w~granicy, gdy krzywizna czasoprzestrzeni
  dąży do zera.

  Odnośnie drugiej uwagi, że~te rozważania nie mają sensu fizycznego, to
  fakt, że~w~zakrzywionej czasoprzestrzeni nie ma uniwersalnego stanu próżni
  stoi za słynnym zjawiskiem promieniowania czarnych dziur. Słynny, bo miało
  dobry marketing. Obliczenia ilościowe są skomplikowane, ale ten
  efekt można podsumować w~następujący sposób. Jeśli wybierzemy „stan
  próżni” pusty w~układzie odniesienia nieskończonej przeszłości czarnej
  dziury, to nie będzie on pusty w~układzie odniesienia nieskończonej
  przyszłości tej dziury.

\end{frame}
% ##################





% ##################
\begin{frame}
  \frametitle{Społeczność AQFT taka jest}



  \begin{figure}

    \label{fig:IT-guys-and-users}

    \centering


    \includegraphics[scale=0.48]
    {./Presentation-pictures/AQFT-community-Im-about.jpeg}

  \end{figure}

\end{frame}
% ##################





% % ##################
% \begin{frame}
%   \frametitle{Komentarz do 2}


%   W~czasoprzestrzenni Minkowskiego zbiór operatorów anihilacji
%   $a^{ ( \mu ) }_{ \vec{ k } }( t )$ wszystkich układów inercjalnych
%   jest stosunkowo mały. Ogranicza go warunek równoważny
%   niezmienniczości próżni:
%   \begin{equation}
%     \label{eq:Czemu-rozwazanie-10}
%     U^{ \dagger }( L, a ) a^{ ( \mu ) }_{ \vec{ k } }( t ) U( L, a )
%     =
%     a^{ ( \mu ) }_{ \vec{ k }' }( t' ),
%   \end{equation}
%   przy czym $( L, a )$ przekształca $t$, $\vec{ k }$ na $t'$,
%   $\vec{ k }'$.

%   W~zakrzywionej czasoprzestrzenie zwykle nie będzie żadnego takiego
%   ograniczenia, więc klasa operatorów anihilacji będzie zbyt duża, by
%   znaleźć dla nich wspólną próżnię.

% \end{frame}
% % ##################





% ##################
\begin{frame}
  \frametitle{Trochę radykalnych wniosków}


  Pojęcie cząstki kwantowej ma sens w układach inercjalnych w~płaskiej
  czasoprzestrzeni. W~czasoprzestrzeni zakrzywionej mamy problem typu:
  ile jest cząstki w~cząstce kwantowej? Można postawić pytanie: czy
  nazywanie takiego obiektu cząstką to nie jest tylko rezultat bezwładności
  języka?

  Efekty takie jak promieniowanie czarnych dziur, każą~się nam
  zastanowić, czy szukanie takiej „globalnej” przestrzeni Hilberta to
  nie jest zbyt mocne wymaganie? Takie rozważania są oczywiście mile widziane
  przez społeczność algebraicznej kwantowej teorii pola (\textsc{aqft},
  ang.~\textit{Algebraic Quantum Field Theory}). Dlatego teraz naszkicujemy,
  jak społeczność ta proponuje rozwiązać ten problem.

\end{frame}
% ##################










% ######################################
\section{Algebraiczna QFT w~zakrzywionej czasoprzestrzeni}
% ######################################



% ##################
\begin{frame}
  \frametitle{Szkic podejścia algebraicznego}


  Przypomnijmy, że~zakładamy, iż~geometria czasoprzestrzeni jest ustalona
  i~rozważamy tylko ruch obiektów tak „lekkich”, że~ich wkład do pola
  grawitacyjnego możemy zaniedbać. Jest to duże uproszczenie, ale problem
  i~bez tego jest wystarczająco trudny.

  W~\textsc{aqft} podstawowym obiektem są \alert{lokalne} algebry,
  czyli algebry przypisane do ograniczonych obszarów czasoprzestrzeni
  $\Ocal$. Dokładniej, chcemy by te obszary były zbiorami otwartymi
  o~zwartym domknięciu i~wszystkie te pojęcia mają sens w~dowolnej
  zakrzywionej czasoprzestrzeni.

  Algebra ta $\Acal( \Ocal )$ reprezentuje wszystkie (kwantowe) wielkości
  fizyczne, które można zmierzyć w~obszarze $\Ocal$. Jednocześnie można
  wprowadzić \alert{lokalną} przestrzeń Hilberta, jeśli tylko znajdziemy
  odpowiedni stan na algebrze $\Acal( \Ocal )$. Fakt ten jest treścią
  twierdzenia \textsc{gns} (Gelfand-Naimark-Segal)

\end{frame}
% ##################





% ##################
\begin{frame}
  \frametitle{Problemy podejścia algebraicznego}


  Konstrukcja \textsc{gns} powinna więc pozwolić nam odtworzyć cząstki
  jako byty \alert{lokalne}, co zgadza~się dobrze z~naszą intuicją fizyczną.

  Ten szkic raczej nie pozwali zrozumieć jak działa podejście algebraiczne,
  dlatego należy~się kilka słów komentarza. Podstawowa trudność w~wszystkich
  matematycznych podejścia do \textsc{qft}, takich jak \textsc{aqft}, polega
  na skonstruowaniu konkretnego modelu. W~naszym wypadku chodzi o~zbudowanie
  odpowiedniej rodziny algebr $\Acal( \Ocal )$. W~czterech wymiarach umiemy
  to zrobić jedynie perturbacyjnie.

  W~ogromnym uproszczeniu, problem bierze~się stąd, że~aby skonstruować model
  pola kwantowego w~$d$ wymiarach, musimy zacząć od konstrukcji teorii
  na $d$ wymiarowej sieci, co nie jest żadnym problemem. Następnie należy
  wziąć granicę $L \searrow 0$, gdzie $L$ to odległość między węzłami sieci.
  Kontrola tej granicy dla $d = 4$ przekracza obecnie nasze możliwości.

\end{frame}
% ##################





% ##################
\begin{frame}
  \frametitle{Co udało~się osiągnąć?}


  Konstrukcja algebry $\Acal( \Ocal )$ pola \alert{swobodnego} na
  zakrzywionej czasoprzestrzeni, która ma sensowne własności, gdy chodzi
  o~upływ czasu (mówiąc technicznie, jest globalnie hiperboliczna), jest
  technicznie niełatwe, ale wykonalne.

  Dla modeli oddziałujących udało~się zbudować szereg perturbacyjny
  konkretnych algebr różnych pól kwantowych, ale nie udało~się wykazać, czy
  jest on zbieżny. Omówienie ich konstrukcji przekracza znacznie ramy tego
  wystąpienia, gdyż wymaga dyskusji własności przestrzeni globalnie
  hiperbolicznej, rozwiązań dystrybucyjnych równań różniczkowych na
  rozmaitościach, i~wielu innych tematów.

  Kończąc, można postawić sobie pytanie, czy podejście algebraiczne jest
  poprawnym fizycznie rozwiązaniem tego problemu? Dopiero testy
  eksperymentalne będą mogły rozstrzygnąć to pytanie. My ze swojej strony
  możemy tylko dodać, że~choć \textsc{aqft} zmaga~się z~wieloma poważnymi
  problemami, potrafi zainspirować do myślenia.

\end{frame}
% ##################





% % ##################
% \begin{frame}
%   \frametitle{Pola i~cząstki}


%   Bardziej szczegółowe omówienie tego podejścia przekracza obecnie
%   moje możliwości.

%   Czy to dobra droga? W~tej chwili nie umiem odpowiedzieć.

% \end{frame}
% % ##################










% ######################################
\appendix
% ######################################





% ######################################
\EndingSlide{Dziękuję. Pytania?}
% ######################################










% ##################
\begin{frame}
  \frametitle{Bibliografia}


  K. Fredenhagen, \textit{The impact of~the~algebraic approach on
    perturbative quantum field theory}, [Fre09]. Wystąpienie na
  \textit{Algebraic Quantum Field Theory – the first 50 Years},
  G\"{o}ttingen 2009.

  R.M. Wald, \textit{Axiomatic Quantum Field Theory in~Curved
    Spacetime}, [Wal09]. Wystąpienie na \textit{Algebraic Quantum
    Field Theory – the first 50 Years}, G\"{o}ttingen 2009.

  R. Haag, \textit{Local Algebras: A~look back at the~early years
    and~at~some successes and~missed opportunitie} [Haa09].
  Wystąpienie na \textit{Algebraic Quantum Field Theory – the first 50
    Years}, G\"{o}ttingen 2009.

  K. Fredenhagen, K. Rejzner, \textit{Perturbative algebraic quantum
    field theory}, arXiv: 1208.1428, [FR12].

\end{frame}
% ##################





% ##################
\begin{frame}
  \frametitle{Bibliografia}


  R. Brunetti, K. Fredenhagen, \textit{Microlocal Analysis
    and~Interacting Quantum Field Theories: Renormalization
    on~Physical Backgrounds}, Commun.Math.Phys, \textbf{208} (2000)
  623-661, arXiv: 9903.028, [BF00].

  R. Brunetti, K. Fredenhagen, \textit{Quantum Field Theory on~Curved
    Backgrounds}, Proceedings of the Kompaktkurs „Quantenfeldtheorie
  auf gekruemmten Raumzeiten” held at~Universitaet Potsdam, Germany,
  in 8.-12.10.2007, arXiv: 0901.2063, [BF09].

  M. D\"{u}etsch, K. Fredenhagen, K. J. Keller, K.~Rejzner,
  \textit{Dimensional Regularization in Position Space, and~a Forest
    Formula for Epstein-Glaser Renormalization}, arXiv: 1311.5424,
  [DFKR13].

\end{frame}
% ##################










% ######################################
\appendix
% ######################################





% % ######################################
% \EndingSlide{Dziękuję! Pytania?}
% % ######################################










% ####################################################################
% ####################################################################
% Bibliografia
% \bibliographystyle{plalpha}

% \bibliography{}{}





% ############################

% Koniec dokumentu
\end{document}
