% ---------------------------------------------------------------------
% Basic configuration of Beamera and Jagiellonian
% ---------------------------------------------------------------------
\RequirePackage[l2tabu, orthodox]{nag}



\ifx\PresentationStyle\notset
\def\PresentationStyle{dark}
\fi



\documentclass[10pt,t]{beamer}
\mode<presentation>
\usetheme[style=\PresentationStyle,logoColor=monochromaticJUwhite,JUlogotitle=yes]{jagiellonian}



% ---------------------------------------
% Configuration files of Jagiellonian loceted in catalog preambule
% ---------------------------------------
% Configuration for polish language
% Need description
\usepackage[polish]{babel}
% Need description
\usepackage[MeX]{polski}



% ------------------------------
% Better support of polish chars in technical parts of PDF
% ------------------------------
\hypersetup{pdfencoding=auto,psdextra}

% Package "textpos" give as enviroment "textblock" which is very usefull in
% arranging text on slides.

% This is standard configuration of "textpos"
\usepackage[overlay,absolute]{textpos}

% If you need to see bounds of "textblock's" comment line above and uncomment
% one below.

% Caution! When showboxes option is on significant ammunt of space is add
% to the top of textblock and as such, everyting put in them gone down.
% We need to check how to remove this bug.

% \usepackage[showboxes,overlay,absolute]{textpos}



% Setting scale length for package "textpos"
\setlength{\TPHorizModule}{10mm}
\setlength{\TPVertModule}{\TPHorizModule}


% ---------------------------------------
% TikZ
% ---------------------------------------
% Importing TikZ libraries
\usetikzlibrary{arrows.meta}
\usetikzlibrary{positioning}





% % Configuration package "bm" that need for making bold symbols
% \newcommand{\bmmax}{0}
% \newcommand{\hmmax}{0}
% \usepackage{bm}




% ---------------------------------------
% Packages for scientific texts
% ---------------------------------------
% \let\lll\undefined  % Sometimes you must use this line to allow
% "amsmath" package to works with packages with packages for polish
% languge imported
% /preambul/LanguageSettings/JagiellonianPolishLanguageSettings.tex.
% This comments (probably) removes polish letter Ł.
\usepackage{amsmath}  % Packages from American Mathematical Society (AMS)
\usepackage{amssymb}
\usepackage{amscd}
\usepackage{amsthm}
\usepackage{siunitx}  % Package for typsetting SI units.
\usepackage{upgreek}  % Better looking greek letters.
% Example of using upgreek: pi = \uppi


\usepackage{calrsfs}  % Zmienia czcionkę kaligraficzną w \mathcal
% na ładniejszą. Może w innych miejscach robi to samo, ale o tym nic
% nie wiem.










% ---------------------------------------
% Packages written for lectures "Geometria 3D dla twórców gier wideo"
% ---------------------------------------
% \usepackage{./ProgramowanieSymulacjiFizykiPaczki/ProgramowanieSymulacjiFizyki}
% \usepackage{./ProgramowanieSymulacjiFizykiPaczki/ProgramowanieSymulacjiFizykiIndeksy}
% \usepackage{./ProgramowanieSymulacjiFizykiPaczki/ProgramowanieSymulacjiFizykiTikZStyle}





% !!!!!!!!!!!!!!!!!!!!!!!!!!!!!!
% !!!!!!!!!!!!!!!!!!!!!!!!!!!!!!
% EVIL STUFF
\if\JUlogotitle1
\edef\LogoJUPath{LogoJU_\JUlogoLang/LogoJU_\JUlogoShape_\JUlogoColor.pdf}
\titlegraphic{\hfill\includegraphics[scale=0.22]
{./JagiellonianPictures/\LogoJUPath}}
\fi
% ---------------------------------------
% Commands for handling colors
% ---------------------------------------


% Command for setting normal text color for some text in math modestyle
% Text color depend on used style of Jagiellonian

% Beamer version of command
\newcommand{\TextWithNormalTextColor}[1]{%
  {\color{jNormalTextFGColor}
    \setbeamercolor{math text}{fg=jNormalTextFGColor} {#1}}
}

% Article and similar classes version of command
% \newcommand{\TextWithNormalTextColor}[1]{%
%   {\color{jNormalTextsFGColor} {#1}}
% }



% Beamer version of command
\newcommand{\NormalTextInMathMode}[1]{%
  {\color{jNormalTextFGColor}
    \setbeamercolor{math text}{fg=jNormalTextFGColor} \text{#1}}
}


% Article and similar classes version of command
% \newcommand{\NormalTextInMathMode}[1]{%
%   {\color{jNormalTextsFGColor} \text{#1}}
% }




% Command that sets color of some mathematical text to the same color
% that has normal text in header (?)

% Beamer version of the command
\newcommand{\MathTextFrametitleFGColor}[1]{%
  {\color{jFrametitleFGColor}
    \setbeamercolor{math text}{fg=jFrametitleFGColor} #1}
}

% Article and similar classes version of the command
% \newcommand{\MathTextWhiteColor}[1]{{\color{jFrametitleFGColor} #1}}





% Command for setting color of alert text for some text in math modestyle

% Beamer version of the command
\newcommand{\MathTextAlertColor}[1]{%
  {\color{jOrange} \setbeamercolor{math text}{fg=jOrange} #1}
}

% Article and similar classes version of the command
% \newcommand{\MathTextAlertColor}[1]{{\color{jOrange} #1}}





% Command that allow you to sets chosen color as the color of some text into
% math mode. Due to some nuances in the way that Beamer handle colors
% it not work in all cases. We hope that in the future we will improve it.

% Beamer version of the command
\newcommand{\SetMathTextColor}[2]{%
  {\color{#1} \setbeamercolor{math text}{fg=#1} #2}
}


% Article and similar classes version of the command
% \newcommand{\SetMathTextColor}[2]{{\color{#1} #2}}










% ---------------------------------------
% Commands for few special slides
% ---------------------------------------
\newcommand{\EndingSlide}[1]{%
  \begin{frame}[standout]

    \begingroup

    \color{jFrametitleFGColor}

    #1

    \endgroup

  \end{frame}
}










% ---------------------------------------
% Commands for setting background pictures for some slides
% ---------------------------------------
\newcommand{\TitleBackgroundPicture}
{./JagiellonianPictures/Backgrounds/LajkonikDark.png}
\newcommand{\SectionBackgroundPicture}
{./JagiellonianPictures/Backgrounds/LajkonikLight.png}



\newcommand{\TitleSlideWithPicture}{%
  \begingroup

  \usebackgroundtemplate{%
    \includegraphics[height=\paperheight]{\TitleBackgroundPicture}}

  \maketitle

  \endgroup
}





\newcommand{\SectionSlideWithPicture}[1]{%
  \begingroup

  \usebackgroundtemplate{%
    \includegraphics[height=\paperheight]{\SectionBackgroundPicture}}

  \setbeamercolor{titlelike}{fg=normal text.fg}

  \section{#1}

  \endgroup
}










% ---------------------------------------
% Commands for lectures "Geometria 3D dla twórców gier wideo"
% Polish version
% ---------------------------------------
% Komendy teraz wykomentowane były potrzebne, gdy loga były na niebieskim
% tle, nie na białym. A są na białym bo tego chcieli w biurze projektu.
% \newcommand{\FundingLogoWhitePicturePL}
% {./PresentationPictures/CommonPictures/logotypFundusze_biale_bez_tla2.pdf}
\newcommand{\FundingLogoColorPicturePL}
{./PresentationPictures/CommonPictures/European_Funds_color_PL.pdf}
% \newcommand{\EULogoWhitePicturePL}
% {./PresentationPictures/CommonPictures/logotypUE_biale_bez_tla2.pdf}
\newcommand{\EUSocialFundLogoColorPicturePL}
{./PresentationPictures/CommonPictures/EU_Social_Fund_color_PL.pdf}
% \newcommand{\ZintegrUJLogoWhitePicturePL}
% {./PresentationPictures/CommonPictures/zintegruj-logo-white.pdf}
\newcommand{\ZintegrUJLogoColorPicturePL}
{./PresentationPictures/CommonPictures/ZintegrUJ_color.pdf}
\newcommand{\JULogoColorPicturePL}
{./JagiellonianPictures/LogoJU_PL/LogoJU_A_color.pdf}





\newcommand{\GeometryThreeDSpecialBeginningSlidePL}{%
  \begin{frame}[standout]

    \begin{textblock}{11}(1,0.7)

      \begin{flushleft}

        \mdseries

        \footnotesize

        \color{jFrametitleFGColor}

        Materiał powstał w ramach projektu współfinansowanego ze środków
        Unii Europejskiej w ramach Europejskiego Funduszu Społecznego
        POWR.03.05.00-00-Z309/17-00.

      \end{flushleft}

    \end{textblock}





    \begin{textblock}{10}(0,2.2)

      \tikz \fill[color=jBackgroundStyleLight] (0,0) rectangle (12.8,-1.5);

    \end{textblock}


    \begin{textblock}{3.2}(1,2.45)

      \includegraphics[scale=0.3]{\FundingLogoColorPicturePL}

    \end{textblock}


    \begin{textblock}{2.5}(3.7,2.5)

      \includegraphics[scale=0.2]{\JULogoColorPicturePL}

    \end{textblock}


    \begin{textblock}{2.5}(6,2.4)

      \includegraphics[scale=0.1]{\ZintegrUJLogoColorPicturePL}

    \end{textblock}


    \begin{textblock}{4.2}(8.4,2.6)

      \includegraphics[scale=0.3]{\EUSocialFundLogoColorPicturePL}

    \end{textblock}

  \end{frame}
}



\newcommand{\GeometryThreeDTwoSpecialBeginningSlidesPL}{%
  \begin{frame}[standout]

    \begin{textblock}{11}(1,0.7)

      \begin{flushleft}

        \mdseries

        \footnotesize

        \color{jFrametitleFGColor}

        Materiał powstał w ramach projektu współfinansowanego ze środków
        Unii Europejskiej w ramach Europejskiego Funduszu Społecznego
        POWR.03.05.00-00-Z309/17-00.

      \end{flushleft}

    \end{textblock}





    \begin{textblock}{10}(0,2.2)

      \tikz \fill[color=jBackgroundStyleLight] (0,0) rectangle (12.8,-1.5);

    \end{textblock}


    \begin{textblock}{3.2}(1,2.45)

      \includegraphics[scale=0.3]{\FundingLogoColorPicturePL}

    \end{textblock}


    \begin{textblock}{2.5}(3.7,2.5)

      \includegraphics[scale=0.2]{\JULogoColorPicturePL}

    \end{textblock}


    \begin{textblock}{2.5}(6,2.4)

      \includegraphics[scale=0.1]{\ZintegrUJLogoColorPicturePL}

    \end{textblock}


    \begin{textblock}{4.2}(8.4,2.6)

      \includegraphics[scale=0.3]{\EUSocialFundLogoColorPicturePL}

    \end{textblock}

  \end{frame}





  \TitleSlideWithPicture
}



\newcommand{\GeometryThreeDSpecialEndingSlidePL}{%
  \begin{frame}[standout]

    \begin{textblock}{11}(1,0.7)

      \begin{flushleft}

        \mdseries

        \footnotesize

        \color{jFrametitleFGColor}

        Materiał powstał w ramach projektu współfinansowanego ze środków
        Unii Europejskiej w~ramach Europejskiego Funduszu Społecznego
        POWR.03.05.00-00-Z309/17-00.

      \end{flushleft}

    \end{textblock}





    \begin{textblock}{10}(0,2.2)

      \tikz \fill[color=jBackgroundStyleLight] (0,0) rectangle (12.8,-1.5);

    \end{textblock}


    \begin{textblock}{3.2}(1,2.45)

      \includegraphics[scale=0.3]{\FundingLogoColorPicturePL}

    \end{textblock}


    \begin{textblock}{2.5}(3.7,2.5)

      \includegraphics[scale=0.2]{\JULogoColorPicturePL}

    \end{textblock}


    \begin{textblock}{2.5}(6,2.4)

      \includegraphics[scale=0.1]{\ZintegrUJLogoColorPicturePL}

    \end{textblock}


    \begin{textblock}{4.2}(8.4,2.6)

      \includegraphics[scale=0.3]{\EUSocialFundLogoColorPicturePL}

    \end{textblock}





    \begin{textblock}{11}(1,4)

      \begin{flushleft}

        \mdseries

        \footnotesize

        \RaggedRight

        \color{jFrametitleFGColor}

        Treść niniejszego wykładu jest udostępniona na~licencji
        Creative Commons (\textsc{cc}), z~uzna\-niem autorstwa
        (\textsc{by}) oraz udostępnianiem na tych samych warunkach
        (\textsc{sa}). Rysunki i~wy\-kresy zawarte w~wykładzie są
        autorstwa dr.~hab.~Pawła Węgrzyna et~al. i~są dostępne
        na tej samej licencji, o~ile nie wskazano inaczej.
        W~prezentacji wykorzystano temat Beamera Jagiellonian,
        oparty na~temacie Metropolis Matthiasa Vogelgesanga,
        dostępnym na licencji \LaTeX{} Project Public License~1.3c
        pod adresem: \colorhref{https://github.com/matze/mtheme}
        {https://github.com/matze/mtheme}.

        Projekt typograficzny: Iwona Grabska-Gradzińska \\
        Skład: Kamil Ziemian;
        Korekta: Wojciech Palacz \\
        Modele: Dariusz Frymus, Kamil Nowakowski \\
        Rysunki i~wykresy: Kamil Ziemian, Paweł Węgrzyn, Wojciech Palacz

      \end{flushleft}

    \end{textblock}

  \end{frame}
}



\newcommand{\GeometryThreeDTwoSpecialEndingSlidesPL}[1]{%
  \begin{frame}[standout]


    \begin{textblock}{11}(1,0.7)

      \begin{flushleft}

        \mdseries

        \footnotesize

        \color{jFrametitleFGColor}

        Materiał powstał w ramach projektu współfinansowanego ze środków
        Unii Europejskiej w~ramach Europejskiego Funduszu Społecznego
        POWR.03.05.00-00-Z309/17-00.

      \end{flushleft}

    \end{textblock}





    \begin{textblock}{10}(0,2.2)

      \tikz \fill[color=jBackgroundStyleLight] (0,0) rectangle (12.8,-1.5);

    \end{textblock}


    \begin{textblock}{3.2}(1,2.45)

      \includegraphics[scale=0.3]{\FundingLogoColorPicturePL}

    \end{textblock}


    \begin{textblock}{2.5}(3.7,2.5)

      \includegraphics[scale=0.2]{\JULogoColorPicturePL}

    \end{textblock}


    \begin{textblock}{2.5}(6,2.4)

      \includegraphics[scale=0.1]{\ZintegrUJLogoColorPicturePL}

    \end{textblock}


    \begin{textblock}{4.2}(8.4,2.6)

      \includegraphics[scale=0.3]{\EUSocialFundLogoColorPicturePL}

    \end{textblock}





    \begin{textblock}{11}(1,4)

      \begin{flushleft}

        \mdseries

        \footnotesize

        \RaggedRight

        \color{jFrametitleFGColor}

        Treść niniejszego wykładu jest udostępniona na~licencji
        Creative Commons (\textsc{cc}), z~uzna\-niem autorstwa
        (\textsc{by}) oraz udostępnianiem na tych samych warunkach
        (\textsc{sa}). Rysunki i~wy\-kresy zawarte w~wykładzie są
        autorstwa dr.~hab.~Pawła Węgrzyna et~al. i~są dostępne
        na tej samej licencji, o~ile nie wskazano inaczej.
        W~prezentacji wykorzystano temat Beamera Jagiellonian,
        oparty na~temacie Metropolis Matthiasa Vogelgesanga,
        dostępnym na licencji \LaTeX{} Project Public License~1.3c
        pod adresem: \colorhref{https://github.com/matze/mtheme}
        {https://github.com/matze/mtheme}.

        Projekt typograficzny: Iwona Grabska-Gradzińska \\
        Skład: Kamil Ziemian;
        Korekta: Wojciech Palacz \\
        Modele: Dariusz Frymus, Kamil Nowakowski \\
        Rysunki i~wykresy: Kamil Ziemian, Paweł Węgrzyn, Wojciech Palacz

      \end{flushleft}

    \end{textblock}

  \end{frame}





  \begin{frame}[standout]

    \begingroup

    \color{jFrametitleFGColor}

    #1

    \endgroup

  \end{frame}
}



\newcommand{\GeometryThreeDSpecialEndingSlideVideoPL}{%
  \begin{frame}[standout]

    \begin{textblock}{11}(1,0.7)

      \begin{flushleft}

        \mdseries

        \footnotesize

        \color{jFrametitleFGColor}

        Materiał powstał w ramach projektu współfinansowanego ze środków
        Unii Europejskiej w~ramach Europejskiego Funduszu Społecznego
        POWR.03.05.00-00-Z309/17-00.

      \end{flushleft}

    \end{textblock}





    \begin{textblock}{10}(0,2.2)

      \tikz \fill[color=jBackgroundStyleLight] (0,0) rectangle (12.8,-1.5);

    \end{textblock}


    \begin{textblock}{3.2}(1,2.45)

      \includegraphics[scale=0.3]{\FundingLogoColorPicturePL}

    \end{textblock}


    \begin{textblock}{2.5}(3.7,2.5)

      \includegraphics[scale=0.2]{\JULogoColorPicturePL}

    \end{textblock}


    \begin{textblock}{2.5}(6,2.4)

      \includegraphics[scale=0.1]{\ZintegrUJLogoColorPicturePL}

    \end{textblock}


    \begin{textblock}{4.2}(8.4,2.6)

      \includegraphics[scale=0.3]{\EUSocialFundLogoColorPicturePL}

    \end{textblock}





    \begin{textblock}{11}(1,4)

      \begin{flushleft}

        \mdseries

        \footnotesize

        \RaggedRight

        \color{jFrametitleFGColor}

        Treść niniejszego wykładu jest udostępniona na~licencji
        Creative Commons (\textsc{cc}), z~uzna\-niem autorstwa
        (\textsc{by}) oraz udostępnianiem na tych samych warunkach
        (\textsc{sa}). Rysunki i~wy\-kresy zawarte w~wykładzie są
        autorstwa dr.~hab.~Pawła Węgrzyna et~al. i~są dostępne
        na tej samej licencji, o~ile nie wskazano inaczej.
        W~prezentacji wykorzystano temat Beamera Jagiellonian,
        oparty na~temacie Metropolis Matthiasa Vogelgesanga,
        dostępnym na licencji \LaTeX{} Project Public License~1.3c
        pod adresem: \colorhref{https://github.com/matze/mtheme}
        {https://github.com/matze/mtheme}.

        Projekt typograficzny: Iwona Grabska-Gradzińska;
        Skład: Kamil Ziemian \\
        Korekta: Wojciech Palacz;
        Modele: Dariusz Frymus, Kamil Nowakowski \\
        Rysunki i~wykresy: Kamil Ziemian, Paweł Węgrzyn, Wojciech Palacz \\
        Montaż: Agencja Filmowa Film \& Television Production~-- Zbigniew
        Masklak

      \end{flushleft}

    \end{textblock}

  \end{frame}
}





\newcommand{\GeometryThreeDTwoSpecialEndingSlidesVideoPL}[1]{%
  \begin{frame}[standout]

    \begin{textblock}{11}(1,0.7)

      \begin{flushleft}

        \mdseries

        \footnotesize

        \color{jFrametitleFGColor}

        Materiał powstał w ramach projektu współfinansowanego ze środków
        Unii Europejskiej w~ramach Europejskiego Funduszu Społecznego
        POWR.03.05.00-00-Z309/17-00.

      \end{flushleft}

    \end{textblock}





    \begin{textblock}{10}(0,2.2)

      \tikz \fill[color=jBackgroundStyleLight] (0,0) rectangle (12.8,-1.5);

    \end{textblock}


    \begin{textblock}{3.2}(1,2.45)

      \includegraphics[scale=0.3]{\FundingLogoColorPicturePL}

    \end{textblock}


    \begin{textblock}{2.5}(3.7,2.5)

      \includegraphics[scale=0.2]{\JULogoColorPicturePL}

    \end{textblock}


    \begin{textblock}{2.5}(6,2.4)

      \includegraphics[scale=0.1]{\ZintegrUJLogoColorPicturePL}

    \end{textblock}


    \begin{textblock}{4.2}(8.4,2.6)

      \includegraphics[scale=0.3]{\EUSocialFundLogoColorPicturePL}

    \end{textblock}





    \begin{textblock}{11}(1,4)

      \begin{flushleft}

        \mdseries

        \footnotesize

        \RaggedRight

        \color{jFrametitleFGColor}

        Treść niniejszego wykładu jest udostępniona na~licencji
        Creative Commons (\textsc{cc}), z~uzna\-niem autorstwa
        (\textsc{by}) oraz udostępnianiem na tych samych warunkach
        (\textsc{sa}). Rysunki i~wy\-kresy zawarte w~wykładzie są
        autorstwa dr.~hab.~Pawła Węgrzyna et~al. i~są dostępne
        na tej samej licencji, o~ile nie wskazano inaczej.
        W~prezentacji wykorzystano temat Beamera Jagiellonian,
        oparty na~temacie Metropolis Matthiasa Vogelgesanga,
        dostępnym na licencji \LaTeX{} Project Public License~1.3c
        pod adresem: \colorhref{https://github.com/matze/mtheme}
        {https://github.com/matze/mtheme}.

        Projekt typograficzny: Iwona Grabska-Gradzińska;
        Skład: Kamil Ziemian \\
        Korekta: Wojciech Palacz;
        Modele: Dariusz Frymus, Kamil Nowakowski \\
        Rysunki i~wykresy: Kamil Ziemian, Paweł Węgrzyn, Wojciech Palacz \\
        Montaż: Agencja Filmowa Film \& Television Production~-- Zbigniew
        Masklak

      \end{flushleft}

    \end{textblock}

  \end{frame}





  \begin{frame}[standout]


    \begingroup

    \color{jFrametitleFGColor}

    #1

    \endgroup

  \end{frame}
}










% ---------------------------------------
% Commands for lectures "Geometria 3D dla twórców gier wideo"
% English version
% ---------------------------------------
% \newcommand{\FundingLogoWhitePictureEN}
% {./PresentationPictures/CommonPictures/logotypFundusze_biale_bez_tla2.pdf}
\newcommand{\FundingLogoColorPictureEN}
{./PresentationPictures/CommonPictures/European_Funds_color_EN.pdf}
% \newcommand{\EULogoWhitePictureEN}
% {./PresentationPictures/CommonPictures/logotypUE_biale_bez_tla2.pdf}
\newcommand{\EUSocialFundLogoColorPictureEN}
{./PresentationPictures/CommonPictures/EU_Social_Fund_color_EN.pdf}
% \newcommand{\ZintegrUJLogoWhitePictureEN}
% {./PresentationPictures/CommonPictures/zintegruj-logo-white.pdf}
\newcommand{\ZintegrUJLogoColorPictureEN}
{./PresentationPictures/CommonPictures/ZintegrUJ_color.pdf}
\newcommand{\JULogoColorPictureEN}
{./JagiellonianPictures/LogoJU_EN/LogoJU_A_color.pdf}



\newcommand{\GeometryThreeDSpecialBeginningSlideEN}{%
  \begin{frame}[standout]

    \begin{textblock}{11}(1,0.7)

      \begin{flushleft}

        \mdseries

        \footnotesize

        \color{jFrametitleFGColor}

        This content was created as part of a project co-financed by the
        European Union within the framework of the European Social Fund
        POWR.03.05.00-00-Z309/17-00.

      \end{flushleft}

    \end{textblock}





    \begin{textblock}{10}(0,2.2)

      \tikz \fill[color=jBackgroundStyleLight] (0,0) rectangle (12.8,-1.5);

    \end{textblock}


    \begin{textblock}{3.2}(0.7,2.45)

      \includegraphics[scale=0.3]{\FundingLogoColorPictureEN}

    \end{textblock}


    \begin{textblock}{2.5}(4.15,2.5)

      \includegraphics[scale=0.2]{\JULogoColorPictureEN}

    \end{textblock}


    \begin{textblock}{2.5}(6.35,2.4)

      \includegraphics[scale=0.1]{\ZintegrUJLogoColorPictureEN}

    \end{textblock}


    \begin{textblock}{4.2}(8.4,2.6)

      \includegraphics[scale=0.3]{\EUSocialFundLogoColorPictureEN}

    \end{textblock}

  \end{frame}
}



\newcommand{\GeometryThreeDTwoSpecialBeginningSlidesEN}{%
  \begin{frame}[standout]

    \begin{textblock}{11}(1,0.7)

      \begin{flushleft}

        \mdseries

        \footnotesize

        \color{jFrametitleFGColor}

        This content was created as part of a project co-financed by the
        European Union within the framework of the European Social Fund
        POWR.03.05.00-00-Z309/17-00.

      \end{flushleft}

    \end{textblock}





    \begin{textblock}{10}(0,2.2)

      \tikz \fill[color=jBackgroundStyleLight] (0,0) rectangle (12.8,-1.5);

    \end{textblock}


    \begin{textblock}{3.2}(0.7,2.45)

      \includegraphics[scale=0.3]{\FundingLogoColorPictureEN}

    \end{textblock}


    \begin{textblock}{2.5}(4.15,2.5)

      \includegraphics[scale=0.2]{\JULogoColorPictureEN}

    \end{textblock}


    \begin{textblock}{2.5}(6.35,2.4)

      \includegraphics[scale=0.1]{\ZintegrUJLogoColorPictureEN}

    \end{textblock}


    \begin{textblock}{4.2}(8.4,2.6)

      \includegraphics[scale=0.3]{\EUSocialFundLogoColorPictureEN}

    \end{textblock}

  \end{frame}





  \TitleSlideWithPicture
}



\newcommand{\GeometryThreeDSpecialEndingSlideEN}{%
  \begin{frame}[standout]

    \begin{textblock}{11}(1,0.7)

      \begin{flushleft}

        \mdseries

        \footnotesize

        \color{jFrametitleFGColor}

        This content was created as part of a project co-financed by the
        European Union within the framework of the European Social Fund
        POWR.03.05.00-00-Z309/17-00.

      \end{flushleft}

    \end{textblock}





    \begin{textblock}{10}(0,2.2)

      \tikz \fill[color=jBackgroundStyleLight] (0,0) rectangle (12.8,-1.5);

    \end{textblock}


    \begin{textblock}{3.2}(0.7,2.45)

      \includegraphics[scale=0.3]{\FundingLogoColorPictureEN}

    \end{textblock}


    \begin{textblock}{2.5}(4.15,2.5)

      \includegraphics[scale=0.2]{\JULogoColorPictureEN}

    \end{textblock}


    \begin{textblock}{2.5}(6.35,2.4)

      \includegraphics[scale=0.1]{\ZintegrUJLogoColorPictureEN}

    \end{textblock}


    \begin{textblock}{4.2}(8.4,2.6)

      \includegraphics[scale=0.3]{\EUSocialFundLogoColorPictureEN}

    \end{textblock}





    \begin{textblock}{11}(1,4)

      \begin{flushleft}

        \mdseries

        \footnotesize

        \RaggedRight

        \color{jFrametitleFGColor}

        The content of this lecture is made available under a~Creative
        Commons licence (\textsc{cc}), giving the author the credits
        (\textsc{by}) and putting an obligation to share on the same terms
        (\textsc{sa}). Figures and diagrams included in the lecture are
        authored by Paweł Węgrzyn et~al., and are available under the same
        license unless indicated otherwise.\\ The presentation uses the
        Beamer Jagiellonian theme based on Matthias Vogelgesang’s
        Metropolis theme, available under license \LaTeX{} Project
        Public License~1.3c at: \colorhref{https://github.com/matze/mtheme}
        {https://github.com/matze/mtheme}.

        Typographic design: Iwona Grabska-Gradzińska \\
        \LaTeX{} Typesetting: Kamil Ziemian \\
        Proofreading: Wojciech Palacz,
        Monika Stawicka \\
        3D Models: Dariusz Frymus, Kamil Nowakowski \\
        Figures and charts: Kamil Ziemian, Paweł Węgrzyn, Wojciech Palacz

      \end{flushleft}

    \end{textblock}

  \end{frame}
}



\newcommand{\GeometryThreeDTwoSpecialEndingSlidesEN}[1]{%
  \begin{frame}[standout]


    \begin{textblock}{11}(1,0.7)

      \begin{flushleft}

        \mdseries

        \footnotesize

        \color{jFrametitleFGColor}

        This content was created as part of a project co-financed by the
        European Union within the framework of the European Social Fund
        POWR.03.05.00-00-Z309/17-00.

      \end{flushleft}

    \end{textblock}





    \begin{textblock}{10}(0,2.2)

      \tikz \fill[color=jBackgroundStyleLight] (0,0) rectangle (12.8,-1.5);

    \end{textblock}


    \begin{textblock}{3.2}(0.7,2.45)

      \includegraphics[scale=0.3]{\FundingLogoColorPictureEN}

    \end{textblock}


    \begin{textblock}{2.5}(4.15,2.5)

      \includegraphics[scale=0.2]{\JULogoColorPictureEN}

    \end{textblock}


    \begin{textblock}{2.5}(6.35,2.4)

      \includegraphics[scale=0.1]{\ZintegrUJLogoColorPictureEN}

    \end{textblock}


    \begin{textblock}{4.2}(8.4,2.6)

      \includegraphics[scale=0.3]{\EUSocialFundLogoColorPictureEN}

    \end{textblock}





    \begin{textblock}{11}(1,4)

      \begin{flushleft}

        \mdseries

        \footnotesize

        \RaggedRight

        \color{jFrametitleFGColor}

        The content of this lecture is made available under a~Creative
        Commons licence (\textsc{cc}), giving the author the credits
        (\textsc{by}) and putting an obligation to share on the same terms
        (\textsc{sa}). Figures and diagrams included in the lecture are
        authored by Paweł Węgrzyn et~al., and are available under the same
        license unless indicated otherwise.\\ The presentation uses the
        Beamer Jagiellonian theme based on Matthias Vogelgesang’s
        Metropolis theme, available under license \LaTeX{} Project
        Public License~1.3c at: \colorhref{https://github.com/matze/mtheme}
        {https://github.com/matze/mtheme}.

        Typographic design: Iwona Grabska-Gradzińska \\
        \LaTeX{} Typesetting: Kamil Ziemian \\
        Proofreading: Wojciech Palacz,
        Monika Stawicka \\
        3D Models: Dariusz Frymus, Kamil Nowakowski \\
        Figures and charts: Kamil Ziemian, Paweł Węgrzyn, Wojciech Palacz

      \end{flushleft}

    \end{textblock}

  \end{frame}





  \begin{frame}[standout]

    \begingroup

    \color{jFrametitleFGColor}

    #1

    \endgroup

  \end{frame}
}



\newcommand{\GeometryThreeDSpecialEndingSlideVideoVerOneEN}{%
  \begin{frame}[standout]

    \begin{textblock}{11}(1,0.7)

      \begin{flushleft}

        \mdseries

        \footnotesize

        \color{jFrametitleFGColor}

        This content was created as part of a project co-financed by the
        European Union within the framework of the European Social Fund
        POWR.03.05.00-00-Z309/17-00.

      \end{flushleft}

    \end{textblock}





    \begin{textblock}{10}(0,2.2)

      \tikz \fill[color=jBackgroundStyleLight] (0,0) rectangle (12.8,-1.5);

    \end{textblock}


    \begin{textblock}{3.2}(0.7,2.45)

      \includegraphics[scale=0.3]{\FundingLogoColorPictureEN}

    \end{textblock}


    \begin{textblock}{2.5}(4.15,2.5)

      \includegraphics[scale=0.2]{\JULogoColorPictureEN}

    \end{textblock}


    \begin{textblock}{2.5}(6.35,2.4)

      \includegraphics[scale=0.1]{\ZintegrUJLogoColorPictureEN}

    \end{textblock}


    \begin{textblock}{4.2}(8.4,2.6)

      \includegraphics[scale=0.3]{\EUSocialFundLogoColorPictureEN}

    \end{textblock}





    \begin{textblock}{11}(1,4)

      \begin{flushleft}

        \mdseries

        \footnotesize

        \RaggedRight

        \color{jFrametitleFGColor}

        The content of this lecture is made available under a Creative
        Commons licence (\textsc{cc}), giving the author the credits
        (\textsc{by}) and putting an obligation to share on the same terms
        (\textsc{sa}). Figures and diagrams included in the lecture are
        authored by Paweł Węgrzyn et~al., and are available under the same
        license unless indicated otherwise.\\ The presentation uses the
        Beamer Jagiellonian theme based on Matthias Vogelgesang’s
        Metropolis theme, available under license \LaTeX{} Project
        Public License~1.3c at: \colorhref{https://github.com/matze/mtheme}
        {https://github.com/matze/mtheme}.

        Typographic design: Iwona Grabska-Gradzińska;
        \LaTeX{} Typesetting: Kamil Ziemian \\
        Proofreading: Wojciech Palacz,
        Monika Stawicka \\
        3D Models: Dariusz Frymus, Kamil Nowakowski \\
        Figures and charts: Kamil Ziemian, Paweł Węgrzyn, Wojciech
        Palacz \\
        Film editing: Agencja Filmowa Film \& Television Production~--
        Zbigniew Masklak

      \end{flushleft}

    \end{textblock}

  \end{frame}
}



\newcommand{\GeometryThreeDSpecialEndingSlideVideoVerTwoEN}{%
  \begin{frame}[standout]

    \begin{textblock}{11}(1,0.7)

      \begin{flushleft}

        \mdseries

        \footnotesize

        \color{jFrametitleFGColor}

        This content was created as part of a project co-financed by the
        European Union within the framework of the European Social Fund
        POWR.03.05.00-00-Z309/17-00.

      \end{flushleft}

    \end{textblock}





    \begin{textblock}{10}(0,2.2)

      \tikz \fill[color=jBackgroundStyleLight] (0,0) rectangle (12.8,-1.5);

    \end{textblock}


    \begin{textblock}{3.2}(0.7,2.45)

      \includegraphics[scale=0.3]{\FundingLogoColorPictureEN}

    \end{textblock}


    \begin{textblock}{2.5}(4.15,2.5)

      \includegraphics[scale=0.2]{\JULogoColorPictureEN}

    \end{textblock}


    \begin{textblock}{2.5}(6.35,2.4)

      \includegraphics[scale=0.1]{\ZintegrUJLogoColorPictureEN}

    \end{textblock}


    \begin{textblock}{4.2}(8.4,2.6)

      \includegraphics[scale=0.3]{\EUSocialFundLogoColorPictureEN}

    \end{textblock}





    \begin{textblock}{11}(1,4)

      \begin{flushleft}

        \mdseries

        \footnotesize

        \RaggedRight

        \color{jFrametitleFGColor}

        The content of this lecture is made available under a Creative
        Commons licence (\textsc{cc}), giving the author the credits
        (\textsc{by}) and putting an obligation to share on the same terms
        (\textsc{sa}). Figures and diagrams included in the lecture are
        authored by Paweł Węgrzyn et~al., and are available under the same
        license unless indicated otherwise.\\ The presentation uses the
        Beamer Jagiellonian theme based on Matthias Vogelgesang’s
        Metropolis theme, available under license \LaTeX{} Project
        Public License~1.3c at: \colorhref{https://github.com/matze/mtheme}
        {https://github.com/matze/mtheme}.

        Typographic design: Iwona Grabska-Gradzińska;
        \LaTeX{} Typesetting: Kamil Ziemian \\
        Proofreading: Wojciech Palacz,
        Monika Stawicka \\
        3D Models: Dariusz Frymus, Kamil Nowakowski \\
        Figures and charts: Kamil Ziemian, Paweł Węgrzyn, Wojciech
        Palacz \\
        Film editing: IMAVI -- Joanna Kozakiewicz, Krzysztof Magda, Nikodem
        Frodyma

      \end{flushleft}

    \end{textblock}

  \end{frame}
}



\newcommand{\GeometryThreeDSpecialEndingSlideVideoVerThreeEN}{%
  \begin{frame}[standout]

    \begin{textblock}{11}(1,0.7)

      \begin{flushleft}

        \mdseries

        \footnotesize

        \color{jFrametitleFGColor}

        This content was created as part of a project co-financed by the
        European Union within the framework of the European Social Fund
        POWR.03.05.00-00-Z309/17-00.

      \end{flushleft}

    \end{textblock}





    \begin{textblock}{10}(0,2.2)

      \tikz \fill[color=jBackgroundStyleLight] (0,0) rectangle (12.8,-1.5);

    \end{textblock}


    \begin{textblock}{3.2}(0.7,2.45)

      \includegraphics[scale=0.3]{\FundingLogoColorPictureEN}

    \end{textblock}


    \begin{textblock}{2.5}(4.15,2.5)

      \includegraphics[scale=0.2]{\JULogoColorPictureEN}

    \end{textblock}


    \begin{textblock}{2.5}(6.35,2.4)

      \includegraphics[scale=0.1]{\ZintegrUJLogoColorPictureEN}

    \end{textblock}


    \begin{textblock}{4.2}(8.4,2.6)

      \includegraphics[scale=0.3]{\EUSocialFundLogoColorPictureEN}

    \end{textblock}





    \begin{textblock}{11}(1,4)

      \begin{flushleft}

        \mdseries

        \footnotesize

        \RaggedRight

        \color{jFrametitleFGColor}

        The content of this lecture is made available under a Creative
        Commons licence (\textsc{cc}), giving the author the credits
        (\textsc{by}) and putting an obligation to share on the same terms
        (\textsc{sa}). Figures and diagrams included in the lecture are
        authored by Paweł Węgrzyn et~al., and are available under the same
        license unless indicated otherwise.\\ The presentation uses the
        Beamer Jagiellonian theme based on Matthias Vogelgesang’s
        Metropolis theme, available under license \LaTeX{} Project
        Public License~1.3c at: \colorhref{https://github.com/matze/mtheme}
        {https://github.com/matze/mtheme}.

        Typographic design: Iwona Grabska-Gradzińska;
        \LaTeX{} Typesetting: Kamil Ziemian \\
        Proofreading: Wojciech Palacz,
        Monika Stawicka \\
        3D Models: Dariusz Frymus, Kamil Nowakowski \\
        Figures and charts: Kamil Ziemian, Paweł Węgrzyn, Wojciech
        Palacz \\
        Film editing: Agencja Filmowa Film \& Television Production~--
        Zbigniew Masklak \\
        Film editing: IMAVI -- Joanna Kozakiewicz, Krzysztof Magda, Nikodem
        Frodyma

      \end{flushleft}

    \end{textblock}

  \end{frame}
}



\newcommand{\GeometryThreeDTwoSpecialEndingSlidesVideoVerOneEN}[1]{%
  \begin{frame}[standout]

    \begin{textblock}{11}(1,0.7)

      \begin{flushleft}

        \mdseries

        \footnotesize

        \color{jFrametitleFGColor}

        This content was created as part of a project co-financed by the
        European Union within the framework of the European Social Fund
        POWR.03.05.00-00-Z309/17-00.

      \end{flushleft}

    \end{textblock}





    \begin{textblock}{10}(0,2.2)

      \tikz \fill[color=jBackgroundStyleLight] (0,0) rectangle (12.8,-1.5);

    \end{textblock}


    \begin{textblock}{3.2}(0.7,2.45)

      \includegraphics[scale=0.3]{\FundingLogoColorPictureEN}

    \end{textblock}


    \begin{textblock}{2.5}(4.15,2.5)

      \includegraphics[scale=0.2]{\JULogoColorPictureEN}

    \end{textblock}


    \begin{textblock}{2.5}(6.35,2.4)

      \includegraphics[scale=0.1]{\ZintegrUJLogoColorPictureEN}

    \end{textblock}


    \begin{textblock}{4.2}(8.4,2.6)

      \includegraphics[scale=0.3]{\EUSocialFundLogoColorPictureEN}

    \end{textblock}





    \begin{textblock}{11}(1,4)

      \begin{flushleft}

        \mdseries

        \footnotesize

        \RaggedRight

        \color{jFrametitleFGColor}

        The content of this lecture is made available under a Creative
        Commons licence (\textsc{cc}), giving the author the credits
        (\textsc{by}) and putting an obligation to share on the same terms
        (\textsc{sa}). Figures and diagrams included in the lecture are
        authored by Paweł Węgrzyn et~al., and are available under the same
        license unless indicated otherwise.\\ The presentation uses the
        Beamer Jagiellonian theme based on Matthias Vogelgesang’s
        Metropolis theme, available under license \LaTeX{} Project
        Public License~1.3c at: \colorhref{https://github.com/matze/mtheme}
        {https://github.com/matze/mtheme}.

        Typographic design: Iwona Grabska-Gradzińska;
        \LaTeX{} Typesetting: Kamil Ziemian \\
        Proofreading: Wojciech Palacz,
        Monika Stawicka \\
        3D Models: Dariusz Frymus, Kamil Nowakowski \\
        Figures and charts: Kamil Ziemian, Paweł Węgrzyn,
        Wojciech Palacz \\
        Film editing: Agencja Filmowa Film \& Television Production~--
        Zbigniew Masklak

      \end{flushleft}

    \end{textblock}

  \end{frame}





  \begin{frame}[standout]


    \begingroup

    \color{jFrametitleFGColor}

    #1

    \endgroup

  \end{frame}
}



\newcommand{\GeometryThreeDTwoSpecialEndingSlidesVideoVerTwoEN}[1]{%
  \begin{frame}[standout]

    \begin{textblock}{11}(1,0.7)

      \begin{flushleft}

        \mdseries

        \footnotesize

        \color{jFrametitleFGColor}

        This content was created as part of a project co-financed by the
        European Union within the framework of the European Social Fund
        POWR.03.05.00-00-Z309/17-00.

      \end{flushleft}

    \end{textblock}





    \begin{textblock}{10}(0,2.2)

      \tikz \fill[color=jBackgroundStyleLight] (0,0) rectangle (12.8,-1.5);

    \end{textblock}


    \begin{textblock}{3.2}(0.7,2.45)

      \includegraphics[scale=0.3]{\FundingLogoColorPictureEN}

    \end{textblock}


    \begin{textblock}{2.5}(4.15,2.5)

      \includegraphics[scale=0.2]{\JULogoColorPictureEN}

    \end{textblock}


    \begin{textblock}{2.5}(6.35,2.4)

      \includegraphics[scale=0.1]{\ZintegrUJLogoColorPictureEN}

    \end{textblock}


    \begin{textblock}{4.2}(8.4,2.6)

      \includegraphics[scale=0.3]{\EUSocialFundLogoColorPictureEN}

    \end{textblock}





    \begin{textblock}{11}(1,4)

      \begin{flushleft}

        \mdseries

        \footnotesize

        \RaggedRight

        \color{jFrametitleFGColor}

        The content of this lecture is made available under a Creative
        Commons licence (\textsc{cc}), giving the author the credits
        (\textsc{by}) and putting an obligation to share on the same terms
        (\textsc{sa}). Figures and diagrams included in the lecture are
        authored by Paweł Węgrzyn et~al., and are available under the same
        license unless indicated otherwise.\\ The presentation uses the
        Beamer Jagiellonian theme based on Matthias Vogelgesang’s
        Metropolis theme, available under license \LaTeX{} Project
        Public License~1.3c at: \colorhref{https://github.com/matze/mtheme}
        {https://github.com/matze/mtheme}.

        Typographic design: Iwona Grabska-Gradzińska;
        \LaTeX{} Typesetting: Kamil Ziemian \\
        Proofreading: Wojciech Palacz,
        Monika Stawicka \\
        3D Models: Dariusz Frymus, Kamil Nowakowski \\
        Figures and charts: Kamil Ziemian, Paweł Węgrzyn,
        Wojciech Palacz \\
        Film editing: IMAVI -- Joanna Kozakiewicz, Krzysztof Magda, Nikodem
        Frodyma

      \end{flushleft}

    \end{textblock}

  \end{frame}





  \begin{frame}[standout]


    \begingroup

    \color{jFrametitleFGColor}

    #1

    \endgroup

  \end{frame}
}



\newcommand{\GeometryThreeDTwoSpecialEndingSlidesVideoVerThreeEN}[1]{%
  \begin{frame}[standout]

    \begin{textblock}{11}(1,0.7)

      \begin{flushleft}

        \mdseries

        \footnotesize

        \color{jFrametitleFGColor}

        This content was created as part of a project co-financed by the
        European Union within the framework of the European Social Fund
        POWR.03.05.00-00-Z309/17-00.

      \end{flushleft}

    \end{textblock}





    \begin{textblock}{10}(0,2.2)

      \tikz \fill[color=jBackgroundStyleLight] (0,0) rectangle (12.8,-1.5);

    \end{textblock}


    \begin{textblock}{3.2}(0.7,2.45)

      \includegraphics[scale=0.3]{\FundingLogoColorPictureEN}

    \end{textblock}


    \begin{textblock}{2.5}(4.15,2.5)

      \includegraphics[scale=0.2]{\JULogoColorPictureEN}

    \end{textblock}


    \begin{textblock}{2.5}(6.35,2.4)

      \includegraphics[scale=0.1]{\ZintegrUJLogoColorPictureEN}

    \end{textblock}


    \begin{textblock}{4.2}(8.4,2.6)

      \includegraphics[scale=0.3]{\EUSocialFundLogoColorPictureEN}

    \end{textblock}





    \begin{textblock}{11}(1,4)

      \begin{flushleft}

        \mdseries

        \footnotesize

        \RaggedRight

        \color{jFrametitleFGColor}

        The content of this lecture is made available under a Creative
        Commons licence (\textsc{cc}), giving the author the credits
        (\textsc{by}) and putting an obligation to share on the same terms
        (\textsc{sa}). Figures and diagrams included in the lecture are
        authored by Paweł Węgrzyn et~al., and are available under the same
        license unless indicated otherwise. \\ The presentation uses the
        Beamer Jagiellonian theme based on Matthias Vogelgesang’s
        Metropolis theme, available under license \LaTeX{} Project
        Public License~1.3c at: \colorhref{https://github.com/matze/mtheme}
        {https://github.com/matze/mtheme}.

        Typographic design: Iwona Grabska-Gradzińska;
        \LaTeX{} Typesetting: Kamil Ziemian \\
        Proofreading: Leszek Hadasz, Wojciech Palacz,
        Monika Stawicka \\
        3D Models: Dariusz Frymus, Kamil Nowakowski \\
        Figures and charts: Kamil Ziemian, Paweł Węgrzyn,
        Wojciech Palacz \\
        Film editing: Agencja Filmowa Film \& Television Production~--
        Zbigniew Masklak \\
        Film editing: IMAVI -- Joanna Kozakiewicz, Krzysztof Magda, Nikodem
        Frodyma


      \end{flushleft}

    \end{textblock}

  \end{frame}





  \begin{frame}[standout]


    \begingroup

    \color{jFrametitleFGColor}

    #1

    \endgroup

  \end{frame}
}











% ---------------------------------------
% Packages, libraries and their configuration
% ---------------------------------------
\usepackage{mathcommands}










% ---------------------------------------------------------------------
\title[Promieniowanie Hawkinga]{Promieniowanie Hawkinga czarnych
  dziur~-- Krótki wstęp wprowadzenie do~zagadnienia}
\subtitle{Seminarium Astrofizyki}

\author{Kamil Ziemian \\
  \texttt{kziemianfvt@gmail.com} }


\institute{Zakład Teorii Pola, \\
  Uniwersytet Jagielloński w~Krakowie}

\date[9 stycznia 2019~r.]{9 stycznia 2019~r.}
% ---------------------------------------------------------------------





% ####################################################################
% Początek dokumentu
\begin{document}
% ####################################################################





% Wyrównanie do lewej z łamaniem wyrazów

\RaggedRight





% ######################################
\maketitle % Tytuł całego tekstu
% ######################################










% ##################
\begin{frame}
  \frametitle{Uwagi ogólne}


  Podstawowe źródło z~którego korzystałem. \\
  Większość przedstawionych tu~materiału pochodzi z~pracy Hawkinga
  \textit{Particle Creation by~Black Holes} z~1975 roku
  \cite{HawkingParticleCreationByBlackHole1975}. Praca ta jednak
  omawia dużo zagadnień o~których nie mamy czasu tutaj mówić, jak
  choćby problem dynamiki rozważanego układu.

  O~seminarium \\
  Zdecydowałem~się nie usuwać z~wystąpienia wszystkich moim
  osobistych preferencji, przekonań i~uwag, mając nadzieję,
  że~dzięki temu będzie ono bardziej interesujące.
  Za~przesadną stronniczość, której nie udało się usunąć,
  przepraszam wszystkich Państwa.

\end{frame}
% ##################










% ######################################
% \section{Wstęp kulturowy}
\section{Wstęp historyczny}
% ######################################



% ##################
\begin{frame}
  \frametitle{Wstęp kulturowy}


  \begin{figure}

    \centering

    \includegraphics[height=1.5in]
    {./PresentationPictures/Stephen-Hawking.jpg}


    \caption{Stephen W. Hawking (1942--2018 r.)}

  \end{figure}

  Znany głównie jako aktor telewizyjny.
  Wystąpił m.in.~w~\textit{Star Trek: Next Generation},
  \textit{Simpsons}, \textit{Futurama} i~\textit{Big Bang Theory}.

\end{frame}
% ##################





% ##################
\begin{frame}
  \frametitle{Wstęp kulturowy}

  O~jego osobie \\
  Obok Einsteina zapewne najbardziej ikoniczne uosobienie naukowca
  w~XX i~XXI wieku. Napisał co~najmniej jedną bardzo złą książkę
  (jak twierdzą poważni krytycy) \textit{A Brief History of Time}
  oraz~co najmniej jedną wspaniałą, stworzoną wraz z~Georgem
  Ellis’em: \textit{The Large Scale Structure~of Space-Time}.

  Ale\ldots \\
  Z~nieznanych powodów jego osiągnięcia naukowe, są~tak naprawdę
  bardzo słabo znane i~rozumiane wśród ogółu ludzi.

\end{frame}
% ##################





% ##################
\begin{frame}
  \frametitle{Zarys historyczny}


  \begin{itemize}
    \RaggedRight

  \item Jeśli twierdzimy (ktoś może być innego zdania), że~świat
    jest jednością i~dlatego musi istnieć jedna teoria fizyczna go
    opisująca, to należy znaleźć teorię która ujmuje jednocześnie
    efekty mechaniki kwantowej i~ogólnej teorii względności.
    Nazwijmy tę teorie kwantową grawitacją.

  \item Kwantowej grawitacji nie ma (długa historia).

  \item I~chyba jeszcze trochę nie będzie.

  \item Lata 60-te XX wieku to jeden ze~złotych okresów ogólnej
    teorii względności. Wtedy też znalezione przez Karl
    Schwarzschild w~1916 roku rozwiązanie równań Einsteina zostało
    zreinterpretowane jako przedstawiające nowy fenomen fizyczny:
    czarną dziurę. Nazwę podobno zaproponował John Archibald
    Wheeler, wybitna postać.

  \item W~pracy z~1973 roku, James M. Bardeen, Brandon Carter
    i~Stephen W.~Hawking proponują 4 prawa mechaniki czarnych dziur
    \cite{BardeenCarterHawkingFourLawsOfBHMechanics1973}.

  \end{itemize}

\end{frame}
% ##################





% ##################
\begin{frame}
  \frametitle{Wprowadzenie historyczne}


  Zarys historii
  \begin{itemize}
    \RaggedRight

  \item W tym samym roku Jacob Bekenstein sugeruje, że~te prawa~są
    w~istocie prawami termodynamiki czarnych dziur i~podaje
    m. in. wzory na~ich temperaturę i~entropię,
    \cite{BekensteinBlackHolesAndEntropy1973}.

  \end{itemize}

  To rodzi kilka pytań
  \begin{itemize}
    \RaggedRight

  \item Jeżeli mamy wzór na makroskopową entropię, po powinniśmy móc
    ją obliczyć z modelu mikroskopowego. Czyli zliczyć stany
    i~zlogarytmować otrzymają liczbę.

  \item Skoro ciało ma niezerową temperaturę powinno emitować
    wysyłać jakieś promieniowanie. Jednak klasycznie czarna dziura
    nie może promieniować.

  \end{itemize}

  Zapewne z~tych powodów Hawking w latach 70-tych, zaproponował
  by~przebadać układ: pole kwantowe +~niekwantowa ogólna teoria
  względności; licząc, że~efekty kwantowe pomogą wyjaśnić problem
  sprzeczności między zasadami OTW, a~zasadami termodynamiki.

\end{frame}
% ##################










% ######################################
\section[Podstawy teorii]{Podstawy teorii promieniowania Hawkinga}
% ######################################



% ##################
\begin{frame}
  \frametitle{Podstawy modelu}


  Pole kwantowe w~szczególnej teorii względności \\
  Teoria ta jest względnie dobrze rozumiana i~wielokrotnie
  potwierdzona eksperymentalnie. Pole bezmasowego, pozbawionego
  spinu bozonu (pole skalarne) spełnia swobodne równanie
  \begin{equation}
    \label{eq:Promieniowanie-Hawkinga-01}
    \eta^{ \mu \nu } \partial_{ \mu } \partial_{ \nu } \widehat{ \phi }( x ) = 0.
  \end{equation}
  Załóżmy, że~chcemy napisać to równanie na klasycznej
  czasoprzestrzeni OTW. Jak to zrobić?

  Problem \\
  Okazuje~się, że~jest to bardzo subtelne i~podstępne zagadnienie.
  Tutaj przedstawię podejście Hawkinga, nie wdając~się zbytnio
  w~dyskusję na~ile jest ono uzasadnione.

\end{frame}
% ##################





% ##################
\begin{frame}
  \frametitle{Podstawy modelu}


  Podejście Hawkinga
  \begin{itemize}
    \RaggedRight

  \item Czasoprzestrzeń jest niekwantowa i~posiada metrykę
    $g_{ \mu \nu }( x )$, spełniającą równania Einsteina.

  \item Równanie pola kwantowe obecnego w~tej ogólnej
    czasoprzestrzeni otrzymujemy z~równania
    \eqref{eq:Promieniowanie-Hawkinga-01} przez dokonanie dość
    oczywistych podstawień
    \begin{equation}
      \label{eq:Promieniowanie-Hawkinga-02}
      \eta^{ \mu \nu } \to g^{ \mu \nu }( x ), \quad
      \partial_{ \mu } \to \nabla_{ \mu },
    \end{equation}
    otrzymując
    \begin{equation}
      \label{eq:Promieniowanie-Hawkinga-03}
      g^{ \mu \nu }( x ) \nabla_{ \mu } \nabla_{ \nu }
      \widehat{ \phi }( x ) = 0.
    \end{equation}

  \item Równania Einsteina mają postać:
    \begin{equation}
      \label{eq:Promieniowanie-Hawkinga-04}
      R_{ \mu \nu }( x ) - \frac{ 1 }{ 2 } R( x )\, g_{ \mu \nu }( x )
      = \langle \widehat{ T }_{ \mu \nu }( x ) \rangle.
    \end{equation}
    To równanie ma sens, jeśli wartość oczekiwana
    $\langle \widehat{ T }_{ \mu \nu }( x ) \rangle$ kwantowego
    tensora $\widehat{ T }_{ \mu \nu }( x )$ jest odpowiednio
    regularnym polem tensorowym.

  \end{itemize}

\end{frame}
% ##################





% ##################
\begin{frame}
  \frametitle{Pole kwantowe}


  Rozkład na mody normalne \\
  Postulujemy istnienie rozkładu pola analogicznie do~pola
  swobodnego w~czasoprzestrzeni Minkowskiego
  \begin{equation}
    \label{eq:Promieniowanie-Hawkinga-05}
    \widehat{ \phi }( x ) =
    \sum_{ i }( f_{ i }( x )\, \widehat{ a }_{ i }
    + \bar{ f }_{ i }( x )\, \widehat{ a }^{ \dagger }_{ i } ),
  \end{equation}
  gdzie $f_{ i }( x )$ są zupełnym układem ortonormalnych,
  zespolonych rozwiązań równania falowego, zawierającymi tylko
  dodatnie częstości. Stan próżni jest określony przez warunek
  \begin{equation}
    \label{eq:Promieniowanie-Hawkinga-06}
    \widehat{ a }_{ i }| 0 \rangle = 0, \quad \forall\, i.
  \end{equation}

  W~czasoprzestrzeni Minkowskiego \\
  \begin{equation}
    \label{eq:Promieniowanie-Hawkinga-07}
    f_{ i }( x ) =
    \frac{ 1 }{ \sqrt{ 2\pi \omega_{ 0 }( i ) } }
    \exp( -i \omega_{ 0 }( i ) t + i \vec{ k }( i )
    \cdot \vec{ \omega }( i ) ),
  \end{equation}

\end{frame}
% ##################





% ##################
\begin{frame}
  \frametitle{Pole kwantowe}


  Tu zaczynają~się prawdziwe problemy \\
  Ponieważ w~czasoprzestrzeni Minkowskiego mamy
  \begin{equation}
    \label{eq:Promieniowanie-Hawkinga-08}
    f_{ i }( x ) =
    \frac{ 1 }{ \sqrt{ 2\pi \omega_{ 0 }( i ) } }
    \exp( -i \omega_{ 0 }( i ) t + i \vec{ k }( i )
    \cdot \vec{ \omega }( i ) ),
  \end{equation}
  więc, aby~ustalić które częstości są dodatnie potrzebujemy
  wiedzieć ,,w~którą stronę płynie czas''. Jednak w~ogólnej
  zakrzywionej czasoprzestrzeni nie ma czegoś takiego jak
  \textbf{globalny} czas, więc mamy problem.

  Operator anihilacji \\
  To niejako z~definicji operator stojący przy funkcji zawierającej
  tylko dodatnie częstości. Skoro nie wiem, które częstości~są
  dodatnie, to nie mamy pojęcia który operator anihiluje, a~który
  kreuje cząstki!!!

\end{frame}
% ##################





% ##################
\begin{frame}
  \frametitle{Pole kwantowe i~zakrzywiona czasoprzestrzeń}


  Co można zrobić? \\
  \begin{itemize}
    \RaggedRight

  \item[1.] Poddać~się i~porzucić badanie tego problemu.

  \item[2.] Dojść do~wniosku, że~na zakrzywionej czasoprzestrzeni
    pojęcie cząstki nie ma sensu (z~punktu widzenia fizyki
    teoretycznej).

  \item[3.] Rozpatrzeć przypadek, gdzie coś da~się powiedzieć.

  \end{itemize}

  Wyjście 3 \\
  Jeżeli czasoprzestrzeń $M$ zawiera obszar który jest płaski
  (asymptotycznie płaski), wtedy istnieje w~nim kanoniczny czas
  i~używają go możemy zdefiniować rodzinę rozwiązań równań pola
  skalarnego o~dodatnich częstościach. Niemniej jeśli mamy dwa takie
  obszary nie możemy zwykle powiązać czasu w~jednym obszarze
  z~czasem w~drugim. Co~to oznacza?

\end{frame}
% ##################





% ##################
\begin{frame}
  \frametitle{Pole kwantowe i~zakrzywiona czasoprzestrzeń}


  Rozkład na mody normalne \\
  \begin{subequations}
    \begin{equation}
      \label{eq:Promieniowanie-Hawkinga-09a}
      \widehat{ \phi }( x ) =
      \sum_{ i }( f_{ i }( x )\, \widehat{ a }_{ i }
      + \bar{ f }_{ i }( x )\, \widehat{ a }^{ \dagger }_{ i } ),
    \end{equation}
    \begin{equation}
      \label{eq:Promieniowanie-Hawkinga-09b}
      \widehat{ a }_{ i }| 0 \rangle = 0, \quad \forall\, i.
    \end{equation}
  \end{subequations}

  Modelowy przykład \\
  Rozpatrzmy czasoprzestrzeń w~której mamy „następujący po~sobie”
  \begin{itemize}
    \RaggedRight

  \item[1.] obszar płaski;

  \item[2.] obszar z~niezerową krzywizną;

  \item[3.] obszar płaski.

  \end{itemize}

  Oznaczmy układ zupełny rozwiązań w~obszarze~1 przez
  $f_{ i }( x )$, natomiast w~obszarze $g_{ i }( x )$, zaś~próżnię
  w~tych dwóch obszarach jako~$| 0_{ 1 } \rangle$
  i~$| 0_{ 3 } \rangle$. Tym samym
  \begin{equation}
    \label{eq:Promieniowanie-Hawkinga-10}
    \widehat{ a }_{ f, i } | 0_{ 1 } \rangle = 0, \quad
    \widehat{ a }_{ g, i } | 0_{ 3 } \rangle = 0.
  \end{equation}

\end{frame}
% ##################





% ##################
\begin{frame}
  \frametitle{Pole kwantowe i~zakrzywiona czasoprzestrzeń}


  Ale \\
  Skoro w~ogólności układy funkcji $f_{ i }( x )$ i~$g_{ i }( x )$~są różne,
  więc tym samym $\widehat{ a }_{ f, i } \neq \widehat{ a }_{ g, i }$. Z~tego
  wynika, że
  \begin{equation}
    \label{eq:Promieniowanie-Hawkinga-11}
    \widehat{ a }_{ g, i }\, | 0_{ 1 } \rangle \neq 0.
  \end{equation}

  Bardzo dziwny wniosek \\
  „Przejście” przez dany obszar zakrzywienia powoduje kreacje
  nowych cząstek. Konkretnie tych cząstek dla których operatorów
  anihilacji zachodzi warunek \eqref{eq:Promieniowanie-Hawkinga-11}. Ich
  ilość jest dana przez
  \begin{equation}
    \label{eq:Promieniowanie-Hawkinga-12}
    N =
    \langle 0_{ 1 } | \sum_{ i } \widehat{ a }_{ g, i }^{ \dagger }
    \widehat{ a }_{ g, i }\, | 0_{ 1 } \rangle.
  \end{equation}

  Moja skromna opinia \\
  Wiele elementów tego podejścia jest wątpliwych. Zamiast tego można
  pójść np.~drogą porzucenia pojęcia cząstki w~zakrzywionej
  czasoprzestrzeni.
  % Nie rozumiem fizyki tego zjawiska.
  % Bardziej mnie przekonuje, że~pojęcie cząstki nie ma
  % teoretycznego
  % sensu w~zakrzywionej czasoprzestrzeni.

\end{frame}
% ##################










% ######################################
\section[Przypadek czarnej dziury Schwarzschilda]{Efekt Hawkinga
  dla~czarnej dziury Schwarzschilda}
% ######################################



% ##################
\begin{frame}
  \frametitle{Efekt Hawkinga dla~czarnej dziury Schwarzschilda}


  Uwaga \\
  Aby obliczyć ilość „cząstek wypromieniowanych przez czarną dziurę”
  \begin{equation}
    \label{eq:Promieniowanie-Hawkigna-13}
    N =
    \langle 0_{ 1 } | \sum_{ i } \widehat{ a }_{ g, i }^{ \dagger }
    \widehat{ a }_{ g, i }\, | 0_{ 1 } \rangle,
  \end{equation}
  Hawking wykonał całkiem pomysłowy i~dość żmudny rachunek.
  W~dalszym ciągu naszkicuję jak on~wyglądał oraz do~jakich wniosków
  prowadzi.

  Czarna dziura Schwarzschilda \\
  Metrykę dla tego przepadku można odczytać z~wzoru
  \begin{equation}
    \label{eq:Promieniowanie-Hawkinga-14}
    ds^{ 2 } =
    -\left(1 - \frac{ 2M }{ r } \right) dt^{ 2 }
    + \left(1 - \frac{ 2M }{ r } \right)^{ -1 } dr^{ 2 }
    + r^{ 2 } ( d \theta^{ 2 } + \sin^{ 2 }\theta d\phi^{ 2 } ).
  \end{equation}
  Przyjęliśmy „naturalny układ jednostek” $c = 1$, $G = 1$, $\hbar = 1$,
  $k_{ \mathrm{B} } = 1$.

\end{frame}
% ##################





% ##################
\begin{frame}{Pole kwantowe i~zakrzywiona czasoprzestrzeń}
  \frametitle{Efekt Hawkinga dla~czarnej dziury Schwarzschilda}


  Warunek ortogonalności \\
  Będziemy rozpatrywali rodziny rozwiązań równania pola skalarnego,
  \begin{equation}
    \label{eq:Promieniowanie-Hawkinga-15}
    g^{ \mu \nu }( x ) \nabla_{ \mu } \nabla_{ \nu }
    \widehat{ \phi }( x ) = 0.
  \end{equation}
  spełniające warunek:
  \begin{equation}
    \label{eq:Promieniowanie-Hawkinga-16}
    \frac{ 1 }{ 2 } i\int_{ S }( f_{ i }( x )
    \nabla_{ a }\bar{ f }_{ j }( x ) - f_{ j }( x )
    \nabla_{ a } \bar{ f }_{ i }( x ) ) \, d \sigma_{ a }
    = \delta_{ i j },
  \end{equation}
  gdzie $S$ to~wybrana powierzchnia.

  Powierzchnie~$S$ \\
  W~czasoprzestrzeni Schwarzschilda~są trzy interesujące nas
  powierzchnie.
  \begin{itemize}
    \RaggedRight

  \item[1.] Horyzont czarnej dziury $H$.

  \item[2.] Nieskończona przeszłość świetlna $\mathcal{J}^{ - }$.

  \item[3.] Nieskończona przyszłość świetlna $\mathcal{J}^{ + }$.

  \end{itemize}

\end{frame}
% ##################





% ##################
\begin{frame}{Podstawowe wiadomości}
  \frametitle{Efekt Hawkinga dla~czarnej dziury Schwarzschilda}


  Rodziny funkcji
  \begin{itemize}
    \RaggedRight

  \item Przez $f_{ i }( x )$ oznaczamy zupełny układ rozwiązań
    równania pola $\widehat{ \phi }( x )$, ortogonalnych względem
    całki po~$\mathcal{J}^{ - }$. Przedstawiają cząstki przylatujące
    z~nieskończonej przeszłości czasowej.

  \item $p_{ i }( x )$~są ortogonalne względem całki
    na~$\mathcal{J}^{ + }$ i~znikają wraz z~pierwszymi pochodnymi
    na~$H$. Przestawiają cząstki wylatujące z~czarnej dziury
    i~zmierzające do~nieskończonej przyszłość czasowej.

  \item $q_{ i }( x )$~są ortogonalne na~$H$ i~znikają
    na~$\mathcal{J}^{ + }$. Przedstawia cząstki ,,złapane'' wokół
    czarnej dziury

  \end{itemize}

\end{frame}
% ##################





% ##################
\begin{frame}
  \frametitle{Efekt Hawkinga dla~czarnej dziury Schwarzschilda}


  Dwa przedstawienia pola $\widehat{ \phi }( x )$ \\
  Pole $\widehat{ \phi }( x )$ jest określone w~całej czasoprzestrzeni,
  jeśli zadana jest jego wartość, wraz z~pierwszymi pochodnymi
  na~$\mathcal{J}^{ - }$, lub jednocześnie
  na~$H$ i~$\mathcal{J}^{ + }$. Możliwe są więc dwa następujące jego
  rozkłady na~mody normalne:
  \begin{equation}
    \label{eq:Promieniowanie-Hawkinga-17}
    \widehat{ \phi }( x ) =
    \sum_{ i }( f_{ i }( x )\, \widehat{ a }_{ f,\, i }
    + \bar{ f }_{ i }( x )\, \widehat{ a }^{ \dagger }_{ f,\, i } ),
  \end{equation}
  oraz
  \begin{equation}
    \label{eq:Promieniowanie-Hawkinga-18}
    \widehat{ \phi }( x ) =
    \sum_{ i }( p_{ i }( x )\, \widehat{ b }_{ p,\, i }
    + \bar{ p }_{ i }( x )\, \widehat{ b }^{ \dagger }_{ p,\, i }
    + q_{ i }( x )\, \widehat{ c }_{ q,\, i }
    + \bar{ q }_{ i }( x )\, \widehat{ c }^{ \dagger }_{ q,\, i } ).
  \end{equation}
  Chcemy policzyć cząstek zaobserwowanych w~$\mathcal{J}^{ + }$,
  czyli
  \begin{equation}
    \label{eq:Promieniowanie-Hawkinga-19}
    N =
    \langle 0_{ - } | \sum_{ i } \widehat{ b }^{ \dagger }_{ p,\, i }
    \widehat{ b }_{ p,\, i } | 0_{ - } \rangle.
  \end{equation}

\end{frame}
% ##################





% ##################
\begin{frame}
  \frametitle{Efekt Hawkinga dla~czarnej dziury Schwarzschilda}


  Ponieważ wiemy jak $\widehat{ a }_{ f,\, i }^{ \dagger }$,
  $\widehat{ a }_{ f,\, i }$ działają na~$| 0_{ - } \rangle$,
  problem będzie rozwiązany, jeśli wyrazimy za~ich pomocą
  $\widehat{ b }_{ p,\, i }^{ \dagger }$,
  $\widehat{ b }_{ p,\, i }$. By~to zrobić, zauważmy,
  że~$p_{ i }( x )$ i~$q_{ i }( x )$, możemy znów rozłożyć na
  funkcje $f_{ i }( x )$:
  \begin{subequations}
    \begin{equation}
      \label{eq:Promieniowanie-Hawkinga-20a}
      p_{ i }( x )
      = \sum_{ j }( \alpha_{ i j } f_{ j }( x ) + \beta_{ i j }
      \bar{ f }_{ j }( x ) ),
    \end{equation}
    \begin{equation}
      \label{eq:Promieniowanie-Hawkinga-20b}
      q_{ i }( x ) = \sum_{ j }( \gamma_{ i j } f_{ j }( x )
      + \eta_{ i j } \bar{ f }_{ j }( x ) ).
    \end{equation}
  \end{subequations}
  Podstawiają to~dostajemy
  \begin{equation}
    \label{eq:Promieniowanie-Hawkinga-21}
    \widehat{ \phi }( x )
    = \sum_{ i }( p_{ i }( x )\, \widehat{ b }_{ p,\, i }
    + \bar{ p }_{ i }( x )\, \widehat{ b }^{ \dagger }_{ p,\, i }
    + q_{ i }( x )\, \widehat{ c }_{ q,\, i }
    + \bar{ q }_{ i }( x )\, \widehat{ c }^{ \dagger }_{ q,\, i } ),
  \end{equation}
  skąd
  \begin{subequations}
    \begin{equation}
      \label{eq:Promieniowanie-Hawkinga-22a}
      \widehat{ b }_{ i } = \sum_{ j }( \bar{ \alpha }_{ i j }\,
      \widehat{ a }_{ j } - \bar{ \beta }_{ i j }\,
      \widehat{ a }^{ \dagger }_{ j } ).
    \end{equation}
    \begin{equation}
      \label{eq:Promieniowanie-Hawkinga-22b}
      \widehat{ c }_{ i } = \sum_{ j }( \bar{ \gamma }_{ i j }\,
      \widehat{ a }_{ j }
      - \bar{ \eta }_{ i j }\, \widehat{ a }^{ \dagger }_{ j } ).
    \end{equation}
  \end{subequations}

\end{frame}
% ##################





% ##################
\begin{frame}{Szkic obliczeń}
  \frametitle{Efekt Hawkinga dla~czarnej dziury Schwarzschilda}


  Wniosek \\
  Liczba cząstek emitowana w $i$-tym modzie wynosi więc:
  \begin{equation}
    \label{eq:Promieniowanie-Hawkinga-23}
    N_{ i } =
    \langle 0 | \widehat{ b }^{ \dagger }_{ i } \widehat{ b }_{ i } | 0 \rangle
    = \sum_{ j } | \beta_{ i j } |^{ 2 }.
  \end{equation}
  W~tym miejscu problem został w~zasadzie zredukowany do~dość
  skomplikowanego zagadnienia z~teorii rozpraszania.

  Uwagi o~obliczaniu $\beta_{ i j }$ \\
  Ich wyliczenie, wymaga dość żmudnych rachunków, uwzględniających
  rozkład na~odpowiednie mody sferyczne, operowania pakietami
  falowymi, etc. Dlatego przedstawię tylko konieczne wzory
  i~oznaczenia.

  Uwaga o~tych rachunkach \\
  Dodatkowo Hawking skorzystał w~tych rachunkach z~przybliżenia
  optyki geometrycznej, czego zasadność bywała krytykowana (zob.
  \cite{FredenhagenHaagDerivationOfHawkingRadiation1990}).

\end{frame}
% ##################





% ##################
\begin{frame}
  \frametitle{Efekt Hawkinga dla~czarnej dziury Schwarzschilda}


  Rozkład „sferyczny” \\
  We~współrzędnych „sferycznych” zupełny układ funkcji jest dany
  przez $f_{ \omega l m }( x )$, ,,numerowanych'' ciągłym
  wskaźnikiem częstości $\omega$. Tym samym zmieniają~się też
  indeksy współczynników $\alpha_{ \omega l m\, \omega' l' m' }$
  i~$\beta_{ \omega l m\, \omega' l' m' }$. Definiujemy
  \begin{equation}
    \label{eq:Promieniowanie-Hawkinga-24}
    p_{ \omega l m }( x ) =
    \int\limits_{ 0 }^{ +\infty } ( \alpha_{ \omega l m\, \omega' l m }
    f_{ \omega' l m }( x ) + \beta_{ \omega l m\, \omega' l m }
    \bar{ f }_{ \omega' l m }( x ) ) \, d \omega'.
  \end{equation}
  Niech $p_{ \omega l m }^{ ( 1 ) }( x )$ oznacza rozwiązanie, które
  propaguje~się wstecz w czasie, rozprasza na~statycznej czarnej
  dziurze Schwarzchilda i~dociera na~$\mathcal{J}^{ - }$ z~tą samą
  częstością~$\omega$. Dokonujemy rozkładu
  \begin{equation}
    \label{eq:Promieniowanie-Hawkinga-25}
    p_{ \omega l m }( x ) =
    p_{ \omega l m }^{ ( 1 ) }( x ) + p_{ \omega l m }^{ ( 2 ) }( x ).
  \end{equation}
  \begin{equation}
    f_{ \omega' l m } =
    ( 2 \pi )^{ - \frac{ 1 }{ 2 } } r^{ - 1 }
    ( \omega' )^{ - \frac{ 1 }{ 2 } } F_{ \omega ' }( r ) e^{ i \omega' v }
    Y_{ l m }( \theta, \varphi ),
  \end{equation}
  \begin{equation}
    f_{ \omega' l m }
    = ( 2 \pi )^{ - \frac{ 1 }{ 2 } }
    r^{ - 1 } ( \omega' )^{ - \frac{ 1 }{ 2 } } P_{ \omega ' }( r )
    e^{ i \omega' u } Y_{ l m }( \theta, \varphi ),
  \end{equation}

\end{frame}
% ##################





% ##################
\begin{frame}
  \frametitle{Efekt Hawkinga dla~czarnej dziury Schwarzschilda}


  Analogicznie
  \begin{subequations}
    \begin{align}
      \label{eq:Promieniowanie-Hawkinga-26}
      \alpha_{ \omega l m\, \omega' l m }
      &= \alpha_{ \omega l m\, \omega' l m }^{ ( 1 ) }
        + \alpha_{ \omega l m\, \omega' l m }^{ ( 2 ) }, \\
      \beta_{ \omega l m\, \omega' l m }
      &= \beta_{ \omega l m\, \omega' l m }^{ ( 1 ) }
        + \beta_{ \omega l m\, \omega' l m }^{ ( 2 ) }.
    \end{align}
  \end{subequations}

  Definiujemy nową rodzinę modów \\
  \begin{equation}
    \label{eq:Promieniowanie-Hawkinga-27}
    p_{ j n l m }( x ) =
    \varepsilon^{ -\frac{ 1 }{ 2 } } \int\limits_{ j \varepsilon }^{ ( j + 1 ) \varepsilon }
    e^{ -2 \pi i \, \mathrm{n} \omega / \varepsilon } p_{ \omega l m }( x ) \, d \omega, \quad
    j = 0, 1, 2, \ldots
  \end{equation}
  gdzie $\varepsilon > 0$. Potrzebujemy jeszcze jednej wielkości
  \begin{equation}
    \label{eq:Promieniowanie-Hawkinga-28}
    \Gamma_{ j n } =
    \int\limits_{ 0 }^{ +\infty }
    \left( \abs{ \alpha_{ j n \omega' l m }^{ ( 2 ) } }^{ 2 }
      - \abs{ \beta_{ j n \omega' l m }^{ ( 2 ) } }^{ 2 } \right) \,
    d \omega'.
  \end{equation}

\end{frame}
% ##################





% ##################
\begin{frame}
  \frametitle{Końcowy wynik i~jego znaczenie}


  Ilość cząstek wyemitowanych w~modzie $p_{ j n l m }( x )$ wynosi
  \begin{equation}
    \label{eq:Promieniowanie-Hawkinga-29}
    N_{ j n l m } =
    \Gamma_{ j n } \; \frac{ 1 }{ e^{  \frac{ 2 \pi }{ \kappa } \omega } - 1 }.
  \end{equation}
  $\kappa$ to tak zwana grawitacja powierzchniowa (ang.~surface
  gravity).

  Sens \\
  Po~analizie $\Gamma_{ j n }$ okazuje~się, że~ten wzór opisujący
  „emisję cząstek z~powierzchni czarnej dziury Schwarzschilda”,
  jest identyczny jak dla ciała dla którego
  $E / k_{ \textrm{B} } T = 2\pi \omega / \kappa$. Biorąc
  $E = \hbar \omega$ otrzymujemy
  \begin{equation}
    \label{eq:HawkingPromieniowanie-30}
    T =
    \frac{ \hbar \kappa }{ 2\pi k_{ \mathrm{B} } } =
    \frac{ \hbar }{ 2\pi c } A \approx
    10^{ -6 } \left( \frac{ M_{ \odot } }{ M_{ BH } } \right)
    {}^{ \circ } \mathrm{K},
  \end{equation}
  gdzie $A$ to powierzchnia czarnej dziury. Według Hawkinga podobny
  wyniki zachodzi dla pola elektromagnetycznego i~zlinearyzowanej
  grawitacji.

\end{frame}
% ##################





% ##################
\begin{frame}
  \frametitle{Uwagi końcowe}


  Od~1975 roku upłynęło dużo czasu i~całkiem sporo~się działo
  również jeśli chodzi o~wyprowadzenie oraz~konsekwencje efektu
  Hawkinga, poddano krytyce metodę użytą w~oryginalnej pracy, jak
  i~pomniejsze jej wyniki oraz~stwierdzenia. Na~dokładniejszą
  dyskusję nie ma tu jednak czasu.

\end{frame}
% ##################










% ######################################
\appendix
% ######################################





% ######################################
\EndingSlide{Dziękuję! Pytania?}
% ######################################










% ##################
\begin{frame}
  \frametitle{Bibliografia}


  \bibliographystyle{plalpha}

  \bibliography{ArticPhilNatur}{}

\end{frame}
% ##################





% ##################
\begin{frame}
  \frametitle{Grawitacja powierzchniowa czarnej dziury}


  Założenia \\
  Rozważmy statyczną, osiowo symetryczną, asymptotycznie płaską
  czasoprzestrzeń z~obecną w niej czarną dziurą. To~implikuje
  istnienie dwóch pól~Killing'a: $K^{ \mu }$~odpowiedzialne
  za~przesunięcie w~czasie i~$\tilde{ K }^{ \mu }$ odpowiedzialne
  za~obroty przestrzenne. Istnienie przestrzennopodobnego pola
  Killinga $\tilde{ K }^{ \mu }$ pozwala nam wybrać asymptotycznie
  płaską hiperpowierzchnię $B$ do~której jest ono styczne i~która
  przecina horyzont zdarzeń na~dwie części

\end{frame}
% ##################





% ##################
\begin{frame}
  \frametitle{Grawitacja powierzchniowa czarnej dziury}


  Trochę więcej pojęć \\
  Rozważmy parametr $t$ krzywych całkowych pola $K^{ \mu }$ i~wektor
  świetlny obliczony na horyzoncie zdarzeń:
  \begin{equation}
    \label{eq:Promieniowanie-Hawkinga-31}
    l^{ \mu } =  \frac{ d x^{ \mu } }{ d t }.
  \end{equation}
  Wprowadźmy wektor $n^{ \mu }$ styczny do~hiperpowierzchni $B$
  i~znormalizowany do
  \begin{equation}
    \label{eq:Promieniowanie-Hawkinga-32}
    l^{ \mu } n_{ \mu } = -1.
  \end{equation}
  Grawitacja powierzchniowa w~określonym punkcie jest zdefiniowana
  jako:
  \begin{equation}
    \label{eq:Promieniowanie-Hawkinga-33}
    \kappa = - \big( \nabla_{ \mu } l_{ \nu } \big) n^{ \mu } l^{ \nu }.
  \end{equation}
  Dla czarnej dziury Schwarzschilda jest ona stała na~całych
  horyzoncie zdarzeń.

\end{frame}
% ##################





% ##################
\begin{frame}
  \frametitle{Grawitacja powierzchniowa czarnej dziury}


  Sens matematyczno-fizyczny \\
  $\kappa$ wyraża stopień w jakim $t$ nie jest parametrem
  afinicznym.

\end{frame}
% ##################





% ##################
\begin{frame}
  \frametitle{Inne prace}


  Według mnie interesujący nurt \\
  Klaus Fredenhagena i~Rudolfa Haaga w~swej pracy
  \textit{On~the~derivation~of Hawking radiation associated with
    the~formation~of a~black hole} z~1990 roku
  (\cite{FredenhagenHaagDerivationOfHawkingRadiation1990}),
  kontynuują podejście Hawkinga, opierając je jednak na~inny
  sposobie postawieniu problemu w~kwantowej teorii pola: analizie
  funkcji Wightmana. Uzyskują dodatkowe wyniki i~potencjalne
  ograniczenia dla~zależności pokazanych przez Hawkinga.

  Wynik oraz uwagi Fredenhagena i~Haaga
  \begin{itemize}
    \RaggedRight

  \item „W~kolapsie grawitacyjny promieniowanie Hawkinga dla~dużych
    czasów jest ściśle związane z~granicą skalowania na~sferze, gdy
    promień gwizdy przechodzi przez promień Schwarzschilda (przy
    założeniu, że~zaniedbujemy oddziaływanie promieniowania
    na~metrykę).”

  \item Dla pola swobodnego, otrzymali wyniki jak Hawking. Analiza
    jednak sugeruje, że~raczej nie istnieje żaden prosty,
    uniwersalny wzór.

  \end{itemize}

\end{frame}
% ##################





% ##################
\begin{frame}
  \frametitle{Inne podejście do~zagadnienia}


  Wynik oraz uwagi Fredenhagena i~Haaga
  \begin{itemize}
    \RaggedRight

  \item Praca argumentuje, że~asymptotyczna swoboda QCD może mieć
    duże znaczenie dla prawdziwej sytuacji.

  \item Wykorzystanie przez Hawkinga w~jego oryginalnej pracy
    przybliżenia optyki geometrycznej nie jest w~pełni uzasadnione,
    gdyż, mówiąc tym językiem, promień światła będzie musiał
    przechodzić przez obszar gdzie ekstremalnie szybko zmienia~się
    współczynnik załamania.

  \end{itemize}

  Promieniowanie Hawkinga w~teorii bez cząstek \\
  „Uczniowie” Fredenhagena i~Haaga, Valter Moretti oraz~Nicola
  Pinamonti kontynuowali te~badania, wyprowadzając efekt Hawkinga
  w~podejściu do~teorii pola, które nie wymaga pojęcia cząstki.
  Wyniki ich można znaleźć w~pracy Moretti, Pinamonti, \textit{State
    independence for~tunneling processes through black hole horizons
    and~Hawking radiation}, \cite[]. Na~ten temat nie mogę niestety
  wiele powiedzieć.

\end{frame}
% ##################










% ############################

% Koniec dokumentu
\end{document}
