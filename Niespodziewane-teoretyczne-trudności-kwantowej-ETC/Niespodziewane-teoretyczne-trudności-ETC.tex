% ---------------------------------------------------------------------
% Basic configuration of Beamera and Jagiellonian
% ---------------------------------------------------------------------
\RequirePackage[l2tabu, orthodox]{nag}



\ifx\PresentationStyle\notset
\def\PresentationStyle{dark}
\fi



\documentclass[10pt,t]{beamer}
\mode<presentation>
\usetheme[style=\PresentationStyle,logoColor=monochromaticJUwhite,JUlogotitle=yes]{jagiellonian}



% ---------------------------------------
% Configuration files of Jagiellonian loceted in catalog preambule
% ---------------------------------------
% Configuration for polish language
% Need description
\usepackage[polish]{babel}
% Need description
\usepackage[MeX]{polski}



% ------------------------------
% Better support of polish chars in technical parts of PDF
% ------------------------------
\hypersetup{pdfencoding=auto,psdextra}

% Package "textpos" give as enviroment "textblock" which is very usefull in
% arranging text on slides.

% This is standard configuration of "textpos"
\usepackage[overlay,absolute]{textpos}

% If you need to see bounds of "textblock's" comment line above and uncomment
% one below.

% Caution! When showboxes option is on significant ammunt of space is add
% to the top of textblock and as such, everyting put in them gone down.
% We need to check how to remove this bug.

% \usepackage[showboxes,overlay,absolute]{textpos}



% Setting scale length for package "textpos"
\setlength{\TPHorizModule}{10mm}
\setlength{\TPVertModule}{\TPHorizModule}


% ---------------------------------------
% TikZ
% ---------------------------------------
% Importing TikZ libraries
\usetikzlibrary{arrows.meta}
\usetikzlibrary{positioning}





% % Configuration package "bm" that need for making bold symbols
% \newcommand{\bmmax}{0}
% \newcommand{\hmmax}{0}
% \usepackage{bm}




% ---------------------------------------
% Packages for scientific texts
% ---------------------------------------
% \let\lll\undefined  % Sometimes you must use this line to allow
% "amsmath" package to works with packages with packages for polish
% languge imported
% /preambul/LanguageSettings/JagiellonianPolishLanguageSettings.tex.
% This comments (probably) removes polish letter Ł.
\usepackage{amsmath}  % Packages from American Mathematical Society (AMS)
\usepackage{amssymb}
\usepackage{amscd}
\usepackage{amsthm}
\usepackage{siunitx}  % Package for typsetting SI units.
\usepackage{upgreek}  % Better looking greek letters.
% Example of using upgreek: pi = \uppi


\usepackage{calrsfs}  % Zmienia czcionkę kaligraficzną w \mathcal
% na ładniejszą. Może w innych miejscach robi to samo, ale o tym nic
% nie wiem.










% ---------------------------------------
% Packages written for lectures "Geometria 3D dla twórców gier wideo"
% ---------------------------------------
% \usepackage{./ProgramowanieSymulacjiFizykiPaczki/ProgramowanieSymulacjiFizyki}
% \usepackage{./ProgramowanieSymulacjiFizykiPaczki/ProgramowanieSymulacjiFizykiIndeksy}
% \usepackage{./ProgramowanieSymulacjiFizykiPaczki/ProgramowanieSymulacjiFizykiTikZStyle}





% !!!!!!!!!!!!!!!!!!!!!!!!!!!!!!
% !!!!!!!!!!!!!!!!!!!!!!!!!!!!!!
% EVIL STUFF
\if\JUlogotitle1
\edef\LogoJUPath{LogoJU_\JUlogoLang/LogoJU_\JUlogoShape_\JUlogoColor.pdf}
\titlegraphic{\hfill\includegraphics[scale=0.22]
{./JagiellonianPictures/\LogoJUPath}}
\fi
% ---------------------------------------
% Commands for handling colors
% ---------------------------------------


% Command for setting normal text color for some text in math modestyle
% Text color depend on used style of Jagiellonian

% Beamer version of command
\newcommand{\TextWithNormalTextColor}[1]{%
  {\color{jNormalTextFGColor}
    \setbeamercolor{math text}{fg=jNormalTextFGColor} {#1}}
}

% Article and similar classes version of command
% \newcommand{\TextWithNormalTextColor}[1]{%
%   {\color{jNormalTextsFGColor} {#1}}
% }



% Beamer version of command
\newcommand{\NormalTextInMathMode}[1]{%
  {\color{jNormalTextFGColor}
    \setbeamercolor{math text}{fg=jNormalTextFGColor} \text{#1}}
}


% Article and similar classes version of command
% \newcommand{\NormalTextInMathMode}[1]{%
%   {\color{jNormalTextsFGColor} \text{#1}}
% }




% Command that sets color of some mathematical text to the same color
% that has normal text in header (?)

% Beamer version of the command
\newcommand{\MathTextFrametitleFGColor}[1]{%
  {\color{jFrametitleFGColor}
    \setbeamercolor{math text}{fg=jFrametitleFGColor} #1}
}

% Article and similar classes version of the command
% \newcommand{\MathTextWhiteColor}[1]{{\color{jFrametitleFGColor} #1}}





% Command for setting color of alert text for some text in math modestyle

% Beamer version of the command
\newcommand{\MathTextAlertColor}[1]{%
  {\color{jOrange} \setbeamercolor{math text}{fg=jOrange} #1}
}

% Article and similar classes version of the command
% \newcommand{\MathTextAlertColor}[1]{{\color{jOrange} #1}}





% Command that allow you to sets chosen color as the color of some text into
% math mode. Due to some nuances in the way that Beamer handle colors
% it not work in all cases. We hope that in the future we will improve it.

% Beamer version of the command
\newcommand{\SetMathTextColor}[2]{%
  {\color{#1} \setbeamercolor{math text}{fg=#1} #2}
}


% Article and similar classes version of the command
% \newcommand{\SetMathTextColor}[2]{{\color{#1} #2}}










% ---------------------------------------
% Commands for few special slides
% ---------------------------------------
\newcommand{\EndingSlide}[1]{%
  \begin{frame}[standout]

    \begingroup

    \color{jFrametitleFGColor}

    #1

    \endgroup

  \end{frame}
}










% ---------------------------------------
% Commands for setting background pictures for some slides
% ---------------------------------------
\newcommand{\TitleBackgroundPicture}
{./JagiellonianPictures/Backgrounds/LajkonikDark.png}
\newcommand{\SectionBackgroundPicture}
{./JagiellonianPictures/Backgrounds/LajkonikLight.png}



\newcommand{\TitleSlideWithPicture}{%
  \begingroup

  \usebackgroundtemplate{%
    \includegraphics[height=\paperheight]{\TitleBackgroundPicture}}

  \maketitle

  \endgroup
}





\newcommand{\SectionSlideWithPicture}[1]{%
  \begingroup

  \usebackgroundtemplate{%
    \includegraphics[height=\paperheight]{\SectionBackgroundPicture}}

  \setbeamercolor{titlelike}{fg=normal text.fg}

  \section{#1}

  \endgroup
}










% ---------------------------------------
% Commands for lectures "Geometria 3D dla twórców gier wideo"
% Polish version
% ---------------------------------------
% Komendy teraz wykomentowane były potrzebne, gdy loga były na niebieskim
% tle, nie na białym. A są na białym bo tego chcieli w biurze projektu.
% \newcommand{\FundingLogoWhitePicturePL}
% {./PresentationPictures/CommonPictures/logotypFundusze_biale_bez_tla2.pdf}
\newcommand{\FundingLogoColorPicturePL}
{./PresentationPictures/CommonPictures/European_Funds_color_PL.pdf}
% \newcommand{\EULogoWhitePicturePL}
% {./PresentationPictures/CommonPictures/logotypUE_biale_bez_tla2.pdf}
\newcommand{\EUSocialFundLogoColorPicturePL}
{./PresentationPictures/CommonPictures/EU_Social_Fund_color_PL.pdf}
% \newcommand{\ZintegrUJLogoWhitePicturePL}
% {./PresentationPictures/CommonPictures/zintegruj-logo-white.pdf}
\newcommand{\ZintegrUJLogoColorPicturePL}
{./PresentationPictures/CommonPictures/ZintegrUJ_color.pdf}
\newcommand{\JULogoColorPicturePL}
{./JagiellonianPictures/LogoJU_PL/LogoJU_A_color.pdf}





\newcommand{\GeometryThreeDSpecialBeginningSlidePL}{%
  \begin{frame}[standout]

    \begin{textblock}{11}(1,0.7)

      \begin{flushleft}

        \mdseries

        \footnotesize

        \color{jFrametitleFGColor}

        Materiał powstał w ramach projektu współfinansowanego ze środków
        Unii Europejskiej w ramach Europejskiego Funduszu Społecznego
        POWR.03.05.00-00-Z309/17-00.

      \end{flushleft}

    \end{textblock}





    \begin{textblock}{10}(0,2.2)

      \tikz \fill[color=jBackgroundStyleLight] (0,0) rectangle (12.8,-1.5);

    \end{textblock}


    \begin{textblock}{3.2}(1,2.45)

      \includegraphics[scale=0.3]{\FundingLogoColorPicturePL}

    \end{textblock}


    \begin{textblock}{2.5}(3.7,2.5)

      \includegraphics[scale=0.2]{\JULogoColorPicturePL}

    \end{textblock}


    \begin{textblock}{2.5}(6,2.4)

      \includegraphics[scale=0.1]{\ZintegrUJLogoColorPicturePL}

    \end{textblock}


    \begin{textblock}{4.2}(8.4,2.6)

      \includegraphics[scale=0.3]{\EUSocialFundLogoColorPicturePL}

    \end{textblock}

  \end{frame}
}



\newcommand{\GeometryThreeDTwoSpecialBeginningSlidesPL}{%
  \begin{frame}[standout]

    \begin{textblock}{11}(1,0.7)

      \begin{flushleft}

        \mdseries

        \footnotesize

        \color{jFrametitleFGColor}

        Materiał powstał w ramach projektu współfinansowanego ze środków
        Unii Europejskiej w ramach Europejskiego Funduszu Społecznego
        POWR.03.05.00-00-Z309/17-00.

      \end{flushleft}

    \end{textblock}





    \begin{textblock}{10}(0,2.2)

      \tikz \fill[color=jBackgroundStyleLight] (0,0) rectangle (12.8,-1.5);

    \end{textblock}


    \begin{textblock}{3.2}(1,2.45)

      \includegraphics[scale=0.3]{\FundingLogoColorPicturePL}

    \end{textblock}


    \begin{textblock}{2.5}(3.7,2.5)

      \includegraphics[scale=0.2]{\JULogoColorPicturePL}

    \end{textblock}


    \begin{textblock}{2.5}(6,2.4)

      \includegraphics[scale=0.1]{\ZintegrUJLogoColorPicturePL}

    \end{textblock}


    \begin{textblock}{4.2}(8.4,2.6)

      \includegraphics[scale=0.3]{\EUSocialFundLogoColorPicturePL}

    \end{textblock}

  \end{frame}





  \TitleSlideWithPicture
}



\newcommand{\GeometryThreeDSpecialEndingSlidePL}{%
  \begin{frame}[standout]

    \begin{textblock}{11}(1,0.7)

      \begin{flushleft}

        \mdseries

        \footnotesize

        \color{jFrametitleFGColor}

        Materiał powstał w ramach projektu współfinansowanego ze środków
        Unii Europejskiej w~ramach Europejskiego Funduszu Społecznego
        POWR.03.05.00-00-Z309/17-00.

      \end{flushleft}

    \end{textblock}





    \begin{textblock}{10}(0,2.2)

      \tikz \fill[color=jBackgroundStyleLight] (0,0) rectangle (12.8,-1.5);

    \end{textblock}


    \begin{textblock}{3.2}(1,2.45)

      \includegraphics[scale=0.3]{\FundingLogoColorPicturePL}

    \end{textblock}


    \begin{textblock}{2.5}(3.7,2.5)

      \includegraphics[scale=0.2]{\JULogoColorPicturePL}

    \end{textblock}


    \begin{textblock}{2.5}(6,2.4)

      \includegraphics[scale=0.1]{\ZintegrUJLogoColorPicturePL}

    \end{textblock}


    \begin{textblock}{4.2}(8.4,2.6)

      \includegraphics[scale=0.3]{\EUSocialFundLogoColorPicturePL}

    \end{textblock}





    \begin{textblock}{11}(1,4)

      \begin{flushleft}

        \mdseries

        \footnotesize

        \RaggedRight

        \color{jFrametitleFGColor}

        Treść niniejszego wykładu jest udostępniona na~licencji
        Creative Commons (\textsc{cc}), z~uzna\-niem autorstwa
        (\textsc{by}) oraz udostępnianiem na tych samych warunkach
        (\textsc{sa}). Rysunki i~wy\-kresy zawarte w~wykładzie są
        autorstwa dr.~hab.~Pawła Węgrzyna et~al. i~są dostępne
        na tej samej licencji, o~ile nie wskazano inaczej.
        W~prezentacji wykorzystano temat Beamera Jagiellonian,
        oparty na~temacie Metropolis Matthiasa Vogelgesanga,
        dostępnym na licencji \LaTeX{} Project Public License~1.3c
        pod adresem: \colorhref{https://github.com/matze/mtheme}
        {https://github.com/matze/mtheme}.

        Projekt typograficzny: Iwona Grabska-Gradzińska \\
        Skład: Kamil Ziemian;
        Korekta: Wojciech Palacz \\
        Modele: Dariusz Frymus, Kamil Nowakowski \\
        Rysunki i~wykresy: Kamil Ziemian, Paweł Węgrzyn, Wojciech Palacz

      \end{flushleft}

    \end{textblock}

  \end{frame}
}



\newcommand{\GeometryThreeDTwoSpecialEndingSlidesPL}[1]{%
  \begin{frame}[standout]


    \begin{textblock}{11}(1,0.7)

      \begin{flushleft}

        \mdseries

        \footnotesize

        \color{jFrametitleFGColor}

        Materiał powstał w ramach projektu współfinansowanego ze środków
        Unii Europejskiej w~ramach Europejskiego Funduszu Społecznego
        POWR.03.05.00-00-Z309/17-00.

      \end{flushleft}

    \end{textblock}





    \begin{textblock}{10}(0,2.2)

      \tikz \fill[color=jBackgroundStyleLight] (0,0) rectangle (12.8,-1.5);

    \end{textblock}


    \begin{textblock}{3.2}(1,2.45)

      \includegraphics[scale=0.3]{\FundingLogoColorPicturePL}

    \end{textblock}


    \begin{textblock}{2.5}(3.7,2.5)

      \includegraphics[scale=0.2]{\JULogoColorPicturePL}

    \end{textblock}


    \begin{textblock}{2.5}(6,2.4)

      \includegraphics[scale=0.1]{\ZintegrUJLogoColorPicturePL}

    \end{textblock}


    \begin{textblock}{4.2}(8.4,2.6)

      \includegraphics[scale=0.3]{\EUSocialFundLogoColorPicturePL}

    \end{textblock}





    \begin{textblock}{11}(1,4)

      \begin{flushleft}

        \mdseries

        \footnotesize

        \RaggedRight

        \color{jFrametitleFGColor}

        Treść niniejszego wykładu jest udostępniona na~licencji
        Creative Commons (\textsc{cc}), z~uzna\-niem autorstwa
        (\textsc{by}) oraz udostępnianiem na tych samych warunkach
        (\textsc{sa}). Rysunki i~wy\-kresy zawarte w~wykładzie są
        autorstwa dr.~hab.~Pawła Węgrzyna et~al. i~są dostępne
        na tej samej licencji, o~ile nie wskazano inaczej.
        W~prezentacji wykorzystano temat Beamera Jagiellonian,
        oparty na~temacie Metropolis Matthiasa Vogelgesanga,
        dostępnym na licencji \LaTeX{} Project Public License~1.3c
        pod adresem: \colorhref{https://github.com/matze/mtheme}
        {https://github.com/matze/mtheme}.

        Projekt typograficzny: Iwona Grabska-Gradzińska \\
        Skład: Kamil Ziemian;
        Korekta: Wojciech Palacz \\
        Modele: Dariusz Frymus, Kamil Nowakowski \\
        Rysunki i~wykresy: Kamil Ziemian, Paweł Węgrzyn, Wojciech Palacz

      \end{flushleft}

    \end{textblock}

  \end{frame}





  \begin{frame}[standout]

    \begingroup

    \color{jFrametitleFGColor}

    #1

    \endgroup

  \end{frame}
}



\newcommand{\GeometryThreeDSpecialEndingSlideVideoPL}{%
  \begin{frame}[standout]

    \begin{textblock}{11}(1,0.7)

      \begin{flushleft}

        \mdseries

        \footnotesize

        \color{jFrametitleFGColor}

        Materiał powstał w ramach projektu współfinansowanego ze środków
        Unii Europejskiej w~ramach Europejskiego Funduszu Społecznego
        POWR.03.05.00-00-Z309/17-00.

      \end{flushleft}

    \end{textblock}





    \begin{textblock}{10}(0,2.2)

      \tikz \fill[color=jBackgroundStyleLight] (0,0) rectangle (12.8,-1.5);

    \end{textblock}


    \begin{textblock}{3.2}(1,2.45)

      \includegraphics[scale=0.3]{\FundingLogoColorPicturePL}

    \end{textblock}


    \begin{textblock}{2.5}(3.7,2.5)

      \includegraphics[scale=0.2]{\JULogoColorPicturePL}

    \end{textblock}


    \begin{textblock}{2.5}(6,2.4)

      \includegraphics[scale=0.1]{\ZintegrUJLogoColorPicturePL}

    \end{textblock}


    \begin{textblock}{4.2}(8.4,2.6)

      \includegraphics[scale=0.3]{\EUSocialFundLogoColorPicturePL}

    \end{textblock}





    \begin{textblock}{11}(1,4)

      \begin{flushleft}

        \mdseries

        \footnotesize

        \RaggedRight

        \color{jFrametitleFGColor}

        Treść niniejszego wykładu jest udostępniona na~licencji
        Creative Commons (\textsc{cc}), z~uzna\-niem autorstwa
        (\textsc{by}) oraz udostępnianiem na tych samych warunkach
        (\textsc{sa}). Rysunki i~wy\-kresy zawarte w~wykładzie są
        autorstwa dr.~hab.~Pawła Węgrzyna et~al. i~są dostępne
        na tej samej licencji, o~ile nie wskazano inaczej.
        W~prezentacji wykorzystano temat Beamera Jagiellonian,
        oparty na~temacie Metropolis Matthiasa Vogelgesanga,
        dostępnym na licencji \LaTeX{} Project Public License~1.3c
        pod adresem: \colorhref{https://github.com/matze/mtheme}
        {https://github.com/matze/mtheme}.

        Projekt typograficzny: Iwona Grabska-Gradzińska;
        Skład: Kamil Ziemian \\
        Korekta: Wojciech Palacz;
        Modele: Dariusz Frymus, Kamil Nowakowski \\
        Rysunki i~wykresy: Kamil Ziemian, Paweł Węgrzyn, Wojciech Palacz \\
        Montaż: Agencja Filmowa Film \& Television Production~-- Zbigniew
        Masklak

      \end{flushleft}

    \end{textblock}

  \end{frame}
}





\newcommand{\GeometryThreeDTwoSpecialEndingSlidesVideoPL}[1]{%
  \begin{frame}[standout]

    \begin{textblock}{11}(1,0.7)

      \begin{flushleft}

        \mdseries

        \footnotesize

        \color{jFrametitleFGColor}

        Materiał powstał w ramach projektu współfinansowanego ze środków
        Unii Europejskiej w~ramach Europejskiego Funduszu Społecznego
        POWR.03.05.00-00-Z309/17-00.

      \end{flushleft}

    \end{textblock}





    \begin{textblock}{10}(0,2.2)

      \tikz \fill[color=jBackgroundStyleLight] (0,0) rectangle (12.8,-1.5);

    \end{textblock}


    \begin{textblock}{3.2}(1,2.45)

      \includegraphics[scale=0.3]{\FundingLogoColorPicturePL}

    \end{textblock}


    \begin{textblock}{2.5}(3.7,2.5)

      \includegraphics[scale=0.2]{\JULogoColorPicturePL}

    \end{textblock}


    \begin{textblock}{2.5}(6,2.4)

      \includegraphics[scale=0.1]{\ZintegrUJLogoColorPicturePL}

    \end{textblock}


    \begin{textblock}{4.2}(8.4,2.6)

      \includegraphics[scale=0.3]{\EUSocialFundLogoColorPicturePL}

    \end{textblock}





    \begin{textblock}{11}(1,4)

      \begin{flushleft}

        \mdseries

        \footnotesize

        \RaggedRight

        \color{jFrametitleFGColor}

        Treść niniejszego wykładu jest udostępniona na~licencji
        Creative Commons (\textsc{cc}), z~uzna\-niem autorstwa
        (\textsc{by}) oraz udostępnianiem na tych samych warunkach
        (\textsc{sa}). Rysunki i~wy\-kresy zawarte w~wykładzie są
        autorstwa dr.~hab.~Pawła Węgrzyna et~al. i~są dostępne
        na tej samej licencji, o~ile nie wskazano inaczej.
        W~prezentacji wykorzystano temat Beamera Jagiellonian,
        oparty na~temacie Metropolis Matthiasa Vogelgesanga,
        dostępnym na licencji \LaTeX{} Project Public License~1.3c
        pod adresem: \colorhref{https://github.com/matze/mtheme}
        {https://github.com/matze/mtheme}.

        Projekt typograficzny: Iwona Grabska-Gradzińska;
        Skład: Kamil Ziemian \\
        Korekta: Wojciech Palacz;
        Modele: Dariusz Frymus, Kamil Nowakowski \\
        Rysunki i~wykresy: Kamil Ziemian, Paweł Węgrzyn, Wojciech Palacz \\
        Montaż: Agencja Filmowa Film \& Television Production~-- Zbigniew
        Masklak

      \end{flushleft}

    \end{textblock}

  \end{frame}





  \begin{frame}[standout]


    \begingroup

    \color{jFrametitleFGColor}

    #1

    \endgroup

  \end{frame}
}










% ---------------------------------------
% Commands for lectures "Geometria 3D dla twórców gier wideo"
% English version
% ---------------------------------------
% \newcommand{\FundingLogoWhitePictureEN}
% {./PresentationPictures/CommonPictures/logotypFundusze_biale_bez_tla2.pdf}
\newcommand{\FundingLogoColorPictureEN}
{./PresentationPictures/CommonPictures/European_Funds_color_EN.pdf}
% \newcommand{\EULogoWhitePictureEN}
% {./PresentationPictures/CommonPictures/logotypUE_biale_bez_tla2.pdf}
\newcommand{\EUSocialFundLogoColorPictureEN}
{./PresentationPictures/CommonPictures/EU_Social_Fund_color_EN.pdf}
% \newcommand{\ZintegrUJLogoWhitePictureEN}
% {./PresentationPictures/CommonPictures/zintegruj-logo-white.pdf}
\newcommand{\ZintegrUJLogoColorPictureEN}
{./PresentationPictures/CommonPictures/ZintegrUJ_color.pdf}
\newcommand{\JULogoColorPictureEN}
{./JagiellonianPictures/LogoJU_EN/LogoJU_A_color.pdf}



\newcommand{\GeometryThreeDSpecialBeginningSlideEN}{%
  \begin{frame}[standout]

    \begin{textblock}{11}(1,0.7)

      \begin{flushleft}

        \mdseries

        \footnotesize

        \color{jFrametitleFGColor}

        This content was created as part of a project co-financed by the
        European Union within the framework of the European Social Fund
        POWR.03.05.00-00-Z309/17-00.

      \end{flushleft}

    \end{textblock}





    \begin{textblock}{10}(0,2.2)

      \tikz \fill[color=jBackgroundStyleLight] (0,0) rectangle (12.8,-1.5);

    \end{textblock}


    \begin{textblock}{3.2}(0.7,2.45)

      \includegraphics[scale=0.3]{\FundingLogoColorPictureEN}

    \end{textblock}


    \begin{textblock}{2.5}(4.15,2.5)

      \includegraphics[scale=0.2]{\JULogoColorPictureEN}

    \end{textblock}


    \begin{textblock}{2.5}(6.35,2.4)

      \includegraphics[scale=0.1]{\ZintegrUJLogoColorPictureEN}

    \end{textblock}


    \begin{textblock}{4.2}(8.4,2.6)

      \includegraphics[scale=0.3]{\EUSocialFundLogoColorPictureEN}

    \end{textblock}

  \end{frame}
}



\newcommand{\GeometryThreeDTwoSpecialBeginningSlidesEN}{%
  \begin{frame}[standout]

    \begin{textblock}{11}(1,0.7)

      \begin{flushleft}

        \mdseries

        \footnotesize

        \color{jFrametitleFGColor}

        This content was created as part of a project co-financed by the
        European Union within the framework of the European Social Fund
        POWR.03.05.00-00-Z309/17-00.

      \end{flushleft}

    \end{textblock}





    \begin{textblock}{10}(0,2.2)

      \tikz \fill[color=jBackgroundStyleLight] (0,0) rectangle (12.8,-1.5);

    \end{textblock}


    \begin{textblock}{3.2}(0.7,2.45)

      \includegraphics[scale=0.3]{\FundingLogoColorPictureEN}

    \end{textblock}


    \begin{textblock}{2.5}(4.15,2.5)

      \includegraphics[scale=0.2]{\JULogoColorPictureEN}

    \end{textblock}


    \begin{textblock}{2.5}(6.35,2.4)

      \includegraphics[scale=0.1]{\ZintegrUJLogoColorPictureEN}

    \end{textblock}


    \begin{textblock}{4.2}(8.4,2.6)

      \includegraphics[scale=0.3]{\EUSocialFundLogoColorPictureEN}

    \end{textblock}

  \end{frame}





  \TitleSlideWithPicture
}



\newcommand{\GeometryThreeDSpecialEndingSlideEN}{%
  \begin{frame}[standout]

    \begin{textblock}{11}(1,0.7)

      \begin{flushleft}

        \mdseries

        \footnotesize

        \color{jFrametitleFGColor}

        This content was created as part of a project co-financed by the
        European Union within the framework of the European Social Fund
        POWR.03.05.00-00-Z309/17-00.

      \end{flushleft}

    \end{textblock}





    \begin{textblock}{10}(0,2.2)

      \tikz \fill[color=jBackgroundStyleLight] (0,0) rectangle (12.8,-1.5);

    \end{textblock}


    \begin{textblock}{3.2}(0.7,2.45)

      \includegraphics[scale=0.3]{\FundingLogoColorPictureEN}

    \end{textblock}


    \begin{textblock}{2.5}(4.15,2.5)

      \includegraphics[scale=0.2]{\JULogoColorPictureEN}

    \end{textblock}


    \begin{textblock}{2.5}(6.35,2.4)

      \includegraphics[scale=0.1]{\ZintegrUJLogoColorPictureEN}

    \end{textblock}


    \begin{textblock}{4.2}(8.4,2.6)

      \includegraphics[scale=0.3]{\EUSocialFundLogoColorPictureEN}

    \end{textblock}





    \begin{textblock}{11}(1,4)

      \begin{flushleft}

        \mdseries

        \footnotesize

        \RaggedRight

        \color{jFrametitleFGColor}

        The content of this lecture is made available under a~Creative
        Commons licence (\textsc{cc}), giving the author the credits
        (\textsc{by}) and putting an obligation to share on the same terms
        (\textsc{sa}). Figures and diagrams included in the lecture are
        authored by Paweł Węgrzyn et~al., and are available under the same
        license unless indicated otherwise.\\ The presentation uses the
        Beamer Jagiellonian theme based on Matthias Vogelgesang’s
        Metropolis theme, available under license \LaTeX{} Project
        Public License~1.3c at: \colorhref{https://github.com/matze/mtheme}
        {https://github.com/matze/mtheme}.

        Typographic design: Iwona Grabska-Gradzińska \\
        \LaTeX{} Typesetting: Kamil Ziemian \\
        Proofreading: Wojciech Palacz,
        Monika Stawicka \\
        3D Models: Dariusz Frymus, Kamil Nowakowski \\
        Figures and charts: Kamil Ziemian, Paweł Węgrzyn, Wojciech Palacz

      \end{flushleft}

    \end{textblock}

  \end{frame}
}



\newcommand{\GeometryThreeDTwoSpecialEndingSlidesEN}[1]{%
  \begin{frame}[standout]


    \begin{textblock}{11}(1,0.7)

      \begin{flushleft}

        \mdseries

        \footnotesize

        \color{jFrametitleFGColor}

        This content was created as part of a project co-financed by the
        European Union within the framework of the European Social Fund
        POWR.03.05.00-00-Z309/17-00.

      \end{flushleft}

    \end{textblock}





    \begin{textblock}{10}(0,2.2)

      \tikz \fill[color=jBackgroundStyleLight] (0,0) rectangle (12.8,-1.5);

    \end{textblock}


    \begin{textblock}{3.2}(0.7,2.45)

      \includegraphics[scale=0.3]{\FundingLogoColorPictureEN}

    \end{textblock}


    \begin{textblock}{2.5}(4.15,2.5)

      \includegraphics[scale=0.2]{\JULogoColorPictureEN}

    \end{textblock}


    \begin{textblock}{2.5}(6.35,2.4)

      \includegraphics[scale=0.1]{\ZintegrUJLogoColorPictureEN}

    \end{textblock}


    \begin{textblock}{4.2}(8.4,2.6)

      \includegraphics[scale=0.3]{\EUSocialFundLogoColorPictureEN}

    \end{textblock}





    \begin{textblock}{11}(1,4)

      \begin{flushleft}

        \mdseries

        \footnotesize

        \RaggedRight

        \color{jFrametitleFGColor}

        The content of this lecture is made available under a~Creative
        Commons licence (\textsc{cc}), giving the author the credits
        (\textsc{by}) and putting an obligation to share on the same terms
        (\textsc{sa}). Figures and diagrams included in the lecture are
        authored by Paweł Węgrzyn et~al., and are available under the same
        license unless indicated otherwise.\\ The presentation uses the
        Beamer Jagiellonian theme based on Matthias Vogelgesang’s
        Metropolis theme, available under license \LaTeX{} Project
        Public License~1.3c at: \colorhref{https://github.com/matze/mtheme}
        {https://github.com/matze/mtheme}.

        Typographic design: Iwona Grabska-Gradzińska \\
        \LaTeX{} Typesetting: Kamil Ziemian \\
        Proofreading: Wojciech Palacz,
        Monika Stawicka \\
        3D Models: Dariusz Frymus, Kamil Nowakowski \\
        Figures and charts: Kamil Ziemian, Paweł Węgrzyn, Wojciech Palacz

      \end{flushleft}

    \end{textblock}

  \end{frame}





  \begin{frame}[standout]

    \begingroup

    \color{jFrametitleFGColor}

    #1

    \endgroup

  \end{frame}
}



\newcommand{\GeometryThreeDSpecialEndingSlideVideoVerOneEN}{%
  \begin{frame}[standout]

    \begin{textblock}{11}(1,0.7)

      \begin{flushleft}

        \mdseries

        \footnotesize

        \color{jFrametitleFGColor}

        This content was created as part of a project co-financed by the
        European Union within the framework of the European Social Fund
        POWR.03.05.00-00-Z309/17-00.

      \end{flushleft}

    \end{textblock}





    \begin{textblock}{10}(0,2.2)

      \tikz \fill[color=jBackgroundStyleLight] (0,0) rectangle (12.8,-1.5);

    \end{textblock}


    \begin{textblock}{3.2}(0.7,2.45)

      \includegraphics[scale=0.3]{\FundingLogoColorPictureEN}

    \end{textblock}


    \begin{textblock}{2.5}(4.15,2.5)

      \includegraphics[scale=0.2]{\JULogoColorPictureEN}

    \end{textblock}


    \begin{textblock}{2.5}(6.35,2.4)

      \includegraphics[scale=0.1]{\ZintegrUJLogoColorPictureEN}

    \end{textblock}


    \begin{textblock}{4.2}(8.4,2.6)

      \includegraphics[scale=0.3]{\EUSocialFundLogoColorPictureEN}

    \end{textblock}





    \begin{textblock}{11}(1,4)

      \begin{flushleft}

        \mdseries

        \footnotesize

        \RaggedRight

        \color{jFrametitleFGColor}

        The content of this lecture is made available under a Creative
        Commons licence (\textsc{cc}), giving the author the credits
        (\textsc{by}) and putting an obligation to share on the same terms
        (\textsc{sa}). Figures and diagrams included in the lecture are
        authored by Paweł Węgrzyn et~al., and are available under the same
        license unless indicated otherwise.\\ The presentation uses the
        Beamer Jagiellonian theme based on Matthias Vogelgesang’s
        Metropolis theme, available under license \LaTeX{} Project
        Public License~1.3c at: \colorhref{https://github.com/matze/mtheme}
        {https://github.com/matze/mtheme}.

        Typographic design: Iwona Grabska-Gradzińska;
        \LaTeX{} Typesetting: Kamil Ziemian \\
        Proofreading: Wojciech Palacz,
        Monika Stawicka \\
        3D Models: Dariusz Frymus, Kamil Nowakowski \\
        Figures and charts: Kamil Ziemian, Paweł Węgrzyn, Wojciech
        Palacz \\
        Film editing: Agencja Filmowa Film \& Television Production~--
        Zbigniew Masklak

      \end{flushleft}

    \end{textblock}

  \end{frame}
}



\newcommand{\GeometryThreeDSpecialEndingSlideVideoVerTwoEN}{%
  \begin{frame}[standout]

    \begin{textblock}{11}(1,0.7)

      \begin{flushleft}

        \mdseries

        \footnotesize

        \color{jFrametitleFGColor}

        This content was created as part of a project co-financed by the
        European Union within the framework of the European Social Fund
        POWR.03.05.00-00-Z309/17-00.

      \end{flushleft}

    \end{textblock}





    \begin{textblock}{10}(0,2.2)

      \tikz \fill[color=jBackgroundStyleLight] (0,0) rectangle (12.8,-1.5);

    \end{textblock}


    \begin{textblock}{3.2}(0.7,2.45)

      \includegraphics[scale=0.3]{\FundingLogoColorPictureEN}

    \end{textblock}


    \begin{textblock}{2.5}(4.15,2.5)

      \includegraphics[scale=0.2]{\JULogoColorPictureEN}

    \end{textblock}


    \begin{textblock}{2.5}(6.35,2.4)

      \includegraphics[scale=0.1]{\ZintegrUJLogoColorPictureEN}

    \end{textblock}


    \begin{textblock}{4.2}(8.4,2.6)

      \includegraphics[scale=0.3]{\EUSocialFundLogoColorPictureEN}

    \end{textblock}





    \begin{textblock}{11}(1,4)

      \begin{flushleft}

        \mdseries

        \footnotesize

        \RaggedRight

        \color{jFrametitleFGColor}

        The content of this lecture is made available under a Creative
        Commons licence (\textsc{cc}), giving the author the credits
        (\textsc{by}) and putting an obligation to share on the same terms
        (\textsc{sa}). Figures and diagrams included in the lecture are
        authored by Paweł Węgrzyn et~al., and are available under the same
        license unless indicated otherwise.\\ The presentation uses the
        Beamer Jagiellonian theme based on Matthias Vogelgesang’s
        Metropolis theme, available under license \LaTeX{} Project
        Public License~1.3c at: \colorhref{https://github.com/matze/mtheme}
        {https://github.com/matze/mtheme}.

        Typographic design: Iwona Grabska-Gradzińska;
        \LaTeX{} Typesetting: Kamil Ziemian \\
        Proofreading: Wojciech Palacz,
        Monika Stawicka \\
        3D Models: Dariusz Frymus, Kamil Nowakowski \\
        Figures and charts: Kamil Ziemian, Paweł Węgrzyn, Wojciech
        Palacz \\
        Film editing: IMAVI -- Joanna Kozakiewicz, Krzysztof Magda, Nikodem
        Frodyma

      \end{flushleft}

    \end{textblock}

  \end{frame}
}



\newcommand{\GeometryThreeDSpecialEndingSlideVideoVerThreeEN}{%
  \begin{frame}[standout]

    \begin{textblock}{11}(1,0.7)

      \begin{flushleft}

        \mdseries

        \footnotesize

        \color{jFrametitleFGColor}

        This content was created as part of a project co-financed by the
        European Union within the framework of the European Social Fund
        POWR.03.05.00-00-Z309/17-00.

      \end{flushleft}

    \end{textblock}





    \begin{textblock}{10}(0,2.2)

      \tikz \fill[color=jBackgroundStyleLight] (0,0) rectangle (12.8,-1.5);

    \end{textblock}


    \begin{textblock}{3.2}(0.7,2.45)

      \includegraphics[scale=0.3]{\FundingLogoColorPictureEN}

    \end{textblock}


    \begin{textblock}{2.5}(4.15,2.5)

      \includegraphics[scale=0.2]{\JULogoColorPictureEN}

    \end{textblock}


    \begin{textblock}{2.5}(6.35,2.4)

      \includegraphics[scale=0.1]{\ZintegrUJLogoColorPictureEN}

    \end{textblock}


    \begin{textblock}{4.2}(8.4,2.6)

      \includegraphics[scale=0.3]{\EUSocialFundLogoColorPictureEN}

    \end{textblock}





    \begin{textblock}{11}(1,4)

      \begin{flushleft}

        \mdseries

        \footnotesize

        \RaggedRight

        \color{jFrametitleFGColor}

        The content of this lecture is made available under a Creative
        Commons licence (\textsc{cc}), giving the author the credits
        (\textsc{by}) and putting an obligation to share on the same terms
        (\textsc{sa}). Figures and diagrams included in the lecture are
        authored by Paweł Węgrzyn et~al., and are available under the same
        license unless indicated otherwise.\\ The presentation uses the
        Beamer Jagiellonian theme based on Matthias Vogelgesang’s
        Metropolis theme, available under license \LaTeX{} Project
        Public License~1.3c at: \colorhref{https://github.com/matze/mtheme}
        {https://github.com/matze/mtheme}.

        Typographic design: Iwona Grabska-Gradzińska;
        \LaTeX{} Typesetting: Kamil Ziemian \\
        Proofreading: Wojciech Palacz,
        Monika Stawicka \\
        3D Models: Dariusz Frymus, Kamil Nowakowski \\
        Figures and charts: Kamil Ziemian, Paweł Węgrzyn, Wojciech
        Palacz \\
        Film editing: Agencja Filmowa Film \& Television Production~--
        Zbigniew Masklak \\
        Film editing: IMAVI -- Joanna Kozakiewicz, Krzysztof Magda, Nikodem
        Frodyma

      \end{flushleft}

    \end{textblock}

  \end{frame}
}



\newcommand{\GeometryThreeDTwoSpecialEndingSlidesVideoVerOneEN}[1]{%
  \begin{frame}[standout]

    \begin{textblock}{11}(1,0.7)

      \begin{flushleft}

        \mdseries

        \footnotesize

        \color{jFrametitleFGColor}

        This content was created as part of a project co-financed by the
        European Union within the framework of the European Social Fund
        POWR.03.05.00-00-Z309/17-00.

      \end{flushleft}

    \end{textblock}





    \begin{textblock}{10}(0,2.2)

      \tikz \fill[color=jBackgroundStyleLight] (0,0) rectangle (12.8,-1.5);

    \end{textblock}


    \begin{textblock}{3.2}(0.7,2.45)

      \includegraphics[scale=0.3]{\FundingLogoColorPictureEN}

    \end{textblock}


    \begin{textblock}{2.5}(4.15,2.5)

      \includegraphics[scale=0.2]{\JULogoColorPictureEN}

    \end{textblock}


    \begin{textblock}{2.5}(6.35,2.4)

      \includegraphics[scale=0.1]{\ZintegrUJLogoColorPictureEN}

    \end{textblock}


    \begin{textblock}{4.2}(8.4,2.6)

      \includegraphics[scale=0.3]{\EUSocialFundLogoColorPictureEN}

    \end{textblock}





    \begin{textblock}{11}(1,4)

      \begin{flushleft}

        \mdseries

        \footnotesize

        \RaggedRight

        \color{jFrametitleFGColor}

        The content of this lecture is made available under a Creative
        Commons licence (\textsc{cc}), giving the author the credits
        (\textsc{by}) and putting an obligation to share on the same terms
        (\textsc{sa}). Figures and diagrams included in the lecture are
        authored by Paweł Węgrzyn et~al., and are available under the same
        license unless indicated otherwise.\\ The presentation uses the
        Beamer Jagiellonian theme based on Matthias Vogelgesang’s
        Metropolis theme, available under license \LaTeX{} Project
        Public License~1.3c at: \colorhref{https://github.com/matze/mtheme}
        {https://github.com/matze/mtheme}.

        Typographic design: Iwona Grabska-Gradzińska;
        \LaTeX{} Typesetting: Kamil Ziemian \\
        Proofreading: Wojciech Palacz,
        Monika Stawicka \\
        3D Models: Dariusz Frymus, Kamil Nowakowski \\
        Figures and charts: Kamil Ziemian, Paweł Węgrzyn,
        Wojciech Palacz \\
        Film editing: Agencja Filmowa Film \& Television Production~--
        Zbigniew Masklak

      \end{flushleft}

    \end{textblock}

  \end{frame}





  \begin{frame}[standout]


    \begingroup

    \color{jFrametitleFGColor}

    #1

    \endgroup

  \end{frame}
}



\newcommand{\GeometryThreeDTwoSpecialEndingSlidesVideoVerTwoEN}[1]{%
  \begin{frame}[standout]

    \begin{textblock}{11}(1,0.7)

      \begin{flushleft}

        \mdseries

        \footnotesize

        \color{jFrametitleFGColor}

        This content was created as part of a project co-financed by the
        European Union within the framework of the European Social Fund
        POWR.03.05.00-00-Z309/17-00.

      \end{flushleft}

    \end{textblock}





    \begin{textblock}{10}(0,2.2)

      \tikz \fill[color=jBackgroundStyleLight] (0,0) rectangle (12.8,-1.5);

    \end{textblock}


    \begin{textblock}{3.2}(0.7,2.45)

      \includegraphics[scale=0.3]{\FundingLogoColorPictureEN}

    \end{textblock}


    \begin{textblock}{2.5}(4.15,2.5)

      \includegraphics[scale=0.2]{\JULogoColorPictureEN}

    \end{textblock}


    \begin{textblock}{2.5}(6.35,2.4)

      \includegraphics[scale=0.1]{\ZintegrUJLogoColorPictureEN}

    \end{textblock}


    \begin{textblock}{4.2}(8.4,2.6)

      \includegraphics[scale=0.3]{\EUSocialFundLogoColorPictureEN}

    \end{textblock}





    \begin{textblock}{11}(1,4)

      \begin{flushleft}

        \mdseries

        \footnotesize

        \RaggedRight

        \color{jFrametitleFGColor}

        The content of this lecture is made available under a Creative
        Commons licence (\textsc{cc}), giving the author the credits
        (\textsc{by}) and putting an obligation to share on the same terms
        (\textsc{sa}). Figures and diagrams included in the lecture are
        authored by Paweł Węgrzyn et~al., and are available under the same
        license unless indicated otherwise.\\ The presentation uses the
        Beamer Jagiellonian theme based on Matthias Vogelgesang’s
        Metropolis theme, available under license \LaTeX{} Project
        Public License~1.3c at: \colorhref{https://github.com/matze/mtheme}
        {https://github.com/matze/mtheme}.

        Typographic design: Iwona Grabska-Gradzińska;
        \LaTeX{} Typesetting: Kamil Ziemian \\
        Proofreading: Wojciech Palacz,
        Monika Stawicka \\
        3D Models: Dariusz Frymus, Kamil Nowakowski \\
        Figures and charts: Kamil Ziemian, Paweł Węgrzyn,
        Wojciech Palacz \\
        Film editing: IMAVI -- Joanna Kozakiewicz, Krzysztof Magda, Nikodem
        Frodyma

      \end{flushleft}

    \end{textblock}

  \end{frame}





  \begin{frame}[standout]


    \begingroup

    \color{jFrametitleFGColor}

    #1

    \endgroup

  \end{frame}
}



\newcommand{\GeometryThreeDTwoSpecialEndingSlidesVideoVerThreeEN}[1]{%
  \begin{frame}[standout]

    \begin{textblock}{11}(1,0.7)

      \begin{flushleft}

        \mdseries

        \footnotesize

        \color{jFrametitleFGColor}

        This content was created as part of a project co-financed by the
        European Union within the framework of the European Social Fund
        POWR.03.05.00-00-Z309/17-00.

      \end{flushleft}

    \end{textblock}





    \begin{textblock}{10}(0,2.2)

      \tikz \fill[color=jBackgroundStyleLight] (0,0) rectangle (12.8,-1.5);

    \end{textblock}


    \begin{textblock}{3.2}(0.7,2.45)

      \includegraphics[scale=0.3]{\FundingLogoColorPictureEN}

    \end{textblock}


    \begin{textblock}{2.5}(4.15,2.5)

      \includegraphics[scale=0.2]{\JULogoColorPictureEN}

    \end{textblock}


    \begin{textblock}{2.5}(6.35,2.4)

      \includegraphics[scale=0.1]{\ZintegrUJLogoColorPictureEN}

    \end{textblock}


    \begin{textblock}{4.2}(8.4,2.6)

      \includegraphics[scale=0.3]{\EUSocialFundLogoColorPictureEN}

    \end{textblock}





    \begin{textblock}{11}(1,4)

      \begin{flushleft}

        \mdseries

        \footnotesize

        \RaggedRight

        \color{jFrametitleFGColor}

        The content of this lecture is made available under a Creative
        Commons licence (\textsc{cc}), giving the author the credits
        (\textsc{by}) and putting an obligation to share on the same terms
        (\textsc{sa}). Figures and diagrams included in the lecture are
        authored by Paweł Węgrzyn et~al., and are available under the same
        license unless indicated otherwise. \\ The presentation uses the
        Beamer Jagiellonian theme based on Matthias Vogelgesang’s
        Metropolis theme, available under license \LaTeX{} Project
        Public License~1.3c at: \colorhref{https://github.com/matze/mtheme}
        {https://github.com/matze/mtheme}.

        Typographic design: Iwona Grabska-Gradzińska;
        \LaTeX{} Typesetting: Kamil Ziemian \\
        Proofreading: Leszek Hadasz, Wojciech Palacz,
        Monika Stawicka \\
        3D Models: Dariusz Frymus, Kamil Nowakowski \\
        Figures and charts: Kamil Ziemian, Paweł Węgrzyn,
        Wojciech Palacz \\
        Film editing: Agencja Filmowa Film \& Television Production~--
        Zbigniew Masklak \\
        Film editing: IMAVI -- Joanna Kozakiewicz, Krzysztof Magda, Nikodem
        Frodyma


      \end{flushleft}

    \end{textblock}

  \end{frame}





  \begin{frame}[standout]


    \begingroup

    \color{jFrametitleFGColor}

    #1

    \endgroup

  \end{frame}
}











% ---------------------------------------
% Packages, libraries and their configuration
% ---------------------------------------
\usepackage{mathcommands}





% ---------------------------------------
% Configuration for this particular presentation
% ---------------------------------------










% ---------------------------------------------------------------------
\title{Niespodziewane teoretyczne trudności kwantowej teorii pola w~OTW}

\author{Kamil Ziemian \\
  \texttt{kziemianfvt@gmail.com}}


% \institute{Jagiellonian University in~Cracow}

% \date[23 November 2018]{Seminar~of Field Theory Department \\
%   23 November 2018}
% --------------------------------------------------------------------










% ####################################################################
% Początek dokumentu
\begin{document}
% ####################################################################





% Wyrównanie do lewej z łamaniem wyrazów

\RaggedRight





% ######################################
\maketitle % Tytuł całego tekstu
% ######################################





% ######################################
\begin{frame}
  \frametitle{Spis treści}


  \tableofcontents % Spis treści

\end{frame}
% ######################################




% ##################
\begin{frame}
  \frametitle{Problem przed którym stajemy}


  Będziemy rozważać problem sformułowania teorii pola kwantowego,
  żyjącego na ustalonej, klasycznej, zakrzywionej czasoprzestrzeni.
  Pomijamy więc zupełnie jego wkład do pola grawitacyjnego.

  Jest to umiarkowanie ambitny cel. Lecz i~on napotyka na szereg
  głębokich problemów.

\end{frame}
% ##################





% ##################
\begin{frame}
  \frametitle{Będę upraszczał}


  Dzisiaj nie ma ani czas ani prelegenta, by zagłębiać~się we wszystko
  z~pełną precyzją. Dlatego część rzeczy pominę, część uproszczę mniej
  lub bardziej.

\end{frame}
% ##################





% ##################
\begin{frame}
  \frametitle{Uwaga techniczna}


  Istnieje tylko \alert{jedna} przestrzeń Hilberta $\Hcal$ dla
  nierelatywistycznej mechaniki kwantowej. Dla kwantowej teorii pola
  jest ich nieprzeliczalnie wiele, albo więcej.

  Trochę bardziej ściśle. Twierdzenie Stone’a-von Neumanna głosi, że
  mechanika kwantowa żyje w~ośrodkowej przestrzeni Hilberta, a~ta jest
  tylko jedna (z~dokładnością do transformacji unitarnych).

  Kwantowej teorii pola nie można sformułować w~ośrodkowej przestrzeni
  Hilberta, a~przestrzeni nieośrodkowych znamy już nieprzeliczalnie
  wiele. Co więcej skonstruowano już nieprzeliczalnie wiele kwantowych
  teorii pola żyjących w~unitarnie nierównoważnych przestrzeniach
  Hilberta.

\end{frame}
% ##################





% ##################
\begin{frame}
  \frametitle{Pole elektromagnetyczne}


  Elektrodynamika Clerka Maxwella bez ładunków i~prądów jest teorią
  czysto geometryczną, więc przeniesienie jej do OTW nie stwarza
  problemu.

  Pole zapisujemy za pomocą tensora Clerka Maxwell.
  \begin{equation}
    \label{eq:Niespodziewane-teoretyczne-01}
    F_{ \mu \nu }( x )
    =
    \begin{pmatrix}
      \hphantom{-} 0 & \frac{ 1 }{ c } E_{ x }( x ) & \frac{ 1 }{ c }
      E_{ y }( x )
      & \frac{ 1 }{ c } E_{ z }( x ) \\
      -\frac{ 1 }{ c } E_{ x }( x ) & \hphantom{-} 0 & -B_{ z }( x )
      & \hphantom{-} B_{ y }( x ) \\
      -\frac{ 1 }{ c } E_{ y }( x ) & \hphantom{-} B_{ z }( x )
      & \hphantom{-} 0 & -B_{ z }( x ) \\
      -\frac{ 1 }{ c } E_{ z }( x ) & -B_{ y }( x ) & \hphantom{-} B_{
        x }( x ) & \hphantom{-} 0
    \end{pmatrix}
  \end{equation}

  Jeśli wprowadzimy czteroformę pola
  $F( x ) = F_{ \mu \nu }( x ) \, dx^{ \mu } \wedge dx^{ \nu }$
  i~czteroformę potencjału $A( x ) = A_{ \mu }( x ) \, dx^{ \mu }$ to
  równania Clerka Maxwella przyjmują formę (trochę upraszczam)
  \begin{equation}
    \label{eq:Niespodziewane-teoretyczne-02}
    F( x ) = dA( x ), \quad
    dF( x ) = 0.
  \end{equation}

\end{frame}
% ##################





% ##################
\begin{frame}
  \frametitle{Uproszczona kwantyzacja w~czasoprzestrzeni Minkowskiego}


  1. Wybierz inercjalny układ współrzędnych.

  2. Wypisz w~nim równania Clerka Maxwella.
  \begin{equation}
    \label{eq:Niespodziewane-teoretyczne-03}
    \square \, A_{ \mu }( x ) = 0, \quad
    \partial^{ \nu } A_{ \nu } = 0
  \end{equation}

  3. Zapisz rozwiązanie równań Clerka Maxwella jako
  \begin{equation}
    \label{eq:Niespodziewane-teoretyczne-04}
    \begin{split}
      A( x )
      &=
        \int d^{ 3 }k \, \sum_{ \mu = \pm 1 } \frac{ 1 }{ \omega( \veck ) }
        \big( \vec{ e }^{ ( \mu ) }( \veck ) a_{ \veck }^{ ( \mu ) }( t )
        e^{ -i \omega( \veck ) t + i \veck \cdot \vecx } \, + \\
      &\hphantom{=} + \, \overline{ \vece }^{ ( \mu ) }( \veck )
        \overline{ a }_{ \veck }^{ ( \mu ) }( t )
        e^{ i \omega( \veck ) t - i \veck \cdot \vecx } \big)
    \end{split}
  \end{equation}

  4. Podnieś $A( x )$, $a_{ \veck }^{ ( \mu ) }$ do rangi operatorów.

\end{frame}
% ##################





% ##################
\begin{frame}
  \frametitle{Uproszczona kwantyzacja w~czasoprzestrzeni Minkowskiego}


  \begin{equation}
    \label{eq:Niespodziewane-teoretyczne-05}
    \begin{split}
      \widehat{A}( x )
      &=
        \int d^{ 3 }k \, \sum_{ \mu = \pm 1 } \frac{ 1 }{ \omega( \veck ) }
        \big( \vec{ e }^{ ( \mu ) }( \veck )
        \widehat{a}_{ \veck }^{ ( \mu ) }( t )
        e^{ -i \omega( \veck ) t + i \veck \cdot \vecx } \, + \\
      &\hphantom{=} + \, \overline{ \vece }^{ ( \mu ) }( \veck )
        \widehat{a}_{ \veck }^{ \dagger \, ( \mu ) }( t )
        e^{ i \omega( \veck ) t - i \veck \cdot \vecx } \big)
    \end{split}
  \end{equation}

  5. Zinterpretuj operator przy
  $\exp( -i \omega( \veck ) t + i \veck \cdot \vecx )$
  jako operator anihilacji. Znajdź więc stan, taki że
  \begin{equation}
    \label{eq:Niespodziewane-teoretyczne-06}
    \widehat{a}_{ \vec{ k } }^{ ( \mu ) }( t ) | 0 \rangle = 0, \quad
    \forall \vec{ k }, \mu.
  \end{equation}

  6. Zbuduj przestrzeń Hilberta ze stanów zawierających 0, 1, 2, 3,
  \ldots, cząstek.
  \begin{equation}
    \label{eq:Niespodziewane-teoretyczne-07}
    | 0 \rangle, \widehat{a}^{ \dagger \, ( \mu ) }_{ \veck }( t ) | 0 \rangle,
    \widehat{a}^{ \dagger \, ( \mu ) }_{ \veck }( t )
    \widehat{a}^{ \dagger \, ( \nu ) }_{ \veck' }( t ) | 0 \rangle,
    \widehat{a}^{ \dagger \, ( \mu ) }_{ \veck }( t )
    \widehat{a}^{ \dagger \, ( \nu ) }_{ \veck' }( t )
    \widehat{a}^{ \dagger \, ( \rho ) }_{ \veck'' }( t ) | 0 \rangle, \ldots
  \end{equation}

\end{frame}
% ##################





% ##################
\begin{frame}
  \frametitle{Co może pójść nie tak?}


  \begin{equation}
    \label{eq:Niespodziewane-teoretyczne-08}
    \widehat{a}_{ \veck }^{ ( \mu ) }( t ) | 0 \rangle = 0, \quad
    \forall \veck, \mu.
  \end{equation}
  Jest zbyt wiele stanów spełniających tą relację. Wybór każdego
  z~nich daje inną przestrzeń Hilberta budowanej wedle przepisu
  \begin{equation}
    \label{eq:Niespodziewane-teoretyczne-09}
    | 0 \rangle,
    \widehat{a}^{ \dagger \, ( \mu ) }_{ \veck }( t ) | 0 \rangle,
    \widehat{a}^{ \dagger \, ( \mu ) }_{ \veck }( t )
    \widehat{a}^{ \dagger \, ( \nu ) }_{ \veck' }( t ) | 0 \rangle,
    \widehat{a}^{ \dagger \, ( \mu ) }_{ \veck }( t )
    \widehat{a}^{ \dagger \, ( \nu ) }_{ \veck' }( t )
    \widehat{a}^{ \dagger \, ( \rho ) }_{ \veck'' }( t ) | 0 \rangle, \ldots
  \end{equation}
  Każda taka przestrzeń to inna fizyka.

  Ale istnieje tylko \alert{jeden} stan $| 0 \rangle$ który jest taki
  sam w~każdym inercjalnym układzie współrzędnych. Inaczej mówiąc,
  który spełnia relację:
  \begin{equation}
    \label{eq:Niespodziewane-teoretyczne-10}
    U( L, \veca ) | 0 \rangle = | 0 \rangle,
  \end{equation}
  gdzie $L$ to transformacja Lorentza, a~$\veca$ to czterowektor
  translacji.

\end{frame}
% ##################





% ##################
\begin{frame}
  \frametitle{Trochę szczegółów}


  Pole kwantowe $\widehat{A}( x )$ wygląda tak samo w~każdym
  inercjalnym układzie współrzędnych i~istnieje tylko jeden stan
  $| 0 \rangle$, który wygląda tak samo w~każdym z~tych układów.

  Mówiąc prościej, może nawet za prosto. Jest tylko jeden „stan
  próżni”, który jest pusty w~\alert{każdym} inercjalnym układzie
  odniesienia. Pozostałe są puste w~jednych, a~zawierają cząstki
  w~innych. „Dziwne stany próżni”.

  Dobrze nam znane pojęcie cząstki jest uratowane!

\end{frame}
% ##################





% ##################
\begin{frame}
  \frametitle{Teraz widać co się może popsuć}


  Na zakrzywionej czasoprzestrzeni nie można liczyć na istnienie grupy
  symetria tak dużej, jak grupa Poincar\'{e}’ego. Tym samym nie jest
  możliwe znalezienie stanu próżni który jest wszędzie pusty. Bardzo
  dziwne.

  W~zakrzywionej czasoprzestrzeni takiego uniwersalnego stanu próżni
  nie da~się zwykle znaleźć, w~jednych układach będzie pusty w~innych
  nie. To w~istocie leży u~podstaw słynnego promieniowania czarnych
  dziur (miało dobry marketing).

\end{frame}
% ##################





% ##################
\begin{frame}
  \frametitle{Dlaczego nie można liczyć na uniwersalną próżnię?}


  W~czasoprzestrzenni Minkowskiego zbiór operatorów anihilacji
  $\widehat{a}^{ ( \mu ) }_{ \veck }( t )$ wszystkich układów
  inercjalnych jest stosunkowo mały. Ogranicza go warunek równoważny
  niezmienniczości próżni:
  \begin{equation}
    \label{eq:Niespodziewane-teoretyczne-11}
    \widehat{U}^{ \dagger }( L, a ) \, \widehat{a}^{ ( \mu ) }_{ \veck }( t ) \,
    \widehat{U}( L, a )
    =
    \widehat{a}^{ ( \mu ) }_{ \veck' }( t' ),
  \end{equation}
  przy czym $( L, a )$ przekształca $t$, $\veck$ na $t'$, $\veck'$.

  W~zakrzywionej czasoprzestrzenie zwykle nie będzie żadnego takiego
  ograniczenia, więc klasa operatorów anihilacji będzie zbyt duża, by
  znaleźć dla nich wspólną próżnię, różną od wektora $0$.

\end{frame}
% ##################





% ##################
\begin{frame}
  \frametitle{Lista problemów}


  \begin{itemize}
    \RaggedRight

  \item Brak dobrego stanu próżni $| 0 \rangle$.

  \item Brak dobrej przestrzeni Hilberta zbudowanej ze stanu próżni.

  \item Bez dobrej przestrzeni Hilberta i grupy Poincar\'{e}’ego nie
    podamy stanów cząstek swobodnych.

  \item Jak nie ma cząstek swobodnych, to nie ma też macierzy $S$.

  \item W~skutek poprzednich niewiadoma jak wybrać propagatora
    Feynmana.

  \end{itemize}

\end{frame}
% ##################





% ##################
\begin{frame}
  \frametitle{Uwagi}


  W~wystąpieniu postaram się ominąć wszystkie sprawy techniczne,
  które nie~są potrzebne do zrozumienia teorii. Z~tego względu część
  wzorów nie jest „dobrze określona”, ale~wszystkie te problemy
  można obejść.

  Notacja stosowana w~cytowanych pracach jest dość hermetyczna. Będę
  wypisywał formuły możliwie jawnie.

  Z~podobnego ducha wyrasta podejście do QFT na~zakrzywionych
  czasoprzestrzeniach oparte o~OPE. Zagadnienie relacji
  tych sformułowań nie będzie poruszane.

\end{frame}
% ##################





% ##################
\begin{frame}
  \frametitle{Uwagi}


  Najbardziej interesujący aspekt prezentowanej teorii,
  czyli przepisy rachunkowe~są obecnie dynamiczne rozwijane tak,
  że~mówienie o~nich w~chwili obecnej musi chyba oscylować między
  Scyllą przesadnie skomplikowanych rachunków, a~Charybdą być może
  płonnych nadziei.

  Zagadnienie rozbieżności podczerwonych jest będzie zupełnie
  pominięte w~rozważaniach.

\end{frame}
% ##################





% ##################
\begin{frame}
  \frametitle{Motywacje}


  Obecny stan teorii fizycznych wskazuje na potrzebę istnienia
  kwantowej teorii grawitacji. Jednak pomimo istnienia podejść
  takich jak teoria strun czy grawitacja pętlowa, oraz opisania
  pewnych efektów kwantowych na zakrzywionych czasoprzestrzeniach
  (promieniowanie Hawkinga), zadowalająca teoria nie jest znana.

  Zamiast radykalnego przebudowania teorii, można pójść bardziej
  tradycyjną drogą i~spróbować najpierw przenieść QFT na zadane
  zakrzywione czasoprzestrzenie.

\end{frame}
% ##################





% ##################
\begin{frame}
  \frametitle{Algebraiczna QFT $\MathTextFrametitleFGColor{\subset}$
    Lokalna Fizyka Kwantowa}


  Jest to pewne podejście, które pozawala obejść trudności o~których
  mówiłem. Mówię akurat o~nim z~prostego powodu: tylko na nim~się coś
  znam.

  Wadą tego podejścia jest to, że~nie udało~się na razie wyciągnąć
  z~niego konkretnych wniosków, które dałoby~się porównać
  z~eksperymentem. Tak było przynajmniej do 2018 roku, bo nowszych
  źródeł nie udało mi~się przejrzeć.


  Podejście to zostało zaproponowane przez Rudolfa Haaga (1922--2016)
  w~latach 50 i~60 XX wieku. U~podstaw leżała chęć „redukcji teorii
  do~wielkości obserwowalnych„ i~ograniczyć „nadmiarowość”
  przestrzeni Hilberta w~teoriach z~regułami nadwyboru. Przykładowo
  jak mamy regułę nadwyboru ładunku elektrycznego.

\end{frame}
% ##################





% ##################
\begin{frame}
  \frametitle{Algebraiczna QFT $\MathTextFrametitleFGColor{\subset}$
    Lokalna Fizyka Kwantowa}


  Konsekwencje
  \begin{itemize}
    \RaggedRight

  \item Lokalność („zasada-bliskiego-działania”).

  \item Algebry z~inwolucją jako centralny obiekt teorii.

  \end{itemize}

\end{frame}
% ##################





% ##################
\begin{frame}
  \frametitle{Algebraiczna QFT $\MathTextFrametitleFGColor{\subset}$
    Lokalna Fizyka Kwantowa}


  Rozważmy czasoprzestrzeń $M$, może to być czasoprzestrzeń
  Minkowskiego albo zakrzywiona.

  Za podstawową zjawiska fizycznego uznajemy, że~zdarzyło się
  „w~jakimś miejscu i~jakimś czasie”. Tą własność lokalności zjawisk
  („zasada-bliskiego-działania”), formalizujemy następująco.

  Niech $O$ będzie otwartym podzbiorem $M$, którego domknięcie jest
  zwarte. Niech $\Acal( O )$ będzie zbiorem wszystkim obserwabli
  = rodzajów zjawisk które możemy zmierzyć w~zbiorze $O$.

  Przyjmujemy, że~$\Acal( O )$ tworzy algebrę z~gwiazdką albo
  $C^{ * }$-algebrę.

\end{frame}
% ##################





% ##################
\begin{frame}
  \frametitle{Algebraiczna QFT $\MathTextFrametitleFGColor{\subset}$
    Lokalna Fizyka Kwantowa}


  Spróbuje to uczynić bardziej dotykalnym. W~mechanice kwantowej
  w~reprezentacji położeń mamy $X = x$
  i~$P = -i \, \hbar \, \partial_{ x }$. Rachunkowo łatwo (przymykając
  oczy na matematykę) pokazać:
  \begin{equation}
    \label{eq:Niespodziewane-teoretyczne-12}
    X P - P X = i \, \hbar \, I.
  \end{equation}
  Wybór tych operatorów nie jest jednak jednoznaczny, możemy bowiem
  wziąć $X = i \, \hbar \, \partial_{ p }$ i~$P = p$. Oba opisy są
  równoważne.

  Możemy spojrzeć na to w~następujący sposób: podstawowym obiektem
  jest algebra definiowana przez relację
  \eqref{eq:Niespodziewane-teoretyczne-12}, którą można ją
  przetłumaczyć na przestrzenie Hilberta na wiele różnych sposobów.

\end{frame}
% ##################





% ##################
\begin{frame}
  \frametitle{Algebraiczna QFT $\MathTextFrametitleFGColor{\subset}$
    Lokalna Fizyka Kwantowa}


  Rozważmy przestrzeń dualną do $\Acal( O )$, jako do
  przestrzeni wektorowej: $\Acal^{ * }( O )$.

  „Dobre” elementy $\Acal^{ * }( O )$ mają interpretację
  stanów układu fizycznego. Wynika to z~tego, że jeśli
  $a \in \Acal( O )$, $\omega \in \Acal^{ * }( O )$ to
  $\omega( a )$ jest liczbą, którą rozumiemy jak wartość oczekiwaną
  pomiarów zjawiska $a$, np. prędkości elektronu, w~stanie $\omega$.
  \begin{equation}
    \label{eq:Niespodziewane-teoretyczne-13}
    \langle a \rangle_{ \omega } = \omega( a )
  \end{equation}

  Pełen rozkład prawdopodobieństwa da się odtworzyć jeśli znamy
  wszystkie momenty: $\langle a^{ n } \rangle_{ \omega }$,
  $n = 1, 2, 3, \ldots$.

  Jeśli mamy stan $\omega$ to konstrukcje takie jak GNS mówią nam, że
  możemy algebrę $\Acal( O )$ przenieść na przestrzeń Hilberta,
  tak że $a \in \Acal( O )$ stają się operatorami,
  a~$\omega \in \Acal^{ * }( O )$ wektorami stanu.

\end{frame}
% ##################





% ##################
\begin{frame}
  \frametitle{Algebraiczna QFT $\MathTextFrametitleFGColor{\subset}$
    Lokalna Fizyka Kwantowa}


  \textbf{Lokalne algebry.}
  Każdemu „ograniczonemu” obszarowi $\Ocal \subset M$ przypisujemy algebrę
  $\Ufrak( \Ocal ) \subset \Ufrak( M )$~-- zbiór obserwabli które można w~nim
  zmierzyć. Jeśli dwa obszary $\Ocal_{ 1 }$ i~$\Ocal_{ 2 }$ są rozdzielone
  przestrzennie to elementy ich algebr komutują. Plus pewne
  ograniczenia fizyczne.

  \textbf{Konsekwencje.}
  Formalizm ten daje na tyle dużą swobodę, że~pomimo sporych technicznych
  trudności wymagających „uogólnienia symetrii teorii” formalizm ten ma
  sens na interesujących nas czasoprzestrzeniach.

\end{frame}
% ##################





% ##################
\begin{frame}
  \frametitle{Przyczynowość}


  Niech zbiory $O_{ 1 }$ i $O_{ 2 }$ będą przestrzenie rozdzielone.
  Istnieje wspólna algebra zawierająca $\Acal( O_{ 1 } )$
  i~$\Acal( O_{ 1 } )$, do tego mnożenie argumentów między tymi
  algebrami jest przemienne. Równoważnie, komutator operatorów znika.

  W~podobny sposób da się zakodować inne własności fizyczne w~tych
  lokalnych algebrach.

\end{frame}
% ##################





% ###################
\begin{frame}
  \frametitle{Konstrukcja teorii}


  Element klasyczny
  \begin{equation}
    \label{eq:Niespodziewane-teoretyczne-14}
    S_{ \mathrm{Free} }( \varphi ) =
    \int d \mathrm{vol}( x )\, \big( g_{ \mu \nu }( x ) \nabla^{ \mu } \varphi( x )
    \nabla^{ \nu } \varphi( x ) - ( m^{ 2 } + \xi R ) \varphi^{ 2 }( x ) \big),
  \end{equation}
  prowadzi do równania:
  \begin{equation}
    \label{eq:Niespodziewane-teoretyczne-15}
    ( \Box  + m^{ 2 } + \xi R ) \varphi = 0.
  \end{equation}
  Przy przyjętych założeniach ma ono dobrze określony propagator
  przyśpieszony i~opóźniony $\Delta^{ A, R }$.

  Możemy teraz utworzyć:
  \begin{subequations}
    \begin{align}
      \label{eq:Niespodziewane-teoretyczne-16-A}
      \Delta( x, y )
      &= \Delta^{ A }( x, y ) - \Delta^{ R }( x, y ), \\
      \label{eq:Niespodziewane-teoretyczne-16-B}
      \Delta( x, y )
      &= \frac{ 1 }{ 2 } \big( \Delta( x, y ) + i H( x, y ) \big).
    \end{align}
  \end{subequations}

\end{frame}
% ##################





% ##################
\begin{frame}
  \frametitle{Uproszczenie formalne}


  Niech $F$ będzie, w~ogólności nieliniowym, funkcjonałem:
  \begin{equation}
    \label{eq:Niespodziewane-teoretyczne-17}
    F: \Ccal^{ \infty }( M ) \rightarrow \Cbb,
  \end{equation}
  takimi, że
  \begin{equation}
    \label{eq:Niespodziewane-teoretyczne-18}
    \frac{ d }{ d \lambda } F( \varphi + \lambda h ) = \langle F^{ ( 1 ) }( \varphi ), h \rangle =
    \int d \mathrm{vol}( x )\, \big( F^{ ( 1 ) }( \varphi ) \big)( x ) h( x ),
  \end{equation}
  i~analogicznie dla wyższych pochodnych.

  Każdy funkcjonał utożsamiamy z~formalnym szeregiem potęgowym:
  \begin{equation}
    \label{eq:Niespodziewane-teoretyczne-19}
    F :=
    \sum\limits_{ i = 0 }^{ \infty } \frac{ i^{ n } \, \hbar^{ n } }{ n! } F^{ ( n ) }.
  \end{equation}

\end{frame}
% ##################





% ##################
\begin{frame}
  \frametitle{Konstrukcja teorii swobodnej}


  Wówczas
  \begin{equation}
    \label{eq:Niespodziewane-teoretyczne-20}
    \begin{split}
      \left\langle F^{ ( 1 ) }, \frac{ 1 }{ 2 } \Delta G^{ ( 1 ) } \right\rangle( \varphi )
      &=
        \frac{ 1 }{ 2 }
        \int d \mathrm{vol}( x ) \, d \mathrm{vol}( y ) \,
        F^{ ( 1 ) }( \varphi )( x ) \\
      &\hspace{3em} \times \Delta( x, y ) G^{ ( 1 ) }( \varphi )( y ).
    \end{split}
  \end{equation}
  ma przynajmniej formalny sense. Analogicznie robimy dla wyższych
  pochodnych.

  Możemy teraz zdefiniować operację $\star$:
  \begin{equation}
    \label{eq:Niespodziewane-teoretyczne-21}
    F \star G =
    \sum\limits_{ i = 0 }^{ \infty } \frac{ i^{ n } \hbar^{ n } }{ n! }
    \left\langle F^{ ( n ) }, \left( \tfrac{ 1 }{ 2 } \Delta \right)^{ \otimes n }
      G^{ ( n ) } \right\rangle.
    \end{equation}

\end{frame}
% ##################





% ##################
\begin{frame}
  \frametitle{Konstrukcja teorii swobodnej}

  Inwolucja.
  \begin{equation}
    \label{eq:Niespodziewane-teoretyczne-22}
    F^{ * }( \varphi ) = \overline{ F( \varphi ) }.
  \end{equation}

  \textbf{Ważne.}
  Do końca prezentacji będziemy pracować na poziomie tej algebry.

  \textbf{Uwaga.}
  Niemożliwość ograniczania~się do funkcjonałów liniowych wynika
  z~własności grupy renormalizacji.

\end{frame}
% ##################





% ##################
\begin{frame}
  \frametitle{Konstrukcja teorii z~oddziaływaniem}


  Algebra pełnej teorii
  \begin{equation}
    \label{eq:Niespodziewane-teoretyczne-23}
    S_{ \mathrm{Full} } = S_{ \mathrm{Free} } + S_{ \mathrm{I} }.
  \end{equation}
  Aby otrzymać teorię z~oddziaływaniem potrzebujemy jedynie
  zmienić iloczyn $\star = \star_{ \mathrm{Free} }$ na
  $\star_{ \mathrm{I} } = \star_{ \mathrm{Full} }$. Biorąc wzór z teorii
  rozpraszania, możemy to zrobić za pomocą operatorów M\o llera
  $R_{ S_{ I } }$:
  \begin{equation}
    \label{eq:Niespodziewane-teoretyczne-24}
    F \star_{ \mathrm{I} } G :=
    R_{ S_{ \mathrm{I} } }^{ -1 }( R_{ S_{ \mathrm{I} } } F
    \star R_{ S_{ \mathrm{I} } } G ).
  \end{equation}
  Ale aby zdefiniować operatory M\o llera potrzebujemy iloczynu
  chronologicznego.

  \textbf{Problem.}
  Iloczyn chronologiczny wymaga brania iloczynów dystrybucji, co skutkuje
  pojawieniem się rozbieżności ultrafioletowych.

\end{frame}
% ##################





% ##################
\begin{frame}
  \frametitle{Rozwinięcie perturbacyjne}

  \textbf{Zagadnienie renormalizacji.}
  Czasoprzestrzeń zakrzywiona nie ma określonej przestrzeni pędów,
  jednak można podać odpowiednią procedurę w przestrzeni położeń.
  Taką procedurę dla przypadku czasoprzestrzeni Minkowskiego
  podali Epstain i~Glaser [EG73].

  Została ona uogólniona na zadane zakrzywione czasoprzestrzenie
  przez Brunettiego i~Fredenhagena [BF00].

\end{frame}
% ##################





% ##################
\begin{frame}
  \frametitle{Rozwinięcie perturbacyjne}


  Renormalizacja przyczynowa Epstaina-Glasera [EG73].
  Polega na indukcyjnej konstrukcji iloczynów chronologicznych
  $\Tcal_{ n }$ przy której nie pojawiają się żadne
  rozbieżności w~ultrafiolecie. Konkretniej, rządamy by:
  \begin{itemize}
    \RaggedRight

  \item[1.] $\Tcal_{ 0 } = 1$.

  \item[2.]
    $\Tcal_{ 1 } = \exp\left( \int d \mathrm{vol}( x ) \,
      d \mathrm{vol}( y ) \, H( x, y )
      \frac{ \delta^{ 2 } }{ \delta \varphi( x ) \delta \varphi( y ) } \right)$.

  \item[3.]
    $\Tcal_{ n }( F_{ 1 }, \ldots, F_{ n } ) =
    \Tcal_{ k }( F_{ 1 }, \ldots, F_{ k } )
    \star \Tcal_{ n - k }( F_{ k + 1 },\ldots, F_{ n } )$ jeżeli funkcjonały
    $F_{ 1 }, \ldots, F_{ k }$ i~$F_{ k + 1 }, \ldots, F_{ n }$ są
    „rozdzielone przestrzennie”.

  \end{itemize}


  \textbf{Grupa renormalizacji.}
  Niejednoznaczność powyższej konstrukcji jest scharakteryzowana
  przez grupę renormalizacji St\"{u}ckelberga-Petermana.

\end{frame}
% ##################





% ##################
\begin{frame}
  \frametitle{Rozwinięcie perturbacyjne}


  Formalne rozwiązanie procedury Epstaina-Glasera [FR12]
  \begin{subequations}
    \begin{equation}
      \label{eq:Niespodziewane-teoretyczne-25-A}
      \Tcal_{ n }( F_{ 1 },\ldots, F_{ n } )( \varphi ) =
      e^{ \sum_{ i < j } D_{ i, j } } F( \varphi_{ 1 } ) \cdot \ldots
      \cdot F_{ n }( \varphi_{ n } ) |_{ \varphi_{ 1 } = \ldots = \varphi_{ n } = \varphi }.
    \end{equation}

    \vspace{-3em}



    \begin{equation}
      \begin{split}
        D_{ i j }
        &= i \, \hbar \left\langle \Delta^{ F }, \frac{ \delta^{ 2 } }{ \delta \varphi_{ i } \delta \varphi_{ j } }
          \right\rangle \\
        &= i \, \hbar \int d \mathrm{vol}( x ) \, d \mathrm{vol}( y ) \,
        \Delta^{ F }( x, y ) \frac{ \delta^{ 2 } }{ \delta \varphi_{ i }( x ) \delta \varphi_{ j }( y ) }.
      \end{split}
    \end{equation}
  \end{subequations}

  Można to oczywiście zapisać za pomocą grafów Feynmana.

\end{frame}
% ##################





% ##################
\begin{frame}
  \frametitle{Podsumowanie i~perspektywy}


  \begin{itemize}
    \RaggedRight

  \item Praca na poziomie algebraiczny pozwala ominąć wiele
    trudności.

  \item Istnieją prace dotyczące kwasiklasycznej reakcji pola
    na~czasoprzestrzeń.

  \item Można w~tym formalizmie opisać klasyczną teorię pola
    i~wówczas da się otrzymać wyniki o zbieżności badanych
    szeregów.

  \item Pełna Renormalizacja Epstaina-Glasera jest dobrze umotywowana
    fizycznie, lecz w~realnych obliczeniach jest niezwykle skomplikowana.
    Kluczowe więc dla całego podejścia jest to czy da się uprościć tę
    część teorii. Najnowsze prace skupiają się nad możliwością użycia
    kombinacji procedury E-G z~renormalizacją wymiarową, jednak nad
    zakrzywionych czasoprzestrzeniach wymaga to sformułowania
    w~reprezentacji położeń. Do tej pory osiągnięto pozytywne efekty
    w~czasoprzestrzeni Minkowskiego [DFKR13].

  \end{itemize}

\end{frame}
% ##################





% ##################
\begin{frame}[standout]


  { \color{jFrametitleFGColor} Nietypowe problemy mogą wymagać
    niestandardowych rozwiązań. Czy to konkretne jest właściwe to czas
    pokaże. }

\end{frame}
% ##################










% ######################################
\appendix
% ######################################





% ######################################
\EndingSlide{Pytania? Dziękuję za uwagę.}
% ######################################










% ##################
\begin{frame}
  \frametitle{Bibliografia}


  \begin{itemize}
    \RaggedRight

  \item [Fre09] K. Fredenhagen, \textit{The impact of~the~algebraic
      approach on perturbative quantum field theory}.
    Wystąpienie na \textit{Algebraic Quantum Field Theory~– the first 50
      Years}, G\"{o}ttingen 2009.

  \item [Wal09] R. M. Wald, textit{Axiomatic Quantum Field Theory
      in~Curved Spacetime}. Wystąpienie na \textit{Algebraic
      Quantum Field Theory~– the first 50 Years}, G\"{o}ttingen 2009.

  \item [Haa09] R. Haag, \textit{Local Algebras: A~look back at
    the~early years and~at~some successes and~missed opportunitie}.
    Wystąpienie na \textit{Algebraic Quantum Field Theory~– the first 50
    Years}, G\"{o}ttingen 2009.

  \item [FR12] K. Fredenhagen, K. Rejzner, \textit{Perturbative
    algebraic quantum field theory}, arXiv: 1208.1428.

  \end{itemize}

\end{frame}
% ##################





% ##################
\begin{frame}
  \frametitle{Bibliografia}


  \begin{itemize}
    \RaggedRight

  \item [EG73] H. Epstein, V. Glaser, \textit{The role of locality
    in~perturbation theory}, Ann. Inst. H. Poincar\'{e} A
    \textbf{19} (1973) 211.

  \item [BF00] R. Brunetti, K. Fredenhagen, \textit{Microlocal Analysis
    and~Interacting Quantum Field Theories: Renormalization
    on~Physical Backgrounds}, Commun.Math.Phys, \textbf{208} (2000)
    623-661, arXiv: 9903.028.

  \item [BF09] R. Brunetti, K. Fredenhagen, \textit{Quantum Field
    Theory on~Curved Backgrounds}, Proceedings of the
    Kompaktkurs \textit{Quantenfeldtheorie auf gekruemmten Raumzeiten} held
    at~Universitaet Potsdam, Germany, in 8.-12.10.2007, arXiv:
    0901.2063.

  \item [DFKR13] M. D\"{u}etsch, K. Fredenhagen, K. J. Keller,
    K.~Rejzner, \textit{Dimensional Regularization in Position Space,
      and~a Forest Formula for Epstein-Glaser Renormalization}, arXiv:
    1311.5424.

  \end{itemize}

\end{frame}
% ##################










% #####################################################################
% #####################################################################
% Bibliografia
\bibliographystyle{plalpha}

\bibliography{PhilNaturBooks}{}





% ############################

% Koniec dokumentu
\end{document}