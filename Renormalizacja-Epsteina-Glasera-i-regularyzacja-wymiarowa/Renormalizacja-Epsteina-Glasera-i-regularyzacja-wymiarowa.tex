% ---------------------------------------------------------------------
% Basic configuration of Beamera and Jagiellonian
% ---------------------------------------------------------------------
\RequirePackage[l2tabu, orthodox]{nag}



\ifx\PresentationStyle\notset
\def\PresentationStyle{dark}
\fi



\documentclass[10pt,t]{beamer}
\mode<presentation>
\usetheme[style=\PresentationStyle,logoLang=Latin,logoColor=monochromaticJUwhite,JUlogotitle=yes]{jagiellonian}



% ---------------------------------------
% Configuration files of Jagiellonian loceted in catalog preambule
% ---------------------------------------
% Configuration for polish language
% Need description
\usepackage[polish]{babel}
% Need description
\usepackage[MeX]{polski}



% ------------------------------
% Better support of polish chars in technical parts of PDF
% ------------------------------
\hypersetup{pdfencoding=auto,psdextra}

% Package "textpos" give as enviroment "textblock" which is very usefull in
% arranging text on slides.

% This is standard configuration of "textpos"
\usepackage[overlay,absolute]{textpos}

% If you need to see bounds of "textblock's" comment line above and uncomment
% one below.

% Caution! When showboxes option is on significant ammunt of space is add
% to the top of textblock and as such, everyting put in them gone down.
% We need to check how to remove this bug.

% \usepackage[showboxes,overlay,absolute]{textpos}



% Setting scale length for package "textpos"
\setlength{\TPHorizModule}{10mm}
\setlength{\TPVertModule}{\TPHorizModule}


% ---------------------------------------
% TikZ
% ---------------------------------------
% Importing TikZ libraries
\usetikzlibrary{arrows.meta}
\usetikzlibrary{positioning}





% % Configuration package "bm" that need for making bold symbols
% \newcommand{\bmmax}{0}
% \newcommand{\hmmax}{0}
% \usepackage{bm}




% ---------------------------------------
% Packages for scientific texts
% ---------------------------------------
% \let\lll\undefined  % Sometimes you must use this line to allow
% "amsmath" package to works with packages with packages for polish
% languge imported
% /preambul/LanguageSettings/JagiellonianPolishLanguageSettings.tex.
% This comments (probably) removes polish letter Ł.
\usepackage{amsmath}  % Packages from American Mathematical Society (AMS)
\usepackage{amssymb}
\usepackage{amscd}
\usepackage{amsthm}
\usepackage{siunitx}  % Package for typsetting SI units.
\usepackage{upgreek}  % Better looking greek letters.
% Example of using upgreek: pi = \uppi


\usepackage{calrsfs}  % Zmienia czcionkę kaligraficzną w \mathcal
% na ładniejszą. Może w innych miejscach robi to samo, ale o tym nic
% nie wiem.










% ---------------------------------------
% Packages written for lectures "Geometria 3D dla twórców gier wideo"
% ---------------------------------------
% \usepackage{./ProgramowanieSymulacjiFizykiPaczki/ProgramowanieSymulacjiFizyki}
% \usepackage{./ProgramowanieSymulacjiFizykiPaczki/ProgramowanieSymulacjiFizykiIndeksy}
% \usepackage{./ProgramowanieSymulacjiFizykiPaczki/ProgramowanieSymulacjiFizykiTikZStyle}





% !!!!!!!!!!!!!!!!!!!!!!!!!!!!!!
% !!!!!!!!!!!!!!!!!!!!!!!!!!!!!!
% EVIL STUFF
\if\JUlogotitle1
\edef\LogoJUPath{LogoJU_\JUlogoLang/LogoJU_\JUlogoShape_\JUlogoColor.pdf}
\titlegraphic{\hfill\includegraphics[scale=0.22]
{./JagiellonianPictures/\LogoJUPath}}
\fi
% ---------------------------------------
% Commands for handling colors
% ---------------------------------------


% Command for setting normal text color for some text in math modestyle
% Text color depend on used style of Jagiellonian

% Beamer version of command
\newcommand{\TextWithNormalTextColor}[1]{%
  {\color{jNormalTextFGColor}
    \setbeamercolor{math text}{fg=jNormalTextFGColor} {#1}}
}

% Article and similar classes version of command
% \newcommand{\TextWithNormalTextColor}[1]{%
%   {\color{jNormalTextsFGColor} {#1}}
% }



% Beamer version of command
\newcommand{\NormalTextInMathMode}[1]{%
  {\color{jNormalTextFGColor}
    \setbeamercolor{math text}{fg=jNormalTextFGColor} \text{#1}}
}


% Article and similar classes version of command
% \newcommand{\NormalTextInMathMode}[1]{%
%   {\color{jNormalTextsFGColor} \text{#1}}
% }




% Command that sets color of some mathematical text to the same color
% that has normal text in header (?)

% Beamer version of the command
\newcommand{\MathTextFrametitleFGColor}[1]{%
  {\color{jFrametitleFGColor}
    \setbeamercolor{math text}{fg=jFrametitleFGColor} #1}
}

% Article and similar classes version of the command
% \newcommand{\MathTextWhiteColor}[1]{{\color{jFrametitleFGColor} #1}}





% Command for setting color of alert text for some text in math modestyle

% Beamer version of the command
\newcommand{\MathTextAlertColor}[1]{%
  {\color{jOrange} \setbeamercolor{math text}{fg=jOrange} #1}
}

% Article and similar classes version of the command
% \newcommand{\MathTextAlertColor}[1]{{\color{jOrange} #1}}





% Command that allow you to sets chosen color as the color of some text into
% math mode. Due to some nuances in the way that Beamer handle colors
% it not work in all cases. We hope that in the future we will improve it.

% Beamer version of the command
\newcommand{\SetMathTextColor}[2]{%
  {\color{#1} \setbeamercolor{math text}{fg=#1} #2}
}


% Article and similar classes version of the command
% \newcommand{\SetMathTextColor}[2]{{\color{#1} #2}}










% ---------------------------------------
% Commands for few special slides
% ---------------------------------------
\newcommand{\EndingSlide}[1]{%
  \begin{frame}[standout]

    \begingroup

    \color{jFrametitleFGColor}

    #1

    \endgroup

  \end{frame}
}










% ---------------------------------------
% Commands for setting background pictures for some slides
% ---------------------------------------
\newcommand{\TitleBackgroundPicture}
{./JagiellonianPictures/Backgrounds/LajkonikDark.png}
\newcommand{\SectionBackgroundPicture}
{./JagiellonianPictures/Backgrounds/LajkonikLight.png}



\newcommand{\TitleSlideWithPicture}{%
  \begingroup

  \usebackgroundtemplate{%
    \includegraphics[height=\paperheight]{\TitleBackgroundPicture}}

  \maketitle

  \endgroup
}





\newcommand{\SectionSlideWithPicture}[1]{%
  \begingroup

  \usebackgroundtemplate{%
    \includegraphics[height=\paperheight]{\SectionBackgroundPicture}}

  \setbeamercolor{titlelike}{fg=normal text.fg}

  \section{#1}

  \endgroup
}










% ---------------------------------------
% Commands for lectures "Geometria 3D dla twórców gier wideo"
% Polish version
% ---------------------------------------
% Komendy teraz wykomentowane były potrzebne, gdy loga były na niebieskim
% tle, nie na białym. A są na białym bo tego chcieli w biurze projektu.
% \newcommand{\FundingLogoWhitePicturePL}
% {./PresentationPictures/CommonPictures/logotypFundusze_biale_bez_tla2.pdf}
\newcommand{\FundingLogoColorPicturePL}
{./PresentationPictures/CommonPictures/European_Funds_color_PL.pdf}
% \newcommand{\EULogoWhitePicturePL}
% {./PresentationPictures/CommonPictures/logotypUE_biale_bez_tla2.pdf}
\newcommand{\EUSocialFundLogoColorPicturePL}
{./PresentationPictures/CommonPictures/EU_Social_Fund_color_PL.pdf}
% \newcommand{\ZintegrUJLogoWhitePicturePL}
% {./PresentationPictures/CommonPictures/zintegruj-logo-white.pdf}
\newcommand{\ZintegrUJLogoColorPicturePL}
{./PresentationPictures/CommonPictures/ZintegrUJ_color.pdf}
\newcommand{\JULogoColorPicturePL}
{./JagiellonianPictures/LogoJU_PL/LogoJU_A_color.pdf}





\newcommand{\GeometryThreeDSpecialBeginningSlidePL}{%
  \begin{frame}[standout]

    \begin{textblock}{11}(1,0.7)

      \begin{flushleft}

        \mdseries

        \footnotesize

        \color{jFrametitleFGColor}

        Materiał powstał w ramach projektu współfinansowanego ze środków
        Unii Europejskiej w ramach Europejskiego Funduszu Społecznego
        POWR.03.05.00-00-Z309/17-00.

      \end{flushleft}

    \end{textblock}





    \begin{textblock}{10}(0,2.2)

      \tikz \fill[color=jBackgroundStyleLight] (0,0) rectangle (12.8,-1.5);

    \end{textblock}


    \begin{textblock}{3.2}(1,2.45)

      \includegraphics[scale=0.3]{\FundingLogoColorPicturePL}

    \end{textblock}


    \begin{textblock}{2.5}(3.7,2.5)

      \includegraphics[scale=0.2]{\JULogoColorPicturePL}

    \end{textblock}


    \begin{textblock}{2.5}(6,2.4)

      \includegraphics[scale=0.1]{\ZintegrUJLogoColorPicturePL}

    \end{textblock}


    \begin{textblock}{4.2}(8.4,2.6)

      \includegraphics[scale=0.3]{\EUSocialFundLogoColorPicturePL}

    \end{textblock}

  \end{frame}
}



\newcommand{\GeometryThreeDTwoSpecialBeginningSlidesPL}{%
  \begin{frame}[standout]

    \begin{textblock}{11}(1,0.7)

      \begin{flushleft}

        \mdseries

        \footnotesize

        \color{jFrametitleFGColor}

        Materiał powstał w ramach projektu współfinansowanego ze środków
        Unii Europejskiej w ramach Europejskiego Funduszu Społecznego
        POWR.03.05.00-00-Z309/17-00.

      \end{flushleft}

    \end{textblock}





    \begin{textblock}{10}(0,2.2)

      \tikz \fill[color=jBackgroundStyleLight] (0,0) rectangle (12.8,-1.5);

    \end{textblock}


    \begin{textblock}{3.2}(1,2.45)

      \includegraphics[scale=0.3]{\FundingLogoColorPicturePL}

    \end{textblock}


    \begin{textblock}{2.5}(3.7,2.5)

      \includegraphics[scale=0.2]{\JULogoColorPicturePL}

    \end{textblock}


    \begin{textblock}{2.5}(6,2.4)

      \includegraphics[scale=0.1]{\ZintegrUJLogoColorPicturePL}

    \end{textblock}


    \begin{textblock}{4.2}(8.4,2.6)

      \includegraphics[scale=0.3]{\EUSocialFundLogoColorPicturePL}

    \end{textblock}

  \end{frame}





  \TitleSlideWithPicture
}



\newcommand{\GeometryThreeDSpecialEndingSlidePL}{%
  \begin{frame}[standout]

    \begin{textblock}{11}(1,0.7)

      \begin{flushleft}

        \mdseries

        \footnotesize

        \color{jFrametitleFGColor}

        Materiał powstał w ramach projektu współfinansowanego ze środków
        Unii Europejskiej w~ramach Europejskiego Funduszu Społecznego
        POWR.03.05.00-00-Z309/17-00.

      \end{flushleft}

    \end{textblock}





    \begin{textblock}{10}(0,2.2)

      \tikz \fill[color=jBackgroundStyleLight] (0,0) rectangle (12.8,-1.5);

    \end{textblock}


    \begin{textblock}{3.2}(1,2.45)

      \includegraphics[scale=0.3]{\FundingLogoColorPicturePL}

    \end{textblock}


    \begin{textblock}{2.5}(3.7,2.5)

      \includegraphics[scale=0.2]{\JULogoColorPicturePL}

    \end{textblock}


    \begin{textblock}{2.5}(6,2.4)

      \includegraphics[scale=0.1]{\ZintegrUJLogoColorPicturePL}

    \end{textblock}


    \begin{textblock}{4.2}(8.4,2.6)

      \includegraphics[scale=0.3]{\EUSocialFundLogoColorPicturePL}

    \end{textblock}





    \begin{textblock}{11}(1,4)

      \begin{flushleft}

        \mdseries

        \footnotesize

        \RaggedRight

        \color{jFrametitleFGColor}

        Treść niniejszego wykładu jest udostępniona na~licencji
        Creative Commons (\textsc{cc}), z~uzna\-niem autorstwa
        (\textsc{by}) oraz udostępnianiem na tych samych warunkach
        (\textsc{sa}). Rysunki i~wy\-kresy zawarte w~wykładzie są
        autorstwa dr.~hab.~Pawła Węgrzyna et~al. i~są dostępne
        na tej samej licencji, o~ile nie wskazano inaczej.
        W~prezentacji wykorzystano temat Beamera Jagiellonian,
        oparty na~temacie Metropolis Matthiasa Vogelgesanga,
        dostępnym na licencji \LaTeX{} Project Public License~1.3c
        pod adresem: \colorhref{https://github.com/matze/mtheme}
        {https://github.com/matze/mtheme}.

        Projekt typograficzny: Iwona Grabska-Gradzińska \\
        Skład: Kamil Ziemian;
        Korekta: Wojciech Palacz \\
        Modele: Dariusz Frymus, Kamil Nowakowski \\
        Rysunki i~wykresy: Kamil Ziemian, Paweł Węgrzyn, Wojciech Palacz

      \end{flushleft}

    \end{textblock}

  \end{frame}
}



\newcommand{\GeometryThreeDTwoSpecialEndingSlidesPL}[1]{%
  \begin{frame}[standout]


    \begin{textblock}{11}(1,0.7)

      \begin{flushleft}

        \mdseries

        \footnotesize

        \color{jFrametitleFGColor}

        Materiał powstał w ramach projektu współfinansowanego ze środków
        Unii Europejskiej w~ramach Europejskiego Funduszu Społecznego
        POWR.03.05.00-00-Z309/17-00.

      \end{flushleft}

    \end{textblock}





    \begin{textblock}{10}(0,2.2)

      \tikz \fill[color=jBackgroundStyleLight] (0,0) rectangle (12.8,-1.5);

    \end{textblock}


    \begin{textblock}{3.2}(1,2.45)

      \includegraphics[scale=0.3]{\FundingLogoColorPicturePL}

    \end{textblock}


    \begin{textblock}{2.5}(3.7,2.5)

      \includegraphics[scale=0.2]{\JULogoColorPicturePL}

    \end{textblock}


    \begin{textblock}{2.5}(6,2.4)

      \includegraphics[scale=0.1]{\ZintegrUJLogoColorPicturePL}

    \end{textblock}


    \begin{textblock}{4.2}(8.4,2.6)

      \includegraphics[scale=0.3]{\EUSocialFundLogoColorPicturePL}

    \end{textblock}





    \begin{textblock}{11}(1,4)

      \begin{flushleft}

        \mdseries

        \footnotesize

        \RaggedRight

        \color{jFrametitleFGColor}

        Treść niniejszego wykładu jest udostępniona na~licencji
        Creative Commons (\textsc{cc}), z~uzna\-niem autorstwa
        (\textsc{by}) oraz udostępnianiem na tych samych warunkach
        (\textsc{sa}). Rysunki i~wy\-kresy zawarte w~wykładzie są
        autorstwa dr.~hab.~Pawła Węgrzyna et~al. i~są dostępne
        na tej samej licencji, o~ile nie wskazano inaczej.
        W~prezentacji wykorzystano temat Beamera Jagiellonian,
        oparty na~temacie Metropolis Matthiasa Vogelgesanga,
        dostępnym na licencji \LaTeX{} Project Public License~1.3c
        pod adresem: \colorhref{https://github.com/matze/mtheme}
        {https://github.com/matze/mtheme}.

        Projekt typograficzny: Iwona Grabska-Gradzińska \\
        Skład: Kamil Ziemian;
        Korekta: Wojciech Palacz \\
        Modele: Dariusz Frymus, Kamil Nowakowski \\
        Rysunki i~wykresy: Kamil Ziemian, Paweł Węgrzyn, Wojciech Palacz

      \end{flushleft}

    \end{textblock}

  \end{frame}





  \begin{frame}[standout]

    \begingroup

    \color{jFrametitleFGColor}

    #1

    \endgroup

  \end{frame}
}



\newcommand{\GeometryThreeDSpecialEndingSlideVideoPL}{%
  \begin{frame}[standout]

    \begin{textblock}{11}(1,0.7)

      \begin{flushleft}

        \mdseries

        \footnotesize

        \color{jFrametitleFGColor}

        Materiał powstał w ramach projektu współfinansowanego ze środków
        Unii Europejskiej w~ramach Europejskiego Funduszu Społecznego
        POWR.03.05.00-00-Z309/17-00.

      \end{flushleft}

    \end{textblock}





    \begin{textblock}{10}(0,2.2)

      \tikz \fill[color=jBackgroundStyleLight] (0,0) rectangle (12.8,-1.5);

    \end{textblock}


    \begin{textblock}{3.2}(1,2.45)

      \includegraphics[scale=0.3]{\FundingLogoColorPicturePL}

    \end{textblock}


    \begin{textblock}{2.5}(3.7,2.5)

      \includegraphics[scale=0.2]{\JULogoColorPicturePL}

    \end{textblock}


    \begin{textblock}{2.5}(6,2.4)

      \includegraphics[scale=0.1]{\ZintegrUJLogoColorPicturePL}

    \end{textblock}


    \begin{textblock}{4.2}(8.4,2.6)

      \includegraphics[scale=0.3]{\EUSocialFundLogoColorPicturePL}

    \end{textblock}





    \begin{textblock}{11}(1,4)

      \begin{flushleft}

        \mdseries

        \footnotesize

        \RaggedRight

        \color{jFrametitleFGColor}

        Treść niniejszego wykładu jest udostępniona na~licencji
        Creative Commons (\textsc{cc}), z~uzna\-niem autorstwa
        (\textsc{by}) oraz udostępnianiem na tych samych warunkach
        (\textsc{sa}). Rysunki i~wy\-kresy zawarte w~wykładzie są
        autorstwa dr.~hab.~Pawła Węgrzyna et~al. i~są dostępne
        na tej samej licencji, o~ile nie wskazano inaczej.
        W~prezentacji wykorzystano temat Beamera Jagiellonian,
        oparty na~temacie Metropolis Matthiasa Vogelgesanga,
        dostępnym na licencji \LaTeX{} Project Public License~1.3c
        pod adresem: \colorhref{https://github.com/matze/mtheme}
        {https://github.com/matze/mtheme}.

        Projekt typograficzny: Iwona Grabska-Gradzińska;
        Skład: Kamil Ziemian \\
        Korekta: Wojciech Palacz;
        Modele: Dariusz Frymus, Kamil Nowakowski \\
        Rysunki i~wykresy: Kamil Ziemian, Paweł Węgrzyn, Wojciech Palacz \\
        Montaż: Agencja Filmowa Film \& Television Production~-- Zbigniew
        Masklak

      \end{flushleft}

    \end{textblock}

  \end{frame}
}





\newcommand{\GeometryThreeDTwoSpecialEndingSlidesVideoPL}[1]{%
  \begin{frame}[standout]

    \begin{textblock}{11}(1,0.7)

      \begin{flushleft}

        \mdseries

        \footnotesize

        \color{jFrametitleFGColor}

        Materiał powstał w ramach projektu współfinansowanego ze środków
        Unii Europejskiej w~ramach Europejskiego Funduszu Społecznego
        POWR.03.05.00-00-Z309/17-00.

      \end{flushleft}

    \end{textblock}





    \begin{textblock}{10}(0,2.2)

      \tikz \fill[color=jBackgroundStyleLight] (0,0) rectangle (12.8,-1.5);

    \end{textblock}


    \begin{textblock}{3.2}(1,2.45)

      \includegraphics[scale=0.3]{\FundingLogoColorPicturePL}

    \end{textblock}


    \begin{textblock}{2.5}(3.7,2.5)

      \includegraphics[scale=0.2]{\JULogoColorPicturePL}

    \end{textblock}


    \begin{textblock}{2.5}(6,2.4)

      \includegraphics[scale=0.1]{\ZintegrUJLogoColorPicturePL}

    \end{textblock}


    \begin{textblock}{4.2}(8.4,2.6)

      \includegraphics[scale=0.3]{\EUSocialFundLogoColorPicturePL}

    \end{textblock}





    \begin{textblock}{11}(1,4)

      \begin{flushleft}

        \mdseries

        \footnotesize

        \RaggedRight

        \color{jFrametitleFGColor}

        Treść niniejszego wykładu jest udostępniona na~licencji
        Creative Commons (\textsc{cc}), z~uzna\-niem autorstwa
        (\textsc{by}) oraz udostępnianiem na tych samych warunkach
        (\textsc{sa}). Rysunki i~wy\-kresy zawarte w~wykładzie są
        autorstwa dr.~hab.~Pawła Węgrzyna et~al. i~są dostępne
        na tej samej licencji, o~ile nie wskazano inaczej.
        W~prezentacji wykorzystano temat Beamera Jagiellonian,
        oparty na~temacie Metropolis Matthiasa Vogelgesanga,
        dostępnym na licencji \LaTeX{} Project Public License~1.3c
        pod adresem: \colorhref{https://github.com/matze/mtheme}
        {https://github.com/matze/mtheme}.

        Projekt typograficzny: Iwona Grabska-Gradzińska;
        Skład: Kamil Ziemian \\
        Korekta: Wojciech Palacz;
        Modele: Dariusz Frymus, Kamil Nowakowski \\
        Rysunki i~wykresy: Kamil Ziemian, Paweł Węgrzyn, Wojciech Palacz \\
        Montaż: Agencja Filmowa Film \& Television Production~-- Zbigniew
        Masklak

      \end{flushleft}

    \end{textblock}

  \end{frame}





  \begin{frame}[standout]


    \begingroup

    \color{jFrametitleFGColor}

    #1

    \endgroup

  \end{frame}
}










% ---------------------------------------
% Commands for lectures "Geometria 3D dla twórców gier wideo"
% English version
% ---------------------------------------
% \newcommand{\FundingLogoWhitePictureEN}
% {./PresentationPictures/CommonPictures/logotypFundusze_biale_bez_tla2.pdf}
\newcommand{\FundingLogoColorPictureEN}
{./PresentationPictures/CommonPictures/European_Funds_color_EN.pdf}
% \newcommand{\EULogoWhitePictureEN}
% {./PresentationPictures/CommonPictures/logotypUE_biale_bez_tla2.pdf}
\newcommand{\EUSocialFundLogoColorPictureEN}
{./PresentationPictures/CommonPictures/EU_Social_Fund_color_EN.pdf}
% \newcommand{\ZintegrUJLogoWhitePictureEN}
% {./PresentationPictures/CommonPictures/zintegruj-logo-white.pdf}
\newcommand{\ZintegrUJLogoColorPictureEN}
{./PresentationPictures/CommonPictures/ZintegrUJ_color.pdf}
\newcommand{\JULogoColorPictureEN}
{./JagiellonianPictures/LogoJU_EN/LogoJU_A_color.pdf}



\newcommand{\GeometryThreeDSpecialBeginningSlideEN}{%
  \begin{frame}[standout]

    \begin{textblock}{11}(1,0.7)

      \begin{flushleft}

        \mdseries

        \footnotesize

        \color{jFrametitleFGColor}

        This content was created as part of a project co-financed by the
        European Union within the framework of the European Social Fund
        POWR.03.05.00-00-Z309/17-00.

      \end{flushleft}

    \end{textblock}





    \begin{textblock}{10}(0,2.2)

      \tikz \fill[color=jBackgroundStyleLight] (0,0) rectangle (12.8,-1.5);

    \end{textblock}


    \begin{textblock}{3.2}(0.7,2.45)

      \includegraphics[scale=0.3]{\FundingLogoColorPictureEN}

    \end{textblock}


    \begin{textblock}{2.5}(4.15,2.5)

      \includegraphics[scale=0.2]{\JULogoColorPictureEN}

    \end{textblock}


    \begin{textblock}{2.5}(6.35,2.4)

      \includegraphics[scale=0.1]{\ZintegrUJLogoColorPictureEN}

    \end{textblock}


    \begin{textblock}{4.2}(8.4,2.6)

      \includegraphics[scale=0.3]{\EUSocialFundLogoColorPictureEN}

    \end{textblock}

  \end{frame}
}



\newcommand{\GeometryThreeDTwoSpecialBeginningSlidesEN}{%
  \begin{frame}[standout]

    \begin{textblock}{11}(1,0.7)

      \begin{flushleft}

        \mdseries

        \footnotesize

        \color{jFrametitleFGColor}

        This content was created as part of a project co-financed by the
        European Union within the framework of the European Social Fund
        POWR.03.05.00-00-Z309/17-00.

      \end{flushleft}

    \end{textblock}





    \begin{textblock}{10}(0,2.2)

      \tikz \fill[color=jBackgroundStyleLight] (0,0) rectangle (12.8,-1.5);

    \end{textblock}


    \begin{textblock}{3.2}(0.7,2.45)

      \includegraphics[scale=0.3]{\FundingLogoColorPictureEN}

    \end{textblock}


    \begin{textblock}{2.5}(4.15,2.5)

      \includegraphics[scale=0.2]{\JULogoColorPictureEN}

    \end{textblock}


    \begin{textblock}{2.5}(6.35,2.4)

      \includegraphics[scale=0.1]{\ZintegrUJLogoColorPictureEN}

    \end{textblock}


    \begin{textblock}{4.2}(8.4,2.6)

      \includegraphics[scale=0.3]{\EUSocialFundLogoColorPictureEN}

    \end{textblock}

  \end{frame}





  \TitleSlideWithPicture
}



\newcommand{\GeometryThreeDSpecialEndingSlideEN}{%
  \begin{frame}[standout]

    \begin{textblock}{11}(1,0.7)

      \begin{flushleft}

        \mdseries

        \footnotesize

        \color{jFrametitleFGColor}

        This content was created as part of a project co-financed by the
        European Union within the framework of the European Social Fund
        POWR.03.05.00-00-Z309/17-00.

      \end{flushleft}

    \end{textblock}





    \begin{textblock}{10}(0,2.2)

      \tikz \fill[color=jBackgroundStyleLight] (0,0) rectangle (12.8,-1.5);

    \end{textblock}


    \begin{textblock}{3.2}(0.7,2.45)

      \includegraphics[scale=0.3]{\FundingLogoColorPictureEN}

    \end{textblock}


    \begin{textblock}{2.5}(4.15,2.5)

      \includegraphics[scale=0.2]{\JULogoColorPictureEN}

    \end{textblock}


    \begin{textblock}{2.5}(6.35,2.4)

      \includegraphics[scale=0.1]{\ZintegrUJLogoColorPictureEN}

    \end{textblock}


    \begin{textblock}{4.2}(8.4,2.6)

      \includegraphics[scale=0.3]{\EUSocialFundLogoColorPictureEN}

    \end{textblock}





    \begin{textblock}{11}(1,4)

      \begin{flushleft}

        \mdseries

        \footnotesize

        \RaggedRight

        \color{jFrametitleFGColor}

        The content of this lecture is made available under a~Creative
        Commons licence (\textsc{cc}), giving the author the credits
        (\textsc{by}) and putting an obligation to share on the same terms
        (\textsc{sa}). Figures and diagrams included in the lecture are
        authored by Paweł Węgrzyn et~al., and are available under the same
        license unless indicated otherwise.\\ The presentation uses the
        Beamer Jagiellonian theme based on Matthias Vogelgesang’s
        Metropolis theme, available under license \LaTeX{} Project
        Public License~1.3c at: \colorhref{https://github.com/matze/mtheme}
        {https://github.com/matze/mtheme}.

        Typographic design: Iwona Grabska-Gradzińska \\
        \LaTeX{} Typesetting: Kamil Ziemian \\
        Proofreading: Wojciech Palacz,
        Monika Stawicka \\
        3D Models: Dariusz Frymus, Kamil Nowakowski \\
        Figures and charts: Kamil Ziemian, Paweł Węgrzyn, Wojciech Palacz

      \end{flushleft}

    \end{textblock}

  \end{frame}
}



\newcommand{\GeometryThreeDTwoSpecialEndingSlidesEN}[1]{%
  \begin{frame}[standout]


    \begin{textblock}{11}(1,0.7)

      \begin{flushleft}

        \mdseries

        \footnotesize

        \color{jFrametitleFGColor}

        This content was created as part of a project co-financed by the
        European Union within the framework of the European Social Fund
        POWR.03.05.00-00-Z309/17-00.

      \end{flushleft}

    \end{textblock}





    \begin{textblock}{10}(0,2.2)

      \tikz \fill[color=jBackgroundStyleLight] (0,0) rectangle (12.8,-1.5);

    \end{textblock}


    \begin{textblock}{3.2}(0.7,2.45)

      \includegraphics[scale=0.3]{\FundingLogoColorPictureEN}

    \end{textblock}


    \begin{textblock}{2.5}(4.15,2.5)

      \includegraphics[scale=0.2]{\JULogoColorPictureEN}

    \end{textblock}


    \begin{textblock}{2.5}(6.35,2.4)

      \includegraphics[scale=0.1]{\ZintegrUJLogoColorPictureEN}

    \end{textblock}


    \begin{textblock}{4.2}(8.4,2.6)

      \includegraphics[scale=0.3]{\EUSocialFundLogoColorPictureEN}

    \end{textblock}





    \begin{textblock}{11}(1,4)

      \begin{flushleft}

        \mdseries

        \footnotesize

        \RaggedRight

        \color{jFrametitleFGColor}

        The content of this lecture is made available under a~Creative
        Commons licence (\textsc{cc}), giving the author the credits
        (\textsc{by}) and putting an obligation to share on the same terms
        (\textsc{sa}). Figures and diagrams included in the lecture are
        authored by Paweł Węgrzyn et~al., and are available under the same
        license unless indicated otherwise.\\ The presentation uses the
        Beamer Jagiellonian theme based on Matthias Vogelgesang’s
        Metropolis theme, available under license \LaTeX{} Project
        Public License~1.3c at: \colorhref{https://github.com/matze/mtheme}
        {https://github.com/matze/mtheme}.

        Typographic design: Iwona Grabska-Gradzińska \\
        \LaTeX{} Typesetting: Kamil Ziemian \\
        Proofreading: Wojciech Palacz,
        Monika Stawicka \\
        3D Models: Dariusz Frymus, Kamil Nowakowski \\
        Figures and charts: Kamil Ziemian, Paweł Węgrzyn, Wojciech Palacz

      \end{flushleft}

    \end{textblock}

  \end{frame}





  \begin{frame}[standout]

    \begingroup

    \color{jFrametitleFGColor}

    #1

    \endgroup

  \end{frame}
}



\newcommand{\GeometryThreeDSpecialEndingSlideVideoVerOneEN}{%
  \begin{frame}[standout]

    \begin{textblock}{11}(1,0.7)

      \begin{flushleft}

        \mdseries

        \footnotesize

        \color{jFrametitleFGColor}

        This content was created as part of a project co-financed by the
        European Union within the framework of the European Social Fund
        POWR.03.05.00-00-Z309/17-00.

      \end{flushleft}

    \end{textblock}





    \begin{textblock}{10}(0,2.2)

      \tikz \fill[color=jBackgroundStyleLight] (0,0) rectangle (12.8,-1.5);

    \end{textblock}


    \begin{textblock}{3.2}(0.7,2.45)

      \includegraphics[scale=0.3]{\FundingLogoColorPictureEN}

    \end{textblock}


    \begin{textblock}{2.5}(4.15,2.5)

      \includegraphics[scale=0.2]{\JULogoColorPictureEN}

    \end{textblock}


    \begin{textblock}{2.5}(6.35,2.4)

      \includegraphics[scale=0.1]{\ZintegrUJLogoColorPictureEN}

    \end{textblock}


    \begin{textblock}{4.2}(8.4,2.6)

      \includegraphics[scale=0.3]{\EUSocialFundLogoColorPictureEN}

    \end{textblock}





    \begin{textblock}{11}(1,4)

      \begin{flushleft}

        \mdseries

        \footnotesize

        \RaggedRight

        \color{jFrametitleFGColor}

        The content of this lecture is made available under a Creative
        Commons licence (\textsc{cc}), giving the author the credits
        (\textsc{by}) and putting an obligation to share on the same terms
        (\textsc{sa}). Figures and diagrams included in the lecture are
        authored by Paweł Węgrzyn et~al., and are available under the same
        license unless indicated otherwise.\\ The presentation uses the
        Beamer Jagiellonian theme based on Matthias Vogelgesang’s
        Metropolis theme, available under license \LaTeX{} Project
        Public License~1.3c at: \colorhref{https://github.com/matze/mtheme}
        {https://github.com/matze/mtheme}.

        Typographic design: Iwona Grabska-Gradzińska;
        \LaTeX{} Typesetting: Kamil Ziemian \\
        Proofreading: Wojciech Palacz,
        Monika Stawicka \\
        3D Models: Dariusz Frymus, Kamil Nowakowski \\
        Figures and charts: Kamil Ziemian, Paweł Węgrzyn, Wojciech
        Palacz \\
        Film editing: Agencja Filmowa Film \& Television Production~--
        Zbigniew Masklak

      \end{flushleft}

    \end{textblock}

  \end{frame}
}



\newcommand{\GeometryThreeDSpecialEndingSlideVideoVerTwoEN}{%
  \begin{frame}[standout]

    \begin{textblock}{11}(1,0.7)

      \begin{flushleft}

        \mdseries

        \footnotesize

        \color{jFrametitleFGColor}

        This content was created as part of a project co-financed by the
        European Union within the framework of the European Social Fund
        POWR.03.05.00-00-Z309/17-00.

      \end{flushleft}

    \end{textblock}





    \begin{textblock}{10}(0,2.2)

      \tikz \fill[color=jBackgroundStyleLight] (0,0) rectangle (12.8,-1.5);

    \end{textblock}


    \begin{textblock}{3.2}(0.7,2.45)

      \includegraphics[scale=0.3]{\FundingLogoColorPictureEN}

    \end{textblock}


    \begin{textblock}{2.5}(4.15,2.5)

      \includegraphics[scale=0.2]{\JULogoColorPictureEN}

    \end{textblock}


    \begin{textblock}{2.5}(6.35,2.4)

      \includegraphics[scale=0.1]{\ZintegrUJLogoColorPictureEN}

    \end{textblock}


    \begin{textblock}{4.2}(8.4,2.6)

      \includegraphics[scale=0.3]{\EUSocialFundLogoColorPictureEN}

    \end{textblock}





    \begin{textblock}{11}(1,4)

      \begin{flushleft}

        \mdseries

        \footnotesize

        \RaggedRight

        \color{jFrametitleFGColor}

        The content of this lecture is made available under a Creative
        Commons licence (\textsc{cc}), giving the author the credits
        (\textsc{by}) and putting an obligation to share on the same terms
        (\textsc{sa}). Figures and diagrams included in the lecture are
        authored by Paweł Węgrzyn et~al., and are available under the same
        license unless indicated otherwise.\\ The presentation uses the
        Beamer Jagiellonian theme based on Matthias Vogelgesang’s
        Metropolis theme, available under license \LaTeX{} Project
        Public License~1.3c at: \colorhref{https://github.com/matze/mtheme}
        {https://github.com/matze/mtheme}.

        Typographic design: Iwona Grabska-Gradzińska;
        \LaTeX{} Typesetting: Kamil Ziemian \\
        Proofreading: Wojciech Palacz,
        Monika Stawicka \\
        3D Models: Dariusz Frymus, Kamil Nowakowski \\
        Figures and charts: Kamil Ziemian, Paweł Węgrzyn, Wojciech
        Palacz \\
        Film editing: IMAVI -- Joanna Kozakiewicz, Krzysztof Magda, Nikodem
        Frodyma

      \end{flushleft}

    \end{textblock}

  \end{frame}
}



\newcommand{\GeometryThreeDSpecialEndingSlideVideoVerThreeEN}{%
  \begin{frame}[standout]

    \begin{textblock}{11}(1,0.7)

      \begin{flushleft}

        \mdseries

        \footnotesize

        \color{jFrametitleFGColor}

        This content was created as part of a project co-financed by the
        European Union within the framework of the European Social Fund
        POWR.03.05.00-00-Z309/17-00.

      \end{flushleft}

    \end{textblock}





    \begin{textblock}{10}(0,2.2)

      \tikz \fill[color=jBackgroundStyleLight] (0,0) rectangle (12.8,-1.5);

    \end{textblock}


    \begin{textblock}{3.2}(0.7,2.45)

      \includegraphics[scale=0.3]{\FundingLogoColorPictureEN}

    \end{textblock}


    \begin{textblock}{2.5}(4.15,2.5)

      \includegraphics[scale=0.2]{\JULogoColorPictureEN}

    \end{textblock}


    \begin{textblock}{2.5}(6.35,2.4)

      \includegraphics[scale=0.1]{\ZintegrUJLogoColorPictureEN}

    \end{textblock}


    \begin{textblock}{4.2}(8.4,2.6)

      \includegraphics[scale=0.3]{\EUSocialFundLogoColorPictureEN}

    \end{textblock}





    \begin{textblock}{11}(1,4)

      \begin{flushleft}

        \mdseries

        \footnotesize

        \RaggedRight

        \color{jFrametitleFGColor}

        The content of this lecture is made available under a Creative
        Commons licence (\textsc{cc}), giving the author the credits
        (\textsc{by}) and putting an obligation to share on the same terms
        (\textsc{sa}). Figures and diagrams included in the lecture are
        authored by Paweł Węgrzyn et~al., and are available under the same
        license unless indicated otherwise.\\ The presentation uses the
        Beamer Jagiellonian theme based on Matthias Vogelgesang’s
        Metropolis theme, available under license \LaTeX{} Project
        Public License~1.3c at: \colorhref{https://github.com/matze/mtheme}
        {https://github.com/matze/mtheme}.

        Typographic design: Iwona Grabska-Gradzińska;
        \LaTeX{} Typesetting: Kamil Ziemian \\
        Proofreading: Wojciech Palacz,
        Monika Stawicka \\
        3D Models: Dariusz Frymus, Kamil Nowakowski \\
        Figures and charts: Kamil Ziemian, Paweł Węgrzyn, Wojciech
        Palacz \\
        Film editing: Agencja Filmowa Film \& Television Production~--
        Zbigniew Masklak \\
        Film editing: IMAVI -- Joanna Kozakiewicz, Krzysztof Magda, Nikodem
        Frodyma

      \end{flushleft}

    \end{textblock}

  \end{frame}
}



\newcommand{\GeometryThreeDTwoSpecialEndingSlidesVideoVerOneEN}[1]{%
  \begin{frame}[standout]

    \begin{textblock}{11}(1,0.7)

      \begin{flushleft}

        \mdseries

        \footnotesize

        \color{jFrametitleFGColor}

        This content was created as part of a project co-financed by the
        European Union within the framework of the European Social Fund
        POWR.03.05.00-00-Z309/17-00.

      \end{flushleft}

    \end{textblock}





    \begin{textblock}{10}(0,2.2)

      \tikz \fill[color=jBackgroundStyleLight] (0,0) rectangle (12.8,-1.5);

    \end{textblock}


    \begin{textblock}{3.2}(0.7,2.45)

      \includegraphics[scale=0.3]{\FundingLogoColorPictureEN}

    \end{textblock}


    \begin{textblock}{2.5}(4.15,2.5)

      \includegraphics[scale=0.2]{\JULogoColorPictureEN}

    \end{textblock}


    \begin{textblock}{2.5}(6.35,2.4)

      \includegraphics[scale=0.1]{\ZintegrUJLogoColorPictureEN}

    \end{textblock}


    \begin{textblock}{4.2}(8.4,2.6)

      \includegraphics[scale=0.3]{\EUSocialFundLogoColorPictureEN}

    \end{textblock}





    \begin{textblock}{11}(1,4)

      \begin{flushleft}

        \mdseries

        \footnotesize

        \RaggedRight

        \color{jFrametitleFGColor}

        The content of this lecture is made available under a Creative
        Commons licence (\textsc{cc}), giving the author the credits
        (\textsc{by}) and putting an obligation to share on the same terms
        (\textsc{sa}). Figures and diagrams included in the lecture are
        authored by Paweł Węgrzyn et~al., and are available under the same
        license unless indicated otherwise.\\ The presentation uses the
        Beamer Jagiellonian theme based on Matthias Vogelgesang’s
        Metropolis theme, available under license \LaTeX{} Project
        Public License~1.3c at: \colorhref{https://github.com/matze/mtheme}
        {https://github.com/matze/mtheme}.

        Typographic design: Iwona Grabska-Gradzińska;
        \LaTeX{} Typesetting: Kamil Ziemian \\
        Proofreading: Wojciech Palacz,
        Monika Stawicka \\
        3D Models: Dariusz Frymus, Kamil Nowakowski \\
        Figures and charts: Kamil Ziemian, Paweł Węgrzyn,
        Wojciech Palacz \\
        Film editing: Agencja Filmowa Film \& Television Production~--
        Zbigniew Masklak

      \end{flushleft}

    \end{textblock}

  \end{frame}





  \begin{frame}[standout]


    \begingroup

    \color{jFrametitleFGColor}

    #1

    \endgroup

  \end{frame}
}



\newcommand{\GeometryThreeDTwoSpecialEndingSlidesVideoVerTwoEN}[1]{%
  \begin{frame}[standout]

    \begin{textblock}{11}(1,0.7)

      \begin{flushleft}

        \mdseries

        \footnotesize

        \color{jFrametitleFGColor}

        This content was created as part of a project co-financed by the
        European Union within the framework of the European Social Fund
        POWR.03.05.00-00-Z309/17-00.

      \end{flushleft}

    \end{textblock}





    \begin{textblock}{10}(0,2.2)

      \tikz \fill[color=jBackgroundStyleLight] (0,0) rectangle (12.8,-1.5);

    \end{textblock}


    \begin{textblock}{3.2}(0.7,2.45)

      \includegraphics[scale=0.3]{\FundingLogoColorPictureEN}

    \end{textblock}


    \begin{textblock}{2.5}(4.15,2.5)

      \includegraphics[scale=0.2]{\JULogoColorPictureEN}

    \end{textblock}


    \begin{textblock}{2.5}(6.35,2.4)

      \includegraphics[scale=0.1]{\ZintegrUJLogoColorPictureEN}

    \end{textblock}


    \begin{textblock}{4.2}(8.4,2.6)

      \includegraphics[scale=0.3]{\EUSocialFundLogoColorPictureEN}

    \end{textblock}





    \begin{textblock}{11}(1,4)

      \begin{flushleft}

        \mdseries

        \footnotesize

        \RaggedRight

        \color{jFrametitleFGColor}

        The content of this lecture is made available under a Creative
        Commons licence (\textsc{cc}), giving the author the credits
        (\textsc{by}) and putting an obligation to share on the same terms
        (\textsc{sa}). Figures and diagrams included in the lecture are
        authored by Paweł Węgrzyn et~al., and are available under the same
        license unless indicated otherwise.\\ The presentation uses the
        Beamer Jagiellonian theme based on Matthias Vogelgesang’s
        Metropolis theme, available under license \LaTeX{} Project
        Public License~1.3c at: \colorhref{https://github.com/matze/mtheme}
        {https://github.com/matze/mtheme}.

        Typographic design: Iwona Grabska-Gradzińska;
        \LaTeX{} Typesetting: Kamil Ziemian \\
        Proofreading: Wojciech Palacz,
        Monika Stawicka \\
        3D Models: Dariusz Frymus, Kamil Nowakowski \\
        Figures and charts: Kamil Ziemian, Paweł Węgrzyn,
        Wojciech Palacz \\
        Film editing: IMAVI -- Joanna Kozakiewicz, Krzysztof Magda, Nikodem
        Frodyma

      \end{flushleft}

    \end{textblock}

  \end{frame}





  \begin{frame}[standout]


    \begingroup

    \color{jFrametitleFGColor}

    #1

    \endgroup

  \end{frame}
}



\newcommand{\GeometryThreeDTwoSpecialEndingSlidesVideoVerThreeEN}[1]{%
  \begin{frame}[standout]

    \begin{textblock}{11}(1,0.7)

      \begin{flushleft}

        \mdseries

        \footnotesize

        \color{jFrametitleFGColor}

        This content was created as part of a project co-financed by the
        European Union within the framework of the European Social Fund
        POWR.03.05.00-00-Z309/17-00.

      \end{flushleft}

    \end{textblock}





    \begin{textblock}{10}(0,2.2)

      \tikz \fill[color=jBackgroundStyleLight] (0,0) rectangle (12.8,-1.5);

    \end{textblock}


    \begin{textblock}{3.2}(0.7,2.45)

      \includegraphics[scale=0.3]{\FundingLogoColorPictureEN}

    \end{textblock}


    \begin{textblock}{2.5}(4.15,2.5)

      \includegraphics[scale=0.2]{\JULogoColorPictureEN}

    \end{textblock}


    \begin{textblock}{2.5}(6.35,2.4)

      \includegraphics[scale=0.1]{\ZintegrUJLogoColorPictureEN}

    \end{textblock}


    \begin{textblock}{4.2}(8.4,2.6)

      \includegraphics[scale=0.3]{\EUSocialFundLogoColorPictureEN}

    \end{textblock}





    \begin{textblock}{11}(1,4)

      \begin{flushleft}

        \mdseries

        \footnotesize

        \RaggedRight

        \color{jFrametitleFGColor}

        The content of this lecture is made available under a Creative
        Commons licence (\textsc{cc}), giving the author the credits
        (\textsc{by}) and putting an obligation to share on the same terms
        (\textsc{sa}). Figures and diagrams included in the lecture are
        authored by Paweł Węgrzyn et~al., and are available under the same
        license unless indicated otherwise. \\ The presentation uses the
        Beamer Jagiellonian theme based on Matthias Vogelgesang’s
        Metropolis theme, available under license \LaTeX{} Project
        Public License~1.3c at: \colorhref{https://github.com/matze/mtheme}
        {https://github.com/matze/mtheme}.

        Typographic design: Iwona Grabska-Gradzińska;
        \LaTeX{} Typesetting: Kamil Ziemian \\
        Proofreading: Leszek Hadasz, Wojciech Palacz,
        Monika Stawicka \\
        3D Models: Dariusz Frymus, Kamil Nowakowski \\
        Figures and charts: Kamil Ziemian, Paweł Węgrzyn,
        Wojciech Palacz \\
        Film editing: Agencja Filmowa Film \& Television Production~--
        Zbigniew Masklak \\
        Film editing: IMAVI -- Joanna Kozakiewicz, Krzysztof Magda, Nikodem
        Frodyma


      \end{flushleft}

    \end{textblock}

  \end{frame}





  \begin{frame}[standout]


    \begingroup

    \color{jFrametitleFGColor}

    #1

    \endgroup

  \end{frame}
}











% ---------------------------------------
% Packages, libraries and their configuration
% ---------------------------------------
\usepackage{latexgeneralcommands}
\usepackage{mathcommands}





% ---------------------------------------
% Configuration for this particular presentation
% ---------------------------------------
\newcommand{\DIAG}{\textrm{DIAG}}
\newcommand{\divDegree}{\textrm{div}}










% ---------------------------------------------------------------------
\title{Renormalizacja Epsteina-Glasera i~regularyzacja
  wymiarowa w~algebraicznej kwantowej teorii pola na~przestrzeni
  Minkowskiego}

\author{Kamil Ziemian \\
  \texttt{kziemianfvt@gmail.com}}


\institute{Uniwersytet Jagielloński w~Krakowie}

\date[23 I 2015]{Seminarium Zakładu Teorii Pola, 23 stycznia 2015}
% --------------------------------------------------------------------










% ####################################################################
% Początek dokumentu
\begin{document}
% ####################################################################





% Wyrównanie do lewej z łamaniem wyrazów

\RaggedRight





% ######################################
\maketitle % Tytuł całego tekstu
% ######################################





% ######################################
\begin{frame}
  \frametitle{Plan prezentacji}


  \tableofcontents % Spis treści

\end{frame}
% ######################################










% ######################################
\section{Przedłużanie dystrybucji}
% ######################################



% ##################
\begin{frame}
  \frametitle{Problem mnożenia dystrybucji}


  \textbf{Ważna uwaga}
  Renormalizacja Epsteina-Glaser usuwa tylko rozbieżności w~UV
  (ultrafiolecie), zagadnienie rozbieżność IR (podczerwnoych)
  \textbf{zupełnie pomijamy}.

  \vspace{\spaceFour}



  \textbf{Mnożenie dystrybucji}
  \begin{align}
    \label{eq:Remornalizacja-E-G-01-A}
    &\delta( x ) \delta( y ) \textrm{ -- dobrze określona dystrybucja na }
      \Rbb^{ 2 }, \\
    \label{eq:Remornalizacja-E-G-01-B}
    &\delta^{ 2 }( x ) \textrm{ -- dobrze określona dystrybucja dla }
      x \neq 0.
  \end{align}

  \vspace{\spaceFour}



  \textbf{Dalsze problemy.}
  W~rachunku zaburzeń musimy mnożyć przez siebie propagatory Feynmana,
  który jest dobrze określoną dystrybucją na~$\Rbb^{ 2 }$. Znajomość fizyki
  podpowiada nam, że~jeśli położenia dwóch wierzchołków
  oddziaływań~się pokryją, może powstać problem.

\end{frame}
% ##################





% ##################
\begin{frame}
  \frametitle{Przedłużanie dystrybucji}


  \textbf{Renormalizacja.} Będziemy przez nią rozumieć przedłużenie
  dystrybucji określonej na~$\Dcal\left( ( \Rbb )^{ n } \setminus \DIAG \right)$,
  do~dystrybucji na~$\Dcal( ( \Rbb )^{ n } )$
  \begin{equation}
    \label{eq:Epstein-Glaser-02}
    \DIAG =
    \left\{ ( x_{ 1 }, x_{ 2 }, \ldots, x_{ n } ) \; | \;
      \exists\, i \neq j : x_{ i } = x_{ j } \right\}.
  \end{equation}

  \vspace{\spaceFour}



  \textbf{Kwestia jednoznaczności.}
  Jeśli dystrybucja $u \in \Dcal'\left( \Rbb \setminus \{ 0 \} \right)$ ma~rozszerzenie
  $\dot{ u } \in \Dcal'\left( \Rbb \right)$, to
  \begin{equation}
    \label{eq:Epstein-Glaser-03}
    \dot{ u }' = \dot{ u } + \sum_{ \alpha } \delta^{ ( \alpha ) }, \quad
    \alpha = ( \alpha_{ 1 }, \alpha_{ 2 }, \ldots, \alpha_{ d } ),
  \end{equation}
  też jest rozszerzeniem, przy czym suma jest w~powyższym równaniu
  jest skończona. Jest to jedyna swoboda jaka istnieje. Przez
  \textbf{regularyzację} będziemy rozumieli wprowadzenie parametru
  czyniącego przedłużenie dystrybucji jednoznacznym.

\end{frame}
% ##################





% ##########
\begin{frame}
  \frametitle{Twierdzenie o~przedłużaniu dystrybucji}


  \textbf{Wykładnik skalowania Steinmanna}
  \begin{equation}
    \label{eq:Epstein-Glaser-03}
    \sd =
    \inf \left\{ s \: : \: \forall \phi \in \Dcal,
      \lim_{ \rho \to 0^{ + } } \rho^{ s } \langle u, \rho^{ -d } \phi( \rho^{ -1 } x ) \rangle = 0 \right\}.
  \end{equation}
  Jest to uogólnienie stopnia jednorodności funkcji rzeczywistej,
  pozwala on~na~pewną charakteryzację przedłużeń dystrybucji.

  \vspace{\spaceFour}



  \textbf{Własności}
  \begin{itemize}

  \item $\sd( \partial_{ ( \alpha ) } u ) = \sd( u ) + | \alpha |$;

  \item $\sd( x^{ ( \alpha ) } u ) = \sd( u ) - | \alpha |$.

  \end{itemize}

  \vspace{\spaceFour}



  \textbf{Stopień rozbieżności dystrybucji} \\
  \begin{equation}
    \divDegree( u ) = \sd( u ) - d.
  \end{equation}

\end{frame}
% ##################





% ##################
\begin{frame}
  \frametitle{Twierdzenie o~przedłużaniu dystrybucji}


  \textbf{Brunetti, Fredenhagen [BF00]} \\
  Niech $u \in \Dcal'( \Rbb^{ d } \setminus \{ 0 \} )$. Jeśli:
  \begin{itemize}

  \item $\sd( u ) < d$, to~istnieje jednoznaczne przedłużenie
    takie, że $\sd( u ) = \sd( \dot{ u } )$;

  \item $d \leq \sd( u ) < \infty$,
    $\sd( u ) = \sd( \dot{ u } ) = \sd( \dot{ u }' )$ to
    \begin{equation}
      \label{eq:Epstein-Glaser-04}
      \dot{ u }' =
      \dot{ u } + \sum\limits_{ | \alpha | \leq \divDegree( u ) } C_{ \alpha } \delta^{ ( \alpha ) };
    \end{equation}

  \item $\sd( u ) = \infty$
    (np.~$u( f ) = \int e^{ \frac{ 1 }{ x } } f( x ) \, dx$),
    to~rozszerzenie nie istnieje.

  \end{itemize}

  \vspace{\spaceFour}



  \textbf{Zauważmy}
  \begin{equation}
    \label{eq:Epstein-Glaser-05}
    \sd( \delta^{ ( \alpha ) } ) = d + | \alpha |.
  \end{equation}

\end{frame}
% ##################





% ##################
\begin{frame}
  \frametitle{Dowód}


  \textbf{Przypadek} $\sd( u ) < d$ \\
  \begin{equation}
    \label{eq:Epstein-Glaser-06}
    \dot{ u } = \lim\limits_{ \rho \to +\infty } \eta( \rho x ) u,
  \end{equation}
  $\eta( x ) \in \Ccal^{ \infty }( \Rbb^{ d } )$, $\eta( x ) = 0$, jeśli $| x | < 1$,
  $\eta( x ) = 1$, jeśli $| x | > 2$.

  \vspace{\spaceFour}



  \textbf{Przypadek} $\sd( u ) \geq d$ \\
  Aby~sprowadzić do~poprzedniego przypadku definiujemy przestrzeń
  funkcji gładkich znikających do~rzędu
  $\lambda = \divDegree( u ) = \sd( u ) - d$:
  \begin{equation}
    \label{eq:Epstein-Glaser-07}
    \Dcal_{ \lambda }( \Rbb^{ d } ) =
    \{ \, f \in \Dcal( \Rbb^{ d } ) : \partial_{ \alpha } f( 0 ) = 0, \quad
    | \alpha | = \lambda \}.
  \end{equation}
  Korzystając z~odpowiedniej postaci reszty we~wzorze Taylora
  otrzymujemy:
  \begin{equation}
    \label{eq:Epstein-Glaser-08}
    f( x ) = \sum_{ | \alpha | = [ \lambda ] + 1 } x^{ \alpha }
    g_{ \alpha }( x ).
  \end{equation}

\end{frame}
% ##################





% ##################
\begin{frame}
  \frametitle{Dowód}


  \textbf{Przypadek} $\sd( u ) \geq d$ \\
  \begin{subequations}
    \begin{align}
      \label{eq:Epstein-Glaser-08-A}
      \Dcal_{ \lambda }( \Rbb^{ d } )
      &= \{ f \in \Dcal( \Rbb^{ d } ) : \partial_{ \alpha } f( 0 ) = 0, \quad | \alpha | = \lambda \}, \\
      \label{eq:EpsteinGlaser-08-B}
      f( x ) &= \sum_{ | \alpha | = [ \lambda ] + 1 } x^{ \alpha } g_{ \alpha }( x ), \\
      \label{eq:EpsteinGlaser-08-C}
      \sd( x^{ \alpha } u ) &= \sd( u ) - | \alpha |.
    \end{align}
  \end{subequations}

  \textbf{Zauważmy} \\
  Jeśli $\divDegree( u ) < | \alpha |$, to~$\sd( u ) - | \alpha | < d$.

  \vspace{\spaceFour}



  \textbf{Idea} \\
  Należy z~każdej funkcji wyciąć część która za~słabo gaśnie w~0.
  Jednak taka operacja nie jest jednoznaczna.
  \begin{equation}
    \label{eq:Epstein-Glaser-09}
    \dot{ u }' = \dot{ u } + \sum\limits_{ | \alpha | \leq \textrm{div}( u ) }
    C_{ \alpha } \delta^{ ( \alpha ) };
  \end{equation}
  $f \in \Dcal_{ \lambda }( \Rbb^{ d } )$. Ponieważ
  $\sd( u ) - ( [ \lambda ] + 1 ) < \sd( u ) - ( \sd( u ) - d ) = d$,
  więc możemy określić:
  \begin{equation}
    \label{eq:Epstein-Glaser-10}
    \langle \dot{ u }, f \rangle = \sum \langle x^{ \alpha } u, g_{ \alpha } \rangle.
  \end{equation}

\end{frame}
% ##################





% ##################
\begin{frame}
  \frametitle{Dowód}


  \textbf{Pomysł.}
  Aby zdefiniować $\dot{ u }$ na dowolnym $f \in \Dcal( \Rbb^{ d } )$ należy
  wyciąć z~$f$ człony nieznikające w~0.
  \begin{equation}
    \label{eq:Epstein-Glaser-11}
    f_{ \textrm{cut} }( x ) =
    f( x ) - \sum f^{ ( \alpha ) }( 0 ) w_{ \alpha }( x ),
  \end{equation}
  gdzie $w_{ \alpha } \in \Dcal( \Rbb^{ d } )$ jest rodziną funkcji
  „wycinających”:
  \begin{equation}
    \label{eq:Epstein-Glaser-12}
    \partial_{ \alpha } w_{ \beta }( 0 ) = \delta^{ \alpha }_{ \beta }.
  \end{equation}

  \vspace{\spaceFour}



  \textbf{Koniec dowodu}
  \begin{equation}
    \label{eq:Epstein-Glaser-13}
    \langle \dot{ u }, f \rangle =
    \langle u, f_{ \textrm{cut} } \rangle
    + \sum\limits_{ | \alpha | \leq \divDegree( u ) }
    \langle \dot{ u }, w_{ \alpha } \rangle \delta^{ \alpha }.
  \end{equation}
  Ta~równość \textbf{definiuje} $\langle \dot{ u }, w_{ \alpha } \rangle$.

\end{frame}
% ##################










% ######################################
\section{Regularyzacja wymiarowa w~przestrzeni położeń}
% ######################################



% ##################
\begin{frame}
  \frametitle{Regularyzacja wymiarowa dla~$d$~nieparzystego}


  \textbf{Propagator Feynmana}
  \begin{equation}
    \label{eq:Epstein-Glaser-14}
    \Delta_{ F }( x ) =
    \theta( x^{ 0 } ) H^{ \nu }( x ) + \theta( -x^{ 0 } ) H^{ \nu }( -x ) + f( x ),
  \end{equation}
  gdzie $H^{ \nu }$ jest funkcją Hadamarda,
  $\nu = \frac{ d }{ 2 } - 1$, $f( x )$ jest gładką funkcją.

  \vspace{\spaceThree}



  \textbf{Przestrzenna część funkcji Hadamarda}
  \begin{equation}
    \label{eq:Epstein-Glaser-15}
    H^{ \nu }( x ) =
    \frac{ ( 2 \pi )^{ 2 - \nu } m^{ \nu } }{ 4 \sin( \nu \pi ) }
    \left( -x^{ 2 } \right)^{ -\frac{ \nu }{ 2 } }
    I_{ -\nu }( \sqrt{ -m^{ 2 } x^{ 2 } } ), \, x^{ 2 } < 0,
  \end{equation}
  gdzie $I_{ \nu }$ i~$I_{ -\nu }$ to~funkcje Bessela pierwszego rodzaju.
  Używając metod analizy zespolonej można otrzymać stąd rozwiązanie
  dla~dowolnego wektora~$x$.

  \vspace{\spaceFour}



  Wzór powyższy ma~sens póki $\sin( \nu \pi ) \neq 0$,
  czyli~$\nu \in \Cbb \setminus \Nbb_{ 0 }$. Sugeruje
  to~naturalne uogólnienie procedury regularyzacji wymiarowej.

\end{frame}
% ##################





% ##################
\begin{frame}
  \frametitle{Regularyzacja wymiarowa dla~$d$~parzystego}


  \textbf{Przyjmujemy} \\
  \begin{equation}
    \label{eq:Epstein-Glaser-16}
    \nu = \frac{ d + \xi }{ 2 } - 1.
  \end{equation}

  \vspace{\spaceThree}



  \textbf{Regularyzacji wymiarowa funkcji Hadamarda}
  \begin{equation}
    \label{eq:Epstein-Glaser-17}
    \begin{split}
      H^{ \mu, \xi }( z )
      &= ( 2 \pi )^{ -\frac{ d }{ 2 } } m^{ -d }
        \left( m \sqrt{ -z^{ 2 } } \right)^{ 1 - \frac{ d + \xi }{ 2 } }
        \left[ \left( \frac{ m }{ \mu } \right) K_{ \frac{ d + \xi }{ 2 }
        - 1 }\left( m \sqrt{ -z^{ 2 } } \right) \, + \right. \\
      &\hphantom{=} + \, \left. ( -1 )^{ \frac{ d }{ 2 } - 1 } \frac{ \pi }{ 2
          \sin\left( \xi \frac{ \pi }{ 2 } \right) } \left\{ \left(
            \frac{ m }{ \mu } \right)^{ \xi } - 1 \right\} I_{ \frac{
            d + \xi }{ 2 } - 1 }\left( m\sqrt{ -z^{ 2 } } \right) \right].
    \end{split}
  \end{equation}
  Z~funkcji tej trzeba jeszcze wziąć wartość brzegową.

  Ze~względów wymiarowych do~formuły tej został wprowadzony parametr
  $\mu$ o~wymiarze masy.

\end{frame}
% ##################





% ##################
\begin{frame}
  \frametitle{Regularyzacja wymiarowa}


  \textbf{Istnieje granica}
  \begin{equation}
    \label{eq:Epstein-Glaser-18}
    \lim\limits_{ \xi \to 0 } \Delta_{ F }^{ \mu, \xi }( x, y ) \to
    \Delta_{ F }^{ \mu }( x, y ),
  \end{equation}
  i~daje dystrybucję na~$\Dcal\big( ( \Rbb^{ 2 } )^{ 2 } \big)$.

\end{frame}
% ##################










% ######################################
\section{Procedura indukcyjna Epsteina-Glasera}
% ######################################



% ##################
\begin{frame}
  \frametitle{Procedura indukcyjna Epsteina-Glasera}


  \textbf{Nieprzyjemna cecha rzeczywistości.}
  Ponieważ $\Delta_{ F }( x_{ 1 }, x_{ 2 } )$ jest dystrybucją, więc~sens
  ma~tylko jego całka z~odpowiednią „funkcją poprawiającą”
  $g( x ) \in \Scal( \Rbb^{ d } )$:
  \begin{equation}
    \label{eq:Epstein-Glaser-19}
    A( x_{ 1 }, x_{ 4 } ) =
    \int d x_{ 2 }^{ 4 } \, d x_{ 3 }^{ 4 } \; \Delta_{ F }( x_{ 1 }, x_{ 2 } )
    \Delta_{ F }( x_{ 2 }, x_{ 3 } ) \Delta_{ F }( x_{ 3 }, x_{ 4 } )
    g( x_{ 2 } ) g( x_{ 3 } ).
  \end{equation}
  Funkcja poprawiająca nie~ma sensu fizycznego dlatego też należy
  wziąć granicę $g( x ) \to 1$.

  \vspace{\spaceThree}



  \textbf{Problem.}
  Wzięcie granicy $g( x ) \to 1$ w~sposób spójny i~dający sensowny
  wynik, oznaczałoby rozwiązanie problemów podczerwonych. Funkcja
  $g( x )$ obcina oddziaływania na~„skończonej odległości”,
  eliminując ogony podczerwone.

\end{frame}
% ##################





% ##################
\begin{frame}
  \frametitle{Procedura indukcyjna Epsteina-Glasera}


  \textbf{Notacja} \\
  $E( \Gamma )$ oznaczać będzie zbiór linii grafu $\Gamma$,
  $V( \Gamma )$ --~zbiór wierzchołków, $\absOne{ V( \Gamma ) }$ --~liczbę
  wierzchołków.

  \textbf{Wymiarowo zregularyzowany propagator Feynmana} \\
  \begin{equation}
    \label{eq:Epstein-Glaser-20}
    \Delta_{ F }^{ m,\, \mu,\, \xi }( x, y ) =
    \theta( x^{ 0 } ) H^{ m,\, \mu,\, \xi }( x ) + \theta( -x^{ 0 } )
    H^{ m,\, \mu,\, \xi }( -x ),
  \end{equation}
  zaznaczyliśmy tu~jawnie zależność od~masy.

\end{frame}
% ##################





% ##################
\begin{frame}
  \frametitle{Procedura indukcyjna Epsteina-Glasera}


  \textbf{Macierz $S$.}
  Jest sumą iloczynów wyrażeń „algebraicznych” i~dystrybucji:
  \begin{equation}
    \label{eq:Epstein-Glaser-21}
    \widetilde{S}_{ \Gamma }^{ \mu,\, \xi } =
    \prod\limits_{ e \in E( \Gamma ) } \Delta_{ F }^{ m,\, \mu,\, \xi }( x_{ e }, y_{ e } ).
  \end{equation}
  $\widetilde{S}_{ \Gamma }^{ \mu,\, \xi }$ jest dystrybucją tylko
  na~$\Dcal( \Rbb^{ d | V( \Gamma ) | } \setminus \DIAG )$
  \begin{equation}
    \label{eq:Epstein-Glaser-22}
    \DIAG =
    \{ ( x_{ 1 }, x_{ 2 }, \ldots, x_{ n } ) \; | \; \exists\, i \neq j : x_{ i } = x_{ j } \}.
  \end{equation}
  Nie~jest więc określone co~się dzieje, gdy~dwa wierzchołki~się
  zetkną.

\end{frame}
% ##################





% ##################
\begin{frame}
  \frametitle{Procedura indukcyjna Epsteina-Glasera}


  Od~tej chwili interesujące nas grafy $\Gamma$ zawierają tylko wierzchołki
  i~linie je łączące.

  \textbf{Warunek przyczynowości.}
  Jeżeli wielkość zależna od~grafu
  $A_{ \Gamma }( g _{ 1 } + g_{ 2 } )$, zależy od~funkcji takich,
  że~nośniki $g_{ i }$~są przyczynowo odseparowane, to~musi zachodzić:
  \begin{equation}
    \label{eq:Epstein-Glaser-23}
    A_{ \Gamma }( g _{ 1 } + g_{ 2 } ) =
    A_{ \Gamma_{ 1 } }( g _{ 1 } ) A_{ \Gamma_{ 2 } }( g_{ 2 } ),
  \end{equation}
  jako mnożenie liczb.

\end{frame}
% ##################





% ##################
\begin{frame}
  \frametitle{Procedura indukcyjna Epsteina-Glasera}


  \begin{subequations}
    \begin{align}
      \label{eq:Epstein-Glaser-24-A}
      &A_{ \Gamma }( g _{ 1 } + g_{ 2 } ) =
        A_{ \Gamma_{ 1 } }( g _{ 1 } ) A_{ \Gamma_{ 2 } }( g_{ 2 } ), \\
      \label{eq:Epstein-Glaser-24-B}
      &\dot{ A }_{ \Gamma }' =
        \dot{ A }_{ \Gamma }
        + \sum\limits_{ \alpha \leq \textrm{div}( u ) } C_{ \alpha } \delta^{ ( \alpha ) }, \quad
        \sd( A_{ \Gamma } ) > d.
    \end{align}
  \end{subequations}

  \textbf{Dwa wierzchołki}
  \begin{equation}
    \label{eq:Epstein-Glaser-25}
    A_{ \Gamma }^{ \mu,\, \xi } =
    \Delta_{ F }^{ m,\, \mu,\, \xi }( x - y, y - x ) \Delta_{ F }^{ m,\, \mu,\, \xi }( x - y, y - x ),
  \end{equation}
  dystrybucja ta~jest określona
  na~$\Dcal( \Rbb^{ 2 d } \setminus \{ ( x, x ) \} )$. Używając
  niezmienniczości translacyjnej problemu dostajemy dystrybucję
  $\widetilde{ A }_{ \Gamma }^{ \mu,\, \xi }$ określoną
  na~$\Dcal( \Rbb^{ d } \setminus \{ 0 \} )$.

  \textbf{Renormalizacja}
  \begin{equation}
    \label{eq:Epstein-Glaser-26}
    \widetilde{ A }_{ \Gamma }^{ \mu } =
    \lim\limits_{ \xi \to 0 } ( \Delta_{ F }^{ m,\, \mu,\, \xi } )^{ 2 } ( 2 ( x - y ), 0 )
    + \sum_{ \alpha } C_{ \alpha } \delta^{ ( \alpha ) }.
  \end{equation}

\end{frame}
% ##################





% ##################
\begin{frame}
  \frametitle{Procedura indukcyjna Epsteina-Glasera}


  \textbf{Warunek przyczynowości}
  \begin{subequations}
    \begin{align}
      \label{eq:Epstein-Glaser-27-A}
      &A_{ \Gamma }( g _{ 1 } + g_{ 2 } ) =
        A_{ \Gamma_{ 1 } }( g _{ 1 } ) A_{ \Gamma_{ 2 } }( g_{ 2 } ), \\
      \label{eq:Epstein-Glaser-27-B}
      &\dot{ A }_{ \Gamma }' =
        \dot{ A }_{ \Gamma }
        + \sum\limits_{ \alpha \leq \textrm{div}( u ) } C_{ \alpha } \delta^{ ( \alpha ) }, \quad
        \sd( A_{ \Gamma } ) > d.
    \end{align}
  \end{subequations}


  \textbf{Trzy wierzchołki.}
  Umiemy już zetknąć dwa wierzchołki, niezdefiniowana jest operacja na
  trzech. Jednak warunek przyczynowości którym rozbijamy graf na
  iloczyn dwu- i~jednowierzchołkowego, prowadzi to do redukcji
  dystrybucji na $\Dcal( \Rbb^{ 3 d } \setminus \{ ( x, x, x ) \} )$ do
  dystrybucji na $\Dcal( \Rbb^{ 2 d } \setminus \{ 0 \} )$.

\end{frame}
% ##################





% ##################
\begin{frame}
  \frametitle{Procedura indukcyjna Epsteina-Glasera}


  \begin{subequations}
    \begin{align}
      \label{eq:Epstein-Glaser-28-A}
      &A_{ \Gamma }( g _{ 1 } + g_{ 2 } ) =
        A_{ \Gamma_{ 1 } }( g _{ 1 } ) A_{ \Gamma_{ 2 } }( g_{ 2 } ), \\
      \label{eq:Epstein-Glaser-28-B}
      &\dot{ A }_{ \Gamma }' =
        \dot{ A }_{ \Gamma }
        + \sum\limits_{ \alpha \leq \textrm{div}( u ) } C_{ \alpha } \delta^{ ( \alpha ) }, \quad
        \sd( A_{ \Gamma } ) > d.
    \end{align}
  \end{subequations}

  \textbf{Scaling expansion (Hollands, Wald)} \\
  \begin{equation}
    \label{eq:Epstein-Glaser-29}
    \prod_{ l = 1 }^{ k } \Delta_{ F }^{ m,\, \mu,\, \xi }( x_{ i }, y_{ i } ) =
    \sum_{ s_{ k } = 0 }^{ \infty } ( m^{ 2 } )^{ s_{ k } }
    \Delta_{ F }^{ \vec{ s }_{ k }, \mu, \xi }( x_{ 1 }, y_{ 1 }, \ldots,
    x_{ k }, y_{ k } ),
  \end{equation}
  przy czym,
  \begin{equation}
    \label{eq:Epstein-Glaser-30}
    \sd( \Delta_{ F }^{ \vec{ s }_{ k },\, \mu,\, \xi } ) =
    d + \Real( \xi ) - 2 - 2 s_{ k }.
  \end{equation}

  \textbf{Grupa renormalizacji} \\
  Swoboda dobierania stałych $C_{ \alpha }$ przy
  $\delta^{ ( \alpha ) }$ jest opisana grupą renormalizacji.

\end{frame}
% ##################





% ##################
\begin{frame}
  \frametitle{Bibliografia}


  \begin{itemize}
    \RaggedRight

  \item K. J. Keller, \textit{Dimensional Regularization in~Position
      Space and~a~Forest Formula for Regularized Epstein\dywiz Glaser
      Renormalization}, arXiv:~1006.2148v1.

  \item H. Epstein, V. Glaser, \textit{The~role~of locality
      in~perturbation theory}, Ann. Inst. H. Poincar\'{e} A
    \textbf{19} (1973) 211.

  \item R. Brunetti, K. Fredenhagen [BF00], \textit{Microlocal
      Analysis and~Interacting Quantum Field Theories: Renormalization
      on~Physical Backgrounds}, Commun.Math.Phys, \textbf{208} (2000)
    623-661, arXiv:~9903.028.

  \item G. Scharf, \textit{Finite quantum electrodynamics}, Springer,
    1995.

  \item R. Brunetti, K. Fredenhagen, \textit{Quantum Field
      Theory\linebreak on~Curved Backgrounds}, Proceedings of the
    Kompaktkurs \emph{Quantenfeldtheorie auf gekruemmten
      Raumzeiten}
    held at~Universitaet Potsdam, Germany, in~8--12.10.2007,
    arXiv:
    0901.2063.

  \item M. D\"{u}etsch, K. Fredenhagen, K. J.
    Keller, K.~Rejzner, \textit{Dimensional Regularization in~Position
      Space, and~a Forest Formula for Epstein-Glaser Renormalization},
    arXiv: 1311.5424, [DFKR13].

  \end{itemize}

\end{frame}
% ##################










% ####################################################################
% ####################################################################
% Bibliografia
\bibliographystyle{plalpha}

\bibliography{}{}





% ############################

% Koniec dokumentu
\end{document}
