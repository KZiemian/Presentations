% ------------------------------
% Paczki potrzebne w danym projekcie
% ------------------------------
% \usepackage{tabu}  % Lepsze tabele
\newcommand{\bmmax}{0}  % Górne ograniczenie na ilość "alfabetów
% matematycznych", które może wykorzystać paczka "bm" do tworzenia
% "Bold symbols". Zapobiega przekroczenia standardowej maksymalnej
% ilości jakie dopuszcza TeX (i pdfLaTeX), wynoszącej 16.
\newcommand{\hmmax}{0}
\usepackage{bm}  % Bold Math symbols, obecnie preferowany sposób.




% ------------------------------
% Pakiety do tekstów z nauk przyrodniczych
% ------------------------------
% \let\lll\undefined  % Amsmath gryzie się z pakietami do języka
% % polskiego, bo oba definiują komendę \lll. Aby rozwiązać ten problem
% % oddefiniowuję tę komendę, ale może tym samym pozbywam się dużego Ł.
\usepackage{amsmath}  % Podstawowe wsparcie od American
% Mathematical Society (w skrócie AMS)
% \usepackage{amsfonts, amssymb, amscd, amsthm}  % Dalsze wsparcie od AMS
\usepackage{amssymb, amscd, amsthm}  % Dalsze wsparcie od AMS
\usepackage{siunitx}  % Do prostszego pisania jednostek fizycznych
\usepackage{upgreek}  % Ładniejsze greckie litery
% Przykładowa składnia: pi = \uppi
% \usepackage{icomma} % Przecinek w trybie matematyczny ma właściwy odstęp



% ------------------------------
% TikZ
% ------------------------------
% \usepackage{tikz}  % Potężny pakiet PGF/TikZ.
% Włączenie konkretnych bibliotek pakietu TikZ
% \usetikzlibrary{arrows.meta}
% \usetikzlibrary{angles}
% \usetikzlibrary{3d}
% \usetikzlibrary{positioning}
% \usetikzlibrary{intersections}
% \usetikzlibrary{decorations.markings}
% \usetikzlibrary{calc}





% ------------------------------
% Komendy i ustawienia do wykładów ,,''.
% ------------------------------
\newcommand{\titlebackground}{./pictures/Clerk_Maxwell.jpg}
% Dla trybu Dark
% \newcommand{\sectionbackground}{./pictures/SzablonTytBluedark.png}
% Dla trybu Light
\newcommand{\sectionbackground}{./pictures/Newton.jpg}



\newcommand{\jagielloniantitlepage}{
  \begingroup

  \setbeamercolor{background canvas}{bg=JagiellonianBlue}

  \maketitle

  \endgroup

  \addtocounter{framenumber}{1}
}

\newcommand{\jagielloniantitlepagewithpicture}{
  \addtocounter{framenumber}{1}

  \begingroup
  \usebackgroundtemplate{ % \hspace*{-11.5em}
    \includegraphics[height=\paperheight]{\titlebackground}}

  \maketitle

  \endgroup
}


\newcommand{\jagielloniansectionwithpicture}[1]{
  \begingroup
  \usebackgroundtemplate{ % \hspace*{-11.5em}
    \includegraphics[height=\paperheight]{\sectionbackground}}

  \setbeamercolor{titlelike}{fg=normal text.fg}

  \section{#1}

  \endgroup
}



\newcommand{\jagiellonianendslide}[1]{
  \begin{frame}[standout]


    {\color{jStrongWhite} #1}

    % \begin{textblock}{5.5}(0.65,3)
    %   \includegraphics[scale=0.45]{./pictures/Fundusze_Europejskie.png}
    % \end{textblock}

    % \begin{textblock}{3}(3.65,3.9)
    %   \includegraphics{./pictures/CC-by-sa.png}
    % \end{textblock}

  \end{frame}
}





% ############################
% Beamer i LaTeX

% Zmiana koloru fontu matematycznego
\newcommand{\beamermathcolor}[2]{{\color{#1}\setbeamercolor{math text}
  {fg=#1} #2}}

% Zmiana koloru fontu na foreground BEAMER-koloru ,,normal text''
\newcommand{\normaltextcolor}[1]{{\color{normal text.fg}{#1}}}

% Zmiana koloru fontu matematycznego na jMathTextForegorundWhite
\newcommand{\beamermathcolorwhite}[1]{{\color{jStrongWhite}
    \setbeamercolor{math text}{fg=jStrongWhite} #1}}

\newcommand{\alertmath}[1]{{\color{jLightBrown}
  \setbeamercolor{math text}{fg=jLightBrown} #1}}


\newcommand{\colorhref}[2]{\href{#1}{\color{jLightBrown} #2}}





% ------------------------------
% Pakiety napisane do wykładów z ,,???'''.
% ------------------------------
% \usepackage{./SymulacjeFizykiPaczki/}
% \usepackage{./SymulacjeFizykiPaczki/SymulacjeFizyki}
