% Włączenie numerowania rozdziałów w spisie treści
\setbeamertemplate{section in toc}[sections numbered]
% Wybór rodzaju fontu???
\usefonttheme{serif}



% ------------------------------
% Podstawowe paczki (niezwiązane z ustawieniami języka)
% ------------------------------
% Ustawienia fontów (może zadziałają)
\usepackage{fontspec}  % Potrzebne do polskich znaków.
\setmainfont{TeX Gyre Pagella}

% \usepackage[utf8]{inputenc}  % Włączenie kodowania UTF-8, co daje dostęp
% \usepackage{tgpagella}  % Włącza fonty TeX Gyre Pagella

% \defaultfontfeatures{Ligatures=TeX}
% \newfontfeature{Microtype}{protrusion=default;expansion=default;}
% \usepackage{mathfix}  %
\usepackage{microtype}  % Twierdzi, że poprawi rozmiar odstępów w tekście.



\usepackage[polish]{babel}  % Tłumaczy na polski automatyczne teksty LaTeXa
% i pomaga z typografią.
\usepackage[MeX]{polski}  % Polonizacja LaTeXa, bez niej będzie pracował
% w języku angielskim.



% ------------------------------
% Podstawowe paczki (niezwiązane z ustawieniami języka)
% ------------------------------



% ------------------------------
% Lepsze wsparcie polskich znaków w takich miejscach jak konspekty PDFa
% ------------------------------
\hypersetup{pdfencoding=auto,psdextra}
