% Autor: Kamil Ziemian

% ---------------------------------------------------------------------
% Podstawowe ustawienia Beamera i używane pakiety
% ---------------------------------------------------------------------
\RequirePackage[l2tabu, orthodox]{nag}  % Wykrywa przestarzałe i niewłaściwe
% sposoby używania LaTeXa. Więcej jest w l2tabu English version.

\documentclass[10pt]{beamer}  % Klasa dokumentu
\mode<presentation>  % Rodzaj tworzonych slajdów Beamera
\usetheme[style=dark]{jagiellonian}  % Temat graficzny



% General configuration of jagiellonian
\setbeamertemplate{section in toc}[sections numbered]  % Numeracja
% pozycji w spisie treści
% Wybór fontu
\usefonttheme{serif}
\usepackage{mathpazo}  % Nowoczesna paczka do Palatino



\usepackage[polish]{babel}  % Tłumaczy na polski automatyczne teksty LaTeXa
% i pomaga z typografią.
\usepackage[MeX]{polski}  % Polonizacja LaTeXa, bez niej będzie pracował
% w języku angielskim.
\usepackage[utf8]{inputenc}  % Włączenie kodowania UTF-8, co daje dostęp
% do polskich znaków.
\usepackage[T1]{fontenc}  % Potrzebne do używania fontów Latin Modern.



% ------------------------------
% Podstawowe paczki (niezwiązane z ustawieniami języka)
% ------------------------------
% \usepackage{mathfix}   %
\usepackage{microtype}  % Twierdzi, że poprawi rozmiar odstępów w tekście.
\usepackage{graphicx}  % Wprowadza bardzo potrzebne komendy do wstawiania
% grafiki.



% ------------------------------
% Lepsze wsparcie polskich znaków w takich miejscach jak konspekty PDFa
% ------------------------------
\hypersetup{pdfencoding=auto,psdextra}

% ------------------------------
% Paczki potrzebne w danym projekcie
% ------------------------------
% \usepackage{tabu}  % Lepsze tabele
\newcommand{\bmmax}{0}  % Górne ograniczenie na ilość "alfabetów
% matematycznych", które może wykorzystać paczka "bm" do tworzenia
% "Bold symbols". Zapobiega przekroczenia standardowej maksymalnej
% ilości jakie dopuszcza TeX (i pdfLaTeX), wynoszącej 16.
\newcommand{\hmmax}{0}
\usepackage{bm}  % Bold Math symbols, obecnie preferowany sposób.




% ------------------------------
% Pakiety do tekstów z nauk przyrodniczych
% ------------------------------
% \let\lll\undefined  % Amsmath gryzie się z pakietami do języka
% % polskiego, bo oba definiują komendę \lll. Aby rozwiązać ten problem
% % oddefiniowuję tę komendę, ale może tym samym pozbywam się dużego Ł.
\usepackage{amsmath}  % Podstawowe wsparcie od American
% Mathematical Society (w skrócie AMS)
% \usepackage{amsfonts, amssymb, amscd, amsthm}  % Dalsze wsparcie od AMS
\usepackage{amssymb, amscd, amsthm}  % Dalsze wsparcie od AMS
\usepackage{siunitx}  % Do prostszego pisania jednostek fizycznych
\usepackage{upgreek}  % Ładniejsze greckie litery
% Przykładowa składnia: pi = \uppi
% \usepackage{icomma} % Przecinek w trybie matematyczny ma właściwy odstęp



% ------------------------------
% TikZ
% ------------------------------
% \usepackage{tikz}  % Potężny pakiet PGF/TikZ.
% Włączenie konkretnych bibliotek pakietu TikZ
% \usetikzlibrary{arrows.meta}
% \usetikzlibrary{angles}
% \usetikzlibrary{3d}
% \usetikzlibrary{positioning}
% \usetikzlibrary{intersections}
% \usetikzlibrary{decorations.markings}
% \usetikzlibrary{calc}





% ------------------------------
% Komendy i ustawienia do wykładów ,,''.
% ------------------------------
\newcommand{\titlebackground}{./pictures/Clerk_Maxwell.jpg}
% Dla trybu Dark
% \newcommand{\sectionbackground}{./pictures/SzablonTytBluedark.png}
% Dla trybu Light
\newcommand{\sectionbackground}{./pictures/Newton.jpg}



\newcommand{\jagielloniantitlepage}{
  \begingroup

  \setbeamercolor{background canvas}{bg=JagiellonianBlue}

  \maketitle

  \endgroup

  \addtocounter{framenumber}{1}
}

\newcommand{\jagielloniantitlepagewithpicture}{
  \addtocounter{framenumber}{1}

  \begingroup
  \usebackgroundtemplate{ % \hspace*{-11.5em}
    \includegraphics[height=\paperheight]{\titlebackground}}

  \maketitle

  \endgroup
}


\newcommand{\jagielloniansectionwithpicture}[1]{
  \begingroup
  \usebackgroundtemplate{ % \hspace*{-11.5em}
    \includegraphics[height=\paperheight]{\sectionbackground}}

  \setbeamercolor{titlelike}{fg=normal text.fg}

  \section{#1}

  \endgroup
}



\newcommand{\jagiellonianendslide}[1]{
  \begin{frame}[standout]


    {\color{jStrongWhite} #1}

    % \begin{textblock}{5.5}(0.65,3)
    %   \includegraphics[scale=0.45]{./pictures/Fundusze_Europejskie.png}
    % \end{textblock}

    % \begin{textblock}{3}(3.65,3.9)
    %   \includegraphics{./pictures/CC-by-sa.png}
    % \end{textblock}

  \end{frame}
}





% ############################
% Beamer i LaTeX

% Zmiana koloru fontu matematycznego
\newcommand{\beamermathcolor}[2]{{\color{#1}\setbeamercolor{math text}
  {fg=#1} #2}}

% Zmiana koloru fontu na foreground BEAMER-koloru ,,normal text''
\newcommand{\normaltextcolor}[1]{{\color{normal text.fg}{#1}}}

% Zmiana koloru fontu matematycznego na jMathTextForegorundWhite
\newcommand{\beamermathcolorwhite}[1]{{\color{jStrongWhite}
    \setbeamercolor{math text}{fg=jStrongWhite} #1}}

\newcommand{\alertmath}[1]{{\color{jLightBrown}
  \setbeamercolor{math text}{fg=jLightBrown} #1}}


\newcommand{\colorhref}[2]{\href{#1}{\color{jLightBrown} #2}}





% ------------------------------
% Pakiety napisane do wykładów z ,,???'''.
% ------------------------------
% \usepackage{./SymulacjeFizykiPaczki/}
% \usepackage{./SymulacjeFizykiPaczki/SymulacjeFizyki}






% ------------------------------
% Textpos - TEXT POSition, do dobrego zarządzania ułożeniem obiektów
% na ,,kartce'' tekstu
% ------------------------------
\usepackage[showboxes,overlay,absolute]{textpos}
\setlength{\TPHorizModule}{20mm}
\setlength{\TPVertModule}{\TPHorizModule}
% \textblockorigin{10mm}{10mm}





% ------------------------------
% Paczki, biblioteki i ich ustawienia dla tego pliku
% ------------------------------
\usepackage{tikz}
% \usepackage{tkz-euclide}

\usepackage{verse}
\usepackage{chemfig}
\usepackage{algorithm2e}





% ------------------------------
% Pakiety napisane przez użytkownika.
% Mają być w tym samym katalogu to ten plik .tex
% ------------------------------
\usepackage{latexshortcuts}
\usepackage{mathshortcuts}










% ---------------------------------------------------------------------
\title{Ti\emph{k}Z}

\author{Kamil Ziemian \\
  \texttt{kziemianfvt@gmail.com}}

% \institute{Jagiellonian University in~Cracow}
% \institute{Uniwersytet Jagielloński w~Krakowie}

% \date[16 March 2020]% {Seminarium astrofizyczne PAU \\
  % 11 grudnia 2019}
% ---------------------------------------------------------------------










% ####################################################################
% Początek dokumentu
\begin{document}
% ####################################################################



% ######################################
\maketitle % Tytuł całego tekstu
% ######################################



% ##################
\begin{frame}
  \frametitle{Hello World}

  \begin{figure}
    % \centering

    \includegraphics[scale=0.69]{./SeminarPictures/DEK.jpg}
    \includegraphics[scale=0.5]{./SeminarPictures/LL.jpg}

    \caption{Donald Ervin Knuth (1938~--~), Leslie Lamport (1941~--~).}

  \end{figure}

\end{frame}
% ##################





% ##################
\begin{frame}
  \frametitle{Enter the master}


  W~1977 roku Donald E.~Knuth, legenda informatyki, stwierdził, że~aby
  w~pełni wykorzystać potencjał, coraz szerzej stosownego wtedy składu
  cyfrowego oraz przeciwdziałać obniżającej~się jakości typograficznej
  (wprowadzenie nowej lepszej i~prostszej technologi zwykle obniża
  jakość wykonania), stworzył własny język programowania i~program
  do~składu tekstu. Temu programowi nadał nazwę \TeX. Kolejne wersje
  \TeX a~są zbieżne do $\pi$, obecna to $3.14159265$ (to typowe dla
  Knutha).

  \TeX{} to zapisana w~alfabecie łacińskim pierwsza sylaba
  ($\tau\varepsilon\chi$) słowa $\tau\varepsilon\chi\nu\eta$ (techne), od~którego pochodzi słowo technika
  i~technologia. Po grecku to oznacza czynności takie jak budowa
  statków, kowalstwo, malowanie, rzeźbienie czy hodowla koni.
  $\pi o\iota\eta\sigma\iota\zeta$, poiesis = gr.~robić, jest czymś innym, może dlatego,
  że~jak powiada Sokrates w~pierwszym przykładzie fanowskiego
  dziennikarstwa muzycznego, pod tytułem ,,Ion'', poeci nic nie
  umieją.

\end{frame}
% ##################





% ##################
\begin{frame}
  \frametitle{Odrobina historia}

  Zapewne dlatego, że~pisanie w~samym \TeX u jest dla wielu zbyt
  skomplikowane
  jak~na~tworzenie plików tekstowych, na początku lat 80 XX~wieku, \\
  Leslie Lamport tworzy nakładkę na~\TeX a, która zajmuje~się za~nas
  większością \TeX nicznych szczegółów (teraz możecie odetchnąć
  z~ulgą). Otrzymany w~ten sposób system składu nosi nazwę \LaTeX,
  obecna wersja to~\LaTeX{} 2$_{ \varepsilon }$.

  \emph{La} jest od Lamporta, stąd powinno~się mówić \emph{la\dywiz
    tech}, \emph{lej\dywiz tech}, sam Donald Knuth tego nie wie.
  \LaTeX{} 2$_{ \varepsilon }$ czyta się jako \emph{la\dywiz tech dwa~epsilon}.

\end{frame}
% ##################





% ##################
\begin{frame}
  \frametitle{O nadziei: dialog w pytaniach i~odpowiedziach pomiędzy \\
    A.~Cowleyem i~R.~Crashawem, fragment \\
    (tłumaczył Stanisław
    Barańczak)}

  \settowidth{\versewidth}{Zarówno, gdyś spełniona, jak i
    zawiedziona!} { \tiny
    \begin{verse}[\versewidth]
      \setlength{\vrightskip}{-10em} \poemlines{30}

      C o w l e y \\
      \vin Nadziejo, której kruchy żywot kona \\
      Zarówno, gdyś spełniona, jak i zawiedziona! \\
      Ty, którą Zło i Dobro niszczy w równej mierze, \\
      A Los wiecznie na rogi alternatyw bierze. \\
      \vin Czczym cieniem jesteś: doszczętnie niknąca \\
      \vin Tak w noc najgłębszą, jak przy blasku słońca. \\
      \vin Ze wszelkich błogosławieństw, które Los nam rodzi, \\
      \vin\vin Żadne cię nie nagrodzi. \\
      Jeśli sądzić po skutkach wszystko, co istnieje, \\
      Najbardziej beznadziejne są ludzkie nadzieje. \\!

      C r a s h a w \\
      \vin Nadziejo! Tyś jest Niebios długiem, Ziemi darem, \\
      Nie zaistniałych rzeczy realnych wymiarem. \\
      Najpewniejsza, choć najmniej uchwytna! Zaiste, \\
      Dzięki tobie Nic nasze jest tak wyraziste. \\
      \vin Chmuro z ognia, i cieniem i światłem będąca, \\
      \vin Tyś naszym życiem w śmierci, w nocy -- blaskiem słońca. \\
      \vin Ze wszelakich nieszczęść, które Los w nadmiarze płodzi, \\
      \vin\vin Żadne ci nie zaszkodzi. \\
      Los na twój widok chowa swoje tępe rogi, \\
      Jak słabowity księżyc w czas poranku błogi. \\!

    \end{verse}
  }
\end{frame}
% ##################





% ##################
\begin{frame}[fragile]
  \frametitle{Wzory chemiczne}


  \chemfig{H-[:52.24]\lewis{1:3:,O}-[::-104.48]H}

  \chemfig{C(-[:0]H)(-[:90]H)(-[:180]H)(-[:270]H)}

\end{frame}
% ##################





% ##################
\begin{frame}[fragile]
  \frametitle{Algorytmy}


\begin{algorithm}[H]
 \KwData{this text}
 \KwResult{how to write algorithm with \LaTeX2e }
 initialization\;
 \While{not at end of this document}{
  read current\;
  \eIf{understand}{
   go to next section\;
   current section becomes this one\;
   }{
   go back to the beginning of current section\;
  }
 }
 \caption{How to write algorithms}
\end{algorithm}

\end{frame}
% ##################





% ##################
\begin{frame}
  \frametitle{Typy procesorów tekstu (brr, co za groźne nazwa)}


    \begin{itemize}
    \item[--] WYSIWYG = \emph{what you see is what you get} (to co
      widzisz, jest tym co~dostaniesz). Przykłady: MS Word,
      OpenOffice.

    \item[--] WYSIWYM = \emph{what you see is what you mean} (to co
      widzisz, to jest to o~czym myślałeś). Przykłady: \LaTeX.

    \end{itemize}

    Jeśli zaczynasz pisać w~\LaTeX u, to jest to raczej WYWINWYG =
    \emph{what you want is not what you get} (to czego chciałeś, to
    nie jest to co dostałeś). Ale~od tego~są te~warsztaty, aby~ułatwić
    wam przejście tego trudnego początkowego etapu. Jak w~takim
    wypadku wygląda plika \LaTeX a?

\end{frame}
% ##################





% % ##################
% \begin{frame}[fragile]
%   \frametitle{Nasz pierwszy tekst}

%   \begin{block}{Klasyczna szkoła}
% \begin{verbatim}
% \documentclass{article}



% \begin{document}



% Hello World!



% \end{document}
% \end{verbatim}
%   \end{block}

% \end{frame}
% % ##################





% % ##################
% \begin{frame}
%   \frametitle{Białe znaki w~\LaTeX u}


%     \LaTeX{}
%     traktuje jedną spację i~100 spacji tak samo: jako jeden odstęp.
%     Tab traktuje jako spację. Przejście do nowej linii, \red{traktuje
%       jako jeden odstęp.} Jedną pustą linię traktuje jako komendę,
%     że~w~tym miejscu ma {\color{red} skończyć akapit} (częsty błąd
%     początkujących), zaś~100 pustych linii jak jedną pustą linię.

% \end{frame}
% % ##################





% % ##################
% \begin{frame}
%   \frametitle{Białe znaki w~\LaTeX u}


%   \begin{tikzpicture}
%     \tkzInit[xmax=6,ymax=6,xmin=-6,ymin=-6]
%    \tkzGrid
%    \tkzAxeXY
%    % \draw[ thick,latex-latex] (-1,4) -- (4,-6) node[anchor=south west] {$a$}; % two points for drawing 2x+y=2
%    \tkzText[above](0,6.75){Desired Output}
%   \end{tikzpicture}

% \end{frame}
% % ##################


% % ##################
% \begin{frame}
%   \frametitle{Białe znaki w~\LaTeX u}

%   \begin{block}{Organizacja tekstu za pomocą białych znaków}
%     Plik dobrze napisany, to plik który łatwiej~się poprawia. A~każdy
%     musi go~poprawiać.
%     \begin{itemize}
%     \item[--] Ponieważ przejście do~nowej to tyle samo co spacja,
%       warto dzielić tekst na~krótsze linie (ja stosuje standard 79
%       znaków). Dzięki temu nie musimy scrolować tekstu by~go zmienić,
%       wszystko jest na ekranie. \\
%       Poza tym komentarze lepiej działają, ale~o~tym za chwilę.
%     \item[--] Ja wyznaję zasadę, że~dwie ważne części pliku źródłowego
%       powinien oddzielać odstęp 3~pustych linii.
%     \item[--] Pomiędzy mniej ważnymi fragmentami powinny znajdować~się
%       2~lub~1 puste linie. Przy założeniu, że~przejście do~nowego
%       akapitu nie psuje wyglądu tekstu.
%     \end{itemize}
%   \end{block}

%   \begin{block}{Ważne}
%     Nie musicie tego robić tak jak ja, jednak gorąco polecam przyjąć
%     jakiś standard. To~niezmiernie ułatwia pracę z~plikiem źródłowym.
%   \end{block}

% \end{frame}
% % ##################

% % ######################################
% \begin{frame}
%   \frametitle{Za i~przeciw używania \LaTeX a}


%   \begin{block}{Przeciw}
%     \begin{enumerate}
%     \item Mój kolega miał wygłosić seminarium na temat
%       \emph{O~wyższości pisania w~Wordzie nad \LaTeX em}; niestety
%       pozostało ono w~sferze planów, więc konkretnych argumentów nie
%       znam.
%     \item Aby móc go używać, trzeba~się trochę rzeczy nauczyć.
%     \item Na początku pliki bez przerwy nie będą ci~się działać.
%     \item A~jak już~zadziałają, to nie będą zawierać tego co
%       chciałeś.
%     \item Rozdziela formę od treści.
%     \end{enumerate}
%   \end{block}

% \end{frame}



% \begin{frame}
%   \frametitle{Za i~przeciw używania \LaTeX a}

%   \begin{block}{Za (ta prezentacja jest zrobiona w~\LaTeX u)}
%     \begin{enumerate}
%     \item Jest naprawdę ładny.
%     \item Kiedy~się go nauczysz, większość rzeczy zrobi za ciebie.
%     \item Możesz w~nim zmienić praktycznie wszystko co chcesz.
%     \item Ponieważ ktoś zapewne już zrobił to za ciebie, więc musisz
%       tylko
%       użyć \\
%       {\color{blue} doktora Google}.
%     \item % Jeśli~się do niego podejście
%       Przy odrobinie sprytu, znacznie upraszcza życie.
%     \item Rozdziela formę od treści.
%     \end{enumerate}
%   \end{block}

% \end{frame}



% \begin{frame}
%   \frametitle{Komiks na dobry początek}


%   \begin{figure}
%     %     \centering
%     \includegraphics[scale = 0.4]{Programing_language.jpg}
%     \caption{Cała prawda o~\LaTeX u,
%     \colorlink{http://www.buzzfeed.com/lukelewis/28-things-only-developers-will-find-funny}
%     (kliknij w~niego).}
%   \end{figure}

% \end{frame}



% \begin{frame}
%   \frametitle{Co można zrobić w~\LaTeX u?}

%   \begin{block}{Na~przykład}
%     \begin{itemize}
%     \item[--] Tą prezentację.
%     \item[--] Pracę licencjacką, magisterską, doktorską, etc.
%     \item[--] Artykuły, eseje, felietony.
%     \item[--] Sprawozdania na pracownię.
%     \item[--] Książkę.
%     \item[--] Erratę do~książki;).
%     \item[--] Grafikę matematyczną: \\
%       \vspace{1em}
%       \begin{center}
%         \setlength{\unitlength}{0.65mm}
%         \begin{picture}
%           (60,40)
%           \put(30,20){\vector(1,0){30}}
%           \put(30,20){\vector(4,1){20}} \put(30,20){\vector(3,1){25}}
%           \put(30,20){\vector(2,1){30}} \put(30,20){\vector(1,2){10}}
%           \thicklines \put(30,20){\vector(-4,1){30}}
%           \put(30,20){\vector(-1,4){5}} \thinlines
%           \put(30,20){\vector(-1,-1){5}}
%           \put(30,20){\vector(-1,-4){5}}
%         \end{picture}
%       \end{center}
%     \end{itemize}
%   \end{block}

% \end{frame}



% \begin{frame}
%   \frametitle{Co można robić w~\LaTeX u?}

%   \begin{block}{Na~przykład}
%     \begin{itemize}
%     \item[--] Grafy Feynmana (choć to sprawa niebanalna): \\
%       \begin{center}
%         \feynmandiagram [horizontal=a to b] { i1 -- [fermion] a --
%         [fermion] i2, a -- [photon] b, f1 -- [fermion] b -- [fermion]
%         f2, };
%       \end{center}
%     \item[--] Wzory chemiczne: \\
%       \begin{center}
%         \chemfig{A*5(-B=C-D-E=)}
%       \end{center}
%     \item[--] Jak~się postarać, to można pisać po~hebrajsku
%       (hieroglify też~się da).
%     \item[--] Ogólnie rzecz biorąc wszystko, tylko trzeba
%       {\color{purple} pogooglować}.
%     \end{itemize}
%   \end{block}

% \end{frame}
















% \section{Jak pisać w~\LaTeX u?}



% \begin{frame}
%   \frametitle{Kilka rad na~dobry początek}

%   \begin{block}{Mądrości programistów}
%     Tworzenie tekstu w~\LaTeX u to trochę jak programowanie, więc
%     warto pamiętać o~kilku mądrościach programistów.
%     \begin{itemize}
%     \item[--] Najtrudniejsza rzecz to skompilować pierwszy tekst do
%       PDFa. Potem jest już z~górki.
%     \item[--] Aby nauczyć~się pisać w~\LaTeX u, trzeba pisać w~\LaTeX
%       u.
%     \item[--] Twój główny wróg na samym początku to literówki, \\
%       np.~\tb{\tbs beign} zamiast \tb{\tbs begin}. Nie zniechęcaj~się
%       tym.
%     \item[--] To jak ,,kody'' \LaTeX a wygląda ma znaczenie. Pisz go
%       tak, by w~jakimś stopniu oddawał logikę tekstu.
%     \end{itemize}
%   \end{block}

%   \begin{block}{Przykład}       % C-x C-+
%     Niech będzie nim ten slajd.
%   \end{block}

% \end{frame}



% \begin{frame}
%   \frametitle{Kilka rad na~dobry początek}

%   \begin{block}{Mądrości programistów}
%     \begin{itemize}
%     \item[--] Naucz~się pisać bezwzrokowo, to wcale nie jest trudne.
%     \item[--] Dobre środowisko to skarb. Poznaj jego podstawowe skróty
%       klawiszowe i~możliwości.
%     \end{itemize}
%   \end{block}

%   \begin{block}{Polecane środowiska}
%     \begin{enumerate}
%     \item \TeX Maker;
%     \item Kile (warto połączyć z~następnym);
%     \item Share\LaTeX.com, \colorlink{https://www.sharelatex.com/};
%     \item WinEdit;
%     \item OverLeaf.com, \colorlink{https://www.overleaf.com/};
%     \item Vim z~\LaTeX suite i~Vim\dywiz\LaTeX;
%     \item Lyx;
%     \item Emacs z AUC\TeX\dywiz em.
%     \end{enumerate}
%   \end{block}


% \end{frame}




% \begin{frame}[fragile]
%   \frametitle{Nasz pierwszy tekst}

%   \begin{block}{Plik źródłowy}
%     W~edytorze tekstu (MS Word, OpenOffice~się nie nadają) należy
%     utworzyć plik z~rozszerzeniem ,,.tex'', tak zwany plik źródłowy.
%     Niech nosi nazwę ,,Nasz-pierwszy-tekst-w-LaTeXu.tex''. Niektóre
%     dystrybucje \LaTeX a nie zaakceptują pliku ,,Nasz pierwszy tekst
%     w~LaTeXu.tex''.
%   \end{block}

%   \begin{block}{Kompilacja}
%     Aby powstał gotowy dokument, najlepiej w~formacie PDF, musimy
%     skompilować (groźne słowo związane z~takimi rzeczami jak C/C++
%     i~Java, ale~nie musicie~się tego bać) plik źródłowy. Zwykle to~się
%     nam nie uda i~kompilator wyrzuci komunikat o~błędach.
%   \end{block}

%   \begin{block}{W~dobrze skonfigurowany środowisku}
%     Wystarczy wcisnąć jeden klawisz, aby~skompilować plik i~wieloma
%     rzeczami nie trzeba~się martwić. W~\TeX Makerze jest to~F1.
%   \end{block}

% \end{frame}



% \begin{frame}[fragile]
%   \frametitle{Nasz pierwszy tekst}

%   \begin{block}{Klasyczna szkoła}
% \begin{verbatim}
% \documentclass{article}



% \begin{document}



% Hello World!



% \end{document}
% \end{verbatim}
%   \end{block}

% \end{frame}



% \begin{frame}[fragile]
%   \frametitle{Nasz pierwszy tekst}

%   \begin{block}{Nowa szkoła (w~pierwszej linii jest L2TABU)}
% \begin{verbatim}
% \RequirePackage[l2tabu, orthodox]{nag}
% \documentclass{article}



% \begin{document}



% Hello World!



% \end{document}
% \end{verbatim}
%   \end{block}

% \end{frame}



% \begin{frame}
%   \frametitle{Wynik kompilacji}

%   \begin{block}{Pliki generowane}
%     \LaTeX{} tworząc plik PDFa produkuje masę plików z~rozszerzeniami
%     ,,.aux'', ,,.bbl'', ,,.blg'', ,,.log'', ,,.nav'', na~szczęście
%     w~większości przypadków nie musimy~się nimi przejmować.
%   \end{block}

%   \begin{block}{Ale czasem}
%     Rzadko, bo rzadko, ale~się zdarza, że~choć wszystko w~pliku
%     źródłowym jest~dobrze, to~z~powodu błędu w~którymś z~tych
%     dodatkowych plików, kompilacja nie może dojść do skutku. Wtedy
%     trzeba usunąć je wszystkie i~skompilować jeszcze
%     raz plik źródłowy.
%   \end{block}

%   \begin{block}{\red{Uwaga}}
%     Pamiętajcie by~przy okazji nie usunąć pliku z~rozszerzeniem
%     \red{,,.tex''!!!} Dobre środowisko ma wbudowaną opcję, usunięcia
%     wszystkich zbędnych plików za nas. Np.~w~\TeX Makerze mamy: Narzędzia
%     $\ra$ Wyczyść.
%   \end{block}

% \end{frame}



% \begin{frame}[fragile]
%   \frametitle{Nasz pierwszy tekst}

%   \begin{block}{Podział pliku źródłowego}
%     \begin{enumerate}
%     \item Preambuła.
%     \item Część główna.
%     \end{enumerate}
%   \end{block}

%   \begin{block}{Preambuła}
%     Preambuła zaczyna~się w~pierwszej linii pliku, zaś kończy na
%     {\color{purple} \verb+\begin{document}+}. Zawiera informacje,
%       które służą \LaTeX owi stworzenia gotowego tekstu, rozszerzają
%       jego standardowe możliwości \\
%       (na chyba nieograniczoną liczbę sposobów) oraz komendy które
%       sami zdefiniowaliśmy.
% \begin{verbatim}
% \RequirePackage[l2tabu, orthodox]{nag}
% \documentclass{article}

% \begin{document}
% \end{verbatim}
%   \end{block}

% \end{frame}



% \begin{frame}[fragile]
%   \frametitle{Co robi ta preambuła?}

%   \begin{block}{Linia po linii}
%     \begin{enumerate}
%     \item \verb+\RequirePackage[l2tabu, orthodox]{nag}+ --~na razie
%       najlepiej potraktować to jako magię dla zaawansowanych.
%       W~procesie kompilacji wykrywa przestarzała lub niepotrzebne
%       paczki, które od~czasów starożytnych mistrzów \LaTeX a kopiujemy
%       do~swojej preambuły oraz~błędnie stosowane komendy \LaTeX a,
%       wszystko to~według standardów opisanych w~,,An essential guide
%       to \LaTeX$2_{ \veps }$ usage'' Marka Trettina (popularnie
%       l2tabu,
%       \colorlink{ftp://sunsite.icm.edu.pl/pub/CTAN/info/l2tabu/english/l2tabuen.pdf}).
%       Np.~w~tej prezentacji \tb{nag} wykrył, że~paczka ,,epsfig'' jest
%       przestarzała i~należy ją zmienić na ,,graphicx''.
%     \item \verb+\documentclass{article}+ --~definiuje klasę dokumentu \\
%       (co za odkrycie), czyli najbardziej ogólne reguły tworzenia
%       danego dokumentu przez \LaTeX a. Podstawowe, klasyczne klasy
%       \LaTeX a to \tb{article}, \tb{report}, \tb{book}, \tb{letter},
%       plus niekończenie wiele dodatkowych, a~jeśli jesteś odpowiednio
%       dobry, możesz napisać też swoją. \\
%       Ja używam \tb{article} w~99.99\% przypadków.
%     \end{enumerate}
%   \end{block}

% \end{frame}



% \begin{frame}[fragile]
%   \frametitle{Co robi ta preambuła?}

%   \begin{block}{Polecenia \LaTeX a}
%     \LaTeX, jak przystało na język programowania, posiada odpowiednie
%     polecenia, które zawsze zaczynają~się od znaku ,,\tbs''
%     (backslash, ,,w\dywiz tył\dywiz ciach''). Co zaczyna~się symbolem
%     ,,\tbs'' jest poleceniem \LaTeX a. Widzimy więc, że~poleceniami~są
%     \verb+\RequirePackage[l2tabu, orthodox]{nag}+
%     i~\verb+\documentclass{article}+.
%   \end{block}

%   \begin{block}{Polecenia w~preambule}
%     Najbardziej podstawowe wyglądają tak
%     \begin{center}
%       \verb+\nazwa-polecenia+[\emph{argumenty-dodatkowe}]\{\emph{argumenty-obowiązkowe}\}
%     \end{center}
%     Argumenty dodatkowe należy oddzielać przecinkami i~można podawać
%     w~dowolnej kolejności.
%   \end{block}

% \end{frame}



% \begin{frame}[fragile]
%   \frametitle{Pierwsza przeróbka preambuły}

%   \begin{block}{Polecenia w~preambule}
%     Najbardziej podstawowe wyglądają tak
%     \begin{center}
%       \verb+\nazwa-polecenia+[\emph{argumenty-dodatkowe}]\{\emph{argumenty-obowiązkowe}\}
%     \end{center}
%     Argumenty dodatkowe należy oddzielać przecinkami i~można podawać
%     w~dowolnej kolejności.
%   \end{block}

%   \begin{block}{Zróbmy coś takiego}
%     \verb+\documentclass[a4paper, 12pt]{article}+ \\
%     Widzicie różnicę? Pt = point, jednak ze standardowych jednostek
%     składu tekstu, równa 1/72 cala. Cal to 2.54 centymetra, więc
%     w~bardziej zrozumiałych jednostkach 1 pt = 3.52 mm.
%   \end{block}

% \end{frame}



% \begin{frame}[fragile]
%   \frametitle{Wróćmy do~tekstu}

%   \begin{block}{Część główna}
%     Zawiera tekst, który (po obrobieniu przez \LaTeX a) znajdzie~się
%     w~gotowym dokumencie.
%     \begin{verbatim}
% \begin{document}



% Hello World!



% \end{document}
% \end{verbatim}
%   \end{block}

% \end{frame}



% \begin{frame}
%   \frametitle{Białe znaki w~\LaTeX u}

%   \begin{block}{}
%     Białe znaki to \tb{spacja}, \tb{tab} i~\tb{enter}. \LaTeX{}
%     traktuje jedną spację i~100 spacji tak samo: jako jeden odstęp.
%     Tab traktuje jako spację. Przejście do nowej linii, \red{traktuje
%       jako jeden odstęp.} Jedną pustą linię traktuje jako komendę,
%     że~w~tym miejscu ma {\color{red} skończyć akapit} (częsty błąd
%     początkujących), zaś~100 pustych linii jak jedną pustą linię.
%   \end{block}

% \end{frame}



% \begin{frame}
%   \frametitle{Białe znaki w~\LaTeX u}

%   \begin{block}{Organizacja tekstu za pomocą białych znaków}
%     Plik dobrze napisany, to plik który łatwiej~się poprawia. A~każdy
%     musi go~poprawiać.
%     \begin{itemize}
%     \item[--] W~preambule białe znaki nie mają znaczenie, co nie
%       znaczy, że~nie należy ich używać. Wręcz przeciwnie!
%     \item[--] Ponieważ przejście do~nowej to tyle samo co spacja,
%       warto dzielić tekst na~krótsze linie (ja stosuje standard 79
%       znaków). Dzięki temu nie musimy scrolować tekstu by~go zmienić,
%       wszystko jest na ekranie. \\
%       Poza tym komentarze lepiej działają, ale~o~tym za chwilę.
%     \item[--] Ja wyznaję zasadę, że~dwie ważne części pliku źródłowego
%       powinien oddzielać odstęp 3~pustych linii.
%     \item[--] Pomiędzy mniej ważnymi fragmentami powinny znajdować~się
%       2~lub~1 puste linie. Przy założeniu, że~przejście do~nowego
%       akapitu nie psuje wyglądu tekstu.
%     \end{itemize}
%   \end{block}

% \end{frame}



% \begin{frame}[fragile]
%   \frametitle{Przykład}

%   \begin{block}{Stąd taki dziki wygląd tego pliku}

%   \begin{verbatim}
% \RequirePackage[l2tabu, orthodox]{nag}
% \documentclass{article}



% \begin{document}



% Hello World!



% \end{document}
% \end{verbatim}
%   \end{block}

% \end{frame}



% \begin{frame}
%   \frametitle{Białe znaki w~\LaTeX u}

%   \begin{block}{Organizacja tekstu za pomocą białych znaków}
%     Plik dobrze napisany, to plik który łatwiej~się poprawia. A~każdy
%     musi go~poprawiać.
%     \begin{itemize}
%     \item[--] Ponieważ przejście do~nowej to tyle samo co spacja,
%       warto dzielić tekst na~krótsze linie (ja stosuje standard 79
%       znaków). Dzięki temu nie musimy scrolować tekstu by~go zmienić,
%       wszystko jest na ekranie. \\
%       Poza tym komentarze lepiej działają, ale~o~tym za chwilę.
%     \item[--] Ja wyznaję zasadę, że~dwie ważne części pliku źródłowego
%       powinien oddzielać odstęp 3~pustych linii.
%     \item[--] Pomiędzy mniej ważnymi fragmentami powinny znajdować~się
%       2~lub~1 puste linie. Przy założeniu, że~przejście do~nowego
%       akapitu nie psuje wyglądu tekstu.
%     \end{itemize}
%   \end{block}

%   \begin{block}{Ważne}
%     Nie musicie tego robić tak jak ja, jednak gorąco polecam przyjąć
%     jakiś standard. To~niezmiernie ułatwia pracę z~plikiem źródłowym.
%   \end{block}

% \end{frame}



% \begin{frame}[fragile]
%   \frametitle{Polskie znaki}

%   \begin{block}{Na razie pewnie nie da~się ich pisać}
%     \LaTeX{} został stworzony do pracy z~językiem angielskim, stąd
%     aby~używać wygodnie znaków polskich, należy rozszerzyć jego
%     możliwości.
%   \end{block}

%   \begin{block}{Mądrość starożytnych, znaleziona w~bibliotece dr.~Google}
%     Powiada że~należy tak zmodyfikować preambułę
% \begin{verbatim}
% \documentclass[a4paper,11pt]{article}
% \usepackage[utf8]{inputenc}
% \usepackage[polish]{babel}
% \usepackage[MeX]{polski}
% \end{verbatim}
%   \end{block}

%   \begin{block}{\tbs usepackage}
%     Dosłownie ,,użyj paczki''. Importuje paczkę, która rozszerza
%     możliwości standardowego \LaTeX a.
%   \end{block}

% \end{frame}



% \begin{frame}[fragile]
%   \frametitle{Co warto zawsze mieć w~preambule?}

%   \begin{block}{Według mnie to}
% \begin{verbatim}
% \let\lll\undefined
% \usepackage{amsmath, amsfonts, amssymb, amscd, amsthm}
% \usepackage{calrsfs}
% \usepackage{xcolor}
% \end{verbatim}
%   \end{block}

% \end{frame}



% \begin{frame}[fragile]
%   \frametitle{Co warto zawsze mieć w~preambule?}

%   \begin{block}{}
% \begin{verbatim}
% \usepackage{vmargin}
% % ------------------------------------------------------
% % MARGINS
% % ------------------------------------------------------
% \setmarginsrb
% { 0.7in} % left margin
% { 0.6in} % top margin
% { 0.7in} % right margin
% { 0.8in} % bottom margin
% {  20pt} % head height
% {0.25in} % head sep
% {   9pt} % foot height
% { 0.3in} % foot sep

% \usepackage{hyperref}
% \usepackage{cleveref}
% \end{verbatim}
%   \end{block}

% \end{frame}



% \begin{frame}[fragile]
%   \frametitle{Polecenia w~części głównej}

%   \begin{block}{Składnia}
%     Polecenie zaczyna~się ,,\tbs'' zaś kończy \tb{spacją} lub nawiasem
%     wąsatym ,,\}''. Spacja w~pierwszym przypadku \tb{nie oznacza}
%     odstępu, lecz koniec polecenia.
%   \end{block}

%   \begin{block}{Wpiszcie taki tekst}
%     \verb+\TeX i~\LaTeX to \emph{nie} jest to samo.+
%   \end{block}

%   \begin{block}{Dlaczego?}
%     WYWINWYG =~to czego chcesz, to nie jest to co dostajesz. Ponieważ
%     spacja po \verb+\TeX+ nie oznacza odstępu, tylko koniec polecenia
%     \LaTeX{} rozumiem, że~odstępu ma tam nie być. Danie dwóch spacji
%     nic nie da, bo dla niego 1 spacja = $N$ spacji = koniec komendy.
%     Są różne szkoły radzenia sobie z~tym:
%     \begin{itemize}
%     \item[--] \verb+\TeX{} i~\LaTeX{} to+;
%     \item[--] \verb+{\TeX} i~{\LaTeX} to+.
%     \end{itemize}
%   \end{block}

% \end{frame}



% \begin{frame}[fragile]
%   \frametitle{Wdowy i~sieroty}

%   \begin{block}{Po co ta $\sim$?}
%     Przyimki takie jak ,,a'', ,,w'', ,,u'', ,,z'', czy też zaimek
%     zwrotny ,,się'' \\
%     źle wyglądają na końcu lub początku linii, bo~nie mają sensu bez
%     słów które~są zaraz obok. Podobno określa~się je jako wdowy
%     i~sieroty. \LaTeX{} sam decyduje jak rozłożyć graficznie słowa
%     ,,na~kartce'', więc często produkuje takie kwiatki. Aby temu
%     zapobiec, należy użyć \tb{twardej spacji}. Od zwykłe spacji
%     różni~się tym, że~\LaTeX{} zrobi wszystko, by~te słowa
%     znalazły~się w~jednej linii, oznacza ją właśnie \verb+~+.
%   \end{block}

%   \begin{block}{Porównanie}
%     \begin{itemize}
%     \item[--] \tb{Poprawnie:} \verb+zrobi~się+.
%     \item[--] \tb{Niepoprawnie:} \verb+zrobi się+.
%     \end{itemize}
%   \end{block}

%   \begin{block}{Dłuższe wyrażenia}
%     Jeśli chcemy by jakiś ciąg słów nigdy nie został rozbity między
%     dwie linie, to~piszemy np.~\verb+\mbox{coś tam, coś tam}+.
%   \end{block}

% \end{frame}



% \begin{frame}[fragile]
%   \frametitle{\% --~nasz najlepszy przyjaciel w~\LaTeX u}

%   \begin{block}{O~co chodzi z~tym~\%?}
%     Jak każdy dobry język programowania, \LaTeX{} posiada komentarze.
%     Wszystko od znaku \%, do końca linii jest zupełnie przez niego
%     ignorowane. Jeśli chcemy mieć znak procenta w~tekście trzeba
%     napisać \verb+\%+.
%   \end{block}

%   \begin{block}{Podstawowe rady}
%     \begin{itemize}
%     \item[--] Jeżeli dopiero zaczynasz pisać w~\LaTeX u i~uznałeś,
%       że~chcesz usunąć tekst z~pliku źródłowego, to~najlepiej go
%       wykomentuj. Zaraz dojdziesz do~wniosku, że~by ci~się przydał.
%     \item[--] Komentarz to najlepszy debugger na świecie, dzięki
%       znanej już Platonowi, to jest bodaj w~dialogu ,,Sofista'',
%       metodzie bisekcji. Za~sekundę do~tego wrócimy.
%     \end{itemize}
%   \end{block}

%   \begin{block}{Skorzystaj z~możliwości swojego środowiska}
%     Każde środowisko ma skrót klawiszowy odpowiedzialne
%     za~wykomentowanie i~odkomentowanie dowolnego bloku tekstu.
%     Dla~\TeX Makera: Ctrl-T, Ctrl-U.
%   \end{block}

% \end{frame}



% \begin{frame}[fragile]
%   \frametitle{Plik~się nie kompiluje}

%   \begin{block}{Początkujący powinien powiesić to nad łóżkiem}
%     Skróty klawiszowe do~komentowania i~odkomentowania bloków tekstu.
%     Dla \TeX Makera:
%     \begin{itemize}
%     \item[--] wykomentuj --~Ctrl-T;
%     \item[--] odkomentuj --~Ctrl-U.
%     \end{itemize}
%     Skoro już przy tym jesteśmy, to~trzeba powiedzieć o~trybie
%     matematycznym.
%   \end{block}

%   \begin{block}{Jeśli mamy błąd kompilacji i~nie wiesz jaki}
%     \begin{enumerate}
%     \item Kompilator podaje nam linię w~której jest błąd, ale
%       często~się myli w~zakresie 10~linii. Jeśli linie~są relatywnie
%       krótkie, to~jest większa szansa, że~poda poprawną.
%     \item Odkomentowujemy tekst, gdzie kompilator wskazuje błąd.
%     \item Kompilujemy jeszcze raz. Jeśli błąd jest wciąż obecny, to
%       znaczy, że~błędów jest więcej albo~wykomentowaliśmy zły
%       fragment.
%     \item Powtarzamy krok 2, aż~kompilator zadziała.
%     \end{enumerate}
%   \end{block}

% \end{frame}



% \begin{frame}[fragile]
%   \frametitle{Plik~się nie kompiluje}

%   \begin{block}{Jeśli mamy błąd kompilacji i~nie wiesz jaki}
%     \begin{enumerate}
%     \item Kompilator podaje nam linię w~której jest błąd, ale
%       często~się myli w~zakresie 10~linii. Jeśli linie~są relatywnie
%       krótkie, to~jest większa szansa, że~poda poprawną.
%     \item Odkomentowujemy tekst, gdzie kompilator wskazuje błąd.
%     \item Kompilujemy jeszcze raz. Jeśli błąd jest wciąż obecny, to
%       znaczy, że~błędów jest więcej albo~zakomentowaliśmy zły
%       fragment.
%     \item Powtarzamy krok 2, aż~kompilator zadziała.
%     \item Przyglądamy~się krótkiemu fragmentowi zakomentowanego
%       tekstu, jeśli~znajdziemy w~nim błąd to go poprawiamy, jeśli nie
%       to go~odkomentowujemy.
%     \item Jeśli zadziałał, to powtarzamy krok 5, jeśli nie, to znaczy,
%       że~przeoczyliśmy błąd i~musimy dokładnie przeanalizować
%       odkomentowany fragment.
%     \item Stosujemy tę procedurę, aż~wszystko działa.
%     \end{enumerate}
%   \end{block}

% \end{frame}



% \begin{frame}[fragile]
%   \frametitle{Warto robić wcięcia}

%   \begin{block}{Dlaczego warto skracać linie?}
%     Relatywnie krótkie linie w~pliku źródłowym pozwalają pracować na
%     małych fragmentach tekstu, a~znacznie łatwiej znaleźć błąd
%     w~krótkim fragmencie tekstu niż w~długim.
%   \end{block}

%   \begin{block}{Przykład trybu matematycznego}
%     \begin{equation}
%       \label{eq:2}
%       \eps > 0, \quad \veps > 0.
%     \end{equation}
%     Aby to otrzymać piszemy
% \begin{verbatim}
% \begin{equation}
%   \label{eq:1}
%   \epsilon > 0, \quad \varepsilon > 0.
% \end{equation}
% \end{verbatim}
%     Wcięcia bardzo ułatwiają później pracę z~\LaTeX em, dobre
%     środowisko zrobi je za was. A~o~trybie matematyczny, więcej opowie
%     wam Wojtek.
%   \end{block}

% \end{frame}



% \begin{frame}[fragile]
%   \frametitle{Tryb matematyczny}

%   \begin{block}{Chcemy mieć}
%     \begin{equation}
%       \label{eq:1}
%       \pd{ 2 }{ u( x, t ) }{ x } - \fr{ 1 }{ c^{ 2 } } \pd{ 2 }{ u(
%         x, t ) }{ t } = 0.
%     \end{equation}
%     Będzie trochę roboty, ale~potem pokażemy jak zmniejszyć jej
%     ilość do~rozsądnych rozmiarów.
%   \end{block}

%   \begin{block}{Słownik}
%     \begin{itemize}
%     \item[--] potęga --~\verb+a^{ 2 }+ = $a^{ 2 }$;
%     \item[--] ułamek --~\verb+\frac{ 1 }{ 2 }+ = $\fr{ 1 }{ 2 }$;
%     \item[--] pochodna cząstkowa --~\verb+\partial+ = $\partial$.
%     \end{itemize}
%   \end{block}

% \end{frame}



% \begin{frame}[fragile]
%   \frametitle{Tryb matematyczny}

%   \begin{block}{Musimy napisać}
% \begin{verbatim}
% \begin{equation}
%   \label{eq:1}
%   \frac{ \partial^{ 2 } u( x, t ) }{ \partial x^{ 2 } }
%   - \frac{ 1 }{ c^{ 2 } } \frac{ \partial^{ 2 } u( x, t ) }
%   { \partial t^{ 2 } }  = 0
% \end{equation}
% \end{verbatim}
%     Większość białych znaków jest tu niepotrzebna, dodałem je dla
%     własnej wygody. Tak samo podział na linie. W~trybie
%     matematycznym {\color{red} nie wolno} zostawiać pustych linii!!!
%   \end{block}

% \end{frame}



% \begin{frame}[fragile]
%   \frametitle{Tryb matematyczny}

%   \begin{block}{Jeśli mamy błąd kompilacji i~nie wiesz jaki}
%     \begin{enumerate}
%     \item Kompilator podaje nam linię w~której jest błąd, ale
%       często~się myli o~jakieś 10~linii. Jeśli linie~są relatywnie
%       krótkie, to~jest większa szansa, że~poda poprawną.
%     \item Odkomentowujemy tekst, gdzie kompilator wskazuje błąd.
%     \item Kompilujemy jeszcze raz. Jeśli błąd jest wciąż obecny, to
%       znaczy, że~błędów jest więcej albo ~zakomentowaliśmy zły
%       fragment.
%     \item Powtarzamy krok 2, aż~kompilator zadziała.
%     \item Przyglądamy~się krótkiemu fragmentowi zakomentowanego
%       tekstu, jeśli~znajdziemy w~nim błąd to go poprawiamy, jeśli
%       nie to go~odkomentowujemy.
%     \item Jeśli zadziałał, to powtarzamy krok 5, jeśli nie, to
%       znaczy, że~przeoczyliśmy błąd i~musimy dokładnie
%       przeanalizować odkomentowany fragment.
%     \item Stosujemy tę procedurę, aż~wszystko działa.
%     \end{enumerate}
%   \end{block}

% \end{frame}



% \begin{frame}[fragile]
%   \frametitle{Uprośćmy sobie życie}

%   \begin{block}{Dlaczego warto skracać linie?}
%     Relatywnie krótkie linie w~pliku źródłowym pozwalają pracować na
%     małych fragmentach tekstu, a~znacznie łatwiej znaleźć błąd
%     w~krótkiej części tekstu niż w~długiej.
%   \end{block}

%   \begin{block}{Przy odrobinie sprytu życie jest prostsze}
%     Chcemy mieć % (trochę to brzydkie, ale~kod prosty)
%     \begin{equation}
%       \label{eq:2}
%       \eps > 0, \veps > 0.
%     \end{equation}
%     Aby to uzyskać trzeba napisać
% \begin{verbatim}
% \begin{equation}
%   \label{eq:2}
%   \epsilon > 0, \varepsilon > 0.
% \end{equation}
% \end{verbatim}
%     Nie ma potrzeby tyle pisać.
%   \end{block}

% \end{frame}



% \begin{frame}[fragile]
%   \frametitle{Uprośćmy sobie życie}

%   \begin{block}{\tbs newcommand}
%     Wstawiamy do preambuły, najlepiej po ostatnim
%     \verb+\usepackage+, linię
% \begin{verbatim}
% \newcommand{\eps}{\epsilon}
% \end{verbatim}
%     Jeżeli dobrze rozumiem, a~to rozumowanie jeszcze mnie nie
%     zawiodło, urządzenie zwane preprocesorem zamieni w~tekście ,,na
%     chama'' każde wystąpienie \verb+\eps+ na \verb+\epsilon+.
%     Analogicznie
% \begin{verbatim}
% \newcommand{\veps}{\varepsilon}
% \end{verbatim}
%     Od razu lepiej:).
%   \end{block}

% \end{frame}



% \begin{frame}[fragile]
%   \frametitle{Uprośćmy sobie życie}

%   \begin{block}{Pochodna cząstkowa}
%     Wstawiamy do~preambuły
% \begin{verbatim}
% \newcommand{\pd}[3]{\frac{ \partial^{ #1 } { #2 } }
% { \partial { #3 }^{ #1 } }}
% \end{verbatim}
%     Lepiej nie rozdzielać, choć można, tej komendy na dwie linie jak
%     powyżej, zrobiłem to, aby tekst ładnie mieścił~się na~slajdzie.
%     Nawiasy wąsate wokół ,,\#liczba'', nie~są konieczne, ale~bez
%     nich \LaTeX{} czasem protestuje.
%   \end{block}

%   \begin{block}{Piszemy}
% \begin{verbatim}
% \pd{ 2 }{ u( x, t ) }{ x }
% - \frac{ 1 }{ c^{ 2 } } \pd{ 2 }{ u( x, t ) }{ t } = 0
% \end{verbatim}
%     Która wersja jest prostsza?
%   \end{block}

% \end{frame}



% \begin{frame}[fragile]
%   \frametitle{Uprośćmy sobie życie}

%   \begin{block}{Objaśnienie}
% \begin{verbatim}
% \newcommand{\nazwa-komendy}[ilość-argumentów]
% {treść komendy #1 #2 #3,...}
% \end{verbatim}
%     Nazwa komendy to chyba jasne. Jeśli komenda ma nie przyjmować
%     żadnych argumentów, jak komenda \verb+\LaTeX+, to pomijamy
%     nawias kwadratowy. Jeśli ma przyjmować np.~5 argumentów, to
%     piszemy [5]. W~treści komendy \LaTeX{} podstawi ,,na chama''
%     pierwszy argument za ,,\#1'', drugi za~,,\#2'', etc.
%   \end{block}

%   \begin{block}{}
%     Tutaj też lepiej nie rozbijać \verb+\newcommand+ na~dwie linie.
%   \end{block}

%   \begin{block}{Użycie komendy}
% \begin{verbatim}
% \nazwa-komendy{pierwszy-argument}{drugi-argument}...
% \end{verbatim}
%   \end{block}

% \end{frame}



% \begin{frame}[fragile]
%   \frametitle{Uprośćmy sobie życie}

%   \begin{block}{Czy rozumiecie już jak działa}
% \begin{verbatim}
% \newcommand{\pd}[3]{\frac{ \partial^{ #1 } { #2 } }
% { \partial { #3 }^{ #1 } } }
% \pd{ 2 }{ u( x, t ) }{ x }
% \end{verbatim}
%     ta komenda?
%   \end{block}

%   \begin{block}{}
%     {\color{red} Ważne}, \verb+\pd{ }{ u( x, t ) }{ x }+ też działa.
%   \end{block}

%   \begin{block}{Inny przykład}
%     \verb+\newcommand{\sizeOne}{8pt}+ --~definicja rozmiaru
%     czcionki.
%   \end{block}

%   \begin{block}{Osobiste doświadczenie}
%     Potrafię mieć 40 \verb+\newcommand+ w~jednym pliku źródłowym,
%     tak bardzo upraszczają mi życie. Np.~jeśli chcę mieć znak $\ra$,
%     to po co mam pisać \verb+\rightarrow+, kiedy mogę \verb+\ra+?
%   \end{block}

% \end{frame}



% \begin{frame}
%   \frametitle{Nie pytaj co możesz zrobić dla swojego środowiska}

%   \begin{block}{To środowisko ma zrobić coś dla ciebie :)}
%     Podstawowe skróty i~polecenia, na przykładzie \TeX Makera.
%     \begin{itemize}
%     \item[--] Ctrl-T --~zakomentuj blok tekstu;
%     \item[--] Ctrl-U --~odkomentuj blok tekstu;
%     \item[--] Ctrl-$>$ - wcięcie bloku;
%     \item[--] Ctrl-$<$ - usunięcie wcięcia bloku;
%     \item[--] Ctrl-R - zastąp;
%     \item[--] Ctrl-F - znajdź;
%     \item[--] Ctrl-M - znajdź następny;
%     \item[--] Ctrl-G - przejdź do linii;
%     \item[--] Narzędzia $\ra$ Wyczyść.
%     \end{itemize}
%     Sprawdź jakie są w~twoim i~ich używaj :).
%   \end{block}

% \end{frame}


% \begin{frame}
%   \frametitle{Co warto zrobić dalej?}

%   \begin{block}{Rady}
%     \begin{itemize}
%     \item Posłuchać następnych wystąpień :).
%     \item Przeczytać \emph{Nie za~krótkie wprowadzenie do~systemu
%         \LaTeX a}.
%     \item Nauczyć~się Bib\TeX a (chyba, że~jest już jakieś lepsze
%       rozwiązanie).
%     \item Pogooglować i~poeksperymentować.
%     \end{itemize}
%   \end{block}

% \end{frame}



% \begin{frame}
%   \frametitle{Literatura}

%   \begin{block}{Podstawowa}
%     \begin{itemize}
%     \item[--] Tobias Oetiker et.al, \emph{Nie za~krótkie wprowadzenie
%         do~systemu \LaTeX{}~2$_{ \varepsilon }$''},
%       \colorlink{http://texdoc.net/texmf-dist/doc/latex/lshort-polish/lshort2e.pdf}.
%     \item[--] \emph{\LaTeX. Wikibooks},
%       \colorlink{https://en.wikibooks.org/wiki/LaTeX}.
%     \item[--] \emph{Share\LaTeX{} guides},
%       \colorlink{https://www.sharelatex.com/learn/Main_Page}.
%     \item[--] Włodzimierz Macewicz, Stanisław Wawrykiewicz,
%       \emph{Wirtualna Akademia \TeX owa},
%       \colorlink{http://www.gust.org.pl/projects/wirtualna-akademia-texowa}.
%     \item[--] Scott Pakin, \emph{The Comprehensive \LaTeX{} Symbol
%         List},
%       \colorlink{http://tug.ctan.org/info/symbols/comprehensive/symbols-a4.pdf}.
%     \end{itemize}
%   \end{block}

%   \begin{block}{Zaawansowana (sam jej nie rozumiem)}
%     \begin{itemize}
%     \item[--] Christian Feuers\"{a}nger, \emph{Notes On Programming in
%         \TeX},
%       \colorlink{http://pgfplots.sourceforge.net/TeX-programming-notes.pdf}.
%     \item[--] Mark Trettin, \emph{An~essential guide to
%         \LaTeX{}~2$_{ \varepsilon }$ usage},
%       \colorlink{ftp://sunsite.icm.edu.pl/pub/CTAN/info/l2tabu/english/l2tabuen.pdf}.
%     \end{itemize}
%   \end{block}

% \end{frame}



\jagiellonianendslide{Teraz cała przygoda z~Ti\emph{k}Z przed wami}



\end{document}