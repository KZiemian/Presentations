% ---------------------------------------------------------------------
% Basic configuration of Beamera class and Jagiellonian theme
% ---------------------------------------------------------------------
\RequirePackage[l2tabu, orthodox]{nag}



\ifx\PresentationStyle\notset
  \def\PresentationStyle{dark}
\fi



\documentclass[10pt,t]{beamer}
\mode<presentation>
\usetheme[style=\PresentationStyle,JUlogotitle=no]{jagiellonian}




% ---------------------------------------
% Procesing configuration files of Jagiellonian theme loceted in directory
% "preambule".
% ---------------------------------------
% Configuration for polish language
% Need description
\usepackage[polish]{babel}
% Need description
\usepackage[MeX]{polski}



% ------------------------------
% Better support of polish chars in technical parts of PDF
% ------------------------------
\hypersetup{pdfencoding=auto,psdextra}

% Package "textpos" give as enviroment "textblock" which is very usefull in
% arranging text on slides.

% This is standard configuration of "textpos"
\usepackage[overlay,absolute]{textpos}

% If you need to see bounds of "textblock's" comment line above and uncomment
% one below.

% Caution! When showboxes option is on significant ammunt of space is add
% to the top of textblock and as such, everyting put in them gone down.
% We need to check how to remove this bug.

% \usepackage[showboxes,overlay,absolute]{textpos}



% Setting scale length for package "textpos"
\setlength{\TPHorizModule}{10mm}
\setlength{\TPVertModule}{\TPHorizModule}


% ---------------------------------------
% Packages written for lectures "Geometria 3D dla twórców gier wideo"
% ---------------------------------------
% \usepackage{./Geometry3DPackages/Geometry3D}
% \usepackage{./Geometry3DPackages/Geometry3DIndices}
% \usepackage{./Geometry3DPackages/Geometry3DTikZStyle}
% \usepackage{./ProgramowanieSymulacjiFizykiPaczki/ProgramowanieSymulacjiFizykiTikZStyle}
% \usepackage{./Geometry3DPackages/mathcommands}


% ---------------------------------------
% TikZ
% ---------------------------------------
% Importing TikZ libraries
\usetikzlibrary{arrows.meta}
\usetikzlibrary{positioning}





% % Configuration package "bm" that need for making bold symbols
% \newcommand{\bmmax}{0}
% \newcommand{\hmmax}{0}
% \usepackage{bm}




% ---------------------------------------
% Packages for scientific texts
% ---------------------------------------
% \let\lll\undefined  % Sometimes you must use this line to allow
% "amsmath" package to works with packages with packages for polish
% languge imported
% /preambul/LanguageSettings/JagiellonianPolishLanguageSettings.tex.
% This comments (probably) removes polish letter Ł.
\usepackage{amsmath}  % Packages from American Mathematical Society (AMS)
\usepackage{amssymb}
\usepackage{amscd}
\usepackage{amsthm}
\usepackage{siunitx}  % Package for typsetting SI units.
\usepackage{upgreek}  % Better looking greek letters.
% Example of using upgreek: pi = \uppi


\usepackage{calrsfs}  % Zmienia czcionkę kaligraficzną w \mathcal
% na ładniejszą. Może w innych miejscach robi to samo, ale o tym nic
% nie wiem.










% ---------------------------------------
% Packages written for lectures "Geometria 3D dla twórców gier wideo"
% ---------------------------------------
% \usepackage{./ProgramowanieSymulacjiFizykiPaczki/ProgramowanieSymulacjiFizyki}
% \usepackage{./ProgramowanieSymulacjiFizykiPaczki/ProgramowanieSymulacjiFizykiIndeksy}
% \usepackage{./ProgramowanieSymulacjiFizykiPaczki/ProgramowanieSymulacjiFizykiTikZStyle}





% !!!!!!!!!!!!!!!!!!!!!!!!!!!!!!
% !!!!!!!!!!!!!!!!!!!!!!!!!!!!!!
% EVIL STUFF
\if\JUlogotitle1
\edef\LogoJUPath{LogoJU_\JUlogoLang/LogoJU_\JUlogoShape_\JUlogoColor.pdf}
\titlegraphic{\hfill\includegraphics[scale=0.22]
{./JagiellonianPictures/\LogoJUPath}}
\fi
% ---------------------------------------
% Commands for handling colors
% ---------------------------------------


% Command for setting normal text color for some text in math modestyle
% Text color depend on used style of Jagiellonian

% Beamer version of command
\newcommand{\TextWithNormalTextColor}[1]{%
  {\color{jNormalTextFGColor}
    \setbeamercolor{math text}{fg=jNormalTextFGColor} {#1}}
}

% Article and similar classes version of command
% \newcommand{\TextWithNormalTextColor}[1]{%
%   {\color{jNormalTextsFGColor} {#1}}
% }



% Beamer version of command
\newcommand{\NormalTextInMathMode}[1]{%
  {\color{jNormalTextFGColor}
    \setbeamercolor{math text}{fg=jNormalTextFGColor} \text{#1}}
}


% Article and similar classes version of command
% \newcommand{\NormalTextInMathMode}[1]{%
%   {\color{jNormalTextsFGColor} \text{#1}}
% }




% Command that sets color of some mathematical text to the same color
% that has normal text in header (?)

% Beamer version of the command
\newcommand{\MathTextFrametitleFGColor}[1]{%
  {\color{jFrametitleFGColor}
    \setbeamercolor{math text}{fg=jFrametitleFGColor} #1}
}

% Article and similar classes version of the command
% \newcommand{\MathTextWhiteColor}[1]{{\color{jFrametitleFGColor} #1}}





% Command for setting color of alert text for some text in math modestyle

% Beamer version of the command
\newcommand{\MathTextAlertColor}[1]{%
  {\color{jOrange} \setbeamercolor{math text}{fg=jOrange} #1}
}

% Article and similar classes version of the command
% \newcommand{\MathTextAlertColor}[1]{{\color{jOrange} #1}}





% Command that allow you to sets chosen color as the color of some text into
% math mode. Due to some nuances in the way that Beamer handle colors
% it not work in all cases. We hope that in the future we will improve it.

% Beamer version of the command
\newcommand{\SetMathTextColor}[2]{%
  {\color{#1} \setbeamercolor{math text}{fg=#1} #2}
}


% Article and similar classes version of the command
% \newcommand{\SetMathTextColor}[2]{{\color{#1} #2}}










% ---------------------------------------
% Commands for few special slides
% ---------------------------------------
\newcommand{\EndingSlide}[1]{%
  \begin{frame}[standout]

    \begingroup

    \color{jFrametitleFGColor}

    #1

    \endgroup

  \end{frame}
}










% ---------------------------------------
% Commands for setting background pictures for some slides
% ---------------------------------------
\newcommand{\TitleBackgroundPicture}
{./JagiellonianPictures/Backgrounds/LajkonikDark.png}
\newcommand{\SectionBackgroundPicture}
{./JagiellonianPictures/Backgrounds/LajkonikLight.png}



\newcommand{\TitleSlideWithPicture}{%
  \begingroup

  \usebackgroundtemplate{%
    \includegraphics[height=\paperheight]{\TitleBackgroundPicture}}

  \maketitle

  \endgroup
}





\newcommand{\SectionSlideWithPicture}[1]{%
  \begingroup

  \usebackgroundtemplate{%
    \includegraphics[height=\paperheight]{\SectionBackgroundPicture}}

  \setbeamercolor{titlelike}{fg=normal text.fg}

  \section{#1}

  \endgroup
}










% ---------------------------------------
% Commands for lectures "Geometria 3D dla twórców gier wideo"
% Polish version
% ---------------------------------------
% Komendy teraz wykomentowane były potrzebne, gdy loga były na niebieskim
% tle, nie na białym. A są na białym bo tego chcieli w biurze projektu.
% \newcommand{\FundingLogoWhitePicturePL}
% {./PresentationPictures/CommonPictures/logotypFundusze_biale_bez_tla2.pdf}
\newcommand{\FundingLogoColorPicturePL}
{./PresentationPictures/CommonPictures/European_Funds_color_PL.pdf}
% \newcommand{\EULogoWhitePicturePL}
% {./PresentationPictures/CommonPictures/logotypUE_biale_bez_tla2.pdf}
\newcommand{\EUSocialFundLogoColorPicturePL}
{./PresentationPictures/CommonPictures/EU_Social_Fund_color_PL.pdf}
% \newcommand{\ZintegrUJLogoWhitePicturePL}
% {./PresentationPictures/CommonPictures/zintegruj-logo-white.pdf}
\newcommand{\ZintegrUJLogoColorPicturePL}
{./PresentationPictures/CommonPictures/ZintegrUJ_color.pdf}
\newcommand{\JULogoColorPicturePL}
{./JagiellonianPictures/LogoJU_PL/LogoJU_A_color.pdf}





\newcommand{\GeometryThreeDSpecialBeginningSlidePL}{%
  \begin{frame}[standout]

    \begin{textblock}{11}(1,0.7)

      \begin{flushleft}

        \mdseries

        \footnotesize

        \color{jFrametitleFGColor}

        Materiał powstał w ramach projektu współfinansowanego ze środków
        Unii Europejskiej w ramach Europejskiego Funduszu Społecznego
        POWR.03.05.00-00-Z309/17-00.

      \end{flushleft}

    \end{textblock}





    \begin{textblock}{10}(0,2.2)

      \tikz \fill[color=jBackgroundStyleLight] (0,0) rectangle (12.8,-1.5);

    \end{textblock}


    \begin{textblock}{3.2}(1,2.45)

      \includegraphics[scale=0.3]{\FundingLogoColorPicturePL}

    \end{textblock}


    \begin{textblock}{2.5}(3.7,2.5)

      \includegraphics[scale=0.2]{\JULogoColorPicturePL}

    \end{textblock}


    \begin{textblock}{2.5}(6,2.4)

      \includegraphics[scale=0.1]{\ZintegrUJLogoColorPicturePL}

    \end{textblock}


    \begin{textblock}{4.2}(8.4,2.6)

      \includegraphics[scale=0.3]{\EUSocialFundLogoColorPicturePL}

    \end{textblock}

  \end{frame}
}



\newcommand{\GeometryThreeDTwoSpecialBeginningSlidesPL}{%
  \begin{frame}[standout]

    \begin{textblock}{11}(1,0.7)

      \begin{flushleft}

        \mdseries

        \footnotesize

        \color{jFrametitleFGColor}

        Materiał powstał w ramach projektu współfinansowanego ze środków
        Unii Europejskiej w ramach Europejskiego Funduszu Społecznego
        POWR.03.05.00-00-Z309/17-00.

      \end{flushleft}

    \end{textblock}





    \begin{textblock}{10}(0,2.2)

      \tikz \fill[color=jBackgroundStyleLight] (0,0) rectangle (12.8,-1.5);

    \end{textblock}


    \begin{textblock}{3.2}(1,2.45)

      \includegraphics[scale=0.3]{\FundingLogoColorPicturePL}

    \end{textblock}


    \begin{textblock}{2.5}(3.7,2.5)

      \includegraphics[scale=0.2]{\JULogoColorPicturePL}

    \end{textblock}


    \begin{textblock}{2.5}(6,2.4)

      \includegraphics[scale=0.1]{\ZintegrUJLogoColorPicturePL}

    \end{textblock}


    \begin{textblock}{4.2}(8.4,2.6)

      \includegraphics[scale=0.3]{\EUSocialFundLogoColorPicturePL}

    \end{textblock}

  \end{frame}





  \TitleSlideWithPicture
}



\newcommand{\GeometryThreeDSpecialEndingSlidePL}{%
  \begin{frame}[standout]

    \begin{textblock}{11}(1,0.7)

      \begin{flushleft}

        \mdseries

        \footnotesize

        \color{jFrametitleFGColor}

        Materiał powstał w ramach projektu współfinansowanego ze środków
        Unii Europejskiej w~ramach Europejskiego Funduszu Społecznego
        POWR.03.05.00-00-Z309/17-00.

      \end{flushleft}

    \end{textblock}





    \begin{textblock}{10}(0,2.2)

      \tikz \fill[color=jBackgroundStyleLight] (0,0) rectangle (12.8,-1.5);

    \end{textblock}


    \begin{textblock}{3.2}(1,2.45)

      \includegraphics[scale=0.3]{\FundingLogoColorPicturePL}

    \end{textblock}


    \begin{textblock}{2.5}(3.7,2.5)

      \includegraphics[scale=0.2]{\JULogoColorPicturePL}

    \end{textblock}


    \begin{textblock}{2.5}(6,2.4)

      \includegraphics[scale=0.1]{\ZintegrUJLogoColorPicturePL}

    \end{textblock}


    \begin{textblock}{4.2}(8.4,2.6)

      \includegraphics[scale=0.3]{\EUSocialFundLogoColorPicturePL}

    \end{textblock}





    \begin{textblock}{11}(1,4)

      \begin{flushleft}

        \mdseries

        \footnotesize

        \RaggedRight

        \color{jFrametitleFGColor}

        Treść niniejszego wykładu jest udostępniona na~licencji
        Creative Commons (\textsc{cc}), z~uzna\-niem autorstwa
        (\textsc{by}) oraz udostępnianiem na tych samych warunkach
        (\textsc{sa}). Rysunki i~wy\-kresy zawarte w~wykładzie są
        autorstwa dr.~hab.~Pawła Węgrzyna et~al. i~są dostępne
        na tej samej licencji, o~ile nie wskazano inaczej.
        W~prezentacji wykorzystano temat Beamera Jagiellonian,
        oparty na~temacie Metropolis Matthiasa Vogelgesanga,
        dostępnym na licencji \LaTeX{} Project Public License~1.3c
        pod adresem: \colorhref{https://github.com/matze/mtheme}
        {https://github.com/matze/mtheme}.

        Projekt typograficzny: Iwona Grabska-Gradzińska \\
        Skład: Kamil Ziemian;
        Korekta: Wojciech Palacz \\
        Modele: Dariusz Frymus, Kamil Nowakowski \\
        Rysunki i~wykresy: Kamil Ziemian, Paweł Węgrzyn, Wojciech Palacz

      \end{flushleft}

    \end{textblock}

  \end{frame}
}



\newcommand{\GeometryThreeDTwoSpecialEndingSlidesPL}[1]{%
  \begin{frame}[standout]


    \begin{textblock}{11}(1,0.7)

      \begin{flushleft}

        \mdseries

        \footnotesize

        \color{jFrametitleFGColor}

        Materiał powstał w ramach projektu współfinansowanego ze środków
        Unii Europejskiej w~ramach Europejskiego Funduszu Społecznego
        POWR.03.05.00-00-Z309/17-00.

      \end{flushleft}

    \end{textblock}





    \begin{textblock}{10}(0,2.2)

      \tikz \fill[color=jBackgroundStyleLight] (0,0) rectangle (12.8,-1.5);

    \end{textblock}


    \begin{textblock}{3.2}(1,2.45)

      \includegraphics[scale=0.3]{\FundingLogoColorPicturePL}

    \end{textblock}


    \begin{textblock}{2.5}(3.7,2.5)

      \includegraphics[scale=0.2]{\JULogoColorPicturePL}

    \end{textblock}


    \begin{textblock}{2.5}(6,2.4)

      \includegraphics[scale=0.1]{\ZintegrUJLogoColorPicturePL}

    \end{textblock}


    \begin{textblock}{4.2}(8.4,2.6)

      \includegraphics[scale=0.3]{\EUSocialFundLogoColorPicturePL}

    \end{textblock}





    \begin{textblock}{11}(1,4)

      \begin{flushleft}

        \mdseries

        \footnotesize

        \RaggedRight

        \color{jFrametitleFGColor}

        Treść niniejszego wykładu jest udostępniona na~licencji
        Creative Commons (\textsc{cc}), z~uzna\-niem autorstwa
        (\textsc{by}) oraz udostępnianiem na tych samych warunkach
        (\textsc{sa}). Rysunki i~wy\-kresy zawarte w~wykładzie są
        autorstwa dr.~hab.~Pawła Węgrzyna et~al. i~są dostępne
        na tej samej licencji, o~ile nie wskazano inaczej.
        W~prezentacji wykorzystano temat Beamera Jagiellonian,
        oparty na~temacie Metropolis Matthiasa Vogelgesanga,
        dostępnym na licencji \LaTeX{} Project Public License~1.3c
        pod adresem: \colorhref{https://github.com/matze/mtheme}
        {https://github.com/matze/mtheme}.

        Projekt typograficzny: Iwona Grabska-Gradzińska \\
        Skład: Kamil Ziemian;
        Korekta: Wojciech Palacz \\
        Modele: Dariusz Frymus, Kamil Nowakowski \\
        Rysunki i~wykresy: Kamil Ziemian, Paweł Węgrzyn, Wojciech Palacz

      \end{flushleft}

    \end{textblock}

  \end{frame}





  \begin{frame}[standout]

    \begingroup

    \color{jFrametitleFGColor}

    #1

    \endgroup

  \end{frame}
}



\newcommand{\GeometryThreeDSpecialEndingSlideVideoPL}{%
  \begin{frame}[standout]

    \begin{textblock}{11}(1,0.7)

      \begin{flushleft}

        \mdseries

        \footnotesize

        \color{jFrametitleFGColor}

        Materiał powstał w ramach projektu współfinansowanego ze środków
        Unii Europejskiej w~ramach Europejskiego Funduszu Społecznego
        POWR.03.05.00-00-Z309/17-00.

      \end{flushleft}

    \end{textblock}





    \begin{textblock}{10}(0,2.2)

      \tikz \fill[color=jBackgroundStyleLight] (0,0) rectangle (12.8,-1.5);

    \end{textblock}


    \begin{textblock}{3.2}(1,2.45)

      \includegraphics[scale=0.3]{\FundingLogoColorPicturePL}

    \end{textblock}


    \begin{textblock}{2.5}(3.7,2.5)

      \includegraphics[scale=0.2]{\JULogoColorPicturePL}

    \end{textblock}


    \begin{textblock}{2.5}(6,2.4)

      \includegraphics[scale=0.1]{\ZintegrUJLogoColorPicturePL}

    \end{textblock}


    \begin{textblock}{4.2}(8.4,2.6)

      \includegraphics[scale=0.3]{\EUSocialFundLogoColorPicturePL}

    \end{textblock}





    \begin{textblock}{11}(1,4)

      \begin{flushleft}

        \mdseries

        \footnotesize

        \RaggedRight

        \color{jFrametitleFGColor}

        Treść niniejszego wykładu jest udostępniona na~licencji
        Creative Commons (\textsc{cc}), z~uzna\-niem autorstwa
        (\textsc{by}) oraz udostępnianiem na tych samych warunkach
        (\textsc{sa}). Rysunki i~wy\-kresy zawarte w~wykładzie są
        autorstwa dr.~hab.~Pawła Węgrzyna et~al. i~są dostępne
        na tej samej licencji, o~ile nie wskazano inaczej.
        W~prezentacji wykorzystano temat Beamera Jagiellonian,
        oparty na~temacie Metropolis Matthiasa Vogelgesanga,
        dostępnym na licencji \LaTeX{} Project Public License~1.3c
        pod adresem: \colorhref{https://github.com/matze/mtheme}
        {https://github.com/matze/mtheme}.

        Projekt typograficzny: Iwona Grabska-Gradzińska;
        Skład: Kamil Ziemian \\
        Korekta: Wojciech Palacz;
        Modele: Dariusz Frymus, Kamil Nowakowski \\
        Rysunki i~wykresy: Kamil Ziemian, Paweł Węgrzyn, Wojciech Palacz \\
        Montaż: Agencja Filmowa Film \& Television Production~-- Zbigniew
        Masklak

      \end{flushleft}

    \end{textblock}

  \end{frame}
}





\newcommand{\GeometryThreeDTwoSpecialEndingSlidesVideoPL}[1]{%
  \begin{frame}[standout]

    \begin{textblock}{11}(1,0.7)

      \begin{flushleft}

        \mdseries

        \footnotesize

        \color{jFrametitleFGColor}

        Materiał powstał w ramach projektu współfinansowanego ze środków
        Unii Europejskiej w~ramach Europejskiego Funduszu Społecznego
        POWR.03.05.00-00-Z309/17-00.

      \end{flushleft}

    \end{textblock}





    \begin{textblock}{10}(0,2.2)

      \tikz \fill[color=jBackgroundStyleLight] (0,0) rectangle (12.8,-1.5);

    \end{textblock}


    \begin{textblock}{3.2}(1,2.45)

      \includegraphics[scale=0.3]{\FundingLogoColorPicturePL}

    \end{textblock}


    \begin{textblock}{2.5}(3.7,2.5)

      \includegraphics[scale=0.2]{\JULogoColorPicturePL}

    \end{textblock}


    \begin{textblock}{2.5}(6,2.4)

      \includegraphics[scale=0.1]{\ZintegrUJLogoColorPicturePL}

    \end{textblock}


    \begin{textblock}{4.2}(8.4,2.6)

      \includegraphics[scale=0.3]{\EUSocialFundLogoColorPicturePL}

    \end{textblock}





    \begin{textblock}{11}(1,4)

      \begin{flushleft}

        \mdseries

        \footnotesize

        \RaggedRight

        \color{jFrametitleFGColor}

        Treść niniejszego wykładu jest udostępniona na~licencji
        Creative Commons (\textsc{cc}), z~uzna\-niem autorstwa
        (\textsc{by}) oraz udostępnianiem na tych samych warunkach
        (\textsc{sa}). Rysunki i~wy\-kresy zawarte w~wykładzie są
        autorstwa dr.~hab.~Pawła Węgrzyna et~al. i~są dostępne
        na tej samej licencji, o~ile nie wskazano inaczej.
        W~prezentacji wykorzystano temat Beamera Jagiellonian,
        oparty na~temacie Metropolis Matthiasa Vogelgesanga,
        dostępnym na licencji \LaTeX{} Project Public License~1.3c
        pod adresem: \colorhref{https://github.com/matze/mtheme}
        {https://github.com/matze/mtheme}.

        Projekt typograficzny: Iwona Grabska-Gradzińska;
        Skład: Kamil Ziemian \\
        Korekta: Wojciech Palacz;
        Modele: Dariusz Frymus, Kamil Nowakowski \\
        Rysunki i~wykresy: Kamil Ziemian, Paweł Węgrzyn, Wojciech Palacz \\
        Montaż: Agencja Filmowa Film \& Television Production~-- Zbigniew
        Masklak

      \end{flushleft}

    \end{textblock}

  \end{frame}





  \begin{frame}[standout]


    \begingroup

    \color{jFrametitleFGColor}

    #1

    \endgroup

  \end{frame}
}










% ---------------------------------------
% Commands for lectures "Geometria 3D dla twórców gier wideo"
% English version
% ---------------------------------------
% \newcommand{\FundingLogoWhitePictureEN}
% {./PresentationPictures/CommonPictures/logotypFundusze_biale_bez_tla2.pdf}
\newcommand{\FundingLogoColorPictureEN}
{./PresentationPictures/CommonPictures/European_Funds_color_EN.pdf}
% \newcommand{\EULogoWhitePictureEN}
% {./PresentationPictures/CommonPictures/logotypUE_biale_bez_tla2.pdf}
\newcommand{\EUSocialFundLogoColorPictureEN}
{./PresentationPictures/CommonPictures/EU_Social_Fund_color_EN.pdf}
% \newcommand{\ZintegrUJLogoWhitePictureEN}
% {./PresentationPictures/CommonPictures/zintegruj-logo-white.pdf}
\newcommand{\ZintegrUJLogoColorPictureEN}
{./PresentationPictures/CommonPictures/ZintegrUJ_color.pdf}
\newcommand{\JULogoColorPictureEN}
{./JagiellonianPictures/LogoJU_EN/LogoJU_A_color.pdf}



\newcommand{\GeometryThreeDSpecialBeginningSlideEN}{%
  \begin{frame}[standout]

    \begin{textblock}{11}(1,0.7)

      \begin{flushleft}

        \mdseries

        \footnotesize

        \color{jFrametitleFGColor}

        This content was created as part of a project co-financed by the
        European Union within the framework of the European Social Fund
        POWR.03.05.00-00-Z309/17-00.

      \end{flushleft}

    \end{textblock}





    \begin{textblock}{10}(0,2.2)

      \tikz \fill[color=jBackgroundStyleLight] (0,0) rectangle (12.8,-1.5);

    \end{textblock}


    \begin{textblock}{3.2}(0.7,2.45)

      \includegraphics[scale=0.3]{\FundingLogoColorPictureEN}

    \end{textblock}


    \begin{textblock}{2.5}(4.15,2.5)

      \includegraphics[scale=0.2]{\JULogoColorPictureEN}

    \end{textblock}


    \begin{textblock}{2.5}(6.35,2.4)

      \includegraphics[scale=0.1]{\ZintegrUJLogoColorPictureEN}

    \end{textblock}


    \begin{textblock}{4.2}(8.4,2.6)

      \includegraphics[scale=0.3]{\EUSocialFundLogoColorPictureEN}

    \end{textblock}

  \end{frame}
}



\newcommand{\GeometryThreeDTwoSpecialBeginningSlidesEN}{%
  \begin{frame}[standout]

    \begin{textblock}{11}(1,0.7)

      \begin{flushleft}

        \mdseries

        \footnotesize

        \color{jFrametitleFGColor}

        This content was created as part of a project co-financed by the
        European Union within the framework of the European Social Fund
        POWR.03.05.00-00-Z309/17-00.

      \end{flushleft}

    \end{textblock}





    \begin{textblock}{10}(0,2.2)

      \tikz \fill[color=jBackgroundStyleLight] (0,0) rectangle (12.8,-1.5);

    \end{textblock}


    \begin{textblock}{3.2}(0.7,2.45)

      \includegraphics[scale=0.3]{\FundingLogoColorPictureEN}

    \end{textblock}


    \begin{textblock}{2.5}(4.15,2.5)

      \includegraphics[scale=0.2]{\JULogoColorPictureEN}

    \end{textblock}


    \begin{textblock}{2.5}(6.35,2.4)

      \includegraphics[scale=0.1]{\ZintegrUJLogoColorPictureEN}

    \end{textblock}


    \begin{textblock}{4.2}(8.4,2.6)

      \includegraphics[scale=0.3]{\EUSocialFundLogoColorPictureEN}

    \end{textblock}

  \end{frame}





  \TitleSlideWithPicture
}



\newcommand{\GeometryThreeDSpecialEndingSlideEN}{%
  \begin{frame}[standout]

    \begin{textblock}{11}(1,0.7)

      \begin{flushleft}

        \mdseries

        \footnotesize

        \color{jFrametitleFGColor}

        This content was created as part of a project co-financed by the
        European Union within the framework of the European Social Fund
        POWR.03.05.00-00-Z309/17-00.

      \end{flushleft}

    \end{textblock}





    \begin{textblock}{10}(0,2.2)

      \tikz \fill[color=jBackgroundStyleLight] (0,0) rectangle (12.8,-1.5);

    \end{textblock}


    \begin{textblock}{3.2}(0.7,2.45)

      \includegraphics[scale=0.3]{\FundingLogoColorPictureEN}

    \end{textblock}


    \begin{textblock}{2.5}(4.15,2.5)

      \includegraphics[scale=0.2]{\JULogoColorPictureEN}

    \end{textblock}


    \begin{textblock}{2.5}(6.35,2.4)

      \includegraphics[scale=0.1]{\ZintegrUJLogoColorPictureEN}

    \end{textblock}


    \begin{textblock}{4.2}(8.4,2.6)

      \includegraphics[scale=0.3]{\EUSocialFundLogoColorPictureEN}

    \end{textblock}





    \begin{textblock}{11}(1,4)

      \begin{flushleft}

        \mdseries

        \footnotesize

        \RaggedRight

        \color{jFrametitleFGColor}

        The content of this lecture is made available under a~Creative
        Commons licence (\textsc{cc}), giving the author the credits
        (\textsc{by}) and putting an obligation to share on the same terms
        (\textsc{sa}). Figures and diagrams included in the lecture are
        authored by Paweł Węgrzyn et~al., and are available under the same
        license unless indicated otherwise.\\ The presentation uses the
        Beamer Jagiellonian theme based on Matthias Vogelgesang’s
        Metropolis theme, available under license \LaTeX{} Project
        Public License~1.3c at: \colorhref{https://github.com/matze/mtheme}
        {https://github.com/matze/mtheme}.

        Typographic design: Iwona Grabska-Gradzińska \\
        \LaTeX{} Typesetting: Kamil Ziemian \\
        Proofreading: Wojciech Palacz,
        Monika Stawicka \\
        3D Models: Dariusz Frymus, Kamil Nowakowski \\
        Figures and charts: Kamil Ziemian, Paweł Węgrzyn, Wojciech Palacz

      \end{flushleft}

    \end{textblock}

  \end{frame}
}



\newcommand{\GeometryThreeDTwoSpecialEndingSlidesEN}[1]{%
  \begin{frame}[standout]


    \begin{textblock}{11}(1,0.7)

      \begin{flushleft}

        \mdseries

        \footnotesize

        \color{jFrametitleFGColor}

        This content was created as part of a project co-financed by the
        European Union within the framework of the European Social Fund
        POWR.03.05.00-00-Z309/17-00.

      \end{flushleft}

    \end{textblock}





    \begin{textblock}{10}(0,2.2)

      \tikz \fill[color=jBackgroundStyleLight] (0,0) rectangle (12.8,-1.5);

    \end{textblock}


    \begin{textblock}{3.2}(0.7,2.45)

      \includegraphics[scale=0.3]{\FundingLogoColorPictureEN}

    \end{textblock}


    \begin{textblock}{2.5}(4.15,2.5)

      \includegraphics[scale=0.2]{\JULogoColorPictureEN}

    \end{textblock}


    \begin{textblock}{2.5}(6.35,2.4)

      \includegraphics[scale=0.1]{\ZintegrUJLogoColorPictureEN}

    \end{textblock}


    \begin{textblock}{4.2}(8.4,2.6)

      \includegraphics[scale=0.3]{\EUSocialFundLogoColorPictureEN}

    \end{textblock}





    \begin{textblock}{11}(1,4)

      \begin{flushleft}

        \mdseries

        \footnotesize

        \RaggedRight

        \color{jFrametitleFGColor}

        The content of this lecture is made available under a~Creative
        Commons licence (\textsc{cc}), giving the author the credits
        (\textsc{by}) and putting an obligation to share on the same terms
        (\textsc{sa}). Figures and diagrams included in the lecture are
        authored by Paweł Węgrzyn et~al., and are available under the same
        license unless indicated otherwise.\\ The presentation uses the
        Beamer Jagiellonian theme based on Matthias Vogelgesang’s
        Metropolis theme, available under license \LaTeX{} Project
        Public License~1.3c at: \colorhref{https://github.com/matze/mtheme}
        {https://github.com/matze/mtheme}.

        Typographic design: Iwona Grabska-Gradzińska \\
        \LaTeX{} Typesetting: Kamil Ziemian \\
        Proofreading: Wojciech Palacz,
        Monika Stawicka \\
        3D Models: Dariusz Frymus, Kamil Nowakowski \\
        Figures and charts: Kamil Ziemian, Paweł Węgrzyn, Wojciech Palacz

      \end{flushleft}

    \end{textblock}

  \end{frame}





  \begin{frame}[standout]

    \begingroup

    \color{jFrametitleFGColor}

    #1

    \endgroup

  \end{frame}
}



\newcommand{\GeometryThreeDSpecialEndingSlideVideoVerOneEN}{%
  \begin{frame}[standout]

    \begin{textblock}{11}(1,0.7)

      \begin{flushleft}

        \mdseries

        \footnotesize

        \color{jFrametitleFGColor}

        This content was created as part of a project co-financed by the
        European Union within the framework of the European Social Fund
        POWR.03.05.00-00-Z309/17-00.

      \end{flushleft}

    \end{textblock}





    \begin{textblock}{10}(0,2.2)

      \tikz \fill[color=jBackgroundStyleLight] (0,0) rectangle (12.8,-1.5);

    \end{textblock}


    \begin{textblock}{3.2}(0.7,2.45)

      \includegraphics[scale=0.3]{\FundingLogoColorPictureEN}

    \end{textblock}


    \begin{textblock}{2.5}(4.15,2.5)

      \includegraphics[scale=0.2]{\JULogoColorPictureEN}

    \end{textblock}


    \begin{textblock}{2.5}(6.35,2.4)

      \includegraphics[scale=0.1]{\ZintegrUJLogoColorPictureEN}

    \end{textblock}


    \begin{textblock}{4.2}(8.4,2.6)

      \includegraphics[scale=0.3]{\EUSocialFundLogoColorPictureEN}

    \end{textblock}





    \begin{textblock}{11}(1,4)

      \begin{flushleft}

        \mdseries

        \footnotesize

        \RaggedRight

        \color{jFrametitleFGColor}

        The content of this lecture is made available under a Creative
        Commons licence (\textsc{cc}), giving the author the credits
        (\textsc{by}) and putting an obligation to share on the same terms
        (\textsc{sa}). Figures and diagrams included in the lecture are
        authored by Paweł Węgrzyn et~al., and are available under the same
        license unless indicated otherwise.\\ The presentation uses the
        Beamer Jagiellonian theme based on Matthias Vogelgesang’s
        Metropolis theme, available under license \LaTeX{} Project
        Public License~1.3c at: \colorhref{https://github.com/matze/mtheme}
        {https://github.com/matze/mtheme}.

        Typographic design: Iwona Grabska-Gradzińska;
        \LaTeX{} Typesetting: Kamil Ziemian \\
        Proofreading: Wojciech Palacz,
        Monika Stawicka \\
        3D Models: Dariusz Frymus, Kamil Nowakowski \\
        Figures and charts: Kamil Ziemian, Paweł Węgrzyn, Wojciech
        Palacz \\
        Film editing: Agencja Filmowa Film \& Television Production~--
        Zbigniew Masklak

      \end{flushleft}

    \end{textblock}

  \end{frame}
}



\newcommand{\GeometryThreeDSpecialEndingSlideVideoVerTwoEN}{%
  \begin{frame}[standout]

    \begin{textblock}{11}(1,0.7)

      \begin{flushleft}

        \mdseries

        \footnotesize

        \color{jFrametitleFGColor}

        This content was created as part of a project co-financed by the
        European Union within the framework of the European Social Fund
        POWR.03.05.00-00-Z309/17-00.

      \end{flushleft}

    \end{textblock}





    \begin{textblock}{10}(0,2.2)

      \tikz \fill[color=jBackgroundStyleLight] (0,0) rectangle (12.8,-1.5);

    \end{textblock}


    \begin{textblock}{3.2}(0.7,2.45)

      \includegraphics[scale=0.3]{\FundingLogoColorPictureEN}

    \end{textblock}


    \begin{textblock}{2.5}(4.15,2.5)

      \includegraphics[scale=0.2]{\JULogoColorPictureEN}

    \end{textblock}


    \begin{textblock}{2.5}(6.35,2.4)

      \includegraphics[scale=0.1]{\ZintegrUJLogoColorPictureEN}

    \end{textblock}


    \begin{textblock}{4.2}(8.4,2.6)

      \includegraphics[scale=0.3]{\EUSocialFundLogoColorPictureEN}

    \end{textblock}





    \begin{textblock}{11}(1,4)

      \begin{flushleft}

        \mdseries

        \footnotesize

        \RaggedRight

        \color{jFrametitleFGColor}

        The content of this lecture is made available under a Creative
        Commons licence (\textsc{cc}), giving the author the credits
        (\textsc{by}) and putting an obligation to share on the same terms
        (\textsc{sa}). Figures and diagrams included in the lecture are
        authored by Paweł Węgrzyn et~al., and are available under the same
        license unless indicated otherwise.\\ The presentation uses the
        Beamer Jagiellonian theme based on Matthias Vogelgesang’s
        Metropolis theme, available under license \LaTeX{} Project
        Public License~1.3c at: \colorhref{https://github.com/matze/mtheme}
        {https://github.com/matze/mtheme}.

        Typographic design: Iwona Grabska-Gradzińska;
        \LaTeX{} Typesetting: Kamil Ziemian \\
        Proofreading: Wojciech Palacz,
        Monika Stawicka \\
        3D Models: Dariusz Frymus, Kamil Nowakowski \\
        Figures and charts: Kamil Ziemian, Paweł Węgrzyn, Wojciech
        Palacz \\
        Film editing: IMAVI -- Joanna Kozakiewicz, Krzysztof Magda, Nikodem
        Frodyma

      \end{flushleft}

    \end{textblock}

  \end{frame}
}



\newcommand{\GeometryThreeDSpecialEndingSlideVideoVerThreeEN}{%
  \begin{frame}[standout]

    \begin{textblock}{11}(1,0.7)

      \begin{flushleft}

        \mdseries

        \footnotesize

        \color{jFrametitleFGColor}

        This content was created as part of a project co-financed by the
        European Union within the framework of the European Social Fund
        POWR.03.05.00-00-Z309/17-00.

      \end{flushleft}

    \end{textblock}





    \begin{textblock}{10}(0,2.2)

      \tikz \fill[color=jBackgroundStyleLight] (0,0) rectangle (12.8,-1.5);

    \end{textblock}


    \begin{textblock}{3.2}(0.7,2.45)

      \includegraphics[scale=0.3]{\FundingLogoColorPictureEN}

    \end{textblock}


    \begin{textblock}{2.5}(4.15,2.5)

      \includegraphics[scale=0.2]{\JULogoColorPictureEN}

    \end{textblock}


    \begin{textblock}{2.5}(6.35,2.4)

      \includegraphics[scale=0.1]{\ZintegrUJLogoColorPictureEN}

    \end{textblock}


    \begin{textblock}{4.2}(8.4,2.6)

      \includegraphics[scale=0.3]{\EUSocialFundLogoColorPictureEN}

    \end{textblock}





    \begin{textblock}{11}(1,4)

      \begin{flushleft}

        \mdseries

        \footnotesize

        \RaggedRight

        \color{jFrametitleFGColor}

        The content of this lecture is made available under a Creative
        Commons licence (\textsc{cc}), giving the author the credits
        (\textsc{by}) and putting an obligation to share on the same terms
        (\textsc{sa}). Figures and diagrams included in the lecture are
        authored by Paweł Węgrzyn et~al., and are available under the same
        license unless indicated otherwise.\\ The presentation uses the
        Beamer Jagiellonian theme based on Matthias Vogelgesang’s
        Metropolis theme, available under license \LaTeX{} Project
        Public License~1.3c at: \colorhref{https://github.com/matze/mtheme}
        {https://github.com/matze/mtheme}.

        Typographic design: Iwona Grabska-Gradzińska;
        \LaTeX{} Typesetting: Kamil Ziemian \\
        Proofreading: Wojciech Palacz,
        Monika Stawicka \\
        3D Models: Dariusz Frymus, Kamil Nowakowski \\
        Figures and charts: Kamil Ziemian, Paweł Węgrzyn, Wojciech
        Palacz \\
        Film editing: Agencja Filmowa Film \& Television Production~--
        Zbigniew Masklak \\
        Film editing: IMAVI -- Joanna Kozakiewicz, Krzysztof Magda, Nikodem
        Frodyma

      \end{flushleft}

    \end{textblock}

  \end{frame}
}



\newcommand{\GeometryThreeDTwoSpecialEndingSlidesVideoVerOneEN}[1]{%
  \begin{frame}[standout]

    \begin{textblock}{11}(1,0.7)

      \begin{flushleft}

        \mdseries

        \footnotesize

        \color{jFrametitleFGColor}

        This content was created as part of a project co-financed by the
        European Union within the framework of the European Social Fund
        POWR.03.05.00-00-Z309/17-00.

      \end{flushleft}

    \end{textblock}





    \begin{textblock}{10}(0,2.2)

      \tikz \fill[color=jBackgroundStyleLight] (0,0) rectangle (12.8,-1.5);

    \end{textblock}


    \begin{textblock}{3.2}(0.7,2.45)

      \includegraphics[scale=0.3]{\FundingLogoColorPictureEN}

    \end{textblock}


    \begin{textblock}{2.5}(4.15,2.5)

      \includegraphics[scale=0.2]{\JULogoColorPictureEN}

    \end{textblock}


    \begin{textblock}{2.5}(6.35,2.4)

      \includegraphics[scale=0.1]{\ZintegrUJLogoColorPictureEN}

    \end{textblock}


    \begin{textblock}{4.2}(8.4,2.6)

      \includegraphics[scale=0.3]{\EUSocialFundLogoColorPictureEN}

    \end{textblock}





    \begin{textblock}{11}(1,4)

      \begin{flushleft}

        \mdseries

        \footnotesize

        \RaggedRight

        \color{jFrametitleFGColor}

        The content of this lecture is made available under a Creative
        Commons licence (\textsc{cc}), giving the author the credits
        (\textsc{by}) and putting an obligation to share on the same terms
        (\textsc{sa}). Figures and diagrams included in the lecture are
        authored by Paweł Węgrzyn et~al., and are available under the same
        license unless indicated otherwise.\\ The presentation uses the
        Beamer Jagiellonian theme based on Matthias Vogelgesang’s
        Metropolis theme, available under license \LaTeX{} Project
        Public License~1.3c at: \colorhref{https://github.com/matze/mtheme}
        {https://github.com/matze/mtheme}.

        Typographic design: Iwona Grabska-Gradzińska;
        \LaTeX{} Typesetting: Kamil Ziemian \\
        Proofreading: Wojciech Palacz,
        Monika Stawicka \\
        3D Models: Dariusz Frymus, Kamil Nowakowski \\
        Figures and charts: Kamil Ziemian, Paweł Węgrzyn,
        Wojciech Palacz \\
        Film editing: Agencja Filmowa Film \& Television Production~--
        Zbigniew Masklak

      \end{flushleft}

    \end{textblock}

  \end{frame}





  \begin{frame}[standout]


    \begingroup

    \color{jFrametitleFGColor}

    #1

    \endgroup

  \end{frame}
}



\newcommand{\GeometryThreeDTwoSpecialEndingSlidesVideoVerTwoEN}[1]{%
  \begin{frame}[standout]

    \begin{textblock}{11}(1,0.7)

      \begin{flushleft}

        \mdseries

        \footnotesize

        \color{jFrametitleFGColor}

        This content was created as part of a project co-financed by the
        European Union within the framework of the European Social Fund
        POWR.03.05.00-00-Z309/17-00.

      \end{flushleft}

    \end{textblock}





    \begin{textblock}{10}(0,2.2)

      \tikz \fill[color=jBackgroundStyleLight] (0,0) rectangle (12.8,-1.5);

    \end{textblock}


    \begin{textblock}{3.2}(0.7,2.45)

      \includegraphics[scale=0.3]{\FundingLogoColorPictureEN}

    \end{textblock}


    \begin{textblock}{2.5}(4.15,2.5)

      \includegraphics[scale=0.2]{\JULogoColorPictureEN}

    \end{textblock}


    \begin{textblock}{2.5}(6.35,2.4)

      \includegraphics[scale=0.1]{\ZintegrUJLogoColorPictureEN}

    \end{textblock}


    \begin{textblock}{4.2}(8.4,2.6)

      \includegraphics[scale=0.3]{\EUSocialFundLogoColorPictureEN}

    \end{textblock}





    \begin{textblock}{11}(1,4)

      \begin{flushleft}

        \mdseries

        \footnotesize

        \RaggedRight

        \color{jFrametitleFGColor}

        The content of this lecture is made available under a Creative
        Commons licence (\textsc{cc}), giving the author the credits
        (\textsc{by}) and putting an obligation to share on the same terms
        (\textsc{sa}). Figures and diagrams included in the lecture are
        authored by Paweł Węgrzyn et~al., and are available under the same
        license unless indicated otherwise.\\ The presentation uses the
        Beamer Jagiellonian theme based on Matthias Vogelgesang’s
        Metropolis theme, available under license \LaTeX{} Project
        Public License~1.3c at: \colorhref{https://github.com/matze/mtheme}
        {https://github.com/matze/mtheme}.

        Typographic design: Iwona Grabska-Gradzińska;
        \LaTeX{} Typesetting: Kamil Ziemian \\
        Proofreading: Wojciech Palacz,
        Monika Stawicka \\
        3D Models: Dariusz Frymus, Kamil Nowakowski \\
        Figures and charts: Kamil Ziemian, Paweł Węgrzyn,
        Wojciech Palacz \\
        Film editing: IMAVI -- Joanna Kozakiewicz, Krzysztof Magda, Nikodem
        Frodyma

      \end{flushleft}

    \end{textblock}

  \end{frame}





  \begin{frame}[standout]


    \begingroup

    \color{jFrametitleFGColor}

    #1

    \endgroup

  \end{frame}
}



\newcommand{\GeometryThreeDTwoSpecialEndingSlidesVideoVerThreeEN}[1]{%
  \begin{frame}[standout]

    \begin{textblock}{11}(1,0.7)

      \begin{flushleft}

        \mdseries

        \footnotesize

        \color{jFrametitleFGColor}

        This content was created as part of a project co-financed by the
        European Union within the framework of the European Social Fund
        POWR.03.05.00-00-Z309/17-00.

      \end{flushleft}

    \end{textblock}





    \begin{textblock}{10}(0,2.2)

      \tikz \fill[color=jBackgroundStyleLight] (0,0) rectangle (12.8,-1.5);

    \end{textblock}


    \begin{textblock}{3.2}(0.7,2.45)

      \includegraphics[scale=0.3]{\FundingLogoColorPictureEN}

    \end{textblock}


    \begin{textblock}{2.5}(4.15,2.5)

      \includegraphics[scale=0.2]{\JULogoColorPictureEN}

    \end{textblock}


    \begin{textblock}{2.5}(6.35,2.4)

      \includegraphics[scale=0.1]{\ZintegrUJLogoColorPictureEN}

    \end{textblock}


    \begin{textblock}{4.2}(8.4,2.6)

      \includegraphics[scale=0.3]{\EUSocialFundLogoColorPictureEN}

    \end{textblock}





    \begin{textblock}{11}(1,4)

      \begin{flushleft}

        \mdseries

        \footnotesize

        \RaggedRight

        \color{jFrametitleFGColor}

        The content of this lecture is made available under a Creative
        Commons licence (\textsc{cc}), giving the author the credits
        (\textsc{by}) and putting an obligation to share on the same terms
        (\textsc{sa}). Figures and diagrams included in the lecture are
        authored by Paweł Węgrzyn et~al., and are available under the same
        license unless indicated otherwise. \\ The presentation uses the
        Beamer Jagiellonian theme based on Matthias Vogelgesang’s
        Metropolis theme, available under license \LaTeX{} Project
        Public License~1.3c at: \colorhref{https://github.com/matze/mtheme}
        {https://github.com/matze/mtheme}.

        Typographic design: Iwona Grabska-Gradzińska;
        \LaTeX{} Typesetting: Kamil Ziemian \\
        Proofreading: Leszek Hadasz, Wojciech Palacz,
        Monika Stawicka \\
        3D Models: Dariusz Frymus, Kamil Nowakowski \\
        Figures and charts: Kamil Ziemian, Paweł Węgrzyn,
        Wojciech Palacz \\
        Film editing: Agencja Filmowa Film \& Television Production~--
        Zbigniew Masklak \\
        Film editing: IMAVI -- Joanna Kozakiewicz, Krzysztof Magda, Nikodem
        Frodyma


      \end{flushleft}

    \end{textblock}

  \end{frame}





  \begin{frame}[standout]


    \begingroup

    \color{jFrametitleFGColor}

    #1

    \endgroup

  \end{frame}
}











% ------------------------------
% Importing packages, libraries and setting their configuration.
% ------------------------------





% ------------------------------
% Special configuration for this particular presentation.
% ------------------------------










% ---------------------------------------------------------------------
\title{Systemy operacyjne, laboratorium}
\subtitle{Zdjęcia ilustrujące historię systemów opracyjnych}

\author{Kamil Ziemian}


\date{}
% ---------------------------------------------------------------------










% ####################################################################
% Beginning of the document.
\begin{document}
% ####################################################################





% Text is adjusted to the left and words are broken at the end of the line.
% Number of chars: 62k+.
\RaggedRight





% ######################################
\maketitle
% ######################################





% ##################
\begin{frame}
  \frametitle{Spis treści}


  \tableofcontents

\end{frame}
% ##################










% ######################################
\section{Ludzie dzięki który powstał system GNU/Linux}
% ######################################



% ##################
\begin{frame}
  \frametitle{Ludzie dzięki którym powstał system GNU/Linux}


  \begin{enumerate}
    \setlength{\itemsep}{1em}

  \item Getting credits for the classes (zaliczenie; pre-examination)

  \item Taking final written examination

  \end{enumerate}

\end{frame}
% ##################





% % ##################
% \begin{frame}
%   \frametitle{Literature}


%   Eric Lengyel, \textit{Mathematics for~3D Game Programming and~Computer
%     Graphics}.

%   John Flynt, Danny Kodicek, \textit{Mathematics and~Physics
%     for~Programmers}.

%   John Flynt, Boris Meltreger, \textit{Beginning Math Concepts for~Game
%     Developers}.

%   Peter Shirley, Michael Ashikhmin, \textit{Fundamentals of~Computer
%     Graphics}.

%   Wendy Stahler, Dustin Clingman, Kaveh Kahriz, \textit{Beginning Math
%     and~Physics for~Game Programmers}.

%   Wendy Stahler, \emph{Fundamentals~of Math and~Physics for~Game
%     Programmers}.

%   Christopher Tremblay, \emph{Mathematics for~Game Developers}.

%   James Van Verth, Lars Bishop, \emph{Essential Mathematics for Games
%     \&~Interactive Applications: A~Programmer’s Guide}.

% \end{frame}
% % ##################





% % ##################
% \begin{frame}
%   \frametitle{Literature}


%   Fletcher Dunn, Ian Parberry, \emph{3D Math Primer for~Graphics
%     and~Game Development}.

%   Web resources.

% \end{frame}
% % ##################





% % ##################
% \begin{frame}
%   \frametitle{\alert{Important information}}


%   The slides are only \alert{auxiliary} to these lectures.
%   To understand all the problems discussed here, please follow the lectures
%   and literature.
%   % They can be
%   % \alert{not enough} to~understand all issues without
%   % listening to a lecture or reaching for literature.

%   All materials are a work within the meaning of the Act of 4~February
%   1994~r. on copyright and related rights (OJ~U.~no.~90, pos.~631). These
%   materials may not be distributed without providing information about the
%   authors.

% \end{frame}
% % ##################





% % ##################
% \begin{frame}
%   \frametitle{Kind request to listeners}


%   We will be grateful for your opinions on how to improve\\ these lectures.

%   If you spot any unclear phrases, illegible drawings, errors,
%   typos, etc, please let us know:\\ just provide the lecture’s  title
%   and a number of the slide.


%   % comments and proposals how to make
%   % lectures better, indication of unclear wording, illegible
%   % drawings, for~small/too large fonts on~slide,~etc. As well as
%   % please inform us about found errors, typos, which
%   % may be present  in these lectures.\\ In this case, we ask
%   % of giving the title of the lecture and~the number of the slide on~which the error is.


% \end{frame}
% % ##################










% % ######################################
% \SectionSlideWithPicture{Graphics pipeline (OpenGL)}
% % ######################################



% % ##################
% \begin{frame}
%   \frametitle{Graphics pipeline (rendering pipeline)}


%   \begin{figure}

%     \begin{tikzpicture}
%       \node[diagrams block 1] (pobranie wierzcholkow) {Vertex \\Specification};

%       \node[diagrams block 1,below=of pobranie wierzcholkow] (szader
%       wierzcholkow) {Vertex Shader};

%       \draw[thick connection line] (pobranie wierzcholkow) -- (szader
%       wierzcholkow);



%       \node[diagrams block 1,below=of szader wierzcholkow] (szader
%       sterowania teselacja) {Tessellation Control Shader};

%       \draw[thick connection line] (szader wierzcholkow) -- (szader
%       sterowania teselacja);



%       \node[diagrams block 1,right=of szader sterowania teselacja]
%       (teselacja) {Tessellation};

%       \draw[thick connection line] (szader sterowania teselacja) --
%       (teselacja);



%       \node[diagrams block 1,above=of teselacja] (szader wyliczania
%       teselacji) {Tessellation Evaluation Shader};

%       \draw[thick connection line] (teselacja) -- (szader wyliczania
%       teselacji);



%       \node[diagrams block 1,above=of szader wyliczania teselacji]
%       (szader geometrii) {Geometry Shader};

%       \draw[thick connection line] (szader wyliczania teselacji) --
%       (szader geometrii);



%       \node[diagrams block 1,right=of szader geometrii] (rasteryzacja)
%       {Rasterization};

%       \draw[thick connection line] (szader geometrii) -- (rasteryzacja);



%       \node[diagrams block 1,below=of rasteryzacja] (szader fragmentow)
%       {Fragment Shader};

%       \draw[thick connection line] (rasteryzacja) -- (szader fragmentow);



%       \node[diagrams block 1,below=of szader fragmentow] (bufor ramki)
%       {Frame buffer\\ Operations};

%       \draw[thick connection line] (szader fragmentow) -- (bufor ramki);
%     \end{tikzpicture}

%   \end{figure}

% \end{frame}
% % ##################





% % ##################
% \begin{frame}
%   \frametitle{About rendering pipeline on
%     \href{https://www.khronos.org/}{https://www.khronos.org/}}


%   \begingroup

%   \footnotesize

%   The~OpenGL rendering pipeline is~initiated when you per- \\
%   form a~rendering operation. Rendering operations require \\
%   the~presence~of a~properly-defined vertex array object and \\
%   a~linked Program Object or~Program Pipeline Object which \\
%   provides the~shaders for the~programmable pipeline stages.


%   Once initiated, the pipeline operates in the following order: \\[0.6em]
%   Vertex Processing: \vspace{-0.5em}
%   \begin{enumerate}
%     \footnotesize
%   \item Each vertex retrieved from the vertex arrays \\
%     (as defined by the \textsc{vao}) is acted upon by a Vertex \\
%     Shader. Each vertex in the stream is processed \\
%     in turn into an~output vertex.

%   \item Optional primitive tessellation stages.

%   \item Optional Geometry Shader primitive processing. \\
%     The~output is a sequence of primitives. \\
%     \ldots

%   \end{enumerate}

%   \endgroup





%   \begin{textblock}{9.1}(1,8.3)

%     \begingroup

%     \footnotesize

%     \textbf{The article is continued on:} \\
%     \colorhref{https://www.khronos.org/opengl/wiki/Rendering\_Pipeline\_Overview}
%     {https://www.khronos.org/opengl/wiki/Rendering\_Pipeline\_Overview}.

%     \endgroup

%   \end{textblock}





%   \begin{textblock}{3.35}(8.8,1.5)

%     \begin{tikzpicture}[node distance=0.6em, block loc/.style =
%       {rectangle, rounded corners, minimum height=1.75em, text
%         width=11em, scale=0.8, text centered,
%         text=jMathTextFGStyleDark}]


%       \node[block loc,fill=jAxisBlue] (vertex specification) {Vertex
%         Specification};

%       \node[block loc,fill=jOrange,below=of vertex specification]
%       (vertex shader) {Vertex Shader};

%       \draw[thick connection line] (vertex specification) -- (vertex
%       shader);


%       \node[block loc,fill=jOrange,below=of vertex shader]
%       (tessellation) {Tessellation};

%       \draw[thick connection line] (vertex shader) -- (tessellation);


%       \node[block loc,fill=jOrange,below=of tessellation] (geometry
%       shader) {Geometry Shader};

%       \draw[thick connection line] (tessellation) -- (geometry shader);


%       \node[block loc,fill=jAxisBlue,below=of geometry shader] (vertex
%       post-processing) {Vertex Post-Processing};

%       \draw[thick connection line] (geometry shader) -- (vertex
%       post-processing);


%       \node[block loc,fill=jAxisBlue,below=of vertex post-processing]
%       (primitive assembly) {Primitive Assembley};

%       \draw[thick connection line] (vertex post-processing) --
%       (primitive assembly);


%       \node[block loc,fill=jAxisBlue,below=of primitive assembly]
%       (rasterization) {Rasterization};

%       \draw[thick connection line] (primitive assembly) --
%       (rasterization);


%       \node[block loc,fill=jOrange,below=of rasterization]
%       (fragment shader) {Fragment Shader};

%       \draw[thick connection line] (rasterization) -- (fragment shader);


%       \node[block loc,fill=jAxisBlue,below=of fragment shader]
%       (per-sample operations) {Per-Sample Operations};

%       \draw[thick connection line] (fragment shader) -- (per-sample
%       operations);
%     \end{tikzpicture}

%   \end{textblock}

% \end{frame}
% % ##################





% % ##################
% \begin{frame}
%   \frametitle{Graphics pipeline (rendering pipeline)}


%   \begin{itemize}
%     \setlength{\itemsep}{1em}

%   \item \textbf{Pipelining} --~a technique of building processors for implementing the division of execution  of the program\\ into specialized groups of instructions\\ (execution subsequent stages).


%   \item \textbf{Parallelism} --~performing
%     multiple instructions simultaneously.


%   \end{itemize}


%   The above assumptions allow   enormous computing efficiency\\
%   on a graphics processing unit  (\textsc{gpu}).


% \end{frame}
% % ##################










% % ######################################
% \SectionSlideWithPicture{Basic concepts of~computer~graphics}
% % ######################################



% % ##################
% \begin{frame}
%   \frametitle{Basic concepts of computer graphics}


%   \textbf{Model} --~a mathematical representation of an object in the 3D world.

%   \vspace{0.5em}


%   \textbf{Rendering} --~obtaining a computer image based on the model\\
%   (displayed by a computer, e.g. on a screen).

%   \vspace{0.5em}


%   \textbf{Perspective} --~a drawing technique of a 3D object on a plane;\\ a generated image makes a three-dimensional impression. \\
%   Perspective is based on the physiology of human vision and optics.\\
%   Already Plato wrote about the use of perspective in painting.


% \end{frame}
% % ##################





% % ##################
% \begin{frame}
%   \frametitle{A sample scene}


%   \begin{textblock}{11}(1,1.65)

%     \includegraphics[scale=0.44]
%     {./PresentationPictures/Lecture_01/renderowanie01.png}

%   \end{textblock}


%   \begin{textblock}{1}(3.4,5.1)


%     \begingroup

%     \color{jBackgroundStyleLight}

%     model

%     \endgroup

%   \end{textblock}


%   \begin{textblock}{1}(5.5,4.1)

%     \begingroup

%     \color{jBackgroundStyleLight}

%     model

%     \endgroup

%   \end{textblock}


%   \begin{textblock}{1}(9.6,3.3)

%     \begingroup

%     \color{jBackgroundStyleLight}

%     model

%     \endgroup

%   \end{textblock}


%   \begin{textblock}{1.8}(9.4,1.8)

%     \begingroup

%     \color{jBackgroundStyleLight}

%     lighting

%     \endgroup

%   \end{textblock}


%   \begin{textblock}{1.2}(4.7,7.5)

%     \begingroup

%     \color{jBackgroundStyleLight}

%     camera

%     \endgroup

%   \end{textblock}

% \end{frame}
% % ##################





% % ##################
% \begin{frame}
%   \frametitle{Rendering without perspective}

%   \vspace{-1em}


%   \begin{figure}

%     \includegraphics[scale=0.45]
%     {./PresentationPictures/Lecture_01/renderowanie02.png}

%   \end{figure}

% \end{frame}
% % ##################





% % ##################
% \begin{frame}
%   \frametitle{Rendering with perspective}

%   \vspace{-1em}


%   \begin{figure}

%     \includegraphics[scale=0.45]
%     {./PresentationPictures/Lecture_01/renderowanie03.png}

%   \end{figure}

% \end{frame}
% % ##################





% % ##################
% \begin{frame}
%   \frametitle{Another sample scene}


%   \begin{textblock}{7.2}(3,1.5)

%     \includegraphics[scale=0.2]
%     {./PresentationPictures/Lecture_01/LOS_scena_1.png}
%     % Skrót: Lajkonik, Obwarzanek, Smok

%   \end{textblock}

% \end{frame}
% % ##################





% % ##################
% \begin{frame}
%   \frametitle{Rendering without perspective}


%   \begin{textblock}{11}(1,2)

%     \includegraphics[scale=0.167,trim=80 0 0 0,clip]
%     {./PresentationPictures/Lecture_01/LOS_rend_bez_perspektywy.png}
%     % LOS: Lajkonik, Obwarzanek, Smok

%   \end{textblock}

% \end{frame}
% % ##################





% % ##################
% \begin{frame}
%   \frametitle{Rendering with perspective}


%   \begin{textblock}{10.9}(1,2)

%     \includegraphics[scale=0.167,trim=90 0 0 0,clip]
%     {./PresentationPictures/Lecture_01/LOS_rend_z_perspektywa.png}
%     % LOS: Lajkonik, Obwarzanek, Smok

%   \end{textblock}

% \end{frame}
% % ##################





% % ##################
% \begin{frame}
%   \frametitle{Basic concepts of computer graphics}



%   \textbf{Point} --~an object in 3D space which has no size.

%   \vspace{0.5em}


%   \textbf{Pixel} (= \emph{picture}
%   +~\emph{element}) --~the smallest element of the displayed image on the
%   screen.


% \end{frame}
% % ##################





% % ##################
% \begin{frame}
%   \frametitle{Data structures}


%   \textbf{Vertex} --~a data structure which stores the information about a point
%   and data related to this point.\\ A vertex contains the data which describes a point\\
%   (as part of the model).

%   \textbf{Fragment}  --~a data structure  which stores the information\\
%   produced by the rasterizer.\\ A fragment contains the data which describes a pixel\\ (as
%   part of the frame buffer).

%   \vspace{1em}


%   point $\to$ vertex \\[0.3em]
%   pixel $\to$ fragment

% \end{frame}
% % ##################





% % ##################
% \begin{frame}
%   \frametitle{From vertices to fragments}


%   \begin{textblock}{10.5}(1,2)

%     \begin{tikzpicture}

%       \node[scale=0.9] at (-2.3,3.5) {points (vertices)};


%       % Układ współrzędnych
%       \draw[axis arrow,color=jAxisRed] (-3.4,0) -- (-0.65,0);

%       \node[symbol] at (-0.95,-0.4) {$x$};


%       \draw[axis arrow,color=jAxisGreen] (-3.4,0) -- (-3.4,2.75);

%       \node[symbol] at (-3,2.45) {$y$};


%       \draw[axis arrow,color=jAxisBlue] (-3.4,0,0) -- (-3.4,0,2.75);

%       \node[symbol] at (-4.35,-0.6) {$z$};



%       \node[scale=0.9] at (-2.3,-2) {continuous space};



%       % Siatka pikseli
%       \node[scale=0.9] at (4.1,3.5) {pixels (fragments)};


%       \draw[very thin,step=0.25,color=jMathTextFGColor!45!white]
%       (2.5,-1.5) grid (5.75,1.5);

%       \draw[very thin,color=jMathTextFGColor!45!white] (2.5,-1.5) --
%       (2.5,1.5);


%       \node[scale=0.9] at (4.15,-2) {discrete space};

%       \node[scale=0.9] at (4.15,-2.4) {(granular space)};



%       % Duża strzałka
%       \draw[thick connection arrow] (-0.25,0) -- +(2.35,0);

%       \node[scale=0.9,color=jOrange] at (0.72,0.5) {rendering};

%     \end{tikzpicture}

%   \end{textblock}

% \end{frame}
% % ##################






% % ##################
% \begin{frame}
%   \frametitle{Basic concepts of computer graphics}


%   \textbf{Lighting} --~modeling of light sources.

%   \vspace{0.5em}


%   \textbf{Shading} --~creating the impression of three-dimensionality\\ by
%   dimming the colors of the drawn images of objects\\ (differentiating the
%   brightness of pixels of the same color).\\ Shading is based on calculations
%   related to the emitted, reflected\\ or transmitted light from the
%   objects to the observer.

%   \vspace{0.5em}


%   \textbf{Shaders} --~applications that perform shading calculations\\ in
%   cooperation with graphics hardware, e.g. a graphics card.

% \end{frame}
% % ##################





% % ##################
% \begin{frame}
%   \frametitle{Each model is based on  a set of vertices}


%   \begin{figure}

%     \includegraphics
%     {./PresentationPictures/Lecture_01/Malpka_Suzanne_obraz.png}

%     \includegraphics
%     {./PresentationPictures/Lecture_01/Malpka_Suzanne_opis.png}

%     Suzanne Monkey (Blender) [1]

%   \end{figure}

% \end{frame}
% % ##################





% % ##################
% \begin{frame}
%   \frametitle{Each model is based on  a set of vertices}


%   \begin{figure}

%     \includegraphics[scale=0.185]
%     {./PresentationPictures/Lecture_01/Lajkonik_wierzcholki.png}

%     Lajkonik

%   \end{figure}

% \end{frame}
% % ##################





% % ##################
% \begin{frame}
%   \frametitle{Vertex attributes}


%   The data entered by the user on the vertex shader input\\ (the beginning
%   of the calculation on the graphics card) \\is called vertex  attributes.
%   The attributes can be scalars,\\ vectors, or matrices, and their values
%   can be singular\\ or double-precision numbers. In the source code of
%   a~shader,\\ you can refer to attributes by arbitrarily chosen names.

% \end{frame}
% % ##################





% % ##################
% \begin{frame}
%   \frametitle{Vertex attributes}


%   The old versions of OpenGL used the following attributes.

%   \vspace{1em}


%   \texttt{gl\_Vertex} \hphantom{aaaaaaaaaaa} \hspace{0.45em}
%   Position (vec4) \\
%   \texttt{gl\_Normal} \hphantom{aaaaaaaaaaa} \hspace{0.45em}
%   Normal \hspace{-0.1em} (vec4) \\
%   \texttt{gl\_Color} \hphantom{aaaaaaaaaaaa} \hspace{0.45em}
%   Primary color~of vertex (vec4) \\
%   \texttt{gl\_MultiTexCoord0} \hphantom{aaaa} Texture coordinate~of
%   texture unit~0 (vec4) \\
%   \texttt{gl\_MultiTexCoord1} \hphantom{aaaa} Texture coordinate~of
%   texture unit~1 (vec4) \\
%   \texttt{gl\_MultiTexCoord2} \hphantom{aaaa} Texture coordinate~of
%   texture unit~2 (vec4) \\
%   \texttt{gl\_MultiTexCoord3} \hphantom{aaaa} Texture coordinate~of
%   texture unit~3 (vec4) \\
%   \texttt{gl\_MultiTexCoord4} \hphantom{aaaa} Texture coordinate~of
%   texture unit~4 (vec4) \\
%   \texttt{gl\_MultiTexCoord5} \hphantom{aaaa} Texture coordinate~of
%   texture unit~5 (vec4) \\
%   \texttt{gl\_MultiTexCoord6} \hphantom{aaaa} Texture coordinate~of
%   texture unit~6 (vec4) \\
%   \texttt{gl\_MultiTexCoord7} \hphantom{aaaa} Texture coordinate~of
%   texture unit~7 (vec4) \\
%   \texttt{gl\_FogCoord} \hphantom{aaaaaaaaa} \hspace{0.45em} Fog Coord
%   (float)

% \end{frame}
% % ##################






% % ##################
% \begin{frame}
%   \frametitle{Basic concepts of computer graphics}


%   \textbf{Geometric primitives}, also called \textbf{basic objects},\\ are
%   the basic geometric figures used for building models.

%   In the OpenGL library, primitives are used as patterns to interpret\\ the
%   vertex stream – with which geometric figure a given fragment\\ of the
%   stream should be associated.

% \end{frame}
% % ##################





% % ##################
% \begin{frame}
%   \frametitle{Primitives in the OpenGL library}


%   \begin{itemize}
%     \RaggedRight

%   \item Point

%   \item Line

%   \item Line strip

%   \item Line loop

%   \item Triangle

%   \item Triangle strip

%   \item Triangle fan

%   \end{itemize}

% \end{frame}
% % ##################





% % ##################
% \begin{frame}
%   \frametitle{Primitives in the OpenGL library}


%   Primitives deemed as not recommended in OpenGL 3.0\\ and rejected in OpenGL
%   3.1.

%   \vspace{-0.5em}

%   \begin{itemize}
%     \RaggedRight

%   \item Quad

%   \item Quad strip

%   \item Polygon

%   \end{itemize}

%   \vspace{1em}


%   OpenGL 3.2 has introduced additional primitives\\ for use   with the
%   geometry shader: \textsc{gl\_lines\_adjacency}, \\
%   \textsc{gl\_line\_strip\_adjacency},
%   \textsc{gl\_triangles\_adjacency}, \\
%   \textsc{gl\_triangle\_strip\_adjacency}. In OpenGL 4.0,\\ an additional
%   primitive for tessellation: \textsc{gl\_patches}.

% \end{frame}
% % ##################





% % ##################
% \begin{frame}
%   \frametitle{Primitives in the OpenGL library}


%   \begin{figure}

%     \begin{tikzpicture}[color=jMathTextFGColor, text color
%       loc/.style={color=jNormalTextFGColor}]


%       % GL_POINTS
%       \fill (0,0) circle [radius=1pt] node[text color loc,left] {0};

%       \fill (0.3,0.5) circle [radius=1pt] node[text color loc,left]
%       {1};

%       \fill (0.4,-0.5) circle [radius=1pt] node[text color loc,right]
%       {2};

%       \fill (1,-0.3) circle [radius=1pt] node[text color loc,right]
%       {3};

%       \fill (1.1,0.5) circle [radius=1pt] node[text color loc,right]
%       {4};

%       \node[text color loc] at (0.6,-1.2)
%       {\textsc{gl}\_\textsc{points}};





%       % GL_LINES
%       \begin{scope}[xshift=4.5cm,yshift=0.5cm]

%         \draw[very thick] (0,0) node[text color loc,left] {0} --
%         (1,-1) node[text color loc,right] {1};

%         \draw[very thick] (0.2,-0.9) node[text color loc,left] {2} --
%         (1.7,0) node[text color loc,above left] {3};

%         \draw[very thick] (1.65,-0.8) node[text color loc,below] {4}
%         -- (2.05,-0.2) node[text color loc,right] {5};

%         \node[text color loc] at (1.05,-1.7)
%         {\textsc{gl}\_\textsc{lines}};

%       \end{scope}





%       % GL_LINE_STRIP
%       \begin{scope}[xshift=-0.3cm,yshift=-4cm]

%         \draw[very thick] (0,0) node[text color loc,left] {0} --
%         (0.6,1) node[text color loc,above left] {1} -- (0.75,0)
%         node[text color loc,left] {2} -- (1.6,0.95) node[text color
%         loc,above] {3} -- (2.35,0.2) node[text color loc,right] {4} --
%         (0.95,0.8) node[text color loc,above] {5} -- (1.5,-0.1)
%         node[text color loc,right] {6};

%         \node[text color loc] at (1.1,-0.8)
%         {\textsc{gl}\_\textsc{line}\_\textsc{strip}};

%       \end{scope}





%       % % GL_LINE_LOOP
%       \begin{scope}[xshift=4.8cm,yshift=-3.5cm]

%         \draw[very thick] (0,0) node[text color loc,below] {0} --
%         (-0.05,0.8) node[text color loc,left] {1} -- (1.2,0.7)
%         node[text color loc,above] {2} -- (1.9,-0.2) node[text color
%         loc,right] {3} -- (0.8,-0.6) node[text color loc,left] {4} --
%         (0.75,0.15) node[text color loc,right] {5} -- cycle;

%         \node[text color loc] at (1,-1.3)
%         {\textsc{gl}\_\textsc{line}\_\textsc{loop}};

%       \end{scope}

%     \end{tikzpicture}

%   \end{figure}

% \end{frame}
% % ##################





% % ##################
% \begin{frame}
%   \frametitle{Primitives in the OpenGL library}


%   \begin{figure}

%     \centering

%     \begin{tikzpicture}[color=jMathTextFGColor,text color
%       loc/.style={color=jNormalTextFGColor}]


%       % GL_TRIANGLES
%       \fill[color=jMathTextFGColor,opacity=0.45] (0,0) -- (1.1,-0.4) --
%       (0.6,-1.5) -- cycle;

%       \draw[very thick] (0,0) node[text color loc,left] {0} --
%       (1.1,-0.4) node[text color loc,right] {1} -- (0.6,-1.5)
%       node[text color loc,left] {2} -- cycle;


%       \fill[color=jMathTextFGColor,opacity=0.45] (2.1,-1.4) -- (2.85,-0.1)
%       -- (3.6,-1.4) -- cycle;

%       \draw[very thick] (2.1,-1.4) node[text color loc,left] {3} --
%       (2.85,-0.1) node[text color loc,above] {4} -- (3.6,-1.4)
%       node[text color loc,right] {5} -- cycle;


%       \fill[color=jMathTextFGColor,opacity=0.45] (4.5,-1.8) -- (5.3,-1.8)
%       -- (4.95,-0.5) -- cycle;

%       \draw[very thick] (4.5,-1.8) node[text color loc,left] {6} --
%       (5.3,-1.8) node[text color loc,right] {7} -- (4.95,-0.5)
%       node[text color loc,above] {8} -- cycle;

%       \node[text color loc] at (2.9,-2.5)
%       {\textsc{gl}\_\textsc{triangles}};





%       % GL_TRIANGLE_STRIP
%       \begin{scope}[xshift=-1.4cm,yshift=-5cm]

%         \fill[color=jMathTextFGColor,opacity=0.45] (0,0) -- (0.8,1.2) --
%         (2,1) -- (3.3,1) -- (3,-0.3) -- (1,0) -- cycle;

%         \draw[very thick] (0,0) node[text color loc,left] {0} --
%         (0.8,1.2) node[text color loc,above] {1} -- (2,1) node[text
%         color loc,above] {3} -- (3.3,1) node[text color loc,right] {5}
%         -- (3,-0.3) node[right,text color loc] {4} -- (1,0) node[text
%         color loc,below] {2} -- cycle;

%         % 1 --> 2
%         \draw[very thick] (0.8,1.2) -- (1,0);

%         % 2 --> 3
%         \draw[very thick] (1,0) -- (2,1);

%         % 3 -> 4
%         \draw[very thick] (2,1) -- (3,-0.3);

%         \node[text color loc] at (1.6,-1.2)
%         {\textsc{gl}\_\textsc{triangle}\_\textsc{strip}};

%       \end{scope}





%       % GL_TRIANGLE_FAN
%       \begin{scope}[xshift=4.9cm,yshift=-5cm]

%         \fill[color=jMathTextFGColor,opacity=0.45] (0,0) -- (-1.5,0.3) --
%         (-1.1,0.7) -- (0,1.1) -- (1,1.1) -- (1.8,0) -- cycle;

%         \draw[very thick] (0,0) node[text color loc,below] {0} --
%         (-1.5,0.3) node[text color loc,left] {1} -- (-1.1,0.7)
%         node[text color loc,above] {2} -- (0,1.1) node[text color
%         loc,above] {3} -- (1,1.1) node[text color loc,above left] {4}
%         -- (1.8,0) node[text color loc,right] {5} -- cycle;

%         % 2 --> 0
%         \draw[very thick] (-1.1,0.7) -- (0,0);

%         % 3 --> 0
%         \draw[very thick] (0,1.1) -- (0,0);

%         % 4 --> 0
%         \draw[very thick] (1,1.1) -- (0,0);

%         \node[text color loc] at (0.2,-1.2)
%         {\textsc{gl}\_\textsc{triangle}\_\textsc{fan}};

%       \end{scope}

%     \end{tikzpicture}

%   \end{figure}

% \end{frame}
% % ##################





% % ##################
% \begin{frame}
%   \frametitle{Primitives in the OpenGL library (deprecated since ver.~3.1)}


%   \begin{figure}

%     \begin{tikzpicture}[color=jMathTextFGColor, text color
%       loc/.style={color=jNormalTextFGColor}]


%       % GL POLYGON
%       \fill[color=jMathTextFGColor,opacity=0.45] (0,0) -- (0.3,0.9) --
%       (1,0.9) -- (2,-0.1) -- (0.7,0.3) -- cycle;

%       \draw[very thick] (0,0) node[text color loc,left] {0} --
%       (0.3,0.9) node[text color loc,above] {1} -- (1,0.9) node[text
%       color loc,above right] {2} -- (2,-0.1) node[text color
%       loc,below] {3} -- (0.7,0.3) node[text color loc,below] {4} --
%       cycle;

%       \node[text color loc] at (0.9,-1)
%       {\textsc{gl}\_\textsc{polygon}};





%       % GL_QUADS
%       \begin{scope}[xshift=4.7cm,yshift=0.9cm]

%         \fill[color=jMathTextFGColor,opacity=0.45] (0,0) -- (0.4,-1) --
%         (1.2,-0.9) -- (0.8,0.2) -- cycle;

%         \draw[very thick] (0,0) node[text color loc,above] {0} --
%         (0.4,-1) node[text color loc,below] {1} -- (1.2,-0.9)
%         node[text color loc,below] {2} -- (0.8,0.2) node[text color
%         loc,above] {3} -- cycle;


%         \fill[color=jMathTextFGColor,opacity=0.45] (1.9,-1) -- (1.9,0.2)
%         -- (3.4,0.2) -- (2.8,-1) -- cycle;

%         \draw[very thick] (1.9,-1) node[text color loc,below] {4} --
%         (1.9,0.2) node[text color loc,above] {5} -- (3.4,0.2)
%         node[text color loc,above] {6} -- (2.8,-1) node[text color
%         loc,below] {7} -- cycle;

%         \node[text color loc] at (1.6,-1.9)
%         {\textsc{gl}\_\textsc{quads}};
%       \end{scope}





%       % GL_QUAD_STRIP
%       \begin{scope}[xshift=2.3cm,yshift=-3.8cm]

%         \fill[color=jMathTextFGColor,opacity=0.45] (0,0) -- (0,0.8) --
%         (0.5,1.1) -- (1.5,1.1) -- (2.2,0.7) -- (2.2,0.3) -- (1.4,-0.3)
%         -- (0.5,0) -- cycle;

%         \draw[very thick] (0,0) node[text color loc,left] {0} --
%         (0,0.8) node[text color loc,left] {1} -- (0.5,1.1) node[text
%         color loc,above] {3} -- (1.5,1.1) node[text color loc,above]
%         {5} -- (2.2,0.7) node[text color loc,above right] {7} --
%         (2.2,0.3) node[text color loc,below right] {6} -- (1.4,-0.3)
%         node[text color loc,below] {4} -- (0.5,0) node[text color
%         loc,below] {2} -- cycle;

%         % 3 --> 2
%         \draw[very thick] (0.5,1.1) -- (0.5,0);

%         % 5 --> 4
%         \draw[very thick] (1.5,1.1) -- (1.4,-0.3);

%         \node[text color loc] at (1.2,-1.25)
%         {\textsc{gl}\_\textsc{quad}\_\textsc{strip}};
%       \end{scope}

%     \end{tikzpicture}

%   \end{figure}

% \end{frame}
% % ##################





% % ##################
% \begin{frame}
%   \frametitle{Texturing and blending}


%   \textbf{Texturing} --~overlaying additional   images (textures) on
%   surfaces stretched on vertices (faces).

%   \textbf{Blending} --~mixing colors to produce another color.






%   \begin{textblock}{5.1}(1,3.9)

%     \includegraphics[scale=0.14]
%     {./PresentationPictures/Lecture_01/Lajkonik_wierzcholki.png}

%   \end{textblock}


%   \begin{textblock}{5.1}(6.8,3.9)

%     \includegraphics[scale=0.14]
%     {./PresentationPictures/Lecture_01/Lajkonik_tekstury.png}

%   \end{textblock}


%   \begin{textblock}{2.6}(5.1,6.1)

%     \begin{tikzpicture}

%       \draw[thick connection arrow] (0,0) -- (2.35,0);

%       \node[scale=0.9,color=jOrange] at (0.97,0.5) {rendering};

%     \end{tikzpicture}

%   \end{textblock}

% \end{frame}
% % ##################





% % ##################
% \begin{frame}
%   \frametitle{Texturing and blending}


%   \begin{textblock}{5.1}(1,2.5)

%     \includegraphics[scale=0.14]
%     {./PresentationPictures/Lecture_01/LOS_scena_1.png}

%   \end{textblock}


%   \begin{textblock}{5.1}(6.8,2.5)

%     \includegraphics[scale=0.14]
%     {./PresentationPictures/Lecture_01/LOS_scena_2.png}

%   \end{textblock}


%   \begin{textblock}{2.6}(5.1,2.9)

%     \begin{tikzpicture}

%       \draw[thick connection arrow] (0,0) -- (2.35,0);

%       \node[scale=0.9,color=jOrange] at (0.97,0.5) {rendering};

%     \end{tikzpicture}

%   \end{textblock}

% \end{frame}
% % ##################





% % ##################
% \begin{frame}
%   \frametitle{Communication between CPU and GPU}


%   \begin{figure}

%     \centering

%     \begin{tikzpicture}[node distance=1.7em, small block loc/.style =
%       {rectangle, rounded corners, fill=jAxisBlue, text width=4em,
%         text centered, minimum height=4em,
%         text=jMathTextFGStyleDark}, block loc/.style = {rectangle,
%         fill=jAxisBlue, text centered, text width=6em, minimum
%         height=4em, text=jMathTextFGStyleDark},
%       block circle loc/.style = {circle, minimum size=3.3em, text
%         centered, text=jMathTextFGStyleDark}]


%       \node[block circle loc,fill=jOrange,scale=1.3] at (0,0) (CPU)
%       {CPU};

%       \node[small block loc,right=of CPU] (Aplikacja) {Application};

%       \draw[thick connection arrow] (CPU) -- (Aplikacja);



%       \node[small block loc,right=of Aplikacja] (OpenGL DirectX) {OpenGL
%         DirectX};

%       \draw[thick connection arrow] (Aplikacja) -- (OpenGL DirectX);



%       \node[small block loc,align=left,right=of OpenGL DirectX]
%       (Sterownik)
%       {{\small Video card \\
%           driver}};

%       \draw[thick connection arrow] (OpenGL DirectX) -- (Sterownik);



%       \node[block circle loc,fill=jOrange,scale=1.3,right=of
%       Sterownik] (GPU) {GPU};

%       \draw[thick connection arrow] (Sterownik) -- (GPU);


%       \node[block loc,below=6em of CPU] (Pamiec Operacyjna) {Main Memory};

%       \draw[{Triangle[scale width=0.6,scale
%         length=0.3]}-{Triangle[scale width=0.6,scale length=0.3]},
%       line width=7.5,color=jNormalTextFGStyleLight] (CPU) -- (Pamiec
%       Operacyjna);



%       \node[block loc,below=6em of GPU] (Pamiec Wideo) {Video Memory};

%       \draw[{Triangle[scale width=0.6,scale
%         length=0.3]}-{Triangle[scale width=0.6,scale length=0.3]},
%       line width=7.5,color=jNormalTextFGStyleLight] (GPU) -- (Pamiec
%       Wideo);


%       \draw[thick connection arrow] (Pamiec Operacyjna) -- (Pamiec
%       Wideo);

%       \node[align=left] at (4.3,-2.7) {vertex data \\
%         texture data \\
%         shader parameters};

%     \end{tikzpicture}

%   \end{figure}

% \end{frame}
% % ##################





% % ##################
% \begin{frame}
%   \frametitle{Data stored in the video memory}


%   \begin{itemize}
%     \RaggedRight

%     \setlength{\itemsep}{0.3em}

%   \item Pixel data of the image visible in the viewport --\\
%     front image buffer

%   \item Pixel data of the image invisible in the viewport --\\
%     back image buffer

%   \item Data that describes how deep a pixel lies in the image --\\ depth
%     buffer (also called z-buffer)

%   \item Data  used  to enable or disable drawing
%     on a per-pixel basis -- stencil buffer

%   \item Vertex data -- vertex buffers

%   \item Texture data -- texture maps

%   \end{itemize}

% \end{frame}
% % ##################





% % ##################
% \begin{frame}
%   \frametitle{Data stored in the video memory}


%   In the video memory we can store any data, just like in the
%   \textsc{ram}. Compared to the usual \textsc{ram}, the access to this data is
%   not through\\ pointers, but through buffers, textures and other objects.

%   Examples of data objects \vspace{-0.5em}
%   \begin{itemize}
%     \RaggedRight

%     \setlength{\itemsep}{0.3em}

%   \item Vertex Buffer
%     Objects

%   \item Pixel Buffer Objects

%   \item Shader Storage
%     Buffer Objects

%   \item Uniform Buffer
%     Objects

%   \item Sampler Objects

%   \end{itemize}

% \end{frame}
% % ##################





% % ##################
% \begin{frame}
%   \frametitle{Graphics pipeline (rendering pipeline)}


%   \begin{figure}

%     \begin{tikzpicture}

%       \node[diagrams block 1] (pobranie wierzcholkow)
%       {Vertex \\
%         Specification};

%       \node[diagrams block 1,below=of pobranie wierzcholkow] (szader
%       wierzcholkow) {Vertex Shader};

%       \draw[thick connection line] (pobranie wierzcholkow) -- (szader
%       wierzcholkow);



%       \node[diagrams block 1,below=of szader wierzcholkow] (szader
%       sterowania teselacja) {Tessellation Control Shader};

%       \draw[thick connection line] (szader wierzcholkow) -- (szader
%       sterowania teselacja);



%       \node[diagrams block 1,right=of szader sterowania teselacja]
%       (teselacja) {Tessellation};

%       \draw[thick connection line] (szader sterowania teselacja) --
%       (teselacja);



%       \node[diagrams block 1,above=of teselacja] (szader wyliczania
%       teselacji) {Tessellation Evaluation Shader};

%       \draw[thick connection line] (teselacja) -- (szader wyliczania
%       teselacji);



%       \node[diagrams block 1,above=of szader wyliczania teselacji]
%       (szader geometrii) {Geometry Shader};

%       \draw[thick connection line] (szader wyliczania teselacji) --
%       (szader geometrii);



%       \node[diagrams block 1,right=of szader geometrii] (rasteryzacja)
%       {Rasterization};

%       \draw[thick connection line] (szader geometrii) -- (rasteryzacja);



%       \node[diagrams block 1,below=of rasteryzacja] (szader fragmentow)
%       {Fragment Shader};

%       \draw[thick connection line] (rasteryzacja) -- (szader fragmentow);



%       \node[diagrams block 1,below=of szader fragmentow] (bufor ramki)
%       {Frame buffer \\ Operations};

%       \draw[thick connection line] (szader fragmentow) -- (bufor ramki);

%     \end{tikzpicture}

%   \end{figure}

% \end{frame}
% % ##################





% % ##################
% \begin{frame}
%   \frametitle{Basic concepts of computer graphics}


%   The graphics pipeline can be divided into two parts.

%   \begin{itemize}
%     \setlength{\itemsep}{0.75em}

%   \item \textbf{Front end} --~processing vertices and primitives, creating points, lines and triangles (assembling primitives), up to passing data\\ to rasterization.

%   \item \textbf{Back end} --~pixel processing, depth checking, fragment shading, color blending,  updating the final image.


%   \end{itemize}

% \end{frame}
% % ##################





% % ##################
% \begin{frame}
%   \frametitle{Basic concepts of computer graphics}


%   \textbf{Vertex shader}~--
%   a program that runs once for each downloaded\\ vertex or  control point of a patch (if you use tessellation).

%   \vspace{0.5em}


%   \textbf{Tessellation control shader}~--   a program that runs once for each  patch. A \textbf{patch} is a high-order primitive.\\ It is accessible only
%   if we use tessellation.


%   \vspace{0.5em}


%   \textbf{Tessellation} --~a fixed function transforming patches into smaller and simpler primitives (e.g. triangles).

%   \vspace{0.5em}


%   \textbf{Tessellation evaluation shader} --   a program that runs once for each vertex created by the tessellation mechanism.


% \end{frame}
% % ##################





% % ##################
% \begin{frame}
%   \frametitle{Basic concepts of computer graphics}


%   \textbf{Geometry shader} --~a program that runs once for each primitive. \\It has access to all vertices of the primitive. It can transform one type of primitives
%   into another, e.g. points can be transformed to triangles.


%   \vspace{0.5em}


%   \textbf{Compute shader} --~a program
%   that runs outside the graphics pipeline. It has no specific
%   role and can deal with any calculation;  it can have any
%   input/output. Used in graphics for postprocessing.


%   \vspace{0.5em}


%   \textbf{Clipping} --~after  creating primitives from the vertices, they are clipped in order to be contained in the visible area.

%   \vspace{0.5em}


%   \textbf{Viewport} --~a window area,
%   in which OpenGL renders its image.


% \end{frame}
% % ##################





% % ##################
% \begin{frame}
%   \frametitle{Vertex transformations (OpenGL nomenclature)}


%   \begin{textblock}{10.8}(1,1.2)

%     \begin{tikzpicture}[node distance=2.6em,scale=0.7, block loc/.style
%       = {rectangle, rounded corners, fill=jAxisBlue, text width=6em,
%         text centered, minimum height=4.1em,
%         text=jMathTextFGStyleDark}]


%       \node[block loc] (przestrzen obiektu)
%       {Object \\[-0.2em]
%         space};

%       \node[block loc,right=of przestrzen obiektu] (przestrzen swiata)
%       {World \\[-0.2em]
%         space};

%       \draw[thick connection arrow] (przestrzen obiektu) -- (przestrzen
%       swiata);


%       \node[block loc,right=of przestrzen swiata] (przestrzen kamery)
%       {Camera \\[-0.2em]
%         space};

%       \draw[thick connection arrow] (przestrzen swiata) -- (przestrzen
%       kamery);


%       \node[block loc,below=2.6em of przestrzen kamery] (jednorodna
%       przestrzen)
%       {Homogeneous \\[-0.2em]
%         clip space};

%       \draw[thick connection arrow] (przestrzen kamery) -- (jednorodna
%       przestrzen);


%       \node[block loc,below=2.6em of jednorodna przestrzen] (przestrzen
%       widoku) {Viewport space};

%       \draw[thick connection arrow] (jednorodna przestrzen) --
%       (przestrzen widoku);



%       \draw[thick connection arrow,rounded corners] (przestrzen obiektu)
%       -- ++(0,2) -| (przestrzen kamery);

%       \node at (4.8,2.65) {\alert{model-view transformation}};


%       \node at (11.77,-1.6) (rzutowanie) {\alert{projection}};


%       \node[rectangle,align=left] at (12,-5.05) (transformacja okna)
%       {\alert{viewport} \\[-0.2em]
%         \alert{transformation}};

%     \end{tikzpicture}

%   \end{textblock}

% \end{frame}
% % ##################





% % ##################
% \begin{frame}[label=Przestrzen-uklad-wspolrzednych-1]
%   \frametitle{Spaces --~coordinate systems}


%   \begin{textblock}{2.4}(10,1.4)

%     \hyperlink{Uwagi-pojecie-przestrzeni-Przestrzen-afiniczna-i-wektorowa-1}
%     {\beamergotobutton{Terminological notes}}

%   \end{textblock}


%   \vspace{1.8em}


%   \textbf{Object space}, or \textbf{model space}, is associated
%   with the local coordinate system native to the specific model.
%   In this space, we usually specify the positions of the vertices of the model.


%   \vspace{0.5em}


%   \textbf{World space} is 	associated with the global coordinate system\\ native to the scene. In this space, we usually specify the positions\\ and orientations of all models.


%   \vspace{0.5em}


%   \textbf{Camera space}, or \textbf{eye space}, is associated with the
%   coordinate\\ system  defined by the camera, the eye or  the screen.\\ The axes
%   $x$ and ~$y$ lie in the plane of the screen,\\ and the axis ~$z$ is perpendicular
%   to the screen.


% \end{frame}
% % ##################





% % ##################
% \begin{frame}
%   \frametitle{Spaces --~coordinate systems}


%   \textbf{Homogeneous clip space} is associated with the coordinate\\ system that has
%   highlighted visibility area.\\ It is a cube with vertices
%   in $( -1 \hspace{0.5em} {-1} \hspace{0.5em} {-1} )^{ T }$,
%   $( 1 \hspace{0.5em} {-1} \hspace{0.5em} {-1} )^{ T }$, \ldots,
%   $( 1 \hspace{0.5em} 1 \hspace{0.5em} 1 )^{ T }$.\\ Graphic primitives
%   which do not fit entirely inside the cube\\ are clipped.


%   \vspace{0.5em}


%   \textbf{Window space}, or \textbf{viewport space}, is
%   associated with the coordinate system defined by the view window.\\
%   It contains the information about  pixel coordinates and their depth.


% \end{frame}
% % ##################





% % ##################
% \begin{frame}
%   \frametitle{Rasterization and fragment operations}


%   \textbf{Rasterization} --~taking a graphic primitive  and converting it\\ into a raster image.\\ Graphic primitives are obtained after the transformation of
%   vertices of the model to the window space.\\ A raster image is a series of pixels.


%   \vspace{0.5em}


%   \textbf{Fragment} --~the information on
%   the localization of each pixel\\ of any primitive with its depth,
%   interpolated vertex colors \\
%   and interpolated texture coordinates.


% \end{frame}
% % ##################





% % ##################
% \begin{frame}
%   \frametitle{Rasterization and fragment operations}


%   \begin{textblock}{7.4}(2.7,1.6)

%     \begin{tikzpicture}[node distance=4em]
%       \draw[thick connection arrow] (-0.8,-0.5) -- (-1.35,-2);

%       \draw[thick connection arrow] (0.825,-0.5) -- (1.4,-2);


%       \node[diagrams block 1] (rasteryzacja) {Rasterization};


%       \node[diagrams block 1] (selekcja) [above=of rasteryzacja] {Selection\\
%         visible\\ primitives};


%       \draw[thick connection arrow] (selekcja) -- (rasteryzacja);


%       \node[diagrams block 1] at (-2.2,-2.8) (cieniowanie) {Fragment shading};


%       \node[diagrams block 1] at (2.25,-2.8) (operacje) {Fragment operations};
%     \end{tikzpicture}

%   \end{textblock}

% \end{frame}
% % ##################





% % ##################
% \begin{frame}
%   \frametitle{Fragment operations}


%   \textbf{Pixel ownership test} --~an application that determines
%   if a fragment lies in the region of the currently visible viewport.

%   \vspace{0.5em}


%   \textbf{Scissor test} --~an application that may specify a rectangle\\ in
%   the viewport, to which rendering should be restricted.

%   \vspace{0.5em}


%   \textbf{Alpha test} --~selective display of pixels based on
%   transparency \\conditions.


%   \vspace{0.5em}


%   \textbf{Stencil test} --~selective display of pixels based on the data specified \\in the stencil buffer.

%   \vspace{0.5em}


%   \textbf{Depth test} --~selective display of pixels based on the data specified\\ in the depth buffer.

%   \vspace{0.5em}


%   \textbf{Blending} --~determining the final color of a pixel by
%   mixing  all color components.



% \end{frame}
% % ##################





% % ##################
% \begin{frame}
%   \frametitle{Basic concepts of computer graphics}


%   \textbf{Fragment shader} --~a program that runs once for each fragment\\ of the rendered primitive.

%   \vspace{0.5em}


%   \textbf{Double buffering} --~a drawing technique of using
%   two memory buffers for image.


%   \vspace{0.5em}


%   \textbf{Aliasing} --~a loss of information related to the finite
%   resolution\\ of image reproduction.


%   \vspace{0.5em}


%   \textbf{Anti-aliasing} --~a rendering technique with smoothing
%   edges \\(we distinguish, among others: \textsc{fsaa}, \textsc{msaa},
%   \textsc{mfaa}, \textsc{csaa}, \textsc{cfaa}, \textsc{mlaa},
%   \textsc{smaa}, \textsc{hraa}, \textsc{fxaa}, \textsc{txaa},
%   \textsc{taa}, \textsc{ssaa}).


%   \vspace{0.5em}


%   \textbf{View frustum} --~a visible
%   area of space.


% \end{frame}
% % ##################





% % ##################
% \begin{frame}
%   \frametitle{Links}


%   [1] \colorhref{www.blender.org}{www.blender.org}

% \end{frame}
% % ##################










% % ######################################
% \appendix
% % ######################################





% % ##################
% \GeometryThreeDTwoSpecialEndingSlidesEN{Questions? Thank you for your attention.}
% % ##################



% % % ##################
% % \jagiellonianendslide{Dziękuję za~uwagę.}
% % % ##################










% % ######################################
% \SectionSlideWithPicture{Terminological notes}
% % ######################################



% % ##################
% \begin{frame}[label=Uwagi-pojecie-przestrzeni-Przestrzen-afiniczna-i-wektorowa-1]
%   \frametitle{The concept of space. Affine and vector space}


%   In geometry, we use two kinds of space.\\
%   \textbf{Vector space} is the corresponding set of vectors.\\
%   \textbf{Affine space} is a set of points that can be
%   moved by vectors from some vector space.

%   An example of affine space can be a two-dimensional plane.\\
%   It has no center, any two points of it can be
%   connected by a vector,\\ and at any point one can hook a system of
%   coordinates.

%   An example of a vector space can be the set of all possible
%   velocities of a point moving in the plane. Such
%   space has a distinguished\\ element, the zero vector (zero
%   velocity of the point, the resting point), which can be considered its ``center''.





%   \begin{textblock}{2.1}(1,8.7)

%     \hyperlink{Przestrzen-uklad-wspolrzednych-1}
%     {\beamerreturnbutton{Back to lecture}}

%   \end{textblock}


%   \begin{textblock}{2.1}(10,8.7)

%     \hyperlink{Uwagi-pojecie-przestrzeni-Przestrzen-afiniczna-i-wektorowa-2}
%     {\beamergotobutton{Continued \hspace{3.5em}}}

%   \end{textblock}

% \end{frame}
% % ##################





% % ##################
% \begin{frame}[label=Uwagi-pojecie-przestrzeni-Przestrzen-afiniczna-i-wektorowa-2]
%   \frametitle{The concept of space. Space and coordinate system}


%   In this lecture we use the word ``space'' in both senses,
%   its exact\\ meaning should be read from the context. For example,
%   ``scene space'' can mean both the set of all points
%   forming the scene, i.e. the affine space, as well as the space of
%   position vectors of the scene points\\ with respect to selected point
%   $\pointO$, i.e. the vector space.

%   \vspace{0.5em}


%   \textbf{Coordinate system of vector space} is based on a system of three
%   linearly independent vectors
%   $<\! \vecWscript{1}, \vecWscript{2}, \vecWscript{3} \!>$ (other
%   cases we do not consider). Since
%   $\vecV = \Vscript{x} \vecWscript{1} + \Vscript{y} \vecWscript{2} +
%   \Vscript{z} \vecWscript{3}$, this coordinate system allows us to
%   replace in  calculations any vector $\vecV$ with a column of numbers
%   $\begin{bmatrix} \Vscript{x} & \Vscript{y} &
%                                                \Vscript{z} \end{bmatrix}^{ T }$. The numbers $\Vscript{x}$,
%                                              $\Vscript{y}$, $\Vscript{z}$ we call
%                                              \textbf{coordinates\\ of vector $\vecV$ in the basis}
%                                              $<\! \vecWscript{1}, \vecWscript{2}, \vecWscript{3} \!>$.
%                                              Note that without\\ knowledge of the base, the coordinates of the vector are
%                                              useless.




%                                              \begin{textblock}{2.1}(1,8.7)

%                                                \hyperlink{Przestrzen-uklad-wspolrzednych-1}
%                                                {\beamerreturnbutton{Back to lecture}}

%                                              \end{textblock}


%                                              \begin{textblock}{2.1}(4,8.7)

%                                                \hyperlink{Uwagi-pojecie-przestrzeni-Przestrzen-afiniczna-i-wektorowa-1}
%                                                {\beamerreturnbutton{Previous part \hspace{1em}}}

%                                              \end{textblock}


%                                              \begin{textblock}{2.1}(10,8.7)

%                                                \hyperlink{Uwagi-pojecie-przestrzeni-Przestrzen-afiniczna-i-wektorowa-3}
%                                                {\beamergotobutton{Continued \hspace{3.5em}}}

%                                              \end{textblock}

%                                            \end{frame}
%                                            % ##################





%                                            % ##################
%                                            \begin{frame}[label=Uwagi-pojecie-przestrzeni-Przestrzen-afiniczna-i-wektorowa-3]
%                                              \frametitle{The concept of space. Space and coordinate system}


%                                              \textbf{Coordinate system of affine space} is point
%                                              $\pointQ$ of this space\\ and a system of three linearly independent
%                                              vectors $<\! \vecWscript{1}, \vecWscript{2}, \vecWscript{3} \!>$\\\
%                                              from the corresponding vector space. The coordinate system
%                                              is given as
%                                              $<\! \pointQ; \vecWscript{1}, \vecWscript{2}, \vecWscript{3} \!>$.
%                                              Since{ } $\pointP = \pointQ + \vecV $
%                                              $= \pointQ + \Pscript{x} \vecWscript{1} + \Pscript{y} \vecWscript{2}
%                                              + \Pscript{z} \vecWscript{3}$, \\ this coordinate system allows us to
%                                              substitute in  calculations any\\ point $\pointP$ with a column
%                                              $\Big[ \Pscript{x} \hspace{0.6em} \Pscript{y} \hspace{0.6em}
%                                              \Pscript{z} \Big]^{ T }$.
%                                              The numbers $\Pscript{x}$, $\Pscript{y}$,
%                                              $\Pscript{z}$ are called \textbf{coordinates of  point $\pointP$ in the basis
%                                                $<\! \pointQ; \vecWscript{1}, \vecWscript{2}, \vecWscript{3}
%                                                \!>$}. \\[0.1em]
%                                              Note that without knowledge of the base, the coordinates are useless.

%                                              \vspace{0.5em}


%                                              \textbf{Homogeneous coordinates} are coordinates in which each
%                                              point\\ or vector is translated into a column of four
%                                              coordinates. For a point, the last coordinate determines the positive ``weight'' of the
%                                              coordinates and for a vector it is equal to $0$. We can
%                                              define this kind\\ of coordinates for each of the spaces discussed in the lecture.






%                                              \begin{textblock}{2.1}(1,8.7)

%                                                \hyperlink{Przestrzen-uklad-wspolrzednych-1}
%                                                {\beamerreturnbutton{Back to lecture}}

%                                              \end{textblock}


%                                              \begin{textblock}{2.1}(4,8.7)

%                                                \hyperlink{Uwagi-pojecie-przestrzeni-Przestrzen-afiniczna-i-wektorowa-2}
%                                                {\beamerreturnbutton{Previous part \hspace{1em}}}

%                                              \end{textblock}


%                                              \begin{textblock}{2.1}(10,8.7)

%                                                \hyperlink{Uwagi-uklady-wspolrzednych-w-przestrzeni-Rodzaje-ukladow-1}
%                                                {\beamergotobutton{Continued \hspace{3.5em}}}

%                                              \end{textblock}

%                                            \end{frame}
%                                            % ##################





%                                            % ##################
%                                            \begin{frame}[label=Uwagi-uklady-wspolrzednych-w-przestrzeni-Rodzaje-ukladow-1]
%                                              \frametitle{Coordinate systems in computer graphics}


%                                              \textbf{World (coordinate) system} is a coordinate system
%                                              hooked\\ permanently to a selected point in space. The word
%                                              ``coordinate''\\ in the name will often be omitted for convenience, both
%                                              in this\\ and all subsequent cases.

%                                              Synonyms include: \textbf{scene (coordinate) system}, \textbf{global (coordinate) system}, \textbf{environment (coordinate) system}.

%                                              \vspace{1em}


%                                              \textbf{Camera (coordinate) system} is a coordinate system used in computer graphics defined relative to the field
%                                              of vision (viewing frustum). \\In this system, we perform calculations related to
%                                              the determination of objects to be rendered.

%                                              Synonyms include: \textbf{eye (coordinate) system}, \textbf{viewing frustum\\ (coordinate) system}, \textbf{observer (coordinate) system}.





%                                              \begin{textblock}{2.1}(1,8.7)

%                                                \hyperlink{Przestrzen-uklad-wspolrzednych-1}
%                                                {\beamerreturnbutton{Back to lecture}}

%                                              \end{textblock}


%                                              \begin{textblock}{2.1}(4,8.7)

%                                                \hyperlink{Uwagi-pojecie-przestrzeni-Przestrzen-afiniczna-i-wektorowa-3}
%                                                {\beamerreturnbutton{Previous part \hspace{1em}}}

%                                              \end{textblock}


%                                              \begin{textblock}{2.1}(10,8.7)

%                                                \hyperlink{Uwagi-uklady-wspolrzednych-w-przestrzeni-Rodzaje-ukladow-2}
%                                                {\beamergotobutton{Continued \hspace{3.5em}}}

%                                              \end{textblock}

%                                            \end{frame}
%                                            % ##################





%                                            % ##################
%                                            \begin{frame}[label=Uwagi-uklady-wspolrzednych-w-przestrzeni-Rodzaje-ukladow-2]
%                                              \frametitle{Coordinate systems in computer graphics}


%                                              \textbf{Object (coordinate) system} is a system that is permanently
%                                              attached to some object that moves around the scene. If
%                                              this object moves, then along with it moves the origin of the coordinate system. If it rotates, then simultaneously the axes of the
%                                              coordinate system rotate. Systems of spheres, boxes, etc., are examples here.

%                                              Synonyms include:  \textbf{local (coordinate) system}, \textbf{model (coordinate) system}.







%                                              \begin{textblock}{2.1}(1,8.7)

%                                                \hyperlink{Przestrzen-uklad-wspolrzednych-1}
%                                                {\beamerreturnbutton{Back to lecture}}

%                                              \end{textblock}


%                                              \begin{textblock}{2.1}(4,8.7)

%                                                \hyperlink{Uwagi-uklady-wspolrzednych-w-przestrzeni-Rodzaje-ukladow-1}
%                                                {\beamerreturnbutton{Previous part \hspace{1em}}}

%                                              \end{textblock}


%                                              \begin{textblock}{2.1}(10,8.7)

%                                                \hyperlink{Uwagi-uklady-wspolrzednych-w-przestrzeni-Rodzaje-ukladow-3}
%                                                {\beamergotobutton{Continued \hspace{3.5em}}}

%                                              \end{textblock}

%                                            \end{frame}
%                                            % ##################





%                                            % ##################
%                                            \begin{frame}[label=Uwagi-uklady-wspolrzednych-w-przestrzeni-Rodzaje-ukladow-3]
%                                              \frametitle{Coordinate systems in computer graphics}


%                                              \textbf{Homogeneous clip space}  is the three-dimensional affine space\\ obtained after the transformation of the 3D-projection of the visible field (cf. Lecture 6, S.~35), that is, the projection of three-dimensional space onto a plane
%                                              (\emph{viewport}) and a line (\emph{depth}).

%                                              \vspace{0.5em}


%                                              \textbf{Projective coordinate system} in homogeneous clip space\\ is the homogeneous coordinate system defined for the
%                                              appropriate\\ 3D-projection transformation image. Usually, it is either
%                                              perspective or orthographic projection.

%                                              \vspace{0.5em}


%                                              \textbf{Normalized device coordinate system} (\textsc{ndc}) are  	three-dimensional coordinates  for a  homogeneous clip space. We obtain
%                                              them\\ by dividing the first three homogeneous coordinates by the fourth. Coordinates \textsc{ndc} belong to the interval $[-1, 1]$.






%                                              \begin{textblock}{2.1}(1,8.7)

%                                                \hyperlink{Przestrzen-uklad-wspolrzednych-1}
%                                                {\beamerreturnbutton{Back to lecture}}

%                                              \end{textblock}


%                                              \begin{textblock}{2.1}(4,8.7)

%                                                \hyperlink{Uwagi-uklady-wspolrzednych-w-przestrzeni-Rodzaje-ukladow-2}
%                                                {\beamerreturnbutton{Previous part \hspace{1em}}}

%                                              \end{textblock}

%                                            \end{frame}
%                                            % ##################










% ####################################################################

% End of the document
\end{document}
