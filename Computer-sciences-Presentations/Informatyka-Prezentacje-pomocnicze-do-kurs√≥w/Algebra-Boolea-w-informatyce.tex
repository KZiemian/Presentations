% ------------------------------------------------------------------------------------------------------------------
% Basic configuration of Beamera class and Jagiellonian theme
% ------------------------------------------------------------------------------------------------------------------
\RequirePackage[l2tabu, orthodox]{nag}



\ifx\PresentationStyle\notset
  \def\PresentationStyle{dark}
\fi



\documentclass[10pt,t]{beamer}
\mode<presentation>
\usetheme[style=\PresentationStyle,JUlogotitle=no]{jagiellonian}




% ------------------------------------------------------
% Procesing configuration files of Jagiellonian theme loceted in directory
% "preambule"
% ------------------------------------------------------
% Configuration for polish language
% Need description
\usepackage[polish]{babel}
% Need description
\usepackage[MeX]{polski}



% ------------------------------
% Better support of polish chars in technical parts of PDF
% ------------------------------
\hypersetup{pdfencoding=auto,psdextra}

% Package "textpos" give as enviroment "textblock" which is very usefull in
% arranging text on slides.

% This is standard configuration of "textpos"
\usepackage[overlay,absolute]{textpos}

% If you need to see bounds of "textblock's" comment line above and uncomment
% one below.

% Caution! When showboxes option is on significant ammunt of space is add
% to the top of textblock and as such, everyting put in them gone down.
% We need to check how to remove this bug.

% \usepackage[showboxes,overlay,absolute]{textpos}



% Setting scale length for package "textpos"
\setlength{\TPHorizModule}{10mm}
\setlength{\TPVertModule}{\TPHorizModule}


% ---------------------------------------
% Packages written for lectures "Geometria 3D dla twórców gier wideo"
% ---------------------------------------
% \usepackage{./Geometry3DPackages/Geometry3D}
% \usepackage{./Geometry3DPackages/Geometry3DIndices}
% \usepackage{./Geometry3DPackages/Geometry3DTikZStyle}
% \usepackage{./ProgramowanieSymulacjiFizykiPaczki/ProgramowanieSymulacjiFizykiTikZStyle}
% \usepackage{./Geometry3DPackages/mathcommands}


% ---------------------------------------
% TikZ
% ---------------------------------------
% Importing TikZ libraries
\usetikzlibrary{arrows.meta}
\usetikzlibrary{positioning}





% % Configuration package "bm" that need for making bold symbols
% \newcommand{\bmmax}{0}
% \newcommand{\hmmax}{0}
% \usepackage{bm}




% ---------------------------------------
% Packages for scientific texts
% ---------------------------------------
% \let\lll\undefined  % Sometimes you must use this line to allow
% "amsmath" package to works with packages with packages for polish
% languge imported
% /preambul/LanguageSettings/JagiellonianPolishLanguageSettings.tex.
% This comments (probably) removes polish letter Ł.
\usepackage{amsmath}  % Packages from American Mathematical Society (AMS)
\usepackage{amssymb}
\usepackage{amscd}
\usepackage{amsthm}
\usepackage{siunitx}  % Package for typsetting SI units.
\usepackage{upgreek}  % Better looking greek letters.
% Example of using upgreek: pi = \uppi


\usepackage{calrsfs}  % Zmienia czcionkę kaligraficzną w \mathcal
% na ładniejszą. Może w innych miejscach robi to samo, ale o tym nic
% nie wiem.










% ---------------------------------------
% Packages written for lectures "Geometria 3D dla twórców gier wideo"
% ---------------------------------------
% \usepackage{./ProgramowanieSymulacjiFizykiPaczki/ProgramowanieSymulacjiFizyki}
% \usepackage{./ProgramowanieSymulacjiFizykiPaczki/ProgramowanieSymulacjiFizykiIndeksy}
% \usepackage{./ProgramowanieSymulacjiFizykiPaczki/ProgramowanieSymulacjiFizykiTikZStyle}





% !!!!!!!!!!!!!!!!!!!!!!!!!!!!!!
% !!!!!!!!!!!!!!!!!!!!!!!!!!!!!!
% EVIL STUFF
\if\JUlogotitle1
\edef\LogoJUPath{LogoJU_\JUlogoLang/LogoJU_\JUlogoShape_\JUlogoColor.pdf}
\titlegraphic{\hfill\includegraphics[scale=0.22]
{./JagiellonianPictures/\LogoJUPath}}
\fi
% ---------------------------------------
% Commands for handling colors
% ---------------------------------------


% Command for setting normal text color for some text in math modestyle
% Text color depend on used style of Jagiellonian

% Beamer version of command
\newcommand{\TextWithNormalTextColor}[1]{%
  {\color{jNormalTextFGColor}
    \setbeamercolor{math text}{fg=jNormalTextFGColor} {#1}}
}

% Article and similar classes version of command
% \newcommand{\TextWithNormalTextColor}[1]{%
%   {\color{jNormalTextsFGColor} {#1}}
% }



% Beamer version of command
\newcommand{\NormalTextInMathMode}[1]{%
  {\color{jNormalTextFGColor}
    \setbeamercolor{math text}{fg=jNormalTextFGColor} \text{#1}}
}


% Article and similar classes version of command
% \newcommand{\NormalTextInMathMode}[1]{%
%   {\color{jNormalTextsFGColor} \text{#1}}
% }




% Command that sets color of some mathematical text to the same color
% that has normal text in header (?)

% Beamer version of the command
\newcommand{\MathTextFrametitleFGColor}[1]{%
  {\color{jFrametitleFGColor}
    \setbeamercolor{math text}{fg=jFrametitleFGColor} #1}
}

% Article and similar classes version of the command
% \newcommand{\MathTextWhiteColor}[1]{{\color{jFrametitleFGColor} #1}}





% Command for setting color of alert text for some text in math modestyle

% Beamer version of the command
\newcommand{\MathTextAlertColor}[1]{%
  {\color{jOrange} \setbeamercolor{math text}{fg=jOrange} #1}
}

% Article and similar classes version of the command
% \newcommand{\MathTextAlertColor}[1]{{\color{jOrange} #1}}





% Command that allow you to sets chosen color as the color of some text into
% math mode. Due to some nuances in the way that Beamer handle colors
% it not work in all cases. We hope that in the future we will improve it.

% Beamer version of the command
\newcommand{\SetMathTextColor}[2]{%
  {\color{#1} \setbeamercolor{math text}{fg=#1} #2}
}


% Article and similar classes version of the command
% \newcommand{\SetMathTextColor}[2]{{\color{#1} #2}}










% ---------------------------------------
% Commands for few special slides
% ---------------------------------------
\newcommand{\EndingSlide}[1]{%
  \begin{frame}[standout]

    \begingroup

    \color{jFrametitleFGColor}

    #1

    \endgroup

  \end{frame}
}










% ---------------------------------------
% Commands for setting background pictures for some slides
% ---------------------------------------
\newcommand{\TitleBackgroundPicture}
{./JagiellonianPictures/Backgrounds/LajkonikDark.png}
\newcommand{\SectionBackgroundPicture}
{./JagiellonianPictures/Backgrounds/LajkonikLight.png}



\newcommand{\TitleSlideWithPicture}{%
  \begingroup

  \usebackgroundtemplate{%
    \includegraphics[height=\paperheight]{\TitleBackgroundPicture}}

  \maketitle

  \endgroup
}





\newcommand{\SectionSlideWithPicture}[1]{%
  \begingroup

  \usebackgroundtemplate{%
    \includegraphics[height=\paperheight]{\SectionBackgroundPicture}}

  \setbeamercolor{titlelike}{fg=normal text.fg}

  \section{#1}

  \endgroup
}










% ---------------------------------------
% Commands for lectures "Geometria 3D dla twórców gier wideo"
% Polish version
% ---------------------------------------
% Komendy teraz wykomentowane były potrzebne, gdy loga były na niebieskim
% tle, nie na białym. A są na białym bo tego chcieli w biurze projektu.
% \newcommand{\FundingLogoWhitePicturePL}
% {./PresentationPictures/CommonPictures/logotypFundusze_biale_bez_tla2.pdf}
\newcommand{\FundingLogoColorPicturePL}
{./PresentationPictures/CommonPictures/European_Funds_color_PL.pdf}
% \newcommand{\EULogoWhitePicturePL}
% {./PresentationPictures/CommonPictures/logotypUE_biale_bez_tla2.pdf}
\newcommand{\EUSocialFundLogoColorPicturePL}
{./PresentationPictures/CommonPictures/EU_Social_Fund_color_PL.pdf}
% \newcommand{\ZintegrUJLogoWhitePicturePL}
% {./PresentationPictures/CommonPictures/zintegruj-logo-white.pdf}
\newcommand{\ZintegrUJLogoColorPicturePL}
{./PresentationPictures/CommonPictures/ZintegrUJ_color.pdf}
\newcommand{\JULogoColorPicturePL}
{./JagiellonianPictures/LogoJU_PL/LogoJU_A_color.pdf}





\newcommand{\GeometryThreeDSpecialBeginningSlidePL}{%
  \begin{frame}[standout]

    \begin{textblock}{11}(1,0.7)

      \begin{flushleft}

        \mdseries

        \footnotesize

        \color{jFrametitleFGColor}

        Materiał powstał w ramach projektu współfinansowanego ze środków
        Unii Europejskiej w ramach Europejskiego Funduszu Społecznego
        POWR.03.05.00-00-Z309/17-00.

      \end{flushleft}

    \end{textblock}





    \begin{textblock}{10}(0,2.2)

      \tikz \fill[color=jBackgroundStyleLight] (0,0) rectangle (12.8,-1.5);

    \end{textblock}


    \begin{textblock}{3.2}(1,2.45)

      \includegraphics[scale=0.3]{\FundingLogoColorPicturePL}

    \end{textblock}


    \begin{textblock}{2.5}(3.7,2.5)

      \includegraphics[scale=0.2]{\JULogoColorPicturePL}

    \end{textblock}


    \begin{textblock}{2.5}(6,2.4)

      \includegraphics[scale=0.1]{\ZintegrUJLogoColorPicturePL}

    \end{textblock}


    \begin{textblock}{4.2}(8.4,2.6)

      \includegraphics[scale=0.3]{\EUSocialFundLogoColorPicturePL}

    \end{textblock}

  \end{frame}
}



\newcommand{\GeometryThreeDTwoSpecialBeginningSlidesPL}{%
  \begin{frame}[standout]

    \begin{textblock}{11}(1,0.7)

      \begin{flushleft}

        \mdseries

        \footnotesize

        \color{jFrametitleFGColor}

        Materiał powstał w ramach projektu współfinansowanego ze środków
        Unii Europejskiej w ramach Europejskiego Funduszu Społecznego
        POWR.03.05.00-00-Z309/17-00.

      \end{flushleft}

    \end{textblock}





    \begin{textblock}{10}(0,2.2)

      \tikz \fill[color=jBackgroundStyleLight] (0,0) rectangle (12.8,-1.5);

    \end{textblock}


    \begin{textblock}{3.2}(1,2.45)

      \includegraphics[scale=0.3]{\FundingLogoColorPicturePL}

    \end{textblock}


    \begin{textblock}{2.5}(3.7,2.5)

      \includegraphics[scale=0.2]{\JULogoColorPicturePL}

    \end{textblock}


    \begin{textblock}{2.5}(6,2.4)

      \includegraphics[scale=0.1]{\ZintegrUJLogoColorPicturePL}

    \end{textblock}


    \begin{textblock}{4.2}(8.4,2.6)

      \includegraphics[scale=0.3]{\EUSocialFundLogoColorPicturePL}

    \end{textblock}

  \end{frame}





  \TitleSlideWithPicture
}



\newcommand{\GeometryThreeDSpecialEndingSlidePL}{%
  \begin{frame}[standout]

    \begin{textblock}{11}(1,0.7)

      \begin{flushleft}

        \mdseries

        \footnotesize

        \color{jFrametitleFGColor}

        Materiał powstał w ramach projektu współfinansowanego ze środków
        Unii Europejskiej w~ramach Europejskiego Funduszu Społecznego
        POWR.03.05.00-00-Z309/17-00.

      \end{flushleft}

    \end{textblock}





    \begin{textblock}{10}(0,2.2)

      \tikz \fill[color=jBackgroundStyleLight] (0,0) rectangle (12.8,-1.5);

    \end{textblock}


    \begin{textblock}{3.2}(1,2.45)

      \includegraphics[scale=0.3]{\FundingLogoColorPicturePL}

    \end{textblock}


    \begin{textblock}{2.5}(3.7,2.5)

      \includegraphics[scale=0.2]{\JULogoColorPicturePL}

    \end{textblock}


    \begin{textblock}{2.5}(6,2.4)

      \includegraphics[scale=0.1]{\ZintegrUJLogoColorPicturePL}

    \end{textblock}


    \begin{textblock}{4.2}(8.4,2.6)

      \includegraphics[scale=0.3]{\EUSocialFundLogoColorPicturePL}

    \end{textblock}





    \begin{textblock}{11}(1,4)

      \begin{flushleft}

        \mdseries

        \footnotesize

        \RaggedRight

        \color{jFrametitleFGColor}

        Treść niniejszego wykładu jest udostępniona na~licencji
        Creative Commons (\textsc{cc}), z~uzna\-niem autorstwa
        (\textsc{by}) oraz udostępnianiem na tych samych warunkach
        (\textsc{sa}). Rysunki i~wy\-kresy zawarte w~wykładzie są
        autorstwa dr.~hab.~Pawła Węgrzyna et~al. i~są dostępne
        na tej samej licencji, o~ile nie wskazano inaczej.
        W~prezentacji wykorzystano temat Beamera Jagiellonian,
        oparty na~temacie Metropolis Matthiasa Vogelgesanga,
        dostępnym na licencji \LaTeX{} Project Public License~1.3c
        pod adresem: \colorhref{https://github.com/matze/mtheme}
        {https://github.com/matze/mtheme}.

        Projekt typograficzny: Iwona Grabska-Gradzińska \\
        Skład: Kamil Ziemian;
        Korekta: Wojciech Palacz \\
        Modele: Dariusz Frymus, Kamil Nowakowski \\
        Rysunki i~wykresy: Kamil Ziemian, Paweł Węgrzyn, Wojciech Palacz

      \end{flushleft}

    \end{textblock}

  \end{frame}
}



\newcommand{\GeometryThreeDTwoSpecialEndingSlidesPL}[1]{%
  \begin{frame}[standout]


    \begin{textblock}{11}(1,0.7)

      \begin{flushleft}

        \mdseries

        \footnotesize

        \color{jFrametitleFGColor}

        Materiał powstał w ramach projektu współfinansowanego ze środków
        Unii Europejskiej w~ramach Europejskiego Funduszu Społecznego
        POWR.03.05.00-00-Z309/17-00.

      \end{flushleft}

    \end{textblock}





    \begin{textblock}{10}(0,2.2)

      \tikz \fill[color=jBackgroundStyleLight] (0,0) rectangle (12.8,-1.5);

    \end{textblock}


    \begin{textblock}{3.2}(1,2.45)

      \includegraphics[scale=0.3]{\FundingLogoColorPicturePL}

    \end{textblock}


    \begin{textblock}{2.5}(3.7,2.5)

      \includegraphics[scale=0.2]{\JULogoColorPicturePL}

    \end{textblock}


    \begin{textblock}{2.5}(6,2.4)

      \includegraphics[scale=0.1]{\ZintegrUJLogoColorPicturePL}

    \end{textblock}


    \begin{textblock}{4.2}(8.4,2.6)

      \includegraphics[scale=0.3]{\EUSocialFundLogoColorPicturePL}

    \end{textblock}





    \begin{textblock}{11}(1,4)

      \begin{flushleft}

        \mdseries

        \footnotesize

        \RaggedRight

        \color{jFrametitleFGColor}

        Treść niniejszego wykładu jest udostępniona na~licencji
        Creative Commons (\textsc{cc}), z~uzna\-niem autorstwa
        (\textsc{by}) oraz udostępnianiem na tych samych warunkach
        (\textsc{sa}). Rysunki i~wy\-kresy zawarte w~wykładzie są
        autorstwa dr.~hab.~Pawła Węgrzyna et~al. i~są dostępne
        na tej samej licencji, o~ile nie wskazano inaczej.
        W~prezentacji wykorzystano temat Beamera Jagiellonian,
        oparty na~temacie Metropolis Matthiasa Vogelgesanga,
        dostępnym na licencji \LaTeX{} Project Public License~1.3c
        pod adresem: \colorhref{https://github.com/matze/mtheme}
        {https://github.com/matze/mtheme}.

        Projekt typograficzny: Iwona Grabska-Gradzińska \\
        Skład: Kamil Ziemian;
        Korekta: Wojciech Palacz \\
        Modele: Dariusz Frymus, Kamil Nowakowski \\
        Rysunki i~wykresy: Kamil Ziemian, Paweł Węgrzyn, Wojciech Palacz

      \end{flushleft}

    \end{textblock}

  \end{frame}





  \begin{frame}[standout]

    \begingroup

    \color{jFrametitleFGColor}

    #1

    \endgroup

  \end{frame}
}



\newcommand{\GeometryThreeDSpecialEndingSlideVideoPL}{%
  \begin{frame}[standout]

    \begin{textblock}{11}(1,0.7)

      \begin{flushleft}

        \mdseries

        \footnotesize

        \color{jFrametitleFGColor}

        Materiał powstał w ramach projektu współfinansowanego ze środków
        Unii Europejskiej w~ramach Europejskiego Funduszu Społecznego
        POWR.03.05.00-00-Z309/17-00.

      \end{flushleft}

    \end{textblock}





    \begin{textblock}{10}(0,2.2)

      \tikz \fill[color=jBackgroundStyleLight] (0,0) rectangle (12.8,-1.5);

    \end{textblock}


    \begin{textblock}{3.2}(1,2.45)

      \includegraphics[scale=0.3]{\FundingLogoColorPicturePL}

    \end{textblock}


    \begin{textblock}{2.5}(3.7,2.5)

      \includegraphics[scale=0.2]{\JULogoColorPicturePL}

    \end{textblock}


    \begin{textblock}{2.5}(6,2.4)

      \includegraphics[scale=0.1]{\ZintegrUJLogoColorPicturePL}

    \end{textblock}


    \begin{textblock}{4.2}(8.4,2.6)

      \includegraphics[scale=0.3]{\EUSocialFundLogoColorPicturePL}

    \end{textblock}





    \begin{textblock}{11}(1,4)

      \begin{flushleft}

        \mdseries

        \footnotesize

        \RaggedRight

        \color{jFrametitleFGColor}

        Treść niniejszego wykładu jest udostępniona na~licencji
        Creative Commons (\textsc{cc}), z~uzna\-niem autorstwa
        (\textsc{by}) oraz udostępnianiem na tych samych warunkach
        (\textsc{sa}). Rysunki i~wy\-kresy zawarte w~wykładzie są
        autorstwa dr.~hab.~Pawła Węgrzyna et~al. i~są dostępne
        na tej samej licencji, o~ile nie wskazano inaczej.
        W~prezentacji wykorzystano temat Beamera Jagiellonian,
        oparty na~temacie Metropolis Matthiasa Vogelgesanga,
        dostępnym na licencji \LaTeX{} Project Public License~1.3c
        pod adresem: \colorhref{https://github.com/matze/mtheme}
        {https://github.com/matze/mtheme}.

        Projekt typograficzny: Iwona Grabska-Gradzińska;
        Skład: Kamil Ziemian \\
        Korekta: Wojciech Palacz;
        Modele: Dariusz Frymus, Kamil Nowakowski \\
        Rysunki i~wykresy: Kamil Ziemian, Paweł Węgrzyn, Wojciech Palacz \\
        Montaż: Agencja Filmowa Film \& Television Production~-- Zbigniew
        Masklak

      \end{flushleft}

    \end{textblock}

  \end{frame}
}





\newcommand{\GeometryThreeDTwoSpecialEndingSlidesVideoPL}[1]{%
  \begin{frame}[standout]

    \begin{textblock}{11}(1,0.7)

      \begin{flushleft}

        \mdseries

        \footnotesize

        \color{jFrametitleFGColor}

        Materiał powstał w ramach projektu współfinansowanego ze środków
        Unii Europejskiej w~ramach Europejskiego Funduszu Społecznego
        POWR.03.05.00-00-Z309/17-00.

      \end{flushleft}

    \end{textblock}





    \begin{textblock}{10}(0,2.2)

      \tikz \fill[color=jBackgroundStyleLight] (0,0) rectangle (12.8,-1.5);

    \end{textblock}


    \begin{textblock}{3.2}(1,2.45)

      \includegraphics[scale=0.3]{\FundingLogoColorPicturePL}

    \end{textblock}


    \begin{textblock}{2.5}(3.7,2.5)

      \includegraphics[scale=0.2]{\JULogoColorPicturePL}

    \end{textblock}


    \begin{textblock}{2.5}(6,2.4)

      \includegraphics[scale=0.1]{\ZintegrUJLogoColorPicturePL}

    \end{textblock}


    \begin{textblock}{4.2}(8.4,2.6)

      \includegraphics[scale=0.3]{\EUSocialFundLogoColorPicturePL}

    \end{textblock}





    \begin{textblock}{11}(1,4)

      \begin{flushleft}

        \mdseries

        \footnotesize

        \RaggedRight

        \color{jFrametitleFGColor}

        Treść niniejszego wykładu jest udostępniona na~licencji
        Creative Commons (\textsc{cc}), z~uzna\-niem autorstwa
        (\textsc{by}) oraz udostępnianiem na tych samych warunkach
        (\textsc{sa}). Rysunki i~wy\-kresy zawarte w~wykładzie są
        autorstwa dr.~hab.~Pawła Węgrzyna et~al. i~są dostępne
        na tej samej licencji, o~ile nie wskazano inaczej.
        W~prezentacji wykorzystano temat Beamera Jagiellonian,
        oparty na~temacie Metropolis Matthiasa Vogelgesanga,
        dostępnym na licencji \LaTeX{} Project Public License~1.3c
        pod adresem: \colorhref{https://github.com/matze/mtheme}
        {https://github.com/matze/mtheme}.

        Projekt typograficzny: Iwona Grabska-Gradzińska;
        Skład: Kamil Ziemian \\
        Korekta: Wojciech Palacz;
        Modele: Dariusz Frymus, Kamil Nowakowski \\
        Rysunki i~wykresy: Kamil Ziemian, Paweł Węgrzyn, Wojciech Palacz \\
        Montaż: Agencja Filmowa Film \& Television Production~-- Zbigniew
        Masklak

      \end{flushleft}

    \end{textblock}

  \end{frame}





  \begin{frame}[standout]


    \begingroup

    \color{jFrametitleFGColor}

    #1

    \endgroup

  \end{frame}
}










% ---------------------------------------
% Commands for lectures "Geometria 3D dla twórców gier wideo"
% English version
% ---------------------------------------
% \newcommand{\FundingLogoWhitePictureEN}
% {./PresentationPictures/CommonPictures/logotypFundusze_biale_bez_tla2.pdf}
\newcommand{\FundingLogoColorPictureEN}
{./PresentationPictures/CommonPictures/European_Funds_color_EN.pdf}
% \newcommand{\EULogoWhitePictureEN}
% {./PresentationPictures/CommonPictures/logotypUE_biale_bez_tla2.pdf}
\newcommand{\EUSocialFundLogoColorPictureEN}
{./PresentationPictures/CommonPictures/EU_Social_Fund_color_EN.pdf}
% \newcommand{\ZintegrUJLogoWhitePictureEN}
% {./PresentationPictures/CommonPictures/zintegruj-logo-white.pdf}
\newcommand{\ZintegrUJLogoColorPictureEN}
{./PresentationPictures/CommonPictures/ZintegrUJ_color.pdf}
\newcommand{\JULogoColorPictureEN}
{./JagiellonianPictures/LogoJU_EN/LogoJU_A_color.pdf}



\newcommand{\GeometryThreeDSpecialBeginningSlideEN}{%
  \begin{frame}[standout]

    \begin{textblock}{11}(1,0.7)

      \begin{flushleft}

        \mdseries

        \footnotesize

        \color{jFrametitleFGColor}

        This content was created as part of a project co-financed by the
        European Union within the framework of the European Social Fund
        POWR.03.05.00-00-Z309/17-00.

      \end{flushleft}

    \end{textblock}





    \begin{textblock}{10}(0,2.2)

      \tikz \fill[color=jBackgroundStyleLight] (0,0) rectangle (12.8,-1.5);

    \end{textblock}


    \begin{textblock}{3.2}(0.7,2.45)

      \includegraphics[scale=0.3]{\FundingLogoColorPictureEN}

    \end{textblock}


    \begin{textblock}{2.5}(4.15,2.5)

      \includegraphics[scale=0.2]{\JULogoColorPictureEN}

    \end{textblock}


    \begin{textblock}{2.5}(6.35,2.4)

      \includegraphics[scale=0.1]{\ZintegrUJLogoColorPictureEN}

    \end{textblock}


    \begin{textblock}{4.2}(8.4,2.6)

      \includegraphics[scale=0.3]{\EUSocialFundLogoColorPictureEN}

    \end{textblock}

  \end{frame}
}



\newcommand{\GeometryThreeDTwoSpecialBeginningSlidesEN}{%
  \begin{frame}[standout]

    \begin{textblock}{11}(1,0.7)

      \begin{flushleft}

        \mdseries

        \footnotesize

        \color{jFrametitleFGColor}

        This content was created as part of a project co-financed by the
        European Union within the framework of the European Social Fund
        POWR.03.05.00-00-Z309/17-00.

      \end{flushleft}

    \end{textblock}





    \begin{textblock}{10}(0,2.2)

      \tikz \fill[color=jBackgroundStyleLight] (0,0) rectangle (12.8,-1.5);

    \end{textblock}


    \begin{textblock}{3.2}(0.7,2.45)

      \includegraphics[scale=0.3]{\FundingLogoColorPictureEN}

    \end{textblock}


    \begin{textblock}{2.5}(4.15,2.5)

      \includegraphics[scale=0.2]{\JULogoColorPictureEN}

    \end{textblock}


    \begin{textblock}{2.5}(6.35,2.4)

      \includegraphics[scale=0.1]{\ZintegrUJLogoColorPictureEN}

    \end{textblock}


    \begin{textblock}{4.2}(8.4,2.6)

      \includegraphics[scale=0.3]{\EUSocialFundLogoColorPictureEN}

    \end{textblock}

  \end{frame}





  \TitleSlideWithPicture
}



\newcommand{\GeometryThreeDSpecialEndingSlideEN}{%
  \begin{frame}[standout]

    \begin{textblock}{11}(1,0.7)

      \begin{flushleft}

        \mdseries

        \footnotesize

        \color{jFrametitleFGColor}

        This content was created as part of a project co-financed by the
        European Union within the framework of the European Social Fund
        POWR.03.05.00-00-Z309/17-00.

      \end{flushleft}

    \end{textblock}





    \begin{textblock}{10}(0,2.2)

      \tikz \fill[color=jBackgroundStyleLight] (0,0) rectangle (12.8,-1.5);

    \end{textblock}


    \begin{textblock}{3.2}(0.7,2.45)

      \includegraphics[scale=0.3]{\FundingLogoColorPictureEN}

    \end{textblock}


    \begin{textblock}{2.5}(4.15,2.5)

      \includegraphics[scale=0.2]{\JULogoColorPictureEN}

    \end{textblock}


    \begin{textblock}{2.5}(6.35,2.4)

      \includegraphics[scale=0.1]{\ZintegrUJLogoColorPictureEN}

    \end{textblock}


    \begin{textblock}{4.2}(8.4,2.6)

      \includegraphics[scale=0.3]{\EUSocialFundLogoColorPictureEN}

    \end{textblock}





    \begin{textblock}{11}(1,4)

      \begin{flushleft}

        \mdseries

        \footnotesize

        \RaggedRight

        \color{jFrametitleFGColor}

        The content of this lecture is made available under a~Creative
        Commons licence (\textsc{cc}), giving the author the credits
        (\textsc{by}) and putting an obligation to share on the same terms
        (\textsc{sa}). Figures and diagrams included in the lecture are
        authored by Paweł Węgrzyn et~al., and are available under the same
        license unless indicated otherwise.\\ The presentation uses the
        Beamer Jagiellonian theme based on Matthias Vogelgesang’s
        Metropolis theme, available under license \LaTeX{} Project
        Public License~1.3c at: \colorhref{https://github.com/matze/mtheme}
        {https://github.com/matze/mtheme}.

        Typographic design: Iwona Grabska-Gradzińska \\
        \LaTeX{} Typesetting: Kamil Ziemian \\
        Proofreading: Wojciech Palacz,
        Monika Stawicka \\
        3D Models: Dariusz Frymus, Kamil Nowakowski \\
        Figures and charts: Kamil Ziemian, Paweł Węgrzyn, Wojciech Palacz

      \end{flushleft}

    \end{textblock}

  \end{frame}
}



\newcommand{\GeometryThreeDTwoSpecialEndingSlidesEN}[1]{%
  \begin{frame}[standout]


    \begin{textblock}{11}(1,0.7)

      \begin{flushleft}

        \mdseries

        \footnotesize

        \color{jFrametitleFGColor}

        This content was created as part of a project co-financed by the
        European Union within the framework of the European Social Fund
        POWR.03.05.00-00-Z309/17-00.

      \end{flushleft}

    \end{textblock}





    \begin{textblock}{10}(0,2.2)

      \tikz \fill[color=jBackgroundStyleLight] (0,0) rectangle (12.8,-1.5);

    \end{textblock}


    \begin{textblock}{3.2}(0.7,2.45)

      \includegraphics[scale=0.3]{\FundingLogoColorPictureEN}

    \end{textblock}


    \begin{textblock}{2.5}(4.15,2.5)

      \includegraphics[scale=0.2]{\JULogoColorPictureEN}

    \end{textblock}


    \begin{textblock}{2.5}(6.35,2.4)

      \includegraphics[scale=0.1]{\ZintegrUJLogoColorPictureEN}

    \end{textblock}


    \begin{textblock}{4.2}(8.4,2.6)

      \includegraphics[scale=0.3]{\EUSocialFundLogoColorPictureEN}

    \end{textblock}





    \begin{textblock}{11}(1,4)

      \begin{flushleft}

        \mdseries

        \footnotesize

        \RaggedRight

        \color{jFrametitleFGColor}

        The content of this lecture is made available under a~Creative
        Commons licence (\textsc{cc}), giving the author the credits
        (\textsc{by}) and putting an obligation to share on the same terms
        (\textsc{sa}). Figures and diagrams included in the lecture are
        authored by Paweł Węgrzyn et~al., and are available under the same
        license unless indicated otherwise.\\ The presentation uses the
        Beamer Jagiellonian theme based on Matthias Vogelgesang’s
        Metropolis theme, available under license \LaTeX{} Project
        Public License~1.3c at: \colorhref{https://github.com/matze/mtheme}
        {https://github.com/matze/mtheme}.

        Typographic design: Iwona Grabska-Gradzińska \\
        \LaTeX{} Typesetting: Kamil Ziemian \\
        Proofreading: Wojciech Palacz,
        Monika Stawicka \\
        3D Models: Dariusz Frymus, Kamil Nowakowski \\
        Figures and charts: Kamil Ziemian, Paweł Węgrzyn, Wojciech Palacz

      \end{flushleft}

    \end{textblock}

  \end{frame}





  \begin{frame}[standout]

    \begingroup

    \color{jFrametitleFGColor}

    #1

    \endgroup

  \end{frame}
}



\newcommand{\GeometryThreeDSpecialEndingSlideVideoVerOneEN}{%
  \begin{frame}[standout]

    \begin{textblock}{11}(1,0.7)

      \begin{flushleft}

        \mdseries

        \footnotesize

        \color{jFrametitleFGColor}

        This content was created as part of a project co-financed by the
        European Union within the framework of the European Social Fund
        POWR.03.05.00-00-Z309/17-00.

      \end{flushleft}

    \end{textblock}





    \begin{textblock}{10}(0,2.2)

      \tikz \fill[color=jBackgroundStyleLight] (0,0) rectangle (12.8,-1.5);

    \end{textblock}


    \begin{textblock}{3.2}(0.7,2.45)

      \includegraphics[scale=0.3]{\FundingLogoColorPictureEN}

    \end{textblock}


    \begin{textblock}{2.5}(4.15,2.5)

      \includegraphics[scale=0.2]{\JULogoColorPictureEN}

    \end{textblock}


    \begin{textblock}{2.5}(6.35,2.4)

      \includegraphics[scale=0.1]{\ZintegrUJLogoColorPictureEN}

    \end{textblock}


    \begin{textblock}{4.2}(8.4,2.6)

      \includegraphics[scale=0.3]{\EUSocialFundLogoColorPictureEN}

    \end{textblock}





    \begin{textblock}{11}(1,4)

      \begin{flushleft}

        \mdseries

        \footnotesize

        \RaggedRight

        \color{jFrametitleFGColor}

        The content of this lecture is made available under a Creative
        Commons licence (\textsc{cc}), giving the author the credits
        (\textsc{by}) and putting an obligation to share on the same terms
        (\textsc{sa}). Figures and diagrams included in the lecture are
        authored by Paweł Węgrzyn et~al., and are available under the same
        license unless indicated otherwise.\\ The presentation uses the
        Beamer Jagiellonian theme based on Matthias Vogelgesang’s
        Metropolis theme, available under license \LaTeX{} Project
        Public License~1.3c at: \colorhref{https://github.com/matze/mtheme}
        {https://github.com/matze/mtheme}.

        Typographic design: Iwona Grabska-Gradzińska;
        \LaTeX{} Typesetting: Kamil Ziemian \\
        Proofreading: Wojciech Palacz,
        Monika Stawicka \\
        3D Models: Dariusz Frymus, Kamil Nowakowski \\
        Figures and charts: Kamil Ziemian, Paweł Węgrzyn, Wojciech
        Palacz \\
        Film editing: Agencja Filmowa Film \& Television Production~--
        Zbigniew Masklak

      \end{flushleft}

    \end{textblock}

  \end{frame}
}



\newcommand{\GeometryThreeDSpecialEndingSlideVideoVerTwoEN}{%
  \begin{frame}[standout]

    \begin{textblock}{11}(1,0.7)

      \begin{flushleft}

        \mdseries

        \footnotesize

        \color{jFrametitleFGColor}

        This content was created as part of a project co-financed by the
        European Union within the framework of the European Social Fund
        POWR.03.05.00-00-Z309/17-00.

      \end{flushleft}

    \end{textblock}





    \begin{textblock}{10}(0,2.2)

      \tikz \fill[color=jBackgroundStyleLight] (0,0) rectangle (12.8,-1.5);

    \end{textblock}


    \begin{textblock}{3.2}(0.7,2.45)

      \includegraphics[scale=0.3]{\FundingLogoColorPictureEN}

    \end{textblock}


    \begin{textblock}{2.5}(4.15,2.5)

      \includegraphics[scale=0.2]{\JULogoColorPictureEN}

    \end{textblock}


    \begin{textblock}{2.5}(6.35,2.4)

      \includegraphics[scale=0.1]{\ZintegrUJLogoColorPictureEN}

    \end{textblock}


    \begin{textblock}{4.2}(8.4,2.6)

      \includegraphics[scale=0.3]{\EUSocialFundLogoColorPictureEN}

    \end{textblock}





    \begin{textblock}{11}(1,4)

      \begin{flushleft}

        \mdseries

        \footnotesize

        \RaggedRight

        \color{jFrametitleFGColor}

        The content of this lecture is made available under a Creative
        Commons licence (\textsc{cc}), giving the author the credits
        (\textsc{by}) and putting an obligation to share on the same terms
        (\textsc{sa}). Figures and diagrams included in the lecture are
        authored by Paweł Węgrzyn et~al., and are available under the same
        license unless indicated otherwise.\\ The presentation uses the
        Beamer Jagiellonian theme based on Matthias Vogelgesang’s
        Metropolis theme, available under license \LaTeX{} Project
        Public License~1.3c at: \colorhref{https://github.com/matze/mtheme}
        {https://github.com/matze/mtheme}.

        Typographic design: Iwona Grabska-Gradzińska;
        \LaTeX{} Typesetting: Kamil Ziemian \\
        Proofreading: Wojciech Palacz,
        Monika Stawicka \\
        3D Models: Dariusz Frymus, Kamil Nowakowski \\
        Figures and charts: Kamil Ziemian, Paweł Węgrzyn, Wojciech
        Palacz \\
        Film editing: IMAVI -- Joanna Kozakiewicz, Krzysztof Magda, Nikodem
        Frodyma

      \end{flushleft}

    \end{textblock}

  \end{frame}
}



\newcommand{\GeometryThreeDSpecialEndingSlideVideoVerThreeEN}{%
  \begin{frame}[standout]

    \begin{textblock}{11}(1,0.7)

      \begin{flushleft}

        \mdseries

        \footnotesize

        \color{jFrametitleFGColor}

        This content was created as part of a project co-financed by the
        European Union within the framework of the European Social Fund
        POWR.03.05.00-00-Z309/17-00.

      \end{flushleft}

    \end{textblock}





    \begin{textblock}{10}(0,2.2)

      \tikz \fill[color=jBackgroundStyleLight] (0,0) rectangle (12.8,-1.5);

    \end{textblock}


    \begin{textblock}{3.2}(0.7,2.45)

      \includegraphics[scale=0.3]{\FundingLogoColorPictureEN}

    \end{textblock}


    \begin{textblock}{2.5}(4.15,2.5)

      \includegraphics[scale=0.2]{\JULogoColorPictureEN}

    \end{textblock}


    \begin{textblock}{2.5}(6.35,2.4)

      \includegraphics[scale=0.1]{\ZintegrUJLogoColorPictureEN}

    \end{textblock}


    \begin{textblock}{4.2}(8.4,2.6)

      \includegraphics[scale=0.3]{\EUSocialFundLogoColorPictureEN}

    \end{textblock}





    \begin{textblock}{11}(1,4)

      \begin{flushleft}

        \mdseries

        \footnotesize

        \RaggedRight

        \color{jFrametitleFGColor}

        The content of this lecture is made available under a Creative
        Commons licence (\textsc{cc}), giving the author the credits
        (\textsc{by}) and putting an obligation to share on the same terms
        (\textsc{sa}). Figures and diagrams included in the lecture are
        authored by Paweł Węgrzyn et~al., and are available under the same
        license unless indicated otherwise.\\ The presentation uses the
        Beamer Jagiellonian theme based on Matthias Vogelgesang’s
        Metropolis theme, available under license \LaTeX{} Project
        Public License~1.3c at: \colorhref{https://github.com/matze/mtheme}
        {https://github.com/matze/mtheme}.

        Typographic design: Iwona Grabska-Gradzińska;
        \LaTeX{} Typesetting: Kamil Ziemian \\
        Proofreading: Wojciech Palacz,
        Monika Stawicka \\
        3D Models: Dariusz Frymus, Kamil Nowakowski \\
        Figures and charts: Kamil Ziemian, Paweł Węgrzyn, Wojciech
        Palacz \\
        Film editing: Agencja Filmowa Film \& Television Production~--
        Zbigniew Masklak \\
        Film editing: IMAVI -- Joanna Kozakiewicz, Krzysztof Magda, Nikodem
        Frodyma

      \end{flushleft}

    \end{textblock}

  \end{frame}
}



\newcommand{\GeometryThreeDTwoSpecialEndingSlidesVideoVerOneEN}[1]{%
  \begin{frame}[standout]

    \begin{textblock}{11}(1,0.7)

      \begin{flushleft}

        \mdseries

        \footnotesize

        \color{jFrametitleFGColor}

        This content was created as part of a project co-financed by the
        European Union within the framework of the European Social Fund
        POWR.03.05.00-00-Z309/17-00.

      \end{flushleft}

    \end{textblock}





    \begin{textblock}{10}(0,2.2)

      \tikz \fill[color=jBackgroundStyleLight] (0,0) rectangle (12.8,-1.5);

    \end{textblock}


    \begin{textblock}{3.2}(0.7,2.45)

      \includegraphics[scale=0.3]{\FundingLogoColorPictureEN}

    \end{textblock}


    \begin{textblock}{2.5}(4.15,2.5)

      \includegraphics[scale=0.2]{\JULogoColorPictureEN}

    \end{textblock}


    \begin{textblock}{2.5}(6.35,2.4)

      \includegraphics[scale=0.1]{\ZintegrUJLogoColorPictureEN}

    \end{textblock}


    \begin{textblock}{4.2}(8.4,2.6)

      \includegraphics[scale=0.3]{\EUSocialFundLogoColorPictureEN}

    \end{textblock}





    \begin{textblock}{11}(1,4)

      \begin{flushleft}

        \mdseries

        \footnotesize

        \RaggedRight

        \color{jFrametitleFGColor}

        The content of this lecture is made available under a Creative
        Commons licence (\textsc{cc}), giving the author the credits
        (\textsc{by}) and putting an obligation to share on the same terms
        (\textsc{sa}). Figures and diagrams included in the lecture are
        authored by Paweł Węgrzyn et~al., and are available under the same
        license unless indicated otherwise.\\ The presentation uses the
        Beamer Jagiellonian theme based on Matthias Vogelgesang’s
        Metropolis theme, available under license \LaTeX{} Project
        Public License~1.3c at: \colorhref{https://github.com/matze/mtheme}
        {https://github.com/matze/mtheme}.

        Typographic design: Iwona Grabska-Gradzińska;
        \LaTeX{} Typesetting: Kamil Ziemian \\
        Proofreading: Wojciech Palacz,
        Monika Stawicka \\
        3D Models: Dariusz Frymus, Kamil Nowakowski \\
        Figures and charts: Kamil Ziemian, Paweł Węgrzyn,
        Wojciech Palacz \\
        Film editing: Agencja Filmowa Film \& Television Production~--
        Zbigniew Masklak

      \end{flushleft}

    \end{textblock}

  \end{frame}





  \begin{frame}[standout]


    \begingroup

    \color{jFrametitleFGColor}

    #1

    \endgroup

  \end{frame}
}



\newcommand{\GeometryThreeDTwoSpecialEndingSlidesVideoVerTwoEN}[1]{%
  \begin{frame}[standout]

    \begin{textblock}{11}(1,0.7)

      \begin{flushleft}

        \mdseries

        \footnotesize

        \color{jFrametitleFGColor}

        This content was created as part of a project co-financed by the
        European Union within the framework of the European Social Fund
        POWR.03.05.00-00-Z309/17-00.

      \end{flushleft}

    \end{textblock}





    \begin{textblock}{10}(0,2.2)

      \tikz \fill[color=jBackgroundStyleLight] (0,0) rectangle (12.8,-1.5);

    \end{textblock}


    \begin{textblock}{3.2}(0.7,2.45)

      \includegraphics[scale=0.3]{\FundingLogoColorPictureEN}

    \end{textblock}


    \begin{textblock}{2.5}(4.15,2.5)

      \includegraphics[scale=0.2]{\JULogoColorPictureEN}

    \end{textblock}


    \begin{textblock}{2.5}(6.35,2.4)

      \includegraphics[scale=0.1]{\ZintegrUJLogoColorPictureEN}

    \end{textblock}


    \begin{textblock}{4.2}(8.4,2.6)

      \includegraphics[scale=0.3]{\EUSocialFundLogoColorPictureEN}

    \end{textblock}





    \begin{textblock}{11}(1,4)

      \begin{flushleft}

        \mdseries

        \footnotesize

        \RaggedRight

        \color{jFrametitleFGColor}

        The content of this lecture is made available under a Creative
        Commons licence (\textsc{cc}), giving the author the credits
        (\textsc{by}) and putting an obligation to share on the same terms
        (\textsc{sa}). Figures and diagrams included in the lecture are
        authored by Paweł Węgrzyn et~al., and are available under the same
        license unless indicated otherwise.\\ The presentation uses the
        Beamer Jagiellonian theme based on Matthias Vogelgesang’s
        Metropolis theme, available under license \LaTeX{} Project
        Public License~1.3c at: \colorhref{https://github.com/matze/mtheme}
        {https://github.com/matze/mtheme}.

        Typographic design: Iwona Grabska-Gradzińska;
        \LaTeX{} Typesetting: Kamil Ziemian \\
        Proofreading: Wojciech Palacz,
        Monika Stawicka \\
        3D Models: Dariusz Frymus, Kamil Nowakowski \\
        Figures and charts: Kamil Ziemian, Paweł Węgrzyn,
        Wojciech Palacz \\
        Film editing: IMAVI -- Joanna Kozakiewicz, Krzysztof Magda, Nikodem
        Frodyma

      \end{flushleft}

    \end{textblock}

  \end{frame}





  \begin{frame}[standout]


    \begingroup

    \color{jFrametitleFGColor}

    #1

    \endgroup

  \end{frame}
}



\newcommand{\GeometryThreeDTwoSpecialEndingSlidesVideoVerThreeEN}[1]{%
  \begin{frame}[standout]

    \begin{textblock}{11}(1,0.7)

      \begin{flushleft}

        \mdseries

        \footnotesize

        \color{jFrametitleFGColor}

        This content was created as part of a project co-financed by the
        European Union within the framework of the European Social Fund
        POWR.03.05.00-00-Z309/17-00.

      \end{flushleft}

    \end{textblock}





    \begin{textblock}{10}(0,2.2)

      \tikz \fill[color=jBackgroundStyleLight] (0,0) rectangle (12.8,-1.5);

    \end{textblock}


    \begin{textblock}{3.2}(0.7,2.45)

      \includegraphics[scale=0.3]{\FundingLogoColorPictureEN}

    \end{textblock}


    \begin{textblock}{2.5}(4.15,2.5)

      \includegraphics[scale=0.2]{\JULogoColorPictureEN}

    \end{textblock}


    \begin{textblock}{2.5}(6.35,2.4)

      \includegraphics[scale=0.1]{\ZintegrUJLogoColorPictureEN}

    \end{textblock}


    \begin{textblock}{4.2}(8.4,2.6)

      \includegraphics[scale=0.3]{\EUSocialFundLogoColorPictureEN}

    \end{textblock}





    \begin{textblock}{11}(1,4)

      \begin{flushleft}

        \mdseries

        \footnotesize

        \RaggedRight

        \color{jFrametitleFGColor}

        The content of this lecture is made available under a Creative
        Commons licence (\textsc{cc}), giving the author the credits
        (\textsc{by}) and putting an obligation to share on the same terms
        (\textsc{sa}). Figures and diagrams included in the lecture are
        authored by Paweł Węgrzyn et~al., and are available under the same
        license unless indicated otherwise. \\ The presentation uses the
        Beamer Jagiellonian theme based on Matthias Vogelgesang’s
        Metropolis theme, available under license \LaTeX{} Project
        Public License~1.3c at: \colorhref{https://github.com/matze/mtheme}
        {https://github.com/matze/mtheme}.

        Typographic design: Iwona Grabska-Gradzińska;
        \LaTeX{} Typesetting: Kamil Ziemian \\
        Proofreading: Leszek Hadasz, Wojciech Palacz,
        Monika Stawicka \\
        3D Models: Dariusz Frymus, Kamil Nowakowski \\
        Figures and charts: Kamil Ziemian, Paweł Węgrzyn,
        Wojciech Palacz \\
        Film editing: Agencja Filmowa Film \& Television Production~--
        Zbigniew Masklak \\
        Film editing: IMAVI -- Joanna Kozakiewicz, Krzysztof Magda, Nikodem
        Frodyma


      \end{flushleft}

    \end{textblock}

  \end{frame}





  \begin{frame}[standout]


    \begingroup

    \color{jFrametitleFGColor}

    #1

    \endgroup

  \end{frame}
}










% ------------------------------------------------------
% BibLaTeX
% ------------------------------------------------------
% Package biblatex, with biber as its backend, allow us to handle
% bibliography entries that use Unicode symbols outside ASCII
\usepackage[
language=polish,
backend=biber,
style=alphabetic,
url=false,
eprint=true,
]{biblatex}

\addbibresource{PomocniczePrezentacjeBibliography.bib}





% ------------------------------------------------------
% Importing packages, libraries and setting their configuration
% ------------------------------------------------------





% ------------------------------------------------------
% Special configuration for this particular presentation
% ------------------------------------------------------










% ------------------------------------------------------------------------------------------------------------------
\title{Algebry Boola}
\subtitle{}

\author{Kamil Ziemian}


\date{}
% ------------------------------------------------------------------------------------------------------------------










% ####################################################################
% Beginning of the document
\begin{document}
% ####################################################################





% Text is adjusted to the left and words are broken at the end of the line.
% Number of chars: 62k+, 13k+,
\RaggedRight





% ######################################
\maketitle
% ######################################





% ##################
\begin{frame}
  \frametitle{Spis treści}


  \tableofcontents

\end{frame}
% ##################










% ######################################
\section{Ogólne informacje}
% ######################################



% ##################
\begin{frame}
  \frametitle{Uwaga}


  Poniższa prezentacja o~algebrach Boole’a jest przeznaczona dla
  informatyków, więc co~wpływa na~wybór i~sposób prezentacji materiału.
  Proszę mieć to na uwagę w~trakcie jej lektury. Staraliśmy~się również,
  by został on możliwie prosto wyłożony, stąd liczne uproszczenia
  i~pominięcie wielu subtelności jakie ten temat sobą prezentuje.

\end{frame}
% ##################





% ##################
\begin{frame}
  \frametitle{Kim był Boole?}


  \begin{figure}

    \centering

    \includegraphics[scale=0.195]
    {./PresentationsPictures/George-Boole.jpeg}

    \caption{George Boole (1815--1864), brytyjski matematyk i~filozof.
      Wprowadził metody algebraiczne do logiki. \textbf{Algebra Boole’a}
    została nazwana na jego cześć.}

    \label{fig:George-Boole}

  \end{figure}

\end{frame}
% ##################





% ##################
\begin{frame}
  \frametitle{Algebry Boole’a w~informatyce}


  Algebra Boole’a może już być Państwu znana pod nazwą „logiki”,
  „logiki matematycznej”, „rachunku zdań”, etc. Najbardziej poprawną
  nazwą w~tym kontekście jest \textbf{klasyczny rachunku zdań}, jednak
  informacje o~nim ograniczymy do niezbędnego minimum. Osoby zainteresowany
  tym tematem mogą sięgnąć do następujących pozycji.
  \begin{itemize}

  \item Andrzej Grzegorczyk \textit{Zarys logiki matematycznej}
    \parencite{Grzegorczyk-Zarys-logiki-matematycznej-Pub-1975}.



  \item Willard von Ormian Quine \textit{Logika matematyczna}
    \parencite{Quine-Logika-matematyczna-Pub-1974}.



  \item Zdzisław Kraszewski \textit{Logika. Nauka rozumowania}
    \parencite{Kraszewski-Logika-Pub-1984}.

  \end{itemize}

\end{frame}
% ##################





% ##################
\begin{frame}
  \frametitle{Algebry Boole’a w~informatyce}


  Do informatyki algebry Boole’a wprowadził je zapewne Claude E.~Shannon
  (1916--2001), gdy zrozumiał jakie użyteczne w~projektowaniu układów
  elektronicznych mogą być idea Boole’a, które poznał na kursie
  z~filozofii. Obecnie są one powszechnie stosowane tak w~teorii jak
  i~praktyce informatyki, że~każdy dobry informatyk musi~się umieć nimi
  posługiwać.

\end{frame}
% ##################










% ######################################
\section{Podstawy algebry Boole’a}
% ######################################



% ##################
\begin{frame}
  \frametitle{Logiczna prawda i~fałsz}


  Dla algebry Boole’a podstawowe są dwie wielkości oznaczające odpowiednio
  wartość logiczną prawdy i~fałszu. Prawdę logiczną będziemy oznaczać
  symbolem \texttt{true}, zaś~fałsz symbolem \texttt{false}. Nazywamy
  jej \textbf{wartościami boolowskimi} (ang. \textit{boolean values})
  lub \textbf{wartościami logicznymi} (ang. \textit{logical values}).

  W~praktyce istnieje wiele różnych sposobów reprezentowania tych wielkości
  w~komputerze. W~języku Go są one oznaczane przez \texttt{true}
  i~\texttt{false}. W~języku Python przez \texttt{True} i~\texttt{False}
  (proszę zwrócić uwagę na wielkość liter). W~języku C fałsz reprezentuj
  liczba całkowita równa 0, prawdą jest każda inna liczba całkowita.
  Domyślnie, choć jest to tylko umowa między użytkownikami języka~C, jako
  prawdę zwracamy wartość~1. Można również korzystać z~protezy typu
  boolowskiego jaką daje nam \texttt{stdbool.h}. W~języku C++ sytuacja jest
  w~zasadzie taka sama jak w~języku~C++.

\end{frame}
% ##################





% ##################
\begin{frame}
  \frametitle{Logiczna prawda i~fałsz}


  W~tej prezentacji będziemy korzystali z~konwencji języka~Go, które są
  poprawioną wersją tych stosowanych w~języku~C.

\end{frame}
% ##################





% ##################
\begin{frame}
  \frametitle{Zmienne boolowskie}


  W~informatyce oprócz wartości logiczny dużą rolę odgrywają
  \textbf{zmiennymi boolowskimi} \textbf{zmienne logiczne}, zwane też .
  Są~to zmienne które mogą przechowywać wartości logiczne \texttt{true}
  i~\texttt{false}. Tak jak w~przypadku sposobu reprezentowania wartości
  logicznych, również gdy chodzi o~zmienne sprawa ich działania zależy
  w~dużej mierze od używanego języka.

  Rozpatrzmy przypadek języka~C. Ponieważ wartości logiczne reprezentują
  w~nim liczby całkowite, więc jako zmiennej logicznej możemy użyć dowolnej
  zmiennej która przechowuje wartości całkowitoliczbowe. Zmienne tego typu
  w~języku~C są zwykle typu \texttt{int}.

\end{frame}
% ##################





% ##################
\begin{frame}
  \frametitle{Zmienne boolowskie}


  Poniżej prezentujemy przykładowy fragment programu napisanego w~języku~Go
  operujący na zmiennych logicznych. Symbol „\texttt{//}” oznacza, że~tekst
  na prawo od niego jest komentarzem dla człowieka i~jest zupełnie
  ignorowany przez komputer.

  \textit{Uwaga.} To jest tylko fragment programu w~języku~Go i~ze względu
  na brakujące części nie da~się go uruchomić.

  \texttt{// Tworzymy zmienną logiczną "boolVar" i~nadajemy jej} \\
  \texttt{// wartość "false".} \\
  \texttt{var boolVar bool = false} \\
  \vspace{0.8em}
  \texttt{// Zmieniamy wartość logiczną zmiennej "boolVar" na "true".} \\
  \texttt{boolVar = true}

\end{frame}
% ##################





% ##################
\begin{frame}
  \frametitle{Wyrażenie boolowskie}


  \textbf{Wyrażeniem boolowskim} (ang.~\textit{boolean expression}) zwane
  też \textbf{wyrażeniem logicznym} (ang.~\textit{logical expression})
  zwane  to dowolne wyrażenie w~wyniku obliczenia którego zostaje zwrócona
  wartość boolowska \texttt{true} albo \texttt{false}.

  \textit{Ważne.} Wyrażenie jest tylko i~wyłącznie obiekt, który
  \textit{zwraca} pewną wartość. W~przeciwnym razie mamy do czynienia
  z~\textbf{poleceniem} (ang. \textit{statement}).

  Przyjmujemy, że~wartości boolowskie \texttt{true} i~\texttt{false}, które
  zwracają jako swoją wartość same siebie. Nawet jeśli wydaje~się to dziwne
  i~przekombinowane, to taki sposób myślenia ma swoje zastosowania
  w~informatyce.

\end{frame}
% ##################





% ##################
\begin{frame}
  \frametitle{Operatory boolowskie}


  Jeśli funkcja o~nazwie \texttt{funBool} zwraca jako wartość zmienną
  boolowską, to jej wywołanie \\
  \texttt{funBool(argumenty funkcji)} \\
  jest wyrażeniem boolowskim. Takie funkcje spotyka~się w~praktyce, jednak
  częściej wyrażenia boolwskie tworzy~się za pomocą odpowiednich operatorów
  logicznych.

  \textbf{Operatorem boolowskim} (ang. \textit{boolean operator}) lub
  \textbf{operatorem logicznym} (ang. \textit{logcial operator})
  nazywamy symbol reprezentujący \textbf{operacje}, która to operacja
  pozwala z~dwóch wyrażeń boolowskich utworzyć nowe wyrażenie. Cztery
  podstawowe, z~punktu widzenia informatyki, operatory boolowskie to:
  logiczny iloczyn, logiczna suma, logiczna negacja i~operacja porównania.

\end{frame}
% ##################





% ##################
\begin{frame}
  \frametitle{Operatory boolowskie}


  \textbf{Iloczyn logiczny}, zwany też \textbf{koniunkcją logiczną},
  odpowiada zasadniczo temu co~w~języku polskim oznacza spójnik~„i”. Łączy
  on parę dwa wyrażenie boolowskie, zwracając logiczną prawdę wtedy i~tylko
  wtedy gdy wartość pierwszego \textit{i}~drugiego wyrażenia są~prawdziwe.
  W~przeciwny razie zwraca logiczny fałsz.

  Gdy chodzi o~notację dla iloczynu logicznego to ta jest dość zróżnicowana.
  Język C i~większość najpopularniejszych języków programowania oznacza
  go za~pomocą symbolu „\texttt{\&\&}”. Proszę zauważyć, że~w~tych językach
  symbol „\texttt{\&}” oznacza pewną operację, która jednak \textit{nie
    jest} iloczynem logicznym. Język Python oznacza ją za pomocą
  angielskiego symbolu „\texttt{and}”. W~matematyce często zapisuję~się ją
  za pomocą symbolu~„$\wedge$”.

  My~będziemy korzystać z~symbolu „\texttt{\&\&}”.

\end{frame}
% ##################





% ##################
\begin{frame}
  \frametitle{Iloczyn logiczny}


  Zgodnie z~tym co powiedziano wcześniej, wyrażenie \\
  \texttt{true \&\& false} \\
  jest wyrażeniem boolowskim i~zwraca wartość to \texttt{false}.

  Własności iloczynu logicznego można podsumować za pomocą następującej
  tabeli, gdzie \texttt{boolVar1} i~\texttt{boolVar2} to odpowiednie zmienne
  boolowskie. Jeśli zmienne \texttt{boolVar1} i~\texttt{boolVar2} mają
  wartości takie jak w~dwóch pierwszych kolumnach, to wartość iloczynu
  logicznego tych zmiennych podana jest w~trzeciej kolumnie.

  \vspace{1em}





  \begingroup

  \centering

  \begin{tabular}{|c|c|c|}
    \hline
    boolVar1 & boolVar2 & boolVar1 \&\& boolVar2 \\
    \hline
    \texttt{true}  & \texttt{true}  & \texttt{true}  \\
    \texttt{true}  & \texttt{false} & \texttt{false} \\
    \texttt{false} & \texttt{true}  & \texttt{false} \\
    \texttt{false} & \texttt{false} & \texttt{false} \\
    \hline
  \end{tabular}

  \endgroup

\end{frame}
% ##################





% ##################
\begin{frame}
  \frametitle{Tabelki operatorów logicznych}


  \begingroup

  \centering

  \begin{tabular}{|c|c|c|}
    \hline
    boolVar1 & boolVar2 & boolVar1 \&\& boolVar2 \\
    \hline
    \texttt{true}  & \texttt{true}  & \texttt{true}  \\
    \texttt{true}  & \texttt{false} & \texttt{false} \\
    \texttt{false} & \texttt{true}  & \texttt{false} \\
    \texttt{false} & \texttt{false} & \texttt{false} \\
    \hline
  \end{tabular}

  \endgroup

  \vspace{1em}





  Choć w~tabeli powyższe występują zmienne boolowskie, to jest ona
  prawdziwa dla dowolnych wyrażeń boolowskich. Proszę przy tym pamiętać,
  że~każda zmienna boolowska jest dla nas wyrażeniem boolowskim, ale nie
  każde wyrażenie boolowskie jest po prostu zmienną boolowską. Niezależnie
  od tych subtelności, jeśli wartości zwracane przez dwa wyrażenia
  boolowskie są takie jak w~dwóch pierwszych tabelach, to ich iloczyn
  logiczny ma wartości wyrażoną w~trzeciej.

\end{frame}
% ##################





% ##################
\begin{frame}
  \frametitle{Suma logiczna}


  \textbf{Suma logiczna}, zwana też \textbf{alternatywą logiczną} odpowiada
  mniej więcej temu, co w~języku polskim oznacza spójnik „lub”. Łączy ona
  dwa wyrażenia boolowskie dając nową wyrażenie, która wartość \texttt{true}
  jeśli wartość pierwszego \textit{lub} drugiego jest równa \texttt{true}.

  Podobnie jak dla iloczynu logicznego, język C i~za nim większość
  najpopularniejszy języków programowania oznacza sumę logiczną symbolem
  „\texttt{||}”. Ponownie, trzeba mieć na uwagę, że~w~tych językach~symbol
  „\texttt{|}” posiada osobne znaczenie. W~Pythonie operację tą oznacza
  symbol „\texttt{or}”. W~matematyce często stosuje~się symbol~„$\vee$”.

  My będziemy go zapisywać za~pomocą symbolu „\texttt{||}”.

  Na podstawie tego co powiedziano powyżej \\
  \texttt{false || true} \\
  jest wyrażeniem logicznym i~zwraca wartość \texttt{true}.

\end{frame}
% ##################





% ##################
\begin{frame}
  \frametitle{Suma logiczna}


  Podstawowe własności sumy logicznej podaje poniższa tabela.

  \vspace{1em}





  \begingroup

  \centering

  \begin{tabular}{|c|c|c|}
    \hline
    \texttt{boolVar1} & \texttt{boolVar2}
    & \texttt{boolVar1 || boolVar2} \\
    \hline
    \texttt{true}  & \texttt{true}  & \texttt{true}  \\
    \texttt{true}  & \texttt{false} & \texttt{true}  \\
    \texttt{false} & \texttt{true}  & \texttt{true}  \\
    \texttt{false} & \texttt{false} & \texttt{false} \\
    \hline
  \end{tabular}

  \endgroup

\end{frame}
% ##################





% ##################
\begin{frame}
  \frametitle{Negacja logiczna}


  \textbf{Negacja logiczna} odpowiada w~przybliżeniu co polskie słowo „nie”.
  Negacja prawdy logicznej jest fałszem, a~negacja fałszu jest prawdą.

  W~językach programowania negację danego wyrażenia \texttt{expr}, zwykle
  oznacza~się poprzez poprzedzenie go symbolem~„\texttt{!}” i~my też
  będziemy korzystać z~tego zapisu. W~matematyce często stosuje~się symbol
  „\texttt{$\neg$}”. Negację wyrażenia \texttt{expr} zapiszemy więc odpowiednio
  jako \texttt{!expr} lub \texttt{$\neg$expr}.

  Własności negacji logicznej podsumowuje poniższa tabelka.

  \vspace{1em}





  \begingroup

  \centering

  \begin{tabular}{|c|c|}
    \hline
    \texttt{boolVar1} & \texttt{!boolVar1} \\
    \hline
    \texttt{true}  & \texttt{false} \\
    \texttt{false} & \texttt{true}  \\
    \hline
  \end{tabular}

  \endgroup

\end{frame}
% ##################





% ##################
\begin{frame}
  \frametitle{Porównanie logiczne}


  \textbf{Porównanie logiczne} jest prawdą, wtedy i~tylko wtedy, gdy obie
  porównywane strony mają tą samą wartość logiczną.

  W~informatyce do porównywania zwykle używamy podwójnego znaku równości
  „\texttt{==}”, my też z~niego będziemy korzystali. Oczywiście, tutaj też
  muszą być wyjątki od tej reguły, ale prawdziwe źródło komplikacji
  leży gdzie indziej. Zanim o~nich powiemy, przytoczymy tabelę
  podsumowującą własności tej operacji.

  \vspace{1em}





  \begingroup

  \centering

  \begin{tabular}{|c|c|c|}
    \hline
    \texttt{boolVar1} & \texttt{boolVar2}
    & \texttt{boolVar1 == boolVar2} \\
    \hline
    \texttt{true}  & \texttt{true}  & \texttt{true}  \\
    \texttt{true}  & \texttt{false} & \texttt{false} \\
    \texttt{false} & \texttt{true}  & \texttt{false} \\
    \texttt{false} & \texttt{false} & \texttt{true}  \\
    \hline
  \end{tabular}

  \endgroup

\end{frame}
% ##################





% ##################
\begin{frame}
  \frametitle{Porównanie logiczne}


  Podstawowy problem z~operatorem „\texttt{==}” polega na tym,
  że~w~większości języków służy nie tylko do porównywania wyrażeń
  boolowskich, ale do porównywania wszystkich obiektów, których wartości
  można porównywać. Dostajemy w~ten sposób wyrażenie boolowskie.

  Przykładowo, w~języku C, Go czy w~Pythonie można za~jego pomocą
  porównywać liczby całkowite, tworząc odpowiednie wyrażenia boolowskie.
  Tym samym \\
  \texttt{0 == 1} \\
  jest w~tych językach wyrażeniem boolowskim, którego wartość logiczna jest
  równa \texttt{false}. Proszę pamiętać, że~w~każdym z~tych języków
  wartości logiczne są trochę inaczej reprezentowane.

\end{frame}
% ##################





% ##################
\begin{frame}
  \frametitle{Porównanie logiczne}


  Należy zwrócić uwagę, że~operator „\texttt{==}” zwykle sprawdza równość
  \textit{wartość} wyrażeń, nie ich \textit{identyczność}. Przykładowo,
  załóżmy, że~mamy dwie zmienne typu \texttt{int} w~języku~C, niech~się
  nazywają \texttt{intVar1} i~\texttt{intVar2}. Każda z~nich zawiera pewną
  wartość i~możemy~się pytać o~równość tych wartości. Możemy~się też
  zapytać, czy obie zmienne zajmują ten sam obszar pamięci (choćby pamięci
  \textsc{ram}), wtedy pytamy~się o~ich identyczność.

  Tym samym w~języku~C wyrażenie \\
  \texttt{intVar1 == intVar2} \\
  sprawdza, czy \textit{wartość} przechowywane przez obie te
  zmienne są równe, nie czy te zmienne są identyczne. Powtórzymy,
  identyczność zmiennych oznacza, że~są one po prostu dwiema różnymi
  nazwami na~to~samo (ten sam obszar pamięci).

\end{frame}
% ##################










% ######################################
\section{Parsowanie i~ewalucaja wyrażeń boolowskich}
% ######################################



% ##################
\begin{frame}
  \frametitle{Co to jest „parsowanie”?}


  Słowo „parsowanie” (ang.~\textit{parsing}) jest wzięte z~angielskiego
  i~oznacza, przynajmniej dla nas, tyle co „analiza składniowa wyrażenia”
  (ang. \textit{syntax analysis~of expression}), czyli ustalenie co dane
  wyrażenie tak naprawdę znaczy. Można uznać, że~jest to mądrze brzmiące
  słowo na zrozumienie co dany wzór znaczy.

  Przykładowo, co oznacza następujące wyrażenie arytmetyczne? \\
  \texttt{1 + 2 * 3} \\
  Parsowanie tego wyrażenia, które mamy tak wbite do głowy, że~pewnie o~tym
  nie myślimy, mówi nam, że~najpierw musimy wykonać mnożenie
  \texttt{2 * 3}, dostając liczbę \texttt{6}, następnie dodajemy do
  niej liczbę~\texttt{1} dostając w~wyniku~\texttt{7}.

  Teraz omówimy reguły parsowania wyrażeń boolowskich, ale najpierw musimy
  wyjaśnić co to jest ewaluacja przez krótkie spięcie.

\end{frame}
% ##################





% ##################
\begin{frame}
  \frametitle{Ewaluacja przez krótkie spięcie}


  \textbf{Ewaluacja} (ang. \textit{evaluation}) oznacza obliczanie wartości
  danego wyrażenia. Nie potrafię znaleźć lepszego tłumaczenia tego zwrotu
  na~język polski.

  \textbf{Ewaluacja przez krótkie spięcie} (ang. \textit{short-circut
    evaluation}) to powszechnie stosowana w~informatyce technika obliczania
  wyrażeń boolowskich będących ciągiem \texttt{\&\&} lub \texttt{||}.
  Metody ta jest stosowana w~większości najpopularniejszy języków
  programowania.

  Rozpatrzmy wyrażenie boolowskie postaci \\
  \texttt{boolExpr1 \&\& boolExpr2 \&\& boolExpr3 \&\& boolExpr4} \\
  Można je przepisać jako \\
  \texttt{boolExpr1 \&\& (boolExpr2 \&\& (boolExpr3 \&\& boolExpr4))} \\
  Pomimo, że~nie omówiliśmy reguł obliczania tego typu wyrażeń, sens
  powyższej notacji powinien być jasny poprzez analogię z~arytmetyką.

\end{frame}
% ##################





% ##################
\begin{frame}
  \frametitle{Ewaluacja przez krótkie spięcie}


  \texttt{boolExpr1 \&\& (boolExpr2 \&\& (boolExpr3 \&\& boolExpr4))}

  Z~własności iloczynu logicznego wynika, że~jeśli co najmniej jedno
  wyrażenie boolowskie \texttt{boolExpr1}, \texttt{boolExpr2},
  \texttt{boolExpr3}, \texttt{boolExpr4} zwraca wartość \texttt{false}, to
  cały iloczyn ma wartość \texttt{false}.

  Wynik ten automatycznie przenosi~się na wyrażenie o~tej samej formie
  zbudowane z~$N$ wyrażeń boolowskich: \\
  \texttt{boolExpr1 \&\& boolExpr2 \&\& \ldots{} \&\& boolExprN}.

\end{frame}
% ##################





% ##################
\begin{frame}
  \frametitle{Ewaluacja przez krótkie spięcie}


  Rozpatrzmy teraz następujące wyrażenie. \\
  \texttt{false \&\& (false || (true \&\& false)) \&\& (true \&\&
    (false || \\ }
  \hphantom{aaaa}\texttt{|| (true \&\& false)))}

  Mamy tu trzy wyrażenia w~iloczynie logicznym i~powinno być dość
  oczywiste, że~obliczenie drugie i~trzeciego z~nich będzie
  nasz kosztowało trochę pracy. Jednak obliczanie ich jest zupełnie
  zbytecznie, bo skoro pierwsze wyrażenie ma wartość \texttt{false} to
  wartość całego wyrażenie jest równa \texttt{false} i~nie musimy obliczać
  konkretnej wartości dwóch pozostałych wyrażeń. Oszczędza to nam sporo
  pracy.

\end{frame}
% ##################





% ##################
\begin{frame}
  \frametitle{Ewaluacja przez krótkie spięcie}


  \texttt{false \&\& (false || (true \&\& false)) \&\& (true \&\&
    (false || \\ }
  \hphantom{aaaa}\texttt{|| (true \&\& false)))}

  \textit{Uwaga.} W~rozumowaniu przedstawionym na poprzednim slajdzie
  przyjęliśmy, że~interesuje nas tylko wartość logiczna \textit{całego}
  wyrażenie, nie rezultaty zwrócone przez konkretne wyrazy.

  Można napisać kod w~którym wartość zwrócone przez konkretne wyrażenia
  składowe mają znaczenie, ale pisanie takiego kodu jest bardzo złą
  praktyką. Dlatego ten przypadek będziemy ignorować.

\end{frame}
% ##################





% ##################
\begin{frame}
  \frametitle{Ewaluacja przez krótkie spięcie}


  \texttt{false \&\& (false || (true \&\& false)) \&\& (true \&\&
    (false || \\ }
  \hphantom{aaaa}\texttt{|| (true \&\& false)))}

  % \vspace{0.8em}


  Regułę ewaluacji przez krótkie spięcie dla iloczynu logicznego można więc
  opisać w~następujący sposób. Jeśli na najwyższym poziomie analizowane
  wyrażenie jest ciągiem mniejszych wyrażeń połączonych symbolem
  \texttt{\&\&} to obliczamy je od lewej do prawej, aż~dojdziemy do końca
  lub do pierwszego wyrażenia które zwróciło jako swoją wartość
  \texttt{false}. W~tym ostatnim przypadku zwracamy jako wartość całego
  wyrażenia \texttt{false} i~nie obliczamy dalszych wyrazów. W~pozostałych
  przypadkach zwracamy \texttt{true} jako wartość całego wyrażenia.

  Jeśli to wyrażenie jest dla Państwa intuicyjnie zrozumiałe, to jest to
  ważniejsze, niż ewentualne luki w~jego sformułowaniu.

\end{frame}
% ##################





% ##################
\begin{frame}
  \frametitle{Ewaluacja przez krótkie spięcie}


  \texttt{boolExpr1 || boolExpr2 || boolExpr3 || \ldots{} || boolExprN}

  Analogicznie działa ewaluacja przez krótkie spięcie dla ciągu wyrażeń
  połączonych operacją \texttt{||}. Tym razem korzystamy z~tego, że~jeśli
  w~tym ciągu choć jedno wyrażenie ma wartość logiczną \texttt{true} to całe
  to wyrażenie ma wartość ma wartość \texttt{true}.

  Jeśli więc na najwyższym poziomie analizowane wyrażenie jest ciągiem
  mniejszych wyrażeń połączonych operatorem \texttt{||} to obliczamy je od
  lewej do prawej, aż~dojdziemy do końca lub do pierwszego wyrażenia które
  zwróciło jako swoją wartość \texttt{true}. W~tym ostatnim przypadku
  zwracamy jako wartość całego wyrażenia \texttt{true} i~nie obliczamy
  dalszych wyrazów. W~pozostałych przypadkach zwracamy \texttt{false} jako
  wartość całego wyrażenia.

\end{frame}
% ##################





% ##################
\begin{frame}
  \frametitle{Parsowanie i~ewaluacja wyrażeń boolowskich}


  Reguły mówiące w~jakiej kolejności obliczać wyrażenia boolowskie
  są podane poniżej, przy czym najpierw wykonujemy te o~niższym numerze.
  Od razu musimy zaznaczyć, że~zasady te obowiązują w~większości języków
  programowania, acz, bo czemu życie miałoby być tak proste, zdarzają~się
  wyjątki.

  \begin{itemize}

  \item[1)] Jeżeli na najwyższym poziomie wyrażenie jest ciągiem
    \texttt{\&\&} albo \texttt{||} to dokonujemy ewaluacji przez krótkie
    spięcie.



  \item[2)] Wyrażenie w~nawiasach.



  \item[2)] Negacja logiczna.



  \item[3)] Iloczyn logiczny.



  \item[4)] Suma logiczna.



  \item[5)] Porównanie logiczne.

  \end{itemize}

  Tutaj ponownie, intuicyjna zrozumiałość jest ważniejsza, niż precyzyjne
  sformułowanie tych reguł.

\end{frame}
% ##################





% ##################
\begin{frame}
  \frametitle{Przykład~1}


  Rozpatrzmy wyrażenie \\
  \texttt{false || true \&\& (false || !false) || !true \&\& true}

  W~wyniku jego parsowania na podstawie reguł 2--5 dostajemy: \\
  \texttt{false || (true \&\& (false || (!false))) ||
    ((!true) \&\& true)} \\
  Teraz stosujemy ewaluację przez krótkie spięcie. Wyrażenie \texttt{false}
  ma wartość \texttt{false}, więc przechodzi do drugiego z~nich: \\
  \texttt{(true \&\& (false || (!false)))} \\
  Stosując kilkakrotnie regułę~2 dochodzimy do wniosku, że~najpierw musimy
  obliczyć \\
  \texttt{!false == true}. \\
  Mamy więc \\
  \texttt{(true \&\& (false || true))} \\
  Znów z~reguły~2 wynika, że~najpierw należy wyliczyć \\
  \texttt{(false || true) == true}.

\end{frame}
% ##################








% ##################
\begin{frame}
  \frametitle{Przykład~1}


  Mamy więc \\
  \texttt{(true \&\& (false || (!false))) == true \&\& true == true}.

  Z~tego wynika, że~drugi wyrażenie w \\
  \texttt{false || true \&\& (false || !false) || !true \&\& true}, \\
  ma wartość logiczną \texttt{true}, więc na mocy ewaluacji przez krótkie
  spięcie, całe to wyrażenie ma wartość logiczną \texttt{true} i~nie
  ma potrzeby obliczanie wartości trzeciego wyrażenia.

\end{frame}
% ##################





% ##################
\begin{frame}
  \frametitle{Przykład~2}


  Rozpatrzmy teraz kilka bardzo prostych, ale również bardzo użytecznych
  w~informatyce wyrażeń boolowskich, ewaluowanych metodą krótkiego spięcia.

  \texttt{boolExpr1 \&\& boolExpr2} \\
  Załóżmy, że~\texttt{boolExpr1} zwraca nam wartość \texttt{true}. Wówczas
  wyrażenie przyjmuje postać \\
  \texttt{true \&\& boolExpr2}. \\
  W~takiej sytuacji wyrażenie \texttt{boolExpr2} musi zostać obliczone i~na
  podstawie przytoczonej tabelki dla \texttt{\&\&} możemy~się łatwo
  przekonać, że~zachodzi zależność \\
  \texttt{true \&\& boolExpr2 == boolExpr2}.

  Jeśli \texttt{boolExpr1} zwraca wartość \texttt{false}, to~zgodnie
  z~zasadą ewaluacji przez krótkie spięcie wartość \texttt{boolExpr2}
  nie zostanie nigdy wyliczona.

\end{frame}
% ##################






























% % ##################
% \begin{frame}
%   \frametitle{Basic concepts of computer graphics}






% \end{frame}
% % ##################





% % ##################
% \begin{frame}
%   \frametitle{Data structures}




% \end{frame}
% % ##################











% % ##################
% \begin{frame}
%   \frametitle{Basic concepts of computer graphics}




% \end{frame}
% % ##################





% % ##################
% \begin{frame}
%   \frametitle{Each model is based on  a set of vertices}



% \end{frame}
% % ##################





% % ##################
% \begin{frame}
%   \frametitle{Each model is based on  a set of vertices}




% \end{frame}
% % ##################





% % ##################
% \begin{frame}
%   \frametitle{Vertex attributes}




% \end{frame}
% % ##################





% % ##################
% \begin{frame}
%   \frametitle{Vertex attributes}



% \end{frame}
% % ##################






% % ##################
% \begin{frame}
%   \frametitle{Basic concepts of computer graphics}




% \end{frame}
% % ##################





% % ##################
% \begin{frame}
%   \frametitle{Primitives in the OpenGL library}




% \end{frame}
% % ##################





% % ##################
% \begin{frame}
%   \frametitle{Primitives in the OpenGL library}



% \end{frame}
% % ##################





% % ##################
% \begin{frame}
%   \frametitle{Primitives in the OpenGL library}



% \end{frame}
% % ##################





% % ##################
% \begin{frame}
%   \frametitle{Primitives in the OpenGL library}



% \end{frame}
% % ##################





% % ##################
% \begin{frame}
%   \frametitle{Primitives in the OpenGL library (deprecated since ver.~3.1)}






% \end{frame}
% % ##################





% % ##################
% \begin{frame}
%   \frametitle{Texturing and blending}




% \end{frame}
% % ##################





% % ##################
% \begin{frame}
%   \frametitle{Texturing and blending}




% \end{frame}
% % ##################





% % ##################
% \begin{frame}
%   \frametitle{Communication between CPU and GPU}




% \end{frame}
% % ##################





% % ##################
% \begin{frame}
%   \frametitle{Data stored in the video memory}




% \end{frame}
% % ##################





% % ##################
% \begin{frame}
%   \frametitle{Data stored in the video memory}




% \end{frame}
% % ##################





% % ##################
% \begin{frame}
%   \frametitle{Graphics pipeline (rendering pipeline)}




% \end{frame}
% % ##################





% % ##################
% \begin{frame}
%   \frametitle{Basic concepts of computer graphics}




% \end{frame}
% % ##################





% % ##################
% \begin{frame}
%   \frametitle{Basic concepts of computer graphics}





% \end{frame}
% % ##################





% % ##################
% \begin{frame}
%   \frametitle{Basic concepts of computer graphics}





% \end{frame}
% % ##################





% % ##################
% \begin{frame}
%   \frametitle{Vertex transformations (OpenGL nomenclature)}




% \end{frame}
% % ##################





% % ##################
% \begin{frame}[label=Przestrzen-uklad-wspolrzednych-1]
%   \frametitle{Spaces --~coordinate systems}





% \end{frame}
% % ##################





% % ##################
% \begin{frame}
%   \frametitle{Spaces --~coordinate systems}





% \end{frame}
% % ##################





% % ##################
% \begin{frame}
%   \frametitle{Rasterization and fragment operations}





% \end{frame}
% % ##################





% % ##################
% \begin{frame}
%   \frametitle{Rasterization and fragment operations}




% \end{frame}
% % ##################





% % ##################
% \begin{frame}
%   \frametitle{Fragment operations}






% \end{frame}
% % ##################





% % ##################
% \begin{frame}
%   \frametitle{Basic concepts of computer graphics}





% \end{frame}
% % ##################





% % ##################
% \begin{frame}
%   \frametitle{Links}




% \end{frame}
% % ##################










% % ######################################
% \appendix
% % ######################################





% % ##################
% \GeometryThreeDTwoSpecialEndingSlidesEN{Questions? Thank you for your attention.}
% % ##################



% % % ##################
% % \jagiellonianendslide{Dziękuję za~uwagę.}
% % % ##################



% ####################################################################
% ####################################################################
% Bibliography

\printbibliography





% ####################################################################
% End of the document

\end{document}
