% Autor: Kamil Ziemian

% --------------------------------------------------------------------
% Podstawowe ustawienia Beamera i używane pakiety
% --------------------------------------------------------------------
\RequirePackage[l2tabu, orthodox]{nag}  % Wykrywa przestarzałe i niewłaściwe
% sposoby używania LaTeXa. Więcej jest w l2tabu English version.

\documentclass{beamer}  % Klasa dokumentu
\mode<presentation>  % Rodzaj tworzonych slajdów Beamera
\usetheme[progressbar=frametitle,block=fill]{metropolis}  % Temat graficzny
% \usetheme{owl}

% \setbeamertemplate{headline}{}  % Usuwa nagłówek
% \setbeamersize{text margin left=3mm}  % Wielkość lewego marginesu
% \setbeamersize{text margin right=3mm}  % Wielkość prawego marginesu
% \setbeamertemplate{navigation symbols}{}  % Usuwa ikony nawigacji w prawym
% % dolnym rogu



\usepackage[polish]{babel}  % Tłumaczy na polski teksty automatyczne LaTeXa
% i pomaga z typografią.
\usepackage[MeX]{polski}  % Polonizacja LaTeXa, bez niej będzie pracował
% w języku angielskim.
\usepackage[utf8]{inputenc}  % Włączenie kodowania UTF-8, co daje dostęp
% do polskich znaków.
\usepackage{lmodern}  % Wprowadza fonty Latin Modern.
\usepackage[T1]{fontenc}  % Potrzebne do używania fontów Latin Modern.



% ----------------------------
% Podstawowe pakiety (niezwiązane z ustawieniami języka)
% ----------------------------
\usepackage{microtype}  % Twierdzi, że poprawi rozmiar odstępów w tekście.
\usepackage{graphicx}  % Wprowadza bardzo potrzebne komendy do wstawiania
% grafiki.



% ------------------------------
% Pakiety do tekstów z nauk przyrodniczych
% ------------------------------
% \let\lll\undefined  % Amsmath gryzie się z językiem pakietami do języka
% % polskiego, bo oba definiują komendę \lll. Aby rozwiązać ten problem
% % oddefiniowuję tę komendę, ale może tym samym pozbywam się dużego Ł.
% \usepackage{amsmath}  % Podstawowe wsparcie od American Mathematical
% % Society (w skrócie AMS)
\usepackage{amsfonts, amssymb, amscd, amsthm}  % Dalsze wsparcie od AMS
% \usepackage{siunitx}  % Do prostszego pisania jednostek fizycznych
\usepackage{upgreek}  % Ładniejsze greckie litery
% Przykładowa składnia: pi = \uppi
% \usepackage{slashed}  % Pozwala w prosty sposób pisać slash Feynmana.
\usepackage{calrsfs}  % Zmienia czcionkę kaligraficzną w \mathcal
% na ładniejszą. Może w innych miejscach robi to samo, ale o tym nic
% nie wiem.
% \usepackage{tikz}  % Potężny pakiet PGF/TikZ.
% \usetikzlibrary{decorations.markings}  % Włączenie konkretnych bibliotek
% % pakietu TikZ



% ----------------------------
% Pakiety napisane przez użytkownika.
% Mają być w tym samym katalogu to ten plik .tex
% ----------------------------
\usepackage{latexshortcuts}
\usepackage{mathshortcuts}



% --------------------------------------
% Komendy do prostszego pisania prezentacji
% --------------------------------------
% --------------------------------------



% ----------------------------
% Pakiet "hyperref" Polecano by umieszczać go na końcu preambuły.
% ----------------------------
\usepackage{hyperref}  % Pozwala tworzyć hiperlinki i zamienia odwołania
% do bibliografii na hiperlinki.





% --------------------------------------------------------------------
\title[Promieniowanie Hawkinga]{Promieniowanie Hawkinga czarnych dziur
  --~krótki wprowadzenie do~zagadnienia}

\subtitle{Seminarium Astrofizyki}

\author{Kamil Ziemian \\
  \texttt{kziemianfvt@gmail.com} }

\institute{Zakład Teorii Pola, \\
  Uniwersytet Jagielloński w~Krakowie}

\date[9 stycznia 2019~r.]{9 stycznia 2019~r.}
% --------------------------------------------------------------------





% ####################################################################
% Początek dokumentu
\begin{document}
% ####################################################################



% ######################################
\begin{frame}
  \titlepage
\end{frame}
% ######################################



% % ##########
% \begin{frame}
%   \frametitle{Plan wystąpienia}
%   \tableofcontents
% \end{frame}
% % ##########





% ##################
\begin{frame}
  \frametitle{Uwagi ogólne}

  \begin{block}{Podstawowe źródło z~którego korzystałem}
    Większość przedstawionych tu~materiału pochodzi z~pracy Hawkinga
    \emph{Particle Creation by~Black Holes} z~1975 roku
    \cite{HawkingParticleCreationByBlackHole1975}. Praca ta jednak
    omawia dużo zagadnień o~których nie mamy czasu tutaj mówić, jak
    choćby problem dynamiki rozważanego układu.
  \end{block}

  % \textbf{Podstawowe źródło z~którego korzystałem} \\
  %   Większość przedstawionych tu~materiału pochodzi z~pracy Hawkinga
  %   \emph{Particle Creation by~Black Holes} z~1975 roku
  %   \cite{HawkingParticleCreationByBlackHole1975}. Praca ta jednak
  %   omawia dużo zagadnień o~których nie mamy czasu tutaj mówić, jak
  %   choćby problem dynamiki rozważanego układu.

  \begin{block}{O~seminarium}
    Zdecydowałem~się nie usuwać z~wystąpienia wszystkich moim
    osobistych preferencji, przekonań i~uwag, mając nadzieję,
    że~dzięki temu będzie ono bardziej interesujące.
    Za~przesadną stronniczość, której nie udało się usunąć,
    przepraszam wszystkich Państwa.
  \end{block}

\end{frame}
% ##################





% ######################################
% \section{Wstęp kulturowy}
\section{Wstęp historyczny}
% ######################################



% % ##########
% \begin{frame}
%   \frametitle{Wstęp kulturowy}

%   \begin{block}{}
%     \begin{figure}
%       \centering

%       \includegraphics[height=1.5in,
%       width=1.1in]{Stephen-Hawking.jpg}
%       \caption{Stephen W. Hawking (1942--2018 r.)}
%     \end{figure}
%   \end{block}

% %   \begin{block}{Znany głównie jako aktor telewizyjny}
% %     Wystąpił m.in.~w~\emph{Star Trek: Next Generation},
% %     \emph{Simpsons}, \emph{Futurama} i~\emph{Big Bang Theory}.
% %   \end{block}

% \end{frame}
% % ##########





% % ##########
% \begin{frame}
%   \frametitle{Wstęp kulturowy}

%   \begin{block}{O~jego osobie}
%     Obok Einsteina zapewne najbardziej ikoniczne uosobienie naukowca
%     w~XX i~XXI wieku. Napisał co~najmniej jedną bardzo złą książkę
%     (jak twierdzą poważni krytycy) \emph{A Brief History of Time}
%     oraz~co najmniej jedną wspaniałą, stworzoną wraz z~Georgem
%     Ellis'em: \emph{The Large Scale Structure~of Space-Time}.
%   \end{block}

%   \begin{block}{Ale\ldots}
%     Z~nieznanych powodów jego osiągnięcia naukowe, są~tak naprawdę
%     bardzo słabo znane i~rozumiane wśród ogółu ludzi.
%   \end{block}

% \end{frame}
% % ##########





% ##########
\begin{frame}
  \frametitle{Wprowadzenie historyczne}

  % \begin{block}{Uwaga}
  %   \pause Słuchaczy prosi się o niezaśnięcie, bo będzie ich
  %   ciężko
  %   obudzić na następne wystąpienie.
  % \end{block}

  \begin{block}{Zarys historii}
    \begin{itemize}
    \item Jeśli twierdzimy (ktoś może być innego zdania), że~świat
      jest jednością i~dlatego musi istnieć jedna teoria fizyczna go
      opisująca, to należy znaleźć teorię która ujmuje jednocześnie
      efekty mechaniki kwantowej i~ogólnej teorii względności.
      Nazwijmy tę teorie kwantową grawitacją.
    \item Kwantowej grawitacji nie ma (długa historia).
    \item I~chyba jeszcze trochę nie będzie.
    \item Lata 60-te XX wieku to jeden ze~złotych okresów ogólnej
      teorii względności. Wtedy też znalezione przez Karl
      Schwarzschild w~1916 roku rozwiązanie równań Einsteina zostało
      zreinterpretowane jako przedstawiające nowy fenomen fizyczny:
      czarną dziurę. Nazwę podobno zaproponował John Archibald
      Wheeler, wybitna postać.
      % niezwykle wybitna naukowiec.
    \item W~pracy z~1973 roku, James M. Bardeen, Brandon Carter
      i~Stephen W.~Hawking proponują 4 prawa mechaniki czarnych dziur
      \cite{BardeenCarterHawkingFourLawsOfBHMechanics1973}.
    \end{itemize}
  \end{block}

\end{frame}
% ##########





% ##########
\begin{frame}
  \frametitle{Wprowadzenie historyczne}

  % \begin{block}{Uwaga}
  %   \pause Słuchaczy prosi się o niezaśnięcie, bo będzie ich
  %   ciężko
  %   obudzić na następne wystąpienie.
  % \end{block}
  % \pause

  \begin{block}{Zarys historii}
    \begin{itemize}
    \item W tym samym roku Jacob Bekenstein sugeruje, że~te prawa~są
      w~istocie prawami termodynamiki czarnych dziur i~podaje
      m.in.~wzory na~ich temperaturę i~entropię,
      \cite{BekensteinBlackHolesAndEntropy1973}.
    \end{itemize}
  \end{block}

  \begin{block}{To rodzi kilka pytań}
    \begin{itemize}
    \item Jeżeli mamy wzór na makroskopową entropię, po powinniśmy móc
      ją obliczyć z modelu mikroskopowego. Czyli zliczyć stany
      i~zlogarytmować otrzymają liczbę.
    \item Skoro ciało ma niezerową temperaturę powinno emitować
      wysyłać jakieś promieniowanie. Jednak klasycznie czarna dziura
      nie może promieniować.
    \end{itemize}
    Zapewne z~tych powodów Hawking w latach 70-tych, zaproponował
    by~przebadać układ: pole kwantowe +~niekwantowa ogólna teoria
    względności; licząc, że~efekty kwantowe pomogą wyjaśnić problem
    sprzeczności między zasadami OTW, a~zasadami termodynamiki.
  \end{block}

\end{frame}
% ##########





% ######################################
\section[Podstawy teorii]{Podstawy teorii promieniowania Hawkinga}
% ######################################



% ##########
\begin{frame}
  \frametitle{Podstawy modelu}

  \begin{block}{Pole kwantowe w~szczególnej teorii względności}
    Teoria ta jest względnie dobrze rozumiana i~wielokrotnie
    potwierdzona eksperymentalnie. Pole bezmasowego, pozbawionego
    spinu bozonu (pole skalarne) spełnia swobodne równanie
    \begin{equation}
      \label{eq:HawkingPromieniowanie-01}
      \eta^{ \mu \nu } \partial_{ \mu } \partial_{ \nu } \widehat{ \phi }( x )
      = 0.
    \end{equation}
    Załóżmy, że~chcemy napisać to równanie na klasycznej
    czasoprzestrzeni OTW. Jak to zrobić?
  \end{block}

  \begin{block}{Problem}
    Okazuje~się, że~jest to bardzo subtelne i~podstępne zagadnienie.
    Tutaj przedstawię podejście Hawkinga, nie wdając~się zbytnio
    w~dyskusję na~ile jest ono uzasadnione.
  \end{block}

\end{frame}
% ##########





% ##########
\begin{frame}
  \frametitle{Podstawy modelu}

  \begin{block}{Podejście Hawkinga}
    \begin{itemize}
    \item Czasoprzestrzeń jest niekwantowa i~posiada metrykę
      $g_{ \mu \nu }( x )$, spełniającą równania Einsteina.
    \item Równanie pola kwantowe obecnego w~tej ogólnej
      czasoprzestrzeni otrzymujemy z~równania
      \eqref{eq:HawkingPromieniowanie-01} przez dokonanie dość
      oczywistych podstawień
      \begin{equation}
        \label{eq:HawkingPromieniowanie-02}
        \eta^{ \mu \nu } \to g^{ \mu \nu }( x ),
        \quad \partial_{ \mu } \to \nabla_{ \mu },
      \end{equation}
      otrzymując
      \begin{equation}
        \label{eq:HawkingPromieniowanie-03}
        g^{ \mu \nu }( x ) \nabla_{ \mu } \nabla_{ \nu }
        \widehat{ \phi }( x ) = 0.
      \end{equation}
    \item Równania Einsteina mają postać:
      \begin{equation}
        \label{eq:HawkingPromieniowanie-04}
        R_{ \mu \nu }( x ) - \frac{ 1 }{ 2 } R( x )\, g_{ \mu \nu }( x )
        = \langle \widehat{ T }_{ \mu \nu }( x ) \rangle.
      \end{equation}
      To równanie ma sens, jeśli wartość oczekiwana
      $\langle \widehat{ T }_{ \mu \nu }( x ) \rangle$ kwantowego
      tensora $\widehat{ T }_{ \mu \nu }( x )$ jest odpowiednio
      regularnym polem tensorowym.
    \end{itemize}
  \end{block}

\end{frame}
% ##########





% ##########
\begin{frame}{Pole kwantowe}

  \begin{block}{Rozkład na mody normalne}
    Postulujemy istnienie rozkładu pola analogicznie do~pola
    swobodnego w~czasoprzestrzeni Minkowskiego
    \begin{equation}
      \label{eq:HawkingPromieniowanie-05}
      \widehat{ \phi }( x )
      = \sum_{ i }( f_{ i }( x )\, \widehat{ a }_{ i }
      + \bar{ f }_{ i }( x )\, \widehat{ a }^{ \dagger }_{ i } ),
    \end{equation}
    gdzie $f_{ i }( x )$ są zupełnym układem ortonormalnych,
    zespolonych rozwiązań równania falowego, zawierającymi tylko
    dodatnie częstości. Stan próżni jest określony przez warunek
    \begin{equation}
      \label{eq:HawkingPromieniowanie-06}
      \widehat{ a }_{ i }| 0 \rangle = 0, \quad \forall\, i.
    \end{equation}
  \end{block}

  \begin{block}{W~czasoprzestrzeni Minkowskiego}
    \begin{equation}
      \label{eq:HawkingPromieniowanie-07}
      f_{ i }( x ) = \frac{ 1 }{ \sqrt{ 2\pi \omega_{ 0 }( i ) } }
      \exp( -i \omega_{ 0 }( i ) t + i \vec{ k }( i )
      \cdot \vec{ \omega }( i ) ),
    \end{equation}
  \end{block}

\end{frame}
% ##########





% ##########
\begin{frame}{Pole kwantowe}

  \begin{block}{Tu zaczynają~się prawdziwe problemy}
    Ponieważ w~czasoprzestrzeni Minkowskiego mamy
    \begin{equation}
      \label{eq:HawkingPromieniowanie-08}
      f_{ i }( x ) = \frac{ 1 }{ \sqrt{ 2\pi \omega_{ 0 }( i ) } }
      \exp( -i \omega_{ 0 }( i ) t + i \vec{ k }( i )
      \cdot \vec{ \omega }( i ) ),
    \end{equation}
    więc, aby~ustalić które częstości są dodatnie potrzebujemy
    wiedzieć ,,w~którą stronę płynie czas''. Jednak w~ogólnej
    zakrzywionej czasoprzestrzeni nie ma czegoś takiego jak
    \textbf{globalny} czas, więc mamy problem.
  \end{block}

  \begin{block}{Operator anihilacji}
    To niejako z~definicji operator stojący przy funkcji zawierającej
    tylko dodatnie częstości. Skoro nie wiem, które częstości~są
    dodatnie, to nie mamy pojęcia który operator anihiluje, a~który
    kreuje cząstki!!!
  \end{block}

\end{frame}
% ##########





% ##########
\begin{frame}{Pole kwantowe i~zakrzywiona czasoprzestrzeń}

  \begin{block}{Co można zrobić?}
    \begin{itemize}
    \item[1.] Poddać~się i~porzucić badanie tego problemu.
    \item[2.] Dojść do~wniosku, że~na zakrzywionej czasoprzestrzeni
      pojęcie cząstki nie ma sensu (z~punktu widzenia fizyki
      teoretycznej).
    \item[3.] Rozpatrzeć przypadek, gdzie coś da~się powiedzieć.
    \end{itemize}
  \end{block}

  \begin{block}{Wyjście 3}
    Jeżeli czasoprzestrzeń $M$ zawiera obszar który jest płaski
    (asymptotycznie płaski), wtedy istnieje w~nim kanoniczny czas
    i~używają go możemy zdefiniować rodzinę rozwiązań równań pola
    skalarnego o~dodatnich częstościach. Niemniej jeśli mamy dwa takie
    obszary nie możemy zwykle powiązać czasu w~jednym obszarze
    z~czasem w~drugim. Co~to oznacza?
  \end{block}

\end{frame}
% ##########





% ##########
\begin{frame}{Pole kwantowe i~zakrzywiona czasoprzestrzeń}

  \begin{block}{Rozkład na mody normalne}
    \begin{subequations}
      \begin{equation}
        \label{eq:HawkingPromieniowanie-09a}
        \widehat{ \phi }( x )
        = \sum_{ i }( f_{ i }( x )\, \widehat{ a }_{ i }
        + \bar{ f }_{ i }( x )\, \widehat{ a }^{ \dagger }_{ i } ),
      \end{equation}
      \begin{equation}
        \label{eq:HawkingPromieniowanie-09b}
        \widehat{ a }_{ i }| 0 \rangle = 0, \quad \forall\, i.
      \end{equation}
    \end{subequations}
  \end{block}

  \begin{block}{Modelowy przykład}
    Rozpatrzmy czasoprzestrzeń w~której mamy ,,następujący po~sobie''
    \begin{itemize}
    \item[1.] obszar płaski;
    \item[2.] obszar z~niezerową krzywizną;
    \item[3.] obszar płaski.
    \end{itemize}
    Oznaczmy układ zupełny rozwiązań w~obszarze~1 przez
    $f_{ i }( x )$, natomiast w~obszarze $g_{ i }( x )$, zaś~próżnię
    w~tych dwóch obszarach jako~$| 0_{ 1 } \rangle$
    i~$| 0_{ 3 } \rangle$. Tym samym
    \begin{equation}
      \label{eq:HawkingPromieniowanie-10}
      \widehat{ a }_{ f, i } | 0_{ 1 } \rangle = 0, \quad
      \widehat{ a }_{ g, i } | 0_{ 3 } \rangle = 0.
    \end{equation}
  \end{block}

\end{frame}
% ##########





% ##########
\begin{frame}{Pole kwantowe i~zakrzywiona czasoprzestrzeń}

  \begin{block}{Ale}
    Skoro w~ogólności układy funkcji $f_{ i }( x )$
    i~$g_{ i }( x )$~są różne, więc tym samym
    $\widehat{ a }_{ f, i } \neq \widehat{ a }_{ g, i }$. Z~tego
    wynika, że
    \begin{equation}
      \label{eq:HawkingPromieniowanie-11}
      \widehat{ a }_{ g, i }\, | 0_{ 1 } \rangle \neq 0.
    \end{equation}
  \end{block}

  \begin{block}{Bardzo dziwny wniosek}
    ,,Przejście'' przez dany obszar zakrzywienia powoduje kreacje
    nowych cząstek. Konkretnie tych cząstek dla których operatorów
    anihilacji zachodzi warunek \eqref{eq:HawkingPromieniowanie-11}. Ich ilość jest dana
    przez
    \begin{equation}
      \label{eq:HawkingPromieniowanie-12}
      N
      = \langle 0_{ 1 } | \sum_{ i } \widehat{ a }_{ g, i }^{ \dagger }
      \widehat{ a }_{ g, i }\, | 0_{ 1 } \rangle.
    \end{equation}
  \end{block}

  \begin{block}{Moja skromna opinia}
    Wiele elementów tego podejścia jest wątpliwych. Zamiast tego można
    pójść np.~drogą porzucenia pojęcia cząstki w~zakrzywionej
    czasoprzestrzeni.
    % Nie rozumiem fizyki tego zjawiska.
    % Bardziej mnie przekonuje, że~pojęcie cząstki nie ma
    % teoretycznego
    % sensu w~zakrzywionej czasoprzestrzeni.
  \end{block}

\end{frame}
% ##########





% ######################################
\section[Przypadek czarnej dziury Schwarzschilda]{Efekt Hawkinga
  dla~czarnej dziury Schwarzschilda}
% ######################################



% ##########
\begin{frame}
  \frametitle{Efekt Hawkinga dla~czarnej dziury Schwarzschilda}

  \begin{block}{Uwaga}
    Aby obliczyć ilość ,,cząstek wypromieniowanych przez czarną
    dziurę''
    \begin{equation}
      \label{eq:HawkingPromieniowanie-13}
      N
      = \langle 0_{ 1 } | \sum_{ i } \widehat{ a }_{ g, i }^{ \dagger }
      \widehat{ a }_{ g, i }\, | 0_{ 1 } \rangle,
    \end{equation}
    Hawking wykonał całkiem pomysłowy i~dość żmudny rachunek.
    W~dalszym ciągu naszkicuję jak on~wyglądał oraz do~jakich wniosków
    prowadzi.
  \end{block}

  \begin{block}{Czarna dziura Schwarzschilda}
    Metrykę dla tego przepadku można odczytać z~wzoru
    \begin{equation}
      \label{eq:HawkingPromieniowanie-14}
      ds^{ 2 } = -\left(1 - \frac{ 2M }{ r } \right) dt^{ 2 }
      + \left(1 - \frac{ 2M }{ r } \right)^{ -1 } dr^{ 2 }
      + r^{ 2 } ( d \theta^{ 2 } + \sin^{ 2 }\theta d\phi^{ 2 } ).
    \end{equation}
    Przyjęliśmy ,,naturalny układ jednostek'' $c = 1$, $G = 1$,
    $\hbar = 1$, $k_{ \mathrm{B} } = 1$.
  \end{block}

\end{frame}
% ##########





% ##########
\begin{frame}{Pole kwantowe i~zakrzywiona czasoprzestrzeń}
  \frametitle{Efekt Hawkinga dla~czarnej dziury Schwarzschilda}

  \begin{block}{Warunek ortogonalności}
    Będziemy rozpatrywali rodziny rozwiązań równania pola skalarnego,
    \begin{equation}
      \label{eq:HawkingPromieniowanie-15}
      g^{ \mu \nu }( x ) \nabla_{ \mu } \nabla_{ \nu }
      \widehat{ \phi }( x ) = 0.
    \end{equation}
    spełniające warunek:
    \begin{equation}
      \label{eq:HawkingPromieniowanie-16}
      \frac{ 1 }{ 2 } i\int_{ S }( f_{ i }( x )
      \nabla_{ a }\bar{ f }_{ j }( x ) - f_{ j }( x )
      \nabla_{ a } \bar{ f }_{ i }( x ) )\, \de\sigma_{ a }
      = \delta_{ i j },
    \end{equation}
    gdzie $S$ to~wybrana powierzchnia.
  \end{block}

  \begin{block}{Powierzchnie~$S$}
    W~czasoprzestrzeni Schwarzschilda~są trzy interesujące nas
    powierzchnie.
    \begin{itemize}
    \item[1.] Horyzont czarnej dziury $H$.
    \item[2.] Nieskończona przeszłość świetlna $\mathcal{J}^{ - }$.
    \item[3.] Nieskończona przyszłość świetlna $\mathcal{J}^{ + }$.
    \end{itemize}
  \end{block}

\end{frame}
% ##########




% ##########
\begin{frame}{Podstawowe wiadomości}
  \frametitle{Efekt Hawkinga dla~czarnej dziury Schwarzschilda}

  \begin{block}{Rodziny funkcji}
    \begin{itemize}
    \item Przez $f_{ i }( x )$ oznaczamy zupełny układ rozwiązań
      równania pola $\widehat{ \phi }( x )$, ortogonalnych względem
      całki po~$\mathcal{J}^{ - }$. Przedstawiają cząstki przylatujące
      z~nieskończonej przeszłości czasowej.
    \item $p_{ i }( x )$~są ortogonalne względem całki
      na~$\mathcal{J}^{ + }$ i~znikają wraz z~pierwszymi pochodnymi
      na~$H$. Przestawiają cząstki wylatujące z~czarnej dziury
      i~zmierzające do~nieskończonej przyszłość czasowej.
    \item $q_{ i }( x )$~są ortogonalne na~$H$ i~znikają
      na~$\mathcal{J}^{ + }$. Przedstawia cząstki ,,złapane'' wokół
      czarnej dziury
    \end{itemize}
  \end{block}

\end{frame}
% ##########





% ##########
\begin{frame}
  \frametitle{Efekt Hawkinga dla~czarnej dziury Schwarzschilda}

  \begin{block}{Dwa przedstawienia pola $\widehat{ \phi }( x )$}
    Pole $\widehat{ \phi }( x )$ jest określone w~całej
    czasoprzestrzeni, jeśli zadana jest jego wartość, wraz
    z~pierwszymi pochodnymi na~$\mathcal{J}^{ - }$, lub jednocześnie
    na~$H$ i~$\mathcal{J}^{ + }$. Możliwe są więc dwa następujące jego
    rozkłady na~mody normalne:
    \begin{equation}
      \label{eq:HawkingPromieniowanie-17}
      \widehat{ \phi }( x )
      = \sum_{ i }( f_{ i }( x )\, \widehat{ a }_{ f,\, i }
      + \bar{ f }_{ i }( x )\, \widehat{ a }^{ \dagger }_{ f,\, i } ),
    \end{equation}
    oraz
    \begin{equation}
      \label{eq:HawkingPromieniowanie-18}
      \widehat{ \phi }( x )
      = \sum_{ i }( p_{ i }( x )\, \widehat{ b }_{ p,\, i }
      + \bar{ p }_{ i }( x )\, \widehat{ b }^{ \dagger }_{ p,\, i }
      + q_{ i }( x )\, \widehat{ c }_{ q,\, i }
      + \bar{ q }_{ i }( x )\, \widehat{ c }^{ \dagger }_{ q,\, i } ).
    \end{equation}
    Chcemy policzyć cząstek zaobserwowanych w~$\mathcal{J}^{ + }$,
    czyli
    \begin{equation}
      \label{eq:HawkingPromieniowanie-19}
      N = \langle 0_{ - } | \sum_{ i } \widehat{ b }^{ \dagger }_{ p,\, i }
      \widehat{ b }_{ p,\, i } | 0_{ - } \rangle.
    \end{equation}
  \end{block}

\end{frame}
% ##########





% ##########
\begin{frame}
  \frametitle{Efekt Hawkinga dla~czarnej dziury Schwarzschilda}

  \begin{block}{} % {Zadanie}
    Ponieważ wiemy jak $\widehat{ a }_{ f,\, i }^{ \dagger }$,
    $\widehat{ a }_{ f,\, i }$ działają na~$| 0_{ - } \rangle$,
    problem będzie rozwiązany, jeśli wyrazimy za~ich pomocą
    $\widehat{ b }_{ p,\, i }^{ \dagger }$,
    $\widehat{ b }_{ p,\, i }$. By~to zrobić, zauważmy,
    że~$p_{ i }( x )$ i~$q_{ i }( x )$, możemy znów rozłożyć na
    funkcje $f_{ i }( x )$:
    \begin{subequations}
      \begin{equation}
        \label{eq:HawkingPromieniowanie-20a}
        p_{ i }( x )
        = \sum_{ j }( \alpha_{ i j } f_{ j }( x ) + \beta_{ i j }
        \bar{ f }_{ j }( x ) ),
      \end{equation}
      \begin{equation}
        \label{eq:HawkingPromieniowanie-20b}
        q_{ i }( x ) = \sum_{ j }( \gamma_{ i j } f_{ j }( x )
        + \eta_{ i j } \bar{ f }_{ j }( x ) ).
      \end{equation}
    \end{subequations}
    Podstawiają to~dostajemy
    \begin{equation}
      \label{eq:HawkingPromieniowanie-21}
      \widehat{ \phi }( x )
      = \sum_{ i }( p_{ i }( x )\, \widehat{ b }_{ p,\, i }
      + \bar{ p }_{ i }( x )\, \widehat{ b }^{ \dagger }_{ p,\, i }
      + q_{ i }( x )\, \widehat{ c }_{ q,\, i }
      + \bar{ q }_{ i }( x )\, \widehat{ c }^{ \dagger }_{ q,\, i } ),
    \end{equation}
    skąd
    \begin{subequations}
      \begin{equation}
        \label{eq:HawkingPromieniowanie-22}
        \widehat{ b }_{ i } = \sum_{ j }( \bar{ \alpha }_{ i j }\,
        \widehat{ a }_{ j } - \bar{ \beta }_{ i j }\,
        \widehat{ a }^{ \dagger }_{ j } ).
      \end{equation}
      % \begin{equation}
      %   \label{eq:22b}
      %   \widehat{ c }_{ i } = \sum_{ j }( \bar{ \gamma }_{ i j }\,
      %   \widehat{ a }_{ j }
      %   - \bar{ \eta }_{ i j }\, \widehat{ a }^{ \dagger }_{ j } ).
      % \end{equation}
    \end{subequations}
  \end{block}

\end{frame}
% ##########





% ##########
\begin{frame}{Szkic obliczeń}
  \frametitle{Efekt Hawkinga dla~czarnej dziury Schwarzschilda}

  \begin{block}{Wniosek}
    Liczba cząstek emitowana w $i$-tym modzie wynosi więc:
    \begin{equation}
      \label{eq:HawkingPromieniowanie-23}
      N_{ i } = \langle 0 | \widehat{ b }^{ \dagger }_{ i } \widehat{ b }_{ i }
      | 0 \rangle
      = \sum_{ j } | \beta_{ i j } |^{ 2 }.
    \end{equation}
    W~tym miejscu problem został w~zasadzie zredukowany do~dość
    skomplikowanego zagadnienia z~teorii rozpraszania.
  \end{block}

  \begin{block}{Uwagi o~obliczaniu $\beta_{ i j }$}
    Ich wyliczenie, wymaga dość żmudnych rachunków, uwzględniających
    rozkład na~odpowiednie mody sferyczne, operowania pakietami
    falowymi, etc. Dlatego przedstawię tylko konieczne wzory
    i~oznaczenia.
  \end{block}

  \begin{block}{Uwaga o~tych rachunkach}
    Dodatkowo Hawking skorzystał w~tych rachunkach z~przybliżenia
    optyki geometrycznej, czego zasadność bywała krytykowana (zob.
    \cite{FredenhagenHaagDerivationOfHawkingRadiation1990}).
  \end{block}

\end{frame}
% ##########




% ##########
\begin{frame}
  \frametitle{Efekt Hawkinga dla~czarnej dziury Schwarzschilda}

  \begin{block}{Rozkład ,,sferyczny''}
    We~współrzędnych ,,sferycznych'' zupełny układ funkcji jest dany
    przez $f_{ \omega l m }( x )$, ,,numerowanych'' ciągłym
    wskaźnikiem częstości $\omega$. Tym samym zmieniają~się też
    indeksy współczynników $\alpha_{ \omega l m\, \omega' l' m' }$
    i~$\beta_{ \omega l m\, \omega' l' m' }$. Definiujemy
    \begin{equation}
      \label{eq:HawkingPromieniowanie-24}
      p_{ \omega l m }( x )
      = \int\limits_{ 0 }^{ +\infty } ( \alpha_{ \omega l m\, \omega' l m }
      f_{ \omega' l m }( x ) + \beta_{ \omega l m\, \omega' l m }
      \bar{ f }_{ \omega' l m }( x ) )\, \de \omega'.
    \end{equation}
    Niech $p_{ \omega l m }^{ ( 1 ) }( x )$ oznacza rozwiązanie, które
    propaguje~się wstecz w czasie, rozprasza na~statycznej czarnej
    dziurze Schwarzchilda i~dociera na~$\mathcal{J}^{ - }$ z~tą samą
    częstością~$\omega$. Dokonujemy rozkładu
    \begin{equation}
      \label{eq:HawkingPromieniowanie-25}
      p_{ \omega l m }( x ) = p_{ \omega l m }^{ ( 1 ) }( x ) + p_{
        \omega l m }^{ ( 2 ) }( x ).
    \end{equation}
    % \begin{equation}
    %   f_{ \omega' l m } = ( 2 \pi )^{ - \frac{ 1 }{ 2 } } r^{ - 1 }
    %   ( \omega' )^{ - \frac{ 1 }{ 2 } } F_{ \omega ' }( r ) e^{ i \omega' v }
    %   Y_{ l m }( \theta, \varphi ),
    % \end{equation}
    % \begin{equation}
    %   f_{ \omega' l m } = ( 2 \pi )^{ - \frac{ 1 }{ 2 } }
    %   r^{ - 1 } ( \omega' )^{ - \frac{ 1 }{ 2 } } P_{ \omega ' }( r )
    %   e^{ i \omega' u } Y_{ l m }( \theta, \varphi ),
    % \end{equation}
  \end{block}

\end{frame}
% ##########





% ##########
\begin{frame}
  \frametitle{Efekt Hawkinga dla~czarnej dziury Schwarzschilda}

  \begin{block}{Analogicznie}
    \begin{subequations}
      \begin{align}
        \label{eq:HawkingPromieniowanie-26}
        \alpha_{ \omega l m\, \omega' l m }
        &= \alpha_{ \omega l m\, \omega' l m }^{ ( 1 ) }
          + \alpha_{ \omega l m\, \omega' l m }^{ ( 2 ) }, \\
        \beta_{ \omega l m\, \omega' l m }
        &= \beta_{ \omega l m\, \omega' l m }^{ ( 1 ) }
          + \beta_{ \omega l m\, \omega' l m }^{ ( 2 ) }.
      \end{align}
    \end{subequations}
  \end{block}

  \begin{block}{Definiujemy nową rodzinę modów}
    \begin{equation}
      \label{eq:HawkingPromieniowanie-27}
      p_{ j n l m }( x )
      = \veps^{ -\frac{ 1 }{ 2 } } \int\limits_{ j \veps }^{ ( j + 1 ) \veps }
      e^{ -2 \pi i\, \mathrm{n} \omega / \veps } p_{ \omega l m }( x ) \de \omega,
      \quad j = 0, 1, 2, \ldots
    \end{equation}
    gdzie $\veps > 0$. Potrzebujemy jeszcze jednej wielkości
    \begin{equation}
      \label{eq:HawkingPromieniowanie-28}
      \Gamma_{ j n }
      = \int\limits_{ 0 }^{ +\infty }
      \left( \abso{ \alpha_{ j n \omega' l m }^{ ( 2 ) } }^{ 2 }
        - \abso{ \beta_{ j n \omega' l m }^{ ( 2 ) } }^{ 2 } \right)\,
      \de \omega'.
    \end{equation}
  \end{block}

\end{frame}
% ##########





% ##########
\begin{frame}
  \frametitle{Końcowy wynik i~jego znaczenie}

  \begin{block}{} % {Końcowy wynik}
    Ilość cząstek wyemitowanych w~modzie $p_{ j n l m }( x )$ wynosi
    \begin{equation}
      \label{eq:HawkingPromieniowanie-29}
      N_{ j n l m } = \Gamma_{ j n }\;
      \frac{ 1 }{ e^{  \frac{ 2 \pi }{ \kappa } \omega } - 1 }.
    \end{equation}
    $\kappa$ to tak zwana grawitacja powierzchniowa (ang.~surface
    gravity).
  \end{block}

  \begin{block}{Sens}
    Po~analizie $\Ga_{ j n }$ okazuje~się, że~ten wzór opisujący
    ,,emisję cząstek z~powierzchni czarnej dziury Schwarzschilda'',
    jest identyczny jak dla ciała dla którego
    $E / k_{ \textrm{B} } T = 2\pi \omega / \kappa$. Biorąc
    $E = \hbar \omega$ otrzymujemy
    \begin{equation}
      \label{eq:HawkingPromieniowanie-30}
      T = \frac{ \hbar \kappa }{ 2\pi k_{ \mathrm{B} } }
      = \frac{ \hbar }{ 2\pi c } A
      \approx 10^{ -6 } \left( \frac{ M_{ \odot } }{ M_{ BH } } \right)
      {}^{ \circ } \mathrm{K},
    \end{equation}
    gdzie $A$ to powierzchnia czarnej dziury. Według Hawkinga podobny
    wyniki zachodzi dla pola elektromagnetycznego i~zlinearyzowanej
    grawitacji.
  \end{block}

\end{frame}
% ##########





% ##########
\begin{frame}
  \frametitle{Uwagi końcowe}

  \begin{block}{}
    Od~1975 roku upłynęło dużo czasu i~całkiem sporo~się działo
    również jeśli chodzi o~wyprowadzenie oraz~konsekwencje efektu
    Hawkinga, poddano krytyce metodę użytą w~oryginalnej pracy, jak
    i~pomniejsze jej wyniki oraz~stwierdzenia. Na~dokładniejszą
    dyskusję nie ma tu jednak czasu.
  \end{block}

\end{frame}
% ##########





% ##########
\begin{frame}

  \begin{center}
    \LARGE Dziękuję.
  \end{center}

\end{frame}
% ##########





% ##########
\begin{frame}
  \frametitle{Bibliografia}

  \begin{block}{}
    \bibliographystyle{plalpha} \bibliography{ArticPhilNatur}{}
  \end{block}

\end{frame}
% ##########





% ##########
\begin{frame}
  \frametitle{Grawitacja powierzchniowa czarnej dziury}

  \begin{block}{Założenia}
    Rozważmy statyczną, osiowo symetryczną, asymptotycznie płaską
    czasoprzestrzeń z~obecną w niej czarną dziurą. To~implikuje
    istnienie dwóch pól~Killing'a: $K^{ \mu }$~odpowiedzialne
    za~przesunięcie w~czasie i~$\tilde{ K }^{ \mu }$ odpowiedzialne
    za~obroty przestrzenne. Istnienie przestrzennopodobnego pola
    Killinga $\tilde{ K }^{ \mu }$ pozwala nam wybrać asymptotycznie
    płaską hiperpowierzchnię $B$ do~której jest ono styczne i~która
    przecina horyzont zdarzeń na~dwie części
  \end{block}

\end{frame}
% ##########





% ##########
\begin{frame}
  \frametitle{Grawitacja powierzchniowa czarnej dziury}

  \begin{block}{Trochę więcej pojęć}
    Rozważmy parametr $t$ krzywych całkowych pola $K^{ \mu }$ i~wektor
    świetlny obliczony na horyzoncie zdarzeń:
    \begin{equation}
      \label{eq:HawkingPromieniowanie-31}
      l^{ \mu } =  \dd{}{ x^{ \mu } }{ t }.
    \end{equation}
    Wprowadźmy wektor $n^{ \mu }$ styczny do~hiperpowierzchni $B$
    i~znormalizowany do
    \begin{equation}
      \label{eq:HawkingPromieniowanie-32}
      l^{ \mu } n_{ \mu } = -1.
    \end{equation}
    Grawitacja powierzchniowa w~określonym punkcie jest zdefiniowana
    jako:
    \begin{equation}
      \label{eq:HawkingPromieniowanie-33}
      \kappa = - \big( \nabla_{ \mu } l_{ \nu } \big) n^{ \mu } l^{ \nu }.
    \end{equation}
    Dla czarnej dziury Schwarzschilda jest ona stała na~całych
    horyzoncie zdarzeń.
  \end{block}

\end{frame}
% ##########





% ##########
\begin{frame}
  \frametitle{Grawitacja powierzchniowa czarnej dziury}

  \begin{block}{Sens matematyczno-fizyczny}
    $\kappa$ wyraża stopień w jakim $t$ nie jest parametrem
    afinicznym.
  \end{block}

\end{frame}
% ##########





% ##########
\begin{frame}
  \frametitle{Inne prace}

  \begin{block}{Według mnie interesujący nurt}
    Klaus Fredenhagena i~Rudolfa Haaga w~swej pracy
    \emph{On~the~derivation~of Hawking radiation associated with
      the~formation~of a~black hole} z~1990 roku
    (\cite{FredenhagenHaagDerivationOfHawkingRadiation1990}),
    kontynuują podejście Hawkinga, opierając je jednak na~inny
    sposobie postawieniu problemu w~kwantowej teorii pola: analizie
    funkcji Wightmana. Uzyskują dodatkowe wyniki i~potencjalne
    ograniczenia dla~zależności pokazanych przez Hawkinga.
  \end{block}

  \begin{block}{Wynik oraz uwagi Fredenhagena i~Haaga}
    \begin{itemize}
    \item ,,W~kolapsie grawitacyjny promieniowanie Hawkinga dla~dużych
      czasów jest ściśle związane z~granicą skalowania na~sferze, gdy
      promień gwizdy przechodzi przez promień Schwarzschilda (przy
      założeniu, że~zaniedbujemy oddziaływanie promieniowania
      na~metrykę).''
    \item Dla pola swobodnego, otrzymali wyniki jak Hawking. Analiza
      jednak sugeruje, że~raczej nie istnieje żaden prosty,
      uniwersalny wzór.
    \end{itemize}
  \end{block}

\end{frame}
% ##########





% ##########
\begin{frame}
  \frametitle{Inne podejście do~zagadnienia}

  \begin{block}{Wynik oraz uwagi Fredenhagena i~Haaga}
    \begin{itemize}
    \item Praca argumentuje, że~asymptotyczna swoboda QCD może mieć
      duże znaczenie dla prawdziwej sytuacji.
    \item Wykorzystanie przez Hawkinga w~jego oryginalnej pracy
      przybliżenia optyki geometrycznej nie jest w~pełni uzasadnione,
      gdyż, mówiąc tym językiem, promień światła będzie musiał
      przechodzić przez obszar gdzie ekstremalnie szybko zmienia~się
      współczynnik załamania.
    \end{itemize}
  \end{block}

  \begin{block}{Promieniowanie Hawkinga w~teorii bez cząstek}
    ,,Uczniowie'' Fredenhagena i~Haaga, Valter Moretti oraz~Nicola
    Pinamonti kontynuowali te~badania, wyprowadzając efekt Hawkinga
    w~podejściu do~teorii pola, które nie wymaga pojęcia cząstki.
    Wyniki ich można znaleźć w~pracy Moretti, Pinamonti, \emph{State
      independence for~tunneling processes through black hole horizons
      and~Hawking radiation}, \cite[]. Na~ten temat nie mogę niestety
    wiele powiedzieć.
  \end{block}

\end{frame}
% ##########





% ############################

% Koniec dokumentu
\end{document}
