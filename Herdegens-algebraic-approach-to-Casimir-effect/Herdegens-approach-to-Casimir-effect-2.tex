% Autor: Kamil Ziemian

% ---------------------------------------------------------------------
% Podstawowe ustawienia Beamera i używane pakiety
% ---------------------------------------------------------------------
\RequirePackage[l2tabu, orthodox]{nag}  % Wykrywa przestarzałe i niewłaściwe
% sposoby używania LaTeXa. Więcej jest w l2tabu English version.

\documentclass[10pt,t]{beamer}  % Klasa dokumentu
\mode<presentation>  % Rodzaj tworzonych slajdów Beamera
\usetheme[style=dark]{jagiellonian}  % Temat graficzny



% General configuration of jagiellonian
\setbeamertemplate{section in toc}[sections numbered]  % Numeracja
% pozycji w spisie treści
% Wybór fontu
\usefonttheme{serif}
\usepackage{mathpazo}  % Nowoczesna paczka do Palatino



\usepackage[polish]{babel}  % Tłumaczy na polski automatyczne teksty LaTeXa
% i pomaga z typografią.
\usepackage[MeX]{polski}  % Polonizacja LaTeXa, bez niej będzie pracował
% w języku angielskim.
\usepackage[utf8]{inputenc}  % Włączenie kodowania UTF-8, co daje dostęp
% do polskich znaków.
\usepackage[T1]{fontenc}  % Potrzebne do używania fontów Latin Modern.



% ------------------------------
% Podstawowe paczki (niezwiązane z ustawieniami języka)
% ------------------------------
% \usepackage{mathfix}   %
\usepackage{microtype}  % Twierdzi, że poprawi rozmiar odstępów w tekście.
\usepackage{graphicx}  % Wprowadza bardzo potrzebne komendy do wstawiania
% grafiki.



% ------------------------------
% Lepsze wsparcie polskich znaków w takich miejscach jak konspekty PDFa
% ------------------------------
\hypersetup{pdfencoding=auto,psdextra}


% ------------------------------
% Paczki potrzebne w danym projekcie
% ------------------------------
% \usepackage{tabu}  % Lepsze tabele
\newcommand{\bmmax}{0}  % Górne ograniczenie na ilość "alfabetów
% matematycznych", które może wykorzystać paczka "bm" do tworzenia
% "Bold symbols". Zapobiega przekroczenia standardowej maksymalnej
% ilości jakie dopuszcza TeX (i pdfLaTeX), wynoszącej 16.
\newcommand{\hmmax}{0}
\usepackage{bm}  % Bold Math symbols, obecnie preferowany sposób.




% ------------------------------
% Pakiety do tekstów z nauk przyrodniczych
% ------------------------------
% \let\lll\undefined  % Amsmath gryzie się z pakietami do języka
% % polskiego, bo oba definiują komendę \lll. Aby rozwiązać ten problem
% % oddefiniowuję tę komendę, ale może tym samym pozbywam się dużego Ł.
\usepackage{amsmath}  % Podstawowe wsparcie od American
% Mathematical Society (w skrócie AMS)
% \usepackage{amsfonts, amssymb, amscd, amsthm}  % Dalsze wsparcie od AMS
\usepackage{amssymb, amscd, amsthm}  % Dalsze wsparcie od AMS
\usepackage{siunitx}  % Do prostszego pisania jednostek fizycznych
\usepackage{upgreek}  % Ładniejsze greckie litery
% Przykładowa składnia: pi = \uppi
% \usepackage{icomma} % Przecinek w trybie matematyczny ma właściwy odstęp



% ------------------------------
% TikZ
% ------------------------------
% \usepackage{tikz}  % Potężny pakiet PGF/TikZ.
% Włączenie konkretnych bibliotek pakietu TikZ
% \usetikzlibrary{arrows.meta}
% \usetikzlibrary{angles}
% \usetikzlibrary{3d}
% \usetikzlibrary{positioning}
% \usetikzlibrary{intersections}
% \usetikzlibrary{decorations.markings}
% \usetikzlibrary{calc}





% ------------------------------
% Komendy i ustawienia do wykładów ,,''.
% ------------------------------
\newcommand{\titlebackground}{./pictures/Clerk_Maxwell.jpg}
% Dla trybu Dark
% \newcommand{\sectionbackground}{./pictures/SzablonTytBluedark.png}
% Dla trybu Light
\newcommand{\sectionbackground}{./pictures/Newton.jpg}



\newcommand{\jagielloniantitlepage}{
  \begingroup

  \setbeamercolor{background canvas}{bg=JagiellonianBlue}

  \maketitle

  \endgroup

  \addtocounter{framenumber}{1}
}

\newcommand{\jagielloniantitlepagewithpicture}{
  \addtocounter{framenumber}{1}

  \begingroup
  \usebackgroundtemplate{ % \hspace*{-11.5em}
    \includegraphics[height=\paperheight]{\titlebackground}}

  \maketitle

  \endgroup
}


\newcommand{\jagielloniansectionwithpicture}[1]{
  \begingroup
  \usebackgroundtemplate{ % \hspace*{-11.5em}
    \includegraphics[height=\paperheight]{\sectionbackground}}

  \setbeamercolor{titlelike}{fg=normal text.fg}

  \section{#1}

  \endgroup
}



\newcommand{\jagiellonianendslide}[1]{
  \begin{frame}[standout]


    {\color{jStrongWhite} #1}

    % \begin{textblock}{5.5}(0.65,3)
    %   \includegraphics[scale=0.45]{./pictures/Fundusze_Europejskie.png}
    % \end{textblock}

    % \begin{textblock}{3}(3.65,3.9)
    %   \includegraphics{./pictures/CC-by-sa.png}
    % \end{textblock}

  \end{frame}
}





% ############################
% Beamer i LaTeX

% Zmiana koloru fontu matematycznego
\newcommand{\beamermathcolor}[2]{{\color{#1}\setbeamercolor{math text}
  {fg=#1} #2}}

% Zmiana koloru fontu na foreground BEAMER-koloru ,,normal text''
\newcommand{\normaltextcolor}[1]{{\color{normal text.fg}{#1}}}

% Zmiana koloru fontu matematycznego na jMathTextForegorundWhite
\newcommand{\beamermathcolorwhite}[1]{{\color{jStrongWhite}
    \setbeamercolor{math text}{fg=jStrongWhite} #1}}

\newcommand{\alertmath}[1]{{\color{jLightBrown}
  \setbeamercolor{math text}{fg=jLightBrown} #1}}


\newcommand{\colorhref}[2]{\href{#1}{\color{jLightBrown} #2}}





% ------------------------------
% Pakiety napisane do wykładów z ,,???'''.
% ------------------------------
% \usepackage{./SymulacjeFizykiPaczki/}
% \usepackage{./SymulacjeFizykiPaczki/SymulacjeFizyki}




% ------------------------------
% Textpos (TEXT POSition) -- pakiet do dobrego zarządzania ułożeniem
% obiektów na „kartce” tekstu
% ------------------------------
\usepackage[showboxes,overlay,absolute]{textpos}
\setlength{\TPHorizModule}{10mm}
\setlength{\TPVertModule}{\TPHorizModule}
% \textblockorigin{10mm}{10mm}





% ------------------------------
% Paczki, biblioteki i ich ustawienia dla tego pliku
% ------------------------------
\usepackage{siunitx}






% ---------------------------------------
% Pakiety których pliki *.sty mają być w tym samym katalogu co ten plik
% ---------------------------------------
\usepackage{latexshortcuts}
\usepackage{mathshortcuts}





% ------------------------------
% Ustawienie dla tego konkretnego pliku
% ------------------------------
\newcommand{\Ac}{\mathcal{A}}
\newcommand{\Ec}{\mathcal{E}}
\newcommand{\Hc}{\mathcal{H}}
\newcommand{\Lc}{\mathcal{L}}
\newcommand{\Rc}{\mathcal{R}}





% ------------------------------
% Pakiet „hyperref”
% Polecano by umieszczać go na końcu preambuły
% ------------------------------
\usepackage{hyperref}  % Pozwala tworzyć hiperlinki i zamienia odwołania
% do bibliografii na hiperlinki.










% ---------------------------------------------------------------------
\title{Herdegen's algebraic approach to Casimir effect}
\subtitle{What two plates and two delta like systems tells as about?}
% \title{Julia}
% \subtitle{2010s proposition for scientific (and other) programming}

\author{Kamil Ziemian \\
  \texttt{kziemianfvt@gmail.com}}

\institute{Jagiellonian University in~Cracow}
                                % \institute{Uniwersytet Jagielloński w~Krakowie}

% \date[11 December 2018]{Seminarium astrofizyczne PAU \\
%   11 grudnia 2019}
\date[17 June 2020]{Virtual LQP Workshop, 17 VI 2020}
% ---------------------------------------------------------------------










% ####################################################################
% Początek dokumentu
\begin{document}
% ####################################################################



% ######################################
\maketitle % Tytuł całego tekstu
% ######################################



% ######################################
\begin{frame}
  \frametitle{Table of contents}


  \tableofcontents % Spis treści

\end{frame}
% ######################################










% ##################
\begin{frame}
  \frametitle{General remarks}


  The main aim of this talk is to present foundations of Herdegen's
  framework and interesting results obtained in it. On the
  mathematical side only main steps are presented, often in simplified
  form. Also some important results are only mention or even omitted.

  All results presented here are derived in as rigorous way as of
  today's mathematics requires. Precise definitions, results and they
  proofs can be found in the works listed at the end of presentation.

  I assume general knowledge of algebraic approach to quantum physics,
  but for the sake of clarity I will repeat well know points,
  stressing this that are important here.

\end{frame}
% ##################









% ######################################
\section{Casimir effect in algebraic settings}
% ######################################



% ##################
\begin{frame}
  \frametitle{Algebraic formalism}


  In this talk quantum system will be described by three elements.

  1) $C^{ * }$ algebra $\Ac$.

  2) Representation $\pi$ of algebra $\Ac$ as operators acting on
  Hilbert space $\Hc$.
  \begin{equation}
    \label{eq:1}
    \pi: \Ac \to \pi( \Ac )
  \end{equation}

  3) Time evolution implemented by family of automorphism of $\Ac$ and
  unitary transformations of $\Hc$.
  \begin{align*}
    &\alpha_{ t }: \Ac \to \Ac\normaltextcolor{,}
      \quad
      A \mapsto \alpha_{ t }A\normaltextcolor{,}
      \quad
      A \in \Ac \\
    &\pi( \alpha_{ t } A ) = U( t ) \pi( A ) U^{ * }( t )\normaltextcolor{,}
      \quad
      U( t ) = \exp( it H )\normaltextcolor{,}
  \end{align*}
  where $H$ is hamiltonian of the system. We assume that it is
  nonnegative.

\end{frame}
% ##################





% ##################
\begin{frame}
  \frametitle{Algebraic formalism}


  Two systems $Q_{ 0 }$ and $Q_{ 1 }$ are physically comparable if and
  only if they have common algebra $\Ac$ and equivalent
  representations $\pi_{ 0 }$ and $\pi_{ 1 }$. If not, comparison of
  such systems can leads to ``unreasonably'' phenomena of physical
  origin, like physically justified infinities.

  Consider for Minkowski spacetime at thermal equilibrium with
  $T = 0 \si{.K}$ and $T = 1 \si{.K}$. Transition from first state to
  the second require infinite amount of energy delivered to the
  system, so we shouldn't be surprised if compering this two states
  leads to infinite results.

  We call interaction $V$ \textbf{singular} if it introduction to the
  system change representation to nonequivalent to $\pi_{ 0 }$. In
  other cases we call it \textbf{nonsingular}.

\end{frame}
% ##################





% ##################
\begin{frame}
  \frametitle{Algebraic problems of two ``plates'' system [Her01]}


  Consider massless scalar field $\phi$ in $1 + 1$ dimensions ($1 + 3$
  is analogous) with two metal ``plates'' at distance $x = 0$ and
  $x = a$. In standard analysis of Casimir effect we require that
  field obey Dirchlet boundary conditions.
  \begin{align}
    &\partial_{ t }^{ 2 } \phi( x, t ) = -\Delta \phi( x, t ) \\
    &\phi( 0 ) = \phi( a ) = 0
  \end{align}
  How algebra $C^{ * }$ for such case looks like?

\end{frame}
% ##################





% ##################
\begin{frame}
  \frametitle{Algebraic problems of two plate system [Her01]}


  In current context algebra for two ``plates'' with Dirichlet
  boundary conditions should be build on the top of symplectic space
  of functions from $L^{ 2 }( \mathbb{R}, dx )$, that are regular
  enough and vanish at the ``plates'' positions $0$ and $a$ (more
  precisely there are sum of functions from $H_{ 0 }^{ 1 }$). But this
  can give us different algebraic model for every value of $a$!

  Situation is even worse when we want to find common algebra for
  plate position vary in range $( -l, l )$ and
  $( a - \varepsilon, a + \varepsilon )$. The functions must now
  vanish on both these intervals, which give us trivial theory.

  Maybe I will have time go back to this problem.

\end{frame}
% ##################





% ##################
\begin{frame}
  \frametitle{Moral of the story}


  Dirichlet boundary conditions belongs to singular class of
  interactions. I~will argue that for these reason we shouldn't be
  surprised be appearing of infinities in such \emph{linear} system,
  since they are probably physical.

  Also, renormalization of physical inifinites can lead to obscuring
  or even losing of important information's about system.

  In the story part I will sketch Herdegen's formalism that allow us
  to investigate systems free from all this problems, which are
  ``close enough'' to Dirichlet case to recover canonical Casimir
  term. I will also explain what mean ``close enough''.

\end{frame}
% ##################








% ######################################
\section{Herdegen's approach}
% ######################################



% ##################
\begin{frame}
  \frametitle{Herdegen framework. Assumptions [Her05]}


  Let $Q$ be quantum system with Hamiltonian $H_{ 0 }$ and $M$ be
  macroscopic system described by classical variables collectively
  denoted $a$. For example $a$ can by just separation of the plates.
  We make following assumptions about they interaction.

  1) Introduction of system $M$ is nonsingular for the $Q$.

  2) When $M$ is in fixed state described by $a_{ 1 }$ time evolution
  is given by
  \begin{align}
    \label{eq:2}
    &\alpha_{ a_{ 1 },\, t } \Ac \mapsto \Ac\normaltextcolor{,} \\
    &\pi( \alpha_{ a_{ 1 }, t } A ) = U_{ a_{ 1 } }( t ) \pi( A )
      U_{ a_{ 1 } }^{ * }( t )\normaltextcolor{,}
      \quad
      U_{ a_{ 1 } }( t ) = \exp( i t H_{ a_{ 1 } } )\normaltextcolor{.}
  \end{align}
  Since we can always add $C \, \textrm{id}$ to Hamiltonian there is
  big ambiguity in definition of $H_{ a }$:
  \begin{equation}
    \label{eq:3}
    H_{ a } \to H_{ a } + C( a ) \, \textrm{id}\normaltextcolor{,}
  \end{equation}
  with arbitrary function $C( a )$. Fortunately, this not introduce
  ambiguity to physical predictions.

\end{frame}
% ##################





% ##################
\begin{frame}
  \frametitle{Herdegen framework. Assumptions [Her05]}


  3) We will consider only \textbf{adiabatic} change of parameters
  $a( t )$.

  4) When joint system $Q$-$M$ evolves, energy of $Q$ is as previously
  given by $H_{ 0 }$, but due to interactions with $M$ quantum states
  evolution is given by family of operators $H_{ M }( t )$.

  5) Due to adiabatic changes of $a( t )$ we can use approximation
  \begin{equation}
    \label{eq:4}
    H_{ M }( t ) = H_{ a( t ) }\normaltextcolor{,}
  \end{equation}
  where $H_{ a( t ) }$ was defined previously as Hamiltonian for fixed
  state of $M$ with parameters $a( t )$.

\end{frame}
% ##################





% ##################
\begin{frame}
  \frametitle{Herdegen framework. Assumptions [Her05]}


  6) For fixed $t$ consider eigenstate of $H_{ a( t ) }$
  \begin{equation}
    \label{eq:5}
    H_{ a( t ) } \psi_{ a( t ) } = E_{ a( t ) } \psi_{ a( t ) }\normaltextcolor{,}
    \quad
    \normaltextcolor{\textrm{for fixed }} t\normaltextcolor{.}
  \end{equation}
  We assume that this eigenstates are not degenerated and depend
  continuously on $a( t )$. Since we are interested in ground state
  this is not very restrictive assumption. Adiabatic evolution
  $\psi( t )$ of state $\psi_{ a( 0 ) }$ is given by
  \begin{equation}
    \label{eq:6}
    \psi( t ) = \exp\big( i \varphi( E_{ a( t ) }, \psi_{ a( t ) }, t ) \big)
    \psi_{ a( t ) }\normaltextcolor{.}
  \end{equation}

  7) Expectation value of any observable $O$ in the state
  $\psi_{ a( t ) }$ is given by
  \begin{equation}
    \label{eq:7}
    \langle O \rangle_{ t }
    =
    ( \psi_{ a( t ) }, O \psi_{ a( t ) } )\normaltextcolor{.}
  \end{equation}
  This expression is independent of any ambiguity in choosing of
  $H_{ a( t ) }$.

\end{frame}
% ##################





% ##################
\begin{frame}
  \frametitle{Herdegen framework. Assumptions [Her05]}


  8) Casimir energy is given by expectation value of $H_{ 0 }$
  (Hamiltonian of isolated system $Q$) in its current ground state
  $\Omega_{ a( t ) }$.
  \begin{equation}
    \label{eq:8}
    \Ec_{ a( t ) } = ( \Omega_{ a( t ) }, H_{ 0 } \Omega_{ a( t ) } )
  \end{equation}

  9) Casimir force is minus derivative of energy with respect to
  $a( t )$.
  \begin{equation}
    \label{eq:9}
    \mathcal{F}_{ a( t ) }
    =
    -\frac{ \partial \Ec_{ a( t ) } }{ \partial a( t ) }
  \end{equation}
  Since we are interested in how energy and force depend on $a$, not
  on $t$, from this moment I will omitting time dependence in $a$.

\end{frame}
% ##################





% ##################
\begin{frame}
  \frametitle{Constructions of algebra and representation [Her05]}


  This was list of our wishes, but if they can't come true, it is
  void. Fortunately there exist quit general construction.

  We need \emph{real} Hilbert space $\Rc$ and operator selfadjoint
  operator $h$ with domain $D( h )$. Operator $h$ need to be strictly
  positive or nonnegative (more about this latter): $h > 0$ or
  $h \geq 0$.

  Consider $\Lc = D( h ) \oplus \Rc \subset \Rc \oplus \Rc$. We denote
  $V_{ i } = v_{ i } \oplus u_{ i }$, $v_{ i } \in D( h )$,
  $u_{ i } \in \Rc$. $\Lc$ is symplectic space with symplectic
  product:
  \begin{equation}
    \label{eq:10}
    \sigma( V_{ 1 }, V_{ 2 } )
    =
    ( v_{ 2 }, u_{ 1 } ) - ( v_{ 1 }, u_{ 2 } )\normaltextcolor{.}
  \end{equation}
  Dynamics on $\Lc$ is given by Hamiltonian
  \begin{equation}
    \label{eq:11}
    \Hc( V )
    =
    \frac{ 1 }{ 2 } [ ( h v, h v ) + ( u, u ) ]\normaltextcolor{.}
  \end{equation}
  We can solve equation of motions for this system and find time
  evolution family of transformations $T_{ t }$.

\end{frame}
% ##################





% ##################
\begin{frame}
  \frametitle{Constructions of algebra and representation [Her05]}


  Let $\mathcal{K} = \Rc \oplus i \Rc$ be complexification of $\Rc$.
  We define operator
  \begin{align*}
    &j: \Lc \to \mathcal{K}\normaltextcolor{,} \\
    &j( V ) = j( v \oplus u ) = h^{ 1 / 2 } v - i h^{ -1/2 } u\normaltextcolor{.}
  \end{align*}
  Operator $j$ requires that $h > 0$. In special cases we can allow it
  to have $0$ eigenvalue.

  After making this more precise and putting some work we can use
  standard construction of Weyl $C^{ * }$ algebra corresponding to
  symplectic space $\Lc$ with representation in Fock space with
  $\mathcal{K}$ as ``one-particle space'' (see the second volume of
  Bratteli, Robinson book [BR87]).

  When you introduce macroscopic bodies $M$ we only need to change
  operator $h$ to $h_{ a }$ and repeat all above construction. This is
  easy to say, but hard to follow.

\end{frame}
% ##################





% ##################
\begin{frame}
  \frametitle{Energy and number of particles [Her05]}


  Ground state representations for algebraic systems constructed this
  way are equivalent if and only if
  \begin{equation}
    \label{eq:12}
    \mathcal{N}_{ a } = ( \Omega_{ a }, N \Omega_{ a } )
    =
    \textrm{Tr}\big[ h^{ -1/2 } ( h_{ a } - h ) h_{ a }^{ -1 }
    ( h_{ a } - h ) h^{ -1/2 } \big] < \infty\normaltextcolor{.}
  \end{equation}
  In the context of quantum field this formula have clear meaning:
  introduction of macroscopic bodies create only finite number of free
  particles.

  \vspace{1em}


  Physics requires that Casimir energy is also finite.
  \begin{equation}
    \label{eq:13}
    \Ec_{ a } = ( \Omega_{ a }, H_{ 0 } \Omega_{ a } )
    =
    \textrm{Tr}\big[ ( h_{ a } - h ) h_{ a }^{ -1 } ( h_{ a } - h ) \big]
    < \infty
  \end{equation}

  Proving of r.h.s. of this formulas takes some time.

\end{frame}
% ##################





% ##################
\begin{frame}
  \frametitle{Local energy density for quantum field [Her05]}


  In this case ground state $\Omega_{ a }$ defines distribution on
  pairs of test functions $f$, $g$ with $L^{ 2 }$ scalar product.
  \begin{equation}
    \label{eq:14}
    T_{ a }( \, f, g )
    =
    \frac{ 1 }{ 4 } \big( \, f, ( h_{ a } - h ) g \big)
    + \frac{ 1 }{ 4 } \big( \nabla f, ( h_{ a }^{ -1 } - h^{ -1 } )
    \nabla g \big)
  \end{equation}
  By definition, outside singular support of distribution this give us
  kernel function $T_{ a }( \vecx, \vecy )$. In such region, and
  \emph{only} in it, we can defined local energy density:
  \begin{equation}
    \label{eq:15}
    \varepsilon_{ a }( \vecx ) = T_{ a }( \vecx, \vecx )\normaltextcolor{.}
  \end{equation}

\end{frame}
% ##################





% ##################
\begin{frame}
  \frametitle{What this framework give us?}


  \begin{itemize}
  \item We can recover in rigorous manner textbook results for two
    plates.

  \item Since energy is equal $( \Omega_{ a }, H_{ 0 } \Omega_{ a } )$
    it must be positive. There is no dubious ``negative energy''.

  \item All physical quantities are under control.

  \item We can quite well understand relation between global and local
    energy.

  \item It is hard to find nonsingular interactions that looks simple
    and ``natural''.

  \item To understand how to use this formalism you need to put some
    work.

  \end{itemize}

\end{frame}
% ##################






% ##################
\begin{frame}
  \frametitle{What you need to do in practice?}


  1) Find promising system described by operators $h$ and $h_{ a }$.
  Show that $\mathcal{N}_{ a } < \infty$ and $\Ec_{ a } < \infty$.

  2) Introduce rescaled family of systems $h_{ \lambda,\, a }$,
  $\lambda \in [ 1, 0 )$ such that $h_{ 1,\, a } = h_{ a }$.

  3) Show that for all $\lambda \in [ 1, 0 )$ you have
  $\mathcal{N}_{ \lambda,\, a } < \infty$,
  $\Ec_{ \lambda,\, a } < \infty$.

  4) Show that in limit $\lambda \searrow 0$ your family tends in the
  sens of resolvent limit (or other well defined way) to interesting
  system: $h_{ \lambda,\, a } \to h_{ I }$. This limit case is most
  probably singular.

  5) Derive asymptotic expansion of Casimir energy
  $\Ec_{ \lambda,\, a }$ around $\lambda = 0$.

  6) Compute energy density $\varepsilon_{ \lambda,\, a }( \vecx )$.
  Distribution $T_{ \lambda,\, a }( \, f, g )$ most probably will be
  regular on $\mathbb{R}^{ 3 }$.

  7) Compute energy density limit for $\lambda \searrow 0$. If limit
  case is singular, singular support of limit distribution will be
  probably nonempty.

\end{frame}
% ##################





% ##################
\begin{frame}
  \frametitle{Solved systems}


  \begin{itemize}
  \item Multidimensional harmonic oscillator, Andrzej Herdegen, 2005,
    [Her05].

  \item Two plate system for scalar and electromagnetic field,
    Herdegen, 2006, [Her06], Andrzej Herdegen and Mariusz Stopa 2010,
    [HS10].

  \item Two point like objects for scalar field, Ziemian, 2020, to be
    published.

  \end{itemize}

\end{frame}
% ##################










% ######################################
\section{Two plates system}
% ######################################



% ##################
\begin{frame}
  \frametitle{Herdegen and Stopa approach to two plates [HS10]}


  Plates are parallel to $xy$ plane, so interesting dynamics is only
  in $z$ directions. After ``integreting out $xy$ directions over unit
  area'', consistency conditions take form
  \begin{align}
    &\textrm{Tr}\big[ ( h_{ z,\, a } - h_{ z } )^{ 2 } \big]
      < \infty\normaltextcolor{,} \\
    &\textrm{Tr}\big[ ( h_{ z,\, a } - h_{ z } ) h_{ z } ( h_{ z,\, z }
      - h_{ z } ) \big] < \infty\normaltextcolor{.}
  \end{align}

  Energy per unit area
  \begin{equation}
    \label{eq:16}
    E_{ a }
    =
    \frac{ 1 }{ 24 \pi }
    \textrm{Tr}\big[ ( h_{ z,\, a } - h_{ z } )
    ( 2 h_{ z,\, a } + h_{ z,\, a } ) ( h_{ z,\, a } - h_{ z } ) \big]
  \end{equation}

  In this special case, since we don't have take $h_{ a }^{ -1 }$, it
  can have $0$ eigenvalue.

\end{frame}
% ##################





% ##################
\begin{frame}
  \frametitle{Herdegen and Stopa approach to two plates [HS10]}


  We will consider simpler case of scalar field. We have
  $\Rc = L^{ 2 }_{ \mathbb{R} }( \mathbb{R}^{ 3 }, d^{ 3 }x )$,
  $\mathcal{K} = L^{ 2 }( \mathbb{R}^{ 3 }, d^{ 3 }x )$,
  $h = \sqrt{ -\Delta }$.
  \begin{equation}
    \label{eq:17}
    h_{ a }^{ \, 2 }
    =
    -\Delta + V
  \end{equation}
  $V$ is projection operator of finite rank. In position space it is
  integral operator with kernel
  \begin{equation}
    \label{eq:18}
    V( z, y )
    =
    \left[ f( z - b ) \overline{ f( y - b ) } + f( z + b )
      \overline{ f( y + b ) } \right]\normaltextcolor{.}
  \end{equation}
  $f( z )$ is smooth complex function with compact support.

  Rescaled system is defined by
  \begin{equation}
    \label{eq:19}
    f_{ \lambda }( z ) = \lambda^{ -1 } f\left( \frac{ z }{ \lambda } \right)
  \end{equation}
  For $\lambda \searrow 0$ system converge in resolvent sens to field
  with Dirichlet boundary conditions in planes $z = \pm b$.

\end{frame}
% ##################





% ##################
\begin{frame}
  \frametitle{Herdegen and Stopa approach to two plates [HS10]}


  Asymptotic expansion of Casimir energy
  \begin{equation}
    \label{eq:20}
    E_{ \lambda,\, a }
    = \frac{ E_{ \infty } }{ \lambda^{ 3 } } + \frac{ c }{ \lambda a^{ 2 } }
    - \frac{ \pi^{ 2 } }{ 1440 a^{ 3 } } + O( \lambda )\normaltextcolor{,}
  \end{equation}
  where $E_{ \infty }$, $c$ are constant.

  $E_{ \infty } / \lambda^{ 3 }$ is two times energy of single plate
  in vacuum, per unit
  area, \\
  $-\pi^{ 2 } / 1440 a^{ 3 }$ this term in this expansion recovers
  standard formula for Casimir force.

  Constant $c$ depends on choice of function $f$ and typically
  $c > 0$, so force become repulsive for large $a$.

  Since $\lambda$ is roughly thickens of the plate, this model is only
  valid when $\lambda < a$ and this guaranties that energy is always
  positive.

\end{frame}
% ##################





% ##################
\begin{frame}
  \frametitle{Herdegen and Stopa approach to two plates [HS10]}


  For $\lambda \neq 0$ local energy density
  $\varepsilon_{ \lambda,\, a }( \vecx )$ is defined on whole
  $\mathbb{R}^{ 3 }$. In the limit $\lambda \searrow 0$ singular
  support of distribution is equal to $\{ -b, +b \}$ (on the $x$
  axis).

  For this reason integration over whole space is mathematically
  unjustified. But, if we do it anyway we arrive at result
  \begin{equation}
    \label{eq:21}
    \int \varepsilon_{ 0,\, a }( x ) \, dx
    =
    -\frac{ \pi^{ 2 } }{ 1440 a^{ 3 } }\normaltextcolor{.}
  \end{equation}
  This expression give us canonical Casimir term for force, but not
  recover total formula. It is also negative, while global energy is
  always positive.

\end{frame}
% ##################










% ######################################
\section{Two delta like systems}
% ######################################



% ##################
\begin{frame}
  \frametitle{Two delta like systems}


  In this case we know only results for scalar field.

  This problem is in many ways like two plates problem.
  $\Rc = L^{ 2 }_{ \mathbb{R} }( \mathbb{R}^{ 3 }, d^{ 3 }x )$,
  $\mathcal{K} = L^{ 2 }( \mathbb{R}^{ 3 }, d^{ 3 }x )$,
  $h = \sqrt{ -\Delta }$.
  \begin{equation}
    h_{ \vec{a} }^{ \, 2 }
    =
    -\Delta + V
  \end{equation}
  $V$ is again projection operator of finite rank.
  \begin{equation}
    \label{eq:22}
    V( \vecx, \vecy )
    =
    \sigma( g ) \left[ g( \vecx - \vec{b} )
      \overline{ g( \vecx - \vec{b} ) } + g( \vecx + \vec{b} )
      \overline{ g( \vecx + \vec{b} ) } \right]
  \end{equation}
  $g( \vecx )$ is smooth, positive, spherical symmetric function with
  compact support.

  Rescaled version of the model
  \begin{equation}
    \label{eq:23}
    g_{ \lambda }( \vecx ) = \lambda^{ -3 } g\left( \frac{ \vecx }{ \lambda }
    \right)
  \end{equation}
  In limit $\lambda \searrow 0$ we have two $\delta$ system, well know
  in literature [Alb88].

\end{frame}
% ##################





% ##################
\begin{frame}
  \frametitle{Two delta like systems}


  Asymptotic expansion of Casimir energy
  \begin{equation}
    \label{eq:24}
    \begin{split}
      &\Ec( a, \lambda ) = \Ec_{ \textrm{self} }( \lambda ) + \frac{ 2
        \alpha }{ \pi^{ 3 } } \bigg[ \frac{ \chi }{ \lambda }
      \int\limits_{ 0 }^{ + \infty } \frac{ e^{ -2l } \, dl }{ (
        \gamma + l )
        [ ( \gamma + l )^{ 2 } - e^{ -2l } ] } + \\
      &+ \frac{ b_{ 1 } \chi }{ \gamma } \int\limits_{ 0 }^{ +\infty }
      \frac{ l^{ 2 } [ 3 ( \gamma + l )^{ 2 } e^{ -2l } - e^{ -4l } ]
      }{ ( \gamma + l )^{ 2 } [ ( \gamma + l )^{ 2 } - e^{ -2l } ]^{ 2
        } } dl - \frac{ 2 }{ \gamma } \int\limits_{ 0 }^{ +\infty }
      \frac{ l e^{ -2l } \, dl }{ ( \gamma + l ) [ ( \gamma + l )^{ 2
        } - e^{ -2l } ] }
      + \\
      &+ \frac{ 1 }{ \gamma } \int\limits_{ 0 }^{ +\infty } \frac{ ( 1
        - l ) e^{ -2l } } { ( \gamma + l )^{ 2 } - e^{ -2l } } \, dl
      \bigg] + O( \lambda )
    \end{split}
  \end{equation}

  \vspace{-2em}

  $\Ec_{ \textrm{self} }( \lambda )$ is two times energy of single $\delta$
  in vacuum, \\
  $a$ is distance between centers of two bodies, \\
  $\alpha$ is parameter of $\delta$ interaction [Alb88], \\
  $\gamma = \alpha a / 2 \pi^{ 2 } > 1$ is constrain on distance $a$
  [Alb88], \\
  $\chi > 0$ and $b_{ 1 } > 0$ are constants describing properties of
  function $g$.

\end{frame}
% ##################





% ##################
\begin{frame}
  \frametitle{Two delta like systems}


  Presented expression for global energy is dominated by model
  dependent terms which can't be removed. Numerical analysis show that
  force is repulsive. This contradict previous result found in
  literature that predicts universal attractive force [Sca05].

  Also we can show that local energy density have universal limit,
  which exclude possibility of recovering global energy by integrating
  local [FP18], [Fer19].

\end{frame}
% ##################





% ##################
\begin{frame}
  \frametitle{Open problems}


  \begin{itemize}
  \item How to compute electromagnetic Casimir effect for two delta
    like objects?

  \item How this framework relates to alternative algebraic formalism
    of Claudio Dappiaggi, Gabriele Nosari and Nicola Pinamonti
    [DNP16]?

  \item Investigate limit $\lambda \searrow 0$ for single sphere.
  \end{itemize}

\end{frame}
% ##################





% ######################################
\appendix
% ######################################





% ##################
\jagiellonianendslide{Thank you. Any question?}
% ##################





% ##################
\begin{frame}
  \frametitle{Papers on Herdegen's approach}


  \begin{thebibliography}{99}
  \bibitem{HerdegenNononsensCasimirForce2001} Herdegen, A.,
    ``No-nonsens Casimir force'', \emph{Acta Phys. Polon.},
    \textbf{B32}, 55--65, 2001,
    \colorhref{https://arxiv.org/abs/hep-th/0008207}{hep-th/0008207}.

  \bibitem{HerdegenQuantumBackreactionI2005} Herdegen, A., ``Quantum
    backreaction (Casimir) effect~I. What are admissible
    idealizations?'', \emph{Ann. Henri Poincare}, \textbf{6}, 657,
    2005, DOI: 10.1007/s00023-005-0219-7, arXiv:
    \colorhref{https://arxiv.org/abs/hep-th/0412132}{hep-th/0412132}.

  \bibitem{HerdegenQuantumBackreactionII2006} Herdegen, A., ``Quantum
    backreaction (Casimir) effect II. Scalar and electromagnetic
    fields'', \emph{Ann. H. Poincare}, \textbf{7}, 253--301, DOI:
    https://doi.org/10.1007/s00023-005-0249-1, arXiv:
    \colorhref{https://arxiv.org/abs/hep-th/0507023}{hep-th/0507023}.

  \bibitem{HerdegenStopaGlobalVsLocal2010} Herdegen, A., and Stopa,
    M., ``Global vs~local casimir effect'', \emph{Ann. Henri
      Poincare}, \textbf{11}, 1171, 2010, DOI:
    10.1007/s00023-010-0053-4, arXiv:
    \colorhref{https://arxiv.org/abs/1007.2139}{1007.2139}.

  \bibitem{NothingMatter} Ziemian, K., ``Algebraic Approach to Casimir
    Force Between Two \delta-like Potentials'', to be published in
    \emph{Ann. Henri Poincare}.

  \end{thebibliography}

\end{frame}
% ##################





% ##################
\begin{frame}
  \frametitle{Bibliography}


  \begin{thebibliography}{99}
  \bibitem{AlbeverioEtAlSolvableModelsInQM1988} Albeverio, S.,
    Gesztesy, F., H{\o}egh-Krohn, R., and Holden, H., \emph{Solvable
      Models in Quantum Mechanics}, Texts and Monographs in Physics,
    Springer, New York, Berlin, Heidelberg, London, Paris, Tokyo,
    1988.

  \bibitem{BordagMohideenMostepanenkoNewDevelopmentsCasimirEffect2001}
    Bordag., M., Mohideen, U., and Mostepanenko, V. M., ``New
    delepoments in the casimir effect'', \emph{Physics Reports},
    \textbf{353}, 1, 2001, DOI: 10.1016/S0370-1573(01)00015-1, arXiv:
    \colorhref{https://arxiv.org/abs/quant-ph/0106045}{quant-ph/0106045}.

  \bibitem{BratteliRobinsonOperatorAlgebras1987} Bratteli, O.,
    Robinson, D. W., \emph{Operator Algebras and Quantum Statistical
      Mechanics}, 1987, Springer.


  \bibitem{CasimirOnTheAttractionBetween1948} Casimir, H. B. G., ``On
    the attraction between two perfectly conducting plates'',
    \emph{Indag. Math.}, \textbf{10}, 261, 1948, [Kon. Ned. Akad.
    Wetensch. Proc.100N3-4, 61 (1997)].

  \end{thebibliography}

\end{frame}
% ##################





% ##################
\begin{frame}
  \frametitle{Bibliography}


  \begin{thebibliography}{99}
  \bibitem{DappiaggiNosariPianamontiCasimirEffectAlgebraic2016}
    Dappiaggi, C., Nosari, G., and Pinamonti, N., ``The casimir effect
    from the point of view of algebraic quantum field theory'',
    \emph{Math. Phys. Anal. Geom.}, \textbf{19}, 12, 2016, arXiv:
    \colorhref{https://arxiv.org/abs/1412.1409}{1412.1409 [math-ph]}.

  \bibitem{FermiCasimirEnergyAnomalyForPointInteraction2019} Fermi,
    D., ``The Casimir energy anomaly for a point interaction'', arXiv:
    \colorhref{https://arxiv.org/abs/1909.00604}{1909.00604 [math-ph]}.

  \bibitem{FermiPizzaccheroLocalZetaRegularizationAndScalarCasimirEffect2015}
    Fermi, D., Pizzocchero, L., \emph{Local zeta regularization and
      the~scalar Casimir effect: A~general approach based on~integral
      kernels}, World Scientific, Singapore, 2015.

  \bibitem{FermiPizzoccheroLocalCasimirPointImpurity2018} Fermi, D.,
    and Pizzocchero, L., ``Local casimir effet for a scalar field in
    presence of a point impurity'', \emph{Symmetry}, \textbf{10}, 38,
    2018, DOI: 10.3390/sym10020038, arXiv:
    \colorhref{https://arxiv.org/abs/1712.10039}{1712.10039
      [math-ph]}.

  \end{thebibliography}

\end{frame}
% ##################





% ##################
\begin{frame}
  \frametitle{Bibliography}


  \begin{thebibliography}{99}
  \bibitem{ReedSimonMethodsOfModernMathematicalPhysicsVolI1980} Reed,
    M., and Simon, B., \emph{Methods of Modern Mathematical Physics I:
      Functional Analysis}, Academic Press, San Diego, New York,
    Boston, London, Sydney, Tokyo, Toronto, revised and enlarged
    edtion edition, 1980.

  \bibitem{ScardicchioCasimirDynamics2005} Scardicchio, A., ``Casimir
    dynamics: Interactions of surfaces with codimension $> 1$ due to
    quantum fluctuations'', \emph{Phys. Rev. D}, \textbf{72}, 065004,
    2005, DOI: 10.1103/PhysRevD.72.065004, arXiv:
    \href{https://arxiv.org/abs/hep-th/0503170}{hep-th/0503170}.

  \bibitem{SpreaficoZerbiniFiniteTemperatureQFTOnNoncompactDomains2009}
    Spreafico, M., and Zerbini, S., ``Finite temperature quantum field
    theory on noncompact domains and application to delta
    interactions'', \emph{Rep. Math. Phys}, \textbf{63}, 163, 2009,
    arXiv: \colorhref{https://arxiv.org/abs/0708.4109}{0708.4109}.

  \end{thebibliography}

\end{frame}
% ##################










% ##################
\jagiellonianendslide{If you have more questions, please write to me
  at kziemianfvt@gmail.com.}
% ##################








































% ####################################################################
% ####################################################################
% Bibliografia
\bibliographystyle{alpha} \bibliography{Bibliography}{}


% ############################

% Koniec dokumentu
\end{document}
