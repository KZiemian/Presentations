% ---------------------------------------------------------------------
% Basic configuration of Beamera and Jagiellonian
% ---------------------------------------------------------------------
\RequirePackage[l2tabu, orthodox]{nag}



\ifx\PresentationStyle\notset
  \def\PresentationStyle{dark}
\fi



\documentclass[10pt,t]{beamer}
\mode<presentation>
\usetheme[style=\PresentationStyle]{jagiellonian}



% ---------------------------------------
% Configuration files of Jagiellonian loceted in catalog preambule
% ---------------------------------------
% Configuration for polish language
% Need description
\usepackage[english]{babel}





% % ------------------------------
% % Better support of polish chars in technical parts of PDF
% % ------------------------------
% \hypersetup{pdfencoding=auto,psdextra}

% Package "textpos" give as enviroment "textblock" which is very usefull in
% arranging text on slides.

% This is standard configuration of "textpos"
\usepackage[overlay,absolute]{textpos}

% If you need to see bounds of "textblock's" comment line above and uncomment
% one below.

% Caution! When showboxes option is on significant ammunt of space is add
% to the top of textblock and as such, everyting put in them gone down.
% We need to check how to remove this bug.

% \usepackage[showboxes,overlay,absolute]{textpos}



% Setting scale length for package "textpos"
\setlength{\TPHorizModule}{10mm}
\setlength{\TPVertModule}{\TPHorizModule}


% ---------------------------------------
% TikZ
% ---------------------------------------
% Importing TikZ libraries
\usetikzlibrary{arrows.meta}
\usetikzlibrary{positioning}





% % Configuration package "bm" that need for making bold symbols
% \newcommand{\bmmax}{0}
% \newcommand{\hmmax}{0}
% \usepackage{bm}




% ---------------------------------------
% Packages for scientific texts
% ---------------------------------------
% \let\lll\undefined  % Sometimes you must use this line to allow
% "amsmath" package to works with packages with packages for polish
% languge imported
% /preambul/LanguageSettings/JagiellonianPolishLanguageSettings.tex.
% This comments (probably) removes polish letter Ł.
\usepackage{amsmath}  % Packages from American Mathematical Society (AMS)
\usepackage{amssymb}
\usepackage{amscd}
\usepackage{amsthm}
\usepackage{siunitx}  % Package for typsetting SI units.
\usepackage{upgreek}  % Better looking greek letters.
% Example of using upgreek: pi = \uppi


\usepackage{calrsfs}  % Zmienia czcionkę kaligraficzną w \mathcal
% na ładniejszą. Może w innych miejscach robi to samo, ale o tym nic
% nie wiem.










% ---------------------------------------
% Packages written for lectures "Geometria 3D dla twórców gier wideo"
% ---------------------------------------
% \usepackage{./ProgramowanieSymulacjiFizykiPaczki/ProgramowanieSymulacjiFizyki}
% \usepackage{./ProgramowanieSymulacjiFizykiPaczki/ProgramowanieSymulacjiFizykiIndeksy}
% \usepackage{./ProgramowanieSymulacjiFizykiPaczki/ProgramowanieSymulacjiFizykiTikZStyle}





% !!!!!!!!!!!!!!!!!!!!!!!!!!!!!!
% !!!!!!!!!!!!!!!!!!!!!!!!!!!!!!
% EVIL STUFF
\if\JUlogotitle1
\edef\LogoJUPath{LogoJU_\JUlogoLang/LogoJU_\JUlogoShape_\JUlogoColor.pdf}
\titlegraphic{\hfill\includegraphics[scale=0.22]
{./JagiellonianPictures/\LogoJUPath}}
\fi
% ---------------------------------------
% Commands for handling colors
% ---------------------------------------


% Command for setting normal text color for some text in math modestyle
% Text color depend on used style of Jagiellonian

% Beamer version of command
\newcommand{\TextWithNormalTextColor}[1]{%
  {\color{jNormalTextFGColor}
    \setbeamercolor{math text}{fg=jNormalTextFGColor} {#1}}
}

% Article and similar classes version of command
% \newcommand{\TextWithNormalTextColor}[1]{%
%   {\color{jNormalTextsFGColor} {#1}}
% }



% Beamer version of command
\newcommand{\NormalTextInMathMode}[1]{%
  {\color{jNormalTextFGColor}
    \setbeamercolor{math text}{fg=jNormalTextFGColor} \text{#1}}
}


% Article and similar classes version of command
% \newcommand{\NormalTextInMathMode}[1]{%
%   {\color{jNormalTextsFGColor} \text{#1}}
% }




% Command that sets color of some mathematical text to the same color
% that has normal text in header (?)

% Beamer version of the command
\newcommand{\MathTextFrametitleFGColor}[1]{%
  {\color{jFrametitleFGColor}
    \setbeamercolor{math text}{fg=jFrametitleFGColor} #1}
}

% Article and similar classes version of the command
% \newcommand{\MathTextWhiteColor}[1]{{\color{jFrametitleFGColor} #1}}





% Command for setting color of alert text for some text in math modestyle

% Beamer version of the command
\newcommand{\MathTextAlertColor}[1]{%
  {\color{jOrange} \setbeamercolor{math text}{fg=jOrange} #1}
}

% Article and similar classes version of the command
% \newcommand{\MathTextAlertColor}[1]{{\color{jOrange} #1}}





% Command that allow you to sets chosen color as the color of some text into
% math mode. Due to some nuances in the way that Beamer handle colors
% it not work in all cases. We hope that in the future we will improve it.

% Beamer version of the command
\newcommand{\SetMathTextsColor}[2]{%
  {\color{#1} \setbeamercolor{math text}{fg=#1} #2}
}


% Article and similar classes version of the command
% \newcommand{\SetMathTextColor}[2]{{\color{#1} #2}}










% ---------------------------------------
% Commands for setting background pictures for some slides
% ---------------------------------------
\newcommand{\TitleBackgroundPicture}
{./PresentationPictures/CommonPictures/Cute_dragon_BG_dark.png}
\newcommand{\SectionBackgroundPicture}
{./PresentationPictures/CommonPictures/Cute_dragon_small_BG_light.png}



\newcommand{\TitleSlideWithPicture}{
  \begingroup

  \usebackgroundtemplate{ % \hspace*{-11.5em}
    \includegraphics[height=\paperheight]{\TitleBackgroundPicture}}

  \maketitle

  \endgroup
}





\newcommand{\SectionSlideWithPicture}[1]{%
  \begingroup

  \usebackgroundtemplate{ % \hspace*{-11.5em}
    \includegraphics[height=\paperheight]{\SectionBackgroundPicture}}

  \setbeamercolor{titlelike}{fg=normal text.fg}

  \section{#1}

  \endgroup
}





\newcommand{\EndingSlide}[1]{%
  \begin{frame}[standout]

    \begingroup

    \color{jFrametitleFGColor}

    #1

    \endgroup

  \end{frame}
}










% ------------------------------------------------------
% BibLaTeX
% ------------------------------------------------------
% Package biblatex, with biber as its backend, allow us to handle
% bibliography entries that use Unicode symbols outside ASCII.
\usepackage[
language=polish,
backend=biber,
style=alphabetic,
url=false,
eprint=true,
]{biblatex}

\addbibresource{Stochastic-quantization-and-singular-stochastic-equations-Bibliography.bib}





% ------------------------------------------------------
% Wonderful package PGF/TikZ
% ------------------------------------------------------

% Node and pics for drawing charts
% \usepackage{./Local-packages/PGF-TikZ-Chart-nodes-and-pics}

% Styles for arrows
% \usepackage{./Local-packages/PGF-TikZ-Arrows-styles}

% Pic for drawing functions
% \usepackage{./Local-packages/PGF-TikZ-Functions-pics}






% ------------------------------------------------------
% Local packages
% ------------------------------------------------------
% Special configuration for this particular presentation
\usepackage{./Local-packages/local-settings}

% Package containing various command useful for working with a text
\usepackage{./Local-packages/general-commands}

% Package containing commands and other code useful for working with
% mathematical text
\usepackage{./Local-packages/math-commands}










% ---------------------------------------------------------------------
\title{Stochastic quantization and singular stochastic
  equations}

\author{Kamil Ziemian \\
  \texttt{kziemianfvt@gmail.com}}


% \institute{}

\date[28 XI 2025]{Seminar of Field Theory Department \\
  28 XI 2025}
% --------------------------------------------------------------------










% ####################################################################
% Początek dokumentu
\begin{document}
% ####################################################################





% ######################################
% Number of chars: 33k+,
% Text is adjusted to the left and words are broken at the end of the line.
\RaggedRight
% ######################################






% ######################################
\maketitle % Tytuł całego tekstu
% ######################################





% ######################################
\begin{frame}
  \frametitle{Table of contents}


  % Put here table of contents.
  \tableofcontents

\end{frame}
% ######################################





% ######################################
\section{Constructive QFT and its problems}
% ######################################





% ##################
\begin{frame}
  \frametitle{Aims of our talk}


  The aim of this talk is twofold. First, we want to justify and give
  overview of procedure called stochastic quantization, that allow us to
  define rigorously some models of quantum fields in $1 + 1$ and $2 + 1$
  dimensions with some promise for future development. This procedure force
  us to deal with \textbf{singular stochastic partial differential
    equations} (SsPDEs for short), which are as complicated to analyze as
  their name is long. Theory of SsPDE is now the research topic of
  Paweł Duch, who was a member of Field Theory Department in the years
  $2012\text{--}2018$. Explaining some of his recent results is the second
  aim of our talk.

  All these topics are very complicated and involve cutting edge
  mathematical research, so we cannot explain them in all the glorious
  details. Our talk is rather a very simple overview of this active field
  of research. Also, you can get PDF of this presentation, writing to us
  e-mail at \texttt{kziemianfvt@gmail.com}.

\end{frame}
% ##################





% ##################
\begin{frame}
  \frametitle{Mathematical physicists always spoils the fun}


  We want to define a simple quantum field theory like $\Phi^{ 4 }$
  in~$d$~dimensional Minkowski space. So we wrote the lagrangian
  \begin{equation}
    \label{eq:Stochastic-quantization-ETC-01}
    \Lcal( x ) =
    \frac{ 1 }{ 2 } \partial_{ \mu } \Phi( x ) \partial^{ \mu } \Phi( x ) - m^{ 2 } \Phi^{ 2 }( x )
    - \frac{ \lambda }{ 4! } \Phi^{ 4 }( x ).
  \end{equation}
  where $x = [ t, x_{ 1 }, x_{ 2 }, \ldots, x_{ d - 1 } ]$ and
  \begin{equation}
    \label{eq:Stochastic-quantization-ETC-02}
    x \cdot x = t^{ \, 2 } - x_{ 1 }^{ \, 2 } - x_{ 2 }^{ \, 2 } -
    x_{ 3 }^{ \, 2 } - \ldots - x_{ d - 1 }^{ \, 2 }.
  \end{equation}
  Then we compute action
  \begin{equation}
    \label{eq:Stochastic-quantization-ETC-03}
    S[ \Phi( x ) ] =
    \int\limits_{ [ 0, 1 ] \times \Rbb^{ d } } d^{ d + 1 }x \, \Lcal[ \Phi( x ) ],
  \end{equation}
  and partition function
  \begin{equation}
    \label{eq:Stochastic-quantization-ETC-04}
    Z[ \, J( x ) ] =
    \int [ \Dcal \Phi( x ) ] \exp\Big( i S[ \phi( x ) ] + i \int d^{ d + 1 }x \, J( x )
    \Phi( x ) \Big),
  \end{equation}
  and we can do physics now. But, mathematical physicists like to spoil
  all the fun, so they point out that measure $[ \Dcal \Phi( x ) ]$ doesn't
  exist.

\end{frame}
% ##################





% ##################
\begin{frame}
  \frametitle{Why \MathTextFrametitleFGColor{$[ \Dcal \Phi( x ) ]$}
    doesn't exists?}


  What is the problem with $[ \Dcal \Phi( x ) ]$? First we need to notice,
  that it should be a~Lebesgue's measure for, let say, the infinite
  dimensional space of functions, like $L^{ 2 }( \Rbb^{ d } )$. In the
  case~of $\Rbb^{ d }$ the crucial property of Lebesgue's measure is
  translational invariance. If we have measurable set $A \subset \Rbb^{ d }$, then
  \begin{equation}
    \label{eq:Stochastic-quantization-ETC-05}
    \forall \, \vecx \in \Rbb^{ d }, \quad
    \text{vol}( A + \vecx ) = \text{vol}( A ).
  \end{equation}
  More precisely, we have the following theorem. \\
  \textit{Let $G$ be a locally compact, topological group. There
    \alert{exists} translational invariant, positive measure on the $G$,
    which is unique, up to multiplication by constant $c > 0$.}

  Existence of Lebesgue's measure is closely tied to its translational
  invariance, but in infinite dimensional space, we cannot have
  translation invariant measure. Simple argument for that goes as follows.

\end{frame}
% ##################





% ##################
\begin{frame}
  \frametitle{Finite dimensional spaces are easy}


  Let $K( 0, \frac{ 1 }{ 4 } )$ will be ball in our space and assume that
  it has non-zero finite measure
  \begin{equation}
    \label{eq:Stochastic-quantization-ETC-06}
    0 < \mu\big( K( 0, \tfrac{ 1 }{ 4 } ) \big) < +\infty.
  \end{equation}
  Since $K( 0, \frac{ 1 }{ 4 } ) \subset K( 0, 1 )$, it implies
  \begin{equation}
    \label{eq:Stochastic-quantization-ETC-07}
    \mu\big( K( 0, \tfrac{ 1 }{ 4 } ) \big) \leq \mu( K( 0, 1 ) ).
  \end{equation}
  In three dimensional space I can take unite ball
  $K( ( 0, 0, 0 ), \tfrac{ 1 }{ 4 } )$ and create six disjoint balls
  contained inside $K( ( 0, 0, 0 ), 1 )$ by shifting this first ball
  by vectors $[ \frac{ 1 }{ 2 }, 0, 0 ]$, $[ -\frac{ 1 }{ 2 }, 0, 0 ]$,
  $[ 0, \frac{ 1 }{ 2 }, 0 ]$, $[ 0, -\frac{ 1 }{ 2 }, 0 ]$, etc. If our
  measure is translational invariant we arrive at result
  \begin{equation}
    \label{eq:Stochastic-quantization-ETC-08}
    6 \, \mu\big( K( ( 0, 0, 0), \tfrac{ 1 }{ 4 } ) \big) \leq
    \mu\big( K( ( 0, 0, 0 ), 1 ) \big).
  \end{equation}
  Which is a perfectly fine. But in infinite dimensional space, we have
  a~serious problem.

\end{frame}
% ##################





% ##################
\begin{frame}
  \frametitle{Infinite dimensional spaces are hard}


  In infinite dimensional space like $L^{ 2 }( \Rbb^{ 3 } )$, we have
  infinitely many orthogonal vectors with the length $\frac{ 1 }{ 2 }$
  to perform such shifts. As a~result we have infinitely many disjoint
  balls with radius $\tfrac{ 1 }{ 4 }$ that are contained
  inside unit ball $K( 0, 1 )$:
  \begin{equation}
    \label{eq:Stochastic-quantization-ETC-09}
    K( x_{ i }, \tfrac{ 1 }{ 4 } ) \subset K( 0, 1 ), \quad i \in I,
  \end{equation}
  where $I$ is some infinite set.

  Assuming translational invariance of our measure, we get
  \begin{equation}
    \label{eq:Stochastic-quantization-ETC-10}
    \infty \, \mu\big( K( 0, \tfrac{ 1 }{ 4 } ) \big) \leq \mu\big( K( 0, 1 ) \big).
  \end{equation}
  This can only be true in two situations. First
  $\mu\big( K( 0, \tfrac{ 1 }{ 4 } ) \big) = 0$, second
  $\mu\big( K( 0, 1 ) \big) = +\infty$. In first case we have contradiction
  with our assumption $\mu\big( K( 0, \frac{ 1 }{ 4 } ) \big) \neq 0$, so this
  is nonsense. In the second case, we arrived at quite reasonable, but
  useless for all application measure, in which the unit ball in
  infinite dimensional space has infinite volume.

\end{frame}
% ##################





% ##################
\begin{frame}
  \frametitle{Life is hard, but we don't give up}


  We thus must abandon search for such translational measures, at least
  in Hilbert spaces, since our reasoning was based on having orthogonal
  vectors. More advanced analysis show us, that we should look for
  measures not at the space of functions, but on the spaces of
  distributions, like $\Dcal'( \Rbb^{ n } )$ and $\Scal'( \Rbb^{ n } )$.
  But, we can't go into details.

  This task is very hard, but we don't give up. It will be important
  for the future, that because we know already how to solve the problem of
  \alert{free} field, we can construct measure for \alert{free}
  evolution~of such \alert{free} field. This measure is a functional
  Guassian measure.

  Since our task of finding such measures is very hard, we need to simplify
  it a bit. First of all in our naive measure
  \begin{equation}
    \label{eq:Stochastic-quantization-ETC-11}
    [ \Dcal \Phi( x ) ] \exp\Big( i S[ \Phi( x ) ] \Big),
  \end{equation}
  we have an oscillating function $\exp( i S )$, which of course makes
  such an expression harder to work with.

\end{frame}
% ##################





% ##################
\begin{frame}
  \frametitle{Euclidian QFT comes for rescue}


  As textbooks of QFT tell us, we should make Wick rotation: $t = i \tau$.
  Which means that our measure is now
  \begin{equation}
    \label{eq:Stochastic-quantization-ETC-12}
    [ \Dcal \Phi_{ \text{Euc} }( x_{ \text{Euc} } ) ]
    \exp\Big( -\!S_{ \text{Euc} }[ \Phi_{ \text{Euc} }( x ) ] \Big)
  \end{equation}
  and scalar product of our vectors
  $x_{ \text{Euc} } = [ \tau, x_{ 1 }, x_{ 2 }, \ldots, x_{ d - 1 } ]$ is, almost,
  Euclidian
  \begin{equation}
    \label{eq:Stochastic-quantization-ETC-13}
    -x_{ \text{Euc} } \cdot x_{ \text{Euc} } =
    \tau^{ 2 } + x_{ 1 }^{ \, 2 } + x_{ 2 }^{ \, 2 } + \ldots + x_{ d - 1 }^{ \, 2 }.
  \end{equation}
  This is why we call it \colorhref{}{Euclidian Quantum Field Theory}
  (EQFT for short). Because naive measure
  \eqref{eq:Stochastic-quantization-ETC-12} is much nicer that
  previous one with $\exp( i S )$ part, a large portion of successes of
  constructive QFT was connected to this approach
  \parencite{Summers-Prespective-on-Constructive-ETC-Pub-2016}.

  We need to mention three more important things. First, you can think
  about EQFT as a stochastic process in the space of distributions. Hard to
  imagine, but you can live with that after a~while.

\end{frame}
% ##################





% ##################
\begin{frame}
  \frametitle{Euclidian QFT comes for rescue}


  Second, Guassian measure of free field in Minkowski space becomes
  Gaussian white noise measure in EQFT. Gaussian white noise, which will
  be denoted as $\eta( x_{ \text{Euc} } )$, is a~very important concept for
  SsPDEs.

  From the point of view of mathematical physics Wick rotation $t = i \tau$ is
  just a bunch of lines on the slide, without any true meaning.
  But \colorhref{https://en.wikipedia.org/wiki/Konrad\_Osterwalder}{Konrad
    Osterwalder} and
  \colorhref{https://en.wikipedia.org/wiki/Robert\_Schrader}{Robert
    Schrader} proved very important reconstruction theorem, which
  states that Euclidean quantum field in $d$~dimensions that fulfils
  given set of requirements, can be transformed into a~quantum field in $d$
  dimensional Minkowski space by analytical continuation
  \parencite{Osterwalder-Schrader-Axioms-for-Euclidean-Part-I-ETC-Pub-1973},
  \parencite{Osterwalder-Schrader-Axioms-for-Euclidean-Part-II-ETC-Pub-1975}.

  This set of requirments is called
  \colorhref{https://en.wikipedia.org/wiki/Schwinger\_function}
  {Osterwalder-Schrader axioms}. The axioms are: temperedness, linear
  growth condition, Euclidian covariance, positivity, symmetry
  and cluster property, but explaining them is far beyond the scope of
  this~talk.

\end{frame}
% ##################










% ######################################
\section{Stochastic quantization}
% ######################################



% ##################
\begin{frame}
  \frametitle{Why stochastic quantization?}


  Stochastic quntization was proposed by
  \colorhref{https://en.wikipedia.org/wiki/Giorgio\_Parisi}
  {Giorgio Parisi} and Yong-shi Wu in the context
  of gauge fields, to avoid the problem of Gribov ambiguities, so we
  need to speak a few words about it. Consider gauge field
  $A_{ \text{Euc}, \, \mu }( x_{ \text{Euc} } )$ and gauge function
  $u( x_{ \text{Euc} } )$. We know, that physical situations described by
  $A_{ \text{Euc}, \, \mu }( x_{ \text{Euc} } )$ and
  \begin{equation}
    \label{eq:Stochastic-quantization-ETC-14}
    \begin{split}
      \widetilde{ A }_{ \text{Euc}, \, \mu }( x_{ \text{Euc} } )
      &=
        u( x_{ \text{Euc} } )^{ -1 } A_{ \text{Euc}, \, \mu }( x_{ \text{Euc} } )
        u( x_{ \text{Euc} } ) - \\
      &\quad - \frac{i}{ g } u( x_{ \text{Euc} } )^{ -1 }
        \partial_{ \mu } u( x_{ \text{Euc} } ),
    \end{split}
  \end{equation}
  where $g$ is coupling constant, are the same. So even naive expression
  \begin{equation}
    \label{eq:Stochastic-quantization-ETC-15}
    \int [ \Dcal A_{ \text{Euc}, \, \mu }( x_{ \text{Euc} } ) ]
    \exp\Big( -\!S[ A_{ \text{Euc}, \, \mu }( x_{ \text{Euc} } ) ] \Big),
  \end{equation}
  or its Minkowski analogue are physically incorrect. Such integrals
  counts the same physical situation many times over, because we integrate
  over $A_{ \text{Euc}, \, \mu }( x_{ \text{Euc} } )$ and all possible
  $\widetilde{ A }_{ \text{Euc}, \, \mu }( x_{ \text{Euc} } )$ connected to it by
  gauge transformation.

\end{frame}
% ##################





% ##################
\begin{frame}
  \frametitle{Gribov ambiguities}


  Standard way of solving that is gauge fixing. We choose function $G$ over
  the space of fields and integrate only over fields that fulfill condition
  $G\big( A_{ \text{Euc}, \, \mu }( x_{ \text{Euc} } ) \big) = 0$. Such a
  procedure is well known from Clerk Maxwell electrodynamics. But as was
  first observed
  by \colorhref{https://en.wikipedia.org/wiki/Vladimir\_Gribov}{Vladimir
    Gribov}, the example of \alert{abelian} electrodynamics is highly
  misleading. In the case of theory with non-abelian gauge groups there are
  very deep topological reasons that prevent us from making ``resonable''
  gauge fixing
  \parencite{Gribov-Quantization-of-non-Abelian-gauge-theories-Pub-1978},
  \parencite{Singer-Some-Remarks-on-the-Gribov-Ambiguity-Pub-1978}. As a
  result even after imposing the condition
  $G\big( A_{ \text{Euc}, \, \mu }( x_{ \text{Euc} } ) \big) = 0$ we can have
  overcounting.

  Parisi and Wu propose to solve that problem, by replacing gauge fixing
  by analysis of stochastic process in the space of \alert{all} fields
  \parencite{Parisi-Wu-Perturbation-Theory-Without-Gauge-Fixing-Pub-1981},
  \parencite{Damgaard-Huffel-Stochastic-quantization-Pub-1987},
  \parencite{Gubinelli-What-is-stochastic-quanstization-ETC-Ver-2022}.
  Simplifying their argument, you \alert{don't} fixe gauge, but evolve
  field in the totally fake time $t_{ \text{F} }$ as a~``diffusion'' process
  in the space of fields in such a~way, that unphysical contribution from
  equivalent fields cancels out.

\end{frame}
% ##################





% ##################
\begin{frame}
  \frametitle{Why must life be so complicated?}


  To be more concrete, we start from field
  $A_{ \text{Euc}, \, \mu }( x_{ \text{Euc} } )$ and add to it arguments
  totally fake time $t_{ \text{F} }$ to get
  $A_{ \text{Euc}, \, \mu }( t_{ \text{F} }, x_{ \text{Euc} } )$. We now evolve
  this field in fake time according to Langevine equation
  \begin{equation}
    \label{eq:Stochastic-quantization-ETC-16}
    \frac{ \partial A_{ \text{Euc}, \, \mu }( t_{ \text{F} }, x_{ \text{Euc} } ) }
    { \partial t_{ \text{F} } } =
    -\frac{ \delta S_{ \text{Euc} } }{ \delta A_{ \text{Euc}, \, \mu } } +
    \eta( t_{ \text{F} }, x_{ \text{Euc} } ),
  \end{equation}
  where $\eta( t_{ \text{F} }, x_{ \text{Euc} } )$ is Gaussian white noise
  in variables $[ t_{ \text{F} }, x_{ \text{Euc} } ]$. Parisi and Wu postulate
  that
  \begin{equation}
    \label{eq:Stochastic-quantization-ETC-17}
    \begin{split}
      &\lim_{ t_{ \text{F} } \nearrow \infty }
        \int [ \Dcal \eta( x_{ \text{Euc} } ) ]
        \exp\!\bigg( -\int d^{ d }x \, dt_{ F } \,
        \eta( t_{ \text{F} }, x_{ \text{Euc} } )^{ 2 } \bigg) \times \\
      & \hspace{2em}
        \times A_{ \text{Euc}, \, \mu }( t_{ \text{F} }, x_{ \text{Euc} } ) =
        \langle A_{ \text{Euc}, \, \mu }( x_{ \text{Euc} } ) \rangle_{ \text{phys} }.
    \end{split}
  \end{equation}
  The expression $[ \Dcal \eta( x_{ \text{Euc} } ) ]
  \exp\!\big( -\int d^{ d }x d t_{ F } \,
  \eta( t_{ \text{F} }, x_{ \text{Euc} } )^{ 2 } \big)$ is mathematicaly well
  defined and on the right hand side we have a physical correct
  exceptetion value of the physical field
  $A_{ \text{Euc}, \, \mu }( x_{ \text{Euc} } )$.

\end{frame}
% ##################





% ##################
\begin{frame}
  \frametitle{Singular stochastic PDEs (SsPDEs)}


  A~lot of effort was put into making these statements more rigorous.
  The main problem is that $\eta( t_{ \text{F} }, x_{ \text{Euc} } )$ is a very
  singular object, you can think about it as a randomly valued distribution.
  As a result field $A_{ \text{Euc}, \, \mu }( t_{ \text{F} }, x_{ \text{Euc} } )$
  also need to be a~distribution:
  \begin{equation}
    \label{eq:Stochastic-quantization-ETC-18}
    \frac{ \partial A_{ \text{Euc}, \, \mu }( t_{ \text{F} }, x_{ \text{Euc} } ) }
    { \partial t_{ \text{F} } } =
    -\frac{ \delta S_{ \text{Euc} } }{ \delta A_{ \text{Euc}, \, \mu } } +
    \eta( t_{ \text{F} }, x_{ \text{Euc} } ),
  \end{equation}
  If we think about free theory then expression
  $-\delta S_{ \text{Euc} } / \delta A_{ \text{Euc}, \, \mu }$ is linear in distribution
  $A_{ \text{Euc}, \, \mu }( t_{ \text{F} }, x_{ \text{Euc} } )$, so life is great.
  But we know that if we want to have interacting fields, we will need
  add least quadratic term in them, something like
  $\psi_{ \text{Euc} }( t_{ \text{F} }, x_{ \text{Euc} } )
  A_{ \text{Euc}, \, \mu }( t_{ \text{F} }, x_{ \text{Euc} } )$. And we know that
  in general, products of distributions have no sense.

  Such stochastic PDEs that are ill posed because of products of
  distributions are called \alert{singular stochastics PDEs} (SsPDEs)
  and are as complicated as their name is long.

\end{frame}
% ##################










% ######################################
\section{SsPDEs and works of Paweł Duch}
% ######################################


% ##################
\begin{frame}
  \frametitle{Making sense of SsPDEs}


  We encounter SsPDEs in various fields of science, so we have good
  reasons to define them mathematically. For us especially important are
  works of Martin Hairer who won
  \colorhref{https://en.wikipedia.org/wiki/Fields\_Medal}{Fields Medal} in
  $2014$ ``\textit{For his outstanding contributions to the theory of
    stochastic partial differential equations, and in particular for
    the creation of a theory of regularity structures for such
    equations}''.

  Hairer and his collaborators first pay attenntion to problems like
  defining \colorhref{https://en.wikipedia.org/wiki/Kardar\%E2\%80\%93Parisi\%E2\%80\%93Zhang\_equation}
  {KPZ equation} (named after Mehran Kadar, Giorgio Paris again ;) and
  Yi-Cheng Zhang), but since they put more attention to QFT
  \parencite{Hairer-Yang-Mills-and-the-Mass-Gap-Ver-2025}. In $2024$
  he together with Ajay Changra, Ilya Chevyrev, Hao Shen published
  article \colorhref{https://arxiv.org/abs/2201.03487}
  {\textit{Stochastic quantisation~of Yang-Mills-Higgs in~3D}}
  \parencite{Chandra-et-al-Stochastic-quantisation-of-Yang-Mills-ETC-Pub-2024}. In August of $2025$ Hairer, Paweł Duch, Jaeyun Yi and Wenhao Zhao
  published on arXiv paper
  \colorhref{https://arxiv.org/abs/2508.07776}
  {\textit{Ergodicity of infinite volume $\Phi_{ 3 }^{ \, 4 }$ at high
      temperature}}
  \parencite{Duch-Hairer-Yi-Zhao-Ergodicity-of-infinite-volume-ETC-Pub-2025}.

\end{frame}
% ##################





% ##################
\begin{frame}
  \frametitle{Paweł Duch's works on flow equation approach}


  Current works of Paweł Duch focus in large part on flow equation approach
  to SsPDES
  \parencite{Duch-Flow-equation-approach-to-singular-stochastic-ETC-Pub-2025},
  \parencite{Duch-Lecture-notes-on-flow-equation-approach-ETC-Pub-2025}.
  It was inspired by renormalization group theory of Kenneth Wilson and
  works of Antti Kupiainen, who adapted it for the~case~of stochastic PDEs
  \parencite{Wilson-Renormalization-Group-and-Critical-ETC-Part-I-Pub-1971},
  \parencite{Kupiainen-Renormalization-Group-and-Stochastic-PDEs-Pub-2016}.
  Because the theory of SsPDEs is very complex, for the~purpose of this
  talk, we need to simplify a lot and introduce some notation. This notation
  may not be the most precise, but useful in sketching the~problem.

  We first fix dimension $d$ and denote $d$-dimensional torus by $T^{ d }$.
  Let $V$ be map transforming space of continuous functions on $T^{ d }$
  into itself:
  \begin{equation}
    \label{eq:Stochastic-quantization-ETC-19}
    V: \Ccal( T^{ d } ) \to \Ccal( T^{ d } ).
  \end{equation}
  The value of $V$ on function $\varphi \in \Ccal( T^{ d } )$ we will be denoting
  by $V[ \varphi ]$. The derivative of $k$~order of $V$ at $\varphi$ in the direction
  of function $\psi$ is defined as
  \begin{equation}
    \label{eq:Stochastic-quantization-ETC-20}
    D^{ k } V[ \varphi ] \cdot \psi^{ \otimes k } :=
    \partial_{ \tau }^{ \, k } \, V[ \varphi + \tau \psi ] \big|_{ \tau = 0 }.
  \end{equation}

\end{frame}
% ##################





% ##################
\begin{frame}
  \frametitle{Toy example}


  For the~sake of simplicity we will work with an equation without
  time dynamics, which is nonetheless inspired by $\Phi_{ d }^{ \, 4 }$
  theory.
  \begin{equation}
    \label{eq:Stochastic-quantization-ETC-21}
    ( 1 - \Delta )^{ \sigma / 2 } \Phi = \eta + \lambda \Phi^{ 3 } - \infty \, \Phi,
  \end{equation}
  where $\Phi \in \Dcal'( \Rbb^{ d } )$. Parameter $\sigma \in ( d / 3, d / 2 ]$ for
  technical reasons. As such $( 1 - \Delta )^{ \sigma / 2 }$ will in general be a well
  defined pseudodifferential operator. The $\eta$ is Gaussian white noise and
  $\infty \, \Phi$ symbolize, that due to presence of $\eta$ we need to ``subtract
  infinity'' from the~ill defined product of distributions $\Phi^{ 3 }$.

  In first step we regularize white noise introducing family of smooth
  functions $\eta_{ \kappa }$, $\kappa \in ( 0, 1 ]$, with property
  \begin{equation}
    \label{eq:Stochastic-quantization-ETC-22}
    \lim_{ \kappa \searrow 0 } \eta_{ \kappa } = \eta.
  \end{equation}
  Now $\Phi_{ \kappa }$ is a solution of the equation
  \eqref{eq:Stochastic-quantization-ETC-21} with regularized noise.

\end{frame}
% ##################





% ##################
\begin{frame}
  \frametitle{Toy example}


  We now need to rewrite our problem as
  \begin{subequations}
    \begin{align}
      \label{eq:Stochastic-quantization-ETC-23-A}
      &( 1 - \Delta )^{ \sigma / 2 } \Phi_{ \kappa } = F_{ \kappa }[ \Phi_{ \kappa } ], \\
      &F_{ \kappa }[ \varphi ] =
        \eta_{ \kappa } + \lambda \varphi_{ \kappa }^{ \, 3 } +
        \sum_{ i = 1 }^{ N( \sigma ) } \lambda^{ i } c_{ \kappa }^{ ( i ) } \varphi,
    \end{align}
  \end{subequations}
  with $N( \sigma ) = \lfloor \sigma / ( 3 \sigma - d ) \rfloor$. We define $F_{ \kappa }$ as
  $F_{ \kappa } : \Ccal( T^{ d } ) \to \Ccal( T^{ d } )$, by imposing boundary
  conditions. The last expression in $F_{ \kappa }$ describe
  standard counter terms that are needed when we take limit $\kappa \searrow 0$.
  Their number grows to infinity as $\sigma \searrow d / 3$.

  Let now $G \in L^{ 1 }( \Rbb^{ d } )$ be fundamental solution of
  $( 1 - \Delta )^{ \sigma / 2 }$:
  \begin{equation}
    \label{eq:Stochastic-quantization-ETC-24}
    ( 1 - \Delta )^{ \sigma / 2 } G( x ) = \delta( x ).
  \end{equation}
  Equation \eqref{eq:Stochastic-quantization-ETC-23-A} is now equivalent to
  \begin{equation}
    \label{eq:Stochastic-quantization-ETC-25}
    \Phi_{ \kappa } = G * F_{ \kappa }[ \Phi_{ \kappa } ].
  \end{equation}

\end{frame}
% ##################





% ##################
\begin{frame}
  \frametitle{Coarse-grained process}


  Next step is to find regularized families of $G_{ \mu }$ and $F_{ \kappa, \, \mu }$,
  such that we have $\mu \in [ 0, 1 ]$, $G_{ \mu } \in L^{ 1 }( \Rbb^{ d } )$,
  $G_{ 0 }$ is equal to fundamental solution of $( 1 - \Delta )^{ \sigma / 2 }$, namely
  $G_{ 0 } = G$, $G_{ 1 } = 0$ and dependents on $\mu$ in sufficient regular
  ways. About $F_{ \kappa, \, \mu }$ we assume that for all $\kappa$ and $\mu$ it is
  polynomial in the fields,
  $F_{ \kappa, \, \mu } : \Ccal( T^{ d } ) \to \Ccal( T^{ d } )$,
  $F_{ \kappa, \, 0 } = F_{ \kappa }$ and depends on both $\kappa$ and $\mu$ in regular way.

  Finding such a family $G_{ \mu }$ is not a problem, the main challenge is to
  find appropriate forms of $F_{ \kappa, \, \mu }$. This unfortunately requires a
  lot of work. In first step we define $\Phi_{ \kappa, \, \mu }$ as
  \begin{equation}
    \label{eq:Stochastic-quantization-ETC-26}
    \Phi_{ \kappa, \, \mu } = G_{ \mu } * F_{ \kappa }[ \Phi_{ \kappa } ].
  \end{equation}
  It is basically taking some Wilsonian average of $\Phi_{ \kappa }$, so
  we call it \textbf{coarse-grained process}.

\end{frame}
% ##################





% ##################
\begin{frame}
  \frametitle{Effective equations}


  We now define \textbf{remainder} $\omega_{ \kappa, \, \mu }$ by equation
  \begin{equation}
    \label{eq:Stochastic-quantization-ETC-27}
    F_{ \kappa }[ \Phi_{ \kappa } ] = F_{ \kappa, \ \mu }[ \Phi_{ \kappa, \, \mu } ] + \omega_{ \kappa, \, \mu }.
  \end{equation}
  As if life wasn't hard enough, we need to define yet another complicated
  function.
  \begin{equation}
    \label{eq:Stochastic-quantization-ETC-28}
    H_{ \kappa, \, \mu }[ \varphi ] =
    \partial_{ \mu } F_{ \kappa, \, \mu }[ \varphi ] +
    \text{D} F_{ \kappa, \, \mu }[ \varphi ] \cdot \big( ( \partial_{ \mu } G_{ \mu }) *
    F_{ \kappa, \, \mu }[ \varphi ] ) \big).
  \end{equation}
  We can now write a very important system of \textbf{effective equations}.
  \begin{subequations}
    \begin{align}
      \label{eq:Stochastic-quantization-ETC-29}
      \Phi_{ \kappa, \mu }
      &=
        -\int\limits_{ \mu }^{ 1 } ( \partial_{ \rho } G_{ \rho } ) *
        \big( F_{ \kappa, \, \rho }[ \Phi_{ \kappa, \, \rho } ] + \omega_{ \kappa, \, \rho } \big) \,
        \text{d}\rho, \\
      \omega_{ \kappa, \mu }
      &=
        -\int\limits_{ 0 }^{ \mu } \big( H_{ \kappa, \, \rho }[ \Phi_{ \kappa, \, \rho } ] +
        \text{D} F_{ \kappa, \, \rho }[ \Phi_{ \kappa, \, \rho } ] \cdot
        ( ( \partial_{ \rho } G_{ \rho } ) * \omega_{ \kappa, \rho } ) \big) \, \text{d}\rho.
    \end{align}
  \end{subequations}
  For the details of their derivation you can see lecture notes
  by Duch
  \parencite{Duch-Lecture-notes-on-flow-equation-approach-ETC-Pub-2025}.

\end{frame}
% ##################





% ##################
\begin{frame}
  \frametitle{The main idea of Hairer's approach}


  The main idea behind the approach proposed by Martin Hairer in this
  particular context, can be explained as follows. For $\kappa \neq 0$ effective
  equations are equivalent to the equation from which we started, or more
  precisely to
  \begin{equation}
    \label{eq:Stochastic-quantization-ETC-30}
    \Phi_{ \kappa, \, \mu } = G_{ \mu } * F_{ \kappa }[ \Phi_{ \kappa } ].
  \end{equation}
  But, this equation breaks down for $\kappa \searrow 0$, while effective equation
  \alert{can} still be meaningful. But, it doesn't came for free
  and we need to find appropriate function $F_{ \kappa, \, \mu }$, obeying condition
  $F_{ \kappa, 0 } = F_{ \kappa }$, to make limit $\kappa \searrow 0$ well defined.

  Paweł Duch proposed to find $F_{ \kappa, \, \mu }$ by deriving flow equations
  for its coefficient and solving it recursively. At is base it is some
  form of well controlled perturbation series, but more close to the spirit
  of Bogoliubov-Epstein-Glasser approach.

\end{frame}
% ##################





% ##################
\begin{frame}
  \frametitle{Flow equation for
    \MathTextFrametitleFGColor{$F_{ \kappa, \, \mu }$}}

  We know that for function $\varphi$, $F_{ \kappa, \, \mu }[ \varphi ]$ is a continuous
  function on $T^{ d }$. But, for technical reasons is better to think about
  it as distribution on the space of Schwartz functions
  $\Scal( \Rbb^{ d } )$. After few technical steps it can be done.
  As such we can define $F_{ \kappa, \, \mu }$ by equation
  \begin{equation}
    \label{eq:Stochastic-quantization-ETC-31}
    \langle F_{ \kappa, \, \mu }[ \varphi ], \psi \rangle :=
    \sum_{ i = 0 }^{ N } \sum_{ m = 0 }^{ 3i } \lambda^{ i }
    \big\langle F_{ \kappa, \, \mu }^{ \, i, \, m }, \psi \otimes \varphi^{ \otimes k } \big\rangle,
  \end{equation}
  with $F_{ \kappa, \, \mu }^{ \, i, \, m }
  \in \Scal'\big( ( \Rbb^{ d } )^{ m + 1 } \big)$. Coefficient
  $F_{ \kappa, \, \mu }^{ \, i, \, m }$, which are tempered distributions, obeys
  flow equations, which unfortunately is extremely complicated, so we can
  only show it, without explaining its full meaning. They also have more
  indices that a normal person can handle.

\end{frame}
% ##################





% ##################
\begin{frame}
  \frametitle{Flow equation for
    \MathTextFrametitleFGColor{$F_{ \kappa, \, \mu }$}}


  Flow equation has the form
  \begin{equation}
    \label{eq:Stochastic-quantization-ETC-32}
    \partial_{ \mu } F_{ \kappa, \, \mu }^{ \, i, \, m } =
    -\sum_{ j = 0 }^{ i } \sum_{ k = 0 }^{ m } ( 1 + k )
    B_{ k }\big( \partial_{ \mu } G_{ \mu }, F_{ \kappa, \mu }^{ \, j, \, 1 + k },
    F_{ \kappa, \, \mu }^{ \, i - j, \, m - k } \big).
  \end{equation}
  Functions $B_{ k }$ on the right hand side are so complicated and
  hard to explain, that we won't even attempt to do this here. We just
  wrote symbolic expressions for them.
  \begin{equation}
    \label{eq:Stochastic-quantization-ETC-33}
    \begin{split}
      & B_{ k }( G, W, U )( x; \text{d}y_{ 1 }, \text{d}y_{ 2 }, \ldots,
      \text{d}y_{ m } ) := \\
      & \hspace{1em} := \frac{ 1 }{ m! } \sum_{ \pi \in \Pcal_{ m } } \;
        \int\limits_{ \Rbb^{ 2 d } } W( x; \text{d}y_{ \pi( 1 ) },
        \text{d}y_{ \pi( 2 ) }, \ldots,  \text{d}y_{ \pi( k ) } ) G( y - z ) \times \\
      & \hspace{8em} \times U( z; \text{d}y_{ \pi( k + 1 ) },
        \text{d}y_{ \pi( k + 2 ) }, \ldots, \text{d}y_{ \pi( m ) } ) \, \text{d}z.
    \end{split}
  \end{equation}
  At this moment you can only trust that this equation has meaning. \\
  It has.

\end{frame}
% ##################





% ##################
\begin{frame}
  \frametitle{Recursive solution of flow equation}


  Paweł Duch solves this equation by following a recursive procedure
  stated below.

  $0)$ Set $F_{ \kappa, \, \mu }^{ \, 0, \, 0 } = \eta_{ \kappa }$
  and $F_{ \kappa, \, \mu }^{ \, i, \, m } = 0$ if $m > 3i$. \\
  $1)$ We assume that $F_{ \kappa, \, \mu }^{ \, i, \, m }$ were constructed
  for $i < j$ or for $i = j$ and $m > l$, for some $j$ and $l$.
  Then \alert{set} $\partial_{ \mu } F_{ \kappa, \, \mu }^{ \, j, \, l }$ equal to the
  RHS of flow equation \eqref{eq:Stochastic-quantization-ETC-32}. \\[0.2em]
  $2)$ Define $F_{ \kappa, \, \mu }^{ \, j, \, l }$ as
  \begin{equation}
    \label{eq:Stochastic-quantization-ETC-34}
    F_{ \kappa, \, \mu }^{ \, j, \, l } =
    F_{ \kappa }^{ \, j, \, l } +
    \int\limits_{ 0 }^{ \mu } \partial_{ \mu } F_{ \kappa, \, \mu }^{ \, j, \, l }.
  \end{equation}
  We must remember that at this moment
  $\partial_{ \mu } F_{ \kappa, \, \mu }^{ \, j, \, l }$ is only defined by the~RHS of flow
  equation.

  At this moment we want to end our talk, leaving you with the message that
  the theory of SsPDEs is an active field of research that hopefully will
  help us construct more models of QFT at least in $2 + 1$ dimension.

\end{frame}
% ##################










% ######################################
\appendix
% ######################################





% ######################################
\EndingSlide{Thank you! Questions?}
% ######################################










% #####################################################################
% #####################################################################
% Bibliography

\printbibliography





% ############################

% Koniec dokumentu
\end{document}
