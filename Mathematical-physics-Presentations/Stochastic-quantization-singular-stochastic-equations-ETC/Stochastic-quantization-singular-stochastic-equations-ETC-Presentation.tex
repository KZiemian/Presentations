% ---------------------------------------------------------------------
% Basic configuration of Beamera and Jagiellonian
% ---------------------------------------------------------------------
\RequirePackage[l2tabu, orthodox]{nag}



\ifx\PresentationStyle\notset
  \def\PresentationStyle{dark}
\fi



\documentclass[10pt,t]{beamer}
\mode<presentation>
\usetheme[style=\PresentationStyle]{jagiellonian}



% ---------------------------------------
% Configuration files of Jagiellonian loceted in catalog preambule
% ---------------------------------------
% Configuration for polish language
% Need description
\usepackage[english]{babel}





% % ------------------------------
% % Better support of polish chars in technical parts of PDF
% % ------------------------------
% \hypersetup{pdfencoding=auto,psdextra}

% Package "textpos" give as enviroment "textblock" which is very usefull in
% arranging text on slides.

% This is standard configuration of "textpos"
\usepackage[overlay,absolute]{textpos}

% If you need to see bounds of "textblock's" comment line above and uncomment
% one below.

% Caution! When showboxes option is on significant ammunt of space is add
% to the top of textblock and as such, everyting put in them gone down.
% We need to check how to remove this bug.

% \usepackage[showboxes,overlay,absolute]{textpos}



% Setting scale length for package "textpos"
\setlength{\TPHorizModule}{10mm}
\setlength{\TPVertModule}{\TPHorizModule}


% ---------------------------------------
% TikZ
% ---------------------------------------
% Importing TikZ libraries
\usetikzlibrary{arrows.meta}
\usetikzlibrary{positioning}





% % Configuration package "bm" that need for making bold symbols
% \newcommand{\bmmax}{0}
% \newcommand{\hmmax}{0}
% \usepackage{bm}




% ---------------------------------------
% Packages for scientific texts
% ---------------------------------------
% \let\lll\undefined  % Sometimes you must use this line to allow
% "amsmath" package to works with packages with packages for polish
% languge imported
% /preambul/LanguageSettings/JagiellonianPolishLanguageSettings.tex.
% This comments (probably) removes polish letter Ł.
\usepackage{amsmath}  % Packages from American Mathematical Society (AMS)
\usepackage{amssymb}
\usepackage{amscd}
\usepackage{amsthm}
\usepackage{siunitx}  % Package for typsetting SI units.
\usepackage{upgreek}  % Better looking greek letters.
% Example of using upgreek: pi = \uppi


\usepackage{calrsfs}  % Zmienia czcionkę kaligraficzną w \mathcal
% na ładniejszą. Może w innych miejscach robi to samo, ale o tym nic
% nie wiem.










% ---------------------------------------
% Packages written for lectures "Geometria 3D dla twórców gier wideo"
% ---------------------------------------
% \usepackage{./ProgramowanieSymulacjiFizykiPaczki/ProgramowanieSymulacjiFizyki}
% \usepackage{./ProgramowanieSymulacjiFizykiPaczki/ProgramowanieSymulacjiFizykiIndeksy}
% \usepackage{./ProgramowanieSymulacjiFizykiPaczki/ProgramowanieSymulacjiFizykiTikZStyle}





% !!!!!!!!!!!!!!!!!!!!!!!!!!!!!!
% !!!!!!!!!!!!!!!!!!!!!!!!!!!!!!
% EVIL STUFF
\if\JUlogotitle1
\edef\LogoJUPath{LogoJU_\JUlogoLang/LogoJU_\JUlogoShape_\JUlogoColor.pdf}
\titlegraphic{\hfill\includegraphics[scale=0.22]
{./JagiellonianPictures/\LogoJUPath}}
\fi
% ---------------------------------------
% Commands for handling colors
% ---------------------------------------


% Command for setting normal text color for some text in math modestyle
% Text color depend on used style of Jagiellonian

% Beamer version of command
\newcommand{\TextWithNormalTextColor}[1]{%
  {\color{jNormalTextFGColor}
    \setbeamercolor{math text}{fg=jNormalTextFGColor} {#1}}
}

% Article and similar classes version of command
% \newcommand{\TextWithNormalTextColor}[1]{%
%   {\color{jNormalTextsFGColor} {#1}}
% }



% Beamer version of command
\newcommand{\NormalTextInMathMode}[1]{%
  {\color{jNormalTextFGColor}
    \setbeamercolor{math text}{fg=jNormalTextFGColor} \text{#1}}
}


% Article and similar classes version of command
% \newcommand{\NormalTextInMathMode}[1]{%
%   {\color{jNormalTextsFGColor} \text{#1}}
% }




% Command that sets color of some mathematical text to the same color
% that has normal text in header (?)

% Beamer version of the command
\newcommand{\MathTextFrametitleFGColor}[1]{%
  {\color{jFrametitleFGColor}
    \setbeamercolor{math text}{fg=jFrametitleFGColor} #1}
}

% Article and similar classes version of the command
% \newcommand{\MathTextWhiteColor}[1]{{\color{jFrametitleFGColor} #1}}





% Command for setting color of alert text for some text in math modestyle

% Beamer version of the command
\newcommand{\MathTextAlertColor}[1]{%
  {\color{jOrange} \setbeamercolor{math text}{fg=jOrange} #1}
}

% Article and similar classes version of the command
% \newcommand{\MathTextAlertColor}[1]{{\color{jOrange} #1}}





% Command that allow you to sets chosen color as the color of some text into
% math mode. Due to some nuances in the way that Beamer handle colors
% it not work in all cases. We hope that in the future we will improve it.

% Beamer version of the command
\newcommand{\SetMathTextsColor}[2]{%
  {\color{#1} \setbeamercolor{math text}{fg=#1} #2}
}


% Article and similar classes version of the command
% \newcommand{\SetMathTextColor}[2]{{\color{#1} #2}}










% ---------------------------------------
% Commands for setting background pictures for some slides
% ---------------------------------------
\newcommand{\TitleBackgroundPicture}
{./PresentationPictures/CommonPictures/Cute_dragon_BG_dark.png}
\newcommand{\SectionBackgroundPicture}
{./PresentationPictures/CommonPictures/Cute_dragon_small_BG_light.png}



\newcommand{\TitleSlideWithPicture}{
  \begingroup

  \usebackgroundtemplate{ % \hspace*{-11.5em}
    \includegraphics[height=\paperheight]{\TitleBackgroundPicture}}

  \maketitle

  \endgroup
}





\newcommand{\SectionSlideWithPicture}[1]{%
  \begingroup

  \usebackgroundtemplate{ % \hspace*{-11.5em}
    \includegraphics[height=\paperheight]{\SectionBackgroundPicture}}

  \setbeamercolor{titlelike}{fg=normal text.fg}

  \section{#1}

  \endgroup
}





\newcommand{\EndingSlide}[1]{%
  \begin{frame}[standout]

    \begingroup

    \color{jFrametitleFGColor}

    #1

    \endgroup

  \end{frame}
}










% ------------------------------------------------------
% BibLaTeX
% ------------------------------------------------------
% Package biblatex, with biber as its backend, allow us to handle
% bibliography entries that use Unicode symbols outside ASCII.
\usepackage[
language=polish,
backend=biber,
style=alphabetic,
url=false,
eprint=true,
]{biblatex}

\addbibresource{Możliwie-proste-wprowadzenie-do-AQFT-Bibliography.bib}





% ------------------------------------------------------
% Wonderful package PGF/TikZ
% ------------------------------------------------------

% Node and pics for drawing charts
% \usepackage{./Local-packages/PGF-TikZ-Chart-nodes-and-pics}

% Styles for arrows
% \usepackage{./Local-packages/PGF-TikZ-Arrows-styles}

% Pic for drawing functions
% \usepackage{./Local-packages/PGF-TikZ-Functions-pics}






% ------------------------------------------------------
% Local packages
% ------------------------------------------------------
% Special configuration for this particular presentation
\usepackage{./Local-packages/local-settings}

% Package containing various command useful for working with a text
\usepackage{./Local-packages/general-commands}

% Package containing commands and other code useful for working with
% mathematical text
\usepackage{./Local-packages/math-commands}










% ---------------------------------------------------------------------
\title{Stochastic quantization, singular stochastic
  equaations and works of Paweł Duch}

\author{Kamil Ziemian \\
  \texttt{kziemianfvt@gmail.com}}


% \institute{}

\date[23 X 2024]{Seminar of Field Theory Department \\
  28 XI 2024}
% --------------------------------------------------------------------










% ####################################################################
% Początek dokumentu
\begin{document}
% ####################################################################





% ######################################
% Number of chars: 29k+,
% Text is adjusted to the left and words are broken at the end of the line.
\RaggedRight
% ######################################






% ######################################
\maketitle % Tytuł całego tekstu
% ######################################





% ######################################
\begin{frame}
  \frametitle{Spis treści}


  \tableofcontents % Spis treści

\end{frame}
% ######################################





% ######################################
\section{Informacje wstępne}
% ######################################


% ##################
\begin{frame}
  \frametitle{Mathematical physicist always spoils the fun}


  We want to define simple qunatum filed theory like $\Phi^{ 4 }$ in $d + 1$
  dimensions, so we wrote the lagrangian
  \begin{equation}
    \label{eq:Stochastic-quantization-ETC-01}
    \Lcal =
    \frac{ 1 }{ 2 } \partial_{ \mu } \Phi( x ) \partial^{ \mu } \Phi( x ) - m^{ 2 } \Phi^{ 2 }( x )
    - \frac{ \lambda }{ 4! } \Phi^{ 4 }( x ).
  \end{equation}
  where $\Phi( x )$ stands for $\Phi( t, x_{ 1 }, x_{ 2 }, \ldots, x_{ d } )$. Then we
  compute action
  \begin{equation}
    \label{eq:Stochastic-quantization-ETC-02}
    S[ \Phi( x ) ] =
    \int_{ [ 0, 1 ] \times \Rbb_{ d } } d^{ d + 1 }x \, \Lcal[ \Phi( x ) ],
  \end{equation}
  and partition function
  \begin{equation}
    \label{eq:Stochastic-quantization-ETC-03}
    Z[ J( x ) ] =
    \int [ \Dcal \Phi( x ) ] \exp\Big( i S[ \phi( x ) ] + \int d^{ d + 1 }x \, J( x )
    \Phi( x ) \Big),
  \end{equation}
  and we can do physics now. But, mathematical physicist like to spoile
  all the fun, so we point out that measure $\Dcal \Phi( x )$ doesn't exists.

\end{frame}
% ##################





% ##################
\begin{frame}
  \frametitle{Why $\Dcal \Phi( x )$ doesn't exists?}


  What is the problem with $\Dcal \Phi( x )$? First we need to notice, that
  it should be a~Lebesgue's measure for the inifinite dimensional space
  of functions, like $L^{ 2 }( \Rbb^{ d + 1 }, d^{ d + 1 }x )$. In the case~of
  $\Rbb^{ d }$ the crucial property of Lebesgue's measure is translational
  invariance. If we have mesurable set $A \subset \Rbb^{ d }$, then
  \begin{equation}
    \label{eq:Niespodziewane-teoretyczne-01}
    \forall \vecx \in \Rbb^{ d }, \quad
    \text{vol}( A + \vecx ) = \text{vol}( A ).
  \end{equation}

  More preciasly, we have a theorem.

  \begin{theorem}
    Let $G$ be localy compact, topogological group. There \alert{exists}
    translational invariant, positive measure on the $G$.
  \end{theorem}

\end{frame}
% ##################





% % ######################################
% \section{Aksjomatyczna i~konstrukcyjna kwantowa teoria pola,
%   czyli o~tym czego nie umiemy zrobić}
% % ######################################









% % ##################
% \begin{frame}
%   \frametitle{Dlaczego tak jest?}


%   By wyjaśnić przyczyny tego smutnego stanu rzeczy, sięgniemy do przykładu
%   teorii którą nazywamy $\varphi^{ 4 }$. Tutaj będzie potrzebna znajomość teorii
%   dystrybucji Schwartza. Czy mogę założyć jej znajomość u~Państwa?

%   Teoria $\varphi^{ 4 }$ zdefiniowana jest ona przez lagrangian
%   (dla prostoty używamy standardowej w~fizyce teoretycznej konwencji
%   do zapisu stałych fizycznych)

%   \vspace{-1.5em}



%   \begin{equation}
%     \label{eq:Niespodziewane-teoretyczne-01}
%     \begin{split}
%       \Lcal( \varphi )
%       &=
%         \frac{ 1 }{ 2 } \left[ ( \partial_{ t } \varphi( t, \vecx ) )^{ 2 } -
%         ( \partial_{ x } \varphi( t, \vecx ) )^{ 2 } - ( \partial_{ y } \varphi( t, \vecx ) )^{ 2 } -
%         ( \partial_{ z } \varphi( t, \vecx ) )^{ 2 } \right] - \\
%       &\hspace{1em} - \frac{ \lambda }{ 4! } \varphi( t, \vecx )^{ 4 }.
%     \end{split}
%   \end{equation}

%   \vspace{-1.8em}



%   Lagrangian ten prowadzi do równania ruchu
%   \begin{equation}
%     \label{eq:Niespodziewane-teoretyczne-01}
%     \begin{split}
%       \partial_{ t }^{ 2 } \varphi( t, \vecx ) - \partial_{ x }^{ 2 } \varphi( t, \vecx ) -
%       \partial_{ y }^{ 2 } \varphi( t, \vecx ) - \partial_{ z }^{ 2 } \varphi( t, \vecx ) +
%       \frac{ \lambda }{ 3! } \varphi( t, \vecx )^{ 3 } = 0.
%     \end{split}
%   \end{equation}

% \end{frame}
% % ##################





% % ##################
% \begin{frame}
%   \frametitle{Dlaczego tak jest?}

%   \vspace{-1.5em}


%   \begin{equation}
%     \label{eq:Niespodziewane-teoretyczne-01}
%     \begin{split}
%       \partial_{ t }^{ 2 } \varphi( t, \vecx ) - \partial_{ x }^{ 2 } \varphi( t, \vecx ) -
%       \partial_{ y }^{ 2 } \varphi( t, \vecx ) - \partial_{ z }^{ 2 } \varphi( t, \vecx ) +
%       \frac{ \lambda }{ 3! } \varphi( t, \vecx )^{ 3 } = 0.
%     \end{split}
%   \end{equation}
%   Gdzie jest problem? Większość podręczników do kwantowej teorii pola
%   powie nam, że~pole $\varphi$ jest funkcją o~wartościach operatorowych, czyli
%   $\varphi( t, \vecx )$ dla konkretnego $t$ i~$\vecx$ to operator działający
%   w~pewnej przestrzeni Hilberta. Bardziej dokładna analiza wskazuje,
%   że~takie podejście jest błędne.

%   Pole $\varphi( t, \vecx )$ nie jest funkcją o~wartościach operatorowych,
%   lecz \alert{dystrybucją (temperowaną) w~sensie Schwartza} o~wartościach
%   operatorowych. Z~tego powodu wartość tej dystrybucji w~punkcie
%   $( t, \vecx )$ jest w~ogólności pozbawiona sensu i~notacja
%   $\varphi( t, \vecx )$ jest tylko zbiorem symboli na tablicy, bez głębszej
%   treści. (Trochę przesadzam, ale nie tak bardzo trochę.)

% \end{frame}
% % ##################





% % ##################
% \begin{frame}
%   \frametitle{Czym jest pole kwantowe?}


%   Skoro pole $\varphi$ jest dystrybucją, to jeśli weźmiemy jakąś funkcję
%   klasy Schwartza $f$, to wówczas

%   \vspace{-1.7em}



%   \begin{equation}
%     \label{eq:Niespodziewane-teoretyczne-03}
%     \varphi( f ), \qquad
%     f \in \Scal( \Rbb^{ 3 } ),
%   \end{equation}

%   \vspace{-1.8em}



%   jest operatorem działającym na pewnej przestrzeni Hilberta. Ten fakt
%   jest przyczyną częstego pojawiania~się funkcji klasy Schwartza w~pewnym
%   podręczniku do algebry liniowej, który może mieli Państwo kiedyś w~rękach.

%   Wróćmy do naszego równania ruchu

%   \vspace{-2em}



%   \begin{equation}
%     \label{eq:Niespodziewane-teoretyczne-01}
%     \begin{split}
%       \partial_{ t }^{ 2 } \varphi( t, \vecx ) - \partial_{ x }^{ 2 } \varphi( t, \vecx ) -
%       \partial_{ y }^{ 2 } \varphi( t, \vecx ) - \partial_{ z }^{ 2 } \varphi( t, \vecx ) +
%       \frac{ \lambda }{ 3! } \varphi( t, \vecx )^{ 3 } = 0.
%     \end{split}
%   \end{equation}
%   W~tym równaniu liczymy drugą pochodną dystrybucji, co zawsze ma sens.
%   Potem dodajemy dystrybucje, tutaj też wszystko jest dobrze. Potem
%   podnosimy do \alert{trzeciej potęgi} dystrybucję. Oj, nie jest dobrze.

% \end{frame}
% % ##################





% % ##################
% \begin{frame}
%   \frametitle{Mamy duży problem}

%   \vspace{-1.5em}


%   \begin{equation}
%     \label{eq:Niespodziewane-teoretyczne-01}
%     \begin{split}
%       \partial_{ t }^{ 2 } \varphi( t, \vecx ) - \partial_{ x }^{ 2 } \varphi( t, \vecx ) -
%       \partial_{ y }^{ 2 } \varphi( t, \vecx ) - \partial_{ z }^{ 2 } \varphi( t, \vecx ) +
%       \frac{ \lambda }{ 3! } \varphi( t, \vecx )^{ 3 } = 0.
%     \end{split}
%   \end{equation}
%   Samo mnożenie dystrybucji jest w~ogólności niewykonalne, tym bardziej
%   nie można w~ogólności podnieść jej do trzeciej potęgi. Niemniej jeśli
%   kogoś nie interesuje zbytnio matematyczna ścisłość, to może spokojnie
%   rozwijać dalej całą teorię w~oparciu o~to pozbawione matematycznego sensu
%   równanie i~swoją intuicję fizyczną.

%   Ludzie niezadowoleni z~tego stanu rzeczy stworzyli aksjomatyczną kwantową
%   teorię pola i~konstrukcyjną kwantową teorię pola. Aksjomatyczna teoria
%   pola zajmuje~się definiowaniem tego, czym jest pole kwantowe poprzez
%   podanie listy aksjomatów jakie to ma spełnić obiekt, by był polem
%   kwantowym. Jest to ta sama sytuacja, gdy definiujemy grupę poprzez
%   podanie jej własności, jak istnienie elementu $e$, który dla każdego
%   $g \in G$ spełnia warunek $e g = g e = g$.

% \end{frame}
% % ##################





% % ##################
% \begin{frame}
%   \frametitle{Mamy kolejny wielki problem}


%   Tak samo jak w~teorii grup, bazując tylko na aksjomatycznej definicji
%   grupy możemy dowodzić pewnych jej własności (np. istnieje tylko jeden
%   element odwrotny), tak samo w~aksjomatycznej kwantowej teorii pola,
%   wychodząc od tych aksjomatów dowodzą własności pól kwantowych.
%   Na obecną chwilę nie ma jednego ogólnie obowiązujący zestaw aksjomatów
%   dla kwantowej teorii pola, z~istniejących warto wymienić
%   \colorhref{https://en.wikipedia.org/wiki/Wightman_axioms}{aksjomaty
%     Wightmana}
%   i~\colorhref{https://ncatlab.org/nlab/show/Osterwalder-Schrader+theorem}{Osterwaldera-Schradera}.

%   Poważna różnica polega na tym, że~mając aksjomaty teorii, grup od razu
%   możemy podać kilka przykładów takich obiektów, podczas gdy podanie
%   konkretnego przykładu pola kwantowego spełniającego zestaw aksjomatów
%   jest niewiarygodnie trudnym zadaniem. Stąd istnieje osobny przedmiot
%   badań, konstrukcyjna kwantowa teoria pola, która zajmuje~się
%   budowanie konkretnych przykładów pól kwantowych spełniających wybrany
%   zestaw aksjomatów.

% \end{frame}
% % ##################





% % ##################
% \begin{frame}
%   \frametitle{I to naprawdę duży problem}


%   Według moje wiedzy, w~roku $2024$ wciąż nie jest znany \alert{żaden}
%   przykład pola kwantowego w~$3 + 1$ wymiarach, które oddziałuje z~pole
%   kwantowym. W~tej liczbie wymiarów znane są przykłady pola kwantowego,
%   które oddziaływuje z~obiektami netwonowskimi, ale nie jesteśmy w~stanie
%   zdefiniować choćby elektrodynamiki kwantowej.

%   Podamy teraz kilka wybranych dat z~historii konstrukcyjnej kwantowej
%   teorii pola.

%   W~latach $1968\text{-}1972$
%   \colorhref{https://en.wikipedia.org/wiki/James\_Glimm}{James Glimm}
%   i~\colorhref{https://en.wikipedia.org/wiki/Arthur\_Jaffe}{Arhtur Jaffe}
%   konstruują teorię $\varphi^{ 4 }$ w~dwóch wymiarach czasoprzestrzennych.

%   $24$ maja $2000$ roku Instytut Matematyki im. Claya ogłasza swoje Problemy
%   Millenijne. Jednym z~nich jest skonstruowanie kwantowej teorii
%   Yanga-Millsa w~$3 + 1$ wymiarach oraz wykazanie, że~posada przerwę masową.
%   Konstrukcja musi spełniać zbiór aksjomatów co najmniej tak
%   silnych jak aksjomaty Wightmana lub Osterwaldera-Schradera.

% \end{frame}
% % ##################





% % ##################
% \begin{frame}
%   \frametitle{Najnowsze wyniki}


%   Autorami oficjalnego sformułowania tego Problemu Millenijnego
%   są \colorhref{https://en.wikipedia.org/wiki/Edward\_Witten}{Edward Witten}
%   i~\colorhref{https://en.wikipedia.org/wiki/Arthur\_Jaffe}{Arthur Jaffe}.
%   Jest ono dostępne pod tym
%   \colorhref{https://www.claymath.org/wp-content/uploads/2022/06/yangmills.pdf}{linkiem}.

%   W~$2022$ roku
%   \colorhref{https://sites.google.com/view/ajaychandra/home}{Ajay Chandra},
%   \colorhref{https://ilyachevyrev.wordpress.com/}{Ilya Chevyrev},
%   \colorhref{https://en.wikipedia.org/wiki/Martin_Hairer}{Martin Hairer}
%   i~\colorhref{https://people.math.wisc.edu/~hshen3/}{Hao Shen} publikują
%   na arXivie wysoce niebanalną pracę
%   \colorhref{https://arxiv.org/abs/2201.03487}{\textit{Stochastic
%       quantisation~of Yang-Mills-Higgs in~3D}}
%   \parencite{Chandra-et-al-Stochastic-quantisation-of-Yang-Mills-ETC-Pub-2024},
%   w~której pokazują jak w~pełni ściśle, choć w~dość pokręcony sposób,
%   skonstruować kwantową teorię Yanga-Millsa wraz z~polem Higgsa, w~$2 + 1$
%   wymiarach.

%   Warto dodać, że~Martin Hairer jest laureatem Medalu Fieldsa za rok $2014$
%   i~„Breakthrough Prize in Mathematics” za rok $2021$
%   i~w~\colorhref{https://www.youtube.com/watch?v=4jR8Sg4PYAA}
%   {wywiadzie} dla kanału Tom Rocks Maths z~tego samego roku stwierdził,
%   że~ich metody działają jeśli wymiar przestrzeni jest równy
%   „$3 + 1 - \varepsilon$”, dla $\varepsilon > 0$, zgodnie jednak z~prawem Murphy’ego,
%   załamują~się dokładnie dla $3 + 1$.

% \end{frame}
% % ##################





% % ##################
% \begin{frame}
%   \frametitle{Najnowsze wyniki}


%   Były doktorant prof. Herdegena,
%   dr~\colorhref{https://pawelduch.github.io/}{Paweł Duch} jest obecnie
%   członkiem \colorhref{https://www.epfl.ch/labs/propde/members/}{grupy}
%   Martina Hairera na École Polytechnique Fédérale de Lausanne
%   w~Szwajcarii.

%   Poza tym w~konstrukcyjnej kwantowej teorii pola musimy~się
%   zmagać z~takimi przeciwnościami losu jak:
%   \begin{itemize}

%   \item Nieośrodkowe przestrzenie Hilberta. Z~tego powodu wybór przestrzeni
%     Hilberta dyktuje \alert{fizyczne} własności teorii.

%   \item Twierdzenie Haaga o~niemożliwości zbudowania obrazu interakcji
%     (obrazu Dirac’a), na którym opiera~się większość wyliczeń dla Modelu
%     Standardowego. A~przynajmniej~się opierała przez kilka dekad.

%   \item Całki po trajektoriach i~ich nieistnienie w~większości przypadków.

%   \item I~wiele innych.

%   \end{itemize}

% \end{frame}
% % ##################










% % ######################################
% \section{Algebraiczna kwantowa teoria pola}
% % ######################################


% % ##################
% \begin{frame}
%   \frametitle{Czym jest AQFT?}


%   Algebraiczna kwantowa teoria pola jest w~zasadzie jednym z~wariantów
%   aksjomatycznej kwantowej teorii pola. Została ona sformułowana przez
%   Rudolfa Haaga ($1922\text{-}2016$), z~dużym wkładem ~Davida Ruelle’a
%   (ur.~$1935$), w~latach $50$-tych i~$60$-tych XX~wieku. Jej dwie podstawowe
%   idee można ująć w~następujący sposób.
%   \begin{itemize}

%   \item[1)] Zjawiska fizyczne są zlokalizowane w~czasie i~przestrzeni
%     („zasada-bliskiego-działania”).

%   \item[2)] Fundamentalne relacje między kwantowymi obserwablami mają
%     charakter algebraiczny.

%   \end{itemize}

%   Ze względu na drugi punkt, w~\textsc{aqft} pola kwantowe są obiektem
%   drugorzędnym, na pierwszy zaś plan wysuwają~się odpowiednie algebry
%   obserwabli. Jeśli mamy taką algebrę, jesteśmy w~stanie odtworzyć
%   odpowiednie pole kwantowe.

% \end{frame}
% % ##################





% % ##################
% \begin{frame}
%   \frametitle{Problemy i~osiągnięcia AQFT}


%   Tak jak wszystkie inne podejścia aksjomatyczne, \textsc{aqft} zmaga~się
%   z~problemem braku modeli konkretnych pól kwantowych. Tym samym nie możemy
%   wyliczyć, powiedzmy, masy protonu jako funkcji kilku mierzalnych
%   parametrów kwarków i~gluonów, w~tym samym sensie, jak nie możemy podać
%   liczby elementów grupy na podstawie jej definicji. To nie jest coś co
%   wynika z~aksjomatów, tylko z~własności konkretnego bytu matematycznego,
%   gdy już określimy go dokładnie (podając jego podstawowe parametry
%   fizyczne).

%   Chromodynamika na sieciach (obliczenia komputerowe) pozwala nam obliczyć
%   masę protonu, ale to~temat na osobne seminarium, prowadzone przez inną
%   osobę.

%   Niemniej, na poziomie wywodzenia z~aksjomatów, \textsc{aqft}~potrafi
%   udowodnić wiele jakościowy rezultatów zgodnych ze światem, który widzimy.
%   Ale brak możliwości obliczenia masy protonu, sprawia, że~nie możemy być
%   z~obecnego stanu rzeczy zadowoleni.

% \end{frame}
% % ##################





% % ##################
% \begin{frame}
%   \frametitle{Lokalizacja zjawisk w~czasoprzestrzeni}

%   \vspace{-0.8em}


%   \begin{figure}

%     \centering


%     \begin{tikzpicture}

%       % x axis
%       \draw[axis arrow] (-5,0) -- (5,0);

%       \pic at (5,0) {x mark for horizontal axis 1};


%       % t axis
%       \draw[axis arrow] (0,-2.5) -- (0,4);

%       \pic at (0,4) {t mark for vertical axis 1};



%       % Light cone
%       \fill[color=blue,opacity=0.4] (0,0) -- (3.5,3.5) -- (-3.5,3.5) -- cycle;

%       \fill[color=blue,opacity=0.4] (0,0) -- (2.2,-2.2) -- (-2.2,-2.2) --
%       cycle;

%       \draw[dashed] (2.2,-2.2) -- (-3.5,3.5);

%       \draw[dashed] (-2.2,-2.2) -- (3.5,3.5);



%       % Region number 1
%       \draw[dashed] plot[smooth cycle]
%       coordinates { (1.2,0) (3,-1) (3.5,1.4) (2.5,1.2) };

%       \node[scale=1.3] at (2.3,0.3) {$\Ocal_{ 1 }$};


%       % Region number 2
%       \draw[dashed] plot[smooth cycle]
%       coordinates { (0.8,0) (2.9,-1.5) (4.2,1.8) (1,3) };

%       \node[scale=1.3] at (2.7,1.8) {$\Ocal_{ 2 }$};


%       % Region number 3
%       \draw[dashed] plot[smooth cycle]
%       coordinates { (-1,-0.2) (-2,1.2) (-3,0.3) (-2,-1) };

%       \node[scale=1.3] at (-2,0.5) {$\Ocal_{ 3 }$};


%       % Event reaching form region 3 to region 1
%       \draw[dashed,thick,color=yellowLoc] (-1.3,0) -- (2,3.3);

%       \fill[color=brown] (-1.3,0) circle [radius=0.09];

%       \node[below] at (-1.3,-0.05) {$E$};

%     \end{tikzpicture}

%     \caption{Trzy ograniczone obszary $\Ocal_{ 1 }$, $\Ocal_{ 2 }$
%       i~$\Ocal_{ 3 }$, oraz zjawisko $E$ wysyłające promień światła.}


%   \end{figure}

% \end{frame}
% % ##################





% % ##################
% \begin{frame}
%   \frametitle{Lokalizacja zjawisk w~czasoprzestrzeni}


%   Z~punktu widzenia fizyki interpretacja tej zasady jest prosta. Każdy
%   fizyczny pomiar trwa skończony czas i~zajmuje skończony obszar
%   przestrzeni. Jeśli trzeba możemy zmodyfikować ten formalizm, by dopuścić
%   nieskończone obszary, ale to rzadko jest potrzebne.

%   By być dokładnym, z~każdym zjawiskiem fizycznym wiążemy otwarty
%   i~ograniczony (w~sensie $\Rbb^{ 4 }$) podzbiór czasoprzestrzeni
%   Minkowskiego, oznaczamy go przez $\Ocal$, w~którym może on zostać
%   zmierzony. Ponieważ wielkości mierzone są reprezentowane przez pewną
%   algebrę, więc obszarowi temu przypisuje algebrę $\Acal( \Ocal )$.

%   Dobrze, ale na jakim zasadzie działają takie algebry? Standardowo
%   w~mechanice kwantowej stany reprezentują wektory z~przestrzeni
%   Hilberta $\psi \in \Hcal$, zaś obserwable operatory, przykładowo $X = x$ \\
%   i~$P = -i \, \hbar \frac{ d }{ dx }$. Korzystając z~tego, i~nie
%   przejmując~się zbytnio matematyką pokazujemy:

%   \vspace{-2em}



%   \begin{equation}
%     [ X, P ] = XP - PX = i \, \hbar \id.
%   \end{equation}

% \end{frame}
% % ##################





% % ##################
% \begin{frame}
%   \frametitle{Algebry $\MathTextFrametitleFGColor{C^{ * }}$}

%   \vspace{-1em}



%   \begin{equation}
%     [ X, P ] = XP - PX = i \, \hbar \id.
%   \end{equation}
%   W~algebraicznej kwantowej teorii pola, jak sama nazwa wskazuje,
%   traktuje takie \alert{algebraiczne} relacje jak te powyżej, jako
%   najbardziej podstawowe własności obserwablami, którymi to obserwablami
%   może być położenie i~pęd elektronu. Przestrzeń Hilberta i~reprezentacja
%   operatorów jako $X = x$ oraz $P = -i \, \hbar \frac{ d }{ dx }$ jest
%   wyprowadzona z~odpowiednich własności algebr.

%   Tutaj potrzebujemy przywołać pojęcie algebry $C^{ * }$. Jej podstawowe
%   cechy to wyabstrahowane własności macierzy kwadratowych $n \times n$
%   o~wyrazach zespolonych. Dokładniej $\Acal$ jest algebrą $C^{ * }$ jeśli:
%   \begin{itemize}

%   \item[1)] $\Acal$ jest przestrzenią wektorową nad $\Cbb$ (dodawanie
%     macierzy i~mnożenie ich przez liczbę).

%   \item[2)] Zdefiniowana jest operacja mnożenia elementów z~$A$ (mnożenie
%     macierzy).

%   \end{itemize}

% \end{frame}
% % ##################





% % ##################
% \begin{frame}
%   \frametitle{Algebry $\MathTextFrametitleFGColor{C^{ * }}$}


%   \begin{itemize}

%   \item[2a)] Mnożenie elementów z~$\Acal$ jest łączne.

%   \item[2b)] Mnożenie jest biliniowe
%     \begin{equation}
%       \label{eq:Niespodziewane-teoretyczne-01}
%       A ( \alpha B + \beta C ) = \alpha A B + \beta A C, \quad
%       ( \alpha A + \beta B ) C = \alpha A C + \beta B C.
%     \end{equation}

%     \vspace{-1.5em}



%   \item[3)] Dana jest operacja zwana inwolucją (sprzężenie hermitowskie
%     macierzy).
%     \begin{equation}
%       \label{eq:Niespodziewane-teoretyczne-01}
%       \Acal \ni A \mapsto A^{ * } \in \Acal.
%     \end{equation}

%     \vspace{-1.5em}



%   \item[3a)] Inwolucja posiada następujące własności

%     \vspace{-2em}



%     \begin{subequations}

%       \begin{align}
%         \label{eq:Niespodziewane-teoretyczne-01}
%         ( \alpha A )^{ * } &= \bar{ \alpha } A^{ * }, \\
%         ( A^{ * } )^{ * } &= A, \\
%         ( A + B )^{ * } &= A^{ * } + B^{ * },
%       \end{align}

%     \end{subequations}

%     \vspace{-2em}



%     gdzie $\bar{ \alpha }$ to liczba sprzężona do~$\alpha$.

%   \end{itemize}

% \end{frame}
% % ##################





% % ##################
% \begin{frame}
%   \frametitle{Algebry $\MathTextFrametitleFGColor{C^{ * }}$}


%   W~zbiorze macierzy diagonalizowalnych możemy wprowadzić normę wzorem
%   \begin{equation}
%     \label{eq:Niespodziewane-teoretyczne-01}
%     \Vert A \Vert = \max\{ \absOne{ \lambda_{ 1 } }, \absOne{ \lambda_{ 2 } }, \ldots,
%     \absOne{ \lambda_{ n } } \},
%   \end{equation}
%   gdzie $\lambda_{ i }$ to jej wartości własne. Jest to norma o~własnościach
%   znanych z~analizy matematycznej, więc zbiór macierze stają~się przestrzeń
%   metryczną. Dla przypadku macierzy niediagonalizowalnej trzeba ten wzór
%   uogólnić, ale nie jest to teraz tak ważne.

%   Wzorując~się na własnościach tej normy, nakładamy na algebry $C^{ * }$
%   następujące warunki.

%   \begin{itemize}

%   \item[4)] $\Acal$ jest przestrzenią unormowaną z~normą $\Vert \cdot \Vert$.

%   \item[4a)] Przy metryce danej przez $\Vert \cdot \Vert$, $\Acal$ jest przestrzenią
%     zupełną.

%   \item[4b)] Norma spełnia dwa dodatkowe warunki.

%     \vspace{-1.7em}



%     \begin{subequations}

%       \begin{align}
%         \label{eq:Niespodziewane-teoretyczne-01}
%         \Vert A B \Vert &\leq \Vert A \Vert \, \Vert B \Vert, \\
%         \Vert A^{ * } A \Vert &= \Vert A \Vert^{ 2 }.
%       \end{align}

%     \end{subequations}


%   \end{itemize}


% \end{frame}
% % ##################





% % ##################
% \begin{frame}
%   \frametitle{Własności algebr
%     $\MathTextFrametitleFGColor{C^{ * }}$}


%   Bez wnikania w~detale techniczne, każdą algebrę $C^{ * }$ możemy
%   wyposarzyć w~element neutralny mnożenia $\UnitAlg$:
%   \begin{equation}
%     \label{eq:Niespodziewane-teoretyczne-01}
%     A \UnitAlg = \UnitAlg A.
%   \end{equation}
%   Dla elementu algebry $C^{ * }$ możemy zdefiniować jego widmo, obliczać
%   relacje komutacji, przy odpowiednich założeniach zdefiniować rozkład
%   polarny, etc.

%   Teoria ta nie jest prosta dla początkujących, dlatego prof. Patryk Mach
%   kiedyś stwierdził, że~problem z~pewnym podręcznikiem do algebry polega na
%   tym, że~spora część jego materiału to jest teoria algebr $C^{ * }$
%   wyłożona w~specjalny przypadku macierzy.

%   Potrzebować będziemy teraz pojęcia przestrzeni dualnej do przestrzeni
%   wektorowej. Albo jeszcze lepiej dualnej do przestrzeni unormowanej
%   (Banach). Czy mam przypomnieć o~co chodzi?

% \end{frame}
% % ##################






% % ##################
% \begin{frame}
%   \frametitle{Definicja stanów}


%   Niech $\Acal^{ \text{d} }$ będzie przestrzenią dualną do~$\Acal$.
%   Element $\omega \in \Acal^{ \text{d} }$ nazywamy \textbf{stanem}, jeśli

%   \vspace{-2.5em}



%   \begin{subequations}

%     \begin{align}
%       \label{eq:Niespodziewane-teoretyczne-01}
%       \omega( \UnitAlg ) &= 1 \in \Rbb, \\
%       \omega( A^{ * } A ) &\geq 0, \quad
%                             A \in \Acal.
%     \end{align}

%   \end{subequations}

%   \vspace{-2em}



%   W~\textsc{aqft} elementy algebry $\Acal( \Ocal )$ reprezentują wielkości
%   fizyczne mierzalne w~obszarze $\Ocal$, zaś stany reprezentują, co dużo
%   mówić, stany układu fizycznego, którego wielkości mierzymy.

% \end{frame}
% % ##################





% % ##################
% \begin{frame}
%   \frametitle{Algebraiczna QFT $\MathTextFrametitleFGColor{\subset}$
%     Lokalna Fizyka Kwantowa}


%   Przykładowo, niech $X \in \Acal( \Ocal )$ będzie obserwablą określającą
%   odpowiadającą położeniu elektronu, zaś $\omega \in \Acal^{ \text{d} }$
%   będzie stanem elektronu, który porusza~się ruchem jednostajnym
%   prostoliniowym. Wówczas
%   \begin{equation}
%     \label{eq:Niespodziewane-teoretyczne-01}
%     \langle X \rangle_{ \omega } := \omega( X ),
%   \end{equation}
%   jest wartością oczekiwaną położenia elektronu w~stanie $\omega$, zmierzoną
%   w~obszarze $\Ocal$. W~mechanice kwantowej zapisalibyśmy to jako
%   \begin{equation}
%     \label{eq:Niespodziewane-teoretyczne-01}
%     \langle X \rangle_{ \psi } = \langle \psi | X | \psi \rangle = ( \psi, X \psi ).
%   \end{equation}

%   \vspace{-2em}



%   W~tym możemy już w~zasadzie odtworzyć aparat probabilistyczny mechaniki
%   kwantowej.

% \end{frame}
% % ##################





% % ##################
% \begin{frame}
%   \frametitle{Konstrukcja GNS}


%   Niezwykle ważny elementem nie tylko teorii algebr $C^{ * }$, i~całej
%   współczesnej matematyki, jest \textbf{konstrukcja \textsc{gns}
%     (Gelfand-Naimark-Segal)}, która mówi,
%   że~jeśli dysponujemy stanem $\omega$ na algebrze $C^{ * }$, to możemy
%   z~niej „wyciągnąć” przestrzeń Hilberta. Nie możemy
%   wchodzić w~to jak dokładnie ona przebiega, powiedzmy pokrótce, że~chodzi
%   o~to, by potraktować element $A, B \in \Acal$ jako wektory przestrzeni
%   Hilberta i~zdefiniować ich iloczyn skalarny wzorem:

%   \vspace{-1.5em}



%   \begin{equation}
%     \label{eq:Niespodziewane-teoretyczne-01}
%     ( A, B ) := \omega( A^{ * } B ).
%   \end{equation}

%   \vspace{-1.5em}



%   Dla nas ważne jest to, że~dzięki temu, po odpowiednio dużej ilości
%   pracy, możemy odtworzyć standardowy formalizm mechaniki kwantowej.
%   Acz nie jest to takie łatwe jakby się mogło wydawać, tym co nie zgłębili
%   arkanów analizy funkcjonalnej.

% \end{frame}
% % ##################





% % ##################
% \begin{frame}
%   \frametitle{Lokalizacja algebr w~czasoprzestrzeni}

%   \vspace{-0.8em}


%   \begin{figure}

%     \centering


%     \begin{tikzpicture}

%       % x axis
%       \draw[axis arrow] (-5,0) -- (5,0);

%       \pic at (5,0) {x mark for horizontal axis 1};


%       % t axis
%       \draw[axis arrow] (0,-2.5) -- (0,4);

%       \pic at (0,4) {t mark for vertical axis 1};



%       % Light cone
%       \fill[color=blue,opacity=0.4] (0,0) -- (3.5,3.5) -- (-3.5,3.5) -- cycle;

%       \fill[color=blue,opacity=0.4] (0,0) -- (2.2,-2.2) -- (-2.2,-2.2) --
%       cycle;

%       \draw[dashed] (2.2,-2.2) -- (-3.5,3.5);

%       \draw[dashed] (-2.2,-2.2) -- (3.5,3.5);



%       % Region number 1
%       \draw[dashed] plot[smooth cycle]
%       coordinates { (1.2,0) (3,-1) (3.5,1.4) (2.5,1.2) };

%       \node[scale=1.3] at (2.5,0.35) {$\Acal( \Ocal_{ 1 } )$};


%       % Region number 2
%       \draw[dashed] plot[smooth cycle]
%       coordinates { (0.8,0) (2.9,-1.5) (4.2,1.8) (1,3) };

%       \node[scale=1.3] at (2.9,1.9) {$\Acal( \Ocal_{ 2 } )$};


%       % Region number 3
%       \draw[dashed] plot[smooth cycle]
%       coordinates { (-1,-0.2) (-2,1.2) (-3,0.3) (-2,-1) };

%       \node[scale=1.3] at (-2.1,0.4) {$\Acal( \Ocal_{ 3 } )$};


%       % Event reaching form region 3 to region 1
%       \draw[dashed,thick,color=yellowLoc] (-1.3,0) -- (2,3.3);

%       \fill[color=brown] (-1.3,0) circle [radius=0.09];

%       \node[below] at (-1.3,-0.05) {$E$};

%     \end{tikzpicture}

%     \caption{Lokalizacja algebr w~czasoprzestrzeni.}


%   \end{figure}

% \end{frame}
% % ##################





% % ##################
% \begin{frame}
%   \frametitle{Aksjomaty AQFT}


%   \textbf{A.1. Warunek izotonii.} Ponieważ zbiór $\Ocal_{ 1 }$ zawiera~się
%   $\Ocal_{ 2 }$, więc żądamy by zachodziło
%   \begin{equation}
%     \label{eq:Niespodziewane-teoretyczne-01}
%     \Acal( \Ocal_{ 1 } ) \subset \Acal( \Ocal_{ 2 } ).
%   \end{equation}
%   Interpretacja tego jest prosta. Jeśli możemy coś zmierzyć zbiorze
%   $\Ocal_{ 1 }$, to możemy to też zmierzyć w~większym zbiorze $\Ocal_{ 2 }$.

%   \textbf{A.2. Przyczynowość Einsteina.} Zbiory $\Ocal_{ 1 }$ i~$\Ocal_{ 3 }$
%   są tak względem siebie ułożone, że~żaden obiekt podróżujący z~prędkością
%   mniejszą bądź równą $c$, nie może z~jednego dotrzeć do drugiego. Tym
%   samym eksperyment przeprowadzony w~jednym z~nich nie może wpłynąć na
%   eksperyment w~drugim. Przypominając sobie, że~dwie macierze które
%   komutują ze sobą, można jednocześnie zdiagonalizować, żądamy by zachodziło
%   \begin{equation}
%     \label{eq:Niespodziewane-teoretyczne-01}
%     [ A, B ] = 0, \quad
%     A \in \Acal( \Ocal_{ 1 } ), B \in \Acal( \Ocal_{ 3 } ).
%   \end{equation}

% \end{frame}
% % ##################





% % ##################
% \begin{frame}
%   \frametitle{Aksjomaty AQFT}


%   \textbf{A.3. Niezmienniczość Poincar\'{e}go.} Niech będzie dany obszar
%   $\Ocal$ i~transformacja Poincar\'{e}go o~nazwie $\rho$, która przekształca
%   go w~obszar $\Ucal$. Żądamy by istniał wówczas izomorfizm $\alpha( \rho )$
%   algebr $C^{ * }$:
%   \begin{equation}
%     \label{eq:Niespodziewane-teoretyczne-01}
%     \alpha( \rho ) : \Acal( \Ocal ) \to \Acal( \Ucal ).
%   \end{equation}
%   Izomorfizm $\alpha( \rho )$ musi spełniać pewne dodatkowe warunki, ich postać
%   jest teraz drugorzędna.

  % \textbf{A.4. Dynamika.} Niech $\Ocal \subset \Ucal$ i~niech $\Ocal$
  % zawiera jakiś fragment hiperpłaszczyzny $h$ zdefiniowanej równaniem
  % $t = \const$, taki że
  % \begin{equation}
  %   \label{eq:Niespodziewane-teoretyczne-01}
  %   h \cap \Ocal = h \cap \Ucal.
  % \end{equation}
  % Wówczas żądamy by zachodziło
  % \begin{equation}
  %   \label{eq:Niespodziewane-teoretyczne-01}
  %   \Acal( \Ocal ) = \Acal( \Ucal ).
  % \end{equation}
  % Ten dziwnie wyglądający warunek sprowadza~się do tego, że~jeśli
  % zadamy równaniu falowemu dane początkowe na hiperpłaszczyźnie
  % $t = \const$, to tym samym możemy znaleźć rozwiązanie obowiązujące
  % na całej przestrzeni. W~takim razie musi być możliwe odtworzenie
  % algebry w~całej przestrzeni, jeśli znamy ją na pewnym otwartym otoczeniu
  % $t = \const$.

% \end{frame}
% % ##################





% % ##################
% \begin{frame}
%   \frametitle{Aksjomaty AQFT}


  % \textbf{A.3. Niezmienniczość Poincar\'{e}go.} Niech będzie dany obszar
  % $\Ocal$ i~transformacja Poincar\'{e}go o~nazwie $\rho$, która przekształca
  % go w~zbiór $\Ucal$. Żądamy by istniał wówczas izomorfizm $\alpha( \rho )$
  % algebr $C^{ * }$:
  % \begin{equation}
  %   \label{eq:Niespodziewane-teoretyczne-01}
  %   \alpha : \Acal( \Ocal ) \to \Acal( \Ucal ).
  % \end{equation}


%   \textbf{A.4. Dynamika.} Niech $\Ocal \subset \Ucal$ i~niech $\Ocal$
%   zawiera jakiś fragment hiperpłaszczyzny $h$ zdefiniowanej równaniem
%   $t = \const$, taki że

%   \vspace{-1.5em}



%   \begin{equation}
%     \label{eq:Niespodziewane-teoretyczne-01}
%     h \cap \Ucal \subset \Ocal.
%   \end{equation}

%   \vspace{-2em}



%   Wówczas żądamy by zachodziło

%   \vspace{-1.5em}



%   \begin{equation}
%     \label{eq:Niespodziewane-teoretyczne-01}
%     \Acal( \Ocal ) = \Acal( \Ucal ).
%   \end{equation}

%   \vspace{-1.8em}



%   Ten dziwnie wyglądający warunek sprowadza~się do tego, że~jeśli
%   zadamy równaniu falowemu dane początkowe na hiperpłaszczyźnie
%   $t = \const$, to tym samym możemy znaleźć rozwiązanie obowiązujące
%   na całej przestrzeni. W~takim razie musi być możliwe odtworzenie
%   algebry w~całej przestrzeni, jeśli znamy ją na pewnym otwartym otoczeniu
%   $t = \const$.

% \end{frame}
% % ##################





% % ##################
% \begin{frame}
%   \frametitle{Podsumowanie}


%   W~tym momencie podstawowe elementy formalizmu są przedstawione, możemy
%   przystąpić do konkretnych rozważań teoretycznych.

%   Oczywiście, nic nie policzymy w~$3 + 1$ wymiarach, bo żadnego konkretnego
%   modelu nie potrafimy tu zbudować. Pozostaje liczyć na~Martina Hairer
%   i~Paweł Ducha. ;)

%   Jeśli ktoś chce~się czegoś więcej dowiedzieć, to może zapytać prof.
%   Herdegena, albo napisać do mnie pod adres \email. To jest ciekawy temat,
%   o~którym dobrze~się rozmawia.

% \end{frame}
% % ##################





% % ##################
% \begin{frame}
%   \frametitle{Koniec}

%   \vspace{7em}


%   \begin{center}

%     \Large

%     Dziękuję bardzo. \\
%     Czy są jakieś pytania?

%   \end{center}

% \end{frame}
% % ##################










% % ######################################
% \appendix
% % ######################################



% % ###################
% \begin{frame}
%   \frametitle{Literatura rozszerzająca}


%   Dobrym wprowadzeniem do zagadnień matematycznej kwantowej teorii
%   pola jest praca Stephen J.~Summersa
%   \colorhref{https://arxiv.org/abs/1203.3991}{\textit{A~Perspective
%       on~Constructive Quantum Field Theory}}, prezentująca historyczny
%   rozwój tej dziedziny
%   \parencite{Summers-Prespective-on-Constructive-ETC-Ver-2016}.

%   Podstawowy wykładem \textsc{aqft} jest książka Rudolpha Haaga
%   \textit{Local Quantum Physics. Fields, Particles, Algebras}
%   \parencite{Haag-Local-Quantum-Physics-Pub-1996}. Obecnie
%   jest ona do ściągnięcia z~tej
%   \colorhref{https://link.springer.com/book/10.1007/978-3-642-61458-3}
%   {strony}. Trzeba zauważyć, że~tekst ten zakłada u~czytelnika dużą wiedzę
%   oraz~umiejętności i~wymaga dogłębnego studiowania.

%   Krótszym, choć wciąż wymagającym wprowadzeniem do tej dziedziny
%   jest artykuł Christophera J.~Fewstera i~Kasi Rejzner
%   \colorhref{https://arxiv.org/abs/1904.04051}{\textit{Algebraic Quantum
%       Field Theory -- an~introduction}}
%   \parencite{Fewster-Rejzner-Algebraic-Quantum-Field-Theory-an-introduction-2019}. Warta uwagi jest również napisana w~formie przeglądu praca
%   Detleva Buchholza i~Klausa Fredenhagena
%   \colorhref{https://arxiv.org/abs/2305.12923}{\textit{{Algebraic quantum
%         field theory: objectives, methods, and results}}}
%   \parencite{Buchholz-Fredenhagen-Algebraic-quantum-field-theory-ETC-2023}.

% \end{frame}
% % ##################





% % ##################
% \begin{frame}
%   \frametitle{Literatura rozszerzająca}


%   Wiele można~się też dowiedzieć z~artykułów Rudolfa Haaga dostępnych
%   na serwisie \colorhref{https://arxiv.org/abs/hep-th/0001006}{arXiv}.

%   Standardową pozycją na temat algebr $C^{ * }$ i~ich zastosowania
%   w~fizyce jest książka Ola Brattelego i~Dereka Robinsona
%   \textit{Operator Algebras and~Quantum Statistical Mechanics}
%   \parencite{Bratteli-Robinson-Operator-algebras-ETC-Vol-I-Pub-2002}.
%   Lektura tej książki potrafi być jednak bardzo wymagająca.

%   Warte uwagi jest również dzieło M. Takesakiego
%   \colorhref{https://www.amazon.com/Theory-Operator-Algebras-Non-Commulative-Geometry/dp/354042248X}{\textit{Theory of Operator Algebras}}.

%   Po referencje do innych poruszonych tematów można~się zwrócić
%   pod adres \email.

% \end{frame}
% % ##################





% % ##################
% \begin{frame}
%   \frametitle{Konstrukcja teorii swobodnej}


%   Wówczas
%   \begin{equation}
%     \label{eq:Niespodziewane-teoretyczne-20}
%     \begin{split}
%       \left\langle F^{ ( 1 ) }, \frac{ 1 }{ 2 } \Delta G^{ ( 1 ) } \right\rangle( \varphi )
%       &=
%         \frac{ 1 }{ 2 }
%         \int d \mathrm{vol}( x ) \, d \mathrm{vol}( y ) \,
%         F^{ ( 1 ) }( \varphi )( x ) \\
%       &\hspace{3em} \times \Delta( x, y ) G^{ ( 1 ) }( \varphi )( y ).
%     \end{split}
%   \end{equation}
%   ma przynajmniej formalny sense. Analogicznie robimy dla wyższych
%   pochodnych.

%   Możemy teraz zdefiniować operację $\star$:
%   \begin{equation}
%     \label{eq:Niespodziewane-teoretyczne-21}
%     F \star G =
%     \sum\limits_{ i = 0 }^{ \infty } \frac{ i^{ n } \hbar^{ n } }{ n! }
%     \left\langle F^{ ( n ) }, \left( \tfrac{ 1 }{ 2 } \Delta \right)^{ \otimes n }
%       G^{ ( n ) } \right\rangle.
%     \end{equation}

% \end{frame}
% % ##################





% % ##################
% \begin{frame}
%   \frametitle{Konstrukcja teorii swobodnej}

%   Inwolucja.
%   \begin{equation}
%     \label{eq:Niespodziewane-teoretyczne-22}
%     F^{ * }( \varphi ) = \overline{ F( \varphi ) }.
%   \end{equation}

%   \textbf{Ważne.}
%   Do końca prezentacji będziemy pracować na poziomie tej algebry.

%   \textbf{Uwaga.}
%   Niemożliwość ograniczania~się do funkcjonałów liniowych wynika
%   z~własności grupy renormalizacji.

% \end{frame}
% % ##################





% % ##################
% \begin{frame}
%   \frametitle{Konstrukcja teorii z~oddziaływaniem}


%   Algebra pełnej teorii
%   \begin{equation}
%     \label{eq:Niespodziewane-teoretyczne-23}
%     S_{ \mathrm{Full} } = S_{ \mathrm{Free} } + S_{ \mathrm{I} }.
%   \end{equation}
%   Aby otrzymać teorię z~oddziaływaniem potrzebujemy jedynie
%   zmienić iloczyn $\star = \star_{ \mathrm{Free} }$ na
%   $\star_{ \mathrm{I} } = \star_{ \mathrm{Full} }$. Biorąc wzór z teorii
%   rozpraszania, możemy to zrobić za pomocą operatorów M\o llera
%   $R_{ S_{ I } }$:
%   \begin{equation}
%     \label{eq:Niespodziewane-teoretyczne-24}
%     F \star_{ \mathrm{I} } G :=
%     R_{ S_{ \mathrm{I} } }^{ -1 }( R_{ S_{ \mathrm{I} } } F
%     \star R_{ S_{ \mathrm{I} } } G ).
%   \end{equation}
%   Ale aby zdefiniować operatory M\o llera potrzebujemy iloczynu
%   chronologicznego.

%   \textbf{Problem.}
%   Iloczyn chronologiczny wymaga brania iloczynów dystrybucji, co skutkuje
%   pojawieniem się rozbieżności ultrafioletowych.

% \end{frame}
% % ##################





% % ##################
% \begin{frame}
%   \frametitle{Rozwinięcie perturbacyjne}

%   \textbf{Zagadnienie renormalizacji.}
%   Czasoprzestrzeń zakrzywiona nie ma określonej przestrzeni pędów,
%   jednak można podać odpowiednią procedurę w przestrzeni położeń.
%   Taką procedurę dla przypadku czasoprzestrzeni Minkowskiego
%   podali Epstain i~Glaser [EG73].

%   Została ona uogólniona na zadane zakrzywione czasoprzestrzenie
%   przez Brunettiego i~Fredenhagena [BF00].

% \end{frame}
% % ##################





% % ##################
% \begin{frame}
%   \frametitle{Rozwinięcie perturbacyjne}


%   Renormalizacja przyczynowa Epstaina-Glasera [EG73].
%   Polega na indukcyjnej konstrukcji iloczynów chronologicznych
%   $\Tcal_{ n }$ przy której nie pojawiają się żadne
%   rozbieżności w~ultrafiolecie. Konkretniej, rządamy by:
%   \begin{itemize}
%     \RaggedRight

%   \item[1.] $\Tcal_{ 0 } = 1$.

%   \item[2.]
%     $\Tcal_{ 1 } = \exp\left( \int d \mathrm{vol}( x ) \,
%       d \mathrm{vol}( y ) \, H( x, y )
%       \frac{ \delta^{ 2 } }{ \delta \varphi( x ) \delta \varphi( y ) } \right)$.

%   \item[3.]
%     $\Tcal_{ n }( F_{ 1 }, \ldots, F_{ n } ) =
%     \Tcal_{ k }( F_{ 1 }, \ldots, F_{ k } )
%     \star \Tcal_{ n - k }( F_{ k + 1 },\ldots, F_{ n } )$ jeżeli funkcjonały
%     $F_{ 1 }, \ldots, F_{ k }$ i~$F_{ k + 1 }, \ldots, F_{ n }$ są
%     „rozdzielone przestrzennie”.

%   \end{itemize}


%   \textbf{Grupa renormalizacji.}
%   Niejednoznaczność powyższej konstrukcji jest scharakteryzowana
%   przez grupę renormalizacji St\"{u}ckelberga-Petermana.

% \end{frame}
% % ##################





% % ##################
% \begin{frame}
%   \frametitle{Rozwinięcie perturbacyjne}


%   Formalne rozwiązanie procedury Epstaina-Glasera [FR12]
%   \begin{subequations}
%     \begin{equation}
%       \label{eq:Niespodziewane-teoretyczne-25-A}
%       \Tcal_{ n }( F_{ 1 },\ldots, F_{ n } )( \varphi ) =
%       e^{ \sum_{ i < j } D_{ i, j } } F( \varphi_{ 1 } ) \cdot \ldots
%       \cdot F_{ n }( \varphi_{ n } ) |_{ \varphi_{ 1 } = \ldots = \varphi_{ n } = \varphi }.
%     \end{equation}

%     \vspace{-3em}



%     \begin{equation}
%       \begin{split}
%         D_{ i j }
%         &= i \, \hbar \left\langle \Delta^{ F }, \frac{ \delta^{ 2 } }{ \delta \varphi_{ i } \delta \varphi_{ j } }
%           \right\rangle \\
%         &= i \, \hbar \int d \mathrm{vol}( x ) \, d \mathrm{vol}( y ) \,
%         \Delta^{ F }( x, y ) \frac{ \delta^{ 2 } }{ \delta \varphi_{ i }( x ) \delta \varphi_{ j }( y ) }.
%       \end{split}
%     \end{equation}
%   \end{subequations}

%   Można to oczywiście zapisać za pomocą grafów Feynmana.

% \end{frame}
% % ##################





% % ##################
% \begin{frame}
%   \frametitle{Podsumowanie i~perspektywy}


%   \begin{itemize}
%     \RaggedRight

%   \item Praca na poziomie algebraiczny pozwala ominąć wiele
%     trudności.

%   \item Istnieją prace dotyczące kwasiklasycznej reakcji pola
%     na~czasoprzestrzeń.

%   \item Można w~tym formalizmie opisać klasyczną teorię pola
%     i~wówczas da się otrzymać wyniki o zbieżności badanych
%     szeregów.

%   \item Pełna Renormalizacja Epstaina-Glasera jest dobrze umotywowana
%     fizycznie, lecz w~realnych obliczeniach jest niezwykle skomplikowana.
%     Kluczowe więc dla całego podejścia jest to czy da się uprościć tę
%     część teorii. Najnowsze prace skupiają się nad możliwością użycia
%     kombinacji procedury E-G z~renormalizacją wymiarową, jednak nad
%     zakrzywionych czasoprzestrzeniach wymaga to sformułowania
%     w~reprezentacji położeń. Do tej pory osiągnięto pozytywne efekty
%     w~czasoprzestrzeni Minkowskiego [DFKR13].

%   \end{itemize}

% \end{frame}
% % ##################





% % ##################
% \begin{frame}[standout]


%   { \color{jFrametitleFGColor} Nietypowe problemy mogą wymagać
%     niestandardowych rozwiązań. Czy to konkretne jest właściwe to czas
%     pokaże. }

% \end{frame}
% % ##################
















% % ######################################
% \EndingSlide{Pytania? Dziękuję za uwagę.}
% % ######################################










% % ##################
% \begin{frame}
%   \frametitle{Bibliografia}


%   \begin{itemize}
%     \RaggedRight

%   \item [Fre09] K. Fredenhagen, \textit{The impact of~the~algebraic
%       approach on perturbative quantum field theory}.
%     Wystąpienie na \textit{Algebraic Quantum Field Theory~– the first 50
%       Years}, G\"{o}ttingen 2009.

%   \item [Wal09] R. M. Wald, textit{Axiomatic Quantum Field Theory
%       in~Curved Spacetime}. Wystąpienie na \textit{Algebraic
%       Quantum Field Theory~– the first 50 Years}, G\"{o}ttingen 2009.

%   \item [Haa09] R. Haag, \textit{Local Algebras: A~look back at
%     the~early years and~at~some successes and~missed opportunitie}.
%     Wystąpienie na \textit{Algebraic Quantum Field Theory~– the first 50
%     Years}, G\"{o}ttingen 2009.

%   \item [FR12] K. Fredenhagen, K. Rejzner, \textit{Perturbative
%     algebraic quantum field theory}, arXiv: 1208.1428.

%   \end{itemize}

% \end{frame}
% % ##################





% % ##################
% \begin{frame}
%   \frametitle{Bibliografia}


%   \begin{itemize}
%     \RaggedRight

%   \item [EG73] H. Epstein, V. Glaser, \textit{The role of locality
%     in~perturbation theory}, Ann. Inst. H. Poincar\'{e} A
%     \textbf{19} (1973) 211.

%   \item [BF00] R. Brunetti, K. Fredenhagen, \textit{Microlocal Analysis
%     and~Interacting Quantum Field Theories: Renormalization
%     on~Physical Backgrounds}, Commun.Math.Phys, \textbf{208} (2000)
%     623-661, arXiv: 9903.028.

%   \item [BF09] R. Brunetti, K. Fredenhagen, \textit{Quantum Field
%     Theory on~Curved Backgrounds}, Proceedings of the
%     Kompaktkurs \textit{Quantenfeldtheorie auf gekruemmten Raumzeiten} held
%     at~Universitaet Potsdam, Germany, in 8.-12.10.2007, arXiv:
%     0901.2063.

%   \item [DFKR13] M. D\"{u}etsch, K. Fredenhagen, K. J. Keller,
%     K.~Rejzner, \textit{Dimensional Regularization in Position Space,
%       and~a Forest Formula for Epstein-Glaser Renormalization}, arXiv:
%     1311.5424.

%   \end{itemize}

% \end{frame}
% % ##################










% #####################################################################
% #####################################################################
% Bibliography

\printbibliography





% ############################

% Koniec dokumentu
\end{document}
