% ------------------------------------------------------------------------------------------------------------------
% Basic configuration of Beamera class and Jagiellonian theme
% ------------------------------------------------------------------------------------------------------------------
\RequirePackage[l2tabu, orthodox]{nag}



\ifx\PresentationStyle\notset
  \def\PresentationStyle{dark}
\fi



% Options: t -- align frame text to the top
\documentclass[10pt,t]{beamer}
\mode<presentation>
\usetheme[style=\PresentationStyle]{jagiellonian}





% ------------------------------------------------------------------------------------
% Procesing configuration files of Jagiellonian theme located in
% the directory "preambule"
% ------------------------------------------------------------------------------------
% Configuration for polish language
% Need description
\usepackage[polish]{babel}
% Need description
\usepackage[MeX]{polski}



% ------------------------------
% Better support of polish chars in technical parts of PDF
% ------------------------------
\hypersetup{pdfencoding=auto,psdextra}

% Package "textpos" give as enviroment "textblock" which is very usefull in
% arranging text on slides.

% This is standard configuration of "textpos"
\usepackage[overlay,absolute]{textpos}

% If you need to see bounds of "textblock's" comment line above and uncomment
% one below.

% Caution! When showboxes option is on significant ammunt of space is add
% to the top of textblock and as such, everyting put in them gone down.
% We need to check how to remove this bug.

% \usepackage[showboxes,overlay,absolute]{textpos}



% Setting scale length for package "textpos"
\setlength{\TPHorizModule}{10mm}
\setlength{\TPVertModule}{\TPHorizModule}


% ---------------------------------------
% TikZ
% ---------------------------------------
% Importing TikZ libraries
\usetikzlibrary{arrows.meta}
\usetikzlibrary{positioning}





% % Configuration package "bm" that need for making bold symbols
% \newcommand{\bmmax}{0}
% \newcommand{\hmmax}{0}
% \usepackage{bm}




% ---------------------------------------
% Packages for scientific texts
% ---------------------------------------
% \let\lll\undefined  % Sometimes you must use this line to allow
% "amsmath" package to works with packages with packages for polish
% languge imported
% /preambul/LanguageSettings/JagiellonianPolishLanguageSettings.tex.
% This comments (probably) removes polish letter Ł.
\usepackage{amsmath}  % Packages from American Mathematical Society (AMS)
\usepackage{amssymb}
\usepackage{amscd}
\usepackage{amsthm}
\usepackage{siunitx}  % Package for typsetting SI units.
\usepackage{upgreek}  % Better looking greek letters.
% Example of using upgreek: pi = \uppi


\usepackage{calrsfs}  % Zmienia czcionkę kaligraficzną w \mathcal
% na ładniejszą. Może w innych miejscach robi to samo, ale o tym nic
% nie wiem.










% ---------------------------------------
% Packages written for lectures "Geometria 3D dla twórców gier wideo"
% ---------------------------------------
% \usepackage{./ProgramowanieSymulacjiFizykiPaczki/ProgramowanieSymulacjiFizyki}
% \usepackage{./ProgramowanieSymulacjiFizykiPaczki/ProgramowanieSymulacjiFizykiIndeksy}
% \usepackage{./ProgramowanieSymulacjiFizykiPaczki/ProgramowanieSymulacjiFizykiTikZStyle}





% !!!!!!!!!!!!!!!!!!!!!!!!!!!!!!
% !!!!!!!!!!!!!!!!!!!!!!!!!!!!!!
% EVIL STUFF
\if\JUlogotitle1
\edef\LogoJUPath{LogoJU_\JUlogoLang/LogoJU_\JUlogoShape_\JUlogoColor.pdf}
\titlegraphic{\hfill\includegraphics[scale=0.22]
{./JagiellonianPictures/\LogoJUPath}}
\fi
% ---------------------------------------
% Commands for handling colors
% ---------------------------------------


% Command for setting normal text color for some text in math modestyle
% Text color depend on used style of Jagiellonian

% Beamer version of command
\newcommand{\TextWithNormalTextColor}[1]{%
  {\color{jNormalTextFGColor}
    \setbeamercolor{math text}{fg=jNormalTextFGColor} {#1}}
}

% Article and similar classes version of command
% \newcommand{\TextWithNormalTextColor}[1]{%
%   {\color{jNormalTextsFGColor} {#1}}
% }



% Beamer version of command
\newcommand{\NormalTextInMathMode}[1]{%
  {\color{jNormalTextFGColor}
    \setbeamercolor{math text}{fg=jNormalTextFGColor} \text{#1}}
}


% Article and similar classes version of command
% \newcommand{\NormalTextInMathMode}[1]{%
%   {\color{jNormalTextsFGColor} \text{#1}}
% }




% Command that sets color of some mathematical text to the same color
% that has normal text in header (?)

% Beamer version of the command
\newcommand{\MathTextFrametitleFGColor}[1]{%
  {\color{jFrametitleFGColor}
    \setbeamercolor{math text}{fg=jFrametitleFGColor} #1}
}

% Article and similar classes version of the command
% \newcommand{\MathTextWhiteColor}[1]{{\color{jFrametitleFGColor} #1}}





% Command for setting color of alert text for some text in math modestyle

% Beamer version of the command
\newcommand{\MathTextAlertColor}[1]{%
  {\color{jOrange} \setbeamercolor{math text}{fg=jOrange} #1}
}

% Article and similar classes version of the command
% \newcommand{\MathTextAlertColor}[1]{{\color{jOrange} #1}}





% Command that allow you to sets chosen color as the color of some text into
% math mode. Due to some nuances in the way that Beamer handle colors
% it not work in all cases. We hope that in the future we will improve it.

% Beamer version of the command
\newcommand{\SetMathTextColor}[2]{%
  {\color{#1} \setbeamercolor{math text}{fg=#1} #2}
}


% Article and similar classes version of the command
% \newcommand{\SetMathTextColor}[2]{{\color{#1} #2}}










% ---------------------------------------
% Commands for few special slides
% ---------------------------------------
\newcommand{\EndingSlide}[1]{%
  \begin{frame}[standout]

    \begingroup

    \color{jFrametitleFGColor}

    #1

    \endgroup

  \end{frame}
}










% ---------------------------------------
% Commands for setting background pictures for some slides
% ---------------------------------------
\newcommand{\TitleBackgroundPicture}
{./JagiellonianPictures/Backgrounds/LajkonikDark.png}
\newcommand{\SectionBackgroundPicture}
{./JagiellonianPictures/Backgrounds/LajkonikLight.png}



\newcommand{\TitleSlideWithPicture}{%
  \begingroup

  \usebackgroundtemplate{%
    \includegraphics[height=\paperheight]{\TitleBackgroundPicture}}

  \maketitle

  \endgroup
}





\newcommand{\SectionSlideWithPicture}[1]{%
  \begingroup

  \usebackgroundtemplate{%
    \includegraphics[height=\paperheight]{\SectionBackgroundPicture}}

  \setbeamercolor{titlelike}{fg=normal text.fg}

  \section{#1}

  \endgroup
}










% ---------------------------------------
% Commands for lectures "Geometria 3D dla twórców gier wideo"
% Polish version
% ---------------------------------------
% Komendy teraz wykomentowane były potrzebne, gdy loga były na niebieskim
% tle, nie na białym. A są na białym bo tego chcieli w biurze projektu.
% \newcommand{\FundingLogoWhitePicturePL}
% {./PresentationPictures/CommonPictures/logotypFundusze_biale_bez_tla2.pdf}
\newcommand{\FundingLogoColorPicturePL}
{./PresentationPictures/CommonPictures/European_Funds_color_PL.pdf}
% \newcommand{\EULogoWhitePicturePL}
% {./PresentationPictures/CommonPictures/logotypUE_biale_bez_tla2.pdf}
\newcommand{\EUSocialFundLogoColorPicturePL}
{./PresentationPictures/CommonPictures/EU_Social_Fund_color_PL.pdf}
% \newcommand{\ZintegrUJLogoWhitePicturePL}
% {./PresentationPictures/CommonPictures/zintegruj-logo-white.pdf}
\newcommand{\ZintegrUJLogoColorPicturePL}
{./PresentationPictures/CommonPictures/ZintegrUJ_color.pdf}
\newcommand{\JULogoColorPicturePL}
{./JagiellonianPictures/LogoJU_PL/LogoJU_A_color.pdf}





\newcommand{\GeometryThreeDSpecialBeginningSlidePL}{%
  \begin{frame}[standout]

    \begin{textblock}{11}(1,0.7)

      \begin{flushleft}

        \mdseries

        \footnotesize

        \color{jFrametitleFGColor}

        Materiał powstał w ramach projektu współfinansowanego ze środków
        Unii Europejskiej w ramach Europejskiego Funduszu Społecznego
        POWR.03.05.00-00-Z309/17-00.

      \end{flushleft}

    \end{textblock}





    \begin{textblock}{10}(0,2.2)

      \tikz \fill[color=jBackgroundStyleLight] (0,0) rectangle (12.8,-1.5);

    \end{textblock}


    \begin{textblock}{3.2}(1,2.45)

      \includegraphics[scale=0.3]{\FundingLogoColorPicturePL}

    \end{textblock}


    \begin{textblock}{2.5}(3.7,2.5)

      \includegraphics[scale=0.2]{\JULogoColorPicturePL}

    \end{textblock}


    \begin{textblock}{2.5}(6,2.4)

      \includegraphics[scale=0.1]{\ZintegrUJLogoColorPicturePL}

    \end{textblock}


    \begin{textblock}{4.2}(8.4,2.6)

      \includegraphics[scale=0.3]{\EUSocialFundLogoColorPicturePL}

    \end{textblock}

  \end{frame}
}



\newcommand{\GeometryThreeDTwoSpecialBeginningSlidesPL}{%
  \begin{frame}[standout]

    \begin{textblock}{11}(1,0.7)

      \begin{flushleft}

        \mdseries

        \footnotesize

        \color{jFrametitleFGColor}

        Materiał powstał w ramach projektu współfinansowanego ze środków
        Unii Europejskiej w ramach Europejskiego Funduszu Społecznego
        POWR.03.05.00-00-Z309/17-00.

      \end{flushleft}

    \end{textblock}





    \begin{textblock}{10}(0,2.2)

      \tikz \fill[color=jBackgroundStyleLight] (0,0) rectangle (12.8,-1.5);

    \end{textblock}


    \begin{textblock}{3.2}(1,2.45)

      \includegraphics[scale=0.3]{\FundingLogoColorPicturePL}

    \end{textblock}


    \begin{textblock}{2.5}(3.7,2.5)

      \includegraphics[scale=0.2]{\JULogoColorPicturePL}

    \end{textblock}


    \begin{textblock}{2.5}(6,2.4)

      \includegraphics[scale=0.1]{\ZintegrUJLogoColorPicturePL}

    \end{textblock}


    \begin{textblock}{4.2}(8.4,2.6)

      \includegraphics[scale=0.3]{\EUSocialFundLogoColorPicturePL}

    \end{textblock}

  \end{frame}





  \TitleSlideWithPicture
}



\newcommand{\GeometryThreeDSpecialEndingSlidePL}{%
  \begin{frame}[standout]

    \begin{textblock}{11}(1,0.7)

      \begin{flushleft}

        \mdseries

        \footnotesize

        \color{jFrametitleFGColor}

        Materiał powstał w ramach projektu współfinansowanego ze środków
        Unii Europejskiej w~ramach Europejskiego Funduszu Społecznego
        POWR.03.05.00-00-Z309/17-00.

      \end{flushleft}

    \end{textblock}





    \begin{textblock}{10}(0,2.2)

      \tikz \fill[color=jBackgroundStyleLight] (0,0) rectangle (12.8,-1.5);

    \end{textblock}


    \begin{textblock}{3.2}(1,2.45)

      \includegraphics[scale=0.3]{\FundingLogoColorPicturePL}

    \end{textblock}


    \begin{textblock}{2.5}(3.7,2.5)

      \includegraphics[scale=0.2]{\JULogoColorPicturePL}

    \end{textblock}


    \begin{textblock}{2.5}(6,2.4)

      \includegraphics[scale=0.1]{\ZintegrUJLogoColorPicturePL}

    \end{textblock}


    \begin{textblock}{4.2}(8.4,2.6)

      \includegraphics[scale=0.3]{\EUSocialFundLogoColorPicturePL}

    \end{textblock}





    \begin{textblock}{11}(1,4)

      \begin{flushleft}

        \mdseries

        \footnotesize

        \RaggedRight

        \color{jFrametitleFGColor}

        Treść niniejszego wykładu jest udostępniona na~licencji
        Creative Commons (\textsc{cc}), z~uzna\-niem autorstwa
        (\textsc{by}) oraz udostępnianiem na tych samych warunkach
        (\textsc{sa}). Rysunki i~wy\-kresy zawarte w~wykładzie są
        autorstwa dr.~hab.~Pawła Węgrzyna et~al. i~są dostępne
        na tej samej licencji, o~ile nie wskazano inaczej.
        W~prezentacji wykorzystano temat Beamera Jagiellonian,
        oparty na~temacie Metropolis Matthiasa Vogelgesanga,
        dostępnym na licencji \LaTeX{} Project Public License~1.3c
        pod adresem: \colorhref{https://github.com/matze/mtheme}
        {https://github.com/matze/mtheme}.

        Projekt typograficzny: Iwona Grabska-Gradzińska \\
        Skład: Kamil Ziemian;
        Korekta: Wojciech Palacz \\
        Modele: Dariusz Frymus, Kamil Nowakowski \\
        Rysunki i~wykresy: Kamil Ziemian, Paweł Węgrzyn, Wojciech Palacz

      \end{flushleft}

    \end{textblock}

  \end{frame}
}



\newcommand{\GeometryThreeDTwoSpecialEndingSlidesPL}[1]{%
  \begin{frame}[standout]


    \begin{textblock}{11}(1,0.7)

      \begin{flushleft}

        \mdseries

        \footnotesize

        \color{jFrametitleFGColor}

        Materiał powstał w ramach projektu współfinansowanego ze środków
        Unii Europejskiej w~ramach Europejskiego Funduszu Społecznego
        POWR.03.05.00-00-Z309/17-00.

      \end{flushleft}

    \end{textblock}





    \begin{textblock}{10}(0,2.2)

      \tikz \fill[color=jBackgroundStyleLight] (0,0) rectangle (12.8,-1.5);

    \end{textblock}


    \begin{textblock}{3.2}(1,2.45)

      \includegraphics[scale=0.3]{\FundingLogoColorPicturePL}

    \end{textblock}


    \begin{textblock}{2.5}(3.7,2.5)

      \includegraphics[scale=0.2]{\JULogoColorPicturePL}

    \end{textblock}


    \begin{textblock}{2.5}(6,2.4)

      \includegraphics[scale=0.1]{\ZintegrUJLogoColorPicturePL}

    \end{textblock}


    \begin{textblock}{4.2}(8.4,2.6)

      \includegraphics[scale=0.3]{\EUSocialFundLogoColorPicturePL}

    \end{textblock}





    \begin{textblock}{11}(1,4)

      \begin{flushleft}

        \mdseries

        \footnotesize

        \RaggedRight

        \color{jFrametitleFGColor}

        Treść niniejszego wykładu jest udostępniona na~licencji
        Creative Commons (\textsc{cc}), z~uzna\-niem autorstwa
        (\textsc{by}) oraz udostępnianiem na tych samych warunkach
        (\textsc{sa}). Rysunki i~wy\-kresy zawarte w~wykładzie są
        autorstwa dr.~hab.~Pawła Węgrzyna et~al. i~są dostępne
        na tej samej licencji, o~ile nie wskazano inaczej.
        W~prezentacji wykorzystano temat Beamera Jagiellonian,
        oparty na~temacie Metropolis Matthiasa Vogelgesanga,
        dostępnym na licencji \LaTeX{} Project Public License~1.3c
        pod adresem: \colorhref{https://github.com/matze/mtheme}
        {https://github.com/matze/mtheme}.

        Projekt typograficzny: Iwona Grabska-Gradzińska \\
        Skład: Kamil Ziemian;
        Korekta: Wojciech Palacz \\
        Modele: Dariusz Frymus, Kamil Nowakowski \\
        Rysunki i~wykresy: Kamil Ziemian, Paweł Węgrzyn, Wojciech Palacz

      \end{flushleft}

    \end{textblock}

  \end{frame}





  \begin{frame}[standout]

    \begingroup

    \color{jFrametitleFGColor}

    #1

    \endgroup

  \end{frame}
}



\newcommand{\GeometryThreeDSpecialEndingSlideVideoPL}{%
  \begin{frame}[standout]

    \begin{textblock}{11}(1,0.7)

      \begin{flushleft}

        \mdseries

        \footnotesize

        \color{jFrametitleFGColor}

        Materiał powstał w ramach projektu współfinansowanego ze środków
        Unii Europejskiej w~ramach Europejskiego Funduszu Społecznego
        POWR.03.05.00-00-Z309/17-00.

      \end{flushleft}

    \end{textblock}





    \begin{textblock}{10}(0,2.2)

      \tikz \fill[color=jBackgroundStyleLight] (0,0) rectangle (12.8,-1.5);

    \end{textblock}


    \begin{textblock}{3.2}(1,2.45)

      \includegraphics[scale=0.3]{\FundingLogoColorPicturePL}

    \end{textblock}


    \begin{textblock}{2.5}(3.7,2.5)

      \includegraphics[scale=0.2]{\JULogoColorPicturePL}

    \end{textblock}


    \begin{textblock}{2.5}(6,2.4)

      \includegraphics[scale=0.1]{\ZintegrUJLogoColorPicturePL}

    \end{textblock}


    \begin{textblock}{4.2}(8.4,2.6)

      \includegraphics[scale=0.3]{\EUSocialFundLogoColorPicturePL}

    \end{textblock}





    \begin{textblock}{11}(1,4)

      \begin{flushleft}

        \mdseries

        \footnotesize

        \RaggedRight

        \color{jFrametitleFGColor}

        Treść niniejszego wykładu jest udostępniona na~licencji
        Creative Commons (\textsc{cc}), z~uzna\-niem autorstwa
        (\textsc{by}) oraz udostępnianiem na tych samych warunkach
        (\textsc{sa}). Rysunki i~wy\-kresy zawarte w~wykładzie są
        autorstwa dr.~hab.~Pawła Węgrzyna et~al. i~są dostępne
        na tej samej licencji, o~ile nie wskazano inaczej.
        W~prezentacji wykorzystano temat Beamera Jagiellonian,
        oparty na~temacie Metropolis Matthiasa Vogelgesanga,
        dostępnym na licencji \LaTeX{} Project Public License~1.3c
        pod adresem: \colorhref{https://github.com/matze/mtheme}
        {https://github.com/matze/mtheme}.

        Projekt typograficzny: Iwona Grabska-Gradzińska;
        Skład: Kamil Ziemian \\
        Korekta: Wojciech Palacz;
        Modele: Dariusz Frymus, Kamil Nowakowski \\
        Rysunki i~wykresy: Kamil Ziemian, Paweł Węgrzyn, Wojciech Palacz \\
        Montaż: Agencja Filmowa Film \& Television Production~-- Zbigniew
        Masklak

      \end{flushleft}

    \end{textblock}

  \end{frame}
}





\newcommand{\GeometryThreeDTwoSpecialEndingSlidesVideoPL}[1]{%
  \begin{frame}[standout]

    \begin{textblock}{11}(1,0.7)

      \begin{flushleft}

        \mdseries

        \footnotesize

        \color{jFrametitleFGColor}

        Materiał powstał w ramach projektu współfinansowanego ze środków
        Unii Europejskiej w~ramach Europejskiego Funduszu Społecznego
        POWR.03.05.00-00-Z309/17-00.

      \end{flushleft}

    \end{textblock}





    \begin{textblock}{10}(0,2.2)

      \tikz \fill[color=jBackgroundStyleLight] (0,0) rectangle (12.8,-1.5);

    \end{textblock}


    \begin{textblock}{3.2}(1,2.45)

      \includegraphics[scale=0.3]{\FundingLogoColorPicturePL}

    \end{textblock}


    \begin{textblock}{2.5}(3.7,2.5)

      \includegraphics[scale=0.2]{\JULogoColorPicturePL}

    \end{textblock}


    \begin{textblock}{2.5}(6,2.4)

      \includegraphics[scale=0.1]{\ZintegrUJLogoColorPicturePL}

    \end{textblock}


    \begin{textblock}{4.2}(8.4,2.6)

      \includegraphics[scale=0.3]{\EUSocialFundLogoColorPicturePL}

    \end{textblock}





    \begin{textblock}{11}(1,4)

      \begin{flushleft}

        \mdseries

        \footnotesize

        \RaggedRight

        \color{jFrametitleFGColor}

        Treść niniejszego wykładu jest udostępniona na~licencji
        Creative Commons (\textsc{cc}), z~uzna\-niem autorstwa
        (\textsc{by}) oraz udostępnianiem na tych samych warunkach
        (\textsc{sa}). Rysunki i~wy\-kresy zawarte w~wykładzie są
        autorstwa dr.~hab.~Pawła Węgrzyna et~al. i~są dostępne
        na tej samej licencji, o~ile nie wskazano inaczej.
        W~prezentacji wykorzystano temat Beamera Jagiellonian,
        oparty na~temacie Metropolis Matthiasa Vogelgesanga,
        dostępnym na licencji \LaTeX{} Project Public License~1.3c
        pod adresem: \colorhref{https://github.com/matze/mtheme}
        {https://github.com/matze/mtheme}.

        Projekt typograficzny: Iwona Grabska-Gradzińska;
        Skład: Kamil Ziemian \\
        Korekta: Wojciech Palacz;
        Modele: Dariusz Frymus, Kamil Nowakowski \\
        Rysunki i~wykresy: Kamil Ziemian, Paweł Węgrzyn, Wojciech Palacz \\
        Montaż: Agencja Filmowa Film \& Television Production~-- Zbigniew
        Masklak

      \end{flushleft}

    \end{textblock}

  \end{frame}





  \begin{frame}[standout]


    \begingroup

    \color{jFrametitleFGColor}

    #1

    \endgroup

  \end{frame}
}










% ---------------------------------------
% Commands for lectures "Geometria 3D dla twórców gier wideo"
% English version
% ---------------------------------------
% \newcommand{\FundingLogoWhitePictureEN}
% {./PresentationPictures/CommonPictures/logotypFundusze_biale_bez_tla2.pdf}
\newcommand{\FundingLogoColorPictureEN}
{./PresentationPictures/CommonPictures/European_Funds_color_EN.pdf}
% \newcommand{\EULogoWhitePictureEN}
% {./PresentationPictures/CommonPictures/logotypUE_biale_bez_tla2.pdf}
\newcommand{\EUSocialFundLogoColorPictureEN}
{./PresentationPictures/CommonPictures/EU_Social_Fund_color_EN.pdf}
% \newcommand{\ZintegrUJLogoWhitePictureEN}
% {./PresentationPictures/CommonPictures/zintegruj-logo-white.pdf}
\newcommand{\ZintegrUJLogoColorPictureEN}
{./PresentationPictures/CommonPictures/ZintegrUJ_color.pdf}
\newcommand{\JULogoColorPictureEN}
{./JagiellonianPictures/LogoJU_EN/LogoJU_A_color.pdf}



\newcommand{\GeometryThreeDSpecialBeginningSlideEN}{%
  \begin{frame}[standout]

    \begin{textblock}{11}(1,0.7)

      \begin{flushleft}

        \mdseries

        \footnotesize

        \color{jFrametitleFGColor}

        This content was created as part of a project co-financed by the
        European Union within the framework of the European Social Fund
        POWR.03.05.00-00-Z309/17-00.

      \end{flushleft}

    \end{textblock}





    \begin{textblock}{10}(0,2.2)

      \tikz \fill[color=jBackgroundStyleLight] (0,0) rectangle (12.8,-1.5);

    \end{textblock}


    \begin{textblock}{3.2}(0.7,2.45)

      \includegraphics[scale=0.3]{\FundingLogoColorPictureEN}

    \end{textblock}


    \begin{textblock}{2.5}(4.15,2.5)

      \includegraphics[scale=0.2]{\JULogoColorPictureEN}

    \end{textblock}


    \begin{textblock}{2.5}(6.35,2.4)

      \includegraphics[scale=0.1]{\ZintegrUJLogoColorPictureEN}

    \end{textblock}


    \begin{textblock}{4.2}(8.4,2.6)

      \includegraphics[scale=0.3]{\EUSocialFundLogoColorPictureEN}

    \end{textblock}

  \end{frame}
}



\newcommand{\GeometryThreeDTwoSpecialBeginningSlidesEN}{%
  \begin{frame}[standout]

    \begin{textblock}{11}(1,0.7)

      \begin{flushleft}

        \mdseries

        \footnotesize

        \color{jFrametitleFGColor}

        This content was created as part of a project co-financed by the
        European Union within the framework of the European Social Fund
        POWR.03.05.00-00-Z309/17-00.

      \end{flushleft}

    \end{textblock}





    \begin{textblock}{10}(0,2.2)

      \tikz \fill[color=jBackgroundStyleLight] (0,0) rectangle (12.8,-1.5);

    \end{textblock}


    \begin{textblock}{3.2}(0.7,2.45)

      \includegraphics[scale=0.3]{\FundingLogoColorPictureEN}

    \end{textblock}


    \begin{textblock}{2.5}(4.15,2.5)

      \includegraphics[scale=0.2]{\JULogoColorPictureEN}

    \end{textblock}


    \begin{textblock}{2.5}(6.35,2.4)

      \includegraphics[scale=0.1]{\ZintegrUJLogoColorPictureEN}

    \end{textblock}


    \begin{textblock}{4.2}(8.4,2.6)

      \includegraphics[scale=0.3]{\EUSocialFundLogoColorPictureEN}

    \end{textblock}

  \end{frame}





  \TitleSlideWithPicture
}



\newcommand{\GeometryThreeDSpecialEndingSlideEN}{%
  \begin{frame}[standout]

    \begin{textblock}{11}(1,0.7)

      \begin{flushleft}

        \mdseries

        \footnotesize

        \color{jFrametitleFGColor}

        This content was created as part of a project co-financed by the
        European Union within the framework of the European Social Fund
        POWR.03.05.00-00-Z309/17-00.

      \end{flushleft}

    \end{textblock}





    \begin{textblock}{10}(0,2.2)

      \tikz \fill[color=jBackgroundStyleLight] (0,0) rectangle (12.8,-1.5);

    \end{textblock}


    \begin{textblock}{3.2}(0.7,2.45)

      \includegraphics[scale=0.3]{\FundingLogoColorPictureEN}

    \end{textblock}


    \begin{textblock}{2.5}(4.15,2.5)

      \includegraphics[scale=0.2]{\JULogoColorPictureEN}

    \end{textblock}


    \begin{textblock}{2.5}(6.35,2.4)

      \includegraphics[scale=0.1]{\ZintegrUJLogoColorPictureEN}

    \end{textblock}


    \begin{textblock}{4.2}(8.4,2.6)

      \includegraphics[scale=0.3]{\EUSocialFundLogoColorPictureEN}

    \end{textblock}





    \begin{textblock}{11}(1,4)

      \begin{flushleft}

        \mdseries

        \footnotesize

        \RaggedRight

        \color{jFrametitleFGColor}

        The content of this lecture is made available under a~Creative
        Commons licence (\textsc{cc}), giving the author the credits
        (\textsc{by}) and putting an obligation to share on the same terms
        (\textsc{sa}). Figures and diagrams included in the lecture are
        authored by Paweł Węgrzyn et~al., and are available under the same
        license unless indicated otherwise.\\ The presentation uses the
        Beamer Jagiellonian theme based on Matthias Vogelgesang’s
        Metropolis theme, available under license \LaTeX{} Project
        Public License~1.3c at: \colorhref{https://github.com/matze/mtheme}
        {https://github.com/matze/mtheme}.

        Typographic design: Iwona Grabska-Gradzińska \\
        \LaTeX{} Typesetting: Kamil Ziemian \\
        Proofreading: Wojciech Palacz,
        Monika Stawicka \\
        3D Models: Dariusz Frymus, Kamil Nowakowski \\
        Figures and charts: Kamil Ziemian, Paweł Węgrzyn, Wojciech Palacz

      \end{flushleft}

    \end{textblock}

  \end{frame}
}



\newcommand{\GeometryThreeDTwoSpecialEndingSlidesEN}[1]{%
  \begin{frame}[standout]


    \begin{textblock}{11}(1,0.7)

      \begin{flushleft}

        \mdseries

        \footnotesize

        \color{jFrametitleFGColor}

        This content was created as part of a project co-financed by the
        European Union within the framework of the European Social Fund
        POWR.03.05.00-00-Z309/17-00.

      \end{flushleft}

    \end{textblock}





    \begin{textblock}{10}(0,2.2)

      \tikz \fill[color=jBackgroundStyleLight] (0,0) rectangle (12.8,-1.5);

    \end{textblock}


    \begin{textblock}{3.2}(0.7,2.45)

      \includegraphics[scale=0.3]{\FundingLogoColorPictureEN}

    \end{textblock}


    \begin{textblock}{2.5}(4.15,2.5)

      \includegraphics[scale=0.2]{\JULogoColorPictureEN}

    \end{textblock}


    \begin{textblock}{2.5}(6.35,2.4)

      \includegraphics[scale=0.1]{\ZintegrUJLogoColorPictureEN}

    \end{textblock}


    \begin{textblock}{4.2}(8.4,2.6)

      \includegraphics[scale=0.3]{\EUSocialFundLogoColorPictureEN}

    \end{textblock}





    \begin{textblock}{11}(1,4)

      \begin{flushleft}

        \mdseries

        \footnotesize

        \RaggedRight

        \color{jFrametitleFGColor}

        The content of this lecture is made available under a~Creative
        Commons licence (\textsc{cc}), giving the author the credits
        (\textsc{by}) and putting an obligation to share on the same terms
        (\textsc{sa}). Figures and diagrams included in the lecture are
        authored by Paweł Węgrzyn et~al., and are available under the same
        license unless indicated otherwise.\\ The presentation uses the
        Beamer Jagiellonian theme based on Matthias Vogelgesang’s
        Metropolis theme, available under license \LaTeX{} Project
        Public License~1.3c at: \colorhref{https://github.com/matze/mtheme}
        {https://github.com/matze/mtheme}.

        Typographic design: Iwona Grabska-Gradzińska \\
        \LaTeX{} Typesetting: Kamil Ziemian \\
        Proofreading: Wojciech Palacz,
        Monika Stawicka \\
        3D Models: Dariusz Frymus, Kamil Nowakowski \\
        Figures and charts: Kamil Ziemian, Paweł Węgrzyn, Wojciech Palacz

      \end{flushleft}

    \end{textblock}

  \end{frame}





  \begin{frame}[standout]

    \begingroup

    \color{jFrametitleFGColor}

    #1

    \endgroup

  \end{frame}
}



\newcommand{\GeometryThreeDSpecialEndingSlideVideoVerOneEN}{%
  \begin{frame}[standout]

    \begin{textblock}{11}(1,0.7)

      \begin{flushleft}

        \mdseries

        \footnotesize

        \color{jFrametitleFGColor}

        This content was created as part of a project co-financed by the
        European Union within the framework of the European Social Fund
        POWR.03.05.00-00-Z309/17-00.

      \end{flushleft}

    \end{textblock}





    \begin{textblock}{10}(0,2.2)

      \tikz \fill[color=jBackgroundStyleLight] (0,0) rectangle (12.8,-1.5);

    \end{textblock}


    \begin{textblock}{3.2}(0.7,2.45)

      \includegraphics[scale=0.3]{\FundingLogoColorPictureEN}

    \end{textblock}


    \begin{textblock}{2.5}(4.15,2.5)

      \includegraphics[scale=0.2]{\JULogoColorPictureEN}

    \end{textblock}


    \begin{textblock}{2.5}(6.35,2.4)

      \includegraphics[scale=0.1]{\ZintegrUJLogoColorPictureEN}

    \end{textblock}


    \begin{textblock}{4.2}(8.4,2.6)

      \includegraphics[scale=0.3]{\EUSocialFundLogoColorPictureEN}

    \end{textblock}





    \begin{textblock}{11}(1,4)

      \begin{flushleft}

        \mdseries

        \footnotesize

        \RaggedRight

        \color{jFrametitleFGColor}

        The content of this lecture is made available under a Creative
        Commons licence (\textsc{cc}), giving the author the credits
        (\textsc{by}) and putting an obligation to share on the same terms
        (\textsc{sa}). Figures and diagrams included in the lecture are
        authored by Paweł Węgrzyn et~al., and are available under the same
        license unless indicated otherwise.\\ The presentation uses the
        Beamer Jagiellonian theme based on Matthias Vogelgesang’s
        Metropolis theme, available under license \LaTeX{} Project
        Public License~1.3c at: \colorhref{https://github.com/matze/mtheme}
        {https://github.com/matze/mtheme}.

        Typographic design: Iwona Grabska-Gradzińska;
        \LaTeX{} Typesetting: Kamil Ziemian \\
        Proofreading: Wojciech Palacz,
        Monika Stawicka \\
        3D Models: Dariusz Frymus, Kamil Nowakowski \\
        Figures and charts: Kamil Ziemian, Paweł Węgrzyn, Wojciech
        Palacz \\
        Film editing: Agencja Filmowa Film \& Television Production~--
        Zbigniew Masklak

      \end{flushleft}

    \end{textblock}

  \end{frame}
}



\newcommand{\GeometryThreeDSpecialEndingSlideVideoVerTwoEN}{%
  \begin{frame}[standout]

    \begin{textblock}{11}(1,0.7)

      \begin{flushleft}

        \mdseries

        \footnotesize

        \color{jFrametitleFGColor}

        This content was created as part of a project co-financed by the
        European Union within the framework of the European Social Fund
        POWR.03.05.00-00-Z309/17-00.

      \end{flushleft}

    \end{textblock}





    \begin{textblock}{10}(0,2.2)

      \tikz \fill[color=jBackgroundStyleLight] (0,0) rectangle (12.8,-1.5);

    \end{textblock}


    \begin{textblock}{3.2}(0.7,2.45)

      \includegraphics[scale=0.3]{\FundingLogoColorPictureEN}

    \end{textblock}


    \begin{textblock}{2.5}(4.15,2.5)

      \includegraphics[scale=0.2]{\JULogoColorPictureEN}

    \end{textblock}


    \begin{textblock}{2.5}(6.35,2.4)

      \includegraphics[scale=0.1]{\ZintegrUJLogoColorPictureEN}

    \end{textblock}


    \begin{textblock}{4.2}(8.4,2.6)

      \includegraphics[scale=0.3]{\EUSocialFundLogoColorPictureEN}

    \end{textblock}





    \begin{textblock}{11}(1,4)

      \begin{flushleft}

        \mdseries

        \footnotesize

        \RaggedRight

        \color{jFrametitleFGColor}

        The content of this lecture is made available under a Creative
        Commons licence (\textsc{cc}), giving the author the credits
        (\textsc{by}) and putting an obligation to share on the same terms
        (\textsc{sa}). Figures and diagrams included in the lecture are
        authored by Paweł Węgrzyn et~al., and are available under the same
        license unless indicated otherwise.\\ The presentation uses the
        Beamer Jagiellonian theme based on Matthias Vogelgesang’s
        Metropolis theme, available under license \LaTeX{} Project
        Public License~1.3c at: \colorhref{https://github.com/matze/mtheme}
        {https://github.com/matze/mtheme}.

        Typographic design: Iwona Grabska-Gradzińska;
        \LaTeX{} Typesetting: Kamil Ziemian \\
        Proofreading: Wojciech Palacz,
        Monika Stawicka \\
        3D Models: Dariusz Frymus, Kamil Nowakowski \\
        Figures and charts: Kamil Ziemian, Paweł Węgrzyn, Wojciech
        Palacz \\
        Film editing: IMAVI -- Joanna Kozakiewicz, Krzysztof Magda, Nikodem
        Frodyma

      \end{flushleft}

    \end{textblock}

  \end{frame}
}



\newcommand{\GeometryThreeDSpecialEndingSlideVideoVerThreeEN}{%
  \begin{frame}[standout]

    \begin{textblock}{11}(1,0.7)

      \begin{flushleft}

        \mdseries

        \footnotesize

        \color{jFrametitleFGColor}

        This content was created as part of a project co-financed by the
        European Union within the framework of the European Social Fund
        POWR.03.05.00-00-Z309/17-00.

      \end{flushleft}

    \end{textblock}





    \begin{textblock}{10}(0,2.2)

      \tikz \fill[color=jBackgroundStyleLight] (0,0) rectangle (12.8,-1.5);

    \end{textblock}


    \begin{textblock}{3.2}(0.7,2.45)

      \includegraphics[scale=0.3]{\FundingLogoColorPictureEN}

    \end{textblock}


    \begin{textblock}{2.5}(4.15,2.5)

      \includegraphics[scale=0.2]{\JULogoColorPictureEN}

    \end{textblock}


    \begin{textblock}{2.5}(6.35,2.4)

      \includegraphics[scale=0.1]{\ZintegrUJLogoColorPictureEN}

    \end{textblock}


    \begin{textblock}{4.2}(8.4,2.6)

      \includegraphics[scale=0.3]{\EUSocialFundLogoColorPictureEN}

    \end{textblock}





    \begin{textblock}{11}(1,4)

      \begin{flushleft}

        \mdseries

        \footnotesize

        \RaggedRight

        \color{jFrametitleFGColor}

        The content of this lecture is made available under a Creative
        Commons licence (\textsc{cc}), giving the author the credits
        (\textsc{by}) and putting an obligation to share on the same terms
        (\textsc{sa}). Figures and diagrams included in the lecture are
        authored by Paweł Węgrzyn et~al., and are available under the same
        license unless indicated otherwise.\\ The presentation uses the
        Beamer Jagiellonian theme based on Matthias Vogelgesang’s
        Metropolis theme, available under license \LaTeX{} Project
        Public License~1.3c at: \colorhref{https://github.com/matze/mtheme}
        {https://github.com/matze/mtheme}.

        Typographic design: Iwona Grabska-Gradzińska;
        \LaTeX{} Typesetting: Kamil Ziemian \\
        Proofreading: Wojciech Palacz,
        Monika Stawicka \\
        3D Models: Dariusz Frymus, Kamil Nowakowski \\
        Figures and charts: Kamil Ziemian, Paweł Węgrzyn, Wojciech
        Palacz \\
        Film editing: Agencja Filmowa Film \& Television Production~--
        Zbigniew Masklak \\
        Film editing: IMAVI -- Joanna Kozakiewicz, Krzysztof Magda, Nikodem
        Frodyma

      \end{flushleft}

    \end{textblock}

  \end{frame}
}



\newcommand{\GeometryThreeDTwoSpecialEndingSlidesVideoVerOneEN}[1]{%
  \begin{frame}[standout]

    \begin{textblock}{11}(1,0.7)

      \begin{flushleft}

        \mdseries

        \footnotesize

        \color{jFrametitleFGColor}

        This content was created as part of a project co-financed by the
        European Union within the framework of the European Social Fund
        POWR.03.05.00-00-Z309/17-00.

      \end{flushleft}

    \end{textblock}





    \begin{textblock}{10}(0,2.2)

      \tikz \fill[color=jBackgroundStyleLight] (0,0) rectangle (12.8,-1.5);

    \end{textblock}


    \begin{textblock}{3.2}(0.7,2.45)

      \includegraphics[scale=0.3]{\FundingLogoColorPictureEN}

    \end{textblock}


    \begin{textblock}{2.5}(4.15,2.5)

      \includegraphics[scale=0.2]{\JULogoColorPictureEN}

    \end{textblock}


    \begin{textblock}{2.5}(6.35,2.4)

      \includegraphics[scale=0.1]{\ZintegrUJLogoColorPictureEN}

    \end{textblock}


    \begin{textblock}{4.2}(8.4,2.6)

      \includegraphics[scale=0.3]{\EUSocialFundLogoColorPictureEN}

    \end{textblock}





    \begin{textblock}{11}(1,4)

      \begin{flushleft}

        \mdseries

        \footnotesize

        \RaggedRight

        \color{jFrametitleFGColor}

        The content of this lecture is made available under a Creative
        Commons licence (\textsc{cc}), giving the author the credits
        (\textsc{by}) and putting an obligation to share on the same terms
        (\textsc{sa}). Figures and diagrams included in the lecture are
        authored by Paweł Węgrzyn et~al., and are available under the same
        license unless indicated otherwise.\\ The presentation uses the
        Beamer Jagiellonian theme based on Matthias Vogelgesang’s
        Metropolis theme, available under license \LaTeX{} Project
        Public License~1.3c at: \colorhref{https://github.com/matze/mtheme}
        {https://github.com/matze/mtheme}.

        Typographic design: Iwona Grabska-Gradzińska;
        \LaTeX{} Typesetting: Kamil Ziemian \\
        Proofreading: Wojciech Palacz,
        Monika Stawicka \\
        3D Models: Dariusz Frymus, Kamil Nowakowski \\
        Figures and charts: Kamil Ziemian, Paweł Węgrzyn,
        Wojciech Palacz \\
        Film editing: Agencja Filmowa Film \& Television Production~--
        Zbigniew Masklak

      \end{flushleft}

    \end{textblock}

  \end{frame}





  \begin{frame}[standout]


    \begingroup

    \color{jFrametitleFGColor}

    #1

    \endgroup

  \end{frame}
}



\newcommand{\GeometryThreeDTwoSpecialEndingSlidesVideoVerTwoEN}[1]{%
  \begin{frame}[standout]

    \begin{textblock}{11}(1,0.7)

      \begin{flushleft}

        \mdseries

        \footnotesize

        \color{jFrametitleFGColor}

        This content was created as part of a project co-financed by the
        European Union within the framework of the European Social Fund
        POWR.03.05.00-00-Z309/17-00.

      \end{flushleft}

    \end{textblock}





    \begin{textblock}{10}(0,2.2)

      \tikz \fill[color=jBackgroundStyleLight] (0,0) rectangle (12.8,-1.5);

    \end{textblock}


    \begin{textblock}{3.2}(0.7,2.45)

      \includegraphics[scale=0.3]{\FundingLogoColorPictureEN}

    \end{textblock}


    \begin{textblock}{2.5}(4.15,2.5)

      \includegraphics[scale=0.2]{\JULogoColorPictureEN}

    \end{textblock}


    \begin{textblock}{2.5}(6.35,2.4)

      \includegraphics[scale=0.1]{\ZintegrUJLogoColorPictureEN}

    \end{textblock}


    \begin{textblock}{4.2}(8.4,2.6)

      \includegraphics[scale=0.3]{\EUSocialFundLogoColorPictureEN}

    \end{textblock}





    \begin{textblock}{11}(1,4)

      \begin{flushleft}

        \mdseries

        \footnotesize

        \RaggedRight

        \color{jFrametitleFGColor}

        The content of this lecture is made available under a Creative
        Commons licence (\textsc{cc}), giving the author the credits
        (\textsc{by}) and putting an obligation to share on the same terms
        (\textsc{sa}). Figures and diagrams included in the lecture are
        authored by Paweł Węgrzyn et~al., and are available under the same
        license unless indicated otherwise.\\ The presentation uses the
        Beamer Jagiellonian theme based on Matthias Vogelgesang’s
        Metropolis theme, available under license \LaTeX{} Project
        Public License~1.3c at: \colorhref{https://github.com/matze/mtheme}
        {https://github.com/matze/mtheme}.

        Typographic design: Iwona Grabska-Gradzińska;
        \LaTeX{} Typesetting: Kamil Ziemian \\
        Proofreading: Wojciech Palacz,
        Monika Stawicka \\
        3D Models: Dariusz Frymus, Kamil Nowakowski \\
        Figures and charts: Kamil Ziemian, Paweł Węgrzyn,
        Wojciech Palacz \\
        Film editing: IMAVI -- Joanna Kozakiewicz, Krzysztof Magda, Nikodem
        Frodyma

      \end{flushleft}

    \end{textblock}

  \end{frame}





  \begin{frame}[standout]


    \begingroup

    \color{jFrametitleFGColor}

    #1

    \endgroup

  \end{frame}
}



\newcommand{\GeometryThreeDTwoSpecialEndingSlidesVideoVerThreeEN}[1]{%
  \begin{frame}[standout]

    \begin{textblock}{11}(1,0.7)

      \begin{flushleft}

        \mdseries

        \footnotesize

        \color{jFrametitleFGColor}

        This content was created as part of a project co-financed by the
        European Union within the framework of the European Social Fund
        POWR.03.05.00-00-Z309/17-00.

      \end{flushleft}

    \end{textblock}





    \begin{textblock}{10}(0,2.2)

      \tikz \fill[color=jBackgroundStyleLight] (0,0) rectangle (12.8,-1.5);

    \end{textblock}


    \begin{textblock}{3.2}(0.7,2.45)

      \includegraphics[scale=0.3]{\FundingLogoColorPictureEN}

    \end{textblock}


    \begin{textblock}{2.5}(4.15,2.5)

      \includegraphics[scale=0.2]{\JULogoColorPictureEN}

    \end{textblock}


    \begin{textblock}{2.5}(6.35,2.4)

      \includegraphics[scale=0.1]{\ZintegrUJLogoColorPictureEN}

    \end{textblock}


    \begin{textblock}{4.2}(8.4,2.6)

      \includegraphics[scale=0.3]{\EUSocialFundLogoColorPictureEN}

    \end{textblock}





    \begin{textblock}{11}(1,4)

      \begin{flushleft}

        \mdseries

        \footnotesize

        \RaggedRight

        \color{jFrametitleFGColor}

        The content of this lecture is made available under a Creative
        Commons licence (\textsc{cc}), giving the author the credits
        (\textsc{by}) and putting an obligation to share on the same terms
        (\textsc{sa}). Figures and diagrams included in the lecture are
        authored by Paweł Węgrzyn et~al., and are available under the same
        license unless indicated otherwise. \\ The presentation uses the
        Beamer Jagiellonian theme based on Matthias Vogelgesang’s
        Metropolis theme, available under license \LaTeX{} Project
        Public License~1.3c at: \colorhref{https://github.com/matze/mtheme}
        {https://github.com/matze/mtheme}.

        Typographic design: Iwona Grabska-Gradzińska;
        \LaTeX{} Typesetting: Kamil Ziemian \\
        Proofreading: Leszek Hadasz, Wojciech Palacz,
        Monika Stawicka \\
        3D Models: Dariusz Frymus, Kamil Nowakowski \\
        Figures and charts: Kamil Ziemian, Paweł Węgrzyn,
        Wojciech Palacz \\
        Film editing: Agencja Filmowa Film \& Television Production~--
        Zbigniew Masklak \\
        Film editing: IMAVI -- Joanna Kozakiewicz, Krzysztof Magda, Nikodem
        Frodyma


      \end{flushleft}

    \end{textblock}

  \end{frame}





  \begin{frame}[standout]


    \begingroup

    \color{jFrametitleFGColor}

    #1

    \endgroup

  \end{frame}
}











% ------------------------------------------------------------------------------------
% Importing packages, libraries and setting their configuration
% ------------------------------------------------------------------------------------





% ------------------------------------------------------
% BibLaTeX
% ------------------------------------------------------
% Package biblatex, with biber as its backend, allow us to handle
% bibliography entries that use Unicode symbols outside ASCII.
\usepackage[
language=polish,
backend=biber,
style=alphabetic,
url=false,
eprint=true,
]{biblatex}

\addbibresource{Cybernetyczne-przygody-Stanisława-Lema-Bibliography.bib}





% ------------------------------------------------------
% Wonderful package PGF/TikZ
% ------------------------------------------------------

% Node and pics for drawing charts
% \usepackage{./Local-packages/PGF-TikZ-Chart-nodes-and-pics}

% Styles for arrows
% \usepackage{./Local-packages/PGF-TikZ-Arrows-styles}





% ------------------------------------------------------
% Local packages
% ------------------------------------------------------
% Special configuration for this particular presentation
\usepackage{./Local-packages/local-settings}

% Package containing various command useful for working with a text
\usepackage{./Local-packages/general-commands}

% Package containing commands and other code useful for working with
% mathematical text
% \usepackage{./Local-packages/math-commands}










% ------------------------------------------------------------------------------------------------------------------
\title{Cybernetyczne przygody Stanisława Lema}

\author{Kamil Ziemian \\
  \email}


% \institute{Uniwersytet Jagielloński w~Krakowie}

\date[14 V~2025~r.]{14 V~2025~r.}
% ------------------------------------------------------------------------------------------------------------------









% ####################################################################
\begin{document}
% ####################################################################





% ######################################
% Number of chars: 48k+,
% Text is adjusted to the left and words are broken at the end of the line.
\RaggedRight
% ######################################





% ######################################
\maketitle
% ######################################





% ######################################
\begin{frame}
  \frametitle{Plan prezentacji}


  \tableofcontents

\end{frame}
% ######################################










% ######################################
\section{O~moich badaniach nad Lemem i~cybernetyką}
% ######################################



% ##################
\begin{frame}
  \frametitle{Jak to~się zaczęło}


  Nie jestem ani zawodowym literaturoznawcą, ani cybernetykiem, wypada więc
  wyjaśnić czemu akurat ja dziś staję przed Państwem. Wszystko
  zaczęło~się od serii rozmów z~dwoma miłośnikami Stanisława Lema,
  \colorhref{https://historia.uj.edu.pl/instytut/pracownicy/marcin-jarzabek}
  {dr.~Marcinem Jarząbkiem}
  i~\colorhref{https://incet.uj.edu.pl/maciej-prochnicki}{dr.~Maciejem
    Próchnickim}, którzy około marca $2023$ roku, zwrócili mi uwagę no to,
  iż przeszedł on fazę fascynacji cybernetyką. Wiedząc już wtedy,
  że~cybernetyka była dziedziną silnie techniczną i~zmatematyzowaną, czym
  postawiłem sobie dwa powiązane pytania. Czym była cybernetyka którą
  studiował Lem i~jak on ją rozumiał?

  By na to odpowiedzieć podjąłem moje amatorskie studia, które
  rodzą wciąż nowe i~nowe pytania, zarówno gdy chodzi o~twórczość Lema,
  jak i~cybernetykę. Najważniejsze z~nich na które szuka dziś odpowiedzi, to
  znaczenie motywów cybernetyczne w~dziełach literackich Lema oraz związek
  jego światopoglądu z~tym
  \colorhref{https://en.wikipedia.org/wiki/Norbert\_Wiener}{Norberta
    Wienera}.

\end{frame}
% ##################





% ##################
\begin{frame}
  \frametitle{Lem i~cybernetyka}


  Wygląda na to, że~pomimo opublikowania przez Pawła Majewskiego monografii
  \textit{Między zwierzęciem a~maszyną. Utopia technologiczna Stanisława
    Lema} w~$2007$~roku
  (\parencite{Majewski-Miedzy-zwierzeciem-a-maszyna-ETC-Pub-2007}), problem
  cybernetycznych przygód Lema pozostaje niezwykle słabo zrozumiany
  zagadnieniem. Wobec tego również ludzie tacy jak ja, mogą~się podjąć jego
  studiowania, z~nadzieją na odkrycie czegoś ciekawego.

  Dużą inspirację do~podjęcia tego tematu stanowiły dla mnie badania
  mojego dobrego znajomego
  \colorhref{https://skos.agh.edu.pl/osoba/szymon-kukulak-10650.html}
  {dr.~Szymona Kukulaka} z~Wydziału Humanistycznego \textsc{agh}, dotyczące
  wpływu odkryć technicznych i~eksploracji Układu Słonecznego na prozę
  Lema, o~których pierwszy raz usłyszałem od~niego na~początku $2015$ roku.
  Również od niego pochodzi wiele informacji zawartych w~tym wystąpieniu.
  Z~wielu jego prac, chciałbym teraz wskazać na jego artykuł \textit{Two
    Faces of Mars: The Red Planet in Stanisław Lem's "The Man from Mars" and
    “Ananke” in Light of Contemporary Scientific Pursuits and Martian
    Fiction} \parencite{Kukulak-Two-Faces-of-Mars-Pub-2023}.

\end{frame}
% ##################





% ##################
\begin{frame}
  \frametitle{Plan badań}


  Jak wskazuje tytuł, dr~Kukulak analizuje w~tym artykule jak literacki
  obraz Marsa ewoluował między \textit{Człowiekiem z~Marsa} a~opowiadaniem
  \textit{Ananke}, w~związku z~nowymi informacjami o~tej planecie
  \parencite{Kukulak-Two-Faces-of-Mars-Pub-2023}. Dochodzi on tam do bardzo
  ciekawego wniosku, że~w~\textit{Ananke} widzimy obraz Marsa, który
  w~zasadzi od razu stał~się przestarzały, w~skutek misji Marinera~$9$
  z~$1971$ roku i~Mariner~$10$ z~$1973$ roku.

  Świadom moich braków wiedzy w~zakresie cybernetyki, nawiązałem kontakt
  z~\colorhref{https://skos.agh.edu.pl/osoba/andrzej-bielecki-7555.html}
  {prof.~Andrzejem Bieleckim} z~Wydziału Elektrotechniki, Automatyki,
  Informatyki i~Inżynierii Biomedycznej \textsc{agh}, cybernetykiem
  i~wielkim miłośnikiem oraz~znawcą twórczość Lema. Jemu również zawdzięczam
  wiele cennych informacji.

  Obecnie jestem w~około $1 / 3$ moich planowanych badań, siłą rzeczy
  ma więcej pytań niż odpowiedzi i~o~wielu ważny rzeczach nie mogę nic
  wartościowego powiedzieć. W~skutek tych braków wiedzy, w~moja analiza
  jest wciąż mocno niespójna.

\end{frame}
% ##################





% ##################
\begin{frame}
  \frametitle{„Niezbadane planety”}


  Jedna z~rzeczy, które będą musiał pominąć, to relacje Lema z~polskim
  środowiskiem cybernetycznym. Problem ten wydaje~się być niewiarygodnie
  zaniedbany, zapewne w~skutek milczącego założenia, że~Lem uczył~się
  cybernetyki od autorów anglo- oraz~rosyjskojęzycznych, przez co
  kwestia tego czy utrzymywał jakikolwiek kontakt intelektualny
  z~polskimi cybernetykami po~$1950$ roku, była ignorowany.

  Wiemy, że~w~latach $1948\text{-}1950$ Lem brał udział w~Konwersatorium
  Naukoznawczym Asystentów Uniwersytetu Jagiellońskiego, kierowanym
  przez
  \colorhref{https://pl.wikipedia.org/wiki/Mieczys\%C5\%82aw\_Choynowski}
  {dr.~Mieczysława Choynowskiego}, które w~tymże roku $1950$ zostało
  rozwiązane i~na pewno wywarło duży wpływ na jego myślenie,
  również o~cybernetyce. Jednak badania związków Lema z~polskim środowiskiem
  cybernetycznym w~późniejszym okresie jego życia, wydają~się być prawie
  nieistniejące.

\end{frame}
% ##################









% ######################################
\section{Cybernetyka przed i~po~$1945$ roku}
% ######################################


% ##################
\begin{frame}
  \frametitle{Cybernetyka po~1945 roku}


  W~powieści Lema \textit{Eden} z~$1958$~roku
  \parencite{Lem-Eden-Pub-2019}, jeden z~sześciu członków
  załogi, Cybernetyk, jest odpowiedzialny za pracę z~automatami, takimi
  jak komputer, odkurzacz autonomiczny, roboty naprawcze i~bojowe
  (Obrońca). Choć powieść zaczyna~się od zdania „W~obliczeniach był błąd.”,
  to katastrofa ich rakiety była spowodowana wprowadzeniem błędnych danych
  do automatu obliczającego trasę ruchu, który, jak czytamy, sam nie
  mógł~się pomylić. Niestety wątek ten zostaje przez Lema szybko
  zmarginalizowany, gdyż automaty zostają zniszczone w~czasie awaryjnego
  lądowania. „Owdowiał”, komentuje Doktor jego sytuację i~zły humor.

  Słowo „automat” jest tu najprawdopodobniej użyte w~znaczeniu
  „cybernetycznym”: „urządzenie (maszyna, aparat, przyrząd)
  umożliwiający realizację procesu produkcyjnego bez osobistego udziału
  człowieka, a~tylko pod jego nadzorem”, str.~$19$
  \parencite{Szylejko-Szylejko-Cybernetyka-bez-matematyki-Pub-1977}.

\end{frame}
% ##################





% ##################
\begin{frame}
  \frametitle{Cybernetyka po~1945 roku}


  Ten przykład ilustruje pewną trudność mówienia o~powojennej cybernetyce,
  która z~naszego punktu widzenia obejmuje problemy należące
  do~informatyki, teorii informacji, badań na sztuczną inteligencją, w~tym
  uczenia maszynowego, robotyki, neurofizjologi, socjologii,
  psychologii,~etc.

  By lepiej zrozumieć ówczesny sens terminu „cybernetyka”, przytoczymy
  słowa Lema z~jego wydanych w~$1957$ \textit{Dialogów}. \\
  \textit{Doniosłość prac cybernetyków na tym się właśnie zasadza,
    że~wykryli odpowiedź na to pytanie [Czym jest informacja?]. Informacja
    jest dzieckiem termodynamiki, postawionej, mówiąc obrazowo, na głowie,
    gdyż stanowi odwrotność entropii.}
  Str.~$56$, \parencite{Lem-Dialogi-Vol-I-Pub-1996}.

  Czy tym co miało czynić z~cybernetyki superdyscyplinarną dziedzinę
  (określenie użyte retrospektywnie przez Lema, por. str.~$6\text{--}7$,
  \parencite{Lem-Dialogi-Vol-I-Pub-1996}), miało być pojęcie informacji?
  Prześledzimy ten problem na podstawie, historii zostawionej nam przez
  Norberta Wienera.

\end{frame}
% ##################





% ##################
\begin{frame}
  \frametitle{Norbert Wiener, ojciec cybernetyki}


  \begin{figure}

    \centering


    \includegraphics[scale=0.23]
    {./Presentation-pictures/Norbert-Wiener-01.jpg}


    \caption{\colorhref{https://en.wikipedia.org/wiki/Norbert\_Wiener}
      {Norbert Wiener} ($1894\text{--}1964$), amerykański filozof,
      informatyk i~matematyk, ojciec cybernetyki.}


    \label{fig:Wiener-01}

  \end{figure}

\end{frame}
% ##################





% ##################
\begin{frame}
  \frametitle{Arturo Rosenblueth, zapomniany współtwórca
    cybernetyki}


  \begin{figure}

    \centering


    \includegraphics[scale=0.37]
    {./Presentation-pictures/Arturo-Rosenblueth-01.png}


    \caption{\colorhref{https://en.wikipedia.org/wiki/Arturo\_Rosenblueth}
      {Arturo Rosenblueth} ($1900\text{--}1970$), meksykański fizjolog,
      lekarz, wskazany przez Wienera jako kluczowa postać w~rozwoju
      cybernetyki.}


    \label{fig:Rosenblueth-01}

  \end{figure}

\end{frame}
% ##################





% ##################
\begin{frame}
  \frametitle{\textit{Cybernetics: Or~control and~Communication \ldots}}


  Norbert Wiener był niewątpliwie wybitnym uczonym, który dokonał
  znaczących odkryć w~wielu dziedzinach wiedzy, my jednak musimy
  skupić~się na jego roli jako ojca cybernetyki, którym został poprzez
  opublikowanie w~roku $1948$ książki
  \colorhref{https://en.wikipedia.org/wiki/Cybernetics\_Or\_Control\_and\_Communication\_in\_the\_Animal\_and\_the\_Machine}
  {\textit{Cybernetics: Or~control and~Communication in the~Animal and
      the~Machine}}. Jak później przyznał pełna była niedoróbek i~jawnych
  błędów.

  Książka ta w~trudny dla mnie do~zrozumienia sposób stała~się fenomenem
  kulturowym, który najlepiej chyba oddają słowa z~jej recenzji, która
  ukazał w~piśmie \textit{Business Week} w~$1949$ roku: \textit{Pod jednym
    względem książka Wienera jest jak Raport Kinseya: reakcja społeczeństwa
    na nią jest tak samo ważna, jak jej treść.}, tłum. wł., cyt.
  za~str.~$408$
  \parencite{Hamilton-The-Charismatic-Cultural-Life-of-ETC-Pub-2017}. Sam
  zaś Wiener uzyskał dzięki niej status celebryty, który to temat jest
  szerzej omawia Sheryl N.~Hamilton w~cytowanym artykule.

  Kulturowa recepcja cybernetyki, to temat bardzo interesujący, który
  niestety musimy w~większości pominąć.

\end{frame}
% ##################





% ##################
\begin{frame}
  \frametitle{Historia cybernetyki według Wienera}


  Muszę jednak wspomnieć, że~urodzony w~$1963$ roku w~Moskwie,
  sowiecko-amerykański historyk, inżynier i~matematyk
  \colorhref{https://en.wikipedia.org/wiki/Slava_Gerovitch}{Slava Gerovitch}
  napisał książkę
  \colorhref{https://mitpress.mit.edu/9780262572255/from-newspeak-to-cyberspeak/}
  {\textit{From Newspeak to~Cyberspeak. A~History of Soviet Cybernetics}},
  o~recepcji cybernetyki w~Związku Sowieckim. Do tej pory inne kraju
  Bloku Wschodniego nie doczekały~się chyba równie dobrego opracowania
  fenomenu cybernetyki, choć u~nas
  \colorhref{https://pl.wikipedia.org/wiki/Jan\_Tr\%C4\%85bka}{Jan Trąbka}
  opublikował \textit{Pół wieku cybernetyki w~Polsce}.

  W~$1950$ Wiener publikuje mniejszą, bardzie
  popularną w~treść książkę
  \colorhref{https://en.wikipedia.org/wiki/The\_Human\_Use\_of\_Human\_Beings}
  {\textit{The~Human Use~of Human Beings. Cybernetics and society}},
  w~Polsce wydana jako \textit{Cybernetyka i~społeczeństwo} w~$1960$~roku.
  W~kontekście prozy Lema jest to również bardzo ważna pozycja.

  Wiener miał tendencje do tego by wiele mówić o~sobie i~swoim
  życiu i~w~rozdziale wprowadzający do pierwszego wydania
  \textit{Cybernetics} zostawił nam swoją wersję historii cybernetyki.

\end{frame}
% ##################





% ##################
\begin{frame}
  \frametitle{Historia cybernetyki według Wienera}


  Do historyków należy ocenienie, jak bardzo relacja
  Wieniera jest wierna rzeczywistości, jest jednak niewątpliwie cennym
  i~ciekawym źródłem wiedzy o~formowaniu~się tej dziedziny i~o~poglądach
  samego Wienera, które mają w~tym kontekście wielką wagę. Ze względu na
  ramy czasowe, wybierzemy z~niej tylko kilka punktów.

  Wydaje~się, że~koło~$1937$ roku Wiener dołącza do serii comiesięcznych
  dyskusji prowadzonych przez Rosenbluetha, pracującego wówczas na Harvard
  Medical Schools. Od początku
  łączy ich silne przekonanie o~potrzebnie badań interdyscyplinarnych,
  acz~dopiero II Wojna Światowa stwarza warunki do bliższej współpracy.
  W~związku z~wojną Wiener pracuje nad działami przeciwlotniczymi, jednym
  z~jego ważnych współpracowników jest pionier konstrukcji komputerów
  \colorhref{https://en.wikipedia.org/wiki/Julian_Bigelow}{Julia
    H.~Bigelow}. Mierzą~się oni z~problemem, że~ze względu na wysokość lotu
  samolotu i~prędkość lotu pocisku, aby zestrzelić dany samolot należy
  celować nie w~jego obecne, lecz przyszłe położenie.

\end{frame}
% ##################





% ##################
\begin{frame}
  \frametitle{Cybernetyka i~sprzężenie zwrotne}


  By rozwiązać ten problem, chcą wyposażyć działa przeciwlotnicze
  w~dobrze już wtedy znany
  mechanizm sprzężenia zwrotnego (ang.~\textit{feedback}). Najpóźniej
  w~$1957$ roku będą już dysponowali koncepcją uczącego~się działa
  przeciwlotniczego. Przedmowa do II-ego wydania \textit{Cybernetics\ldots}
  z~$1961$ roku pokazuje, że~Wiener znał już wtedy, przynajmniej w~ogólnym
  zarysie, podstawowe pojęcia uczenia maszynowego.

  W~związku z~tym warto się zapytać czy~opowiadanie Lema \textit{Terminus}.
  nie jest historią o~przesłaniu dusz (?) ludzi do tytułowego robota, za
  pomocą procedury uczenia maszynowego? Bardzo jestem ciekawy, czy według
  Państwa takie spojrzenie ma w~ogóle sens?

  Wracając do historii Wienera i~jego grupy, w~pracy swojej wspierają~się
  książką L.A. MacColla
  \colorhref{https://archive.org/details/fundamentaltheor0000macc}
  {\textit{Fundamental theory of servomechanisms}}, gdzie opisane jest
  zjawisko \textit{polowania} (ang. \textit{hunting}), kiedy to
  sprzężenie zwrotne, zamiast prowadzić do coraz lepszego działania
  mechanizmu, wprawia go w~coraz silniejsze oscylacje, skutkujące jego
  zniszczenia.

\end{frame}
% ##################





% ##################
\begin{frame}
  \frametitle{Zwierzęta i~maszyny}


  Wiener podaje, że~on i~Bigelow zwrócili~się do Rosenbluetha z~pytanie, czy
  znana jest choroba, która prowadzi do zachowań takich jak polowanie
  w~maszynach? On zaś wskazał im na zjawisko drżenia zamiarowego
  (łac.
  \colorhref{https://en.wikipedia.org/wiki/Intention_tremor}{\textit{tremor
      intentionalis}}). Ta wymiana myśli zostało uznana przez Wienera za
  ważną, zapewne dlatego, że~ugruntowała pogląd, że~cybernetyka dotyczy
  tak samo maszyn jak i~zwierząt.

  Wybiegając naprzód należy dodać, że~w~$1946$ roku Wiener i~Rosenblueth
  podejmą wspólne badania nad mięśniami kota, traktowanymi jako swego
  rodzaju maszyna (serwomechanizm). Badali oni częstotliwość oscylacji
  jego mięśni w~zależności od siły pobudzającego ich impulsu elektrycznego.

  Wiadomo, że~Lema, i~Wienera, bardzo zajmował temat, czy, a~jeśli tak to
  kiedy, maszyna staje~się osobą i~co to przemiana oznacza dla ludzi?
  Obu ich nurtował też problem, że~jeśli człowiek jest tylko maszyną, to jak
  można obronić drogie im wartości humanistyczne?

\end{frame}
% ##################





% ##################
\begin{frame}
  \frametitle{Człowiek/maszyna w~prozie Lema}


  Poza wspominanym już \textit{Terminusem} Lem pisze o~tym choćby
  w~opowiadaniach \textit{Wypadek}, \textit{Polowanie}, \textit{Rozprawa},
  czy powieści \textit{Niezwyciężony}. Wydaje mi~się, że~w~ten sposób można
  interpretować też przynajmniej niektóre \textit{Bajki robotów}. Jeśli
  przy tym weźmiemy pod uwagę propozycję Agnieszki Gajewskiej, żeby czytać
  twórczość Lema również jako próbę przepracowania koszmaru II~Wojny
  Światowej, stosowanie tego podejścia do~takich opowiadań jak
  \textit{Dwa potwory}, wydaje~się wartościowe poznawczo
  \parencite{Gajewska-Zaglada-i-gwiazdy-Pub-2017}. Temat ten wymaga jednak
  głębszej refleksji i~dziś nie chcę niczego przesądzać.

  Prof.~Bielecki zwrócił mi przy tym uwagę, że~istnieje opowiadanie Lema,
  którego nie chciał on nigdy wznawiać i~do~którego pani Gajewskiej
  nie udało~się dotrzeć w~czasie pisania \textit{Zagłady i~gwiazd}, gdyż
  nie jest ono tam wspomniane. Opowiada o~robocie ukrywający~się przez
  chcącym go zniszczyć ludźmi i~zapewne to odsłonięcie swoich przeżyć
  z~czasów II~Wojny Światowej, wyjaśnia późniejsze zachowanie Lema.

\end{frame}
% ##################





% ##################
\begin{frame}
  \frametitle{Lem studiował wszak medycynę}


  Z~historii cybernetyki wiadomo, że~wielu członków tego ruchu w~latach
  $40$-tych i~$50$-tych było fizjologami, lekarzami lub psychologami.
  Przytoczona wyżej historia pozwala lepiej zrozumieć dlaczego ten nurt
  badań przyciągnął tak wielu członków tej profesji. Wiemy też, że~Lem
  studiował medycynę, jak wskazują biografowie, pod wpływem ojca i~bez
  wielkiego entuzjazmu, co jednak prawie na pewno miało wpływ na
  jego początkową fascynację tym przedmiotem.

  Jednym z~wczesnych cybernetyków, z~tej kategorii o~którym potrzebuję
  wspomnieć jest
  \colorhref{https://en.wikipedia.org/wiki/W._Ross_Ashby}{William
    Ross Ashby} ($1903\text{--}1972$), który od~$1930$ roku prowadził
  jako psychiatra badania kliniczne. Jego osoba jest ważna, bo mam
  podejrzenie,
  że~spotkał go Ijon Tichy, gdy Ashby z~niewiadomych powodów ukrywał~się
  pod nazwiskiem prof.~Corcorana. Niemniej to na razie tylko przypuszczenia.

\end{frame}
% ##################





% ##################
\begin{frame}
  \frametitle{Homeostaty Ashby’ego}


  \begin{figure}

    \centering


    \includegraphics[scale=1.1]
    {./Presentation-pictures/W\_Ross\_Ashbys\_1948\_Homeostat.jpeg}


    \caption{\colorhref{https://en.wikipedia.org/wiki/Homeostat}
      {Homeostat} skonstruowane przez W.~Rossa Ashby’ego w~$1948$~roku.}


    \label{fig:Homeostat-01}

  \end{figure}

\end{frame}
% ##################





% ##################
\begin{frame}
  \frametitle{Cybernetyka, antropologia i~socjologia}


  Włączenie takich przedmiotów jak antropologia, psychologia i~socjologia
  w~obręb zainteresowań powstającej cybernetyki odbyło~się najpóźniej
  w~$1946$ roku, gdy w~ramach inicjatywy
  \colorhref{https://en.wikipedia.org/wiki/Josiah\_Macy\_Jr.\_Foundation}
  {Josiah Macy Foundation} zorganizowano w~Nowym Yorku cykl spotkań
  na~temat roli sprzężenia zwrotnego. Z~ramienia fundacji organizował je
  \colorhref{https://en.wikipedia.org/wiki/Frank_Fremont-Smith}
  {Frank Fremont-Smith}, ze~strony naukowej współtwórca teorii sieci
  neuronowych
  \colorhref{https://en.wikipedia.org/wiki/Warren\_Sturgis\_McCulloch}
  {Warren McCulloch}. Z~ich inicjatywy zaproszono tam psychologów,
  socjologów i~antropologów, takich jak małżeństwo antropologów
  \colorhref{https://en.wikipedia.org/wiki/Gregory\_Bateson}
  {Gregory’ego Batesona}
  i~\colorhref{https://en.wikipedia.org/wiki/Margaret\_Mead}
  {Margaret Mead}, czy filozofa
  \colorhref{https://en.wikipedia.org/wiki/F.\_S.\_C.\_Northrop}
  {F.S.C. Northropa}.

  W~obecnym momencie nie rozumiem dobrze stosunku Wiener do socjologii,
  czy antropologi. Z~jednej strony całkowicie popierał decyzję
  Fremonta-Smitha i~McCullocha, pisząc m.in.
  \textit{Jeśli chodzi o~socjologię i~antropologię, znacznie informacji
    i~komunikacji [Czyli cybernetyki. KZ] przejawia~się w~tym,
    że~organizacja wykracza poza jednostkę i~wkracza w~społeczność
    (ang.~„community”).} Tłum. wł., str.~$18$,
  \parencite{Wiener-Cybernetics-Second-edition-Pub-2016}.

\end{frame}
% ##################





% ##################
\begin{frame}
  \frametitle{Cybernetyka, antropologia i~socjologia}


  Z~drugiej strony, gdy Bateson i~Mead zachęcali go by ze względu na
  problemy nękające społeczeństwo w~ówczesnym „czasie zamętu umysłowego”
  (ang. \textit{age~of confusion}), podjął cybernetyczne badania nad
  antropologią, socjologią i~ekonomią, to mimo podzielania ich poglądu
  na sytuację społeczną, zostawił to zadanie innym. Tłumaczył, że~nie
  posiadała tak dużej jako oni ich wiary w~efekty cybernetycznej „terapii
  schorowanego społeczeństwa” str.~$23$,
  \parencite{Wiener-Cybernetics-Second-edition-Pub-2016}.
  Jednocześnie, uważał Bateson i~Mead oraz ich współpracownicy powinni
  podjąć~się badania tych problemów.

  Swoją postawę motywuje on w~dość osobliwy sposób. Mianowicie, metody
  statystyczne znane w~$1948$ roku, wymagały pracy z~dużą próbką danych,
  zebranych w~bardzo stałych warunkach. W~jego ocenie, społeczeństwo
  zmieniało~się wówczas tak szybko, że~takie dane nie~są dostępne.
  Zamiast tego proponuje dwa programy badań.

\end{frame}
% ##################





% ##################
\begin{frame}
  \frametitle{Wiener, moralizator, wieszcz i~poeta}

  Pierwszy to tworzenie mechanicznych protez kończyn ludzkich, które
  w~jakiś sposób „przywracają” czucie w~nich.

  Drugi, to projekt uczynienia fabryki „sztucznym żywym” organizmem.
  \textit{[Od~dawna wiedziałem,~że] nowoczesne, super szybkie [Na
    standardy lat $40$-tych XX~wieku. KZ] maszyny obliczeniowe,~są
    na poziomie koncepcyjnym, idealnym systemem nerwowym dla urządzeń
    odpowiedzialnych za automatyczną kontrolę.} Tłum. wł., str.~$25$,
  \parencite{Wiener-Cybernetics-Second-edition-Pub-2016}.

  Ta idea sprawia, że~Wiener wchodzi w~rolę moralizatora (moja osobista
  ocena jego postawy), wieszcza i~poety, co zilustrujemy kilkoma cytatami.
  \textit{Powiedziałem, że~te nowe możliwości mają nieograniczony potencjał
    dobra i~zła. Z~jednej strony czynią metaforyczną dominację maszyn,
    którą wyobrażał sobie
    \colorhref{https://en.wikipedia.org/wiki/Samuel\_Butler\_(novelist)}
    {Samuel Butler}, najpilniejszym i~niemetaforycznym problemem. Daje
    rasie ludzkiej nowy i~najbardziej wydajny zbiór mechanicznych
    niewolników, by wykonywać ich pracę.} Tłum. wł., str.~$25$,
  \parencite{Wiener-Cybernetics-Second-edition-Pub-2016}.

\end{frame}
% ##################





% ##################
\begin{frame}
  \frametitle{Wiener, moralizator, wieszcz i~poeta}


  \textit{Jednakże, każda praca, która akceptuje warunki konkurencji
    z~pracą niewolniczą, akceptuje warunki pracy niewolniczej i~staje~się
    w~istocie pracą niewolniczą. [\ldots] to właśnie warunku wolnego rynku,
    „piątej wolności”, stały~się znakiem rozpoznawczym tej części
    amerykańskiej opinii, którą reprezentuje National Association~of
    Manufactureres i~Saturday Evening Post.} Tłum. wł., str.~$26$,
  \parencite{Wiener-Cybernetics-Second-edition-Pub-2016}.

  \textit{[Uczynię sytuację bardziej zrozumiałą] jeśli powiem, że~pierwsza
    rewolucja przemysłowa, rewolucja „czarny, diabelskich młynów”
    [ang. „dark satanic mills”, cytat z~
    \colorhref{https://en.wikipedia.org/wiki/And\_did\_those\_feet\_in\_ancient\_time}
    {„Jerozolimy”} Williama Blake’a. KZ]
    odebrała wartość ludzkiej ręce, poprzez kazanie jej konkurować
    z~maszyną. [\ldots] Nowoczesna rewolucja przemysłowa jest podobnie skazana
    na odebranie wartości ludzkiemu mózgowi, przynajmniej, gdy chodzi
    o~prostsze i~bardziej powtarzalne zajęcia.}
  Tłum. wł., str.~$26$,
  \parencite{Wiener-Cybernetics-Second-edition-Pub-2016}.

\end{frame}
% ##################





% ##################
\begin{frame}
  \frametitle{Wiener, moralizator, wieszcz i~poeta}


  \textit{Odpowiedzią, oczywiście, jest społeczeństwo oparte na~ludzkich
    wartościach, nie na sprzedawaniu czy kupowaniu. [\ldots] Uznałem więc
    za~mój obowiązek, przekazać te informacje i~moje rozumienie obecnej
    sytuacji do tych, którzy mają żywotny interes w~warunkach
    i~przyszłości pracy, czyli związków zawodowych. [\ldots] zarówno w~USA
    jak i~w~Anglii, związki zawodowe i~ruch robotniczy są w~rękach bardzo
    ograniczonych intelektualnie ludzi [\ldots] zupełnie nieprzygotowanych by
    podjąć większe polityczne, techniczne, socjologiczne i~ekonomiczne
    problemy, które dotyczą samego istnienia pracy.} \\
  Tłum. wł., str.~$26$,
  \parencite{Wiener-Cybernetics-Second-edition-Pub-2016}.

  \textit{Ci z~nas, którzy wnieśli wkład do nowej nauki jaką jest
    cybernetyka, są sytuacji moralnej, która jest, by użyć eufemizmu,
    niezbyt komfortowa. [\ldots{} Oprócz tego] powinniśmy ograniczyć nasze
    osobiste działania do taki dziedzin, jak fizjologia, czy psychologia,
    najbardziej oddalonych od~wojny i~wyzysku. [\ldots] Piszę to w~$1947$ roku,
    i~muszę rzec, że~moja nadzieja jest bardzo mała.} \\
  Tłum. wł., str.~$27$,
  \parencite{Wiener-Cybernetics-Second-edition-Pub-2016}.

\end{frame}
% ##################





% ##################
\begin{frame}
  \frametitle{Wiener, moralizator, wieszcz i~poeta}


  \textit{W~bardzo rzeczywistym sensie jesteśmy rozbitkami na~planecie
    skazanej na zagładę. Ale nawet u~rozbitków ludzkie zalety i~ludzkie
    wartości nie muszą ginąć, i~należy wydobyć z~nich możliwie wszystko.
    Zatoniemy w~końcu, lecz niech to~się stanie w~sposób nie przynoszący
    ujmy naszej godności.} \\
  Str.~$44$, polskiego wydania \textit{The~Human Use~of Human Beings}
  \parencite{Wiener-Cybernetyka-a-spoleczenstwo-Pub-2016}.

  Z~jakiegoś powodu nie raz czytając Wienera mieliśmy uczucie, że~mam
  przed sobą dzieło Lema. Czy jednak można mówić tutaj o~głębszej
  zależności intelektualnej, to wykażą dopiero dalsze badania.

  Przechodząc do podsumowania tej części, wygląda na to, że~wiele
  osobliwości historii cybernetyki i~jej późniejszych problemów, było
  obecnych już w~zalążku w~projekcie Wienera, który również obdarzył ją
  pewnym szczególną treścią światopoglądową i~zadaniem moralnym. Również
  ten temat wymaga dalszych studiów.

\end{frame}
% ##################










% ######################################
\section{Kilka słów o~Lemie i~cybernetyce}
% ######################################


% ##################
\begin{frame}
  \frametitle{„Okres cybernetyczny”}


  Za dwa główne dzieła cybernetyczne Lema uważa~się wydane w~$1957$ roku
  \textit{Dialogi}, które są świadectwem wielkiej wiary w~moc tej nauki,
  oraz \textit{Summę techologiae} z~$1964$. Tym właśnie dwóm dzieło
  poświęcona jest prawie całkowicie przytaczana już monografia
  Majewskiego
  \parencite{Majewski-Miedzy-zwierzeciem-a-maszyna-ETC-Pub-2007}.
  Jednocześnie poszerzone wydanie \textit{Dialogów} z~roku $1972$ jest już
  świadectwem innego, bardziej zdystansowanego spojrzenia na tę tematykę.
  Błędem byłoby jednak uważać to drugie wydanie, tylko i~wyłącznie za wyraz
  rozczarowania Lema cybernetyką i~moment, gdy ostatecznie wziął z~nią
  rozbrat.

  W~przedmowie do drugiego wydania dialogów, napisanej w~grudniu
  $1971$ roku, Lem stwierdza z~jednej strony, że~cybernetyka nie spełniła
  tego czego najbardziej po niej oczekiwano, jak choćby zunifikowania
  nauk przyrodniczych i~humanistycznych. Z~drugiej strony, uważa on,
  że~ma ona na swoim koncie realne osiągnięcia.

\end{frame}
% ##################





% ##################
\begin{frame}
  \frametitle{Cybernetyka u~Lema}


  Przykładem obu tych rzeczy jest dla niego komputer, który „nie stał się
  równorzędnym osobowym partnerem człowieka”, ale za~to okazał~się w~wielu
  miejscach niezbędnym narzędziem. Por. str. $6\text{--}7$,
  w~\parencite{Lem-Dialogi-Vol-I-Pub-1996}.

  Okres $1957\text{--}1972$ obejmuje czas powstania większości z~tych
  dzieł literackich Lema, które uważane są za najlepsze. Szczęśliwie więc
  dwa wydania \textit{Dialogów} stanowią bardzo użyteczne punkty graniczne,
  dla badania recepcji cybernetyki u~Lema w~tym okresie.

  W~naszej opinii prozy Lema \alert{nie} można sprowadzić do wątków
  cybernetycznych, ale ich zauważenie, pozwala lepiej zrozumieć wiele
  jej elementów. Tutaj lektura \textit{Dialogów} jest kluczowa do ustalenia,
  co Lem wyniósł ze swoich studiów cybernetyki i~które elementy jego
  dzieł mają najpewniej cybernetyczny rodowód. Spróbujmy wielu pytań
  jakie~się tu nasuwają, spróbujmy wymienić choć kilka.

\end{frame}
% ##################





% ##################
\begin{frame}
  \frametitle{Cybernetyka u~Lema}


  Czy omawiany w~\textit{Dialogach} problem „wskrzeszenia” lub
  skopiowania człowieka, jest powiązany z~zwielokrotnionym Ijonem Tichym
  z~\textit{Podróży siódmej}? Czy spotkany przez niego uczony, który
  konstruował sztuczne życia, a~potem sam był przez nie używany do
  przekazywania informacji, był cybernetykiem? Co z~teorią prof. Dońdy,
  że~informację można przekształcić w~energię? Czy wyłączają zmysł kąpiel
  jaką przechodzi młody Pirx, inspirowana była bardziej eksperymentami
  medycznymi, czy cybernetyką? A~co z~różnymi halucynacjami i~złudzeniami
  jakich doświadczają bohaterowie opowiadań o~Pirxie? Czy prace
  o~samoreprodukujących~się maszynach wpłynęły na treść
  \textit{Niezwyciężonego}?

  Zanim zakończymy, chciałbym trzem problemom poświęcić trochę więcej uwagi.
  Po pierwsze, czy wydana w~$1965$ roku \textit{Cyberiada} nie jest,
  przynajmniej częściowo, literacką krytyką cybernetyki?

\end{frame}
% ##################





% ##################
\begin{frame}
  \frametitle{Klęska zbójcy Gębona}


  Przyjrzyjmy~się w~szczególności opowiadaniu \textit{Wyprawa szósta, czyli
    jak Trurl i~Klapaucjusz demona drugiego rodzaju stworzyli, aby zbójcę
    Gębona pokonać}, które nie tylko nawiązuje w~tytule do pojęć
  cybernetycznych, ale też mierzy~się z~tym, co często przytacza~się jako
  największy problem cybernetyki: pojęciem informacji. W~historii tej demon
  drugiego rodzaju nie jest w~stanie odróżnić ważnych informacji, od~tych
  bez większego znaczenia, w~skutek czego zbójca Gębon ginie pod stosem
  wiadomości, które nie~są mu do niczego potrzebne.

  Z~drugiej strony \textit{Wyprawę trzecią, czyli smoki
    prawdopodobieństwa}, może być odczytana jako pewna krytyka statystyki
  i~rachunku prawdopodobieństwa. Czy zakończenie tego opowiadania, nie
  stawia pod znakiem zapytaniem przedstawionej tam probabilistycznej
  interpretacji nieistnienia smoków?

\end{frame}
% ##################





% ##################
\begin{frame}
  \frametitle{Mózg i~staw}


  Kolejne problem związany jest z~hipotezą postawioną przez
  \colorhref{https://en.wikipedia.org/wiki/Stafford\_Beer}{Stafforda Beera}
  i~\colorhref{https://en.wikipedia.org/wiki/Gordon\_Pask}{Gordona Paska}
  około roku $1960$. Niestety, wciąż nie dotarłem do dokładnej daty.
  Wyszli oni od popularnej wtedy tezy, że~mózg jest obiektem posiadającym
  zdolność adaptacji do~otoczenia i~doszli do wniosku, że należy przyjąć
  również tezę odwrotną: każdy układ który adaptuje~się do otoczenia jest
  w~jakimś sensie mózgiem. Beer na podstawie tego stwierdził, że~w~takim
  właśnie sensie staw wodny jest mózgiem
  \parencite{Pickering-Cybernetics-in-Britain-Ver-2022}. Zaś w~roku
  $1961$~opublikowany zostaje \textit{Solaris}.

  Lema bardzo zajmował problem przypadków, a~ja chciałbym wiedzieć, czy to
  jest przypadek, czy nie? A~może zwyczajni widzę tutaj podobieństwa,
  których w~rzeczywistości nie ma?

  Trzeci i~ostatni problem, jaki chciałbym poruszyć opowiadania
  \textit{Ananke}. Pominiemy ciekawe aspekty tego opowiadania opisane
  przez dr.~Kukulaka i~skupimy~się na problemie uczenia maszynowego,
  stanowiącego klucz do katastrofy \textit{Ariela}.

\end{frame}
% ##################





% ##################
\begin{frame}
  \frametitle{Dylematy uczenia maszynowego}


  W~opowiadaniu tym Pirx, zmagający~się z~problemem swojego wieku,
  w~wymagającym bardzo dobrej kondycji zawodzie astronauty, bada
  dlaczego komputer pokładowy doprowadził do katastrofy tego statku.
  W~przeciwieństwie do \textit{Edenu}, w~opowiadaniu tym niezawodność
  „automatów” jest znacznie bardziej problematyczna. Jego treść
  wskazuje, że~gdyby na czas ludzie przejęli kontrolę nad lądowanie, to do
  katastrofy by nie doszło. Z~drugiej strony Pirx jest tam przedstawiony
  jako pilot starej generacji, którego przyzwyczajenia nie pozwalają
  docenić zalet nowych technologii, więc powstaje pytanie, na ile mamy
  tutaj do czynienia z~technologicznym konfliktem między pokoleniami,
  gdzie Pirx jest reprezentantem pokolenia Lema?

  Niezależnie od tego, motyw uczenia maszynowego, który Lem prawie na pewno
  przyswoił sobie w~czasie studiów cybernetycznych, jest w~tym opowiadaniu
  jawnie obecny.

\end{frame}
% ##################





% ##################
\begin{frame}
  \frametitle{Dylematy uczenia maszynowego}


  \textit{Przedstawił [von der Voyt] drogę, jaką przebywa każdy komputer~--
    od montażowej taśmy do~sterowni okrętu. [\ldots] Nie wypełnione jeszcze
    pamięcią [komputery], „nic nie widzące” jak noworodki, jechały do
    Bostonu, gdzie w~zakładach „Syntronics” odbywało~się ich programowanie.
    Po tym kolejnym akcie każdy komputer podlegał procedurze, która jest
    niejako odpowiednikiem nauk szkolnych, gdyż składa~się zarówno
    z~dostarczania pewnych „doświadczeń”, jak i~z~poddawania
    „egzaminom”.} \\
  \textit{Ananke}, str.~$290\text{--}291$,
  \parencite{Lem-Ogrod-ciemnosci-i-inne-opowiadania-Pub-2017}.

  Pirx w~końcu ustala, że~to cierpiący na nerwicę natręctw emerytowany
  pilot Warren Cornelius przeniósł na sztuczną inteligencję swoją chorobę
  i~ją przetrenował, czym doprowadziło do katastrofy \textit{Ariela},
  śmierci trzydziestu osób oraz~swojego samobójstwa. W~tym miejscu
  ciężko uniknąć wniosku, że~mamy do czynienia z~jakże cybernetycznym
  motywem: choroba człowieka ma swój odpowiednik w~chorobie maszyny.
  Choć w~tym wypadku, to człowiek „zaraził” nią komputer.

\end{frame}
% ##################





% ##################
\begin{frame}
  \frametitle{\textit{Ananke} w~czasach uczenia maszynowego}


  Katastrofa \textit{Ariela} wniknęła stąd, że~komputer błędne
  zidentyfikował Marsa, jako asteroidę i~z~trybu lądowania przeszedł
  w~tryb ucieczki. Na pewnym poziomie mamy więc do czynienia, z~błędem
  programu do rozpoznawania obrazów.

  Pozwolę sobie zasugerować dwa pytania, które stawia przed nami to
  opowiadania. Po pierwsze, które elementy opowieści o~uczeniu maszynowym
  pochodzą z~lektur Lema, a~które są wytworem jego fantazji? Po drugie,
  czy dalszy rozwój technologi potwierdził, czy zniwelował obawy Pirxa
  wobec komputerów? W~tym drugim przypadku pytamy~się trochę o~to „Czy
  Lem był dobrym wieszczem?”, co jest trochę usprawiedliwione dzisiejszym
  kontekstem.

  Tutaj chciałbym podziękować mojemu wieloletniemu znajomemu,
  \colorhref{https://www.linkedin.com/in/marabram/}{dr.~Marcinowi Abramowi},
  za~udzielone odpowiedni informacje i~wskazane literatury naukowej
  dotyczącej uczenia maszynowego i~rozpoznawania obrazów.

\end{frame}
% ##################





% ##################
\begin{frame}
  \frametitle{\textit{Ananke} w~czasach uczenia maszynowego}


  Jeśli dobrze zrozumiałem informacje jakie od niego otrzymałem,
  kierowanie osób cierpiących na nerwicę natręctw do pracy ze~sztuczną
  inteligencją jest bardzo złym pomysłem, bo \textsc{ai} ma bardzo silną
  tendencję, by wzmacniać wszelkie skrzywienie w~danych. Jak zaś wynika
  z~naszej lektury artykułu S. Beery, G. Van Horna i~P.~Perona
  \colorhref{https://arxiv.org/abs/1807.04975}
  {\textit{Recognition in Terra Incognita}}
  (\parencite{Beery-Horn-Perona-Recognition-in-Terra-Incognita-Pub-2018}),
  jakość działania programów rozpoznające obiekty bardzo się pogarsza, gdy
  zmieniamy tło.

  W~tym kontekście \textit{Ananke} i~dziś może służyć jako punkt wyjścia
  do głębszych przemyśleń.

\end{frame}
% ##################











% % ##################
% \begin{frame}
%   \frametitle{\textit{Dialogi} i~termin „sztuczna
%     inteligencja”}


%   Jednym z~podstawowych pojęć u~Lem jest przypadek, więc wspomnijmy o~jednym
%   z~nich. Wedle mojej wiedzy, termin \textit{sztuczna inteligencja} został
%   ukuty $1955$~roku przez czterech gigantów informatyki:
%   \colorhref{https://en.wikipedia.org/wiki/John_McCarthy_(computer_scientist)}
%   {Johna McCarthy’ego},
%   \colorhref{https://pl.wikipedia.org/wiki/Marvin\_Minsky}
%   {Marvina Minsky’ego},
%   \colorhref{https://en.wikipedia.org/wiki/Nathaniel\_Rochester\_(computer\_scientist)}{Nathaniela Rochestera}
%   i~\colorhref{https://en.wikipedia.org/wiki/Claude\_Shannon}
%   {Claude’a Shannona}. Termin ten wprowadzili na potrzeby projektu letnich
%   warsztatów, które odbył~się w~roku~$1956$ pod nazwą
%   \colorhref{https://en.wikipedia.org/wiki/Dartmouth_workshop}
%   {\textit{Dartmouth Summer Research Project on Artificial Intelligence}}
%   i~są uważany za~jedno z~najważniejszych wydarzeń w~historii tej
%   dziedziny. Wedle pewnych świadectw, w~nazwie tej konferencji nie użyto
%   słowa „cybernetyka”, ze względu na personalne animozje między niektórymi
%   z~organizatorów a~Norbertem Wienerem. Mało profesjonalne, ale jakże
%   przecież normalne i~ludzkie.

%   Niezależnie od~tego, warsztaty te odbywają~się w~tym samym roku, w~którym
%   Lem kończy \textit{Dialogi}, więc może gdyby ukończył je później,
%   sam termin „sztuczna inteligencja” byłby obecny w~ich tekście.

% \end{frame}
% % ##################





% % ##################
% \begin{frame}
%   \frametitle{Treść \textit{Dialogów}}


%   \textit{Dialogi} zaczynają~się, gdy Filonous napotyka w~„pięknym parku”
%   samotnie rozmyślającego Hylasa. Hylas, w~tym momencie zadeklarowany
%   materialista, wyjaśnia mu, że~gdy technologia osiągnie odpowiedni
%   poziom rozwoju, to będzie mógł zostać „wskrzeszony” przez idealne
%   skopiowanie całego jego ciała. Temat ten dyskutują do mniej więcej
%   jednej trzeciej dialogu numer iii, gdzie Hylas uznaje, że~nie potrafi
%   sprowadzić świadomości człowieka do~konfiguracji materialnych elementów
%   jego ciała. Moje osobiste odczucie jest takie, że~ich rozważania na temat
%   relacji świadomości oraz~materii zawierają zbyt dużo luk i~zbyt szybkich
%   przeskoków, by można było uważać je za filozoficznie zadowalające, ale
%   musimy to odłożyć na bok.

%   W~czasie tych rozmów mocno dyskutują problem tego, czy gdyby istniały
%   dwie idealne kopie jednego człowieka, to która byłaby prawdziwa
%   i~czy człowiek może istnieć jako „zwielokrotniony”. Komediową wersję
%   tego problemu znajdujemy w~\textit{Podróży siódmej} Ijona Tichego.

% \end{frame}
% % ##################





% % ##################
% \begin{frame}
%   \frametitle{Informacja}


%   Konsekwencją wywodu o~wymianie pamięci są następujące słowa wypowiadane
%   przez Filonous, które zasadniczo są parafrazą tego, co można znaleźć
%   w~\textit{Cybernetics\ldots} Wienera. \\
%   \textit{Zastanówmy~się mianowicie, co wymieniały mózgi w~naszym obrazowym
%     przykładzie, cóż to za rzecz była, za zjawisko, któreśmy nazwali
%     skrótowo „ładunkiem pamięci”? Był to, nieprawdaż, całokształt
%     strukturalnych zmian, nabytych przez ten mózg podczas jego istnienia,
%     to znaczy~-- był to zasób pewnej informacji. Kiedy to słowo padło,
%     znaleźliśmy~się w~kardynalnym miejscu naszych rozważań. W~samej rzeczy,
%     problem informacji, jej istoty, jej powstania, gromadzenia,
%     przechowywania i~użytkowania stanowi właściwą treść cybernetyki
%     i~zarazem klucz do zagadek, które przedstawiają systemy typu naszego
%     mózgu.} \\
%   \textsc{hylas}: \textit{Cóż tak osobliwego przedstawia informacja?} \\
%   \textsc{filonous}: \textit{Jest ona czymś wielce osobliwym, mój drogi,
%     albowiem nie jest ani materią, ani~energią, choć stanowi zjawisko
%     całkiem realne.} Str.~$55$, \parencite{Lem-Dialogi-Vol-I-Pub-1996}.

% \end{frame}
% % ##################





% % ##################
% \begin{frame}
%   \frametitle{Informacja}


%   W~powyższym fragmencie Filonous w~zasadzie zdefiniował cybernetykę
%   w~ujęciu Wienera. Również uznanie informacji za równorzędnego partnera
%   materii i~energii jest zaczerpnięte od ojca cybernetyki.

%   Czym jednak jest informacja? Tutaj Lem napotyka na te same problemy,
%   na~które napotkali później cybernetycy, czyli problem z~definicją pojęcia
%   informacji. Mówiąc pół żartem, pół serio, wykorzystywana przez nich
%   definicja informacji, zawiera za mało informacji o~tym czym jest
%   informacja ;).

%   Ponownie powołam~się na prof.~Bieleckiego, który stwierdził, iż~dobrze
%   znana teoria informacji, jest tak naprawdę teorią \alert{przesyłania}
%   informacji, o~samej informacji nie mówi ona zbyt wiele. On sam w~swoich
%   pracach na temat
%   \colorhref{https://www.youtube.com/watch?v=jaAADDFG8oA}{informacji
%     strukturalnej} próbuje wypełnić w~jakimś stopniu tą lukę. Jak zaraz
%   zobaczymy, również Lem dostrzegł różne niedomagania stosowanego
%   w~cybernetyce pojęcia informacji.

% \end{frame}
% % ##################





% % ##################
% \begin{frame}
%   \frametitle{Informacja}


%   Filonous w~następujący sposób wyjaśnia koncepcję informacji. \\

%   \textit{Otóż, wracają teraz do cybernetyki~-- informacja jest odwrotnością
%     entropii. Gdy tamta jest miarą bezładu~-- ta mierzy ład.}
%   Str.~$57$, \parencite{Lem-Dialogi-Vol-I-Pub-1996}.

%   Zestawmy to z~fragmentem wstępu do~\textit{Cybernetics\ldots} Wienera. \\
%   \textit{Pojęcie ilości informacji łączy~się bardzo naturalnie
%     z~klasycznym pojęciem mechaniki statystycznej, mianowicie z~pojęciem
%     \textbf{entropii}. Tak jak informacja zawarta w~systemie jest miarą jego
%     ładu, tak entropia systemu jest miarą jego bezładu. Jedna jest po
%     prostu odwrotnością drugiej (the~negative~of the~other).}
%   Tłum. swobodne, str.~$11\text{--}12$,
%   \parencite{Wiener-Cybernetics-Second-edition-Pub-2016}.

% \end{frame}
% % ##################





% % ##################
% \begin{frame}
%   \frametitle{Informacja}


%   Tego typu sposób myślenia o~informacji, zakorzeniło~się mocniej
%   w~naukach technicznych. Za przykład niech posłużą cytaty z~książki
%   Jerzego Seidlera
%   \textit{Nauka o~informacji. Tom~I: Podstawy, modele źródeł i~wstępne
%     przetwarzanie informacji}, opublikowanej przez Wydawnictwo Naukowe
%   i~Techniczne w~$1983$~roku
%   \parencite{Seidler-Nauka-o-informacji-Vol-I-Pub-1983}.

%   \textit{Podkreślono z~jednej strony uniwersalny charakter pojęć
%     informacji i~sygnału, powiązania dwustronne między nauką o~informacji
%     a~teorią systemów, z~drugiej zaś strony omówiono konkretne przykłady
%     z~telekomunikacji, miernictwa i~automatyki.} Str.~$4$,
%   \parencite{Seidler-Nauka-o-informacji-Vol-I-Pub-1983}.

%   \textit{Pojęciem „informacji” posługujemy~się często. Pojęcie to,
%     podobnie jak na przykład pojęcie „materia” lub „energia”, ma charakter
%     pojęcia pierwotnego i~ścisłe zdefiniowanie go za pomocą prostszych nie
%     jest możliwe. Pozostaje więc jedynie wyjaśnienie sensu tego pojęcia,
%     odpowiadającego jego intuicyjnemu rozumieniu.} Str.~$21$,
%   \parencite{Seidler-Nauka-o-informacji-Vol-I-Pub-1983}.

% \end{frame}
% % ##################


% % ##################
% \begin{frame}
%   \frametitle{Czym my już o~tym nie czytaliśmy?}


%   Czy w~twórczości pisarskiej Lema spotykamy~się z~obiektami, w~których
%   widać cybernetyczną koncepcję sieci? Choć mam kilka pomysłów gdzie zacząć
%   ich szukać, to mam nadzieję, że~w~tej i~innych kwestiach, Państwa
%   znajomość tego pisarza będzie lepsza, niż moja.

%   Niezależnie od tego, zanim zakończymy nasze rozważania, warto zwrócić na
%   chwilę uwagę na opowiadanie \textit{Ananke}. Badania dr.~Kukulaka
%   pokazały jak na tym dziele odbiły~się panujący w~owym czasie, szczęśliwie
%   krótkotrwały, pesymizm dotyczą perspektyw badań Układu Słonecznego,
%   z~naszego jednak punktu widzenia, warto zwrócić uwagę na inny aspekt tego
%   pełnego zniechęcenia opowiadania
%   \parencite{Kukulak-Two-Faces-of-Mars-Pub-2023}.

% \end{frame}
% % ##################








% % ##################
% \begin{frame}
%   \frametitle{\textit{Cyberiada}}


%   W~tym kontekście Filonous przytacza też drugą zasadę termodynamiki,
%   zob.~str.~$57$, \parencite{Lem-Dialogi-Vol-I-Pub-1996}. Literacki
%   dziełem Lema wytykającym braki takiej koncepcji informacji jest
%   . Bohaterowie pokonują w~niej Gębona,
%   wydobywając z~szumu wypełniającego przestrzeń kosmiczną informacje
%   o~średniej ilości piór łabędzia niemego na metr kwadratowy, liczbie
%   ziaren w~główce maku, liczbie ziaren piasku jaka dostaje~się do buta
%   na~plaży, etc., pod których zalewem przepada zbójca.


%   Pisząc to seminarium przyszła mi do głowy hipoteza, że~większość tekstów
%   z~\textit{Cyberiady} jest formą rozliczenia~się Lema z~niespełnionymi
%   obietnicami cybernetyki. Jeśli jest~się świadomym wielkiej roli


% \end{frame}
% % ##################





% % ##################
% \begin{frame}
%   \frametitle{Próg komplikacji minimalnej}


%   Ostatnią ważną z~naszego punktu widzenia koncepcją wprowadzoną w~dialogu
%   iii, jest pojęcie „progu komplikacji minimalnej”. Według Filonousa, jeśli
%   dany układ będzie tak złożony, że~przekroczy próg komplikacji minimalnej,
%   wówczas jest w~stanie stworzyć urządzenie tak samo złożone jako on sam.
%   Jak mówi \\
%   \textit{[\ldots] próg komplikacji minimalnej wyznacza ścisłą, fizykalnie
%     dającą~się mierzyć granicę między światem mechanizmów klasycznych
%     (maszyn) a~światem \textbf{organizmów}. Zauważ, proszę, iż~nie powiadam
%     „a~światem \textbf{żywych} organizmów”. „Życie” jest tu pojęciem
%     węższym, a~„organizacja”~-- pojęciem nadrzędnym, szerszym.}
%   Str.~$68\text{--}69$, \parencite{Lem-Dialogi-Vol-I-Pub-1996}.

%   Te cytaty wskazują na pewną ważną cechę myślenia cybernetycznego, której
%   ja nie dostrzegam w~naszym rozumieniu sztucznej inteligencji. Mianowicie,
%   w~cybernetyce rozmył~się podział między zwierzęciem, a~maszyną, więc
%   w~prowadzonych wtedy rozważaniach, nie pytano tyle o~sztuczną
%   inteligencję, co o~\alert{sztuczne życie}.

% \end{frame}
% % ##################





% % ##################
% \begin{frame}
%   \frametitle{Sztuczna inteligencja vs sztuczne życie}


%   Organizm zwierzęcia i~maszyna były z~punktu widzenia cybernetyki po
%   prostu dwoma układami, które bazowały na tych samych zasadach przesyłu
%   i~przetwarzania informacji, energii i~materii. W~skutek tego nie dzieli
%   je fundamentalna, ontologiczna różnica. Takimi samymi układami co
%   zwierzęta i~maszyny są zresztą również stawy wodne, uniwersytety czy
%   firmy.

%   Nie wydaje mi~się, żebyśmy dzisiaj zadawali sobie pytanie, czy komputer
%   jest tym samym typem obiektu, jak ludzkie ciało, czy to w~filozoficznych
%   rozważaniach, czy też w~filmach fabularnych albo grach video. Raczej
%   przyjmujemy domyślnie, że~to są dość odległe byty. Czymś zupełnie innym
%   jest pytanie, czy \alert{programy} komputerowe myślą i~czy posiadają
%   świadomość? To stawiamy sobie bardzo często.

% \end{frame}
% % ##################





% % ##################
% \begin{frame}
%   \frametitle{Wiener i~Lem}




% \end{frame}
% % ##################





% % ##################
% \begin{frame}
%   \frametitle{Wiener i~Lem}


%   Jeśli choć pobieżnie przekartkujemy \textit{Cybernetykę i~społeczeństwo}
%   Wienera, to znajdziemy tam wiele komentarzy do powyższych problemów, które
%   wyglądają jakby wyszły spod pióra Lema. Choć związek przyczynowy był
%   raczej w~drugą stronę.



%   Rodzi~się więc pytanie, czy Wiener wpłynął swoim poglądami na~filozofię
%   Lema? I~czy ten wpływ przetrwał okres zafascynowania pisarza cybernetyką?
%   W~tym momencie muszę lepiej przestudiować twórczość Wienera, by móc
%   w~tym temacie powiedzieć cokolwiek wartego uwagi. Pytanie to musi więc
%   na razie pozostać otwarte.

% \end{frame}
% % ##################





% % ##################
% \begin{frame}
%   \frametitle{Wróćmy jeszcze do \textit{Dialogów}}


%   Dla szerszego kontekstu warto ponownie zacytować książkę \textit{Nauka
%     o~informacji} Seidlera. \\
%   \textit{Zarysowuje~się też dalej idący proces integracji nauk.
%     W~złożonych systemach informacyjnych można bowiem dopatrzyć~się
%     coraz więcej podobieństw do~procesów informacyjnych zachodzących
%     w~organizmach żywych, a~nawet i~w~społeczeństwie, i~w~rezultacie
%     zaczyna~się pojawiać wzajemne oddziaływanie między do~niedawna tak
%     odległymi dziedzinami jak: technika, biologia i~nauki społeczne.
%     Kluczowym elementem wspólnym dla tych dziedzin jest pojęcie informacji.}
%   Str.~$10$,
%   \parencite{Seidler-Nauka-o-informacji-Vol-I-Pub-1983}.

%   Wróćmy jeszcze na chwilę do~tekstu \textit{Dialogów}. W~dialogach~iv i~v
%   dochodzi do przesunięcia stylu dzieła. Coraz mniej są to dialogi w~pełnym
%   znaczeniu tego słowa, coraz bardziej zaś wykłady Filonousa o~cybernetyce,
%   których samotnym słuchaczem staje~się Hylas.

% \end{frame}
% % ##################





% % ##################
% \begin{frame}
%   \frametitle{Sieci}


%   W~dialogu~iv Filonous i~Hylas rozważają problem świadomości, ewolucji
%   darwinowskiej i~jej podobieństwo, bądź jego braku do budowania maszyn,
%   podłączenia~się ludzi do cudzych systemów nerwowych, jak również dochodzą
%   do wniosku, że~z~„żelaznych brył” można zbudować istotę równie żywą
%   jak~koń czy krowa.

%   W~dialogu~v wprowadzone zostaje pojęcie sieci, które od tego momentu
%   staje~się z~jednym z~podstawowych elementów wykładów Filonousa
%   o~cybernetyce. \\
%   \textsc{hylas}: \textit{Dobrze. Czy masz zamiar mówić teraz o~zbiorze
%     układów zwanych sieciami?} \\
%   \textsc{filonous}: \textit{Tak. Zbiór ten obejmuje układy o~stopniu
%     złożoności większym bądź równym~„w”. Przez „w” rozumiem minimalną
%     złożoność, jaką musi wykazać układ, abyśmy mogli zaliczyć go do
%     zbioru.} \\
%   Str.~$97$, \parencite{Lem-Dialogi-Vol-I-Pub-1996}.

% \end{frame}
% % ##################





% % ##################
% \begin{frame}
%   \frametitle{Sieci neuronowe}


%   Kilka stron dalej, zob.~str.~$111$,
%   \parencite{Lem-Dialogi-Vol-I-Pub-1996}, czytamy już o~sieciach
%   neuronowych rozważanych przez
%   \colorhref{https://en.wikipedia.org/wiki/Warren\_Sturgis\_McCulloch}
%   {Warrena Strugisa McCullocha} ($1898\text{--}1969$)
%   i~\colorhref{https://en.wikipedia.org/wiki/Walter_Pitts}
%   {Waltera Harry’ego Pittsa Jr.} ($1923\text{--}1969$).
%   Autorzy ci w~$1943$
%   opublikowali artykuł \colorhref{https://en.wikipedia.org/wiki/A\_Logical\_Calculus\_of\_the\_Ideas\_Immanent\_in\_Nervous\_Activity}
%   {\textit{A~Logical Calculus of the Ideas Immanent in Nervous Activity}}
%   w~który zaproponowali model neuronu, zbudowanej z~nich sieci, jak również
%   przeprowadzają teoretyczną analizę problemu wykonywania na niej obliczeń
%   logicznych klasycznego rachunku zdań
%   \parencite{Bielecki-Sztuczne-sieci-neuronowe-Slowniki-ETC-Vol-XIII-Ver-2025}.

%   Omawiany artykuł uważa~się za~początek badań sieci neuronowych, które
%   od~roku~$2012$ przeżywają swoją kolejną młodość. Sami zaś McCulloch
%   i~Pitts są pierwszymi cybernetykami, a~może w~ogóle pierwszymi realnie
%   istniejącymi osobami, wymienionymi z~imienia i~nazwiska
%   w~\textit{Dialogach}.

%   % W~dialogu~v wprowadzone zostaje pojęcie sieci, które od tego momentu
%   % staje~się z~jednym z~podstawowych elementów wykładów Filonousa
%   % o~cybernetyce. \\
%   % \textsc{hylas}: \textit{Dobrze. Czy masz zamiar mówić teraz o~zbiorze
%   %   układów zwanych sieciami?} \\
%   % \textsc{filonous}: \textit{Tak. Zbiór ten obejmuje układy o~stopniu
%   %   złożoności większym bądź równym~„w”. Przez „w” rozumiem minimalną
%   %   złożoność, jaką musi wykazać układ, abyśmy mogli zaliczyć go do
%   %   zbioru.} \\
%   % Str.~$97$, \parencite{Lem-Dialogi-Vol-I-Pub-1996}.

%   % Kilka stron dalej, zob.~str.~$111$,
%   % \parencite{Lem-Dialogi-Vol-I-Pub-1996}, czytamy już o~sieciach
%   % neuronowych rozważanych przez
%   % \colorhref{https://en.wikipedia.org/wiki/Warren\_Sturgis\_McCulloch}
%   % {Warrena Strugisa McCullocha} ($1898\text{--}1969$)
%   % i~\colorhref{https://en.wikipedia.org/wiki/Walter_Pitts}
%   % {Waltera Harry’ego Pittsa Jr.} ($1923\text{--}1969$). Autorzy ci w~$1943$
%   % opublikowali artykuł \textit{A~Logical Calculus of the Ideas Immanent in
%   %   Nervous Activity} w~który proponują model neuronu, zbudowanej z~nich
%   % sieci, jak również analizują teoretyczną możliwość przeprowadzania za~jej
%   % pomocą obliczeń logicznych w~ramach klasycznego rachunku zdań
%   % \parencite{Bielecki-Sztuczne-sieci-neuronowe-Slowniki-ETC-Vol-XIII-Ver-2025}.
%   % Tym samym McCulloch i~Pitts zapoczątkowali badanie sieci neuronowych,
%   % które od roku~$2012$ przeżywają swoją kolejną młodość. Są oni też
%   % pierwszymi cybernetykami, a~może w~ogóle pierwszymi istniejącymi osobami,
%   % wymienionymi z~nazwiska w~\textit{Dialogach}.

%   % Lem przy tym doskonale wie, że~cybernetyczne sieci mogę~się uczyć na
%   % podstawie przeszłych doświadczeń, co ilustruje choćby na przykładzie
%   % sieci jaką „jest” niemowlę. Pamiętajmy, że~dla cybernetyka
%   % $\text{zwierzę} = \text{maszyna}$.

% \end{frame}
% % ##################





% % ##################
% \begin{frame}
%   \frametitle{Perceptron}


%   Lem przy tym doskonale wie, że~cybernetyczne sieci mogę~się uczyć na
%   podstawie przeszłych doświadczeń, co ilustruje choćby na przykładzie
%   sieci jaką „jest” niemowlę. Pamiętajmy, że~dla cybernetyka
%   $\text{zwierzę} = \text{maszyna}$. Znał więc, przynajmniej pobieżnie
%   i~we wczesnym stadium rozwoju, pojęcie „sieci neuronowej”, ale według
%   mojej obecnej interpretacji, uważał ją tylko za~szczególny przypadek
%   ogólnego pojęcia „sieci”, którego używa znacznie częściej w~dialogu~v.

%   Warto nadmienić, że~pierwszy układ elektroniczny, działający
%   wedle zasad sieci neuronowej skonstruował amerykański psycholog
%   \colorhref{https://en.wikipedia.org/wiki/Frank\_Rosenblatt}{Frank
%     Rosenblatt} ($1928\text{--}1971$) w~$1957$~roku, przez co bywa nazywany
%   „the father~of deep learning”. Jego układ nosił nazwę
%   \colorhref{https://en.wikipedia.org/wiki/Perceptron}{\textit{perceptronu}}
%   i~na jego projekt duży wpływ miała ówczesna wiedza na temat neuronów
%   znajdujących~się w~siatkówce oka
%   \parencite{Bielecki-Sztuczne-sieci-neuronowe-Slowniki-ETC-Vol-XIII-Ver-2025}. Oznacza to, że~gdy Lem opisywał sieci w~\textit{Dialogach} były
%   one z~jego punktu widzenia koncepcją czysto teoretyczną.

% \end{frame}
% % ##################











% % ##################
% \begin{frame}
%   \frametitle{Czym my już o~tym nie czytaliśmy?}



% \end{frame}
% % ##################










% % ######################################
% \section{Zakończenie i~plany na przyszłość}
% % ######################################



% % ##################
% \begin{frame}
%   \frametitle{Zakończenie}


%   Choć tekst \textit{Dialogów} jest bogatszy, niż zaprezentowana powyżej,
%   przedstawione rozważania dobrze oddają stan moich skromnych badań nad nim.
%   Jak wspomniałem na początku, jestem dopiero tak w~jednej trzeciej
%   planowanych nad cybernetyką i~Lemem, więc wciąż nie umiem odpowiedzieć
%   na~większość pytań.

%   W~planach mam dokładną lekturę dwóch wymienionych wcześniej dzieł
%   Wienera, jak i~zbioru artykułów i~dyskusji \textit{Cybernetyka. Za
%     i~przeciw}, który jest świadectwem debaty nad tą dziedziną, jaka
%   odbyła~się w~Polsce około roku~$1965$. Temat wymaga też dokładnej lektury
%   dzieł Lema oraz prześledzenia znanej chronologii ich powstania.

%   Tematów do badań na pewno nie brakuje.

% \end{frame}
% % ##################










% ######################################
\appendix
% ######################################





% ######################################
\EndingSlide{Dziękuję! Pytania?}
% ######################################









% ####################################################################
% ####################################################################
% Bibliography

\printbibliography










% ############################

% Koniec dokumentu
\end{document}
