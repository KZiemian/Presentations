% ------------------------------------------------------------------------------------------------------------------
% Basic configuration of Beamera class and Jagiellonian theme
% ------------------------------------------------------------------------------------------------------------------
\RequirePackage[l2tabu, orthodox]{nag}



\ifx\PresentationStyle\notset
  \def\PresentationStyle{light}
\fi



% Options: t -- align frame text to the top
\documentclass[10pt,t]{beamer}
\mode<presentation>
\usetheme[style=\PresentationStyle]{jagiellonian}





% ------------------------------------------------------------------------------------
% Procesing configuration files of Jagiellonian theme located in
% the directory "preambule"
% ------------------------------------------------------------------------------------
% Configuration for polish language
% Need description
\usepackage[polish]{babel}
% Need description
\usepackage[MeX]{polski}



% ------------------------------
% Better support of polish chars in technical parts of PDF
% ------------------------------
\hypersetup{pdfencoding=auto,psdextra}

% Package "textpos" give as enviroment "textblock" which is very usefull in
% arranging text on slides.

% This is standard configuration of "textpos"
\usepackage[overlay,absolute]{textpos}

% If you need to see bounds of "textblock's" comment line above and uncomment
% one below.

% Caution! When showboxes option is on significant ammunt of space is add
% to the top of textblock and as such, everyting put in them gone down.
% We need to check how to remove this bug.

% \usepackage[showboxes,overlay,absolute]{textpos}



% Setting scale length for package "textpos"
\setlength{\TPHorizModule}{10mm}
\setlength{\TPVertModule}{\TPHorizModule}


% ---------------------------------------
% TikZ
% ---------------------------------------
% Importing TikZ libraries
\usetikzlibrary{arrows.meta}
\usetikzlibrary{positioning}





% % Configuration package "bm" that need for making bold symbols
% \newcommand{\bmmax}{0}
% \newcommand{\hmmax}{0}
% \usepackage{bm}




% ---------------------------------------
% Packages for scientific texts
% ---------------------------------------
% \let\lll\undefined  % Sometimes you must use this line to allow
% "amsmath" package to works with packages with packages for polish
% languge imported
% /preambul/LanguageSettings/JagiellonianPolishLanguageSettings.tex.
% This comments (probably) removes polish letter Ł.
\usepackage{amsmath}  % Packages from American Mathematical Society (AMS)
\usepackage{amssymb}
\usepackage{amscd}
\usepackage{amsthm}
\usepackage{siunitx}  % Package for typsetting SI units.
\usepackage{upgreek}  % Better looking greek letters.
% Example of using upgreek: pi = \uppi


\usepackage{calrsfs}  % Zmienia czcionkę kaligraficzną w \mathcal
% na ładniejszą. Może w innych miejscach robi to samo, ale o tym nic
% nie wiem.










% ---------------------------------------
% Packages written for lectures "Geometria 3D dla twórców gier wideo"
% ---------------------------------------
% \usepackage{./ProgramowanieSymulacjiFizykiPaczki/ProgramowanieSymulacjiFizyki}
% \usepackage{./ProgramowanieSymulacjiFizykiPaczki/ProgramowanieSymulacjiFizykiIndeksy}
% \usepackage{./ProgramowanieSymulacjiFizykiPaczki/ProgramowanieSymulacjiFizykiTikZStyle}





% !!!!!!!!!!!!!!!!!!!!!!!!!!!!!!
% !!!!!!!!!!!!!!!!!!!!!!!!!!!!!!
% EVIL STUFF
\if\JUlogotitle1
\edef\LogoJUPath{LogoJU_\JUlogoLang/LogoJU_\JUlogoShape_\JUlogoColor.pdf}
\titlegraphic{\hfill\includegraphics[scale=0.22]
{./JagiellonianPictures/\LogoJUPath}}
\fi
% ---------------------------------------
% Commands for handling colors
% ---------------------------------------


% Command for setting normal text color for some text in math modestyle
% Text color depend on used style of Jagiellonian

% Beamer version of command
\newcommand{\TextWithNormalTextColor}[1]{%
  {\color{jNormalTextFGColor}
    \setbeamercolor{math text}{fg=jNormalTextFGColor} {#1}}
}

% Article and similar classes version of command
% \newcommand{\TextWithNormalTextColor}[1]{%
%   {\color{jNormalTextsFGColor} {#1}}
% }



% Beamer version of command
\newcommand{\NormalTextInMathMode}[1]{%
  {\color{jNormalTextFGColor}
    \setbeamercolor{math text}{fg=jNormalTextFGColor} \text{#1}}
}


% Article and similar classes version of command
% \newcommand{\NormalTextInMathMode}[1]{%
%   {\color{jNormalTextsFGColor} \text{#1}}
% }




% Command that sets color of some mathematical text to the same color
% that has normal text in header (?)

% Beamer version of the command
\newcommand{\MathTextFrametitleFGColor}[1]{%
  {\color{jFrametitleFGColor}
    \setbeamercolor{math text}{fg=jFrametitleFGColor} #1}
}

% Article and similar classes version of the command
% \newcommand{\MathTextWhiteColor}[1]{{\color{jFrametitleFGColor} #1}}





% Command for setting color of alert text for some text in math modestyle

% Beamer version of the command
\newcommand{\MathTextAlertColor}[1]{%
  {\color{jOrange} \setbeamercolor{math text}{fg=jOrange} #1}
}

% Article and similar classes version of the command
% \newcommand{\MathTextAlertColor}[1]{{\color{jOrange} #1}}





% Command that allow you to sets chosen color as the color of some text into
% math mode. Due to some nuances in the way that Beamer handle colors
% it not work in all cases. We hope that in the future we will improve it.

% Beamer version of the command
\newcommand{\SetMathTextColor}[2]{%
  {\color{#1} \setbeamercolor{math text}{fg=#1} #2}
}


% Article and similar classes version of the command
% \newcommand{\SetMathTextColor}[2]{{\color{#1} #2}}










% ---------------------------------------
% Commands for few special slides
% ---------------------------------------
\newcommand{\EndingSlide}[1]{%
  \begin{frame}[standout]

    \begingroup

    \color{jFrametitleFGColor}

    #1

    \endgroup

  \end{frame}
}










% ---------------------------------------
% Commands for setting background pictures for some slides
% ---------------------------------------
\newcommand{\TitleBackgroundPicture}
{./JagiellonianPictures/Backgrounds/LajkonikDark.png}
\newcommand{\SectionBackgroundPicture}
{./JagiellonianPictures/Backgrounds/LajkonikLight.png}



\newcommand{\TitleSlideWithPicture}{%
  \begingroup

  \usebackgroundtemplate{%
    \includegraphics[height=\paperheight]{\TitleBackgroundPicture}}

  \maketitle

  \endgroup
}





\newcommand{\SectionSlideWithPicture}[1]{%
  \begingroup

  \usebackgroundtemplate{%
    \includegraphics[height=\paperheight]{\SectionBackgroundPicture}}

  \setbeamercolor{titlelike}{fg=normal text.fg}

  \section{#1}

  \endgroup
}










% ---------------------------------------
% Commands for lectures "Geometria 3D dla twórców gier wideo"
% Polish version
% ---------------------------------------
% Komendy teraz wykomentowane były potrzebne, gdy loga były na niebieskim
% tle, nie na białym. A są na białym bo tego chcieli w biurze projektu.
% \newcommand{\FundingLogoWhitePicturePL}
% {./PresentationPictures/CommonPictures/logotypFundusze_biale_bez_tla2.pdf}
\newcommand{\FundingLogoColorPicturePL}
{./PresentationPictures/CommonPictures/European_Funds_color_PL.pdf}
% \newcommand{\EULogoWhitePicturePL}
% {./PresentationPictures/CommonPictures/logotypUE_biale_bez_tla2.pdf}
\newcommand{\EUSocialFundLogoColorPicturePL}
{./PresentationPictures/CommonPictures/EU_Social_Fund_color_PL.pdf}
% \newcommand{\ZintegrUJLogoWhitePicturePL}
% {./PresentationPictures/CommonPictures/zintegruj-logo-white.pdf}
\newcommand{\ZintegrUJLogoColorPicturePL}
{./PresentationPictures/CommonPictures/ZintegrUJ_color.pdf}
\newcommand{\JULogoColorPicturePL}
{./JagiellonianPictures/LogoJU_PL/LogoJU_A_color.pdf}





\newcommand{\GeometryThreeDSpecialBeginningSlidePL}{%
  \begin{frame}[standout]

    \begin{textblock}{11}(1,0.7)

      \begin{flushleft}

        \mdseries

        \footnotesize

        \color{jFrametitleFGColor}

        Materiał powstał w ramach projektu współfinansowanego ze środków
        Unii Europejskiej w ramach Europejskiego Funduszu Społecznego
        POWR.03.05.00-00-Z309/17-00.

      \end{flushleft}

    \end{textblock}





    \begin{textblock}{10}(0,2.2)

      \tikz \fill[color=jBackgroundStyleLight] (0,0) rectangle (12.8,-1.5);

    \end{textblock}


    \begin{textblock}{3.2}(1,2.45)

      \includegraphics[scale=0.3]{\FundingLogoColorPicturePL}

    \end{textblock}


    \begin{textblock}{2.5}(3.7,2.5)

      \includegraphics[scale=0.2]{\JULogoColorPicturePL}

    \end{textblock}


    \begin{textblock}{2.5}(6,2.4)

      \includegraphics[scale=0.1]{\ZintegrUJLogoColorPicturePL}

    \end{textblock}


    \begin{textblock}{4.2}(8.4,2.6)

      \includegraphics[scale=0.3]{\EUSocialFundLogoColorPicturePL}

    \end{textblock}

  \end{frame}
}



\newcommand{\GeometryThreeDTwoSpecialBeginningSlidesPL}{%
  \begin{frame}[standout]

    \begin{textblock}{11}(1,0.7)

      \begin{flushleft}

        \mdseries

        \footnotesize

        \color{jFrametitleFGColor}

        Materiał powstał w ramach projektu współfinansowanego ze środków
        Unii Europejskiej w ramach Europejskiego Funduszu Społecznego
        POWR.03.05.00-00-Z309/17-00.

      \end{flushleft}

    \end{textblock}





    \begin{textblock}{10}(0,2.2)

      \tikz \fill[color=jBackgroundStyleLight] (0,0) rectangle (12.8,-1.5);

    \end{textblock}


    \begin{textblock}{3.2}(1,2.45)

      \includegraphics[scale=0.3]{\FundingLogoColorPicturePL}

    \end{textblock}


    \begin{textblock}{2.5}(3.7,2.5)

      \includegraphics[scale=0.2]{\JULogoColorPicturePL}

    \end{textblock}


    \begin{textblock}{2.5}(6,2.4)

      \includegraphics[scale=0.1]{\ZintegrUJLogoColorPicturePL}

    \end{textblock}


    \begin{textblock}{4.2}(8.4,2.6)

      \includegraphics[scale=0.3]{\EUSocialFundLogoColorPicturePL}

    \end{textblock}

  \end{frame}





  \TitleSlideWithPicture
}



\newcommand{\GeometryThreeDSpecialEndingSlidePL}{%
  \begin{frame}[standout]

    \begin{textblock}{11}(1,0.7)

      \begin{flushleft}

        \mdseries

        \footnotesize

        \color{jFrametitleFGColor}

        Materiał powstał w ramach projektu współfinansowanego ze środków
        Unii Europejskiej w~ramach Europejskiego Funduszu Społecznego
        POWR.03.05.00-00-Z309/17-00.

      \end{flushleft}

    \end{textblock}





    \begin{textblock}{10}(0,2.2)

      \tikz \fill[color=jBackgroundStyleLight] (0,0) rectangle (12.8,-1.5);

    \end{textblock}


    \begin{textblock}{3.2}(1,2.45)

      \includegraphics[scale=0.3]{\FundingLogoColorPicturePL}

    \end{textblock}


    \begin{textblock}{2.5}(3.7,2.5)

      \includegraphics[scale=0.2]{\JULogoColorPicturePL}

    \end{textblock}


    \begin{textblock}{2.5}(6,2.4)

      \includegraphics[scale=0.1]{\ZintegrUJLogoColorPicturePL}

    \end{textblock}


    \begin{textblock}{4.2}(8.4,2.6)

      \includegraphics[scale=0.3]{\EUSocialFundLogoColorPicturePL}

    \end{textblock}





    \begin{textblock}{11}(1,4)

      \begin{flushleft}

        \mdseries

        \footnotesize

        \RaggedRight

        \color{jFrametitleFGColor}

        Treść niniejszego wykładu jest udostępniona na~licencji
        Creative Commons (\textsc{cc}), z~uzna\-niem autorstwa
        (\textsc{by}) oraz udostępnianiem na tych samych warunkach
        (\textsc{sa}). Rysunki i~wy\-kresy zawarte w~wykładzie są
        autorstwa dr.~hab.~Pawła Węgrzyna et~al. i~są dostępne
        na tej samej licencji, o~ile nie wskazano inaczej.
        W~prezentacji wykorzystano temat Beamera Jagiellonian,
        oparty na~temacie Metropolis Matthiasa Vogelgesanga,
        dostępnym na licencji \LaTeX{} Project Public License~1.3c
        pod adresem: \colorhref{https://github.com/matze/mtheme}
        {https://github.com/matze/mtheme}.

        Projekt typograficzny: Iwona Grabska-Gradzińska \\
        Skład: Kamil Ziemian;
        Korekta: Wojciech Palacz \\
        Modele: Dariusz Frymus, Kamil Nowakowski \\
        Rysunki i~wykresy: Kamil Ziemian, Paweł Węgrzyn, Wojciech Palacz

      \end{flushleft}

    \end{textblock}

  \end{frame}
}



\newcommand{\GeometryThreeDTwoSpecialEndingSlidesPL}[1]{%
  \begin{frame}[standout]


    \begin{textblock}{11}(1,0.7)

      \begin{flushleft}

        \mdseries

        \footnotesize

        \color{jFrametitleFGColor}

        Materiał powstał w ramach projektu współfinansowanego ze środków
        Unii Europejskiej w~ramach Europejskiego Funduszu Społecznego
        POWR.03.05.00-00-Z309/17-00.

      \end{flushleft}

    \end{textblock}





    \begin{textblock}{10}(0,2.2)

      \tikz \fill[color=jBackgroundStyleLight] (0,0) rectangle (12.8,-1.5);

    \end{textblock}


    \begin{textblock}{3.2}(1,2.45)

      \includegraphics[scale=0.3]{\FundingLogoColorPicturePL}

    \end{textblock}


    \begin{textblock}{2.5}(3.7,2.5)

      \includegraphics[scale=0.2]{\JULogoColorPicturePL}

    \end{textblock}


    \begin{textblock}{2.5}(6,2.4)

      \includegraphics[scale=0.1]{\ZintegrUJLogoColorPicturePL}

    \end{textblock}


    \begin{textblock}{4.2}(8.4,2.6)

      \includegraphics[scale=0.3]{\EUSocialFundLogoColorPicturePL}

    \end{textblock}





    \begin{textblock}{11}(1,4)

      \begin{flushleft}

        \mdseries

        \footnotesize

        \RaggedRight

        \color{jFrametitleFGColor}

        Treść niniejszego wykładu jest udostępniona na~licencji
        Creative Commons (\textsc{cc}), z~uzna\-niem autorstwa
        (\textsc{by}) oraz udostępnianiem na tych samych warunkach
        (\textsc{sa}). Rysunki i~wy\-kresy zawarte w~wykładzie są
        autorstwa dr.~hab.~Pawła Węgrzyna et~al. i~są dostępne
        na tej samej licencji, o~ile nie wskazano inaczej.
        W~prezentacji wykorzystano temat Beamera Jagiellonian,
        oparty na~temacie Metropolis Matthiasa Vogelgesanga,
        dostępnym na licencji \LaTeX{} Project Public License~1.3c
        pod adresem: \colorhref{https://github.com/matze/mtheme}
        {https://github.com/matze/mtheme}.

        Projekt typograficzny: Iwona Grabska-Gradzińska \\
        Skład: Kamil Ziemian;
        Korekta: Wojciech Palacz \\
        Modele: Dariusz Frymus, Kamil Nowakowski \\
        Rysunki i~wykresy: Kamil Ziemian, Paweł Węgrzyn, Wojciech Palacz

      \end{flushleft}

    \end{textblock}

  \end{frame}





  \begin{frame}[standout]

    \begingroup

    \color{jFrametitleFGColor}

    #1

    \endgroup

  \end{frame}
}



\newcommand{\GeometryThreeDSpecialEndingSlideVideoPL}{%
  \begin{frame}[standout]

    \begin{textblock}{11}(1,0.7)

      \begin{flushleft}

        \mdseries

        \footnotesize

        \color{jFrametitleFGColor}

        Materiał powstał w ramach projektu współfinansowanego ze środków
        Unii Europejskiej w~ramach Europejskiego Funduszu Społecznego
        POWR.03.05.00-00-Z309/17-00.

      \end{flushleft}

    \end{textblock}





    \begin{textblock}{10}(0,2.2)

      \tikz \fill[color=jBackgroundStyleLight] (0,0) rectangle (12.8,-1.5);

    \end{textblock}


    \begin{textblock}{3.2}(1,2.45)

      \includegraphics[scale=0.3]{\FundingLogoColorPicturePL}

    \end{textblock}


    \begin{textblock}{2.5}(3.7,2.5)

      \includegraphics[scale=0.2]{\JULogoColorPicturePL}

    \end{textblock}


    \begin{textblock}{2.5}(6,2.4)

      \includegraphics[scale=0.1]{\ZintegrUJLogoColorPicturePL}

    \end{textblock}


    \begin{textblock}{4.2}(8.4,2.6)

      \includegraphics[scale=0.3]{\EUSocialFundLogoColorPicturePL}

    \end{textblock}





    \begin{textblock}{11}(1,4)

      \begin{flushleft}

        \mdseries

        \footnotesize

        \RaggedRight

        \color{jFrametitleFGColor}

        Treść niniejszego wykładu jest udostępniona na~licencji
        Creative Commons (\textsc{cc}), z~uzna\-niem autorstwa
        (\textsc{by}) oraz udostępnianiem na tych samych warunkach
        (\textsc{sa}). Rysunki i~wy\-kresy zawarte w~wykładzie są
        autorstwa dr.~hab.~Pawła Węgrzyna et~al. i~są dostępne
        na tej samej licencji, o~ile nie wskazano inaczej.
        W~prezentacji wykorzystano temat Beamera Jagiellonian,
        oparty na~temacie Metropolis Matthiasa Vogelgesanga,
        dostępnym na licencji \LaTeX{} Project Public License~1.3c
        pod adresem: \colorhref{https://github.com/matze/mtheme}
        {https://github.com/matze/mtheme}.

        Projekt typograficzny: Iwona Grabska-Gradzińska;
        Skład: Kamil Ziemian \\
        Korekta: Wojciech Palacz;
        Modele: Dariusz Frymus, Kamil Nowakowski \\
        Rysunki i~wykresy: Kamil Ziemian, Paweł Węgrzyn, Wojciech Palacz \\
        Montaż: Agencja Filmowa Film \& Television Production~-- Zbigniew
        Masklak

      \end{flushleft}

    \end{textblock}

  \end{frame}
}





\newcommand{\GeometryThreeDTwoSpecialEndingSlidesVideoPL}[1]{%
  \begin{frame}[standout]

    \begin{textblock}{11}(1,0.7)

      \begin{flushleft}

        \mdseries

        \footnotesize

        \color{jFrametitleFGColor}

        Materiał powstał w ramach projektu współfinansowanego ze środków
        Unii Europejskiej w~ramach Europejskiego Funduszu Społecznego
        POWR.03.05.00-00-Z309/17-00.

      \end{flushleft}

    \end{textblock}





    \begin{textblock}{10}(0,2.2)

      \tikz \fill[color=jBackgroundStyleLight] (0,0) rectangle (12.8,-1.5);

    \end{textblock}


    \begin{textblock}{3.2}(1,2.45)

      \includegraphics[scale=0.3]{\FundingLogoColorPicturePL}

    \end{textblock}


    \begin{textblock}{2.5}(3.7,2.5)

      \includegraphics[scale=0.2]{\JULogoColorPicturePL}

    \end{textblock}


    \begin{textblock}{2.5}(6,2.4)

      \includegraphics[scale=0.1]{\ZintegrUJLogoColorPicturePL}

    \end{textblock}


    \begin{textblock}{4.2}(8.4,2.6)

      \includegraphics[scale=0.3]{\EUSocialFundLogoColorPicturePL}

    \end{textblock}





    \begin{textblock}{11}(1,4)

      \begin{flushleft}

        \mdseries

        \footnotesize

        \RaggedRight

        \color{jFrametitleFGColor}

        Treść niniejszego wykładu jest udostępniona na~licencji
        Creative Commons (\textsc{cc}), z~uzna\-niem autorstwa
        (\textsc{by}) oraz udostępnianiem na tych samych warunkach
        (\textsc{sa}). Rysunki i~wy\-kresy zawarte w~wykładzie są
        autorstwa dr.~hab.~Pawła Węgrzyna et~al. i~są dostępne
        na tej samej licencji, o~ile nie wskazano inaczej.
        W~prezentacji wykorzystano temat Beamera Jagiellonian,
        oparty na~temacie Metropolis Matthiasa Vogelgesanga,
        dostępnym na licencji \LaTeX{} Project Public License~1.3c
        pod adresem: \colorhref{https://github.com/matze/mtheme}
        {https://github.com/matze/mtheme}.

        Projekt typograficzny: Iwona Grabska-Gradzińska;
        Skład: Kamil Ziemian \\
        Korekta: Wojciech Palacz;
        Modele: Dariusz Frymus, Kamil Nowakowski \\
        Rysunki i~wykresy: Kamil Ziemian, Paweł Węgrzyn, Wojciech Palacz \\
        Montaż: Agencja Filmowa Film \& Television Production~-- Zbigniew
        Masklak

      \end{flushleft}

    \end{textblock}

  \end{frame}





  \begin{frame}[standout]


    \begingroup

    \color{jFrametitleFGColor}

    #1

    \endgroup

  \end{frame}
}










% ---------------------------------------
% Commands for lectures "Geometria 3D dla twórców gier wideo"
% English version
% ---------------------------------------
% \newcommand{\FundingLogoWhitePictureEN}
% {./PresentationPictures/CommonPictures/logotypFundusze_biale_bez_tla2.pdf}
\newcommand{\FundingLogoColorPictureEN}
{./PresentationPictures/CommonPictures/European_Funds_color_EN.pdf}
% \newcommand{\EULogoWhitePictureEN}
% {./PresentationPictures/CommonPictures/logotypUE_biale_bez_tla2.pdf}
\newcommand{\EUSocialFundLogoColorPictureEN}
{./PresentationPictures/CommonPictures/EU_Social_Fund_color_EN.pdf}
% \newcommand{\ZintegrUJLogoWhitePictureEN}
% {./PresentationPictures/CommonPictures/zintegruj-logo-white.pdf}
\newcommand{\ZintegrUJLogoColorPictureEN}
{./PresentationPictures/CommonPictures/ZintegrUJ_color.pdf}
\newcommand{\JULogoColorPictureEN}
{./JagiellonianPictures/LogoJU_EN/LogoJU_A_color.pdf}



\newcommand{\GeometryThreeDSpecialBeginningSlideEN}{%
  \begin{frame}[standout]

    \begin{textblock}{11}(1,0.7)

      \begin{flushleft}

        \mdseries

        \footnotesize

        \color{jFrametitleFGColor}

        This content was created as part of a project co-financed by the
        European Union within the framework of the European Social Fund
        POWR.03.05.00-00-Z309/17-00.

      \end{flushleft}

    \end{textblock}





    \begin{textblock}{10}(0,2.2)

      \tikz \fill[color=jBackgroundStyleLight] (0,0) rectangle (12.8,-1.5);

    \end{textblock}


    \begin{textblock}{3.2}(0.7,2.45)

      \includegraphics[scale=0.3]{\FundingLogoColorPictureEN}

    \end{textblock}


    \begin{textblock}{2.5}(4.15,2.5)

      \includegraphics[scale=0.2]{\JULogoColorPictureEN}

    \end{textblock}


    \begin{textblock}{2.5}(6.35,2.4)

      \includegraphics[scale=0.1]{\ZintegrUJLogoColorPictureEN}

    \end{textblock}


    \begin{textblock}{4.2}(8.4,2.6)

      \includegraphics[scale=0.3]{\EUSocialFundLogoColorPictureEN}

    \end{textblock}

  \end{frame}
}



\newcommand{\GeometryThreeDTwoSpecialBeginningSlidesEN}{%
  \begin{frame}[standout]

    \begin{textblock}{11}(1,0.7)

      \begin{flushleft}

        \mdseries

        \footnotesize

        \color{jFrametitleFGColor}

        This content was created as part of a project co-financed by the
        European Union within the framework of the European Social Fund
        POWR.03.05.00-00-Z309/17-00.

      \end{flushleft}

    \end{textblock}





    \begin{textblock}{10}(0,2.2)

      \tikz \fill[color=jBackgroundStyleLight] (0,0) rectangle (12.8,-1.5);

    \end{textblock}


    \begin{textblock}{3.2}(0.7,2.45)

      \includegraphics[scale=0.3]{\FundingLogoColorPictureEN}

    \end{textblock}


    \begin{textblock}{2.5}(4.15,2.5)

      \includegraphics[scale=0.2]{\JULogoColorPictureEN}

    \end{textblock}


    \begin{textblock}{2.5}(6.35,2.4)

      \includegraphics[scale=0.1]{\ZintegrUJLogoColorPictureEN}

    \end{textblock}


    \begin{textblock}{4.2}(8.4,2.6)

      \includegraphics[scale=0.3]{\EUSocialFundLogoColorPictureEN}

    \end{textblock}

  \end{frame}





  \TitleSlideWithPicture
}



\newcommand{\GeometryThreeDSpecialEndingSlideEN}{%
  \begin{frame}[standout]

    \begin{textblock}{11}(1,0.7)

      \begin{flushleft}

        \mdseries

        \footnotesize

        \color{jFrametitleFGColor}

        This content was created as part of a project co-financed by the
        European Union within the framework of the European Social Fund
        POWR.03.05.00-00-Z309/17-00.

      \end{flushleft}

    \end{textblock}





    \begin{textblock}{10}(0,2.2)

      \tikz \fill[color=jBackgroundStyleLight] (0,0) rectangle (12.8,-1.5);

    \end{textblock}


    \begin{textblock}{3.2}(0.7,2.45)

      \includegraphics[scale=0.3]{\FundingLogoColorPictureEN}

    \end{textblock}


    \begin{textblock}{2.5}(4.15,2.5)

      \includegraphics[scale=0.2]{\JULogoColorPictureEN}

    \end{textblock}


    \begin{textblock}{2.5}(6.35,2.4)

      \includegraphics[scale=0.1]{\ZintegrUJLogoColorPictureEN}

    \end{textblock}


    \begin{textblock}{4.2}(8.4,2.6)

      \includegraphics[scale=0.3]{\EUSocialFundLogoColorPictureEN}

    \end{textblock}





    \begin{textblock}{11}(1,4)

      \begin{flushleft}

        \mdseries

        \footnotesize

        \RaggedRight

        \color{jFrametitleFGColor}

        The content of this lecture is made available under a~Creative
        Commons licence (\textsc{cc}), giving the author the credits
        (\textsc{by}) and putting an obligation to share on the same terms
        (\textsc{sa}). Figures and diagrams included in the lecture are
        authored by Paweł Węgrzyn et~al., and are available under the same
        license unless indicated otherwise.\\ The presentation uses the
        Beamer Jagiellonian theme based on Matthias Vogelgesang’s
        Metropolis theme, available under license \LaTeX{} Project
        Public License~1.3c at: \colorhref{https://github.com/matze/mtheme}
        {https://github.com/matze/mtheme}.

        Typographic design: Iwona Grabska-Gradzińska \\
        \LaTeX{} Typesetting: Kamil Ziemian \\
        Proofreading: Wojciech Palacz,
        Monika Stawicka \\
        3D Models: Dariusz Frymus, Kamil Nowakowski \\
        Figures and charts: Kamil Ziemian, Paweł Węgrzyn, Wojciech Palacz

      \end{flushleft}

    \end{textblock}

  \end{frame}
}



\newcommand{\GeometryThreeDTwoSpecialEndingSlidesEN}[1]{%
  \begin{frame}[standout]


    \begin{textblock}{11}(1,0.7)

      \begin{flushleft}

        \mdseries

        \footnotesize

        \color{jFrametitleFGColor}

        This content was created as part of a project co-financed by the
        European Union within the framework of the European Social Fund
        POWR.03.05.00-00-Z309/17-00.

      \end{flushleft}

    \end{textblock}





    \begin{textblock}{10}(0,2.2)

      \tikz \fill[color=jBackgroundStyleLight] (0,0) rectangle (12.8,-1.5);

    \end{textblock}


    \begin{textblock}{3.2}(0.7,2.45)

      \includegraphics[scale=0.3]{\FundingLogoColorPictureEN}

    \end{textblock}


    \begin{textblock}{2.5}(4.15,2.5)

      \includegraphics[scale=0.2]{\JULogoColorPictureEN}

    \end{textblock}


    \begin{textblock}{2.5}(6.35,2.4)

      \includegraphics[scale=0.1]{\ZintegrUJLogoColorPictureEN}

    \end{textblock}


    \begin{textblock}{4.2}(8.4,2.6)

      \includegraphics[scale=0.3]{\EUSocialFundLogoColorPictureEN}

    \end{textblock}





    \begin{textblock}{11}(1,4)

      \begin{flushleft}

        \mdseries

        \footnotesize

        \RaggedRight

        \color{jFrametitleFGColor}

        The content of this lecture is made available under a~Creative
        Commons licence (\textsc{cc}), giving the author the credits
        (\textsc{by}) and putting an obligation to share on the same terms
        (\textsc{sa}). Figures and diagrams included in the lecture are
        authored by Paweł Węgrzyn et~al., and are available under the same
        license unless indicated otherwise.\\ The presentation uses the
        Beamer Jagiellonian theme based on Matthias Vogelgesang’s
        Metropolis theme, available under license \LaTeX{} Project
        Public License~1.3c at: \colorhref{https://github.com/matze/mtheme}
        {https://github.com/matze/mtheme}.

        Typographic design: Iwona Grabska-Gradzińska \\
        \LaTeX{} Typesetting: Kamil Ziemian \\
        Proofreading: Wojciech Palacz,
        Monika Stawicka \\
        3D Models: Dariusz Frymus, Kamil Nowakowski \\
        Figures and charts: Kamil Ziemian, Paweł Węgrzyn, Wojciech Palacz

      \end{flushleft}

    \end{textblock}

  \end{frame}





  \begin{frame}[standout]

    \begingroup

    \color{jFrametitleFGColor}

    #1

    \endgroup

  \end{frame}
}



\newcommand{\GeometryThreeDSpecialEndingSlideVideoVerOneEN}{%
  \begin{frame}[standout]

    \begin{textblock}{11}(1,0.7)

      \begin{flushleft}

        \mdseries

        \footnotesize

        \color{jFrametitleFGColor}

        This content was created as part of a project co-financed by the
        European Union within the framework of the European Social Fund
        POWR.03.05.00-00-Z309/17-00.

      \end{flushleft}

    \end{textblock}





    \begin{textblock}{10}(0,2.2)

      \tikz \fill[color=jBackgroundStyleLight] (0,0) rectangle (12.8,-1.5);

    \end{textblock}


    \begin{textblock}{3.2}(0.7,2.45)

      \includegraphics[scale=0.3]{\FundingLogoColorPictureEN}

    \end{textblock}


    \begin{textblock}{2.5}(4.15,2.5)

      \includegraphics[scale=0.2]{\JULogoColorPictureEN}

    \end{textblock}


    \begin{textblock}{2.5}(6.35,2.4)

      \includegraphics[scale=0.1]{\ZintegrUJLogoColorPictureEN}

    \end{textblock}


    \begin{textblock}{4.2}(8.4,2.6)

      \includegraphics[scale=0.3]{\EUSocialFundLogoColorPictureEN}

    \end{textblock}





    \begin{textblock}{11}(1,4)

      \begin{flushleft}

        \mdseries

        \footnotesize

        \RaggedRight

        \color{jFrametitleFGColor}

        The content of this lecture is made available under a Creative
        Commons licence (\textsc{cc}), giving the author the credits
        (\textsc{by}) and putting an obligation to share on the same terms
        (\textsc{sa}). Figures and diagrams included in the lecture are
        authored by Paweł Węgrzyn et~al., and are available under the same
        license unless indicated otherwise.\\ The presentation uses the
        Beamer Jagiellonian theme based on Matthias Vogelgesang’s
        Metropolis theme, available under license \LaTeX{} Project
        Public License~1.3c at: \colorhref{https://github.com/matze/mtheme}
        {https://github.com/matze/mtheme}.

        Typographic design: Iwona Grabska-Gradzińska;
        \LaTeX{} Typesetting: Kamil Ziemian \\
        Proofreading: Wojciech Palacz,
        Monika Stawicka \\
        3D Models: Dariusz Frymus, Kamil Nowakowski \\
        Figures and charts: Kamil Ziemian, Paweł Węgrzyn, Wojciech
        Palacz \\
        Film editing: Agencja Filmowa Film \& Television Production~--
        Zbigniew Masklak

      \end{flushleft}

    \end{textblock}

  \end{frame}
}



\newcommand{\GeometryThreeDSpecialEndingSlideVideoVerTwoEN}{%
  \begin{frame}[standout]

    \begin{textblock}{11}(1,0.7)

      \begin{flushleft}

        \mdseries

        \footnotesize

        \color{jFrametitleFGColor}

        This content was created as part of a project co-financed by the
        European Union within the framework of the European Social Fund
        POWR.03.05.00-00-Z309/17-00.

      \end{flushleft}

    \end{textblock}





    \begin{textblock}{10}(0,2.2)

      \tikz \fill[color=jBackgroundStyleLight] (0,0) rectangle (12.8,-1.5);

    \end{textblock}


    \begin{textblock}{3.2}(0.7,2.45)

      \includegraphics[scale=0.3]{\FundingLogoColorPictureEN}

    \end{textblock}


    \begin{textblock}{2.5}(4.15,2.5)

      \includegraphics[scale=0.2]{\JULogoColorPictureEN}

    \end{textblock}


    \begin{textblock}{2.5}(6.35,2.4)

      \includegraphics[scale=0.1]{\ZintegrUJLogoColorPictureEN}

    \end{textblock}


    \begin{textblock}{4.2}(8.4,2.6)

      \includegraphics[scale=0.3]{\EUSocialFundLogoColorPictureEN}

    \end{textblock}





    \begin{textblock}{11}(1,4)

      \begin{flushleft}

        \mdseries

        \footnotesize

        \RaggedRight

        \color{jFrametitleFGColor}

        The content of this lecture is made available under a Creative
        Commons licence (\textsc{cc}), giving the author the credits
        (\textsc{by}) and putting an obligation to share on the same terms
        (\textsc{sa}). Figures and diagrams included in the lecture are
        authored by Paweł Węgrzyn et~al., and are available under the same
        license unless indicated otherwise.\\ The presentation uses the
        Beamer Jagiellonian theme based on Matthias Vogelgesang’s
        Metropolis theme, available under license \LaTeX{} Project
        Public License~1.3c at: \colorhref{https://github.com/matze/mtheme}
        {https://github.com/matze/mtheme}.

        Typographic design: Iwona Grabska-Gradzińska;
        \LaTeX{} Typesetting: Kamil Ziemian \\
        Proofreading: Wojciech Palacz,
        Monika Stawicka \\
        3D Models: Dariusz Frymus, Kamil Nowakowski \\
        Figures and charts: Kamil Ziemian, Paweł Węgrzyn, Wojciech
        Palacz \\
        Film editing: IMAVI -- Joanna Kozakiewicz, Krzysztof Magda, Nikodem
        Frodyma

      \end{flushleft}

    \end{textblock}

  \end{frame}
}



\newcommand{\GeometryThreeDSpecialEndingSlideVideoVerThreeEN}{%
  \begin{frame}[standout]

    \begin{textblock}{11}(1,0.7)

      \begin{flushleft}

        \mdseries

        \footnotesize

        \color{jFrametitleFGColor}

        This content was created as part of a project co-financed by the
        European Union within the framework of the European Social Fund
        POWR.03.05.00-00-Z309/17-00.

      \end{flushleft}

    \end{textblock}





    \begin{textblock}{10}(0,2.2)

      \tikz \fill[color=jBackgroundStyleLight] (0,0) rectangle (12.8,-1.5);

    \end{textblock}


    \begin{textblock}{3.2}(0.7,2.45)

      \includegraphics[scale=0.3]{\FundingLogoColorPictureEN}

    \end{textblock}


    \begin{textblock}{2.5}(4.15,2.5)

      \includegraphics[scale=0.2]{\JULogoColorPictureEN}

    \end{textblock}


    \begin{textblock}{2.5}(6.35,2.4)

      \includegraphics[scale=0.1]{\ZintegrUJLogoColorPictureEN}

    \end{textblock}


    \begin{textblock}{4.2}(8.4,2.6)

      \includegraphics[scale=0.3]{\EUSocialFundLogoColorPictureEN}

    \end{textblock}





    \begin{textblock}{11}(1,4)

      \begin{flushleft}

        \mdseries

        \footnotesize

        \RaggedRight

        \color{jFrametitleFGColor}

        The content of this lecture is made available under a Creative
        Commons licence (\textsc{cc}), giving the author the credits
        (\textsc{by}) and putting an obligation to share on the same terms
        (\textsc{sa}). Figures and diagrams included in the lecture are
        authored by Paweł Węgrzyn et~al., and are available under the same
        license unless indicated otherwise.\\ The presentation uses the
        Beamer Jagiellonian theme based on Matthias Vogelgesang’s
        Metropolis theme, available under license \LaTeX{} Project
        Public License~1.3c at: \colorhref{https://github.com/matze/mtheme}
        {https://github.com/matze/mtheme}.

        Typographic design: Iwona Grabska-Gradzińska;
        \LaTeX{} Typesetting: Kamil Ziemian \\
        Proofreading: Wojciech Palacz,
        Monika Stawicka \\
        3D Models: Dariusz Frymus, Kamil Nowakowski \\
        Figures and charts: Kamil Ziemian, Paweł Węgrzyn, Wojciech
        Palacz \\
        Film editing: Agencja Filmowa Film \& Television Production~--
        Zbigniew Masklak \\
        Film editing: IMAVI -- Joanna Kozakiewicz, Krzysztof Magda, Nikodem
        Frodyma

      \end{flushleft}

    \end{textblock}

  \end{frame}
}



\newcommand{\GeometryThreeDTwoSpecialEndingSlidesVideoVerOneEN}[1]{%
  \begin{frame}[standout]

    \begin{textblock}{11}(1,0.7)

      \begin{flushleft}

        \mdseries

        \footnotesize

        \color{jFrametitleFGColor}

        This content was created as part of a project co-financed by the
        European Union within the framework of the European Social Fund
        POWR.03.05.00-00-Z309/17-00.

      \end{flushleft}

    \end{textblock}





    \begin{textblock}{10}(0,2.2)

      \tikz \fill[color=jBackgroundStyleLight] (0,0) rectangle (12.8,-1.5);

    \end{textblock}


    \begin{textblock}{3.2}(0.7,2.45)

      \includegraphics[scale=0.3]{\FundingLogoColorPictureEN}

    \end{textblock}


    \begin{textblock}{2.5}(4.15,2.5)

      \includegraphics[scale=0.2]{\JULogoColorPictureEN}

    \end{textblock}


    \begin{textblock}{2.5}(6.35,2.4)

      \includegraphics[scale=0.1]{\ZintegrUJLogoColorPictureEN}

    \end{textblock}


    \begin{textblock}{4.2}(8.4,2.6)

      \includegraphics[scale=0.3]{\EUSocialFundLogoColorPictureEN}

    \end{textblock}





    \begin{textblock}{11}(1,4)

      \begin{flushleft}

        \mdseries

        \footnotesize

        \RaggedRight

        \color{jFrametitleFGColor}

        The content of this lecture is made available under a Creative
        Commons licence (\textsc{cc}), giving the author the credits
        (\textsc{by}) and putting an obligation to share on the same terms
        (\textsc{sa}). Figures and diagrams included in the lecture are
        authored by Paweł Węgrzyn et~al., and are available under the same
        license unless indicated otherwise.\\ The presentation uses the
        Beamer Jagiellonian theme based on Matthias Vogelgesang’s
        Metropolis theme, available under license \LaTeX{} Project
        Public License~1.3c at: \colorhref{https://github.com/matze/mtheme}
        {https://github.com/matze/mtheme}.

        Typographic design: Iwona Grabska-Gradzińska;
        \LaTeX{} Typesetting: Kamil Ziemian \\
        Proofreading: Wojciech Palacz,
        Monika Stawicka \\
        3D Models: Dariusz Frymus, Kamil Nowakowski \\
        Figures and charts: Kamil Ziemian, Paweł Węgrzyn,
        Wojciech Palacz \\
        Film editing: Agencja Filmowa Film \& Television Production~--
        Zbigniew Masklak

      \end{flushleft}

    \end{textblock}

  \end{frame}





  \begin{frame}[standout]


    \begingroup

    \color{jFrametitleFGColor}

    #1

    \endgroup

  \end{frame}
}



\newcommand{\GeometryThreeDTwoSpecialEndingSlidesVideoVerTwoEN}[1]{%
  \begin{frame}[standout]

    \begin{textblock}{11}(1,0.7)

      \begin{flushleft}

        \mdseries

        \footnotesize

        \color{jFrametitleFGColor}

        This content was created as part of a project co-financed by the
        European Union within the framework of the European Social Fund
        POWR.03.05.00-00-Z309/17-00.

      \end{flushleft}

    \end{textblock}





    \begin{textblock}{10}(0,2.2)

      \tikz \fill[color=jBackgroundStyleLight] (0,0) rectangle (12.8,-1.5);

    \end{textblock}


    \begin{textblock}{3.2}(0.7,2.45)

      \includegraphics[scale=0.3]{\FundingLogoColorPictureEN}

    \end{textblock}


    \begin{textblock}{2.5}(4.15,2.5)

      \includegraphics[scale=0.2]{\JULogoColorPictureEN}

    \end{textblock}


    \begin{textblock}{2.5}(6.35,2.4)

      \includegraphics[scale=0.1]{\ZintegrUJLogoColorPictureEN}

    \end{textblock}


    \begin{textblock}{4.2}(8.4,2.6)

      \includegraphics[scale=0.3]{\EUSocialFundLogoColorPictureEN}

    \end{textblock}





    \begin{textblock}{11}(1,4)

      \begin{flushleft}

        \mdseries

        \footnotesize

        \RaggedRight

        \color{jFrametitleFGColor}

        The content of this lecture is made available under a Creative
        Commons licence (\textsc{cc}), giving the author the credits
        (\textsc{by}) and putting an obligation to share on the same terms
        (\textsc{sa}). Figures and diagrams included in the lecture are
        authored by Paweł Węgrzyn et~al., and are available under the same
        license unless indicated otherwise.\\ The presentation uses the
        Beamer Jagiellonian theme based on Matthias Vogelgesang’s
        Metropolis theme, available under license \LaTeX{} Project
        Public License~1.3c at: \colorhref{https://github.com/matze/mtheme}
        {https://github.com/matze/mtheme}.

        Typographic design: Iwona Grabska-Gradzińska;
        \LaTeX{} Typesetting: Kamil Ziemian \\
        Proofreading: Wojciech Palacz,
        Monika Stawicka \\
        3D Models: Dariusz Frymus, Kamil Nowakowski \\
        Figures and charts: Kamil Ziemian, Paweł Węgrzyn,
        Wojciech Palacz \\
        Film editing: IMAVI -- Joanna Kozakiewicz, Krzysztof Magda, Nikodem
        Frodyma

      \end{flushleft}

    \end{textblock}

  \end{frame}





  \begin{frame}[standout]


    \begingroup

    \color{jFrametitleFGColor}

    #1

    \endgroup

  \end{frame}
}



\newcommand{\GeometryThreeDTwoSpecialEndingSlidesVideoVerThreeEN}[1]{%
  \begin{frame}[standout]

    \begin{textblock}{11}(1,0.7)

      \begin{flushleft}

        \mdseries

        \footnotesize

        \color{jFrametitleFGColor}

        This content was created as part of a project co-financed by the
        European Union within the framework of the European Social Fund
        POWR.03.05.00-00-Z309/17-00.

      \end{flushleft}

    \end{textblock}





    \begin{textblock}{10}(0,2.2)

      \tikz \fill[color=jBackgroundStyleLight] (0,0) rectangle (12.8,-1.5);

    \end{textblock}


    \begin{textblock}{3.2}(0.7,2.45)

      \includegraphics[scale=0.3]{\FundingLogoColorPictureEN}

    \end{textblock}


    \begin{textblock}{2.5}(4.15,2.5)

      \includegraphics[scale=0.2]{\JULogoColorPictureEN}

    \end{textblock}


    \begin{textblock}{2.5}(6.35,2.4)

      \includegraphics[scale=0.1]{\ZintegrUJLogoColorPictureEN}

    \end{textblock}


    \begin{textblock}{4.2}(8.4,2.6)

      \includegraphics[scale=0.3]{\EUSocialFundLogoColorPictureEN}

    \end{textblock}





    \begin{textblock}{11}(1,4)

      \begin{flushleft}

        \mdseries

        \footnotesize

        \RaggedRight

        \color{jFrametitleFGColor}

        The content of this lecture is made available under a Creative
        Commons licence (\textsc{cc}), giving the author the credits
        (\textsc{by}) and putting an obligation to share on the same terms
        (\textsc{sa}). Figures and diagrams included in the lecture are
        authored by Paweł Węgrzyn et~al., and are available under the same
        license unless indicated otherwise. \\ The presentation uses the
        Beamer Jagiellonian theme based on Matthias Vogelgesang’s
        Metropolis theme, available under license \LaTeX{} Project
        Public License~1.3c at: \colorhref{https://github.com/matze/mtheme}
        {https://github.com/matze/mtheme}.

        Typographic design: Iwona Grabska-Gradzińska;
        \LaTeX{} Typesetting: Kamil Ziemian \\
        Proofreading: Leszek Hadasz, Wojciech Palacz,
        Monika Stawicka \\
        3D Models: Dariusz Frymus, Kamil Nowakowski \\
        Figures and charts: Kamil Ziemian, Paweł Węgrzyn,
        Wojciech Palacz \\
        Film editing: Agencja Filmowa Film \& Television Production~--
        Zbigniew Masklak \\
        Film editing: IMAVI -- Joanna Kozakiewicz, Krzysztof Magda, Nikodem
        Frodyma


      \end{flushleft}

    \end{textblock}

  \end{frame}





  \begin{frame}[standout]


    \begingroup

    \color{jFrametitleFGColor}

    #1

    \endgroup

  \end{frame}
}











% ------------------------------------------------------------------------------------
% Importing packages, libraries and setting their configuration
% ------------------------------------------------------------------------------------





% ------------------------------------------------------
% BibLaTeX
% ------------------------------------------------------
% Package biblatex, with biber as its backend, allow us to handle
% bibliography entries that use Unicode symbols outside ASCII.
\usepackage[
language=polish,
backend=biber,
style=alphabetic,
url=false,
eprint=true,
]{biblatex}

\addbibresource{Cybernetyka-sztuczna-inteligencja-i-Dialogi-ETC-Bibliography.bib}





% ------------------------------------------------------
% Wonderful package PGF/TikZ
% ------------------------------------------------------

% Node and pics for drawing charts
% \usepackage{./Local-packages/PGF-TikZ-Chart-nodes-and-pics}

% Styles for arrows
% \usepackage{./Local-packages/PGF-TikZ-Arrows-styles}





% ------------------------------------------------------
% Local packages
% ------------------------------------------------------
% Special configuration for this particular presentation
\usepackage{./Local-packages/local-settings}

% Package containing various command useful for working with a text
\usepackage{./Local-packages/general-commands}

% Package containing commands and other code useful for working with
% mathematical text
% \usepackage{./Local-packages/math-commands}










% ------------------------------------------------------------------------------------------------------------------
\title{Cybernetyka, sztuczna inteligencja i~\textit{Dialogi}
  Stanisława Lema}

\author{Kamil Ziemian \\
  \email}


% \institute{Uniwersytet Jagielloński w~Krakowie}

\date[12 III~2025~r.]{12 III~2025~r.}
% ------------------------------------------------------------------------------------------------------------------









% ####################################################################
\begin{document}
% ####################################################################





% ######################################
% Number of chars: 47k+, 10k+,
% Text is adjusted to the left and words are broken at the end of the line.
\RaggedRight
% ######################################





% ######################################
\maketitle
% ######################################





% ######################################
\begin{frame}
  \frametitle{Plan prezentacji}


  \tableofcontents

\end{frame}
% ######################################










% ######################################
\section{Wstęp}
% ######################################



% ##################
\begin{frame}
  \frametitle{Skąd zainteresowanie tym problemem?}


  Aby uniknąć nieporozumień, muszę wyjaśnić, że~moje skromne badania nad
  wpływem cybernetyki na~twórczość Lema, nie wynikły z~fascynacji tym
  pisarzem. Wszystko zaczęło~się od rozmów z~dwoma miłośnikami tego pisarza,
  \colorhref{https://historia.uj.edu.pl/instytut/pracownicy/marcin-jarzabek}
  {dr.~Marcinem Jarząbkiem}
  i~\colorhref{https://incet.uj.edu.pl/maciej-prochnicki}{dr.~Maciejem
    Próchnickim}. To oni zwrócili moją uwagę na~cybernetyczny okres w~życiu
  tego twórcy. Ponieważ jak już wtedy wiedziałem, że~cybernetyka była nauką
  albo programem badawczym mocno zakorzenionym w~matematyce, osiągnięciach
  nauk przyrodniczych i~techniki, szybko stanął przede mną problem tego,
  jak bardzo środowisko badaczy literatury zgłębiło ten ezoteryczny z~ich
  punktu widzenia temat?

  Wedle mojej wiedzy na dzień dzisiejszy istnieje tylko jedna, duża pozycja
  analizując wpływ cybernetyki na twórczość Lema, jest to książka Pawła
  Majewskiego \textit{Między zwierzęciem a~maszyną. Utopia technologiczna
    Stanisława Lema} z~roku~$2007$. Niestety, nie udało mi~się jeszcze
  zdobyć jej i~dysponuję tylko informacjami z~drugiej ręki.

\end{frame}
% ##################





% ##################
\begin{frame}
  \frametitle{Stan na dzisiaj}


  Główne pytanie na które chcę obecnie znaleźć odpowiedź, jest takie.
  Czy Lem zaczerpnął pewne idee obecne w~jego dziełach literackich z~prac
  cybernetyków, a~jeśli tak to jakie? W~grudniu zeszłego roku stanął przede
  mną dodatkowy problem badawczy, mianowicie taki, że~publicystyka Lema
  wydaje~się bardzo zaniedbana przez badaczy, którzy skupili~się na jego
  działalności literackiej i~kilku większych dziełach eseistycznych. Nie
  wiem czy sam wkroczę na ten teren.

  Moje skromne badania nad Lemem ogromnie dużo zawdzięczają
  \colorhref{https://skos.agh.edu.pl/osoba/szymon-kukulak-10650.html}
  {dr.~Szymonowi Kukulakowi} z~Wydziału Humanistycznego \textsc{agh}, który
  gościł u~Państwa w~grudniu. Ostatnio jestem też w~kontakcie
  z~\colorhref{https://skos.agh.edu.pl/osoba/andrzej-bielecki-7555.html}
  {prof.~Andrzejem Bieleckim} z~Wydziału Elektrotechniki, Automatyki,
  Informatyki i~Inżynierii Biomedycznej \textsc{agh}. Prof. Bielecki określa
  siebie jako „informatyka, matematyka, cybernetyka i~filozofa nauki”, zna
  też dogłębnie twórczość Lema i~zawdzięczam mu ogromnie wiele cennych
  informacji.

\end{frame}
% ##################





% ##################
\begin{frame}
  \frametitle{Stan na dzisiaj}


  Mam nadzieję, że~w~tym roku~ja, prof.~Bielecki i~dr.~Kukulak nawiążemy
  współpracę i~wspólnie spróbujemy zbadać pewne związki między światem nauki
  i~twórczością Lema. Sam jestem dopiero tak w~jednej trzeciej
  zaplanowanych studiów nad powiązaniami Lema i~cybernetyki, więc w~tej
  chwili mogę sformułować pewne problemy, ale~nie potrafię znaleźć żadnej
  satysfakcjonujące odpowiedzi.

  Dodam, że~prof.~Bielecki obecnie zajmuje~się problemem sformułowania
  teorii informacji wychodzącej poza
  \colorhref{https://en.wikipedia.org/wiki/Information\_theory}{teorię
    Shannona} i~jak nam powiedział, gdy mieliśmy okazję go spotkać,
  inspiracje do nich zaczerpnął po części od~Lema. Mogą Państwo zwrócić~się
  do niego z~prośbą, czy nie chciałby tutaj opowiedzieć o~swoich przygodach
  lemologicznych.

\end{frame}
% ##################








% ######################################
\section{Czym jest cybernetyka?}
% ######################################


% ##################
\begin{frame}
  \frametitle{Czym jest cybernetyka?}


  \colorhref{https://en.wikipedia.org/wiki/Cybernetics}{Cybernetykę}
  najlepiej rozumieć przez analogię z~tym czym jest dzisiaj sztuczna
  inteligencja. Tak jak dziś \textsc{ai}, cybernetyka w~latach
  $1945\text{-}1970$ była zarówno dziedziną badań jak i~fenomenem
  kulturowym. Cybernetykę zamierzano zastosować do elektroniki, fizyki,
  chemii, biologi, socjologii, kierowania gospodarką, psychologii, szeroko
  pojętego mistycyzmu (dzieła
  \colorhref{https://pl.wikipedia.org/wiki/Jan_Tr\%C4\%85bka}{Jana
    Trąbki}), ~etc. Toczyły~się intensywne debaty na temat roli i~znaczenia
  cybernetyki, jak też zagrożeń jakie może ona nieść. W~sprzedaży było
  wiele dzieł dla niespecjalistów wprowadzających do~cybernetyki, takich jak
  \textit{Cybernetyka bez matematyki} A.W. Szylejko i~T.I.~Szylejko, czy
  \textit{Dialogi} Stanisława Lema.

  Gdy chodzi o~wpływ na kulturę to niestety na dzień dzisiejszy muszę~się
  ograniczyć do dwóch bezpośrednich przykładów. W~powieści \textit{Eden}
  Lema jeden z~bohaterów to Cybernetyk. Drugim przykład to sowiecki zbiór
  opowiadań \textit{Miłość i~cybernetyka} Susanny Gieorgijewskiej, wydany
  w~Polsce w~$1975$.

\end{frame}
% ##################





% ##################
\begin{frame}
  \frametitle{Co~się stało z~cybernetyką?}


  Porównanie to jest tym bardziej uzasadnione, że~do około $1965$ sztuczna
  inteligencja była jednym z~działów cybernetyki, dopiero później
  zaczęła być uważana, za część informatyki. Z~lekkim przymrużeniem oka,
  można więc powiedzieć, że~cybernetyka przeżywa dziś swoją trzecią młodość.

  Około $1970$~roku cybernetyka schodzi z~głównej sceny w~aurze
  niespełnionych obietnic i~dla większości ludzi zostaje zredukowana
  do~historycznej anegdoty. By zilustrować nadzieje i~rozczarowanie związane
  z~tą nauką, przytoczymy słowa Stanisława Lema z~\textit{Przedmowy}
  do~\textit{Dialogów}, napisanej w~grudniu $1971$.

  \textit{Pierwszy szkic [dołączony do tego wydania] to konfrontacja
    poglądów zawartych w~„Dialogach” z~realnym biegiem rzeczy w~ciągu
    szesnastu lat, jakie upłynęły między powstaniem książki [wydanej
    w~$1957$~r.] a~chwilą obecną. Konfrontacja ta ujawnia nie
    tylko moją naiwność; [\ldots]}
  Str.~$6$, \parencite{Lem-Dialogi-Vol-I-Pub-1996}.

\end{frame}
% ##################





% ##################
\begin{frame}
  \frametitle{Co~się stało z~cybernetyką?}


  \textit{[\ldots] nie tylko moją naiwność; byłem, jako autor „Dialogów”,
    wyrazicielem poglądu dość
    rozpowszechnionego w~kręgach entuzjastów cybernetyki w~pięćdziesiątych
    latach. Zestawienie opinii pochodzącej z~tych lat ze stanem obecnym jest
    ciekawym przyczynkiem do~historii nauki. Obrazuje mianowicie ową
    ekstrapolacyjną prostolinijność jaką wzbudza w~nauce bodaj każdy jej
    przewrót; perspektywa dalszego postępu wiedzy rysują~się wtedy
    współczesnym tak prosto, jak gdyby ów zawiły, pełen kluczeń i~ślepych
    zaułków ruch poznawania, który doprowadził do kolejnej rewolucji
    w~nauce miał ustać i~ustąpić drogi lawinowemu pomnażaniu wiedzy~--
    właściwie już bez odwrotów i~przeszkód. Regularnie też dochodzi potem
    do rozmijania~się nazbyt optymistycznych nadziei z~rzeczywistością,
    które to zjawisko i~w~odniesieniu do cybernetyki~się sprawdziło. Warto
    dodać, że~reakcją na nie bywa później pesymizm poznawczy, diametralnie
    oponujący wcześniejszy optymizm, jak również, że~taka reakcja przeważnie
    bywa nieporozumieniem: jakkolwiek bowiem cybernetyka nie ziściła
    istotnie tego, [\ldots]}
  Str.~$6$, \parencite{Lem-Dialogi-Vol-I-Pub-1996}.

\end{frame}
% ##################





% ##################
\begin{frame}
  \frametitle{Co~się stało z~cybernetyką?}


  \textit{[\ldots] nie ziściła istotnie tego, czego~się po niej najintensywniej
    spodziewano~-- nie stała~się zwłaszcza lekarstwem leczącym naukę
    ze~schorzenia specjalistycznego (a~miała nim zostać jako inter- czy
    wręcz superdyscyplinarna wiedza, unifikując tak przyrodoznawstwo, jak
    humanistykę)~-- urzeczywistniła to, czego nikt~się po niej nie
    spodziewał. Maszyny cyfrowe nie stały~się co prawda równorzędnymi
    osobowymi partnerami człowieka, ale za to okazały~się niezastąpionym
    już dziś narzędziem w~zarządzaniu światową gospodarką; teoria
    informacji nie okazał~się co prawda nowym kamieniem filozoficznym,
    ale~wtargnęła nawet tam gdzie jej nie oczekiwano, np.~do fizyki
    teoretycznej; przykładów takiego rozmijania~się oczekiwań i~spełnień
    można by wyliczyć wiele.}
  Str.~$6\text{--}7$, \parencite{Lem-Dialogi-Vol-I-Pub-1996}.

  Jak już wspomniano wcześniej, pod postacią sztucznej inteligencji,
  cybernetyka jest z~nami do dziś, czy jednak przetrwała jako osobna nauka?
  Wolałbym pozostawić odpowiedź na to pytanie prof.~Bieleckiemu,
  zob.~przykładowo jego artykuł
  \parencite{Bielecki-Cybernetyka-Slowniki-spoleczne-Vol-XIII-Ver-2025}.

\end{frame}
% ##################










% ######################################
\section{Norbert Wiener, wybitny naukowiec i~celebryta}
% ######################################



% ##################
\begin{frame}
  \frametitle{Norbert Wiener, ojciec cybernetyki}


  \begin{figure}

    \centering


    \includegraphics[scale=0.23]
    {./Presentation-pictures/Norbert-Wiener-01.jpg}


    \caption{\colorhref{https://en.wikipedia.org/wiki/Norbert\_Wiener}
      {Norbert Wiener} ($1894\text{--}1964$), amerykański filozof,
      informatyk i~matematyk, ojciec cybernetyki.}


    \label{fig:Wiener-01}

  \end{figure}

\end{frame}
% ##################





% ##################
\begin{frame}
  \frametitle{Norbert Wiener, ojciec cybernetyki}


  Norbert Wiener dokonał wielu wartościowych odkryć w~kilku dziedzinach.
  W~tym miejscu warto wspomnieć, że~sformułował on sposób opisu ruchów
  Browna, który dziś nazywamy
  \colorhref{https://en.wikipedia.org/wiki/Wiener\_process}{procesem
    Wienera}, który to process indukuje odpowiednią miarę Wienera
  na~przestrzeni funkcji ciągłych. Już sam ten fakt jest godnym odnotowania
  osiągnięciem.

  W~roku $1948$ publikuje książkę
  \colorhref{https://en.wikipedia.org/wiki/Cybernetics\_Or\_Control\_and\_Communication\_in\_the\_Animal\_and\_the\_Machine}
  {\textit{Cybernetics: Or~control and~Communication in the~Animal and
      the~Machine}}, tekst założycielski cybernetyki. Książka ta w~trudny
  do~zrozumienia dla mnie sposób stała~się fenomenem kulturowym, który
  najlepiej chyba oddają słowa z~jej recenzji, która ukazał w~piśmie
  \textit{Business Week} z~$1949$ roku: \textit{Pod jednym względem książka
    Wienera jest jak Raport Kinseya: reakcja społeczeństwa na nią jest tak
    samo ważna, jak jej treść.}, tłum. wł., cyt. za~str.~$408$
  \parencite{Hamilton-The-Charismatic-Cultural-Life-of-ETC-Pub-2017}. Sam
  zaś Wiener uzyskał dzięki niej status celebryty, który to temat jest
  szerzej omawia Sheryl N.~Hamilton w~cytowanym wyżej artykule.

\end{frame}
% ##################





% ##################
\begin{frame}
  \frametitle{Narodziny cybernetyki}


  \textit{Cybernetics\ldots} opublikowana zostaje w~przekładzie polskim
  najpóźniej w~$1971$~roku. W~$1950$ Wiener publikuje mniejszą, bardzie
  popularną książkę
  \colorhref{https://en.wikipedia.org/wiki/The\_Human\_Use\_of\_Human\_Beings}
  {\textit{The Human Use of Human Beings}}, w~Polsce wydana jako
  \textit{Cybernetyka i~społeczeństwo} w~$1960$~roku.

  Tak jak komputery, cybernetyka wyrosła w~dużej mierze z~rozwoju naukowego
  i~technologicznego, który dokonał~się podczas II Wojny Światowej. Sam
  Wiener wspomina rolę jaką jego praca nad zautomatyzowanymi działami
  przeciwlotniczymi (ang.~\textit{anti-aircraft predictors}) odegrała
  w~powstaniu cybernetyki (zob. str.~xii,
  \parencite{Wiener-Cybernetics-Second-edition-Pub-2016}).

  Problem przed jaki stanęli był następujący. Ze względu na wysokość lotu
  samolotów, ich szybkość i~prędkość lotu pocisków przeciwlotniczych, jeśli
  byśmy wycelowali działem przeciwlotniczym idealnie w~samolot, to ponieważ
  od~momentu wystrzelenia pocisku do~dotarcia do celu minie kilka sekund,
  więc samolot najpewniej już opuścił miejsce, w~które wymierzyliśmy.

\end{frame}
% ##################





% ##################
\begin{frame}
  \frametitle{Narodziny cybernetyki}


  Jak dobrze wiedzą ludzie grający w~gry wideo, w~takiej sytuacji nie
  należy celować w~samolot, lecz w~przewidywane przez nas miejsce jego
  położenia po upływie kilku sekund. Urządzenia przy których budowie
  pracował Wiener, miało właśnie pomóc artylerzyście znaleźć to przyszłe
  położenie samolotu.

  Z~przedmowy do drugiego wydania, które Wiener napisał w~marcu $1961$ roku
  widać, że~miał on już do dyspozycji wszystkie podstawowe pojęcia tego,
  co my dziś nazywamy \textit{uczeniem maszynowym} (ang.~\textit{machine
    learning}), a~najpóźniej w~roku~$1957$ jemu i~jego współpracownikom
  była znane idee, która pozwoliłaby działu przeciwlotniczemu uczyć~się
  strzelać do samolotów na podstawie dostarczonych danych. Wiener stwierdza
  wręcz jasno, że~idea uczącej się maszyny, jest tak stara jak sama
  cybernetyka (zob. str.~xii,
  \parencite{Wiener-Cybernetics-Second-edition-Pub-2016}).

  Niemniej na dzisiaj, większość tych problemów wymaga dalszych badań
  z~mojej strony.

\end{frame}
% ##################





% ##################
\begin{frame}
  \frametitle{Narodziny cybernetyki}


  W~przedmowie tej można też znaleźć też dowody na to, że~elektronika
  wywarła bardzo silny wpływ na cybernetykę, co będzie przejawia~się
  później w~stosach diagramów wypełniających książki do~cybernetyki,
  a~które przypominają schematy obwodów elektrycznych.

  Warto też przytoczyć z~niej słowa Wienera, które są wypowiedziane
  w~dobrze nam znanym alarmistycznym tonie. \\
  \textit{Automaty (\textit{The~automata}) które pierwsze wydanie tylko
    przewidywało, dziś są rzeczywistością, a~związane z~nimi zagrożenia dla
    społeczeństwa (\textit{social dangers}) przed którymi ostrzegałem, są
    aż~nadto widoczne.} \\
  Tłum. swobodne, str.~vii,
  \parencite{Wiener-Cybernetics-Second-edition-Pub-2016}.

\end{frame}
% ##################




% ##################
\begin{frame}
  \frametitle{Narodziny cybernetyki}


  Prawem kontrapunktu, warto w~tym miejscu zauważyć, że~choć Wiener był
  głównie matematykiem, to w~pierwszym pokoleniu informatyków było wielu
  fizjologów, psychiatrów i~neurofizjologów. Już w~pierwszej połowie lat
  $40$-tych XX wieku Wiener współpracował
  z~\colorhref{https://en.wikipedia.org/wiki/Arturo_Rosenblueth}
  {Arturo Rosenbluethem Stearnsem} ($1900\text{--}1970$), meksykańskim
  lekarzem i~fizjologiem, który później wniósł duży wkład w~rozwój
  cybernetyki i~któremu Wiener zadedykował swoją książkę
  \textit{Cybernetics\ldots} Ten fakt prawie na~pewno miało duży wpływ na
  rozważania dotyczące życia i~pracy mózgu, które podjął Wiener i~jego
  następcy.

  Duża obecność fizjologów i~psychiatrów, była szczególnie mocna wśród
  cybernetyków brytyjskich. Za przykład niech posłuży
  \colorhref{https://en.wikipedia.org/wiki/W._Ross_Ashby}{William
    Ross Ashby} ($1903\text{-}1972$), który w~roku $1930$ rozpoczął badania
  kliniczne jako psychiatra i~którego chyba spotkał Ijon Tichy, gdy Ashby
  z~niewiadomych powodów ukrywał~się pod nazwiskiem
  prof.~Corcorana. Niemniej to na razie tylko przypuszczenia.

\end{frame}
% ##################





% ##################
\begin{frame}
  \frametitle{Homeostaty Ashby’ego}


  \begin{figure}

    \centering


    \includegraphics[scale=1.1]
    {./Presentation-pictures/W\_Ross\_Ashbys\_1948\_Homeostat.jpeg}


    \caption{\colorhref{https://en.wikipedia.org/wiki/Homeostat}
      {Homeostat} skonstruowane przez W.~Rossa Ashby’ego w~$1948$~roku.}


    \label{fig:Homeostat-01}

  \end{figure}

\end{frame}
% ##################





% ##################
\begin{frame}
  \frametitle{\textit{Cybernetics, or Control and
      Communication\ldots}}


  \textit{Cybernetics\ldots} Wienera jest książką osobliwą i~jak sam przyznał,
  napisaną w~niesprzyjających warunkach, co poskutkowało dużo liczbą błędów,
  również bardzo poważnych (zob. str.~xii,
  \parencite{Wiener-Cybernetics-Second-edition-Pub-2016}). Nie udało mi~się
  zdobyć wydania polskiego, korzystam więc z~angielskiej wersji, będącej
  wznowienia wydania drugiego z~$1961$~roku, w~którym książka ta ma około
  $240$~stron.

  Wstępny obraz tej książki można uzyskać poprzez zestawienie tytułów jej
  rozdziałów, ich długości i~liczby wyróżnionych w~tekście wzorów
  matematycznych.

  \textit{Preface to second edition}, $10$ str., $7$~wz. \\
  \textit{Introduction}, $26$~str., $0$~wz. \\
  \textit{Newtonian and Bergsonian Time}, $14$~str., $0$~wz. \\
  \textit{Groups and Statistical Mechanics}, $16$~str., $26$~wz. \\
  \textit{Time Series, Information and Comunications}, $40$~str.,
  $136$~wz. \\
  \textit{Feedback and Oscillation}, $24$~str., $58$~wz. \\

\end{frame}
% ##################





% ##################
\begin{frame}
  \frametitle{\textit{Cybernetics, or Control and
      Communication\ldots}}


  \textit{Computing Machines and Nervous Systems}, $16$~str., $10$~wz. \\
  \textit{Gestalt and Universals}, $10$~str., $1$~wz. \\
  \textit{Cybernetics and Psychopatology}, $10$~str., $0$~wz. \\
  \textit{Information, Laguage and Society}, $10$~str., $0$~wz.

  Dwa następne rozdziały zostały dodane w~wydaniu z~$1961$~roku. \\
  \textit{On~Learning and Self-Reproducing Machines}, $12$~str., $0$~wz. \\
  \textit{Brain Waves and Self-Organizing Systems}, $24$~str., $36$~wz.

  Lektura rozdziałów z~większą liczbą wzorów, wymaga dość dobrej znajomości
  teorii całki, do~której jak wiemy Wiener wniósł ważny wkład. Jednak to
  temat na zupełnie inne spotkanie.

  Rozdział \textit{On~Learning and Self-Reproducing Machines} wygląda
  szczególnie ciekawie w~kontekście tego, że~w~$1964$ Lem publikuje
  \textit{Niezwyciężonego}, więc może porównują te dwa teksty dojdziemy
  do jakiś ciekawszych wniosków? Czas pokaże.

\end{frame}
% ##################





% ##################
\begin{frame}
  \frametitle{Czym jest mózg?}


  Kończąc już przegląd cybernetyki, chciałbym postawić problem, który mnie
  intryguje. Chodzi o~ideę zaproponowaną w~okolicach $1960$~roku, wciąż nie
  znam dokładnej daty, przez dwóch cybernetyków,
  \colorhref{https://en.wikipedia.org/wiki/Stafford\_Beer}{Stafforda Beera}
  i~\colorhref{https://en.wikipedia.org/wiki/Gordon\_Pask}{Gordona Paska}.
  Przyjęli oni, że~obok wcześniejszej tezy, że~mózg jest obiektem
  posiadającym zdolność adaptacji do~otoczenia, należy uznać też tezę
  odwrotną: każdy układ który adaptuje~się do otoczenia jest w~jakimś
  sensie mózgiem. Beer na podstawie tego stwierdził, że~w~takim właśnie
  sensie staw wodny jest mózgiem
  \parencite{Pickering-Cybernetics-in-Britain-Ver-2022}. Natomiast
  w~$1961$~roku opublikowany zostaje \textit{Solaris}.

  Lema bardzo zajmował problem przypadków, a~ja chciałbym wiedzieć, czy to
  jest przypadek, czy nie? A~może zwyczajni widzę tutaj podobieństwa,
  których w~rzeczywistości nie ma?

\end{frame}
% ##################










% ######################################
\section{\textit{Dialogi} Stanisława Lema}
% ######################################



% ##################
\begin{frame}
  \frametitle{Czym są \textit{Dialogi}?}


  Zacznijmy od przytoczenia pewnych ustaleń Pawła Majewskiego z~jego
  \textit{Między zwierzęciem a~maszyną\ldots} Główny wzorem dla Lema przy
  tworzeniu swoich \textit{Dialogów}, wydaje~się być dzieło \textit{Trzy
    dialogi między Hylasem i~Filonousem}, wydane przez prominentnego
  brytyjskiego filozofa
  \colorhref{https://en.wikipedia.org/wiki/George_Berkeley}
  {George’a Berkeleya} w~$1713$~roku. Berkeley wyłożył w~nim swoją
  filozofię, którą nazywał \textbf{immaterlializmem}. Lem zapożyczył od
  Berkeleya zarówno formę dialogu filozoficznego, imiona bohaterów, jak
  i~podstawowy podział na „dobrego” Filonousa i~„złego” Hylasa. Ich imiona
  są znaczące, gdyż „Filonous” oznacza „umysłowy”, „intelektualny”,
  zaś~„Hylas” tłumaczy~się jako~„cielesny” lub „materialny”.

  Dialogów jest łącznie osiem. Pierwszy z~nich powstał w~latach
  $1948\text{-}1950$, pozostałe siedem w~latach $1954\text{-}1956$. Jak sam
  to określił później Lem, poprzez ich stworzenie stał~się publicznym
  wyrazicielem przekonań entuzjastów cybernetyki tamtego czasu.

\end{frame}
% ##################





% ##################
\begin{frame}
  \frametitle{Czym są \textit{Dialogi}?}


  Wydaje~się, że~w~momencie publikacji w~roku~$1957$, \textit{Dialogi} miały
  być książką wprowadzającą osoby zainteresowane filozofią w~cybernetykę,
  czyli funkcjonowały jako pewien typ literatury popularnonaukowej.
  Nie uświadczysz w~nich ani jednego wzoru, czego nie można powiedzieć
  o~książce Wienera.

  Jako że~moje skromne badania dotyczą wpływu cybernetyki na twórczość
  literacką Lema, \textit{Dialogi} w~których przedstawia on zainteresowanemu
  czytelnikowi tą dziedzinę, były naturalnym punktem początkowym,
  pozwalający ustalić, jaka była jego znajomość tej nauki
  około~$1957$~roku. Przy czym w~dalszej części wystąpienie będę pomijał
  pewne refleksje, jak te odnośnie warstwy literackiej tego dzieła, które
  mi~się nasunęły w~trakcie lektury.

\end{frame}
% ##################





% ##################
\begin{frame}
  \frametitle{\textit{Dialogi} i~termin „sztuczna
    inteligencja”}


  Jednym z~podstawowych pojęć u~Lem jest przypadek, więc wspomnijmy o~jednym
  z~nich. Wedle mojej wiedzy, termin \textit{sztuczna inteligencja} został
  ukuty $1955$~roku przez czterech gigantów informatyki:
  \colorhref{https://en.wikipedia.org/wiki/John_McCarthy_(computer_scientist)}
  {Johna McCarthy’ego},
  \colorhref{https://pl.wikipedia.org/wiki/Marvin\_Minsky}
  {Marvina Minsky’ego},
  \colorhref{https://en.wikipedia.org/wiki/Nathaniel\_Rochester\_(computer\_scientist)}{Nathaniela Rochestera}
  i~\colorhref{https://en.wikipedia.org/wiki/Claude\_Shannon}
  {Claude’a Shannona}. Termin ten wprowadzili na potrzeby projektu letnich
  warsztatów, które odbył~się w~roku~$1956$ pod nazwą
  \colorhref{https://en.wikipedia.org/wiki/Dartmouth_workshop}
  {\textit{Dartmouth Summer Research Project on Artificial Intelligence}}
  i~są uważany za~jedno z~najważniejszych wydarzeń w~historii tej
  dziedziny. Wedle pewnych świadectw, w~nazwie tej konferencji nie użyto
  słowa „cybernetyka”, ze względu na personalne animozje między niektórymi
  z~organizatorów a~Norbertem Wienerem. Mało profesjonalne, ale jakże
  przecież normalne i~ludzkie.

  Niezależnie od~tego, warsztaty te odbywają~się w~tym samym roku, w~którym
  Lem kończy \textit{Dialogi}, więc może gdyby ukończył je później,
  sam termin „sztuczna inteligencja” byłby obecny w~ich tekście.

\end{frame}
% ##################





% ##################
\begin{frame}
  \frametitle{Treść \textit{Dialogów}}


  \textit{Dialogi} zaczynają~się, gdy Filonous napotyka w~„pięknym parku”
  samotnie rozmyślającego Hylasa. Hylas, w~tym momencie zadeklarowany
  materialista, wyjaśnia mu, że~gdy technologia osiągnie odpowiedni
  poziom rozwoju, to będzie mógł zostać „wskrzeszony” przez idealne
  skopiowanie całego jego ciała. Temat ten dyskutują do mniej więcej
  jednej trzeciej dialogu numer iii, gdzie Hylas uznaje, że~nie potrafi
  sprowadzić świadomości człowieka do~konfiguracji materialnych elementów
  jego ciała. Moje osobiste odczucie jest takie, że~ich rozważania na temat
  relacji świadomości oraz~materii zawierają zbyt dużo luk i~zbyt szybkich
  przeskoków, by można było uważać je za filozoficznie zadowalające, ale
  musimy to odłożyć na bok.

  W~czasie tych rozmów mocno dyskutują problem tego, czy gdyby istniały
  dwie idealne kopie jednego człowieka, to która byłaby prawdziwa
  i~czy człowiek może istnieć jako „zwielokrotniony”. Komediową wersję
  tego problemu znajdujemy w~\textit{Podróży siódmej} Ijona Tichego.

\end{frame}
% ##################





% ##################
\begin{frame}
  \frametitle{Uczenie maszynowe i~\textit{Terminus}}


  Nie wiem jak u~Państwa, ale w~moim odczuciu \textit{Podróż siódma} jest
  dość lekkim jak na Lema utworem, którego morałem jest to, że pracę należy
  zostawić dzieciom. I~że~żadne zebranie~się nie odbędzie bez komisji
  skrutacyjnej i~rewizyjnej.

  Wracając do \textit{Dialogów}, gdy Hylas wyczerpie wszystkie swoje
  pomysły, Filonous zaczyna wykładać swój pogląd i~by to zrobić
  opowiada o~świecie, w~którym mózgi mogą wymieniać~się zawartością. Motyw
  wymiany pamięci, a~tym samym osobowości, pojawi~się w~prozie Lema nie raz.

  Jako hipotezę roboczą zaproponuję spojrzenie na opowiadanie
  \textit{Terminus}, jako historię o~wymianie pamięci między ludźmi,
  a~tytułowym robotem, za pomocą procedury uczenia maszynowego. Bardzo
  jestem ciekawy, czy według Państwa takie podejście ma w~ogóle sens?

\end{frame}
% ##################





% ##################
\begin{frame}
  \frametitle{Informacja}


  Konsekwencją wywodu o~wymianie pamięci są następujące słowa wypowiadane
  przez Filonous, które zasadniczo są parafrazą tego, co można znaleźć
  w~\textit{Cybernetics\ldots} Wienera. \\
  \textit{Zastanówmy~się mianowicie, co wymieniały mózgi w~naszym obrazowym
    przykładzie, cóż to za rzecz była, za zjawisko, któreśmy nazwali
    skrótowo „ładunkiem pamięci”? Był to, nieprawdaż, całokształt
    strukturalnych zmian, nabytych przez ten mózg podczas jego istnienia,
    to znaczy~-- był to zasób pewnej informacji. Kiedy to słowo padło,
    znaleźliśmy~się w~kardynalnym miejscu naszych rozważań. W~samej rzeczy,
    problem informacji, jej istoty, jej powstania, gromadzenia,
    przechowywania i~użytkowania stanowi właściwą treść cybernetyki
    i~zarazem klucz do zagadek, które przedstawiają systemy typu naszego
    mózgu.} \\
  \textsc{hylas}: \textit{Cóż tak osobliwego przedstawia informacja?} \\
  \textsc{filonous}: \textit{Jest ona czymś wielce osobliwym, mój drogi,
    albowiem nie jest ani materią, ani~energią, choć stanowi zjawisko
    całkiem realne.} Str.~$55$, \parencite{Lem-Dialogi-Vol-I-Pub-1996}.

\end{frame}
% ##################





% ##################
\begin{frame}
  \frametitle{Informacja}


  W~powyższym fragmencie Filonous w~zasadzie zdefiniował cybernetykę
  w~ujęciu Wienera. Również uznanie informacji za równorzędnego partnera
  materii i~energii jest zaczerpnięte od ojca cybernetyki.

  Czym jednak jest informacja? Tutaj Lem napotyka na te same problemy,
  na~które napotkali później cybernetycy, czyli problem z~definicją pojęcia
  informacji. Mówiąc pół żartem, pół serio, wykorzystywana przez nich
  definicja informacji, zawiera za mało informacji o~tym czym jest
  informacja ;).

  Ponownie powołam~się na prof.~Bieleckiego, który stwierdził, iż~dobrze
  znana teoria informacji, jest tak naprawdę teorią \alert{przesyłania}
  informacji, o~samej informacji nie mówi ona zbyt wiele. On sam w~swoich
  pracach na temat
  \colorhref{https://www.youtube.com/watch?v=jaAADDFG8oA}{informacji
    strukturalnej} próbuje wypełnić w~jakimś stopniu tą lukę. Jak zaraz
  zobaczymy, również Lem dostrzegł różne niedomagania stosowanego
  w~cybernetyce pojęcia informacji.

\end{frame}
% ##################





% ##################
\begin{frame}
  \frametitle{Informacja}


  Filonous w~następujący sposób wyjaśnia koncepcję informacji. \\
  \textit{Doniosłość prac cybernetyków na tym się właśnie zasadza,
    że~wykryli odpowiedź na to pytanie [Czym jest informacja?]. Informacja
    jest dzieckiem termodynamiki, postawionej, mówiąc obrazowo, na głowie,
    gdyż stanowi odwrotność entropii.}
  Str.~$56$, \parencite{Lem-Dialogi-Vol-I-Pub-1996}. \\
  \textit{Otóż, wracają teraz do cybernetyki~-- informacja jest odwrotnością
    entropii. Gdy tamta jest miarą bezładu~-- ta mierzy ład.}
  Str.~$57$, \parencite{Lem-Dialogi-Vol-I-Pub-1996}.

  Zestawmy to z~fragmentem wstępu do~\textit{Cybernetics\ldots} Wienera. \\
  \textit{Pojęcie ilości informacji łączy~się bardzo naturalnie
    z~klasycznym pojęciem mechaniki statystycznej, mianowicie z~pojęciem
    \textbf{entropii}. Tak jak informacja zawarta w~systemie jest miarą jego
    ładu, tak entropia systemu jest miarą jego bezładu. Jedna jest po
    prostu odwrotnością drugiej (the~negative~of the~other).}
  Tłum. swobodne, str.~$11\text{--}12$,
  \parencite{Wiener-Cybernetics-Second-edition-Pub-2016}.

\end{frame}
% ##################





% ##################
\begin{frame}
  \frametitle{Informacja}


  Tego typu sposób myślenia o~informacji, zakorzeniło~się mocniej
  w~naukach technicznych. Za przykład niech posłużą cytaty z~książki
  Jerzego Seidlera
  \textit{Nauka o~informacji. Tom~I: Podstawy, modele źródeł i~wstępne
    przetwarzanie informacji}, opublikowanej przez Wydawnictwo Naukowe
  i~Techniczne w~$1983$~roku
  \parencite{Seidler-Nauka-o-informacji-Vol-I-Pub-1983}.

  \textit{Podkreślono z~jednej strony uniwersalny charakter pojęć
    informacji i~sygnału, powiązania dwustronne między nauką o~informacji
    a~teorią systemów, z~drugiej zaś strony omówiono konkretne przykłady
    z~telekomunikacji, miernictwa i~automatyki.} Str.~$4$,
  \parencite{Seidler-Nauka-o-informacji-Vol-I-Pub-1983}.

  \textit{Pojęciem „informacji” posługujemy~się często. Pojęcie to,
    podobnie jak na przykład pojęcie „materia” lub „energia”, ma charakter
    pojęcia pierwotnego i~ścisłe zdefiniowanie go za pomocą prostszych nie
    jest możliwe. Pozostaje więc jedynie wyjaśnienie sensu tego pojęcia,
    odpowiadającego jego intuicyjnemu rozumieniu.} Str.~$21$,
  \parencite{Seidler-Nauka-o-informacji-Vol-I-Pub-1983}.

\end{frame}
% ##################





% % ##################
% \begin{frame}
%   \frametitle{Informacja}






% \end{frame}
% % ##################





% ##################
\begin{frame}
  \frametitle{\textit{Cyberiada}}


  W~tym kontekście Filonous przytacza też drugą zasadę termodynamiki,
  zob.~str.~$57$, \parencite{Lem-Dialogi-Vol-I-Pub-1996}. Literacki
  dziełem Lema wytykającym braki takiej koncepcji informacji jest
  \textit{Wyprawa szósta, czyli jak Trurl i~Klapaucjusz demona drugiego
    rodzaju stworzyli, aby zbójcę Gębona pokonać}, będą częścią,
  nomen omen, \textit{Cyberiady}. Bohaterowie pokonują w~niej Gębona,
  wydobywając z~szumu wypełniającego przestrzeń kosmiczną informacje
  o~średniej ilości piór łabędzia niemego na metr kwadratowy, liczbie
  ziaren w~główce maku, liczbie ziaren piasku jaka dostaje~się do buta
  na~plaży, etc., pod których zalewem przepada zbójca.


  Pisząc to seminarium przyszła mi do głowy hipoteza, że~większość tekstów
  z~\textit{Cyberiady} jest formą rozliczenia~się Lema z~niespełnionymi
  obietnicami cybernetyki. Jeśli jest~się świadomym wielkiej roli
  rachunku prawdopodobieństwa w~tej nauce, można w~ten sposób odczytać
  \textit{Wyprawę trzecią, czyli smoki prawdopodobieństwa}. W~tym momencie
  nie wiem, czy~to jest dobry pomysł, ale wygląda obiecująco.

\end{frame}
% ##################





% ##################
\begin{frame}
  \frametitle{Próg komplikacji minimalnej}


  Ostatnią ważną z~naszego punktu widzenia koncepcją wprowadzoną w~dialogu
  iii, jest pojęcie „progu komplikacji minimalnej”. Według Filonousa, jeśli
  dany układ będzie tak złożony, że~przekroczy próg komplikacji minimalnej,
  wówczas jest w~stanie stworzyć urządzenie tak samo złożone jako on sam.
  Jak mówi \\
  \textit{[\ldots] próg komplikacji minimalnej wyznacza ścisłą, fizykalnie
    dającą~się mierzyć granicę między światem mechanizmów klasycznych
    (maszyn) a~światem \textbf{organizmów}. Zauważ, proszę, iż~nie powiadam
    „a~światem \textbf{żywych} organizmów”. „Życie” jest tu pojęciem
    węższym, a~„organizacja”~-- pojęciem nadrzędnym, szerszym.}
  Str.~$68\text{--}69$, \parencite{Lem-Dialogi-Vol-I-Pub-1996}.

  Te cytaty wskazują na pewną ważną cechę myślenia cybernetycznego, której
  ja nie dostrzegam w~naszym rozumieniu sztucznej inteligencji. Mianowicie,
  w~cybernetyce rozmył~się podział między zwierzęciem, a~maszyną, więc
  w~prowadzonych wtedy rozważaniach, nie pytano tyle o~sztuczną
  inteligencję, co o~\alert{sztuczne życie}.

\end{frame}
% ##################





% ##################
\begin{frame}
  \frametitle{Sztuczna inteligencja vs sztuczne życie}


  Organizm zwierzęcia i~maszyna były z~punktu widzenia cybernetyki po
  prostu dwoma układami, które bazowały na tych samych zasadach przesyłu
  i~przetwarzania informacji, energii i~materii. W~skutek tego nie dzieli
  je fundamentalna, ontologiczna różnica. Takimi samymi układami co
  zwierzęta i~maszyny są zresztą również stawy wodne, uniwersytety czy
  firmy.

  Nie wydaje mi~się, żebyśmy dzisiaj zadawali sobie pytanie, czy komputer
  jest tym samym typem obiektu, jak ludzkie ciało, czy to w~filozoficznych
  rozważaniach, czy też w~filmach fabularnych albo grach video. Raczej
  przyjmujemy domyślnie, że~to są dość odległe byty. Czymś zupełnie innym
  jest pytanie, czy \alert{programy} komputerowe myślą i~czy posiadają
  świadomość? To stawiamy sobie bardzo często.

\end{frame}
% ##################





% ##################
\begin{frame}
  \frametitle{Wiener i~Lem}


  Wśród pytań na które bardzo chciałbym znać odpowiedź, jest następująca
  kwestia. Wiadomo, że~Lema bardzo zajmował temat, czy, a~jeśli tak to
  kiedy, maszyna staje~się osobą i~co to przemiana oznacza dla ludzi?
  Nurtował go też problem, że~jeśli człowiek jest tylko maszyną, to
  jak można obronić drogie mu wartości humanistyczne?

  Poza wspominanym już \textit{Terminusem} porusza je choćby w~opowiadaniach
  \textit{Wypadek}, \textit{Polowanie}, \textit{Rozprawa}, czy wspomnianej
  już powieści \textit{Niezwyciężony}. Wydaje mi~się, że~w~ten sposób można
  interpretować też przynajmniej niektóre \textit{Bajki robotów}. Jeśli
  przy tym weźmiemy pod uwagę propozycję Agnieszki Gajewskiej, żeby czytać
  twórczość Lema również jako próbę przepracowania koszmaru II~Wojny
  Światowej, stosowanie tego podejścia do~takich opowiadań jak
  \textit{Dwa potwory}, wydaje~się wartościowe poznawczo
  \parencite{Gajewska-Zaglada-i-gwiazdy-Pub-2017}. Temat ten wymaga jednak
  głębszej refleksji i~dziś nie chcę niczego przesądzać.

\end{frame}
% ##################





% ##################
\begin{frame}
  \frametitle{Wiener i~Lem}


  Jeśli choć pobieżnie przekartkujemy \textit{Cybernetykę i~społeczeństwo}
  Wienera, to znajdziemy tam wiele komentarzy do powyższych problemów, które
  wyglądają jakby wyszły spod pióra Lema. Choć związek przyczynowy był
  raczej w~drugą stronę.

  Weźmy choćby taki urywek ze str.~$44$ polskiego wydania
  \textit{The Human Use of Human Beings} z~$1961$~roku
  \parencite{Wiener-Cybernetyka-a-spoleczenstwo-Pub-2016}. \\
  \textit{W~bardzo realnym sensie jesteśmy rozbitkami na~planecie skazanej
    na zagładę. Ale nawet u~rozbitków ludzkie zalety i~ludzkie wartości
    nie muszą ginąć, i~należy wydobyć z~nich możliwie wszystko. Zatoniemy
    w~końcu, lecz niech to~się stanie w~sposób nie przynoszący ujmy naszej
    godności.}

  Rodzi~się więc pytanie, czy Wiener wpłynął swoim poglądami na~filozofię
  Lema? I~czy ten wpływ przetrwał okres zafascynowania pisarza cybernetyką?
  W~tym momencie muszę lepiej przestudiować twórczość Wienera, by móc
  w~tym temacie powiedzieć cokolwiek wartego uwagi. Pytanie to musi więc
  na razie pozostać otwarte.

\end{frame}
% ##################





% ##################
\begin{frame}
  \frametitle{Wróćmy jeszcze do \textit{Dialogów}}


  Dla szerszego kontekstu warto ponownie zacytować książkę \textit{Nauka
    o~informacji} Seidlera. \\
  \textit{Zarysowuje~się też dalej idący proces integracji nauk.
    W~złożonych systemach informacyjnych można bowiem dopatrzyć~się
    coraz więcej podobieństw do~procesów informacyjnych zachodzących
    w~organizmach żywych, a~nawet i~w~społeczeństwie, i~w~rezultacie
    zaczyna~się pojawiać wzajemne oddziaływanie między do~niedawna tak
    odległymi dziedzinami jak: technika, biologia i~nauki społeczne.
    Kluczowym elementem wspólnym dla tych dziedzin jest pojęcie informacji.}
  Str.~$10$,
  \parencite{Seidler-Nauka-o-informacji-Vol-I-Pub-1983}.

  Wróćmy jeszcze na chwilę do~tekstu \textit{Dialogów}. W~dialogach~iv i~v
  dochodzi do przesunięcia stylu dzieła. Coraz mniej są to dialogi w~pełnym
  znaczeniu tego słowa, coraz bardziej zaś wykłady Filonousa o~cybernetyce,
  których samotnym słuchaczem staje~się Hylas.

\end{frame}
% ##################





% ##################
\begin{frame}
  \frametitle{Sieci}


  W~dialogu~iv Filonous i~Hylas rozważają problem świadomości, ewolucji
  darwinowskiej i~jej podobieństwo, bądź jego braku do budowania maszyn,
  podłączenia~się ludzi do cudzych systemów nerwowych, jak również dochodzą
  do wniosku, że~z~„żelaznych brył” można zbudować istotę równie żywą
  jak~koń czy krowa.

  W~dialogu~v wprowadzone zostaje pojęcie sieci, które od tego momentu
  staje~się z~jednym z~podstawowych elementów wykładów Filonousa
  o~cybernetyce. \\
  \textsc{hylas}: \textit{Dobrze. Czy masz zamiar mówić teraz o~zbiorze
    układów zwanych sieciami?} \\
  \textsc{filonous}: \textit{Tak. Zbiór ten obejmuje układy o~stopniu
    złożoności większym bądź równym~„w”. Przez „w” rozumiem minimalną
    złożoność, jaką musi wykazać układ, abyśmy mogli zaliczyć go do
    zbioru.} \\
  Str.~$97$, \parencite{Lem-Dialogi-Vol-I-Pub-1996}.

\end{frame}
% ##################





% ##################
\begin{frame}
  \frametitle{Sieci neuronowe}


  Kilka stron dalej, zob.~str.~$111$,
  \parencite{Lem-Dialogi-Vol-I-Pub-1996}, czytamy już o~sieciach
  neuronowych rozważanych przez
  \colorhref{https://en.wikipedia.org/wiki/Warren\_Sturgis\_McCulloch}
  {Warrena Strugisa McCullocha} ($1898\text{--}1969$)
  i~\colorhref{https://en.wikipedia.org/wiki/Walter_Pitts}
  {Waltera Harry’ego Pittsa Jr.} ($1923\text{--}1969$).
  Autorzy ci w~$1943$
  opublikowali artykuł \colorhref{https://en.wikipedia.org/wiki/A\_Logical\_Calculus\_of\_the\_Ideas\_Immanent\_in\_Nervous\_Activity}
  {\textit{A~Logical Calculus of the Ideas Immanent in Nervous Activity}}
  w~który zaproponowali model neuronu, zbudowanej z~nich sieci, jak również
  przeprowadzają teoretyczną analizę problemu wykonywania na niej obliczeń
  logicznych klasycznego rachunku zdań
  \parencite{Bielecki-Sztuczne-sieci-neuronowe-Slowniki-ETC-Vol-XIII-Ver-2025}.

  Omawiany artykuł uważa~się za~początek badań sieci neuronowych, które
  od~roku~$2012$ przeżywają swoją kolejną młodość. Sami zaś McCulloch
  i~Pitts są pierwszymi cybernetykami, a~może w~ogóle pierwszymi realnie
  istniejącymi osobami, wymienionymi z~imienia i~nazwiska
  w~\textit{Dialogach}.

  % W~dialogu~v wprowadzone zostaje pojęcie sieci, które od tego momentu
  % staje~się z~jednym z~podstawowych elementów wykładów Filonousa
  % o~cybernetyce. \\
  % \textsc{hylas}: \textit{Dobrze. Czy masz zamiar mówić teraz o~zbiorze
  %   układów zwanych sieciami?} \\
  % \textsc{filonous}: \textit{Tak. Zbiór ten obejmuje układy o~stopniu
  %   złożoności większym bądź równym~„w”. Przez „w” rozumiem minimalną
  %   złożoność, jaką musi wykazać układ, abyśmy mogli zaliczyć go do
  %   zbioru.} \\
  % Str.~$97$, \parencite{Lem-Dialogi-Vol-I-Pub-1996}.

  % Kilka stron dalej, zob.~str.~$111$,
  % \parencite{Lem-Dialogi-Vol-I-Pub-1996}, czytamy już o~sieciach
  % neuronowych rozważanych przez
  % \colorhref{https://en.wikipedia.org/wiki/Warren\_Sturgis\_McCulloch}
  % {Warrena Strugisa McCullocha} ($1898\text{--}1969$)
  % i~\colorhref{https://en.wikipedia.org/wiki/Walter_Pitts}
  % {Waltera Harry’ego Pittsa Jr.} ($1923\text{--}1969$). Autorzy ci w~$1943$
  % opublikowali artykuł \textit{A~Logical Calculus of the Ideas Immanent in
  %   Nervous Activity} w~który proponują model neuronu, zbudowanej z~nich
  % sieci, jak również analizują teoretyczną możliwość przeprowadzania za~jej
  % pomocą obliczeń logicznych w~ramach klasycznego rachunku zdań
  % \parencite{Bielecki-Sztuczne-sieci-neuronowe-Slowniki-ETC-Vol-XIII-Ver-2025}.
  % Tym samym McCulloch i~Pitts zapoczątkowali badanie sieci neuronowych,
  % które od roku~$2012$ przeżywają swoją kolejną młodość. Są oni też
  % pierwszymi cybernetykami, a~może w~ogóle pierwszymi istniejącymi osobami,
  % wymienionymi z~nazwiska w~\textit{Dialogach}.

  % Lem przy tym doskonale wie, że~cybernetyczne sieci mogę~się uczyć na
  % podstawie przeszłych doświadczeń, co ilustruje choćby na przykładzie
  % sieci jaką „jest” niemowlę. Pamiętajmy, że~dla cybernetyka
  % $\text{zwierzę} = \text{maszyna}$.

\end{frame}
% ##################





% ##################
\begin{frame}
  \frametitle{Perceptron}


  Lem przy tym doskonale wie, że~cybernetyczne sieci mogę~się uczyć na
  podstawie przeszłych doświadczeń, co ilustruje choćby na przykładzie
  sieci jaką „jest” niemowlę. Pamiętajmy, że~dla cybernetyka
  $\text{zwierzę} = \text{maszyna}$. Znał więc, przynajmniej pobieżnie
  i~we wczesnym stadium rozwoju, pojęcie „sieci neuronowej”, ale według
  mojej obecnej interpretacji, uważał ją tylko za~szczególny przypadek
  ogólnego pojęcia „sieci”, którego używa znacznie częściej w~dialogu~v.

  Warto nadmienić, że~pierwszy układ elektroniczny, działający
  wedle zasad sieci neuronowej skonstruował amerykański psycholog
  \colorhref{https://en.wikipedia.org/wiki/Frank\_Rosenblatt}{Frank
    Rosenblatt} ($1928\text{--}1971$) w~$1957$~roku, przez co bywa nazywany
  „the father~of deep learning”. Jego układ nosił nazwę
  \colorhref{https://en.wikipedia.org/wiki/Perceptron}{\textit{perceptronu}}
  i~na jego projekt duży wpływ miała ówczesna wiedza na temat neuronów
  znajdujących~się w~siatkówce oka
  \parencite{Bielecki-Sztuczne-sieci-neuronowe-Slowniki-ETC-Vol-XIII-Ver-2025}. Oznacza to, że~gdy Lem opisywał sieci w~\textit{Dialogach} były
  one z~jego punktu widzenia koncepcją czysto teoretyczną.

\end{frame}
% ##################





% ##################
\begin{frame}
  \frametitle{Czym my już o~tym nie czytaliśmy?}


  Czy w~twórczości pisarskiej Lema spotykamy~się z~obiektami, w~których
  widać cybernetyczną koncepcję sieci? Choć mam kilka pomysłów gdzie zacząć
  ich szukać, to mam nadzieję, że~w~tej i~innych kwestiach, Państwa
  znajomość tego pisarza będzie lepsza, niż moja.

  Niezależnie od tego, zanim zakończymy nasze rozważania, warto zwrócić na
  chwilę uwagę na opowiadanie \textit{Ananke}. Badania dr.~Kukulaka
  pokazały jak na tym dziele odbiły~się panujący w~owym czasie, szczęśliwie
  krótkotrwały, pesymizm dotyczą perspektyw badań Układu Słonecznego,
  z~naszego jednak punktu widzenia, warto zwrócić uwagę na inny aspekt tego
  pełnego zniechęcenia opowiadania
  \parencite{Kukulak-Two-Faces-of-Mars-Pub-2023}. Mianowicie, w~opowiadaniu
  tym cierpiący na nerwicę natręctw emerytowany pilot Warren Cornelius
  przetrenowuje sztuczną inteligencję, każąc jej
  nadmiernie konsumować informacji i~unikać proszenia ludzkiego nadzorcy
  o~pomoc, co~prowadzi do katastrofy statku kosmicznego „Ariel”, śmierci
  trzydziestu osób oraz~jego samobójstwa.

\end{frame}
% ##################





% ##################
\begin{frame}
  \frametitle{Czym my już o~tym nie czytaliśmy?}


  \textit{[\ldots] o~tym, że~jest [komputer sterujący „Ariela”] przeciążony,
    zawiadomił swoją sterownię, to znaczy~-- ludzi „Ariela” dopiero w~$201$
    sekundzie procedury. Już wtedy dławił~się danymi~-- a~żądał wciąż
    nowych.} \\
  \textit{Ananke}, str.~$288$,
  \parencite{Lem-Ogrod-ciemnosci-i-inne-opowiadania-Pub-2017}.

  \textit{Przedstawił [von der Voyt] drogę, jaką przebywa każdy komputer~--
    od montażowej taśmy do~sterowni okrętu. [\ldots] Nie wypełnione jeszcze
    pamięcią [komputery], „nic nie widzące” jak noworodki, jechały do
    Bostonu, gdzie w~zakładach „Syntronics” odbywało~się ich programowanie.
    Po tym kolejnym akcie każdy komputer podlegał procedurze, która jest
    niejako odpowiednikiem nauk szkolnych, gdyż składa~się zarówno
    z~dostarczania pewnych „doświadczeń”, jak i~z~poddawania „egzaminom”.}
  \textit{Ananke}, str.~$290\text{-}291$,
  \parencite{Lem-Ogrod-ciemnosci-i-inne-opowiadania-Pub-2017}.

\end{frame}
% ##################










% ######################################
\section{Zakończenie i~plany na przyszłość}
% ######################################



% ##################
\begin{frame}
  \frametitle{Zakończenie}


  Choć tekst \textit{Dialogów} jest bogatszy, niż zaprezentowana powyżej,
  przedstawione rozważania dobrze oddają stan moich skromnych badań nad nim.
  Jak wspomniałem na początku, jestem dopiero tak w~jednej trzeciej
  planowanych nad cybernetyką i~Lemem, więc wciąż nie umiem odpowiedzieć
  na~większość pytań.

  W~planach mam dokładną lekturę dwóch wymienionych wcześniej dzieł
  Wienera, jak i~zbioru artykułów i~dyskusji \textit{Cybernetyka. Za
    i~przeciw}, który jest świadectwem debaty nad tą dziedziną, jaka
  odbyła~się w~Polsce około roku~$1965$. Temat wymaga też dokładnej lektury
  dzieł Lema oraz prześledzenia znanej chronologii ich powstania.

  Tematów do badań na pewno nie brakuje.

\end{frame}
% ##################










% ######################################
\appendix
% ######################################





% ######################################
\EndingSlide{Dziękuję! Pytania?}
% ######################################









% ####################################################################
% ####################################################################
% Bibliography

\printbibliography










% ############################

% Koniec dokumentu
\end{document}
