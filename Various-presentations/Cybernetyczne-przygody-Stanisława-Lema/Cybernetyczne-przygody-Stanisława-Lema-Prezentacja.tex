% ------------------------------------------------------------------------------------------------------------------
% Basic configuration of Beamera class and Jagiellonian theme
% ------------------------------------------------------------------------------------------------------------------
\RequirePackage[l2tabu, orthodox]{nag}



\ifx\PresentationStyle\notset
  \def\PresentationStyle{dark}
\fi



% Options: t -- align frame text to the top
\documentclass[10pt,t]{beamer}
\mode<presentation>
\usetheme[style=\PresentationStyle]{jagiellonian}





% ------------------------------------------------------------------------------------
% Procesing configuration files of Jagiellonian theme located in
% the directory "preambule"
% ------------------------------------------------------------------------------------
% Configuration for polish language
% Need description
\usepackage[polish]{babel}
% Need description
\usepackage[MeX]{polski}



% ------------------------------
% Better support of polish chars in technical parts of PDF
% ------------------------------
\hypersetup{pdfencoding=auto,psdextra}

% Package "textpos" give as enviroment "textblock" which is very usefull in
% arranging text on slides.

% This is standard configuration of "textpos"
\usepackage[overlay,absolute]{textpos}

% If you need to see bounds of "textblock's" comment line above and uncomment
% one below.

% Caution! When showboxes option is on significant ammunt of space is add
% to the top of textblock and as such, everyting put in them gone down.
% We need to check how to remove this bug.

% \usepackage[showboxes,overlay,absolute]{textpos}



% Setting scale length for package "textpos"
\setlength{\TPHorizModule}{10mm}
\setlength{\TPVertModule}{\TPHorizModule}


% ---------------------------------------
% TikZ
% ---------------------------------------
% Importing TikZ libraries
\usetikzlibrary{arrows.meta}
\usetikzlibrary{positioning}





% % Configuration package "bm" that need for making bold symbols
% \newcommand{\bmmax}{0}
% \newcommand{\hmmax}{0}
% \usepackage{bm}




% ---------------------------------------
% Packages for scientific texts
% ---------------------------------------
% \let\lll\undefined  % Sometimes you must use this line to allow
% "amsmath" package to works with packages with packages for polish
% languge imported
% /preambul/LanguageSettings/JagiellonianPolishLanguageSettings.tex.
% This comments (probably) removes polish letter Ł.
\usepackage{amsmath}  % Packages from American Mathematical Society (AMS)
\usepackage{amssymb}
\usepackage{amscd}
\usepackage{amsthm}
\usepackage{siunitx}  % Package for typsetting SI units.
\usepackage{upgreek}  % Better looking greek letters.
% Example of using upgreek: pi = \uppi


\usepackage{calrsfs}  % Zmienia czcionkę kaligraficzną w \mathcal
% na ładniejszą. Może w innych miejscach robi to samo, ale o tym nic
% nie wiem.










% ---------------------------------------
% Packages written for lectures "Geometria 3D dla twórców gier wideo"
% ---------------------------------------
% \usepackage{./ProgramowanieSymulacjiFizykiPaczki/ProgramowanieSymulacjiFizyki}
% \usepackage{./ProgramowanieSymulacjiFizykiPaczki/ProgramowanieSymulacjiFizykiIndeksy}
% \usepackage{./ProgramowanieSymulacjiFizykiPaczki/ProgramowanieSymulacjiFizykiTikZStyle}





% !!!!!!!!!!!!!!!!!!!!!!!!!!!!!!
% !!!!!!!!!!!!!!!!!!!!!!!!!!!!!!
% EVIL STUFF
\if\JUlogotitle1
\edef\LogoJUPath{LogoJU_\JUlogoLang/LogoJU_\JUlogoShape_\JUlogoColor.pdf}
\titlegraphic{\hfill\includegraphics[scale=0.22]
{./JagiellonianPictures/\LogoJUPath}}
\fi
% ---------------------------------------
% Commands for handling colors
% ---------------------------------------


% Command for setting normal text color for some text in math modestyle
% Text color depend on used style of Jagiellonian

% Beamer version of command
\newcommand{\TextWithNormalTextColor}[1]{%
  {\color{jNormalTextFGColor}
    \setbeamercolor{math text}{fg=jNormalTextFGColor} {#1}}
}

% Article and similar classes version of command
% \newcommand{\TextWithNormalTextColor}[1]{%
%   {\color{jNormalTextsFGColor} {#1}}
% }



% Beamer version of command
\newcommand{\NormalTextInMathMode}[1]{%
  {\color{jNormalTextFGColor}
    \setbeamercolor{math text}{fg=jNormalTextFGColor} \text{#1}}
}


% Article and similar classes version of command
% \newcommand{\NormalTextInMathMode}[1]{%
%   {\color{jNormalTextsFGColor} \text{#1}}
% }




% Command that sets color of some mathematical text to the same color
% that has normal text in header (?)

% Beamer version of the command
\newcommand{\MathTextFrametitleFGColor}[1]{%
  {\color{jFrametitleFGColor}
    \setbeamercolor{math text}{fg=jFrametitleFGColor} #1}
}

% Article and similar classes version of the command
% \newcommand{\MathTextWhiteColor}[1]{{\color{jFrametitleFGColor} #1}}





% Command for setting color of alert text for some text in math modestyle

% Beamer version of the command
\newcommand{\MathTextAlertColor}[1]{%
  {\color{jOrange} \setbeamercolor{math text}{fg=jOrange} #1}
}

% Article and similar classes version of the command
% \newcommand{\MathTextAlertColor}[1]{{\color{jOrange} #1}}





% Command that allow you to sets chosen color as the color of some text into
% math mode. Due to some nuances in the way that Beamer handle colors
% it not work in all cases. We hope that in the future we will improve it.

% Beamer version of the command
\newcommand{\SetMathTextsColor}[2]{%
  {\color{#1} \setbeamercolor{math text}{fg=#1} #2}
}


% Article and similar classes version of the command
% \newcommand{\SetMathTextColor}[2]{{\color{#1} #2}}










% ---------------------------------------
% Commands for setting background pictures for some slides
% ---------------------------------------
\newcommand{\TitleBackgroundPicture}
{./PresentationPictures/CommonPictures/Cute_dragon_BG_dark.png}
\newcommand{\SectionBackgroundPicture}
{./PresentationPictures/CommonPictures/Cute_dragon_small_BG_light.png}



\newcommand{\TitleSlideWithPicture}{
  \begingroup

  \usebackgroundtemplate{ % \hspace*{-11.5em}
    \includegraphics[height=\paperheight]{\TitleBackgroundPicture}}

  \maketitle

  \endgroup
}





\newcommand{\SectionSlideWithPicture}[1]{%
  \begingroup

  \usebackgroundtemplate{ % \hspace*{-11.5em}
    \includegraphics[height=\paperheight]{\SectionBackgroundPicture}}

  \setbeamercolor{titlelike}{fg=normal text.fg}

  \section{#1}

  \endgroup
}





\newcommand{\EndingSlide}[1]{%
  \begin{frame}[standout]

    \begingroup

    \color{jFrametitleFGColor}

    #1

    \endgroup

  \end{frame}
}










% ------------------------------------------------------------------------------------
% Importing packages, libraries and setting their configuration
% ------------------------------------------------------------------------------------





% ------------------------------------------------------
% BibLaTeX
% ------------------------------------------------------
% Package biblatex, with biber as its backend, allow us to handle
% bibliography entries that use Unicode symbols outside ASCII.
\usepackage[
language=polish,
backend=biber,
style=alphabetic,
url=false,
eprint=true,
]{biblatex}

\addbibresource{Cybernetyczne-przygody-Stanisława-Lema-Bibliography.bib}





% ------------------------------------------------------
% Wonderful package PGF/TikZ
% ------------------------------------------------------

% Node and pics for drawing charts
% \usepackage{./Local-packages/PGF-TikZ-Chart-nodes-and-pics}

% Styles for arrows
% \usepackage{./Local-packages/PGF-TikZ-Arrows-styles}





% ------------------------------------------------------
% Local packages
% ------------------------------------------------------
% Special configuration for this particular presentation
\usepackage{./Local-packages/local-settings}

% Package containing various command useful for working with a text
\usepackage{./Local-packages/general-commands}

% Package containing commands and other code useful for working with
% mathematical text
% \usepackage{./Local-packages/math-commands}










% ------------------------------------------------------------------------------------------------------------------
\title{Cybernetyczne przygody Stanisława Lema}

\author{Kamil Ziemian \\
  \email}


% \institute{Uniwersytet Jagielloński w~Krakowie}

\date[14 V~2025~r.]{14 V~2025~r.}
% ------------------------------------------------------------------------------------------------------------------









% ####################################################################
\begin{document}
% ####################################################################





% ######################################
% Number of chars: 48k+,
% Text is adjusted to the left and words are broken at the end of the line.
\RaggedRight
% ######################################





% ######################################
\maketitle
% ######################################





% ######################################
\begin{frame}
  \frametitle{Plan prezentacji}


  \tableofcontents

\end{frame}
% ######################################










% ######################################
\section{O~moich badaniach nad Lemem i~cybernetyką}
% ######################################



% ##################
\begin{frame}
  \frametitle{Jak to~się zaczęło}


  Nie jestem ani zawodowym literaturoznawcą, ani cybernetykiem, wypada więc
  wyjaśnić czemu akurat ja dzisiaj staję przed Państwem. Wszystko
  zaczęło~się od serii rozmów z~dwoma miłośnikami Stanisława Lema,
  \colorhref{https://historia.uj.edu.pl/instytut/pracownicy/marcin-jarzabek}
  {dr.~Marcinem Jarząbkiem}
  i~\colorhref{https://incet.uj.edu.pl/maciej-prochnicki}{dr.~Maciejem
    Próchnickim}, którzy około marca $2023$ roku, zwrócili mi uwagę no to,
  iż przeszedł on fazę fascynacji cybernetyką. Mając już wtedy szczątkową
  wiedzę na temat tego na temat tego, czym jest cybernetyka, dziedziną
  wiedzy mocno zmatematyzowaną i~techniczną, postawiłem sobie dwa powiązane
  pytania. Czym była cybernetyka którą studiował Lem i~jak on ją rozumiał?

  By na to odpowiedzieć podjąłem amatorskie studia nad cybernetyką, które
  rodziły wciąż nowe i~nowe pytania, zarówno gdy chodzi o~twórczość Lema,
  jak i~cybernetykę. Dwa najważniejsze dziś, to obecność motywów
  cybernetyczne w~dziełach literackich Lema oraz związek jego światopoglądu
  z~tym \colorhref{https://en.wikipedia.org/wiki/Norbert\_Wiener}{Norberta
    Wienera}.

\end{frame}
% ##################





% ##################
\begin{frame}
  \frametitle{Jak to~się zaczęło}


  Wygląda na to, że~pomimo opublikowania przez Pawła Majewskiego monografii
  \textit{Między zwierzęciem a~maszyną. Utopia technologiczna Stanisława
    Lema} w~$2007$~roku, problem cybernetycznych przygód Lema pozostaje
  niezwykle mało przebadanym zagadnieniem, więc również ludzie z~zewnątrz
  jak ja, mogą~się podjąć jego studiowania, z~nadzieją na odkrycie czegoś
  ciekawego.

  Dużą inspirację do~podjęcia tego tematu stanowiły dla mnie też badania
  mojego dobrego znajomego
  \colorhref{https://skos.agh.edu.pl/osoba/szymon-kukulak-10650.html}
  {dr.~Szymona Kukulaka} z~Wydziału Humanistycznego \textsc{agh}, nad
  wpływem odkryć technicznych i~eksploracji Układu Słonecznego na prozę
  Lema, o~których pierwszy raz usłyszałem na początku $2015$ roku. Z~wielu
  jego prac, chciałbym teraz wskazać na jego artykuł \textit{Two Faces of
    Mars: The Red Planet in Stanisław Lem's The Man from Mars" and
    “Ananke” in Light of Contemporary Scientific Pursuits and Martian
    Fiction}.

\end{frame}
% ##################





% ##################
\begin{frame}
  \frametitle{Stan na dzień dzisiejszy}


  Jak wskazuje tytuł dr~Kukulak analizuje w~tym artykule jak literacki
  obraz Marsa ewoluował między \textit{Człowiekiem z~Marsa} a~opowiadaniem
  \textit{Ananke}, w~związku z~nowymi informacjami o~tej planecie
  \parencite{Kukulak-Two-Faces-of-Mars-Pub-2023}. Dochodzi on tam do bardzo
  ciekawego wniosku, że~w~\textit{Ananke} widzimy obraz tej planety, który
  w~zasadzi od razu stał~się przestarzały, w~skutek misji Marinera~$9$
  z~$1971$ roku i~Mariner~$10$ z~$1973$ roku.

  Świadom moich braków wiedzy w~zakresie cybernetyki, nawiązałem kontakt
  z~\colorhref{https://skos.agh.edu.pl/osoba/andrzej-bielecki-7555.html}
  {prof.~Andrzejem Bieleckim} z~Wydziału Elektrotechniki, Automatyki,
  Informatyki i~Inżynierii Biomedycznej \textsc{agh}, wielkim miłośnikiem
  i~znawcą twórczość Lema. Jemu również zawdzięczam ogromnie wiele cennych
  informacji.

  Obecnie jestem w~około $1 / 3$ moich planowanych badań, siłą rzeczy
  ma więcej pytań niż odpowiedzi i~o~wielu ważny rzeczach nie mogę nic
  wartościowego powiedzieć.

\end{frame}
% ##################





% ##################
\begin{frame}
  \frametitle{Stan na dzień dzisiejszy}


  Jedna z~rzeczy, które będą musiał pominąć, to relacja Lema z~polskim
  środowiskiem cybernetycznym, który to temat wydaje~się być szaleńczo
  zaniedbany. Wydaje~się, że~do tej pory przyjmowana milcząco, że~Lem
  uczył~się cybernetyki od autorów anglojęzycznych oraz~rosyjsko języcznych
  i~problem tego, czy Lem utrzymywał jakikolwiek kontakt intelektualny
  z~polskimi cybernetykami po~$1950$ roku był ignorowany.

  Wiemy, że~w~latach $1948\text{-}1950$ Lem brał udział w~Konwersatorium
  Naukoznawczym Asystentów Uniwersytetu Jagiellońskiego, kierowanym
  przez dr.~Mieczysława Choynowskiego, które w~tymże roku $1950$ zostało
  rozwiązane i~na pewno wywarło duży wpływ na jego myślenie,
  również o~cybernetyce. Jednak późniejszy jednak okres Lemowskich
  zainteresowań cybernetyką, wydaje~się bardzo mało przebadany.

\end{frame}
% ##################









% ######################################
\section{Cybernetyka przed i~po~$1945$ roku}
% ######################################


% ##################
\begin{frame}
  \frametitle{Cybernetyka po~1945 roku}


  W~powieści Lema \textit{Eden} z~$1958$~roku, jeden z~sześciu głównych
  bohaterów, Cybernetyk jest odpowiedzialny za pracę z~automatami, które
  w~naszym dzisiejszym rozumieniu obejmują tak komputery, jak i~różnego
  rodzaju roboty, do których ma bardzo osobisty stosunek. Dowiadujemy~się
  bowiem, że~katastrofa
  ich rakiety była spowodowana wprowadzeniem do automatu liczącego błędnych
  danych oraz, że~automat Obrońca, zapewne robot, jak wszystkie inne
  został zniszczony lub unieruchomiony podczas awaryjnego lądowania. Doktor
  komentuje, że~wobec tego Cybernetyk „Owdowiał”.

  Lem użył tu słowa „automat” najprawdopodobniej w~sensie „urządzenie
  (maszyna, aparat, przyrząd) umożliwiający realizację procesu
  produkcyjnego bez osobistego udziału człowieka, a~tylko pod jego
  nadzorem”, str.~$19$. Definicja ta w~zasadzie może obejmować i~komputer
  i~robota bojowego.

\end{frame}
% ##################





% ##################
\begin{frame}
  \frametitle{Cybernetyka po~1945 roku}


  Ten przykład ilustruje pewną trudność mówienia o~powojennej cybernetyce,
  która z~naszego punktu widzenia obejmuje zagadnienia, które
  określilibyśmy jako należące do informatyki, teorii informacji,
  badań na sztuczną inteligencją, w~tym uczenia maszynowego,
  robotyki, neurofizjologi, socjologii, psychologii,~etc.

  By przybliżyć nam sens słowa „cybernetyka” warto przytoczyć słowa Lema
  z~jego wydanego w~$1957$ roku dzieła \textit{Dialogi}, które są w~jakimś
  sensie tekstem popularnonaukowym o~cybernetyce. \\
  \textit{Doniosłość prac cybernetyków na tym się właśnie zasadza,
    że~wykryli odpowiedź na to pytanie [Czym jest informacja?]. Informacja
    jest dzieckiem termodynamiki, postawionej, mówiąc obrazowo, na głowie,
    gdyż stanowi odwrotność entropii.}
  Str.~$56$, \parencite{Lem-Dialogi-Vol-I-Pub-1996}.

  Tym więc co miało z~cybernetyki dziedzinę superdyscyplinarną (określenie
  użyte retrospektywnie przez Lema, por. str.~$6\text{--}7$,
  \parencite{Lem-Dialogi-Vol-I-Pub-1996}), było pojęcie informacji,
  co odsyła nas do dzieł Norbert Wienera.

\end{frame}
% ##################





% ##################
\begin{frame}
  \frametitle{Norbert Wiener, ojciec cybernetyki}


  \begin{figure}

    \centering


    \includegraphics[scale=0.23]
    {./Presentation-pictures/Norbert-Wiener-01.jpg}


    \caption{\colorhref{https://en.wikipedia.org/wiki/Norbert\_Wiener}
      {Norbert Wiener} ($1894\text{--}1964$), amerykański filozof,
      informatyk i~matematyk, ojciec cybernetyki.}


    \label{fig:Wiener-01}

  \end{figure}

\end{frame}
% ##################





% ##################
\begin{frame}
  \frametitle{Arturo Rosenblueth, zapomniany współtwórca
    cybernetyki}


  \begin{figure}

    \centering


    \includegraphics[scale=0.37]
    {./Presentation-pictures/Arturo-Rosenblueth-01.png}


    \caption{\colorhref{https://en.wikipedia.org/wiki/Arturo\_Rosenblueth}
      {Arturo Rosenblueth} ($1900\text{--}1970$), meksykański fizjolog,
      lekarz, wskazany przez Wienera jako kluczowa postać w~rozwoju
      cybernetyki.}


    \label{fig:Rosenblueth-01}

  \end{figure}

\end{frame}
% ##################





% ##################
\begin{frame}
  \frametitle{\textit{Cybernetics: Or~control and~Communication \ldots}}


  Norbert Wiener był niewątpliwie wybitnym uczonym, który dokonał
  znaczących odkryć w~wielu dziedzinach wiedzy, my jednak musimy~się
  skupić na jego roli jako ojca cybernetyki. Został nim poprzez
  opublikowanie w~roku $1948$ książki
  \colorhref{https://en.wikipedia.org/wiki/Cybernetics\_Or\_Control\_and\_Communication\_in\_the\_Animal\_and\_the\_Machine}
  {\textit{Cybernetics: Or~control and~Communication in the~Animal and
      the~Machine}}

  Książka ta w~trudny dla mnie do~zrozumienia sposób stała~się fenomenem
  kulturowym, który najlepiej chyba oddają słowa z~jej recenzji, która
  ukazał w~piśmie \textit{Business Week} z~$1949$ roku: \textit{Pod jednym
    względem książka Wienera jest jak Raport Kinseya: reakcja społeczeństwa
    na nią jest tak samo ważna, jak jej treść.}, tłum. wł., cyt.
  za~str.~$408$
  \parencite{Hamilton-The-Charismatic-Cultural-Life-of-ETC-Pub-2017}. Sam
  zaś Wiener uzyskał dzięki niej status celebryty, który to temat jest
  szerzej omawia Sheryl N.~Hamilton w~cytowanym uprzednio artykule.

  Kulturowa recepcja cybernetyki, to temat bardzo interesujący, który
  niestety musimy pominąć.

\end{frame}
% ##################





% ##################
\begin{frame}
  \frametitle{Historia cybernetyki według Wienera}


  Muszę jednak wspomnieć, że~urodzony w~$1963$ roku w~Moskwie,
  sowiecko-amerykański historyk, inżynier i~matematyk
  \colorhref{https://en.wikipedia.org/wiki/Slava_Gerovitch}{Slava Gerovitch}
  napisał książkę
  \colorhref{https://mitpress.mit.edu/9780262572255/from-newspeak-to-cyberspeak/}
  {\textit{From Newspeak to~Cyberspeak. A~History of Soviet Cybernetics}},
  o~recepcji cybernetyki w~Związku Sowieckim. Do tej pory inne kraju
  Bloku Wschodniego nie doczekały~się chyba równie dobrego opracowania
  fenomenu cybernetyki, choć u~nas Jan Trąbka opublikował
  \textit{Pół wieku cybernetyki w~Polsce}.

  W~$1950$ Wiener publikuje mniejszą, bardzie
  popularną niż \texttt{Cybernetics\ldots} książkę
  \colorhref{https://en.wikipedia.org/wiki/The\_Human\_Use\_of\_Human\_Beings}
  {\textit{The Human Use of Human Beings}}, w~Polsce wydana jako
  \textit{Cybernetyka i~społeczeństwo} w~$1960$~roku. W~kontekście prozy
  Lema jest to również bardzo ważna pozycja.

  Wiener miał tendencje do tego by wiele mówić o~sobie i~swoim
  życiu i~w~rozdziale wprowadzający do pierwszego wydania
  \textit{Cybernetics} zostawił nam swoją relację z~historii powstania
  cybernetyki.

\end{frame}
% ##################





% ##################
\begin{frame}
  \frametitle{Historia cybernetyki według Wienera}


  Do historyków należy ocenienie, jak bardzo relacja
  Wieniera jest wierna rzeczywistości, jest jednak niewątpliwie cennym
  i~ciekawym źródłem wiedzy o~formowaniu~się tej dziedziny i~o~poglądach
  samego Wienera, które wywarły na nią tak silny wpływ.

  Z~historii tej musimy wynotować kilka punktów. Około $1937$ roku
  Wiener dołącza do serii comiesięcznych dyskusji prowadzonych przez
  Rosenbluetha pracującego wówczas na Harvard Medical Schools. Od początku
  łączy ich silne przekonanie o~potrzebnie badań interdyscyplinarnych,
  acz~dopiero II Wojna Światowa stwarza warunki do bliższej współpracy.
  W~związku z~wojną Wiener pracuje nad działami przeciwlotniczymi, jednym
  z~jego ważnych współpracowników jest jeden z~pionierów budowy komputerów
  \colorhref{https://en.wikipedia.org/wiki/Julian_Bigelow}{Julia
    H.~Bigelow}. Mierzą~się oni z~problemem, że~ze względu na wysokość lotu
  samolotu i~prędkość lotu pocisku, aby zestrzelić samolot należy celować
  nie w~jego obecne, lecz przyszłe położenie.

\end{frame}
% ##################





% ##################
\begin{frame}
  \frametitle{Cybernetyka i~sprzężenie zwrotne}


  Planują oni wyposażyć działa przeciwlotnicze w~dobrze już wtedy znany
  mechanizm sprzężenia zwrotnego (ang.~\textit{feedback}). Najpóźniej
  w~$1957$ roku będą już dysponowali koncepcją uczącego~się działa
  przeciwlotniczego. Przedmowa do II-ego wydania \textit{Cybernetics\ldots}
  z~$1961$ roku pokazuje, że~Wiener znał już wtedy, przynajmniej w~ogólnym
  zarysie, podstawowe pojęcia uczenia maszynowego.

  W~tym wypadku warto się zapytać o~opowiadanie Lema \textit{Terminus}.
  Czy nie jest to opowiadania o~przesłaniu duszy (?) ludzi do
  tytułowego robota, za pomocą procedury uczenia maszynowego? Bardzo
  jestem ciekawy, czy według Państwa takie podejście ma w~ogóle sens?

  Wracając do historii Wienera i~jego grupy, w~pracy swojej wspierają~się
  książką L.A. MacColla
  \colorhref{https://archive.org/details/fundamentaltheor0000macc}
  {\textit{Fundamental theory of servomechanisms}}, gdzie opisane jest
  zjawisko \textit{polowania} (ang. \textit{hunting}), kiedy
  sprzężenie zwrotne, zamiast prowadzić do coraz lepszego działania
  mechanizmu, wprawia go w~coraz mocniejsze oscylacje prowadzące do jego
  zniszczenia.

\end{frame}
% ##################





% ##################
\begin{frame}
  \frametitle{Zwierzęta i~maszyny}


  Wiener podaje, że~on i~Bigelow zwrócili~się do Rosenbluetha z~pytanie, czy
  znana jest choroba, która prowadzi do zachowań takich jak polowanie
  w~maszynach? On zaś wskazał im na zjawisko drżenia zamiarowego
  (łac.
  \colorhref{https://en.wikipedia.org/wiki/Intention_tremor}{\textit{tremor
      intentionalis}}). Ten wymiana myśli zostało uznana przez Wienera za
  ważną, zapewne dlatego, że~ugruntowała przekonanie, że~cybernetyka dotyczy
  tak samo maszyn jak i~zwierząt.

  Wybiegając naprzód należy dodać, że~w~$1946$ roku Wiener i~Rosenblueth
  podejmą wspólne badania nad mięśniami kota, traktowanymi jako swego
  rodzaju maszyna (serwomechanizm). Będą oni badali oscylacje jego mięśni
  w~zależności od siły pobudzenia ich impulsem elektrycznym.
  A~gdzie jest granica człowiek-maszyna?

  Wiadomo, że~Lema bardzo zajmował temat, czy, a~jeśli tak to kiedy,
  maszyna staje~się osobą i~co to przemiana oznacza dla ludzi? Nurtował go
  też problem, że~jeśli człowiek jest tylko maszyną, to jak można obronić
  drogie mu wartości humanistyczne?

\end{frame}
% ##################





% ##################
\begin{frame}
  \frametitle{Człowiek/maszyna w~prozie Lema}


  Poza wspominanym już \textit{Terminusem} porusza je choćby w~opowiadaniach
  \textit{Wypadek}, \textit{Polowanie}, \textit{Rozprawa}, czy powieści
  \textit{Niezwyciężony}. Wydaje mi~się, że~w~ten sposób można
  interpretować też przynajmniej niektóre \textit{Bajki robotów}. Jeśli
  przy tym weźmiemy pod uwagę propozycję Agnieszki Gajewskiej, żeby czytać
  twórczość Lema również jako próbę przepracowania koszmaru II~Wojny
  Światowej, stosowanie tego podejścia do~takich opowiadań jak
  \textit{Dwa potwory}, wydaje~się wartościowe poznawczo
  \parencite{Gajewska-Zaglada-i-gwiazdy-Pub-2017}. Temat ten wymaga jednak
  głębszej refleksji i~dziś nie chcę niczego przesądzać.

  Prof.~Bielecki zwrócił mi przy tym uwagę, że~istnieje opowiadanie Lema,
  którego nie chciał on nigdy wznawiać i~do~którego pani Gajewskiej
  nie udało~się dotrzeć w~czasie pisania \textit{Zagłady i~gwiazd}, gdyż
  nie jest ono tam wspomniane. Opowiada o~robocie ukrywający~się przez
  chcącym go zniszczyć ludźmi i~zapewne to odsłonięcie swoich przeżyć
  z~czasów II~Wojny Światowej, poskutkowała postawą Lema.

\end{frame}
% ##################





% ##################
\begin{frame}
  \frametitle{Lem studiował wszak medycynę}


  Z~historii cybernetyki wiadomo, że~wielu członków tego ruchu w~latach
  $40$-tych i~$50$-tych było fizjologami, lekarzami lub psychologami.
  Przytoczona wyżej historia pozwala lepiej zrozumieć dlaczego ten nurt
  badań przyciągnął tak wielu członków tej profesji. Wiemy też, że~Lem
  studiował medycynę, jak wskazują biografowie, pod wpływem ojca i~bez
  wielkiego entuzjazmu, co jednak prawie na pewno miało wpływ na
  jego początkową fascynację tym przedmiotem.

  Jednym z~wczesnych cybernetyków, z~tej kategorii o~którym potrzebuję
  wspomnieć jest
  \colorhref{https://en.wikipedia.org/wiki/W._Ross_Ashby}{William
    Ross Ashby} ($1903\text{-}1972$), który od~$1930$ roku prowadził
  jako psychiatra badania kliniczne. Jego osoba jest ważna, bo mam
  podejrzenie,
  że~spotkał go Ijon Tichy, gdy Ashby z~niewiadomych powodów ukrywał~się
  pod nazwiskiem prof.~Corcorana. Niemniej to na razie tylko przypuszczenia.

\end{frame}
% ##################





% ##################
\begin{frame}
  \frametitle{Homeostaty Ashby’ego}


  \begin{figure}

    \centering


    \includegraphics[scale=1.1]
    {./Presentation-pictures/W\_Ross\_Ashbys\_1948\_Homeostat.jpeg}


    \caption{\colorhref{https://en.wikipedia.org/wiki/Homeostat}
      {Homeostat} skonstruowane przez W.~Rossa Ashby’ego w~$1948$~roku.}


    \label{fig:Homeostat-01}

  \end{figure}

\end{frame}
% ##################





% ##################
\begin{frame}
  \frametitle{Cybernetyka, antropologia i~socjologia}


  Ostatnim fragmentem historii cybernetyki przedstawiony
  w~\textit{Introduction}, o~którym mówimy wspomnieć, jest ten, gdy
  pierwszy raz Wiener pisze o~jej związkach z~antropologią i~socjologią.

  W~latach $40$-tych XX wieku w~Nowym Yorku odbywał~się cykl spotkań
  finansowanych przez
  \colorhref{https://en.wikipedia.org/wiki/Josiah_Macy_Jr._Foundation}
  {Jay???}, którymi zarządzał
  \colorhref{https://en.wikipedia.org/wiki/Frank_Fremont-Smith}
  {Frank Fremont-Smith}.
  Na~wiosnę $1946$ roku, jeden z~twórców teorii sieci neuronowych
  \colorhref{https://en.wikipedia.org/wiki/Warren_Sturgis_McCulloch}
  {Warren McCulloch} nawiązuje współpracę z Jay??? i~razem
  Fremont-Smithem organizują cykl spotkań o~sprzężeniu zwrotnym, na które
  zapraszają również antropologów, filozofów i psychologów. Wiener wymienia
  z~ich uczestników choćby filozofa
  \colorhref{https://en.wikipedia.org/wiki/F._S._C._Northrop}
  {F.S.C. Northropa}.

  Zasadność zaproszenia takich osób komentuje następującymi słowami.

\end{frame}
% ##################





% ##################
\begin{frame}
  \frametitle{Cybernetyka, antropologia i~socjologia}


  Wiener pisze, że~zwrócili~się do niego ?? Mead. Choć uznał, że~to jest
  rzeczywiście istotny problem, ale nie wierzy, iż~cybernetyka będzie tu
  specjalnie użyteczna. Wobec jego poprzednich deklaracji, warto~się
  przyjrzeć co go do takiego stwierdzenia motywuje. Wyjaśniając swoje
  stanowisko w~tej sprawie, Wiener tłumaczy, że~kluczowe dla cybernetyki
  są szeregi czasowe i~statystyka. Statystyka w~jego czasach
  mogła dawać wartościowe rezultaty tylko wtedy, gdy dysponowało~się
  dostatecznie dużą próbkę danych, zebranych w~prawie stałych warunkach.
  Nie czuję~się kompetentny dyskutować, czy dzisiaj statystyka jest
  w~stanie rozwiązywać inne typu problemy. Ponieważ w~ocenie Wienera,
  społeczeństwo zmienia się obecnie (okolice $1948$ roku), tak szybko,
  że~nie mamy sensownej próbki statycznej, zebranej przy
  \alert{prawie stałych} warunkach, że~metody statystyczne nic ciekawego nam
  nie powiedzą.

\end{frame}
% ##################





% ##################
\begin{frame}
  \frametitle{Cybernetyka, antropologia i~socjologia}


  Niemniej, jak wiemy, część cybernetyków podjęło~się tworzenia teorii
  antropologicznych i~społecznych. Rodzi~się oczywiście pytanie, czy
  refleksje Lema na temat człowieka i~społeczeństw, noszą jakieś ślady
  myślenia cybernetycznego? Jest to trudny temat, bowiem zbyt łatwo jest
  wcisnąć siłą jego rozważania w~przyjęte ramy interpretacyjne, dlatego
  dziś nie czuję~się kompetentny o~nim mówić.

  Należy dodać, że~po swoich komentarzach o~antropologi i~socjologii,
  Wiener proponuje zwrócenie~się ku bardziej obiecującej dziedziną:
  robotyce oraz ???
  Dokładniej proponuje on zajęcie~się mechanicznymi protezami rąk i~nóg,
  które pozwolę „odzyskać” czucie w~utraconych kończynach oraz ???

\end{frame}
% ##################





% ##################
\begin{frame}
  \frametitle{Wiener, wieszcz i~poeta}


  Przytoczmy jeszcze kilka wypowiedzi Wieniera, które obrazują często
  ignorowaną stronę jego twórczości: wieszcza i~poeta.

  \textit{W~bardzo realnym sensie jesteśmy rozbitkami na~planecie skazanej
    na zagładę. Ale nawet u~rozbitków ludzkie zalety i~ludzkie wartości
    nie muszą ginąć, i~należy wydobyć z~nich możliwie wszystko. Zatoniemy
    w~końcu, lecz niech to~się stanie w~sposób nie przynoszący ujmy naszej
    godności.} Str.~$44$, polskiego wydania \textit{The~Human Use~of Human
    Beings} \parencite{Wiener-Cybernetyka-a-spoleczenstwo-Pub-2016}.

\end{frame}
% ##################





% ##################
\begin{frame}
  \frametitle{Kilka wniosków}


  Z~historii opowiedzianej przez Wienera wynika, że~cybernetyka jaką
  zaprezentował światu, była nauką, która całkiem świadomie chciała
  być interdyscyplinarna i~dotyczyć zarówno maszyn jak i~istot żywych,
  w~tym człowieka oraz~społeczeństw ludzkich. Również te nurty cybernetyki,
  które w~naszym odczuciu są~wyjątkowo problematyczne, jak cybernetyka
  społeczna, mogą~się wspierać na autorytecie Wienera, pomimo jego
  zastrzeżeń. W~naszej opinii, wiele problemów jakie napotkała później
  cybernetyka i~pewne osobliwości jej historii nie były w~żadnym razie
  przypadkiem, lecz wynikały logicznie z~projektu przedstawionego przez
  jej ojca.

  Lektura przytoczonych fragmentów pism Wienera, sugeruje, że~on i~Lem byli
  sobie bardzo bliscy, nie tylko na poziomie światopoglądu, ale niekiedy
  również na poziomie stylu literackiego. Kwestia ich wzajemnej relacji na
  tych dwóch poziomach, to jedno z~dwóch głównych pytań, na które teraz
  próbujemy znaleźć odpowiedź, dlatego dziś nie chcemy udzielać żadnych
  kategorycznych odpowiedzi na nie.

\end{frame}
% ##################










% ######################################
\section{Kilka słów o~Lemie i~cybernetyce}
% ######################################


% ##################
\begin{frame}
  \frametitle{„Okres cybernetyczny”}


  Za dwa główne dzieła cybernetyczne Lema uważa~się wydane w~$1957$ roku
  \textit{Dialogi}, które jest świadectwem wielkiej wiary w~moc tej nauki,
  oraz \textit{Summę techologiae} z~$1964$. Tym właśnie dwóm dzieło
  poświęcona jest prawie całkowicie przytaczana już monografia
  Majewskiego???. Jednocześnie poszerzone wydanie \textit{Dialogów} z~roku
  $1972$ jest już świadectwem innego, bardziej zdystansowanego
  spojrzenia na tę tematykę. Błędem byłoby jednak uważać to drugie wydanie,
  tylko i~wyłącznie za wyraz rozczarowania Lema tą nauką i~moment, gdy
  ostatecznie wziął z~nią rozbrat.

  W~przedmowie do tego drugiego wydania dialogów, napisanej w~grudniu
  $1971$ roku, Lem stwierdza z~jednej strony, że~cybernetyka nie spełniła
  tego czego najbardziej po niej oczekiwano, jak choćby zunifikowania
  nauk przyrodniczych i~humanistycznych. Z~drugiej strony, uważa on,
  że~ma ona na swoim koncie realne osiągnięcia.

\end{frame}
% ##################





% ##################
\begin{frame}
  \frametitle{Cybernetyka u~Lema}


  Przykładem obu tych rzeczy jest dla niego komputer, który „nie stał się
  równorzędnym osobowym partnerem człowieka”, ale za~to okazał~się w~wielu
  miejscach niezbędnym narzędziem. Por. str. $6\text{--}7$,
  w~\parencite{Lem-Dialogi-Vol-I-Pub-1996}.

  Okres $1957\text{--}1972$ obejmuje czas powstania większości z~tych
  dzieł literackich Lema, które uważane są za najlepsze. Szczęśliwie więc
  dwa wydania \textit{Dialogów} stanowią bardzo użyteczne punkty graniczne,
  do badania recepcji cybernetyki u~Lema w~tym okresie.

  W~naszej opinii prozy Lema \alert{nie} można sprowadzić do wątków
  cybernetycznych, ale ich zauważenie, pozwala lepiej zrozumieć wiele
  jej elementów. Tutaj lektura \textit{Dialogów} jest kluczowa do ustalenia,
  co Lem wyniósł ze swoich studiów cybernetyki i~które elementy jego
  dzieł mają istotnie cybernetyczny rodowód.

\end{frame}
% ##################





% ##################
\begin{frame}
  \frametitle{Czym jest mózg?}


  Potencjalnych pytań jest tutaj bez liku, wymieńmy kilka z~nich.
  Przykładowo, czy omawiany w~\textit{Dialogach} problem „wskrzeszenia” lub
  skopiowania człowieka, jest

  Kończąc już przegląd cybernetyki, chciałbym postawić problem, który mnie
  intryguje. Chodzi o~ideę zaproponowaną w~okolicach $1960$~roku, wciąż nie
  znam dokładnej daty, przez dwóch cybernetyków,
  \colorhref{https://en.wikipedia.org/wiki/Stafford\_Beer}{Stafforda Beera}
  i~\colorhref{https://en.wikipedia.org/wiki/Gordon\_Pask}{Gordona Paska}.
  Przyjęli oni, że~obok wcześniejszej tezy, że~mózg jest obiektem
  posiadającym zdolność adaptacji do~otoczenia, należy uznać też tezę
  odwrotną: każdy układ który adaptuje~się do otoczenia jest w~jakimś
  sensie mózgiem. Beer na podstawie tego stwierdził, że~w~takim właśnie
  sensie staw wodny jest mózgiem
  \parencite{Pickering-Cybernetics-in-Britain-Ver-2022}. Natomiast
  w~$1961$~roku opublikowany zostaje \textit{Solaris}.

  Lema bardzo zajmował problem przypadków, a~ja chciałbym wiedzieć, czy to
  jest przypadek, czy nie? A~może zwyczajni widzę tutaj podobieństwa,
  których w~rzeczywistości nie ma?

\end{frame}
% ##################






























% ##################
\begin{frame}
  \frametitle{Narodziny cybernetyki}



  \textit{[\ldots] nie ziściła istotnie tego, czego~się po niej najintensywniej
    spodziewano~-- nie stała~się zwłaszcza lekarstwem leczącym naukę
    ze~schorzenia specjalistycznego (a~miała nim zostać jako inter- czy
    wręcz superdyscyplinarna wiedza, unifikując tak przyrodoznawstwo, jak
    humanistykę)~-- urzeczywistniła to, czego nikt~się po niej nie
    spodziewał. Maszyny cyfrowe nie stały~się co prawda równorzędnymi
    osobowymi partnerami człowieka, ale za to okazały~się niezastąpionym
    już dziś narzędziem w~zarządzaniu światową gospodarką; teoria
    informacji nie okazał~się co prawda nowym kamieniem filozoficznym,
    ale~wtargnęła nawet tam gdzie jej nie oczekiwano, np.~do fizyki
    teoretycznej; przykładów takiego rozmijania~się oczekiwań i~spełnień
    można by wyliczyć wiele.}

  Jak już wspomniano wcześniej, pod postacią sztucznej inteligencji,
  cybernetyka jest z~nami do dziś, czy jednak przetrwała jako osobna nauka?
  Wolałbym pozostawić odpowiedź na to pytanie prof.~Bieleckiemu,
  zob.~przykładowo jego artykuł
  \parencite{Bielecki-Cybernetyka-Slowniki-spoleczne-Vol-XIII-Ver-2025}.


  \colorhref{https://en.wikipedia.org/wiki/Cybernetics}{Cybernetykę}
  najlepiej rozumieć przez analogię z~tym czym jest dzisiaj sztuczna
  inteligencja. Tak jak dziś \textsc{ai}, cybernetyka w~latach
  $1945\text{-}1970$ była zarówno dziedziną badań jak i~fenomenem
  kulturowym. Cybernetykę zamierzano zastosować do elektroniki, fizyki,
  chemii, biologi, socjologii, kierowania gospodarką, psychologii, szeroko
  pojętego mistycyzmu (dzieła
  \colorhref{https://pl.wikipedia.org/wiki/Jan_Tr\%C4\%85bka}{Jana
    Trąbki}), ~etc. Toczyły~się intensywne debaty na temat roli i~znaczenia
  cybernetyki, jak też zagrożeń jakie może ona nieść. W~sprzedaży było
  wiele dzieł dla niespecjalistów wprowadzających do~cybernetyki, takich jak
  \textit{Cybernetyka bez matematyki} A.W. Szylejko i~T.I.~Szylejko, czy
  \textit{Dialogi} Stanisława Lema.



  Porównanie to jest tym bardziej uzasadnione, że~do około $1965$ sztuczna
  inteligencja była jednym z~działów cybernetyki, dopiero później
  zaczęła być uważana, za część informatyki. Z~lekkim przymrużeniem oka,
  można więc powiedzieć, że~cybernetyka przeżywa dziś swoją trzecią młodość.

  Około $1970$~roku cybernetyka schodzi z~głównej sceny w~aurze
  niespełnionych obietnic i~dla większości ludzi zostaje zredukowana
  do~historycznej anegdoty. By zilustrować nadzieje i~rozczarowanie związane
  z~tą nauką, przytoczymy słowa Stanisława Lema z~\textit{Przedmowy}
  do~\textit{Dialogów}, napisanej w~grudniu $1971$.

  \textit{Pierwszy szkic [dołączony do tego wydania] to konfrontacja
    poglądów zawartych w~„Dialogach” z~realnym biegiem rzeczy w~ciągu
    szesnastu lat, jakie upłynęły między powstaniem książki [wydanej
    w~$1957$~r.] a~chwilą obecną. Konfrontacja ta ujawnia nie
    tylko moją naiwność; [\ldots]}
  Str.~$6$, \parencite{Lem-Dialogi-Vol-I-Pub-1996}.

      \textit{[\ldots] nie tylko moją naiwność; byłem, jako autor „Dialogów”,
    wyrazicielem poglądu dość
    rozpowszechnionego w~kręgach entuzjastów cybernetyki w~pięćdziesiątych
    latach. Zestawienie opinii pochodzącej z~tych lat ze stanem obecnym jest
    ciekawym przyczynkiem do~historii nauki. Obrazuje mianowicie ową
    ekstrapolacyjną prostolinijność jaką wzbudza w~nauce bodaj każdy jej
    przewrót; perspektywa dalszego postępu wiedzy rysują~się wtedy
    współczesnym tak prosto, jak gdyby ów zawiły, pełen kluczeń i~ślepych
    zaułków ruch poznawania, który doprowadził do kolejnej rewolucji
    w~nauce miał ustać i~ustąpić drogi lawinowemu pomnażaniu wiedzy~--
    właściwie już bez odwrotów i~przeszkód. Regularnie też dochodzi potem
    do rozmijania~się nazbyt optymistycznych nadziei z~rzeczywistością,
    które to zjawisko i~w~odniesieniu do cybernetyki~się sprawdziło. Warto
    dodać, że~reakcją na nie bywa później pesymizm poznawczy, diametralnie
    oponujący wcześniejszy optymizm, jak również, że~taka reakcja przeważnie
    bywa nieporozumieniem: jakkolwiek bowiem cybernetyka nie ziściła
    istotnie tego, [\ldots]}
  Str.~$6$, \parencite{Lem-Dialogi-Vol-I-Pub-1996}.


  \textit{[\ldots] nie tylko moją naiwność; byłem, jako autor „Dialogów”,
    wyrazicielem poglądu dość
    rozpowszechnionego w~kręgach entuzjastów cybernetyki w~pięćdziesiątych
    latach. Zestawienie opinii pochodzącej z~tych lat ze stanem obecnym jest
    ciekawym przyczynkiem do~historii nauki. Obrazuje mianowicie ową
    ekstrapolacyjną prostolinijność jaką wzbudza w~nauce bodaj każdy jej
    przewrót; perspektywa dalszego postępu wiedzy rysują~się wtedy
    współczesnym tak prosto, jak gdyby ów zawiły, pełen kluczeń i~ślepych
    zaułków ruch poznawania, który doprowadził do kolejnej rewolucji
    w~nauce miał ustać i~ustąpić drogi lawinowemu pomnażaniu wiedzy~--
    właściwie już bez odwrotów i~przeszkód. Regularnie też dochodzi potem
    do rozmijania~się nazbyt optymistycznych nadziei z~rzeczywistością,
    które to zjawisko i~w~odniesieniu do cybernetyki~się sprawdziło. Warto
    dodać, że~reakcją na nie bywa później pesymizm poznawczy, diametralnie
    oponujący wcześniejszy optymizm, jak również, że~taka reakcja przeważnie
    bywa nieporozumieniem: jakkolwiek bowiem cybernetyka nie ziściła
    istotnie tego, [\ldots]}
  Str.~$6$, \parencite{Lem-Dialogi-Vol-I-Pub-1996}.

\end{frame}
% ##################





% ##################
\begin{frame}
  \frametitle{\textit{Cybernetics, or Control and
      Communication\ldots}}


  \textit{Cybernetics\ldots} Wienera jest książką osobliwą i~jak sam przyznał,
  napisaną w~niesprzyjających warunkach, co poskutkowało dużo liczbą błędów,
  również bardzo poważnych (zob. str.~xii,
  \parencite{Wiener-Cybernetics-Second-edition-Pub-2016}). Nie udało mi~się
  zdobyć wydania polskiego, korzystam więc z~angielskiej wersji, będącej
  wznowienia wydania drugiego z~$1961$~roku, w~którym książka ta ma około
  $240$~stron.

  Wstępny obraz tej książki można uzyskać poprzez zestawienie tytułów jej
  rozdziałów, ich długości i~liczby wyróżnionych w~tekście wzorów
  matematycznych.

  \textit{Preface to second edition}, $10$ str., $7$~wz. \\
  \textit{Introduction}, $26$~str., $0$~wz. \\
  \textit{Newtonian and Bergsonian Time}, $14$~str., $0$~wz. \\
  \textit{Groups and Statistical Mechanics}, $16$~str., $26$~wz. \\
  \textit{Time Series, Information and Comunications}, $40$~str.,
  $136$~wz. \\
  \textit{Feedback and Oscillation}, $24$~str., $58$~wz. \\

\end{frame}
% ##################





% ##################
\begin{frame}
  \frametitle{\textit{Cybernetics, or Control and
      Communication\ldots}}


  \textit{Computing Machines and Nervous Systems}, $16$~str., $10$~wz. \\
  \textit{Gestalt and Universals}, $10$~str., $1$~wz. \\
  \textit{Cybernetics and Psychopatology}, $10$~str., $0$~wz. \\
  \textit{Information, Laguage and Society}, $10$~str., $0$~wz.

  Dwa następne rozdziały zostały dodane w~wydaniu z~$1961$~roku. \\
  \textit{On~Learning and Self-Reproducing Machines}, $12$~str., $0$~wz. \\
  \textit{Brain Waves and Self-Organizing Systems}, $24$~str., $36$~wz.

  Lektura rozdziałów z~większą liczbą wzorów, wymaga dość dobrej znajomości
  teorii całki, do~której jak wiemy Wiener wniósł ważny wkład. Jednak to
  temat na zupełnie inne spotkanie.

  Rozdział \textit{On~Learning and Self-Reproducing Machines} wygląda
  szczególnie ciekawie w~kontekście tego, że~w~$1964$ Lem publikuje
  \textit{Niezwyciężonego}, więc może porównują te dwa teksty dojdziemy
  do jakiś ciekawszych wniosków? Czas pokaże.

\end{frame}
% ##################





% ##################
\begin{frame}
  \frametitle{Czym jest mózg?}


  Kończąc już przegląd cybernetyki, chciałbym postawić problem, który mnie
  intryguje. Chodzi o~ideę zaproponowaną w~okolicach $1960$~roku, wciąż nie
  znam dokładnej daty, przez dwóch cybernetyków,
  \colorhref{https://en.wikipedia.org/wiki/Stafford\_Beer}{Stafforda Beera}
  i~\colorhref{https://en.wikipedia.org/wiki/Gordon\_Pask}{Gordona Paska}.
  Przyjęli oni, że~obok wcześniejszej tezy, że~mózg jest obiektem
  posiadającym zdolność adaptacji do~otoczenia, należy uznać też tezę
  odwrotną: każdy układ który adaptuje~się do otoczenia jest w~jakimś
  sensie mózgiem. Beer na podstawie tego stwierdził, że~w~takim właśnie
  sensie staw wodny jest mózgiem
  \parencite{Pickering-Cybernetics-in-Britain-Ver-2022}. Natomiast
  w~$1961$~roku opublikowany zostaje \textit{Solaris}.

  Lema bardzo zajmował problem przypadków, a~ja chciałbym wiedzieć, czy to
  jest przypadek, czy nie? A~może zwyczajni widzę tutaj podobieństwa,
  których w~rzeczywistości nie ma?

\end{frame}
% ##################










% ######################################
\section{\textit{Dialogi} Stanisława Lema}
% ######################################



% ##################
\begin{frame}
  \frametitle{Czym są \textit{Dialogi}?}


  Zacznijmy od przytoczenia pewnych ustaleń Pawła Majewskiego z~jego
  \textit{Między zwierzęciem a~maszyną\ldots} Główny wzorem dla Lema przy
  tworzeniu swoich \textit{Dialogów}, wydaje~się być dzieło \textit{Trzy
    dialogi między Hylasem i~Filonousem}, wydane przez prominentnego
  brytyjskiego filozofa
  \colorhref{https://en.wikipedia.org/wiki/George_Berkeley}
  {George’a Berkeleya} w~$1713$~roku. Berkeley wyłożył w~nim swoją
  filozofię, którą nazywał \textbf{immaterlializmem}. Lem zapożyczył od
  Berkeleya zarówno formę dialogu filozoficznego, imiona bohaterów, jak
  i~podstawowy podział na „dobrego” Filonousa i~„złego” Hylasa. Ich imiona
  są znaczące, gdyż „Filonous” oznacza „umysłowy”, „intelektualny”,
  zaś~„Hylas” tłumaczy~się jako~„cielesny” lub „materialny”.

  Dialogów jest łącznie osiem. Pierwszy z~nich powstał w~latach
  $1948\text{-}1950$, pozostałe siedem w~latach $1954\text{-}1956$. Jak sam
  to określił później Lem, poprzez ich stworzenie stał~się publicznym
  wyrazicielem przekonań entuzjastów cybernetyki tamtego czasu.

\end{frame}
% ##################











% ##################
\begin{frame}
  \frametitle{\textit{Dialogi} i~termin „sztuczna
    inteligencja”}


  Jednym z~podstawowych pojęć u~Lem jest przypadek, więc wspomnijmy o~jednym
  z~nich. Wedle mojej wiedzy, termin \textit{sztuczna inteligencja} został
  ukuty $1955$~roku przez czterech gigantów informatyki:
  \colorhref{https://en.wikipedia.org/wiki/John_McCarthy_(computer_scientist)}
  {Johna McCarthy’ego},
  \colorhref{https://pl.wikipedia.org/wiki/Marvin\_Minsky}
  {Marvina Minsky’ego},
  \colorhref{https://en.wikipedia.org/wiki/Nathaniel\_Rochester\_(computer\_scientist)}{Nathaniela Rochestera}
  i~\colorhref{https://en.wikipedia.org/wiki/Claude\_Shannon}
  {Claude’a Shannona}. Termin ten wprowadzili na potrzeby projektu letnich
  warsztatów, które odbył~się w~roku~$1956$ pod nazwą
  \colorhref{https://en.wikipedia.org/wiki/Dartmouth_workshop}
  {\textit{Dartmouth Summer Research Project on Artificial Intelligence}}
  i~są uważany za~jedno z~najważniejszych wydarzeń w~historii tej
  dziedziny. Wedle pewnych świadectw, w~nazwie tej konferencji nie użyto
  słowa „cybernetyka”, ze względu na personalne animozje między niektórymi
  z~organizatorów a~Norbertem Wienerem. Mało profesjonalne, ale jakże
  przecież normalne i~ludzkie.

  Niezależnie od~tego, warsztaty te odbywają~się w~tym samym roku, w~którym
  Lem kończy \textit{Dialogi}, więc może gdyby ukończył je później,
  sam termin „sztuczna inteligencja” byłby obecny w~ich tekście.

\end{frame}
% ##################





% ##################
\begin{frame}
  \frametitle{Treść \textit{Dialogów}}


  \textit{Dialogi} zaczynają~się, gdy Filonous napotyka w~„pięknym parku”
  samotnie rozmyślającego Hylasa. Hylas, w~tym momencie zadeklarowany
  materialista, wyjaśnia mu, że~gdy technologia osiągnie odpowiedni
  poziom rozwoju, to będzie mógł zostać „wskrzeszony” przez idealne
  skopiowanie całego jego ciała. Temat ten dyskutują do mniej więcej
  jednej trzeciej dialogu numer iii, gdzie Hylas uznaje, że~nie potrafi
  sprowadzić świadomości człowieka do~konfiguracji materialnych elementów
  jego ciała. Moje osobiste odczucie jest takie, że~ich rozważania na temat
  relacji świadomości oraz~materii zawierają zbyt dużo luk i~zbyt szybkich
  przeskoków, by można było uważać je za filozoficznie zadowalające, ale
  musimy to odłożyć na bok.

  W~czasie tych rozmów mocno dyskutują problem tego, czy gdyby istniały
  dwie idealne kopie jednego człowieka, to która byłaby prawdziwa
  i~czy człowiek może istnieć jako „zwielokrotniony”. Komediową wersję
  tego problemu znajdujemy w~\textit{Podróży siódmej} Ijona Tichego.

\end{frame}
% ##################





% ##################
\begin{frame}
  \frametitle{Informacja}


  Konsekwencją wywodu o~wymianie pamięci są następujące słowa wypowiadane
  przez Filonous, które zasadniczo są parafrazą tego, co można znaleźć
  w~\textit{Cybernetics\ldots} Wienera. \\
  \textit{Zastanówmy~się mianowicie, co wymieniały mózgi w~naszym obrazowym
    przykładzie, cóż to za rzecz była, za zjawisko, któreśmy nazwali
    skrótowo „ładunkiem pamięci”? Był to, nieprawdaż, całokształt
    strukturalnych zmian, nabytych przez ten mózg podczas jego istnienia,
    to znaczy~-- był to zasób pewnej informacji. Kiedy to słowo padło,
    znaleźliśmy~się w~kardynalnym miejscu naszych rozważań. W~samej rzeczy,
    problem informacji, jej istoty, jej powstania, gromadzenia,
    przechowywania i~użytkowania stanowi właściwą treść cybernetyki
    i~zarazem klucz do zagadek, które przedstawiają systemy typu naszego
    mózgu.} \\
  \textsc{hylas}: \textit{Cóż tak osobliwego przedstawia informacja?} \\
  \textsc{filonous}: \textit{Jest ona czymś wielce osobliwym, mój drogi,
    albowiem nie jest ani materią, ani~energią, choć stanowi zjawisko
    całkiem realne.} Str.~$55$, \parencite{Lem-Dialogi-Vol-I-Pub-1996}.

\end{frame}
% ##################





% ##################
\begin{frame}
  \frametitle{Informacja}


  W~powyższym fragmencie Filonous w~zasadzie zdefiniował cybernetykę
  w~ujęciu Wienera. Również uznanie informacji za równorzędnego partnera
  materii i~energii jest zaczerpnięte od ojca cybernetyki.

  Czym jednak jest informacja? Tutaj Lem napotyka na te same problemy,
  na~które napotkali później cybernetycy, czyli problem z~definicją pojęcia
  informacji. Mówiąc pół żartem, pół serio, wykorzystywana przez nich
  definicja informacji, zawiera za mało informacji o~tym czym jest
  informacja ;).

  Ponownie powołam~się na prof.~Bieleckiego, który stwierdził, iż~dobrze
  znana teoria informacji, jest tak naprawdę teorią \alert{przesyłania}
  informacji, o~samej informacji nie mówi ona zbyt wiele. On sam w~swoich
  pracach na temat
  \colorhref{https://www.youtube.com/watch?v=jaAADDFG8oA}{informacji
    strukturalnej} próbuje wypełnić w~jakimś stopniu tą lukę. Jak zaraz
  zobaczymy, również Lem dostrzegł różne niedomagania stosowanego
  w~cybernetyce pojęcia informacji.

\end{frame}
% ##################





% ##################
\begin{frame}
  \frametitle{Informacja}


  Filonous w~następujący sposób wyjaśnia koncepcję informacji. \\

  \textit{Otóż, wracają teraz do cybernetyki~-- informacja jest odwrotnością
    entropii. Gdy tamta jest miarą bezładu~-- ta mierzy ład.}
  Str.~$57$, \parencite{Lem-Dialogi-Vol-I-Pub-1996}.

  Zestawmy to z~fragmentem wstępu do~\textit{Cybernetics\ldots} Wienera. \\
  \textit{Pojęcie ilości informacji łączy~się bardzo naturalnie
    z~klasycznym pojęciem mechaniki statystycznej, mianowicie z~pojęciem
    \textbf{entropii}. Tak jak informacja zawarta w~systemie jest miarą jego
    ładu, tak entropia systemu jest miarą jego bezładu. Jedna jest po
    prostu odwrotnością drugiej (the~negative~of the~other).}
  Tłum. swobodne, str.~$11\text{--}12$,
  \parencite{Wiener-Cybernetics-Second-edition-Pub-2016}.

\end{frame}
% ##################





% ##################
\begin{frame}
  \frametitle{Informacja}


  Tego typu sposób myślenia o~informacji, zakorzeniło~się mocniej
  w~naukach technicznych. Za przykład niech posłużą cytaty z~książki
  Jerzego Seidlera
  \textit{Nauka o~informacji. Tom~I: Podstawy, modele źródeł i~wstępne
    przetwarzanie informacji}, opublikowanej przez Wydawnictwo Naukowe
  i~Techniczne w~$1983$~roku
  \parencite{Seidler-Nauka-o-informacji-Vol-I-Pub-1983}.

  \textit{Podkreślono z~jednej strony uniwersalny charakter pojęć
    informacji i~sygnału, powiązania dwustronne między nauką o~informacji
    a~teorią systemów, z~drugiej zaś strony omówiono konkretne przykłady
    z~telekomunikacji, miernictwa i~automatyki.} Str.~$4$,
  \parencite{Seidler-Nauka-o-informacji-Vol-I-Pub-1983}.

  \textit{Pojęciem „informacji” posługujemy~się często. Pojęcie to,
    podobnie jak na przykład pojęcie „materia” lub „energia”, ma charakter
    pojęcia pierwotnego i~ścisłe zdefiniowanie go za pomocą prostszych nie
    jest możliwe. Pozostaje więc jedynie wyjaśnienie sensu tego pojęcia,
    odpowiadającego jego intuicyjnemu rozumieniu.} Str.~$21$,
  \parencite{Seidler-Nauka-o-informacji-Vol-I-Pub-1983}.

\end{frame}
% ##################


% ##################
\begin{frame}
  \frametitle{Czym my już o~tym nie czytaliśmy?}


  Czy w~twórczości pisarskiej Lema spotykamy~się z~obiektami, w~których
  widać cybernetyczną koncepcję sieci? Choć mam kilka pomysłów gdzie zacząć
  ich szukać, to mam nadzieję, że~w~tej i~innych kwestiach, Państwa
  znajomość tego pisarza będzie lepsza, niż moja.

  Niezależnie od tego, zanim zakończymy nasze rozważania, warto zwrócić na
  chwilę uwagę na opowiadanie \textit{Ananke}. Badania dr.~Kukulaka
  pokazały jak na tym dziele odbiły~się panujący w~owym czasie, szczęśliwie
  krótkotrwały, pesymizm dotyczą perspektyw badań Układu Słonecznego,
  z~naszego jednak punktu widzenia, warto zwrócić uwagę na inny aspekt tego
  pełnego zniechęcenia opowiadania
  \parencite{Kukulak-Two-Faces-of-Mars-Pub-2023}. Mianowicie, w~opowiadaniu
  tym cierpiący na nerwicę natręctw emerytowany pilot Warren Cornelius
  przetrenowuje sztuczną inteligencję, każąc jej
  nadmiernie konsumować informacji i~unikać proszenia ludzkiego nadzorcy
  o~pomoc, co~prowadzi do katastrofy statku kosmicznego „Ariel”, śmierci
  trzydziestu osób oraz~jego samobójstwa.

\end{frame}
% ##################








% ##################
\begin{frame}
  \frametitle{\textit{Cyberiada}}


  W~tym kontekście Filonous przytacza też drugą zasadę termodynamiki,
  zob.~str.~$57$, \parencite{Lem-Dialogi-Vol-I-Pub-1996}. Literacki
  dziełem Lema wytykającym braki takiej koncepcji informacji jest
  \textit{Wyprawa szósta, czyli jak Trurl i~Klapaucjusz demona drugiego
    rodzaju stworzyli, aby zbójcę Gębona pokonać}, będą częścią,
  nomen omen, \textit{Cyberiady}. Bohaterowie pokonują w~niej Gębona,
  wydobywając z~szumu wypełniającego przestrzeń kosmiczną informacje
  o~średniej ilości piór łabędzia niemego na metr kwadratowy, liczbie
  ziaren w~główce maku, liczbie ziaren piasku jaka dostaje~się do buta
  na~plaży, etc., pod których zalewem przepada zbójca.


  Pisząc to seminarium przyszła mi do głowy hipoteza, że~większość tekstów
  z~\textit{Cyberiady} jest formą rozliczenia~się Lema z~niespełnionymi
  obietnicami cybernetyki. Jeśli jest~się świadomym wielkiej roli
  rachunku prawdopodobieństwa w~tej nauce, można w~ten sposób odczytać
  \textit{Wyprawę trzecią, czyli smoki prawdopodobieństwa}. W~tym momencie
  nie wiem, czy~to jest dobry pomysł, ale wygląda obiecująco.

\end{frame}
% ##################





% ##################
\begin{frame}
  \frametitle{Próg komplikacji minimalnej}


  Ostatnią ważną z~naszego punktu widzenia koncepcją wprowadzoną w~dialogu
  iii, jest pojęcie „progu komplikacji minimalnej”. Według Filonousa, jeśli
  dany układ będzie tak złożony, że~przekroczy próg komplikacji minimalnej,
  wówczas jest w~stanie stworzyć urządzenie tak samo złożone jako on sam.
  Jak mówi \\
  \textit{[\ldots] próg komplikacji minimalnej wyznacza ścisłą, fizykalnie
    dającą~się mierzyć granicę między światem mechanizmów klasycznych
    (maszyn) a~światem \textbf{organizmów}. Zauważ, proszę, iż~nie powiadam
    „a~światem \textbf{żywych} organizmów”. „Życie” jest tu pojęciem
    węższym, a~„organizacja”~-- pojęciem nadrzędnym, szerszym.}
  Str.~$68\text{--}69$, \parencite{Lem-Dialogi-Vol-I-Pub-1996}.

  Te cytaty wskazują na pewną ważną cechę myślenia cybernetycznego, której
  ja nie dostrzegam w~naszym rozumieniu sztucznej inteligencji. Mianowicie,
  w~cybernetyce rozmył~się podział między zwierzęciem, a~maszyną, więc
  w~prowadzonych wtedy rozważaniach, nie pytano tyle o~sztuczną
  inteligencję, co o~\alert{sztuczne życie}.

\end{frame}
% ##################





% ##################
\begin{frame}
  \frametitle{Sztuczna inteligencja vs sztuczne życie}


  Organizm zwierzęcia i~maszyna były z~punktu widzenia cybernetyki po
  prostu dwoma układami, które bazowały na tych samych zasadach przesyłu
  i~przetwarzania informacji, energii i~materii. W~skutek tego nie dzieli
  je fundamentalna, ontologiczna różnica. Takimi samymi układami co
  zwierzęta i~maszyny są zresztą również stawy wodne, uniwersytety czy
  firmy.

  Nie wydaje mi~się, żebyśmy dzisiaj zadawali sobie pytanie, czy komputer
  jest tym samym typem obiektu, jak ludzkie ciało, czy to w~filozoficznych
  rozważaniach, czy też w~filmach fabularnych albo grach video. Raczej
  przyjmujemy domyślnie, że~to są dość odległe byty. Czymś zupełnie innym
  jest pytanie, czy \alert{programy} komputerowe myślą i~czy posiadają
  świadomość? To stawiamy sobie bardzo często.

\end{frame}
% ##################





% ##################
\begin{frame}
  \frametitle{Wiener i~Lem}




\end{frame}
% ##################





% ##################
\begin{frame}
  \frametitle{Wiener i~Lem}


  Jeśli choć pobieżnie przekartkujemy \textit{Cybernetykę i~społeczeństwo}
  Wienera, to znajdziemy tam wiele komentarzy do powyższych problemów, które
  wyglądają jakby wyszły spod pióra Lema. Choć związek przyczynowy był
  raczej w~drugą stronę.



  Rodzi~się więc pytanie, czy Wiener wpłynął swoim poglądami na~filozofię
  Lema? I~czy ten wpływ przetrwał okres zafascynowania pisarza cybernetyką?
  W~tym momencie muszę lepiej przestudiować twórczość Wienera, by móc
  w~tym temacie powiedzieć cokolwiek wartego uwagi. Pytanie to musi więc
  na razie pozostać otwarte.

\end{frame}
% ##################





% ##################
\begin{frame}
  \frametitle{Wróćmy jeszcze do \textit{Dialogów}}


  Dla szerszego kontekstu warto ponownie zacytować książkę \textit{Nauka
    o~informacji} Seidlera. \\
  \textit{Zarysowuje~się też dalej idący proces integracji nauk.
    W~złożonych systemach informacyjnych można bowiem dopatrzyć~się
    coraz więcej podobieństw do~procesów informacyjnych zachodzących
    w~organizmach żywych, a~nawet i~w~społeczeństwie, i~w~rezultacie
    zaczyna~się pojawiać wzajemne oddziaływanie między do~niedawna tak
    odległymi dziedzinami jak: technika, biologia i~nauki społeczne.
    Kluczowym elementem wspólnym dla tych dziedzin jest pojęcie informacji.}
  Str.~$10$,
  \parencite{Seidler-Nauka-o-informacji-Vol-I-Pub-1983}.

  Wróćmy jeszcze na chwilę do~tekstu \textit{Dialogów}. W~dialogach~iv i~v
  dochodzi do przesunięcia stylu dzieła. Coraz mniej są to dialogi w~pełnym
  znaczeniu tego słowa, coraz bardziej zaś wykłady Filonousa o~cybernetyce,
  których samotnym słuchaczem staje~się Hylas.

\end{frame}
% ##################





% ##################
\begin{frame}
  \frametitle{Sieci}


  W~dialogu~iv Filonous i~Hylas rozważają problem świadomości, ewolucji
  darwinowskiej i~jej podobieństwo, bądź jego braku do budowania maszyn,
  podłączenia~się ludzi do cudzych systemów nerwowych, jak również dochodzą
  do wniosku, że~z~„żelaznych brył” można zbudować istotę równie żywą
  jak~koń czy krowa.

  W~dialogu~v wprowadzone zostaje pojęcie sieci, które od tego momentu
  staje~się z~jednym z~podstawowych elementów wykładów Filonousa
  o~cybernetyce. \\
  \textsc{hylas}: \textit{Dobrze. Czy masz zamiar mówić teraz o~zbiorze
    układów zwanych sieciami?} \\
  \textsc{filonous}: \textit{Tak. Zbiór ten obejmuje układy o~stopniu
    złożoności większym bądź równym~„w”. Przez „w” rozumiem minimalną
    złożoność, jaką musi wykazać układ, abyśmy mogli zaliczyć go do
    zbioru.} \\
  Str.~$97$, \parencite{Lem-Dialogi-Vol-I-Pub-1996}.

\end{frame}
% ##################





% ##################
\begin{frame}
  \frametitle{Sieci neuronowe}


  Kilka stron dalej, zob.~str.~$111$,
  \parencite{Lem-Dialogi-Vol-I-Pub-1996}, czytamy już o~sieciach
  neuronowych rozważanych przez
  \colorhref{https://en.wikipedia.org/wiki/Warren\_Sturgis\_McCulloch}
  {Warrena Strugisa McCullocha} ($1898\text{--}1969$)
  i~\colorhref{https://en.wikipedia.org/wiki/Walter_Pitts}
  {Waltera Harry’ego Pittsa Jr.} ($1923\text{--}1969$).
  Autorzy ci w~$1943$
  opublikowali artykuł \colorhref{https://en.wikipedia.org/wiki/A\_Logical\_Calculus\_of\_the\_Ideas\_Immanent\_in\_Nervous\_Activity}
  {\textit{A~Logical Calculus of the Ideas Immanent in Nervous Activity}}
  w~który zaproponowali model neuronu, zbudowanej z~nich sieci, jak również
  przeprowadzają teoretyczną analizę problemu wykonywania na niej obliczeń
  logicznych klasycznego rachunku zdań
  \parencite{Bielecki-Sztuczne-sieci-neuronowe-Slowniki-ETC-Vol-XIII-Ver-2025}.

  Omawiany artykuł uważa~się za~początek badań sieci neuronowych, które
  od~roku~$2012$ przeżywają swoją kolejną młodość. Sami zaś McCulloch
  i~Pitts są pierwszymi cybernetykami, a~może w~ogóle pierwszymi realnie
  istniejącymi osobami, wymienionymi z~imienia i~nazwiska
  w~\textit{Dialogach}.

  % W~dialogu~v wprowadzone zostaje pojęcie sieci, które od tego momentu
  % staje~się z~jednym z~podstawowych elementów wykładów Filonousa
  % o~cybernetyce. \\
  % \textsc{hylas}: \textit{Dobrze. Czy masz zamiar mówić teraz o~zbiorze
  %   układów zwanych sieciami?} \\
  % \textsc{filonous}: \textit{Tak. Zbiór ten obejmuje układy o~stopniu
  %   złożoności większym bądź równym~„w”. Przez „w” rozumiem minimalną
  %   złożoność, jaką musi wykazać układ, abyśmy mogli zaliczyć go do
  %   zbioru.} \\
  % Str.~$97$, \parencite{Lem-Dialogi-Vol-I-Pub-1996}.

  % Kilka stron dalej, zob.~str.~$111$,
  % \parencite{Lem-Dialogi-Vol-I-Pub-1996}, czytamy już o~sieciach
  % neuronowych rozważanych przez
  % \colorhref{https://en.wikipedia.org/wiki/Warren\_Sturgis\_McCulloch}
  % {Warrena Strugisa McCullocha} ($1898\text{--}1969$)
  % i~\colorhref{https://en.wikipedia.org/wiki/Walter_Pitts}
  % {Waltera Harry’ego Pittsa Jr.} ($1923\text{--}1969$). Autorzy ci w~$1943$
  % opublikowali artykuł \textit{A~Logical Calculus of the Ideas Immanent in
  %   Nervous Activity} w~który proponują model neuronu, zbudowanej z~nich
  % sieci, jak również analizują teoretyczną możliwość przeprowadzania za~jej
  % pomocą obliczeń logicznych w~ramach klasycznego rachunku zdań
  % \parencite{Bielecki-Sztuczne-sieci-neuronowe-Slowniki-ETC-Vol-XIII-Ver-2025}.
  % Tym samym McCulloch i~Pitts zapoczątkowali badanie sieci neuronowych,
  % które od roku~$2012$ przeżywają swoją kolejną młodość. Są oni też
  % pierwszymi cybernetykami, a~może w~ogóle pierwszymi istniejącymi osobami,
  % wymienionymi z~nazwiska w~\textit{Dialogach}.

  % Lem przy tym doskonale wie, że~cybernetyczne sieci mogę~się uczyć na
  % podstawie przeszłych doświadczeń, co ilustruje choćby na przykładzie
  % sieci jaką „jest” niemowlę. Pamiętajmy, że~dla cybernetyka
  % $\text{zwierzę} = \text{maszyna}$.

\end{frame}
% ##################





% ##################
\begin{frame}
  \frametitle{Perceptron}


  Lem przy tym doskonale wie, że~cybernetyczne sieci mogę~się uczyć na
  podstawie przeszłych doświadczeń, co ilustruje choćby na przykładzie
  sieci jaką „jest” niemowlę. Pamiętajmy, że~dla cybernetyka
  $\text{zwierzę} = \text{maszyna}$. Znał więc, przynajmniej pobieżnie
  i~we wczesnym stadium rozwoju, pojęcie „sieci neuronowej”, ale według
  mojej obecnej interpretacji, uważał ją tylko za~szczególny przypadek
  ogólnego pojęcia „sieci”, którego używa znacznie częściej w~dialogu~v.

  Warto nadmienić, że~pierwszy układ elektroniczny, działający
  wedle zasad sieci neuronowej skonstruował amerykański psycholog
  \colorhref{https://en.wikipedia.org/wiki/Frank\_Rosenblatt}{Frank
    Rosenblatt} ($1928\text{--}1971$) w~$1957$~roku, przez co bywa nazywany
  „the father~of deep learning”. Jego układ nosił nazwę
  \colorhref{https://en.wikipedia.org/wiki/Perceptron}{\textit{perceptronu}}
  i~na jego projekt duży wpływ miała ówczesna wiedza na temat neuronów
  znajdujących~się w~siatkówce oka
  \parencite{Bielecki-Sztuczne-sieci-neuronowe-Slowniki-ETC-Vol-XIII-Ver-2025}. Oznacza to, że~gdy Lem opisywał sieci w~\textit{Dialogach} były
  one z~jego punktu widzenia koncepcją czysto teoretyczną.

\end{frame}
% ##################











% ##################
\begin{frame}
  \frametitle{Czym my już o~tym nie czytaliśmy?}


  \textit{[\ldots] o~tym, że~jest [komputer sterujący „Ariela”] przeciążony,
    zawiadomił swoją sterownię, to znaczy~-- ludzi „Ariela” dopiero w~$201$
    sekundzie procedury. Już wtedy dławił~się danymi~-- a~żądał wciąż
    nowych.} \\
  \textit{Ananke}, str.~$288$,
  \parencite{Lem-Ogrod-ciemnosci-i-inne-opowiadania-Pub-2017}.

  \textit{Przedstawił [von der Voyt] drogę, jaką przebywa każdy komputer~--
    od montażowej taśmy do~sterowni okrętu. [\ldots] Nie wypełnione jeszcze
    pamięcią [komputery], „nic nie widzące” jak noworodki, jechały do
    Bostonu, gdzie w~zakładach „Syntronics” odbywało~się ich programowanie.
    Po tym kolejnym akcie każdy komputer podlegał procedurze, która jest
    niejako odpowiednikiem nauk szkolnych, gdyż składa~się zarówno
    z~dostarczania pewnych „doświadczeń”, jak i~z~poddawania „egzaminom”.}
  \textit{Ananke}, str.~$290\text{-}291$,
  \parencite{Lem-Ogrod-ciemnosci-i-inne-opowiadania-Pub-2017}.

\end{frame}
% ##################










% ######################################
\section{Zakończenie i~plany na przyszłość}
% ######################################



% ##################
\begin{frame}
  \frametitle{Zakończenie}


  Choć tekst \textit{Dialogów} jest bogatszy, niż zaprezentowana powyżej,
  przedstawione rozważania dobrze oddają stan moich skromnych badań nad nim.
  Jak wspomniałem na początku, jestem dopiero tak w~jednej trzeciej
  planowanych nad cybernetyką i~Lemem, więc wciąż nie umiem odpowiedzieć
  na~większość pytań.

  W~planach mam dokładną lekturę dwóch wymienionych wcześniej dzieł
  Wienera, jak i~zbioru artykułów i~dyskusji \textit{Cybernetyka. Za
    i~przeciw}, który jest świadectwem debaty nad tą dziedziną, jaka
  odbyła~się w~Polsce około roku~$1965$. Temat wymaga też dokładnej lektury
  dzieł Lema oraz prześledzenia znanej chronologii ich powstania.

  Tematów do badań na pewno nie brakuje.

\end{frame}
% ##################










% ######################################
\appendix
% ######################################





% ######################################
\EndingSlide{Dziękuję! Pytania?}
% ######################################









% ####################################################################
% ####################################################################
% Bibliography

\printbibliography










% ############################

% Koniec dokumentu
\end{document}
