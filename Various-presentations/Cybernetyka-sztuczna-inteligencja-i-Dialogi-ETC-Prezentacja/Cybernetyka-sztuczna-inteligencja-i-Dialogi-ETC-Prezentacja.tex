% ------------------------------------------------------------------------------------------------------------------
% Basic configuration of Beamera class and Jagiellonian theme
% ------------------------------------------------------------------------------------------------------------------
\RequirePackage[l2tabu, orthodox]{nag}



\ifx\PresentationStyle\notset
  \def\PresentationStyle{dark}
\fi



% Options: t -- align frame text to the top
\documentclass[10pt,t]{beamer}
\mode<presentation>
\usetheme[style=\PresentationStyle]{jagiellonian}





% ------------------------------------------------------------------------------------
% Procesing configuration files of Jagiellonian theme located in
% the directory "preambule"
% ------------------------------------------------------------------------------------
% Configuration for polish language
% Need description
\usepackage[polish]{babel}
% Need description
\usepackage[MeX]{polski}



% ------------------------------
% Better support of polish chars in technical parts of PDF
% ------------------------------
\hypersetup{pdfencoding=auto,psdextra}

% Package "textpos" give as enviroment "textblock" which is very usefull in
% arranging text on slides.

% This is standard configuration of "textpos"
\usepackage[overlay,absolute]{textpos}

% If you need to see bounds of "textblock's" comment line above and uncomment
% one below.

% Caution! When showboxes option is on significant ammunt of space is add
% to the top of textblock and as such, everyting put in them gone down.
% We need to check how to remove this bug.

% \usepackage[showboxes,overlay,absolute]{textpos}



% Setting scale length for package "textpos"
\setlength{\TPHorizModule}{10mm}
\setlength{\TPVertModule}{\TPHorizModule}


% ---------------------------------------
% TikZ
% ---------------------------------------
% Importing TikZ libraries
\usetikzlibrary{arrows.meta}
\usetikzlibrary{positioning}





% % Configuration package "bm" that need for making bold symbols
% \newcommand{\bmmax}{0}
% \newcommand{\hmmax}{0}
% \usepackage{bm}




% ---------------------------------------
% Packages for scientific texts
% ---------------------------------------
% \let\lll\undefined  % Sometimes you must use this line to allow
% "amsmath" package to works with packages with packages for polish
% languge imported
% /preambul/LanguageSettings/JagiellonianPolishLanguageSettings.tex.
% This comments (probably) removes polish letter Ł.
\usepackage{amsmath}  % Packages from American Mathematical Society (AMS)
\usepackage{amssymb}
\usepackage{amscd}
\usepackage{amsthm}
\usepackage{siunitx}  % Package for typsetting SI units.
\usepackage{upgreek}  % Better looking greek letters.
% Example of using upgreek: pi = \uppi


\usepackage{calrsfs}  % Zmienia czcionkę kaligraficzną w \mathcal
% na ładniejszą. Może w innych miejscach robi to samo, ale o tym nic
% nie wiem.










% ---------------------------------------
% Packages written for lectures "Geometria 3D dla twórców gier wideo"
% ---------------------------------------
% \usepackage{./ProgramowanieSymulacjiFizykiPaczki/ProgramowanieSymulacjiFizyki}
% \usepackage{./ProgramowanieSymulacjiFizykiPaczki/ProgramowanieSymulacjiFizykiIndeksy}
% \usepackage{./ProgramowanieSymulacjiFizykiPaczki/ProgramowanieSymulacjiFizykiTikZStyle}





% !!!!!!!!!!!!!!!!!!!!!!!!!!!!!!
% !!!!!!!!!!!!!!!!!!!!!!!!!!!!!!
% EVIL STUFF
\if\JUlogotitle1
\edef\LogoJUPath{LogoJU_\JUlogoLang/LogoJU_\JUlogoShape_\JUlogoColor.pdf}
\titlegraphic{\hfill\includegraphics[scale=0.22]
{./JagiellonianPictures/\LogoJUPath}}
\fi
% ---------------------------------------
% Commands for handling colors
% ---------------------------------------


% Command for setting normal text color for some text in math modestyle
% Text color depend on used style of Jagiellonian

% Beamer version of command
\newcommand{\TextWithNormalTextColor}[1]{%
  {\color{jNormalTextFGColor}
    \setbeamercolor{math text}{fg=jNormalTextFGColor} {#1}}
}

% Article and similar classes version of command
% \newcommand{\TextWithNormalTextColor}[1]{%
%   {\color{jNormalTextsFGColor} {#1}}
% }



% Beamer version of command
\newcommand{\NormalTextInMathMode}[1]{%
  {\color{jNormalTextFGColor}
    \setbeamercolor{math text}{fg=jNormalTextFGColor} \text{#1}}
}


% Article and similar classes version of command
% \newcommand{\NormalTextInMathMode}[1]{%
%   {\color{jNormalTextsFGColor} \text{#1}}
% }




% Command that sets color of some mathematical text to the same color
% that has normal text in header (?)

% Beamer version of the command
\newcommand{\MathTextFrametitleFGColor}[1]{%
  {\color{jFrametitleFGColor}
    \setbeamercolor{math text}{fg=jFrametitleFGColor} #1}
}

% Article and similar classes version of the command
% \newcommand{\MathTextWhiteColor}[1]{{\color{jFrametitleFGColor} #1}}





% Command for setting color of alert text for some text in math modestyle

% Beamer version of the command
\newcommand{\MathTextAlertColor}[1]{%
  {\color{jOrange} \setbeamercolor{math text}{fg=jOrange} #1}
}

% Article and similar classes version of the command
% \newcommand{\MathTextAlertColor}[1]{{\color{jOrange} #1}}





% Command that allow you to sets chosen color as the color of some text into
% math mode. Due to some nuances in the way that Beamer handle colors
% it not work in all cases. We hope that in the future we will improve it.

% Beamer version of the command
\newcommand{\SetMathTextsColor}[2]{%
  {\color{#1} \setbeamercolor{math text}{fg=#1} #2}
}


% Article and similar classes version of the command
% \newcommand{\SetMathTextColor}[2]{{\color{#1} #2}}










% ---------------------------------------
% Commands for setting background pictures for some slides
% ---------------------------------------
\newcommand{\TitleBackgroundPicture}
{./PresentationPictures/CommonPictures/Cute_dragon_BG_dark.png}
\newcommand{\SectionBackgroundPicture}
{./PresentationPictures/CommonPictures/Cute_dragon_small_BG_light.png}



\newcommand{\TitleSlideWithPicture}{
  \begingroup

  \usebackgroundtemplate{ % \hspace*{-11.5em}
    \includegraphics[height=\paperheight]{\TitleBackgroundPicture}}

  \maketitle

  \endgroup
}





\newcommand{\SectionSlideWithPicture}[1]{%
  \begingroup

  \usebackgroundtemplate{ % \hspace*{-11.5em}
    \includegraphics[height=\paperheight]{\SectionBackgroundPicture}}

  \setbeamercolor{titlelike}{fg=normal text.fg}

  \section{#1}

  \endgroup
}





\newcommand{\EndingSlide}[1]{%
  \begin{frame}[standout]

    \begingroup

    \color{jFrametitleFGColor}

    #1

    \endgroup

  \end{frame}
}










% ------------------------------------------------------------------------------------
% Importing packages, libraries and setting their configuration
% ------------------------------------------------------------------------------------





% ------------------------------------------------------
% BibLaTeX
% ------------------------------------------------------
% Package biblatex, with biber as its backend, allow us to handle
% bibliography entries that use Unicode symbols outside ASCII.
\usepackage[
language=polish,
backend=biber,
style=alphabetic,
url=false,
eprint=true,
]{biblatex}

\addbibresource{Cybernetyka-sztuczna-inteligencja-i-Dialogi-ETC-Bibliography.bib}





% ------------------------------------------------------
% Wonderful package PGF/TikZ
% ------------------------------------------------------

% Node and pics for drawing charts
% \usepackage{./Local-packages/PGF-TikZ-Chart-nodes-and-pics}

% Styles for arrows
% \usepackage{./Local-packages/PGF-TikZ-Arrows-styles}





% ------------------------------------------------------
% Local packages
% ------------------------------------------------------
% Special configuration for this particular presentation
\usepackage{./Local-packages/local-settings}

% Package containing various command useful for working with a text
\usepackage{./Local-packages/general-commands}

% Package containing commands and other code useful for working with
% mathematical text
% \usepackage{./Local-packages/math-commands}










% ------------------------------------------------------------------------------------------------------------------
\title{Cybernetyka, sztuczna inteligencja i~„Dialogi”
  Stanisława Lema}

\author{Kamil Ziemian \\
  \email}


% \institute{Uniwersytet Jagielloński w~Krakowie}

\date[12 III~2025~r.]{12 III~2025~r.}
% ------------------------------------------------------------------------------------------------------------------









% ####################################################################
\begin{document}
% ####################################################################





% ######################################
% Number of chars: 47k+, 10k+,
% Text is adjusted to the left and words are broken at the end of the line.
\RaggedRight
% ######################################





% ######################################
\maketitle
% ######################################





% ######################################
\begin{frame}
  \frametitle{Plan prezentacji}


  \tableofcontents

\end{frame}
% ######################################










% ######################################
\section{Wstęp}
% ######################################



% ##################
\begin{frame}
  \frametitle{Skąd zainteresowanie tym problemem?}


  Aby uniknąć nieporozumień, moje bardzo skromne badania nad wpływem
  cybernetyki na~twórczość Lema, nie wynikły z~fascynacji tym pisarzem.
  Wszystko zaczęło~się od rozmów z~dwoma pasjonatami tego twórcy,
  \colorhref{https://historia.uj.edu.pl/instytut/pracownicy/marcin-jarzabek}
  {dr.~Marcinem Jarząbkiem}
  i~\colorhref{https://incet.uj.edu.pl/maciej-prochnicki}{dr.~Maciejem
    Próchnickim}. Dzięki nim dowiedziałem~się tym, iż~Lem przeżył
  okres zafascynowania tą dziedziną wiedzy. Ponieważ jak już wtedy
  wiedziałem, cybernetyka była od początku nauką albo programem badawczym
  mocno korzystającym z~matematyki, osiągnięć nauk przyrodniczych
  i~techniki, szybko stanął przede mną problem tego, jak bardzo środowisko
  badaczy literatury zgłębiło ten ezoteryczny z~ich punktu widzenia temat?

  Wedle mojej wiedzy na dzień dzisiejszy istnieje tylko jedna duża pozycja
  analizując cybernetykę i~jej wpływ na Lema, jest książka Pawła Majewskiego
  \textit{Między zwierzęciem a~maszyną. Utopia technologiczna Stanisława
    Lema} z~roku~$2007$. Niestety, nie udało mi~się jeszcze zdobyć jej
  egzemplarza i~mam o~niej tylko informacje z~drugiej ręki.

\end{frame}
% ##################





% ##################
\begin{frame}
  \frametitle{Stan na dzisiaj}


  Główne pytanie na które chcę obecnie znaleźć możliwie dobrą odpowiedź,
  jest to, czy Lem zaczerpnął pewne idee do swoich dzieł literackich z~prac
  cybernetyków, a~jeśli tak to jakie? W~grudniu zeszłego roku stanął przede
  mną dodatkowy problem badawczy, mianowicie taki, że~publicystyka Lema
  wydaje~się bardzo zaniedbana przez badaczy, którzy skupili~się na jego
  działalności literackiej i~kilku większych dziełach eseistycznych. Nie wiem
  czy będę~się tym zajmował.

  Moje skromne badania nad Lemem ogromnie dużo zawdzięczają
  \colorhref{https://skos.agh.edu.pl/osoba/szymon-kukulak-10650.html}
  {dr.~Szymonowi Kukulakowi} z~Wydziału Humanistycznego \textsc{agh}, który
  gościł u~Państwa w~grudniu. Ostatnio jestem też w~kontakcie
  z~\colorhref{https://skos.agh.edu.pl/osoba/andrzej-bielecki-7555.html}
  {prof.~Andrzejem Bieleckim} z~Wydziału Elektrotechniki, Automatyki,
  Informatyki i~Inżynierii Biomedycznej \textsc{agh}. Prof. Bielecki określa
  siebie jako „informatyka, matematyka, cybernetyka i~filozofa nauki”, zna
  też dogłębnie twórczość Lema, jemu również zawdzięczam ogromnie wiele
  cennych informacji.

\end{frame}
% ##################





% ##################
\begin{frame}
  \frametitle{Stan na dzisiaj}


  Mam nadzieję, że~w~tym roku~ja, prof.~Bielecki i~dr.~Kukulak nawiążemy
  współpracę i~wspólnie spróbujemy zbadać pewne związki odkryć naukowych
  i~twórczość Lema. Powinienem też dodać, że~obecnie jestem najwyżej
  w~jednej trzeciej zaplanowanych, skromnych badań nad związkami Lema
  i~cybernetyki, więc na razie jestem w~stanie sformułować pewne problemy,
  ale~nie znam żadnych odpowiedzi.

  Dodam, że~prof.~Bielecki obecnie zajmuje~się problemem sformułowania
  teorii informacji wychodzącej poza
  \colorhref{https://en.wikipedia.org/wiki/Information\_theory}{teorię
    Shannona} i~jak nam powiedział, gdy mieliśmy okazję go spotkać,
  inspiracje do nich zaczerpnął po części od~Lema. Mogą Państwo zwrócić~się
  do niego z~prośbą, czy nie chciałby tutaj opowiedzieć o~swoich przygodach
  lemologicznych.

\end{frame}
% ##################








% ######################################
\section{Czym jest cybernetyka?}
% ######################################


% ##################
\begin{frame}
  \frametitle{Czym jest cybernetyka?}


  \colorhref{https://en.wikipedia.org/wiki/Cybernetics}{Cybernetykę}
  najlepiej rozumieć przez analogię z~tym czym jest dzisiaj sztuczna
  inteligencja. Tak jak dziś \textsc{ai}, cybernetyka w~latach
  $1945\text{-}1970$ była zarówno dziedziną badań, jak i~fenomenem
  kulturowym. Cybernetykę zamierzano zastosować do elektroniki, fizyki,
  chemii, biologi, socjologii, kierowania gospodarką, psychologii, szeroko
  pojętego mistycyzmu (dzieła
  \colorhref{https://pl.wikipedia.org/wiki/Jan_Tr\%C4\%85bka}{Jana
    Trąbki}), ~etc. Toczyły~się intensywne debaty na temat roli i~znaczenia
  cybernetyki, jak też zagrożeń jakie może ona nieść. W~sprzedaży było
  wiele dzieł dla niespecjalistów wprowadzających do~cybernetyki, takich jak
  \textit{Cybernetyka bez matematyki} A.W. Szylejko i~T.I.~Szylejko, czy
  \textit{Dialogi} Stanisława Lema.

  Gdy chodzi o~wpływ na kulturę to niestety na dzień dzisiejszy muszę~się
  ograniczyć do dwóch znanych mi jawnych, przykładów. W~powieści
  \textit{Eden} Lema jeden z~bohaterów to Cybernetyk. Drugim jest sowiecki
  zbiór opowiadań \textit{Miłość i~cyberentyka} Susanny Gieorgijewksiej,
  w~Polsce wydany w~$1975$.

\end{frame}
% ##################





% ##################
\begin{frame}
  \frametitle{Co~się stało z~cybernetyką?}


  Porównanie to jest tym bardziej uzasadnione, że~do około $1965$ sztuczna
  inteligencja była jednym z~działów cybernetyki, dopiero później
  zaczęła być uważana, za część informatyki. Z~lekkim przymrożeniem oka,
  można więc powiedzieć, że~cybernetyka przeżywa dziś swoją trzecią młodość.

  Około $1970$~roku cybernetyka schodzi z~głównej sceny w~aurze
  niespełnionych obietnic i~dla większości ludzi zostaje zredukowana
  do~historycznej anegdoty. By zilustrować nadzieje i~rozczarowanie związane
  z~tą nauką, przytoczymy słowa Stanisława Lema z~\textit{Przedmowy}
  do~\textit{Dialogów}, napisanej w~grudniu $1971$.

  \textit{Pierwszy szkic [dołączony do tego wydania] to konfrontacja
    poglądów zawartych w~„Dialogach” z~realnym biegiem rzeczy w~ciągu
    szesnastu lat, jakie upłynęły między powstaniem książki a~chwilą
    obecną. Konfrontacja ta ujawnia nie tylko moją naiwność; [\ldots]}
  Str.~$6$ \parencite{Lem-Dialogi-Vol-I-Pub-1996}.

\end{frame}
% ##################





% ##################
\begin{frame}
  \frametitle{Co~się stało z~cybernetyką?}


  \textit{[\ldots] nie tylko moją naiwność; byłem, jako autor „Dialogów”,
    wyrazicielem poglądu dość
    rozpowszechnionego w~kręgach entuzjastów cybernetyki w~pięćdziesiątych
    latach. Zestawienie opinii pochodzącej z~tych lat ze stanem obecnym jest
    ciekawym przyczynkiem do~historii nauki. Obrazuje mianowicie ową
    ekstrapolacyjną prostolinijność jaką wzbudza w~nauce bodaj każdy jej
    przewrót; perspektywa dalszego postępu wiedzy rysuje~się wtedy
    współczesnym tak prosto, jak gdyby ów zawiły, pełen kluczeń i~ślepych
    zaułków ruch poznawania, który doprowadził do następnej rewolucji
    w~nauce miał ustać i~ustąpić drogi leniwemu pomnażaniu wiedzy~--
    właściwie już bez odwrotów i~przeszkód. Regularnie też dochodzi potem
    do rozmijania~się nazbyt optymistycznych nadziei z~rzeczywistością,
    które to zjawisko i~w~odniesieniu do cybernetyki~się sprawdziło. Warto
    dodać, że~reakcją na nie bywa później pesymizm poznawczy, diametralnie
    oponujący wcześniejszy optymizm, jak również, że~taka reakcja przeważnie
    bywa nieporozumieniem: jakkolwiek bowiem cybernetyka nie ziściła
    istotnie tego [\ldots]}
  Str.~$6$ \parencite{Lem-Dialogi-Vol-I-Pub-1996}.

\end{frame}
% ##################





% ##################
\begin{frame}
  \frametitle{Co~się stało z~cybernetyką?}


  \textit{[\ldots] nie ziściła istotnie tego, czego~się po niej najintensywniej
    spodziewano~-- nie stała~się
    zwłaszcza lekarstwem leczącym naukę ze schorzenia specjalistycznego
    (a~miała nim zostać jako nauka super- czy
    wręcz superdyscyplinarna wiedza, unifikując tak przyrodoznawstwo, jak
    humanistykę)~-- urzeczywistniła to, czego nikt~się po niej nie
    spodziewał. Maszyny cyfrowe nie stały~się co prawda równorzędnymi
    osobowymi partnerami człowieka, ale za to okazały~się niezastąpionym już
    dziś narzędziem w~zarządzaniu światową gospodarką; teoria informacji nie
    okazał~się co prawda nowym kamieniem filozoficznym, ale wtargnęła nawet
    tam gdzie jej nie oczekiwano, np.~do fizyki teoretycznej; przykładów
    takiego rozmijania~się oczekiwań i~spełnień można by wyliczać wiele.}
  Str.~$6\text{--}7$ \parencite{Lem-Dialogi-Vol-I-Pub-1996}.

  Jak już wspomniano wcześniej, pod postacią sztucznej inteligencji,
  cybernetyka jest z~nami do dziś, czy jednak przetrwała jako nauka?
  Wolałbym pozostawić odpowiedź na to pytanie prof.~Bieleckiemu, zob.
  przykładowo jego artykuł
  \parencite{Bielecki-Cybernetyka-Slowniki-spoleczne-Vol-XIII-Ver-2025}.

\end{frame}
% ##################










% ######################################
\section{Norbert Wiener, wybitny naukowiec i~celebryta}
% ######################################



% ##################
\begin{frame}
  \frametitle{Norbert Wiener, ojciec cybernetyki}


  \begin{figure}

    \centering


    \includegraphics[scale=0.23]
    {./Presentation-pictures/Norbert-Wiener-01.jpg}


    \caption{\colorhref{https://en.wikipedia.org/wiki/Norbert\_Wiener}
      {Norbert Wiener} ($1894\text{--}1964$), amerykański filozof,
      informatyk i~matematyk, ojciec cybernetyki.}


    \label{fig:Wiener-01}

  \end{figure}

\end{frame}
% ##################





% ##################
\begin{frame}
  \frametitle{Norbert Wiener, ojciec cybernetyki}


  Norbert Wiener dokonał wielu wartościowych odkryć, z~kilku dziedzin.
  W~tym miejscu warto wspomnieć, że~sformułował on opis ruchów Browna,
  który dziś nazywamy
  \colorhref{https://en.wikipedia.org/wiki/Wiener\_process}{procesem
    Wienera}, który to process indukuje odpowiednią miarę Wienera
  na~przestrzeni funkcji ciągłych.

  W~roku $1948$ publikuje książkę
  \colorhref{https://en.wikipedia.org/wiki/Cybernetics:\_Or\_Control\_and\_Communication\_in\_the\_Animal\_and\_the\_Machine}
  {\textit{Cybernetics: Or~control and~Communication in the~Animal and
      the~Machine}}, tekst założycielski cybernetyki. Książka ta w~trudny
  do~zrozumienia dla mnie sposób stała~się fenomenem kulturowym, który
  najlepiej chyba oddają słowa z~recenzji tej książki, która ukazał
  w~piśmie \textit{Business Week} z~$1949$ roku: \textit{Pod jednym
    względem książka Wienera jest jak Raport Kinsey’a: reakcja
    społeczeństwa na nią jest tak samo ważna, jak jej zawartość.}, cyt.
  za~str.~$408$
  \parencite{Hamilton-The-Charismatic-Cultural-Life-of-ETC-Pub-2017}). Sam
  zaś Wiener uzyskał dzięki niej status celebryty, który to temat jest
  szerzej omawia Sheryl N.~Hamilton w~swym artykule
  \parencite{Hamilton-The-Charismatic-Cultural-Life-of-ETC-Pub-2017}.

\end{frame}
% ##################





% ##################
\begin{frame}
  \frametitle{Narodziny cybernetyki}


  \textit{Cybernetics} opublikowana zostaje w~przekładzie polskim najpóźniej
  w~$1971$~roku. W~$1950$ Wiener publikuje mniejszą, bardzie popularną
  książkę \colorhref{https://en.wikipedia.org/wiki/The\_Human\_Use\_of\_Human\_Beings}
  {\textit{The Human Use of Human Beings}}, w~Polsce wydana jako
  \textit{Cybernetyka i~społeczeństwo} w~$1960$~roku.

  Tak jak komputery, cybernetyka wyrosła w~dużej mierze z~rozwoju naukowego
  i~technologicznego, który dokonał~się podczas II Wojny Światowej. Sam
  Wiener wspomina rolę jaką jego praca nad zautomatyzowanymi działami
  przeciwlotniczymi (ang.~\textit{anti-aircraft predictor}) odegrały
  w~powstaniu cybernetyki (por. str.~xii
  \parencite{Wiener-Cybernetics-Second-edition-Pub-2016}).

  Problem przed jaki stanęli był następujący. Ze względu na wysokość lotu
  samolotów, ich szybkość i~prędkość lotu pocisków przeciwlotniczych, jeśli
  byśmy wycelowali działem przeciwlotniczym idealnie w~samolot, to od
  wystrzelenia pocisku do~dotarcia do celu minie kilka sekund, więc ten
  samolot najpewniej już opuścił miejsce, w~które wymierzyliśmy.

\end{frame}
% ##################





% ##################
\begin{frame}
  \frametitle{Narodziny cybernetyki}


  Jak dobrze wiedzą ludzie grający w~gry wideo, w~takiej sytuacji, nie
  należy celować nie w~samolot, lecz w~przewidywane przez nas
  położenie samolotu za~kilka sekund. Urządzenia przy których budowie
  pracował Wiener, miało właśnie pomóc artylerzyście znaleźć to przyszłe
  położenie samolotu.

  Z~przedmowy do drugiego wydania, który Wiener napisał w~marcu $1961$ roku
  widać, że~miał on już do dyspozycji wszystkie podstawowe pojęcia tego
  co my dziś nazywamy \textit{uczeniem maszynowym} (ang.~\textit{machine
    learning}), a~najpóźniej w~roku~$1957$ jemu i~jego współpracownikom
  była znana idea, która pozwoliłaby działu przeciwlotniczemu uczyć~się
  strzelać do samolotów na podstawie dostarczonych danych. Wiener stwierdza
  wręcz jasno, że~idea uczącej się maszyny, jest tak stara jak sama
  cybernetyka (por. str.~xii
  \parencite{Wiener-Cybernetics-Second-edition-Pub-2016}).

  Niemniej na dzień dzisiejszy, większość tych problemów wymaga z~mojej
  strony dalszych badań.

\end{frame}
% ##################





% ##################
\begin{frame}
  \frametitle{Narodziny cybernetyki}


  W~przedmowie tej można też można znaleźć też dowody na to,
  że~elektronika wywarła bardzo silny wpływ na cybernetykę, co będzie
  przejawia~się później w~stosach diagramów wypełniających książki do
  cybernetyki, a~które przypominają obwody elektryczne.

  Warto też przytoczyć słowa Wienera, wypowiedziane w~dobrze nam znany
  apokaliptycznym tonie:
  \textit{Automaty (\textit{The~automata}) które pierwsze wydanie tylko
    przewidywało, dziś są rzeczywistością i~związane z~nimi zagrożenia dla
    społeczeństwa (\textit{social dangers}) przed którymi ostrzegałem, są
    aż~nadto widoczne.} Tłum. swobodne, str.~vii
  \parencite{Wiener-Cybernetics-Second-edition-Pub-2016}.

  % Prawem kontrapunktu, warto w~tym miejscu zauważyć, że~choć Wiener był
  % głównie matematykiem, to w~pierwszym pokoleniu informatyków było wielu
  % psychiatrów i~neurofizjologów. Za przykład niech posłuży
  % \colorhref{https://en.wikipedia.org/wiki/W._Ross_Ashby}{Williams Ross
  %   Ashby} ($1903\text{-}1972$), który w~roku $1930$ rozpoczął badania
  % klincze jako psychiatra i~którego mógł spotkać Ion Tichy, gdy ten
  % z~niewiadomych powodów ukrywał~się pod nazwiskiem prof.~Corcorana.
  % Niemniej to na razie tylko podejrzenia.

\end{frame}
% ##################




% ##################
\begin{frame}
  \frametitle{Narodziny cybernetyki}


  % W~przedmowie tej można też można znaleźć też dowody na to,
  % że~elektronika wywarła bardzo silny wpływ na cybernetykę, co będzie
  % przejawia~się później w~stosach diagramów wypełniających książki do
  % cybernetyki, a~które przypominają obwody elektryczne.

  % Warto też przytoczyć słowa Wienera, wypowiedziane w~dobrze nam znany
  % apokaliptycznym tonie:
  % \textit{Automaty (\textit{The~automata}) które pierwsze wydanie tylko
  %   przewidywało, dziś są rzeczywistością i~związane z~nimi zagrożenia dla
  %   społeczeństwa (\textit{social dangers}) przed którymi ostrzegałem, są
  %   aż~nadto widoczne.} Tłum. swobodne, str.~vii
  % \parencite{Wiener-Cybernetics-Second-edition-Pub-2016}.

  Prawem kontrapunktu, warto w~tym miejscu zauważyć, że~choć Wiener był
  głównie matematykiem, to w~pierwszym pokoleniu informatyków było wielu
  fizjologów, psychiatrów i~neurofizjologów. Już w~pierwszej połowie lat
  $40$-tych XX wieku Wiener współpracował
  z~\colorhref{https://en.wikipedia.org/wiki/Arturo_Rosenblueth}
  {Arturo Rosenbluethem Stearnsem} ($1900\text{-}1970$), meksykańskim
  lekarzem i~fizjologiem, który później wniósł duży wkład w~rozwój
  cybernetyki i~któremu Wiener zadedykował swoją książkę
  \textit{Cybernetics\ldots} Ten fakt prawie na~pewno miało duży wpływ na
  rozważania dotyczące życia i~pracy mózgu, które podjął Wiener i~jego
  następcy cybernetycy.

  Duża obecność fizjologów i~psychiatrów, była najmocniejsza chyba wśród.
  szczególnie mocna wśród cybernetyków brytyjskich. Za przykład niech
  posłuży \colorhref{https://en.wikipedia.org/wiki/W._Ross_Ashby}{Williams
    Ross Ashby} ($1903\text{-}1972$), który w~roku $1930$ rozpoczął badania
  klincze jako psychiatra i~którego chyba spotkał Ion Tichy, gdy Ashby
  z~niewiadomych powodów ukrywał~się pod nazwiskiem
  prof.~Corcorana. Niemniej to na razie tylko przypuszczenia.

\end{frame}
% ##################





% ##################
\begin{frame}
  \frametitle{Homeostaty Ashby’ego}


  \begin{figure}

    \centering


    \includegraphics
    {./Presentation-pictures/W\_Ross\_Ashbys\_1948\_Homeostat.jpeg}


    \caption{\colorhref{https://en.wikipedia.org/wiki/Homeostat}
      {Homeostat} skonstruowane przez W.~Rossa Ashby’ego w~$1948$~roku.}


    \label{fig:Homeostat-01}

  \end{figure}

\end{frame}
% ##################





% ##################
\begin{frame}
  \frametitle{\textit{Cybernetics, or Control and
      Communication\ldots}}


  \textit{Cybernetics} Wienera jest książką osobliwą i~jak sam przyznał
  pisaną w~niesprzyjających warunkach, co poskutkowało dużo liczbą błędów,
  również bardzo poważnych (por. str.~xii
  \parencite{Wiener-Cybernetics-Second-edition-Pub-2016}). Nie udało mi~się
  zdobyć wydania polskiego, korzystam więc z~angielskiej wersji wydania
  drugiego z~$1961$~roku, w~którym książka ta ma około $240$~stron.

  Wstępny obraz tej książki można uzyskać poprzez zestawienie tytułów jej
  rozdziałów, ich długości i~liczby wyróżnionych w~tekście wzorów
  matematycznych.

  \textit{Preface to second edition}, $10$ str., $7$~wz. \\
  \textit{Introduction}, $26$~str., $0$~wz. \\
  \textit{Newtonian and Bergsonian Time}, $14$~str., $0$~wz. \\
  \textit{Groups and Statistical Mechanics}, $16$~str., $26$~wz. \\
  \textit{Time Series, Information and Comunications}, $40$~str.,
  $136$~wz. \\
  \textit{Feedback and Oscillation}, $24$~str., $58$~wz. \\

\end{frame}
% ##################





% ##################
\begin{frame}
  \frametitle{\textit{Cybernetics, or Control and
      Communication\ldots}}


  \textit{Computing Machines and Nervous Systems}, $16$~str., $10$~wz. \\
  \textit{Gestalt and Universals}, $10$~str., $1$~wz. \\
  \textit{Cybernetics and Psychopatology}, $10$~str., $0$~wz. \\
  \textit{Information, Laguage and Society}, $10$~str., $0$~wz.

  Dwa następne rozdziały zostały dodane w~wydaniu z~$1961$~roku. \\
  \textit{On~Learning and Self-Reproducing Machines}, $12$~str., $0$~wz. \\
  \textit{Brain Waves and Self-Organizing Systems}, $24$~str., $36$~wz.

  Lektura rozdziałów z~większą liczbą wzorów, wymaga dość dobrej znajomości
  teorii całki, do~której jak wiemy Wiener wniósł ważny wkład. Jednak to
  temat na zupełnie inne spotkanie.

  Rozdział \textit{On~Learning and Self-Reproducing Machines} wygląda
  szczególnie ciekawie w~kontekście tego, że~w~$1964$ Lem publikuje
  \textit{Niezwyciężonego}, więc może porównują te dwa teksty dojdziemy
  do jakiś ciekawszych wniosków? Czas pokaże.

\end{frame}
% ##################





% ##################
\begin{frame}
  \frametitle{Czym jest mózg?}


  Kończąc już przegląd cybernetyki, chciałbym postawić problem, który mnie
  intryguje. Chodzi o~ideę zaproponowaną w~okolicach $1960$~roku, wciąż nie
  znam dokładnej daty, przez dwóch cybernetyków,
  \colorhref{https://en.wikipedia.org/wiki/Stafford\_Beer}{Stafforda Beer’a}
  i~\colorhref{https://en.wikipedia.org/wiki/Gordon\_Pask}{Gordona Paska}.
  Przyjęli oni, że~obok wcześniejszej tezy, że~mózg jest obiektem
  posiadającym zdolność do~adaptacji do otoczenia, należy uznać też tezę
  odwrotną: każdy układ który adaptuje~się do otoczenia jest w~jakimś
  sensie mózgiem. Beer na podstawie tego stwierdził, że~staw wodny jest
  mózgiem \parencite{Pickering-Cybernetics-in-Britain-Ver-2022}. Zaś
  w~$1961$~roku opublikowany zostaje \textit{Solaris}.

  Lema bardzo zajmował przypadek, a~ja chciałbym wiedzieć, czy to jest
  przypadek, czy nie. A~może zwyczajni widzę tutaj podobieństwa, których
  w~rzeczywistości nie ma?

\end{frame}
% ##################










% ######################################
\section{\textit{Dialogi} Stanisława Lema}
% ######################################



% ##################
\begin{frame}
  \frametitle{Czym są \textit{Dialogi}?}


  Zacznijmy od przytoczenia pewnych ustaleń Pawła Majewskiego z~jego
  \textit{Między zwierzęciem a~maszyną\ldots}. Główny wzorem dla Lema przy
  tworzeniu swoich dialogów, wydaje~się być dzieło \textit{Trzy dialogi
    między Hylasem i~Filonousem}, wydanych przez ważnego brytyjskiego
  filozofa George’a Berkeleya w~$1713$~roku, w~którym Berkeley wyłożył
  swoją filozofię, klasyfikowaną często jako część empiryzmu brytyjskiego.
  Lem zapożyczył od Berkeleya zarówno formę dialogu filozoficznego, imiona
  bohaterów, jak i~podstawowy podział na „dobrego” Filonousa i~„złego”
  Hylasa. Ich imiona są znaczące, gdyż „Filonous” oznacza „umysłowy”,
  „intelektualny”, zaś~„Hylas” tłumaczy~się na~„cielesny”, „materialny”.

  Dialogów jest osiem. Pierwszy z~nich powstał w~latach $1948\text{-}1950$,
  pozostałe siedem w~latach $1954\text{-}1956$. Jak już widzieliśmy
  wcześniej, poprzez napisanie \textit{Dialogów} stał~się publicznym
  wyrazicielem przekonań entuzjastów cybernetyki tamtego czasu.

\end{frame}
% ##################





% ##################
\begin{frame}
  \frametitle{Czym są \textit{Dialogi}?}


  Wydaje~się, że~w~momencie publikacji w~roku~$1957$, \textit{Dialogi} miały
  być książką wprowadzającą osoby zainteresowane filozofią w~cybernetyką,
  czyli funkcjonowały jako pewien tym literatury popularnonaukowej.
  W~książce nie uświadczysz, ani jednego wzoru, czego nie można powiedzieć
  o~książce Wienera.

  Jako, że~moje skromne badania dotyczą wpływu cybernetyki na twórczość
  literacką Lema, \textit{Dialogi} w~których przedstawia on zainteresowanemu
  czytelnikowi cybernetykę, były naturalnym punktem wyjścia, by ustalić,
  co~Lem o~niej wiedział w~roku~$1956$. Pominę przy tym wszystkie pytania
  jakie na temat warstwy literackiej \textit{Dialogów}, jaki mi~się
  nasunęły w~trakcie lektury.

\end{frame}
% ##################





% ##################
\begin{frame}
  \frametitle{\textit{Dialogi} i~termin „sztuczna
    inteligencja”}


  Jednym z~podstawowych pojęć u~Lem jest przypadek, więc wspomnijmy o~jednym
  z~nich. Wedle mojej wiedzy, termin \textit{sztuczna inteligencja} został
  ukuty $1955$~roku przez czterech gigantów informatyki:
  \colorhref{https://en.wikipedia.org/wiki/John_McCarthy_(computer_scientist)}
  {Johna McCarthy’ego},
  \colorhref{https://pl.wikipedia.org/wiki/Marvin\_Minsky}
  {Marvina Minsky’ego},
  \colorhref{https://en.wikipedia.org/wiki/Nathaniel\_Rochester\_(computer\_scientist)}{Nathaniela Rochester’a}
  i~\colorhref{https://en.wikipedia.org/wiki/Claude\_Shannon}
  {Claude’a Shannona}. Termin ten wprowadzili na potrzeby projektu letnich
  warsztatów, które odbył~się w~roku~$1956$ pod nazwą
  \textit{Dartmouth Summer Research Project on Artificial Intelligence}
  i~są uważany za~jedno z~najważniejszych wydarzeń w~historii tej
  dziedziny. Wedle pewnych świadectw, w~nazwie tej konferencji nie użyto
  słowa „cybernetyka”, ze względu na personalne animozje między niektórymi
  z~organizatorów, a~Norbertem Wienerem. Mało profesjonale, ale jakże
  normalne i~ludzkie.

  Niezależnie od~tego, warsztaty te odbywają~się w~tym samym roku w~którym
  Lem kończy pisać \textit{Dialogi}, więc może gdyby ukończył je rok
  później, to termin „sztuczna inteligencja” byłby obecny w~ich tekście.


\end{frame}
% ##################





% ##################
\begin{frame}
  \frametitle{Treść \textit{Dialogów}}


  \textit{Dialogi} zaczynają~się, gdy Filonous napotyka w~„pięknym parku”
  samotnie rozmyślającego Hylasa. Hylas, w~tym momencie zadeklarowany
  materialista, wyjaśnia mu, że~gdy technologi osiągnie odpowiedni
  poziom rozwoju, to będzie mógł zostać „wskrzeszony” przez idealne
  skopiowanie całego jego ciała. Dyskusja między nimi trwa do mniej więcej
  jednej trzeciej dialogu numer iii, gdzie Hylas uznaje, że~nie potrafi
  sprowadzić świadomości człowieka do~konfiguracji materialnych elementów
  jego ciała. Moje osobiste odczucie jest takie, że~ich rozważania na temat
  relacji świadomości oraz~materii zawierają zbyt dużo i~szybkie przeskoki,
  by można było uważać je za filozoficznie zadowalające, ale musimy to
  odłożyć na bok.

  W~tym momencie Filonous zaczyna wykładać swój pogląd i~by to zrobić
  opowiada o~świecie, w~którym mózgi mogą wymieniać~się zawartością. Motyw
  wymiany pamięci, a~tym samym osobowości, pojawi~się w~prozie Lema nie raz.

\end{frame}
% ##################





% ##################
\begin{frame}
  \frametitle{\textit{Dialogi} i~\textit{Terminus}}


  Jako hipotezę roboczą proponuję spojrzenie na opowiadanie
  \textit{Terminus}, jako opowieść o~wymianie pamięci między ludźmi,
  a~tytułowym Terminusem, za pomocą procedury uczenia maszynowego. Bardzo
  jestem ciekawy, czy według Państwa jest to sensowny pomysł, czy nie?

  Konsekwencją wywodu o~wymianie pamięci są następujące słowa wypowiadane
  przez Fileonous, które zasadniczo są parafrazą tego, co można znaleźć
  w~\textit{Cybernetics\ldots} Wienera. \\
  \textit{Zastanówmy~się mianowicie, co wymieniały mózgi???}
  Str.~$55$, \parencite{Lem-Dialogi-Vol-I-Pub-1996}.

\end{frame}
% ##################





% % ##################
% \begin{frame}
%   \frametitle{}




% \end{frame}
% % ##################





% % ##################
% \begin{frame}
%   \frametitle{}




% \end{frame}
% % ##################





% % ##################
% \begin{frame}
%   \frametitle{}




% \end{frame}
% % ##################





% % ##################
% \begin{frame}
%   \frametitle{}




% \end{frame}
% % ##################





% % ##################
% \begin{frame}
%   \frametitle{}




% \end{frame}
% % ##################





% % ##################
% \begin{frame}
%   \frametitle{}




% \end{frame}
% % ##################














% % ##################
% \begin{frame}
%   \frametitle{}




% \end{frame}
% % ##################





% % ##################
% \begin{frame}
%   \frametitle{}




%   \end{frame}
% % ##################





% % ##################
% \begin{frame}
%   \frametitle{}




% \end{frame}
% % ##################





% % ##################
% \begin{frame}
%   \frametitle{}




% \end{frame}
% % ##################





% % ##################
% \begin{frame}
%   \frametitle{}




% \end{frame}
% % ##################





% % ##################
% \begin{frame}
%   \frametitle{}




% \end{frame}
% % ##################










% % ######################################
% \section{}
% % ######################################


% % ##################
% \begin{frame}
%   \frametitle{}




% \end{frame}
% % ##################





% % ##################
% \begin{frame}
%   \frametitle{}




% \end{frame}
% % ##################





% % ##################
% \begin{frame}
%   \frametitle{}




% \end{frame}
% % ##################





% % ##################
% \begin{frame}
%   \frametitle{}




% \end{frame}
% % ##################





% % ##################
% \begin{frame}
%   \frametitle{}




% \end{frame}
% % ##################





% % ##################
% \begin{frame}
%   \frametitle{}




% \end{frame}
% % ##################





% % ##################
% \begin{frame}
%   \frametitle{}




% \end{frame}
% % ##################





% % ##################
% \begin{frame}
%   \frametitle{}




% \end{frame}
% % ##################





% % ##################
% \begin{frame}
%   \frametitle{}




% \end{frame}
% % ##################





% % ##################
% \begin{frame}
%   \frametitle{}




% \end{frame}
% % ##################





% % ##################
% \begin{frame}
%   \frametitle{}




% \end{frame}
% % ##################






% % ##################
% \begin{frame}
%   \frametitle{}




% \end{frame}
% % ##################





% % ##################
% \begin{frame}
%   \frametitle{Pytania, problemy, wątpliwości}




% \end{frame}
% % ##################










% ######################################
\appendix
% ######################################





% ######################################
\EndingSlide{Dziękuję! Pytania?}
% ######################################





% ####################################################################
% ####################################################################
% Bibliography

\printbibliography










% ############################

% Koniec dokumentu
\end{document}
